\chapter{\textmalayalam{അല്‍ ഫാത്തിഹ ( പ്രാരംഭം )}}
\begin{Arabic}
\Huge{\centerline{\basmalah}}\end{Arabic}
\flushright{\begin{Arabic}
\quranayah[1][1]
\end{Arabic}}
\flushleft{\begin{malayalam}
പരമകാരുണികനും കരുണാനിധിയുമായ അല്ലാഹുവിന്റെ നാമത്തില്‍ .
\end{malayalam}}
\flushright{\begin{Arabic}
\quranayah[1][2]
\end{Arabic}}
\flushleft{\begin{malayalam}
സ്തുതി സര്‍വ്വലോക പരിപാലകനായ അല്ലാഹുവിന്നാകുന്നു.
\end{malayalam}}
\flushright{\begin{Arabic}
\quranayah[1][3]
\end{Arabic}}
\flushleft{\begin{malayalam}
പരമകാരുണികനും കരുണാനിധിയും.
\end{malayalam}}
\flushright{\begin{Arabic}
\quranayah[1][4]
\end{Arabic}}
\flushleft{\begin{malayalam}
പ്രതിഫല ദിവസത്തിന്റെ ഉടമസ്ഥന്‍.
\end{malayalam}}
\flushright{\begin{Arabic}
\quranayah[1][5]
\end{Arabic}}
\flushleft{\begin{malayalam}
നിന്നെ മാത്രം ഞങ്ങള്‍ ആരാധിക്കുന്നു. നിന്നോട് മാത്രം ഞങ്ങള്‍ സഹായം തേടുന്നു.
\end{malayalam}}
\flushright{\begin{Arabic}
\quranayah[1][6]
\end{Arabic}}
\flushleft{\begin{malayalam}
ഞങ്ങളെ നീ നേര്‍മാര്‍ഗത്തില്‍ ചേര്‍ക്കേണമേ.
\end{malayalam}}
\flushright{\begin{Arabic}
\quranayah[1][7]
\end{Arabic}}
\flushleft{\begin{malayalam}
നീ അനുഗ്രഹിച്ചവരുടെ മാര്‍ഗത്തില്‍ . കോപത്തിന്ന് ഇരയായവരുടെ മാര്‍ഗത്തിലല്ല. പിഴച്ചുപോയവരുടെ മാര്‍ഗത്തിലുമല്ല.
\end{malayalam}}
\chapter{\textmalayalam{അല്‍ ബഖറ ( പശു )}}
\begin{Arabic}
\Huge{\centerline{\basmalah}}\end{Arabic}
\flushright{\begin{Arabic}
\quranayah[2][1]
\end{Arabic}}
\flushleft{\begin{malayalam}
അലിഫ് ലാം മീം.
\end{malayalam}}
\flushright{\begin{Arabic}
\quranayah[2][2]
\end{Arabic}}
\flushleft{\begin{malayalam}
ഇതാകുന്നു ഗ്രന്ഥം. അതില്‍ സംശയമേയില്ല. സൂക്ഷ്മത പാലിക്കുന്നവര്‍ക്ക് നേര്‍വഴി കാണിക്കുന്നതത്രെ അത്‌.
\end{malayalam}}
\flushright{\begin{Arabic}
\quranayah[2][3]
\end{Arabic}}
\flushleft{\begin{malayalam}
അദൃശ്യകാര്യങ്ങളില്‍ വിശ്വസിക്കുകയും, പ്രാര്‍ത്ഥന അഥവാ നമസ്കാരം മുറപ്രകാരം നിര്‍വഹിക്കുകയും, നാം നല്‍കിയ സമ്പത്തില്‍ നിന്ന് ചെലവഴിക്കുകയും,
\end{malayalam}}
\flushright{\begin{Arabic}
\quranayah[2][4]
\end{Arabic}}
\flushleft{\begin{malayalam}
നിനക്കും നിന്റെമുന്‍ഗാമികള്‍ക്കും നല്‍കപ്പെട്ട സന്ദേശത്തില്‍ വിശ്വസിക്കുകയും, പരലോകത്തില്‍ ദൃഢമായി വിശ്വസിക്കുകയും ചെയ്യുന്നവരത്രെ അവര്‍ (സൂക്ഷ്മത പാലിക്കുന്നവര്‍).
\end{malayalam}}
\flushright{\begin{Arabic}
\quranayah[2][5]
\end{Arabic}}
\flushleft{\begin{malayalam}
അവരുടെ നാഥന്‍ കാണിച്ച നേര്‍വഴിയിലാകുന്നു അവര്‍. അവര്‍ തന്നെയാകുന്നു സാക്ഷാല്‍ വിജയികള്‍.
\end{malayalam}}
\flushright{\begin{Arabic}
\quranayah[2][6]
\end{Arabic}}
\flushleft{\begin{malayalam}
സത്യനിഷേധികളെ സംബന്ധിച്ചിടത്തോളം നീ അവര്‍ക്ക് താക്കീത് നല്‍കിയാലും ഇല്ലെങ്കിലും സമമാകുന്നു. അവര്‍ വിശ്വസിക്കുന്നതല്ല.
\end{malayalam}}
\flushright{\begin{Arabic}
\quranayah[2][7]
\end{Arabic}}
\flushleft{\begin{malayalam}
അവരുടെ മനസ്സുകള്‍ക്കും കാതിനും അല്ലാഹു മുദ്രവെച്ചിരിക്കുകയാണ് . അവരുടെ ദൃഷ്ടികളിന്‍മേലും ഒരു മൂടിയുണ്ട്‌. അവര്‍ക്കാകുന്നു കനത്ത ശിക്ഷയുള്ളത്‌.
\end{malayalam}}
\flushright{\begin{Arabic}
\quranayah[2][8]
\end{Arabic}}
\flushleft{\begin{malayalam}
ഞങ്ങള്‍ അല്ലാഹുവിലും അന്ത്യദിനത്തിലും വിശ്വസിച്ചിരിക്കുന്നു എന്ന് പറയുന്ന ചില ആളുകളുണ്ട് ; (യഥാര്‍ത്ഥത്തില്‍) അവര്‍ വിശ്വാസികളല്ല.
\end{malayalam}}
\flushright{\begin{Arabic}
\quranayah[2][9]
\end{Arabic}}
\flushleft{\begin{malayalam}
അല്ലാഹുവിനെയും വിശ്വാസികളെയും വഞ്ചിക്കുവാനാണ് അവര്‍ ശ്രമിക്കുന്നത്‌. (വാസ്തവത്തില്‍) അവര്‍ ആത്മവഞ്ചന മാത്രമാണ് ചെയ്യുന്നത്‌. അവരത് മനസ്സിലാക്കുന്നില്ല.
\end{malayalam}}
\flushright{\begin{Arabic}
\quranayah[2][10]
\end{Arabic}}
\flushleft{\begin{malayalam}
അവരുടെ മനസ്സുകളില്‍ ഒരുതരം രോഗമുണ്ട്‌. തന്നിമിത്തം അല്ലാഹു അവര്‍ക്ക് രോഗം വര്‍ദ്ധിപ്പിക്കുകയും ചെയ്തു. കള്ളം പറഞ്ഞുകൊണ്ടിരുന്നതിന്റെഫലമായി വേദനയേറിയ ശിക്ഷയാണ് അവര്‍ക്കുണ്ടായിരിക്കുക.
\end{malayalam}}
\flushright{\begin{Arabic}
\quranayah[2][11]
\end{Arabic}}
\flushleft{\begin{malayalam}
നിങ്ങള്‍ നാട്ടില്‍ കുഴപ്പമുണ്ടാക്കാതിരിക്കൂ എന്ന് അവരോട് ആരെങ്കിലും പറഞ്ഞാല്‍, ഞങ്ങള്‍ സല്‍പ്രവര്‍ത്തനങ്ങള്‍ മാത്രമാണല്ലോ ചെയ്യുന്നത് എന്നായിരിക്കും അവരുടെ മറുപടി.
\end{malayalam}}
\flushright{\begin{Arabic}
\quranayah[2][12]
\end{Arabic}}
\flushleft{\begin{malayalam}
എന്നാല്‍ യഥാര്‍ത്ഥത്തില്‍ അവര്‍ തന്നെയാകുന്നു കുഴപ്പക്കാര്‍. പക്ഷെ, അവരത് മനസ്സിലാക്കുന്നില്ല.
\end{malayalam}}
\flushright{\begin{Arabic}
\quranayah[2][13]
\end{Arabic}}
\flushleft{\begin{malayalam}
മറ്റുള്ളവര്‍ വിശ്വസിച്ചത് പോലെ നിങ്ങളും വിശ്വസിക്കൂ എന്ന് അവരോട് ആരെങ്കിലും പറഞ്ഞാല്‍ ഈ മൂഢന്‍മാര്‍ വിശ്വസിച്ചത് പോലെ ഞങ്ങളും വിശ്വസിക്കുകയോ ? എന്നായിരിക്കും അവര്‍ മറുപടി പറയുക. എന്നാല്‍ യഥാര്‍ത്ഥത്തില്‍ അവര്‍ തന്നെയാകുന്നു മൂഢന്‍മാര്‍. പക്ഷെ, അവരത് അറിയുന്നില്ല.
\end{malayalam}}
\flushright{\begin{Arabic}
\quranayah[2][14]
\end{Arabic}}
\flushleft{\begin{malayalam}
വിശ്വാസികളെ കണ്ടുമുട്ടുമ്പോള്‍ അവര്‍ പറയും; ഞങ്ങള്‍ വിശ്വസിച്ചിരിക്കുന്നു എന്ന്‌. അവര്‍ തങ്ങളുടെ (കൂട്ടാളികളായ) പിശാചുക്കളുടെ അടുത്ത് തനിച്ചാകുമ്പോള്‍ അവരോട് പറയും: ഞങ്ങള്‍ നിങ്ങളോടൊപ്പം തന്നെയാകുന്നു. ഞങ്ങള്‍ (മറ്റവരെ) കളിയാക്കുക മാത്രമായിരുന്നു.
\end{malayalam}}
\flushright{\begin{Arabic}
\quranayah[2][15]
\end{Arabic}}
\flushleft{\begin{malayalam}
എന്നാല്‍ അല്ലാഹുവാകട്ടെ, അവരെ പരിഹസിക്കുകയും, അതിക്രമങ്ങളില്‍ വിഹരിക്കുവാന്‍ അവരെ അയച്ചുവിട്ടിരിക്കുകയുമാകുന്നു.
\end{malayalam}}
\flushright{\begin{Arabic}
\quranayah[2][16]
\end{Arabic}}
\flushleft{\begin{malayalam}
സന്മാര്‍ഗം വിറ്റ് പകരം ദുര്‍മാര്‍ഗം വാങ്ങിയവരാകുന്നു അവര്‍. എന്നാല്‍ അവരുടെ കച്ചവടം ലാഭകരമാവുകയോ, അവര്‍ ലക്ഷ്യം പ്രാപിക്കുകയോ ചെയ്തില്ല.
\end{malayalam}}
\flushright{\begin{Arabic}
\quranayah[2][17]
\end{Arabic}}
\flushleft{\begin{malayalam}
അവരെ ഉപമിക്കാവുന്നത് ഒരാളോടാകുന്നു: അയാള്‍ തീ കത്തിച്ചു. പരിസരമാകെ പ്രകാശിതമായപ്പോള്‍ അല്ലാഹു അവരുടെ പ്രകാശം കെടുത്തിക്കളയുകയും ഒന്നും കാണാനാവാതെ ഇരുട്ടില്‍ (തപ്പുവാന്‍) അവരെ വിടുകയും ചെയ്തു.
\end{malayalam}}
\flushright{\begin{Arabic}
\quranayah[2][18]
\end{Arabic}}
\flushleft{\begin{malayalam}
ബധിരരും ഊമകളും അന്ധന്‍മാരുമാകുന്നു അവര്‍. അതിനാല്‍ അവര്‍ (സത്യത്തിലേക്ക്‌) തിരിച്ചുവരികയില്ല.
\end{malayalam}}
\flushright{\begin{Arabic}
\quranayah[2][19]
\end{Arabic}}
\flushleft{\begin{malayalam}
അല്ലെങ്കില്‍ (അവരെ) ഉപമിക്കാവുന്നത് ആകാശത്തുനിന്നു ചൊരിയുന്ന ഒരു പേമാരിയോടാകുന്നു. അതോടൊപ്പം കൂരിരുട്ടും ഇടിയും മിന്നലുമുണ്ട്‌. ഇടിനാദങ്ങള്‍ നിമിത്തം മരണം ഭയന്ന് അവര്‍ വിരലുകള്‍ ചെവിയില്‍ തിരുകുന്നു. എന്നാല്‍ അല്ലാഹു സത്യനിഷേധികളെ വലയം ചെയ്തിരിക്കുകയാണ്‌.
\end{malayalam}}
\flushright{\begin{Arabic}
\quranayah[2][20]
\end{Arabic}}
\flushleft{\begin{malayalam}
മിന്നല്‍ അവരുടെ കണ്ണുകളെ റാഞ്ചിയെടുക്കുമാറാകുന്നു. അത് (മിന്നല്‍) അവര്‍ക്ക് വെളിച്ചം നല്‍കുമ്പോഴെല്ലാം അവര്‍ ആ വെളിച്ചത്തില്‍ നടന്നു പോകും. ഇരുട്ടാകുമ്പോള്‍ അവര്‍ നിന്നു പോകുകയും ചെയ്യും. അല്ലാഹു ഉദ്ദേശിച്ചിരുന്നെങ്കില്‍ അവരുടെ കേള്‍വിയും കാഴ്ചയും അവന്‍ തീരെ നശിപ്പിച്ചുകളയുക തന്നെ ചെയ്യുമായിരുന്നു. നിസ്സംശയം അല്ലാഹു ഏത് കാര്യത്തിനും കഴിവുള്ളവനാണ്‌.
\end{malayalam}}
\flushright{\begin{Arabic}
\quranayah[2][21]
\end{Arabic}}
\flushleft{\begin{malayalam}
ജനങ്ങളേ, നിങ്ങളേയും നിങ്ങളുടെ മുന്‍ഗാമികളേയും സൃഷ്ടിച്ച നിങ്ങളുടെ നാഥനെ നിങ്ങള്‍ ആരാധിക്കുവിന്‍. നിങ്ങള്‍ ദോഷബാധയെ സൂക്ഷിച്ച് ജീവിക്കുവാന്‍ വേണ്ടിയത്രെ അത്‌.
\end{malayalam}}
\flushright{\begin{Arabic}
\quranayah[2][22]
\end{Arabic}}
\flushleft{\begin{malayalam}
നിങ്ങള്‍ക്ക് വേണ്ടി ഭൂമിയെ മെത്തയും ആകാശത്തെ മേല്‍പുരയുമാക്കിത്തരികയും ആകാശത്ത് നിന്ന് വെള്ളം ചൊരിഞ്ഞുതന്നിട്ട് അത് മുഖേന നിങ്ങള്‍ക്ക് ഭക്ഷിക്കുവാനുള്ള കായ്കനികള്‍ ഉല്‍പാദിപ്പിച്ചു തരികയും ചെയ്ത (നാഥനെ). അതിനാല്‍ (ഇതെല്ലാം) അറിഞ്ഞ്കൊണ്ട് നിങ്ങള്‍ അല്ലാഹുവിന് സമന്‍മാരെ ഉണ്ടാക്കരുത്‌.
\end{malayalam}}
\flushright{\begin{Arabic}
\quranayah[2][23]
\end{Arabic}}
\flushleft{\begin{malayalam}
നമ്മുടെ ദാസന് നാം അവതരിപ്പിച്ചുകൊടുത്തതിനെ (വിശുദ്ധ ഖുര്‍ആനെ) പറ്റി നിങ്ങള്‍ സംശയാലുക്കളാണെങ്കില്‍ അതിന്റേത്പോലുള്ള ഒരു അദ്ധ്യായമെങ്കിലും നിങ്ങള്‍ കൊണ്ടുവരിക. അല്ലാഹുവിന് പുറമെ നിങ്ങള്‍ക്കുള്ള സഹായികളേയും വിളിച്ചുകൊള്ളുക. നിങ്ങള്‍ സത്യവാന്‍മാരണെങ്കില്‍ (അതാണല്ലോ വേണ്ടത്‌).
\end{malayalam}}
\flushright{\begin{Arabic}
\quranayah[2][24]
\end{Arabic}}
\flushleft{\begin{malayalam}
നിങ്ങള്‍ക്കത് ചെയ്യാന്‍ കഴിഞ്ഞില്ലെങ്കില്‍ നിങ്ങള്‍ക്കത് ഒരിക്കലും ചെയ്യാന്‍ കഴിയുകയുമില്ല മനുഷ്യരും കല്ലുകളും ഇന്ധനമായി കത്തിക്കപ്പെടുന്ന നരകാഗ്നിയെ നിങ്ങള്‍ കാത്തുസൂക്ഷിച്ചുകൊള്ളുക. സത്യനിഷേധികള്‍ക്കുവേണ്ടി ഒരുക്കിവെക്കപ്പെട്ടതാകുന്നു അത്‌.
\end{malayalam}}
\flushright{\begin{Arabic}
\quranayah[2][25]
\end{Arabic}}
\flushleft{\begin{malayalam}
(നബിയേ,) വിശ്വസിക്കുകയും സല്‍കര്‍മ്മങ്ങള്‍ പ്രവര്‍ത്തിക്കുകയും ചെയ്തവര്‍ക്ക് താഴ്ഭാഗത്ത്കൂടി നദികള്‍ ഒഴുകുന്ന സ്വര്‍ഗത്തോപ്പുകള്‍ ലഭിക്കുവാനുണ്ടെന്ന് സന്തോഷവാര്‍ത്ത അറിയിക്കുക. അതിലെ ഓരോ വിഭവവും ഭക്ഷിക്കുവാനായി നല്‍കപ്പെടുമ്പോള്‍, ഇതിന് മുമ്പ് ഞങ്ങള്‍ക്ക് നല്‍കപ്പെട്ടത് തന്നെയാണല്ലോ ഇതും എന്നായിരിക്കും അവര്‍ പറയുക. (വാസ്തവത്തില്‍) പരസ്പര സാദൃശ്യമുള്ള നിലയില്‍ അതവര്‍ക്ക് നല്‍കപ്പെടുകയാണുണ്ടായത്‌.പരിശുദ്ധരായ ഇണകളും അവര്‍ക്കവിടെ ഉണ്ടായിരിക്കും. അവര്‍ അവിടെ നിത്യവാസികളായിരിക്കുകയും ചെയ്യും.
\end{malayalam}}
\flushright{\begin{Arabic}
\quranayah[2][26]
\end{Arabic}}
\flushleft{\begin{malayalam}
ഏതൊരു വസ്തുവേയും ഉപമയാക്കുന്നതില്‍ അല്ലാഹു ലജ്ജിക്കുകയില്ല; തീര്‍ച്ച. അതൊരു കൊതുകോ അതിലുപരി നിസ്സാരമോ ആകട്ടെ. എന്നാല്‍ വിശ്വാസികള്‍ക്ക് അത് തങ്ങളുടെ നാഥന്റെപക്കല്‍നിന്നുള്ള സത്യമാണെന്ന് ബോധ്യമാകുന്നതാണ്‌. സത്യനിഷേധികളാകട്ടെ ഈ ഉപമകൊണ്ട് അല്ലാഹു എന്താണ് ഉദ്ദേശിക്കുന്നത് എന്ന് ചോദിക്കുകയാണ് ചെയ്യുക. അങ്ങനെ ആ ഉപമ നിമിത്തം ധാരാളം ആളുകളെ അവന്‍ പിഴവിലാക്കുന്നു. ധാരാളം പേരെ നേര്‍വഴിയിലാക്കുകയും ചെയ്യുന്നു. അധര്‍മ്മകാരികളല്ലാത്ത ആരെയും അത് നിമിത്തം അവന്‍ പിഴപ്പിക്കുകയില്ല.
\end{malayalam}}
\flushright{\begin{Arabic}
\quranayah[2][27]
\end{Arabic}}
\flushleft{\begin{malayalam}
അല്ലാഹുവിന്റെഉത്തരവ് അവന്‍ ശക്തിയുക്തം നല്‍കിയതിന് ശേഷം അതിന് വിപരീതം പ്രവര്‍ത്തിക്കുകയും അല്ലാഹു കൂട്ടിചേര്‍ക്കുവാന്‍ കല്‍പിച്ചതിനെ മുറിച്ച് വേര്‍പെടുത്തുകയും ഭൂമിയില്‍ കുഴപ്പമുണ്ടാക്കുകയും ചെയ്യുന്നവരത്രെ അവര്‍ (അധര്‍മ്മകാരികള്‍). അവര്‍ തന്നെയാകുന്നു നഷ്ടക്കാര്‍.
\end{malayalam}}
\flushright{\begin{Arabic}
\quranayah[2][28]
\end{Arabic}}
\flushleft{\begin{malayalam}
നിങ്ങള്‍ക്കെങ്ങനെയാണ് അല്ലാഹുവിനെ നിഷേധിക്കാന്‍ കഴിയുക ? നിങ്ങള്‍ നിര്‍ജീവ വസ്തുക്കളായിരുന്ന അവസ്ഥയ്ക്ക് ശേഷം അവന്‍ നിങ്ങള്‍ക്ക് ജീവന്‍ നല്‍കി. പിന്നെ അവന്‍ നിങ്ങളെ മരിപ്പിക്കുകയും വീണ്ടും ജീവിപ്പിക്കുകയും ചെയ്യുന്നു. പിന്നീട് അവന്‍കലേക്ക് തന്നെ നിങ്ങള്‍ തിരിച്ചുവിളിക്കപ്പെടുകയും ചെയ്യും.
\end{malayalam}}
\flushright{\begin{Arabic}
\quranayah[2][29]
\end{Arabic}}
\flushleft{\begin{malayalam}
അവനാണ് നിങ്ങള്‍ക്ക് വേണ്ടി ഭൂമിയിലുള്ളതെല്ലാം സൃഷ്ടിച്ചു തന്നത്‌. പുറമെ ഏഴ് ആകാശങ്ങളായി ക്രമീകരിച്ചുകൊണ്ട് ഉപരിലോകത്തെ സംവിധാനിച്ചവനും അവന്‍ തന്നെയാണ്‌. അവന്‍ എല്ലാ കാര്യത്തെപ്പറ്റിയും അറിവുള്ളവനാകുന്നു.
\end{malayalam}}
\flushright{\begin{Arabic}
\quranayah[2][30]
\end{Arabic}}
\flushleft{\begin{malayalam}
ഞാനിതാ ഭൂമിയില്‍ ഒരു ഖലീഫയെ നിയോഗിക്കാന്‍ പോകുകയാണ് എന്ന് നിന്റെനാഥന്‍ മലക്കുകളോട് പറഞ്ഞ സന്ദര്‍ഭം (ശ്രദ്ധിക്കുക). അവര്‍ പറഞ്ഞു: അവിടെ കുഴപ്പമുണ്ടാക്കുകയും രക്തം ചിന്തുകയും ചെയ്യുന്നവരെയാണോ നീ നിയോഗിക്കുന്നത്‌? ഞങ്ങളാകട്ടെ നിന്റെമഹത്വത്തെ പ്രകീര്‍ത്തിക്കുകയും, നിന്റെപരിശുദ്ധിയെ വാഴ്ത്തുകയും ചെയ്യുന്നവരല്ലോ. അവന്‍ (അല്ലാഹു) പറഞ്ഞു: നിങ്ങള്‍ക്കറിഞ്ഞുകൂടാത്തത് എനിക്കറിയാം.
\end{malayalam}}
\flushright{\begin{Arabic}
\quranayah[2][31]
\end{Arabic}}
\flushleft{\begin{malayalam}
അവന്‍ (അല്ലാഹു) ആദമിന് നാമങ്ങളെല്ലാം പഠിപ്പിച്ചു. പിന്നീട് ആ പേരിട്ടവയെ അവന്‍ മലക്കുകള്‍ക്ക് കാണിച്ചു. എന്നിട്ടവന്‍ ആജ്ഞാപിച്ചു: നിങ്ങള്‍ സത്യവാന്‍മാരാണെങ്കില്‍ ഇവയുടെ നാമങ്ങള്‍ എനിക്ക് പറഞ്ഞുതരൂ.
\end{malayalam}}
\flushright{\begin{Arabic}
\quranayah[2][32]
\end{Arabic}}
\flushleft{\begin{malayalam}
അവര്‍ പറഞ്ഞു: നിനക്ക് സ്തോത്രം. നീ പഠിപ്പിച്ചുതന്നതല്ലാത്ത യാതൊരു അറിവും ഞങ്ങള്‍ക്കില്ല. നീ തന്നെയാണ് സര്‍വ്വജ്ഞനും അഗാധജ്ഞാനിയും.
\end{malayalam}}
\flushright{\begin{Arabic}
\quranayah[2][33]
\end{Arabic}}
\flushleft{\begin{malayalam}
അനന്തരം അവന്‍ (അല്ലാഹു) പറഞ്ഞു: ആദമേ, ഇവര്‍ക്ക് അവയുടെ നാമങ്ങള്‍ പറഞ്ഞുകൊടുക്കൂ. അങ്ങനെ അവന്‍ (ആദം) അവര്‍ക്ക് ആ നാമങ്ങള്‍ പറഞ്ഞുകൊടുത്തപ്പോള്‍ അവന്‍ (അല്ലാഹു) പറഞ്ഞു: ആകാശ ഭൂമികളിലെ അദൃശ്യകാര്യങ്ങളും, നിങ്ങള്‍ വെളിപ്പെടുത്തുന്നതും, ഒളിച്ചുവെക്കുന്നതുമെല്ലാം എനിക്കറിയാമെന്ന് ഞാന്‍ നിങ്ങളോട് പറഞ്ഞിട്ടില്ലേ?
\end{malayalam}}
\flushright{\begin{Arabic}
\quranayah[2][34]
\end{Arabic}}
\flushleft{\begin{malayalam}
ആദമിനെ നിങ്ങള്‍ പ്രണമിക്കുക എന്ന് നാം മലക്കുകളോട് പറഞ്ഞ സന്ദര്‍ഭം (ശ്രദ്ധിക്കുക) . അവര്‍ പ്രണമിച്ചു; ഇബ്ലീസ് ഒഴികെ. അവന്‍ വിസമ്മതം പ്രകടിപ്പിക്കുകയും അഹംഭാവം നടിക്കുകയും ചെയ്തു. അവന്‍ സത്യനിഷേധികളില്‍ പെട്ടവനായിരിക്കുന്നു.
\end{malayalam}}
\flushright{\begin{Arabic}
\quranayah[2][35]
\end{Arabic}}
\flushleft{\begin{malayalam}
ആദമേ, നീയും നിന്റെഇണയും സ്വര്‍ഗത്തില്‍ താമസിക്കുകയും അതില്‍ നിങ്ങള്‍ ഇച്ഛിക്കുന്നിടത്തു നിന്ന് സുഭിക്ഷമായി ഇരുവരും ഭക്ഷിച്ചുകൊള്ളുകയും ചെയ്യുക. എന്നാല്‍ ഈ വൃക്ഷത്തെ നിങ്ങള്‍ സമീപിച്ചുപോകരുത്‌. എങ്കില്‍ നിങ്ങള്‍ ഇരുവരും അതിക്രമകാരികളായിത്തീരും എന്നു നാം ആജ്ഞാപിച്ചു.
\end{malayalam}}
\flushright{\begin{Arabic}
\quranayah[2][36]
\end{Arabic}}
\flushleft{\begin{malayalam}
എന്നാല്‍ പിശാച് അവരെ അതില്‍ നിന്ന് വ്യതിചലിപ്പിച്ചു. അവര്‍ ഇരുവരും അനുഭവിച്ചിരുന്നതില്‍ (സൌഭാഗ്യം) നിന്ന് അവരെ പുറം തള്ളുകയും ചെയ്തു. നാം (അവരോട്‌) പറഞ്ഞു: നിങ്ങള്‍ ഇറങ്ങിപ്പോകൂ. നിങ്ങളില്‍ ചിലര്‍ ചിലര്‍ക്ക് ശത്രുക്കളാകുന്നു. നിങ്ങള്‍ക്ക് ഭൂമിയില്‍ ഒരു നിശ്ചിത കാലം വരേക്കും വാസസ്ഥലവും ജീവിതവിഭവങ്ങളുമുണ്ടായിരിക്കും.
\end{malayalam}}
\flushright{\begin{Arabic}
\quranayah[2][37]
\end{Arabic}}
\flushleft{\begin{malayalam}
അനന്തരം ആദം തന്റെരക്ഷിതാവിങ്കല്‍ നിന്ന് ചില വചനങ്ങള്‍ സ്വീകരിച്ചു. (ആ വചനങ്ങള്‍ മുഖേന പശ്ചാത്തപിച്ച) ആദമിന് അല്ലാഹു പാപമോചനം നല്‍കി. അവന്‍ പശ്ചാത്താപം ഏറെ സ്വീകരിക്കുന്നവനും കരുണാനിധിയുമത്രെ.
\end{malayalam}}
\flushright{\begin{Arabic}
\quranayah[2][38]
\end{Arabic}}
\flushleft{\begin{malayalam}
നാം പറഞ്ഞു: നിങ്ങളെല്ലാവരും അവിടെ നിന്ന് ഇറങ്ങിപ്പോകുക. എന്നിട്ട് എന്റെപക്കല്‍ നിന്നുള്ള മാര്‍ഗദര്‍ശനം നിങ്ങള്‍ക്ക് വന്നെത്തുമ്പോള്‍ എന്റെആ മാര്‍ഗദര്‍ശനം പിന്‍പറ്റുന്നവരാരോ അവര്‍ക്ക് ഭയപ്പെടേണ്ടതില്ല. അവര്‍ ദുഃഖിക്കേണ്ടിവരികയുമില്ല.
\end{malayalam}}
\flushright{\begin{Arabic}
\quranayah[2][39]
\end{Arabic}}
\flushleft{\begin{malayalam}
അവിശ്വസിക്കുകയും നമ്മുടെ ദൃഷ്ടാന്തങ്ങള്‍ നിഷേധിച്ച് തള്ളുകയും ചെയ്തവരാരോ അവരായിരിക്കും നരകാവകാശികള്‍. അവരതില്‍ നിത്യവാസികളായിരിക്കും.
\end{malayalam}}
\flushright{\begin{Arabic}
\quranayah[2][40]
\end{Arabic}}
\flushleft{\begin{malayalam}
ഇസ്രായീല്‍ സന്തതികളേ, ഞാന്‍ നിങ്ങള്‍ക്ക് ചെയ്തു തന്നിട്ടുള്ള അനുഗ്രഹം നിങ്ങള്‍ ഓര്‍മിക്കുകയും, എന്നോടുള്ള കരാര്‍ നിങ്ങള്‍ നിറവേറ്റുകയും ചെയ്യുവിന്‍. എങ്കില്‍ നിങ്ങളോടുള്ള കരാര്‍ ഞാനും നിറവേറ്റാം. എന്നെ മാത്രമേ നിങ്ങള്‍ ഭയപ്പെടാവൂ.
\end{malayalam}}
\flushright{\begin{Arabic}
\quranayah[2][41]
\end{Arabic}}
\flushleft{\begin{malayalam}
നിങ്ങളുടെ പക്കലുള്ള വേദഗ്രന്ഥങ്ങളെ ശരിവെച്ചുകൊണ്ട് ഞാന്‍ അവതരിപ്പിച്ച സന്ദേശത്തില്‍ (ഖുര്‍ആനില്‍) നിങ്ങള്‍ വിശ്വസിക്കൂ. അതിനെ ആദ്യമായി തന്നെ നിഷേധിക്കുന്നവര്‍ നിങ്ങളാകരുത്‌. തുച്ഛമായ വിലയ്ക്ക് (ഭൗതിക നേട്ടത്തിനു) പകരം എന്റെവചനങ്ങള്‍ നിങ്ങള്‍ വിറ്റുകളയുകയും ചെയ്യരുത്‌. എന്നോട് മാത്രം നിങ്ങള്‍ ഭയഭക്തി പുലര്‍ത്തുക.
\end{malayalam}}
\flushright{\begin{Arabic}
\quranayah[2][42]
\end{Arabic}}
\flushleft{\begin{malayalam}
നിങ്ങള്‍ സത്യം അസത്യവുമായി കൂട്ടിക്കുഴക്കരുത്‌. അറിഞ്ഞുകൊണ്ട് സത്യം മറച്ചുവെക്കുകയും ചെയ്യരുത്‌.
\end{malayalam}}
\flushright{\begin{Arabic}
\quranayah[2][43]
\end{Arabic}}
\flushleft{\begin{malayalam}
പ്രാര്‍ത്ഥന മുറപോലെ നിര്‍വഹിക്കുകയും, സകാത്ത് നല്‍കുകയും, (അല്ലാഹുവിന്റെമുമ്പില്‍) തലകുനിക്കുന്നവരോടൊപ്പം നിങ്ങള്‍ തലകുനിക്കുകയും ചെയ്യുവിന്‍.
\end{malayalam}}
\flushright{\begin{Arabic}
\quranayah[2][44]
\end{Arabic}}
\flushleft{\begin{malayalam}
നിങ്ങള്‍ ജനങ്ങളോട് നന്‍മ കല്‍പിക്കുകയും നിങ്ങളുടെ സ്വന്തം കാര്യത്തില്‍ (അത്‌) മറന്നുകളയുകയുമാണോ ? നിങ്ങള്‍ വേദഗ്രന്ഥം പാരായണം ചെയ്തുകൊണ്ടിരിക്കുന്നുവല്ലോ. നിങ്ങളെന്താണ് ചിന്തിക്കാത്തത് ?
\end{malayalam}}
\flushright{\begin{Arabic}
\quranayah[2][45]
\end{Arabic}}
\flushleft{\begin{malayalam}
സഹനവും നമസ്കാരവും മുഖേന (അല്ലാഹുവിന്റെ) സഹായം തേടുക. അത് (നമസ്കാരം) ഭക്തന്‍മാരല്ലാത്തവര്‍ക്ക് വലിയ (പ്രയാസമുള്ള) കാര്യം തന്നെയാകുന്നു.
\end{malayalam}}
\flushright{\begin{Arabic}
\quranayah[2][46]
\end{Arabic}}
\flushleft{\begin{malayalam}
തങ്ങളുടെ രക്ഷിതാവുമായി കണ്ടുമുട്ടേണ്ടിവരുമെന്നും, അവങ്കലേക്ക് തിരിച്ചുപോകേണ്ടി വരുമെന്നും വിചാരിച്ചുകൊണ്ടിരിക്കുന്നവരത്രെ അവര്‍ (ഭക്തന്‍മാര്‍).
\end{malayalam}}
\flushright{\begin{Arabic}
\quranayah[2][47]
\end{Arabic}}
\flushleft{\begin{malayalam}
ഇസ്രായീല്‍ സന്തതികളേ, നിങ്ങള്‍ക്ക് ഞാന്‍ ചെയ്തു തന്നിട്ടുള്ള അനുഗ്രഹവും, മറ്റു ജനവിഭാഗങ്ങളേക്കാള്‍ നിങ്ങള്‍ക്ക് ഞാന്‍ ശ്രേഷ്ഠത നല്‍കിയതും നിങ്ങള്‍ ഓര്‍ക്കുക.
\end{malayalam}}
\flushright{\begin{Arabic}
\quranayah[2][48]
\end{Arabic}}
\flushleft{\begin{malayalam}
ഒരാള്‍ക്കും മറ്റൊരാള്‍ക്ക് വേണ്ടി ഒരു ഉപകാരവും ചെയ്യാന്‍ പറ്റാത്ത ഒരു ദിവസത്തെ നിങ്ങള്‍ സൂക്ഷിക്കുക. (അന്ന്‌) ഒരാളില്‍ നിന്നും ഒരു ശുപാര്‍ശയും സ്വീകരിക്കപ്പെടുകയില്ല. ഒരാളില്‍നിന്നും ഒരു പ്രായശ്ചിത്തവും മേടിക്കപ്പെടുകയുമില്ല. അവര്‍ക്ക് ഒരു സഹായവും ലഭിക്കുകയുമില്ല.
\end{malayalam}}
\flushright{\begin{Arabic}
\quranayah[2][49]
\end{Arabic}}
\flushleft{\begin{malayalam}
നിങ്ങളുടെ പുരുഷസന്താനങ്ങളെ അറുകൊല ചെയ്തുകൊണ്ടും, നിങ്ങളുടെ സ്ത്രീജനങ്ങളെ ജീവിക്കാന്‍ വിട്ടുകൊണ്ടും നിങ്ങള്‍ക്ക് നിഷ്‌ഠൂര മര്‍ദ്ദനമേല്‍പിച്ചുകൊണ്ടിരുന്ന ഫിര്‍ഔന്റെകൂട്ടരില്‍ നിന്ന് നിങ്ങളെ നാം രക്ഷപ്പെടുത്തിയ സന്ദര്‍ഭം (ഓര്‍മിക്കുക.) നിങ്ങളുടെ രക്ഷിതാവിങ്കല്‍നിന്നുള്ള ഒരു വലിയ പരീക്ഷണമാണ് അതിലുണ്ടായിരുന്നത്‌.
\end{malayalam}}
\flushright{\begin{Arabic}
\quranayah[2][50]
\end{Arabic}}
\flushleft{\begin{malayalam}
കടല്‍ പിളര്‍ന്ന് നിങ്ങളെ കൊണ്ടു പോയി നാം രക്ഷപ്പെടുത്തുകയും, നിങ്ങള്‍ കണ്ടുകൊണ്ടിരിക്കെ ഫിര്‍ഔന്റെകൂട്ടരെ നാം മുക്കിക്കൊല്ലുകയും ചെയ്ത സന്ദര്‍ഭവും (ഓര്‍മിക്കുക).
\end{malayalam}}
\flushright{\begin{Arabic}
\quranayah[2][51]
\end{Arabic}}
\flushleft{\begin{malayalam}
മൂസാ നബിക്ക് നാല്‍പത് രാവുകള്‍ നാം നിശ്ചയിക്കുകയും അദ്ദേഹം (അതിന്നായി) പോയ ശേഷം നിങ്ങള്‍ അക്രമമായി ഒരു കാളക്കുട്ടിയെ (ദൈവമായി) സ്വീകരിക്കുകയും ചെയ്ത സന്ദര്‍ഭവും (ഓര്‍ക്കുക).
\end{malayalam}}
\flushright{\begin{Arabic}
\quranayah[2][52]
\end{Arabic}}
\flushleft{\begin{malayalam}
എന്നിട്ട് അതിന്ന് ശേഷവും നിങ്ങള്‍ക്ക് നാം മാപ്പുനല്‍കി. നിങ്ങള്‍ നന്ദിയുള്ളവരായിരിക്കുവാന്‍ വേണ്ടി.
\end{malayalam}}
\flushright{\begin{Arabic}
\quranayah[2][53]
\end{Arabic}}
\flushleft{\begin{malayalam}
നിങ്ങള്‍ സന്‍മാര്‍ഗം കണ്ടെത്തുന്നതിന് വേണ്ടി വേദ ഗ്രന്ഥവും, സത്യവും അസത്യവും വേര്‍തിരിക്കുന്ന പ്രമാണവും മൂസാനബിക്ക് നാം നല്‍കിയ സന്ദര്‍ഭവും (ഓര്‍ക്കുക).
\end{malayalam}}
\flushright{\begin{Arabic}
\quranayah[2][54]
\end{Arabic}}
\flushleft{\begin{malayalam}
എന്റെസമുദായമേ, കാളക്കുട്ടിയെ (ദൈവമായി) സ്വീകരിച്ചത് മുഖേന നിങ്ങള്‍ നിങ്ങളോട് തന്നെ അന്യായം ചെയ്തിരിക്കുകയാണ്‌. അതിനാല്‍ നിങ്ങള്‍ നിങ്ങളുടെ സ്രഷ്ടാവിലേക്ക് പശ്ചാത്തപിച്ച് മടങ്ങുകയും (പ്രായശ്ചിത്തമായി) നിങ്ങള്‍ നിങ്ങളെതന്നെ നിഗ്രഹിക്കുകയും ചെയ്യുക. നിങ്ങളുടെ സ്രഷ്ടാവിന്റെഅടുക്കല്‍ അതാണ് നിങ്ങള്‍ക്ക് ഗുണകരം എന്ന് മൂസാ തന്റെജനതയോട് പറഞ്ഞ സന്ദര്‍ഭവും (ഓര്‍മിക്കുക). അനന്തരം അല്ലാഹു നിങ്ങളുടെ പശ്ചാത്താപം സ്വീകരിച്ചു. അവന്‍ പശ്ചാത്താപം ഏറെ സ്വീകരിക്കുന്നവനും കരുണാനിധിയുമത്രെ.
\end{malayalam}}
\flushright{\begin{Arabic}
\quranayah[2][55]
\end{Arabic}}
\flushleft{\begin{malayalam}
ഓ; മൂസാ, ഞങ്ങള്‍ അല്ലാഹുവെ പ്രത്യക്ഷമായി കാണുന്നത് വരെ താങ്കളെ ഞങ്ങള്‍ വിശ്വസിക്കുകയേ ഇല്ല എന്ന് നിങ്ങള്‍ പറഞ്ഞ സന്ദര്‍ഭം (ഓര്‍ക്കുക.) തന്നിമിത്തം നിങ്ങള്‍ നോക്കി നില്‍ക്കെ ഇടിത്തീ നിങ്ങളെ പിടികൂടി.
\end{malayalam}}
\flushright{\begin{Arabic}
\quranayah[2][56]
\end{Arabic}}
\flushleft{\begin{malayalam}
പിന്നീട് നിങ്ങളുടെ മരണത്തിന് ശേഷം നിങ്ങളെ നാം എഴുന്നേല്‍പിച്ചു. നിങ്ങള്‍ നന്ദിയുള്ളവരായിത്തീരാന്‍ വേണ്ടി.
\end{malayalam}}
\flushright{\begin{Arabic}
\quranayah[2][57]
\end{Arabic}}
\flushleft{\begin{malayalam}
നിങ്ങള്‍ക്ക് നാം മേഘത്തണല്‍ നല്‍കുകയും മന്നായും കാടപക്ഷികളും ഇറക്കിത്തരികയും ചെയ്തു. നിങ്ങള്‍ക്ക് നാം നല്‍കിയിട്ടുള്ള വിശിഷ്ടമായ വസ്തുക്കളില്‍ നിന്ന് ഭക്ഷിച്ചുകൊള്ളുക (എന്ന് നാം നിര്‍ദേശിച്ചു). അവര്‍ (എന്നിട്ടും നന്ദികേട് കാണിച്ചവര്‍) നമുക്കൊരു ദ്രോഹവും വരുത്തിയിട്ടില്ല. അവര്‍ അവര്‍ക്ക് തന്നെയാണ് ദ്രോഹമുണ്ടാക്കിക്കൊണ്ടിരിക്കുന്നത്‌.
\end{malayalam}}
\flushright{\begin{Arabic}
\quranayah[2][58]
\end{Arabic}}
\flushleft{\begin{malayalam}
നിങ്ങള്‍ ഈ പട്ടണത്തില്‍ പ്രവേശിക്കുവിന്‍. അവിടെ നിങ്ങള്‍ക്ക് ഇഷ്ടമുള്ളിടത്തുനിന്ന് യഥേഷ്ടം ഭക്ഷിച്ചുകൊള്ളുവിന്‍. തലകുനിച്ചുകൊണ്ട് വാതില്‍ കടക്കുകയും പശ്ചാത്താപ വചനം പറയുകയും ചെയ്യുവിന്‍. നിങ്ങളുടെ പാപങ്ങള്‍ നാം പൊറുത്തുതരികയും, സല്‍പ്രവൃത്തികള്‍ ചെയ്യുന്നവര്‍ക്ക് കൂടുതല്‍ കൂടുതല്‍ അനുഗ്രഹങ്ങള്‍ നല്‍കുകയും ചെയ്യുന്നതാണ് എന്ന് നാം പറഞ്ഞ സന്ദര്‍ഭവും (ഓര്‍ക്കുക).
\end{malayalam}}
\flushright{\begin{Arabic}
\quranayah[2][59]
\end{Arabic}}
\flushleft{\begin{malayalam}
എന്നാല്‍ അക്രമികളായ ആളുകള്‍ അവരോട് നിര്‍ദേശിക്കപ്പെട്ട വാക്കിന്നു പകരം മറ്റൊരു വാക്കാണ് ഉപയോഗിച്ചത്‌. അതിനാല്‍ ആ അക്രമികളുടെ മേല്‍ നാം ആകാശത്തു നിന്ന് ശിക്ഷ ഇറക്കി. കാരണം അവര്‍ ധിക്കാരം കാണിച്ചുകൊണ്ടിരുന്നത് തന്നെ.
\end{malayalam}}
\flushright{\begin{Arabic}
\quranayah[2][60]
\end{Arabic}}
\flushleft{\begin{malayalam}
മൂസാ നബി തന്റെജനതയ്ക്കുവേണ്ടി വെള്ളത്തിനപേക്ഷിച്ച സന്ദര്‍ഭവും (ശ്രദ്ധിക്കുക.) അപ്പോള്‍ നാം പറഞ്ഞു: നിന്റെവടികൊണ്ട് പാറമേല്‍ അടിക്കുക. അങ്ങനെ അതില്‍ നിന്ന് പന്ത്രണ്ട് ഉറവുകള്‍ പൊട്ടി ഒഴുകി. ജനങ്ങളില്‍ ഓരോ വിഭാഗവും അവരവര്‍ക്ക് വെള്ളമെടുക്കാനുള്ള സ്ഥലങ്ങള്‍ മനസ്സിലാക്കി. അല്ലാഹുവിന്റെആഹാരത്തില്‍ നിന്ന് നിങ്ങള്‍ തിന്നുകയും കുടിക്കുകയും ചെയ്തുകൊള്ളൂ. ഭൂമിയില്‍ കുഴപ്പമുണ്ടാക്കി നാശകാരികളായിത്തീരരുത് (എന്ന് നാം അവരോട് നിര്‍ദേശിക്കുകയും ചെയ്തു).
\end{malayalam}}
\flushright{\begin{Arabic}
\quranayah[2][61]
\end{Arabic}}
\flushleft{\begin{malayalam}
ഓ; മൂസാ, ഒരേതരം ആഹാരവുമായി ക്ഷമിച്ചുകഴിയുവാന്‍ ഞങ്ങള്‍ക്ക് സാധിക്കുകയില്ല. അതിനാല്‍ മണ്ണില്‍ മുളച്ചുണ്ടാവുന്ന തരത്തിലുള്ള ചീര, വെള്ളരി, ഗോതമ്പ്‌, പയറ്‌, ഉള്ളി മുതലായവ ഞങ്ങള്‍ക്ക് ഉല്‍പാദിപ്പിച്ചുതരുവാന്‍ താങ്കള്‍ താങ്കളുടെ നാഥനോട് പ്രാര്‍ത്ഥിക്കുക എന്ന് നിങ്ങള്‍ പറഞ്ഞ സന്ദര്‍ഭവും (ഓര്‍ക്കുക) മൂസാ പറഞ്ഞു: കൂടുതല്‍ ഉത്തമമായത് വിട്ട് തികച്ചും താണതരത്തിലുള്ളതാണോ നിങ്ങള്‍ പകരം ആവശ്യപ്പെടുന്നത്‌? എന്നാല്‍ നിങ്ങളൊരു പട്ടണത്തില്‍ ചെന്നിറങ്ങിക്കൊള്ളൂ. നിങ്ങള്‍ ആവശ്യപ്പെടുന്നതെല്ലാം നിങ്ങള്‍ക്കവിടെ കിട്ടും. (ഇത്തരം ദുര്‍വാശികള്‍ കാരണമായി) അവരുടെ മേല്‍ നിന്ദ്യതയും പതിത്വവും അടിച്ചേല്‍പിക്കപ്പെടുകയും, അവര്‍ അല്ലാഹുവിന്റെ കോപത്തിന് പാത്രമായിത്തീരുകയും ചെയ്തു. അവര്‍ അല്ലാഹുവിന്റെ ദൃഷ്ടാന്തങ്ങളെ നിഷേധിക്കുകയും, പ്രവാചകന്‍മാരെ അന്യായമായി കൊലപ്പെടുത്തുകയും ചെയ്തതിന്റെ ഫലമായിട്ടാണത് സംഭവിച്ചത്‌. അവര്‍ ധിക്കാരം കാണിക്കുകയും, അതിക്രമം പ്രവര്‍ത്തിക്കുകയും ചെയ്തതിന്റെ ഫലമായാണത് സംഭവിച്ചത്‌.
\end{malayalam}}
\flushright{\begin{Arabic}
\quranayah[2][62]
\end{Arabic}}
\flushleft{\begin{malayalam}
(മുഹമ്മദ് നബിയില്‍) വിശ്വസിച്ചവരോ, യഹൂദമതം സ്വീകരിച്ചവരോ, ക്രൈസ്തവരോ, സാബികളോ ആരാകട്ടെ, അല്ലാഹുവിലും അന്ത്യദിനത്തിലും വിശ്വസിക്കുകയും, സല്‍കര്‍മ്മം പ്രവര്‍ത്തിക്കുകയും ചെയ്തിട്ടുള്ളവര്‍ക്ക് അവരുടെ രക്ഷിതാവിങ്കല്‍ അവര്‍ അര്‍ഹിക്കുന്ന പ്രതിഫലമുണ്ട്‌. അവര്‍ക്ക് ഭയപ്പെടേണ്ടതില്ല. അവര്‍ ദുഃഖിക്കേണ്ടി വരികയുമില്ല.
\end{malayalam}}
\flushright{\begin{Arabic}
\quranayah[2][63]
\end{Arabic}}
\flushleft{\begin{malayalam}
നാം നിങ്ങളോട് കരാര്‍ വാങ്ങുകയും നിങ്ങള്‍ക്ക് മീതെ പര്‍വ്വതത്തെ നാം ഉയര്‍ത്തിപ്പിടിക്കുകയും ചെയ്ത സന്ദര്‍ഭം (ഓര്‍ക്കുക). നിങ്ങള്‍ക്ക് നാം നല്‍കിയത് ഗൌരവബുദ്ധിയോടെ ഏറ്റെടുക്കുകയും, ദോഷബാധയെ സൂക്ഷിക്കുവാന്‍ വേണ്ടി അതില്‍ നിര്‍ദേശിച്ചത് ഓര്‍മിച്ചുകൊണ്ടിരിക്കുകയും ചെയ്യുക (എന്ന് നാം അനുശാസിച്ചു).
\end{malayalam}}
\flushright{\begin{Arabic}
\quranayah[2][64]
\end{Arabic}}
\flushleft{\begin{malayalam}
എന്നിട്ടതിന് ശേഷവും നിങ്ങള്‍ പുറകോട്ട് പോയി. അല്ലാഹുവിന്റെ അനുഗ്രഹവും അവന്റെ കാരുണ്യവും ഇല്ലായിരുന്നുവെങ്കില്‍ നിങ്ങള്‍ നഷ്ടക്കാരില്‍ പെടുമായിരുന്നു.
\end{malayalam}}
\flushright{\begin{Arabic}
\quranayah[2][65]
\end{Arabic}}
\flushleft{\begin{malayalam}
നിങ്ങളില്‍ നിന്ന് സബ്ത്ത് (ശബ്ബത്ത്‌) ദിനത്തില്‍ അതിക്രമം കാണിച്ചവരെ പറ്റി നിങ്ങളറിഞ്ഞിട്ടുണ്ടല്ലോ. അപ്പോള്‍ നാം അവരോട് പറഞ്ഞു: നിങ്ങള്‍ നിന്ദ്യരായ കുരങ്ങന്‍മാരായിത്തീരുക.
\end{malayalam}}
\flushright{\begin{Arabic}
\quranayah[2][66]
\end{Arabic}}
\flushleft{\begin{malayalam}
അങ്ങനെ നാം അതിനെ (ആ ശിക്ഷയെ) അക്കാലത്തും പില്‍ക്കാലത്തുമുള്ളവര്‍ക്ക് ഒരു ഗുണപാഠവും, സൂക്ഷ്മത പാലിക്കുന്നവര്‍ക്ക് ഒരു തത്വോപദേശവുമാക്കി.
\end{malayalam}}
\flushright{\begin{Arabic}
\quranayah[2][67]
\end{Arabic}}
\flushleft{\begin{malayalam}
അല്ലാഹു നിങ്ങളോട് ഒരു പശുവിനെ അറുക്കുവാന്‍ കല്‍പിക്കുന്നു എന്ന് മൂസാ തന്റെ ജനതയോട് പറഞ്ഞ സന്ദര്‍ഭം (ശ്രദ്ധിക്കുക) അവര്‍ പറഞ്ഞു: താങ്കള്‍ ഞങ്ങളെ പരിഹസിക്കുകയാണോ? അദ്ദേഹം (മൂസാ) പറഞ്ഞു: ഞാന്‍ വിവരംകെട്ടവരില്‍ പെട്ടുപോകാതിരിക്കാന്‍ അല്ലാഹുവില്‍ അഭയം പ്രാപിക്കുന്നു.
\end{malayalam}}
\flushright{\begin{Arabic}
\quranayah[2][68]
\end{Arabic}}
\flushleft{\begin{malayalam}
(അപ്പോള്‍) അവര്‍ പറഞ്ഞു: അത് (പശു) ഏത് തരമായിരിക്കണമെന്ന് ഞങ്ങള്‍ക്ക് വിശദീകരിച്ചു തരാന്‍ ഞങ്ങള്‍ക്ക് വേണ്ടി താങ്കളുടെ രക്ഷിതാവിനോട് പ്രാര്‍ത്ഥിക്കണം. മൂസാ പറഞ്ഞു: പ്രായം വളരെ കൂടിയതോ വളരെ കുറഞ്ഞതോ അല്ലാത്ത ഇടപ്രായത്തിലുള്ള ഒരു പശുവായിരിക്കണം അതെന്നാണ് അവന്‍ (അല്ലാഹു) പറയുന്നത്‌. അതിനാല്‍ കല്‍പിക്കപ്പെടുന്ന പ്രകാരം നിങ്ങള്‍ പ്രവര്‍ത്തിക്കുക.
\end{malayalam}}
\flushright{\begin{Arabic}
\quranayah[2][69]
\end{Arabic}}
\flushleft{\begin{malayalam}
അവര്‍ പറഞ്ഞു: അതിന്റെ നിറമെന്തായിരിക്കണമെന്ന് ഞങ്ങള്‍ക്ക് വിശദീകരിച്ചുതരുവാന്‍ ഞങ്ങള്‍ക്ക് വേണ്ടി താങ്കള്‍ താങ്കളുടെ രക്ഷിതാവിനോട് പ്രാര്‍ത്ഥിക്കണം. മൂസാ പറഞ്ഞു: കാണികള്‍ക്ക് കൗതുകം തോന്നിക്കുന്ന, തെളിഞ്ഞ മഞ്ഞനിറമുള്ള ഒരു പശുവായിരിക്കണം അതെന്നാണ് അവന്‍ (അല്ലാഹു) പറയുന്നത്‌.
\end{malayalam}}
\flushright{\begin{Arabic}
\quranayah[2][70]
\end{Arabic}}
\flushleft{\begin{malayalam}
അവര്‍ പറഞ്ഞു: അത് ഏത് തരമാണെന്ന് ഞങ്ങള്‍ക്ക് വ്യക്തമാക്കി തരാന്‍ നിന്റെ രക്ഷിതാവിനോട് ഞങ്ങള്‍ക്ക് വേണ്ടി പ്രാര്‍ത്ഥിക്കുക. തീര്‍ച്ചയായും പശുക്കള്‍ പരസ്പരം സാദൃശ്യമുള്ളതായി ഞങ്ങള്‍ക്ക് തോന്നുന്നു. അല്ലാഹു ഉദ്ദേശിച്ചാല്‍ അവന്റെ മാര്‍ഗനിര്‍ദേശപ്രകാരം തീര്‍ച്ചയായും ഞങ്ങള്‍ പ്രവര്‍ത്തിക്കാം.
\end{malayalam}}
\flushright{\begin{Arabic}
\quranayah[2][71]
\end{Arabic}}
\flushleft{\begin{malayalam}
(അപ്പോള്‍) മൂസാ പറഞ്ഞു: നിലം ഉഴുതുവാനോ വിള നനയ്ക്കുവാനോ ഉപയോഗപ്പെടുത്തുന്നതല്ലാത്ത, പാടുകളൊന്നുമില്ലാത്ത അവികലമായ ഒരു പശുവായിരിക്കണം അതെന്നാണ് അല്ലാഹു പറയുന്നത്‌. അവര്‍ പറഞ്ഞു: ഇപ്പോഴാണ് താങ്കള്‍ ശരിയായ വിവരം വെളിപ്പെടുത്തിയത്‌. അങ്ങനെ അവര്‍ അതിനെ അറുത്തു. അവര്‍ക്കത് നിറവേറ്റുക എളുപ്പമായിരുന്നില്ല.
\end{malayalam}}
\flushright{\begin{Arabic}
\quranayah[2][72]
\end{Arabic}}
\flushleft{\begin{malayalam}
(ഇസ്രായീല്‍ സന്തതികളേ), നിങ്ങള്‍ ഒരാളെ കൊലപ്പെടുത്തുകയും, അന്യോന്യം കുറ്റം ആരോപിച്ചുകൊണ്ട് ഒഴിഞ്ഞ് മാറുകയും ചെയ്ത സന്ദര്‍ഭവും (ഓര്‍ക്കുക.) എന്നാല്‍ നിങ്ങള്‍ ഒളിച്ച് വെക്കുന്നത് അല്ലാഹു വെളിയില്‍ കൊണ്ടുവരിക തന്നെ ചെയ്യും.
\end{malayalam}}
\flushright{\begin{Arabic}
\quranayah[2][73]
\end{Arabic}}
\flushleft{\begin{malayalam}
അപ്പോള്‍ നാം പറഞ്ഞു: നിങ്ങള്‍ അതിന്റെ (പശുവിന്റെ) ഒരംശംകൊണ്ട് ആ മൃതദേഹത്തില്‍ അടിക്കുക. അപ്രകാരം അല്ലാഹു മരണപ്പെട്ടവരെ ജീവിപ്പിക്കുന്നു. നിങ്ങള്‍ ചിന്തിക്കുവാന്‍ വേണ്ടി അവന്റെ ദൃഷ്ടാന്തങ്ങള്‍ നിങ്ങള്‍ക്കവന്‍ കാണിച്ചുതരുന്നു.
\end{malayalam}}
\flushright{\begin{Arabic}
\quranayah[2][74]
\end{Arabic}}
\flushleft{\begin{malayalam}
പിന്നീട് അതിന് ശേഷവും നിങ്ങളുടെ മനസ്സുകള്‍ കടുത്തുപോയി. അവ പാറപോലെയോ അതിനെക്കാള്‍ കടുത്തതോ ആയി ഭവിച്ചു. പാറകളില്‍ ചിലതില്‍ നിന്ന് നദികള്‍ പൊട്ടി ഒഴുകാറുണ്ട്‌. ചിലത് പിളര്‍ന്ന് വെള്ളം പുറത്ത് വരുന്നു. ചിലത് ദൈവഭയത്താല്‍ താഴോട്ട് ഉരുണ്ടു വീഴുകയും ചെയ്യുന്നു. നിങ്ങള്‍ പ്രവര്‍ത്തിക്കുന്ന യാതൊന്നിനെപറ്റിയും അല്ലാഹു ഒട്ടും അശ്രദ്ധനല്ല.
\end{malayalam}}
\flushright{\begin{Arabic}
\quranayah[2][75]
\end{Arabic}}
\flushleft{\begin{malayalam}
(സത്യവിശ്വാസികളേ), നിങ്ങളെ അവര്‍ (യഹൂദര്‍) വിശ്വസിക്കുമെന്ന് നിങ്ങള്‍ മോഹിക്കുകയാണോ? അവരില്‍ ഒരു വിഭാഗം അല്ലാഹുവിന്റെ വചനങ്ങള്‍ കേള്‍ക്കുകയും, അത് ശരിക്കും മനസ്സിലാക്കിയതിന് ശേഷം ബോധപൂര്‍വ്വം തന്നെ അതില്‍ കൃത്രിമം കാണിച്ചുകൊണ്ടിരിക്കുകയുമാണല്ലോ.
\end{malayalam}}
\flushright{\begin{Arabic}
\quranayah[2][76]
\end{Arabic}}
\flushleft{\begin{malayalam}
വിശ്വസിച്ചവരെ കണ്ടുമുട്ടുമ്പോള്‍ അവര്‍ പറയും: ഞങ്ങള്‍ വിശ്വസിച്ചിരിക്കുന്നു എന്ന്‌. അവര്‍ തമ്മില്‍ തനിച്ചുകണ്ടുമുട്ടുമ്പോള്‍ (പരസ്പരം കുറ്റപ്പെടുത്തിക്കൊണ്ട്‌) അവര്‍ പറയും: അല്ലാഹു നിങ്ങള്‍ക്ക് വെളിപ്പെടുത്തിത്തന്ന കാര്യങ്ങള്‍ ഇവര്‍ക്ക് നിങ്ങള്‍ പറഞ്ഞുകൊടുക്കുകയാണോ ? നിങ്ങളുടെ രക്ഷിതാവിന്റെ സന്നിധിയില്‍ അവര്‍ നിങ്ങള്‍ക്കെതിരില്‍ അത് വെച്ച് ന്യായവാദം നടത്താന്‍ വേണ്ടി. നിങ്ങളെന്താണ് ചിന്തിക്കാത്തത് ?
\end{malayalam}}
\flushright{\begin{Arabic}
\quranayah[2][77]
\end{Arabic}}
\flushleft{\begin{malayalam}
എന്നാല്‍ അവര്‍ക്കറിഞ്ഞുകൂടേ; അവര്‍ രഹസ്യമാക്കുന്നതും പരസ്യമാക്കുന്നതുമെല്ലാം അല്ലാഹു അറിയുന്നുണ്ടെന്ന് ?
\end{malayalam}}
\flushright{\begin{Arabic}
\quranayah[2][78]
\end{Arabic}}
\flushleft{\begin{malayalam}
അക്ഷരജ്ഞാനമില്ലാത്ത ചില ആളുകളും അവരില്‍ (ഇസ്രായീല്യരില്‍) ഉണ്ട്‌. ചില വ്യാമോഹങ്ങള്‍ വെച്ച് പുലര്‍ത്തുന്നതല്ലാതെ വേദ ഗ്രന്ഥത്തെപ്പറ്റി അവര്‍ക്ക് ഒന്നുമറിയില്ല. അവര്‍ ഊഹത്തെ അവലംബമാക്കുക മാത്രമാണ് ചെയ്യുന്നത്‌.
\end{malayalam}}
\flushright{\begin{Arabic}
\quranayah[2][79]
\end{Arabic}}
\flushleft{\begin{malayalam}
എന്നാല്‍ സ്വന്തം കൈകള്‍ കൊണ്ട് ഗ്രന്ഥം എഴുതിയുണ്ടാക്കുകയും എന്നിട്ട് അത് അല്ലാഹുവിങ്കല്‍ നിന്ന് ലഭിച്ചതാണെന്ന് പറയുകയും ചെയ്യുന്നവര്‍ക്കാകുന്നു നാശം. അത് മുഖേന വില കുറഞ്ഞ നേട്ടങ്ങള്‍ കരസ്ഥമാക്കാന്‍ വേണ്ടിയാകുന്നു (അവരിത് ചെയ്യുന്നത്‌.) അവരുടെ കൈകള്‍ എഴുതിയ വകയിലും അവര്‍ സമ്പാദിക്കുന്ന വകയിലും അവര്‍ക്ക് നാശം.
\end{malayalam}}
\flushright{\begin{Arabic}
\quranayah[2][80]
\end{Arabic}}
\flushleft{\begin{malayalam}
അവര്‍ (യഹൂദര്‍) പറഞ്ഞു: എണ്ണപ്പെട്ട ദിവസങ്ങളിലല്ലാതെ ഞങ്ങളെ നരക ശിക്ഷ ബാധിക്കുകയേ ഇല്ല. ചോദിക്കുക: നിങ്ങള്‍ അല്ലാഹുവിങ്കല്‍നിന്ന് വല്ല കരാറും വാങ്ങിയിട്ടുണ്ടോ ? എന്നാല്‍ തീര്‍ച്ചയായും അല്ലാഹു തന്റെ കരാര്‍ ലംഘിക്കുകയില്ല. അതല്ല, നിങ്ങള്‍ക്ക് അറിവില്ലാത്തത് അല്ലാഹുവിന്റെ പേരില്‍ നിങ്ങള്‍ പറഞ്ഞുണ്ടാക്കുകയാണോ ?
\end{malayalam}}
\flushright{\begin{Arabic}
\quranayah[2][81]
\end{Arabic}}
\flushleft{\begin{malayalam}
അങ്ങനെയല്ല. ആര്‍ ദുഷ്കൃത്യം ചെയ്യുകയും പാപത്തിന്റെ വലയത്തില്‍ പെടുകയും ചെയ്യുന്നുവോ അവരാകുന്നു നരകാവകാശികള്‍. അവരതില്‍ നിത്യവാസികളായിരിക്കും.
\end{malayalam}}
\flushright{\begin{Arabic}
\quranayah[2][82]
\end{Arabic}}
\flushleft{\begin{malayalam}
വിശ്വസിക്കുകയും സല്‍കര്‍മ്മങ്ങള്‍ അനുഷ്ഠിക്കുകയും ചെയ്തതാരോ അവരാകുന്നു സ്വര്‍ഗാവകാശികള്‍. അവരതില്‍ നിത്യവാസികളായിരിക്കും.
\end{malayalam}}
\flushright{\begin{Arabic}
\quranayah[2][83]
\end{Arabic}}
\flushleft{\begin{malayalam}
അല്ലാഹുവെ അല്ലാതെ നിങ്ങള്‍ ആരാധിക്കരുത്‌; മാതാപിതാക്കള്‍ക്കും ബന്ധുക്കള്‍ക്കും അനാഥകള്‍ക്കും അഗതികള്‍ക്കും നന്‍മ ചെയ്യണം; ജനങ്ങളോട് നല്ല വാക്ക് പറയണം; പ്രാര്‍ത്ഥന മുറ പ്രകാരം നിര്‍വഹിക്കുകയും സകാത്ത് നല്‍കുകയും ചെയ്യണം എന്നെല്ലാം നാം ഇസ്രായീല്യരോട് കരാര്‍ വാങ്ങിയ സന്ദര്‍ഭം (ഓര്‍ക്കുക). (എന്നാല്‍ ഇസ്രായീല്‍ സന്തതികളേ,) പിന്നീട് നിങ്ങളില്‍ കുറച്ച് പേരൊഴികെ മറ്റെല്ലാവരും വിമുഖതയോടെ പിന്‍മാറിക്കളയുകയാണ് ചെയ്തത്‌.
\end{malayalam}}
\flushright{\begin{Arabic}
\quranayah[2][84]
\end{Arabic}}
\flushleft{\begin{malayalam}
നിങ്ങള്‍ അന്യോന്യം രക്തം ചിന്തുകയില്ലെന്നും, സ്വന്തമാളുകളെ കുടിയൊഴിപ്പിക്കുകയില്ലെന്നും നിങ്ങളോട് നാം ഉറപ്പ് വാങ്ങിയ സന്ദര്‍ഭവും (ഓര്‍ക്കുക). എന്നിട്ട് നിങ്ങളത് സമ്മതിച്ച് ശരിവെക്കുകയും ചെയ്തു. നിങ്ങളതിന് സാക്ഷികളുമാകുന്നു.
\end{malayalam}}
\flushright{\begin{Arabic}
\quranayah[2][85]
\end{Arabic}}
\flushleft{\begin{malayalam}
എന്നിട്ടും നിങ്ങളിതാ സ്വജനങ്ങളെ കൊന്നുകൊണ്ടിരിക്കുന്നു. നിങ്ങളിലൊരു വിഭാഗത്തെ തന്നെ അവരുടെ വീടുകളില്‍ നിന്നും ഇറക്കി വിട്ടുകൊണ്ടിരിക്കുന്നു. തികച്ചും കുറ്റകരമായും അതിക്രമപരമായും അവര്‍ക്കെതിരില്‍ നിങ്ങള്‍ അന്യോന്യം സഹായിക്കുകയും ചെയ്യുന്നു. അവര്‍ നിങ്ങളുടെ അടുത്ത് യുദ്ധത്തടവുകാരായി വന്നാല്‍ നിങ്ങള്‍ മോചനമൂല്യം നല്‍കി അവരെ മോചിപ്പിക്കുകയും ചെയ്യുന്നു. യഥാര്‍ത്ഥത്തില്‍ അവരെ പുറം തള്ളുന്നത് തന്നെ നിങ്ങള്‍ക്ക് നിഷിദ്ധമായിരുന്നു. നിങ്ങള്‍ വേദ ഗ്രന്ഥത്തിലെ ചില ഭാഗങ്ങള്‍ വിശ്വസിക്കുകയും മറ്റു ചിലത് തള്ളിക്കളയുകയുമാണോ ? എന്നാല്‍ നിങ്ങളില്‍ നിന്ന് അപ്രകാരം പ്രവര്‍ത്തിക്കുന്നവര്‍ക്ക് ഇഹലോകജീവിതത്തില്‍ അപമാനമല്ലാതെ മറ്റൊരു ഫലവും കിട്ടാനില്ല. ഉയിര്‍ത്തെഴുന്നേല്‍പിന്റെ നാളിലാവട്ടെ അതി കഠിനമായ ശിക്ഷയിലേക്ക് അവര്‍ തള്ളപ്പെടുകയും ചെയ്യും. നിങ്ങളുടെ പ്രവര്‍ത്തനങ്ങളെപ്പറ്റിയൊന്നും അല്ലാഹു അശ്രദ്ധനല്ല.
\end{malayalam}}
\flushright{\begin{Arabic}
\quranayah[2][86]
\end{Arabic}}
\flushleft{\begin{malayalam}
പരലോകം വിറ്റ് ഇഹലോകജീവിതം വാങ്ങിയവരാകുന്നു അത്തരക്കാര്‍. അവര്‍ക്ക് ശിക്ഷയില്‍ ഇളവ് നല്‍കപ്പെടുകയില്ല. അവര്‍ക്ക് ഒരു സഹായവും ലഭിക്കുകയുമില്ല.
\end{malayalam}}
\flushright{\begin{Arabic}
\quranayah[2][87]
\end{Arabic}}
\flushleft{\begin{malayalam}
മൂസായ്ക്ക് നാം ഗ്രന്ഥം നല്‍കി. അദ്ദേഹത്തിന് ശേഷം തുടര്‍ച്ചയായി നാം ദൂതന്‍മാരെ അയച്ചുകൊണ്ടിരുന്നു. മര്‍യമിന്റെ മകനായ ഈസാക്ക് നാം വ്യക്തമായ ദൃഷ്ടാന്തങ്ങള്‍ നല്‍കുകയും, അദ്ദേഹത്തിന് നാം പരിശുദ്ധാത്മാവിന്റെ പിന്‍ബലം നല്‍കുകയും ചെയ്തു. എന്നിട്ട് നിങ്ങളുടെ മനസ്സിന് പിടിക്കാത്ത കാര്യങ്ങളുമായി വല്ല ദൈവദൂതനും നിങ്ങളുടെ അടുത്ത് വരുമ്പോഴൊക്കെ നിങ്ങള്‍ അഹങ്കരിക്കുകയും, ചില ദൂതന്‍മാരെ നിങ്ങള്‍ തള്ളിക്കളയുകയും, മറ്റു ചിലരെ നിങ്ങള്‍ വധിക്കുകയും ചെയ്യുകയാണോ?
\end{malayalam}}
\flushright{\begin{Arabic}
\quranayah[2][88]
\end{Arabic}}
\flushleft{\begin{malayalam}
അവര്‍ പറഞ്ഞു: ഞങ്ങളുടെ മനസ്സുകള്‍ അടഞ്ഞുകിടക്കുകയാണ്‌. എന്നാല്‍ (അതല്ല ശരി) അവരുടെ നിഷേധം കാരണമായി അല്ലാഹു അവരെ ശപിച്ചിരിക്കുകയാണ്‌. അതിനാല്‍ വളരെ കുറച്ചേ അവര്‍ വിശ്വസിക്കുന്നുള്ളൂ.
\end{malayalam}}
\flushright{\begin{Arabic}
\quranayah[2][89]
\end{Arabic}}
\flushleft{\begin{malayalam}
അവരുടെ കൈവശമുള്ള വേദത്തെ ശരിവെക്കുന്ന ഒരു ഗ്രന്ഥം (ഖുര്‍ആന്‍) അല്ലാഹുവിങ്കല്‍ നിന്ന് അവര്‍ക്ക് വന്നുകിട്ടിയപ്പോള്‍ (അവരത് തള്ളിക്കളയുകയാണ് ചെയ്തത്‌). അവരാകട്ടെ (അത്തരം ഒരു ഗ്രന്ഥവുമായി വരുന്ന പ്രവാചകന്‍ മുഖേന) അവിശ്വാസികള്‍ക്കെതിരില്‍ വിജയം നേടികൊടുക്കുവാന്‍ വേണ്ടി മുമ്പ് (അല്ലാഹുവിനോട്‌) പ്രാര്‍ത്ഥിക്കാറുണ്ടായിരുന്നു. അവര്‍ക്ക് സുപരിചിതമായ ആ സന്ദേശം വന്നെത്തിയപ്പോള്‍ അവരത് നിഷേധിക്കുകയാണ് ചെയ്തത്‌. അതിനാല്‍ ആ നിഷേധികള്‍ക്കത്രെ അല്ലാഹുവിന്റെ ശാപം.
\end{malayalam}}
\flushright{\begin{Arabic}
\quranayah[2][90]
\end{Arabic}}
\flushleft{\begin{malayalam}
അല്ലാഹു തന്റെ ദാസന്‍മാരില്‍ നിന്ന് താന്‍ ഇച്ഛിക്കുന്നവരുടെ മേല്‍ തന്റെ അനുഗ്രഹം ഇറക്കികൊടുക്കുന്നതിലുള്ള ഈര്‍ഷ്യ നിമിത്തം അല്ലാഹു അവതരിപ്പിച്ച സന്ദേശത്തെ അവിശ്വസിക്കുക വഴി തങ്ങളുടെ ആത്മാക്കളെ വിറ്റുകൊണ്ടവര്‍ വാങ്ങിയ വില എത്ര ചീത്ത! അങ്ങനെ അവര്‍ കോപത്തിനു മേല്‍ കോപത്തിനു പാത്രമായി തീര്‍ന്നു. സത്യനിഷേധികള്‍ക്കത്രെ നിന്ദ്യമായ ശിക്ഷയുള്ളത്‌.
\end{malayalam}}
\flushright{\begin{Arabic}
\quranayah[2][91]
\end{Arabic}}
\flushleft{\begin{malayalam}
അല്ലാഹു അവതരിപ്പിച്ചതില്‍ (ഖുര്‍ആനില്‍) നിങ്ങള്‍ വിശ്വസിക്കൂ എന്ന് അവരോട് പറയപ്പെട്ടാല്‍, ഞങ്ങള്‍ക്ക് അവതീര്‍ണ്ണമായ സന്ദേശത്തില്‍ ഞങ്ങള്‍ വിശ്വസിക്കുന്നുണ്ട് എന്നാണവര്‍ പറയുക. അതിനപ്പുറമുള്ളത് അവര്‍ നിഷേധിക്കുകയും ചെയ്യുന്നു. അവരുടെ പക്കലുള്ള വേദത്തെ ശരിവെക്കുന്ന സത്യസന്ദേശമാണ് താനും അത് (ഖുര്‍ആന്‍). (നബിയേ,) പറയുക: നിങ്ങള്‍ വിശ്വാസികളാണെങ്കില്‍ പിന്നെ എന്തിനായിരുന്നു മുമ്പൊക്കെ അല്ലാഹുവിന്റെ പ്രവാചകന്‍മാരെ നിങ്ങള്‍ വധിച്ചുകൊണ്ടിരുന്നത്‌?
\end{malayalam}}
\flushright{\begin{Arabic}
\quranayah[2][92]
\end{Arabic}}
\flushleft{\begin{malayalam}
സ്പഷ്ടമായ തെളിവുകളും കൊണ്ട് മൂസാ നിങ്ങളുടെ അടുത്ത് വരികയുണ്ടായി. എന്നിട്ടതിന് ശേഷവും നിങ്ങള്‍ അന്യായമായിക്കൊണ്ട് കാളക്കുട്ടിയെ ദൈവമാക്കുകയാണല്ലോ ചെയ്തത്‌.
\end{malayalam}}
\flushright{\begin{Arabic}
\quranayah[2][93]
\end{Arabic}}
\flushleft{\begin{malayalam}
നിങ്ങളോട് നാം കരാര്‍ വാങ്ങുകയും, നിങ്ങള്‍ക്കു മീതെ പര്‍വ്വതത്തെ നാം ഉയര്‍ത്തിപ്പിടിക്കുകയും ചെയ്ത സന്ദര്‍ഭവും (ശ്രദ്ധിക്കുക). നിങ്ങള്‍ക്ക് നാം നല്‍കിയ സന്ദേശം മുറുകെപിടിക്കുകയും (നമ്മുടെ കല്‍പനകള്‍) ശ്രദ്ധിച്ചു കേള്‍ക്കുകയും ചെയ്യുക (എന്ന് നാം അനുശാസിച്ചു). അപ്പോള്‍ അവര്‍ പറഞ്ഞു: ഞങ്ങള്‍ കേട്ടിരിക്കുന്നു. അനുസരിക്കേണ്ടെന്നു വെക്കുകയും ചെയ്തിരിക്കുന്നു. അവരുടെ നിഷേധസ്വഭാവത്തിന്റെ ഫലമായി കാളക്കുട്ടിയോടുള്ള ഭക്തി അവരുടെ മനസ്സുകളില്‍ ലയിച്ചു ചേര്‍ന്നു കഴിഞ്ഞിരുന്നു. (നബിയേ,) പറയുക: നിങ്ങള്‍ വിശ്വാസികളാണെങ്കില്‍ ആ വിശ്വാസം നിങ്ങളോട് നിര്‍ദേശിക്കുന്ന കാര്യം വളരെ ചീത്തതന്നെ.
\end{malayalam}}
\flushright{\begin{Arabic}
\quranayah[2][94]
\end{Arabic}}
\flushleft{\begin{malayalam}
നീ അവരോട് (യഹൂദരോട്‌) പറയുക: മറ്റാര്‍ക്കും നല്‍കാതെ നിങ്ങള്‍ക്കുമാത്രമായി അല്ലാഹു നീക്കിവെച്ചതാണ് പരലോകവിജയമെങ്കില്‍ നിങ്ങള്‍ മരിക്കുവാന്‍ കൊതിച്ചുകൊള്ളുക. നിങ്ങളുടെ വാദം സത്യമാണെങ്കില്‍ (അതാണല്ലോ വേണ്ടത്‌.)
\end{malayalam}}
\flushright{\begin{Arabic}
\quranayah[2][95]
\end{Arabic}}
\flushleft{\begin{malayalam}
എന്നാല്‍ അവരുടെ കൈകള്‍ മുന്‍കൂട്ടി ചെയ്തുവെച്ചത് (ദുഷ്കൃത്യങ്ങള്‍) കാരണമായി അവരൊരിക്കലും മരണത്തെ കൊതിക്കുകയില്ല. അതിക്രമകാരികളെപറ്റി സൂക്ഷ്മജ്ഞാനമുള്ളവനാകുന്നു അല്ലാഹു.
\end{malayalam}}
\flushright{\begin{Arabic}
\quranayah[2][96]
\end{Arabic}}
\flushleft{\begin{malayalam}
തീര്‍ച്ചയായും ജനങ്ങളില്‍ വെച്ച് ജീവിതത്തോട് ഏറ്റവും ആര്‍ത്തിയുള്ളവരായി അവരെ (യഹൂദരെ) നിനക്ക് കാണാം; ബഹുദൈവവിശ്വാസികളെക്കാള്‍ പോലും. അവരില്‍ ഓരോരുത്തരും കൊതിക്കുന്നത് തനിക്ക് ആയിരം കൊല്ലത്തെ ആയുസ്സ് കിട്ടിയിരുന്നെങ്കില്‍ എന്നാണ്‌. ഒരാള്‍ക്ക് ദീര്‍ഘായുസ്സ് ലഭിക്കുക എന്നത് അയാളെ ദൈവിക ശിക്ഷയില്‍ നിന്ന് അകറ്റിക്കളയുന്ന കാര്യമല്ല. അവര്‍ പ്രവര്‍ത്തിക്കുന്നതെല്ലാം സൂക്ഷ്മമായി അറിയുന്നവനാകുന്നു അല്ലാഹു.
\end{malayalam}}
\flushright{\begin{Arabic}
\quranayah[2][97]
\end{Arabic}}
\flushleft{\begin{malayalam}
(നബിയേ,) പറയുക: (ഖുര്‍ആന്‍ എത്തിച്ചുതരുന്ന) ജിബ്‌രീല്‍ എന്ന മലക്കിനോടാണ് ആര്‍ക്കെങ്കിലും ശത്രുതയെങ്കില്‍ അദ്ദേഹമത് നിന്റെ മനസ്സില്‍ അവതരിപ്പിച്ചത് അല്ലാഹുവിന്റെ ഉത്തരവനുസരിച്ച് മാത്രമാണ്‌. മുന്‍വേദങ്ങളെ ശരിവെച്ചുകൊണ്ടുള്ളതും, വിശ്വാസികള്‍ക്ക് വഴി കാട്ടുന്നതും, സന്തോഷവാര്‍ത്ത നല്‍കുന്നതുമായിട്ടാണ് (അത് അവതരിച്ചിട്ടുള്ളത്‌).
\end{malayalam}}
\flushright{\begin{Arabic}
\quranayah[2][98]
\end{Arabic}}
\flushleft{\begin{malayalam}
ആര്‍ക്കെങ്കിലും അല്ലാഹുവോടും അവന്റെ മലക്കുകളോടും അവന്റെ ദൂതന്‍മാരോടും ജിബ്‌രീലിനോടും മീകാഈലിനോടുമെല്ലാം ശത്രുതയാണെങ്കില്‍ ആ നിഷേധികളുടെ ശത്രുതന്നെയാകുന്നു അല്ലാഹു.
\end{malayalam}}
\flushright{\begin{Arabic}
\quranayah[2][99]
\end{Arabic}}
\flushleft{\begin{malayalam}
നാം നിനക്ക് അവതിരിപ്പിച്ചു തന്നിട്ടുള്ളത് സ്പഷ്ടമായ ദൃഷ്ടാന്തങ്ങളാകുന്നു. ധിക്കാരികളല്ലാതെ മറ്റാരും അവയെ നിഷേധിക്കുകയില്ല.
\end{malayalam}}
\flushright{\begin{Arabic}
\quranayah[2][100]
\end{Arabic}}
\flushleft{\begin{malayalam}
അവര്‍ (യഹൂദര്‍) ഏതൊരു കരാര്‍ ചെയ്തു കഴിയുമ്പോഴും അവരില്‍ ഒരു വിഭാഗം അത് വലിച്ചെറിയുകയാണോ? തന്നെയുമല്ല, അവരില്‍ അധികപേര്‍ക്കും വിശ്വാസം തന്നെയില്ല.
\end{malayalam}}
\flushright{\begin{Arabic}
\quranayah[2][101]
\end{Arabic}}
\flushleft{\begin{malayalam}
അവരുടെ പക്കലുള്ള വേദത്തെ ശരിവെച്ചു കൊണ്ട് അല്ലാഹുവിന്റെ ഒരു ദൂതന്‍ അവരുടെ അടുത്ത് ചെന്നപ്പോള്‍ ആ വേദക്കാരില്‍ ഒരു വിഭാഗം അല്ലാഹുവിന്റെ ഗ്രന്ഥത്തെ യാതൊരു പരിചയവുമില്ലാത്തവരെ പോലെ പുറകോട്ട് വലിച്ചെറിയുകയാണ് ചെയ്തത്‌.
\end{malayalam}}
\flushright{\begin{Arabic}
\quranayah[2][102]
\end{Arabic}}
\flushleft{\begin{malayalam}
സുലൈമാന്‍ നബിയുടെ രാജവാഴ്ചയുടെ (രഹസ്യമെന്ന) പേരില്‍ പിശാചുക്കള്‍ പറഞ്ഞുപരത്തിക്കൊണ്ടിരുന്നത് അവര്‍ (ഇസ്രായീല്യര്‍) പിന്‍പറ്റുകയും ചെയ്തു. സുലൈമാന്‍ നബി ദൈവനിഷേധം കാണിച്ചിട്ടില്ല. എന്നാല്‍ ജനങ്ങള്‍ക്ക് മാന്ത്രികവിദ്യ പഠിപ്പിച്ചുകൊടുത്ത് കൊണ്ട് പിശാചുക്കളാണ് ദൈവ നിഷേധത്തില്‍ ഏര്‍പെട്ടത്‌. ബാബിലോണില്‍ ഹാറൂത്തെന്നും മാറൂത്തെന്നും പേരുള്ള രണ്ടു മാലാഖമാര്‍ക്ക് ലഭിച്ചതിനെയും (പറ്റി പിശാചുക്കള്‍ പറഞ്ഞുണ്ടാക്കിക്കൊണ്ടിരുന്നത് അവര്‍ പിന്തുടര്‍ന്നു). എന്നാല്‍ ഹാറൂത്തും മാറൂത്തും ഏതൊരാള്‍ക്ക് പഠിപ്പിക്കുമ്പോഴും, ഞങ്ങളുടേത് ഒരു പരീക്ഷണം മാത്രമാകുന്നു; അതിനാല്‍ (ഇത് ഉപയോഗിച്ച്‌) ദൈവനിഷേധത്തില്‍ ഏര്‍പെടരുത് എന്ന് അവര്‍ പറഞ്ഞുകൊടുക്കാതിരുന്നില്ല. അങ്ങനെ അവരില്‍ നിന്ന് ഭാര്യാഭര്‍ത്താക്കന്‍മാര്‍ക്കിടയില്‍ ഭിന്നതയുണ്ടാക്കുവാനുള്ള തന്ത്രങ്ങള്‍ ജനങ്ങള്‍ പഠിച്ച് കൊണ്ടിരുന്നു. എന്നാല്‍ അല്ലാഹുവിന്റെ അനുമതി കൂടാതെ അതുകൊണ്ട് യാതൊരാള്‍ക്കും ഒരു ദ്രോഹവും ചെയ്യാന്‍ അവര്‍ക്ക് കഴിയില്ല. അവര്‍ക്ക് തന്നെ ഉപദ്രവമുണ്ടാക്കുന്നതും ഒരു പ്രയോജനവും ചെയ്യാത്തതുമായ കാര്യമാണ് അവര്‍ പഠിച്ചു കൊണ്ടിരുന്നത്‌. അത് (ആ വിദ്യ) ആര്‍ വാങ്ങി (കൈവശപ്പെടുത്തി) യോ അവര്‍ക്ക് പരലോകത്ത് യാതൊരു വിഹിതവുമുണ്ടാവില്ലെന്ന് അവര്‍ ഗ്രഹിച്ചുകഴിഞ്ഞിട്ടുണ്ട്‌. അവരുടെ ആത്മാവുകളെ വിറ്റ് അവര്‍ വാങ്ങിയ വില വളരെ ചീത്ത തന്നെ. അവര്‍ക്ക് വിവരമുണ്ടായിരുന്നെങ്കില്‍!
\end{malayalam}}
\flushright{\begin{Arabic}
\quranayah[2][103]
\end{Arabic}}
\flushleft{\begin{malayalam}
അവര്‍ വിശ്വസിക്കുകയും ദോഷബാധയെ സൂക്ഷിക്കുകയും ചെയ്തിരുന്നാല്‍ അല്ലാഹുവിങ്കല്‍ നിന്ന് ലഭിക്കുന്ന പ്രതിഫലം എത്രയോ ഉത്തമമാകുന്നു. അവരത് മനസ്സിലാക്കിയിരുന്നെങ്കില്‍!
\end{malayalam}}
\flushright{\begin{Arabic}
\quranayah[2][104]
\end{Arabic}}
\flushleft{\begin{malayalam}
ഹേ: സത്യവിശ്വാസികളേ, നിങ്ങള്‍ (നബിയോട്‌) റാഇനാ എന്ന് പറയരുത്‌. പകരം ഉന്‍ളുര്‍നാ എന്ന് പറയുകയും ശ്രദ്ധിച്ച് കേള്‍ക്കുകയും ചെയ്യുക. സത്യനിഷേധികള്‍ക്ക് വേദനയേറിയ ശിക്ഷയുണ്ട്‌.
\end{malayalam}}
\flushright{\begin{Arabic}
\quranayah[2][105]
\end{Arabic}}
\flushleft{\begin{malayalam}
നിങ്ങളുടെ രക്ഷിതാവില്‍ നിന്നും വല്ല നന്‍മയും നിങ്ങളുടെ മേല്‍ ഇറക്കപ്പെടുന്നത് വേദക്കാരിലും ബഹുദൈവാരാധകന്‍മാരിലും പെട്ട സത്യനിഷേധികള്‍ ഒട്ടും ഇഷ്ടപ്പെടുന്നില്ല. അല്ലാഹു അവന്റെ കാരുണ്യം കൊണ്ട് അവന്‍ ഇച്ഛിക്കുന്നവരെ പ്രത്യേകം അനുഗ്രഹിക്കുന്നു. അല്ലാഹു മഹത്തായ അനുഗ്രഹമുള്ളവനാണ്‌.
\end{malayalam}}
\flushright{\begin{Arabic}
\quranayah[2][106]
\end{Arabic}}
\flushleft{\begin{malayalam}
വല്ല ആയത്തും നാം ദുര്‍ബലപ്പെടുത്തുകയോ വിസ്മരിപ്പിക്കുകയോ ചെയ്യുകയാണെങ്കില്‍ പകരം അതിനേക്കാള്‍ ഉത്തമമായതോ അതിന് തുല്യമായതോ നാം കൊണ്ടുവരുന്നതാണ്‌. നിനക്കറിഞ്ഞു കൂടേ; അല്ലാഹു എല്ലാകാര്യത്തിനും കഴിവുള്ളവനാണെന്ന്‌?
\end{malayalam}}
\flushright{\begin{Arabic}
\quranayah[2][107]
\end{Arabic}}
\flushleft{\begin{malayalam}
നിനക്കറിഞ്ഞു കൂടേ അല്ലാഹുവിന്നു തന്നെയാണ് ആകാശഭൂമികളുടെ ആധിപത്യമെന്നും, നിങ്ങള്‍ക്ക് അല്ലാഹുവെ കൂടാതെ ഒരു രക്ഷകനും സഹായിയും ഇല്ലെന്നും?
\end{malayalam}}
\flushright{\begin{Arabic}
\quranayah[2][108]
\end{Arabic}}
\flushleft{\begin{malayalam}
മുമ്പ് മൂസായോട് ചോദിക്കപ്പെട്ടത് പോലുള്ള ചോദ്യങ്ങള്‍ നിങ്ങളുടെ റസൂലിനോടും ചോദിക്കുവാനാണോ നിങ്ങള്‍ ഉദ്ദേശിക്കുന്നത്‌? സത്യവിശ്വാസത്തിന് പകരം സത്യനിഷേധത്തെ സ്വീകരിക്കുന്നവരാരോ അവര്‍ നേര്‍മാര്‍ഗത്തില്‍ നിന്നു വ്യതിചലിച്ചു പോയിരിക്കുന്നു.
\end{malayalam}}
\flushright{\begin{Arabic}
\quranayah[2][109]
\end{Arabic}}
\flushleft{\begin{malayalam}
നിങ്ങള്‍ സത്യവിശ്വാസം സ്വീകരിച്ച ശേഷം നിങ്ങളെ അവിശ്വാസികളാക്കി മാറ്റിയെടുക്കുവാനാണ് വേദക്കാരില്‍ മിക്കവരും ആഗ്രഹിക്കുന്നത്‌. സത്യം വ്യക്തമായി ബോധ്യപ്പെട്ടിട്ടും സ്വാര്‍ത്ഥപരമായ അസൂയ നിമിത്തമാണ് (അവരാ നിലപാട് സ്വീകരിക്കുന്നത്‌.) എന്നാല്‍ (അവരുടെ കാര്യത്തില്‍) അല്ലാഹു അവന്റെ കല്‍പന കൊണ്ടുവരുന്നത് വരെ നിങ്ങള്‍ പൊറുക്കുകയും ക്ഷമിക്കുകയും ചെയ്യുക. നിസ്സംശയം അല്ലാഹു ഏത് കാര്യത്തിനും കഴിവുള്ളവനത്രെ.
\end{malayalam}}
\flushright{\begin{Arabic}
\quranayah[2][110]
\end{Arabic}}
\flushleft{\begin{malayalam}
നിങ്ങള്‍ പ്രാര്‍ത്ഥന മുറപ്രകാരം നിര്‍വഹിക്കുകയും സകാത്ത് നല്‍കുകയും ചെയ്യുക. നിങ്ങളുടെ സ്വന്തം ഗുണത്തിനായി നിങ്ങള്‍ നല്ലതായ എന്തൊന്ന് മുന്‍കൂട്ടി ചെയ്താലും അതിന്റെ ഫലം അല്ലാഹുവിങ്കല്‍ നിങ്ങള്‍ക്ക് കണ്ടെത്താവുന്നതാണ്‌. നിങ്ങള്‍ പ്രവര്‍ത്തിക്കുന്നതെല്ലാം അല്ലാഹു കണ്ടറിയുന്നവനാകുന്നു.
\end{malayalam}}
\flushright{\begin{Arabic}
\quranayah[2][111]
\end{Arabic}}
\flushleft{\begin{malayalam}
ആര്‍ക്കെങ്കിലും) സ്വര്‍ഗത്തില്‍ പ്രവേശിക്കണമെങ്കില്‍ യഹൂദരോ ക്രിസ്ത്യാനികളോ ആവാതെ പറ്റില്ലെന്നാണ് അവര്‍ പറയുന്നത്‌. അതൊക്കെ അവരുടെ വ്യാമോഹങ്ങളത്രെ. എന്നാല്‍ (നബിയേ,) പറയുക; നിങ്ങള്‍ സത്യവാന്‍മാരാണെങ്കില്‍ (അതിന്ന്‌) നിങ്ങള്‍ക്ക് കിട്ടിയ തെളിവ് കൊണ്ടു വരൂ എന്ന്‌.
\end{malayalam}}
\flushright{\begin{Arabic}
\quranayah[2][112]
\end{Arabic}}
\flushleft{\begin{malayalam}
എന്നാല്‍ (കാര്യം) അങ്ങനെയല്ല. ഏതൊരാള്‍ സല്‍കര്‍മ്മകാരിയായിക്കൊണ്ട് അല്ലാഹുവിന്ന് ആത്മസമര്‍പ്പണം ചെയ്തുവോ അവന്ന് തന്റെ രക്ഷിതാവിങ്കല്‍ അതിന്റെ പ്രതിഫലം ഉണ്ടായിരിക്കുന്നതാണ്‌. അത്തരക്കാര്‍ക്ക് യാതൊന്നും ഭയപ്പെടേണ്ടതില്ല ; അവര്‍ ദുഃഖിക്കേണ്ടി വരികയുമില്ല.
\end{malayalam}}
\flushright{\begin{Arabic}
\quranayah[2][113]
\end{Arabic}}
\flushleft{\begin{malayalam}
യഹൂദന്‍മാര്‍ പറഞ്ഞു ; ക്രിസ്ത്യാനികള്‍ക്ക് യാതൊരു അടിസ്ഥാനവുമില്ലെന്ന്‌. ക്രിസ്ത്യാനികള്‍ പറഞ്ഞു; യഹൂദന്‍മാര്‍ക്ക് യാതൊരു അടിസ്ഥാനവുമില്ലെന്ന്‌. അവരെല്ലാവരും വേദഗ്രന്ഥം പാരായണം ചെയ്യുന്നവരാണ് താനും. അങ്ങനെ ഇവര്‍ പറഞ്ഞത് പോലെ തന്നെ വിവരമില്ലാത്ത ചിലരൊക്കെ പറഞ്ഞിട്ടുണ്ട്‌. എന്നാല്‍ അവര്‍ തമ്മില്‍ ഭിന്നിക്കുന്ന വിഷയങ്ങളില്‍ ഉയിര്‍ത്തെഴുന്നേല്‍പിന്റെ നാളില്‍ അല്ലാഹു അവര്‍ക്കിടയില്‍ തീര്‍പ്പുകല്‍പിക്കുന്നതാണ്‌.
\end{malayalam}}
\flushright{\begin{Arabic}
\quranayah[2][114]
\end{Arabic}}
\flushleft{\begin{malayalam}
അല്ലാഹുവിന്റെ പള്ളികളില്‍ അവന്റെ നാമം പ്രകീര്‍ത്തിക്കപ്പെടുന്നതിന് തടസ്സമുണ്ടാക്കുകയും, അവയുടെ (പള്ളികളുടെ) തകര്‍ച്ചയ്ക്കായി ശ്രമിക്കുകയും ചെയ്തവനേക്കാള്‍ വലിയ അതിക്രമകാരി ആരുണ്ട്‌? ഭയപ്പാടോടുകൂടിയല്ലാതെ അവര്‍ക്ക് ആ പള്ളികളില്‍ പ്രവേശിക്കാവതല്ലായിരുന്നു. അവര്‍ക്ക് ഇഹലോകത്ത് നിന്ദ്യതയാണുള്ളത്‌. പരലോകത്താകട്ടെ കഠിനശിക്ഷയും.
\end{malayalam}}
\flushright{\begin{Arabic}
\quranayah[2][115]
\end{Arabic}}
\flushleft{\begin{malayalam}
കിഴക്കും പടിഞ്ഞാറും അല്ലാഹുവിന്റേത് തന്നെയാകുന്നു. നിങ്ങള്‍ എവിടേക്ക് തിരിഞ്ഞ് നിന്ന് പ്രാര്‍ത്ഥിച്ചാലും അവിടെ അല്ലാഹുവിന്റെ മുഖമുണ്ടായിരിക്കും. അല്ലാഹു വിപുലമായ കഴിവുകളുള്ളവനും സര്‍വ്വജ്ഞനുമാകുന്നു.
\end{malayalam}}
\flushright{\begin{Arabic}
\quranayah[2][116]
\end{Arabic}}
\flushleft{\begin{malayalam}
അവര്‍ പറയുന്നു: അല്ലാഹു സന്താനത്തെ സ്വീകരിച്ചിരിക്കുന്നു എന്ന്‌. അവനെത്ര പരിശുദ്ധന്‍! അങ്ങനെയല്ല, ആകാശഭൂമികളിലുള്ളതെല്ലാം തന്നെ അവന്‍റെതാകുന്നു. എല്ലാവരും അവന്ന് കീഴ്പെട്ടിരിക്കുന്നവരാകുന്നു.
\end{malayalam}}
\flushright{\begin{Arabic}
\quranayah[2][117]
\end{Arabic}}
\flushleft{\begin{malayalam}
ആകാശങ്ങളെയും ഭൂമിയെയും മുന്‍ മാതൃകയില്ലാതെ നിര്‍മിച്ചവനത്രെ അവന്‍. അവനൊരു കാര്യം തീരുമാനിച്ചാല്‍ ഉണ്ടാകൂ എന്ന് പറയുക മാത്രമേ വേണ്ടതുള്ളൂ. ഉടനെ അതുണ്ടാകുന്നു.
\end{malayalam}}
\flushright{\begin{Arabic}
\quranayah[2][118]
\end{Arabic}}
\flushleft{\begin{malayalam}
വിവരമില്ലാത്തവര്‍ പറഞ്ഞു: എന്തുകൊണ്ട് ഞങ്ങളോട് (നേരിട്ട്‌) അല്ലാഹു സംസാരിക്കുന്നില്ല? അല്ലെങ്കില്‍ ഞങ്ങള്‍ക്ക് (ബോധ്യമാകുന്ന) ഒരു ദൃഷ്ടാന്തം വന്നുകിട്ടുന്നില്ല? എന്നാല്‍ ഇവര്‍ പറഞ്ഞതു പോലെത്തന്നെ ഇവര്‍ക്ക് മുമ്പുള്ളവരും പറഞ്ഞിട്ടുണ്ട്‌. ഇവര്‍ രണ്ട് കൂട്ടരുടെയും മനസ്സുകള്‍ക്ക് തമ്മില്‍ സാമ്യമുണ്ട്‌. ദൃഢമായി വിശ്വസിക്കുന്ന ജനങ്ങള്‍ക്ക് നാം ദൃഷ്ടാന്തങ്ങള്‍ വ്യക്തമാക്കികൊടുത്തിട്ടുണ്ട്‌.
\end{malayalam}}
\flushright{\begin{Arabic}
\quranayah[2][119]
\end{Arabic}}
\flushleft{\begin{malayalam}
തീര്‍ച്ചയായും നിന്നെ നാം സന്തോഷവാര്‍ത്ത അറിയിക്കുന്നവനും, താക്കീത് നല്‍കുന്നവനുമായിക്കൊണ്ട് സത്യവുമായി അയച്ചിരിക്കുകയാണ്‌. നരകാവകാശികളെപ്പറ്റി നീ ചോദ്യം ചെയ്യപ്പെടുന്നതല്ല.
\end{malayalam}}
\flushright{\begin{Arabic}
\quranayah[2][120]
\end{Arabic}}
\flushleft{\begin{malayalam}
യഹൂദര്‍ക്കോ ക്രൈസ്തവര്‍ക്കോ ഒരിക്കലും നിന്നെപ്പറ്റി തൃപ്തിവരികയില്ല; നീ അവരുടെ മാര്‍ഗം പിന്‍പറ്റുന്നത് വരെ. പറയുക: അല്ലാഹുവിന്റെ മാര്‍ഗദര്‍ശനമാണ് യഥാര്‍ത്ഥ മാര്‍ഗദര്‍ശനം. നിനക്ക് അറിവ് വന്നുകിട്ടിയതിനു ശേഷം അവരുടെ തന്നിഷ്ടങ്ങളെയെങ്ങാനും നീ പിന്‍പറ്റിപ്പോയാല്‍ അല്ലാഹുവില്‍ നിന്ന് നിന്നെ രക്ഷിക്കുവാനോ സഹായിക്കുവാനോ ആരുമുണ്ടാവില്ല.
\end{malayalam}}
\flushright{\begin{Arabic}
\quranayah[2][121]
\end{Arabic}}
\flushleft{\begin{malayalam}
നാം ഈ ഗ്രന്ഥം നല്‍കിയത് ആര്‍ക്കാണോ അവരത് പാരായണത്തിന്റെ മുറപ്രകാരം പാരായണം ചെയ്യുന്നു. അവരതില്‍ വിശ്വസിക്കുന്നു. എന്നാല്‍ ആരതില്‍ അവിശ്വസിക്കുന്നുവോ അവര്‍ തന്നെയാണ് നഷ്ടം പറ്റിയവര്‍.
\end{malayalam}}
\flushright{\begin{Arabic}
\quranayah[2][122]
\end{Arabic}}
\flushleft{\begin{malayalam}
ഇസ്രായീല്‍ സന്തതികളേ, ഞാന്‍ നിങ്ങള്‍ക്ക് ചെയ്തു തന്നിട്ടുള്ള അനുഗ്രഹവും ജനവിഭാഗങ്ങളില്‍ നിങ്ങളെ ഞാന്‍ ഉല്‍കൃഷ്ടരാക്കിയതും നിങ്ങള്‍ ഓര്‍ക്കുക.
\end{malayalam}}
\flushright{\begin{Arabic}
\quranayah[2][123]
\end{Arabic}}
\flushleft{\begin{malayalam}
ഒരാള്‍ക്കും മറ്റൊരാള്‍ക്കുവേണ്ടി ഒരു ഉപകാരവും ചെയ്യുവാന്‍ പറ്റാത്ത, ഒരാളില്‍ നിന്നും ഒരു പ്രായശ്ചിത്തവും സ്വീകരിക്കപ്പെടാത്ത, ഒരാള്‍ക്കും ഒരു ശുപാര്‍ശയും പ്രയോജനപ്പെടാത്ത, ആര്‍ക്കും ഒരു സഹായവും ലഭിക്കാത്ത ഒരു ദിവസത്തെ (ന്യായവിധിയുടെ ദിവസത്തെ) നിങ്ങള്‍ സൂക്ഷിക്കുകയും ചെയ്യുക.
\end{malayalam}}
\flushright{\begin{Arabic}
\quranayah[2][124]
\end{Arabic}}
\flushleft{\begin{malayalam}
ഇബ്രാഹീമിനെ അദ്ദേഹത്തിന്റെ രക്ഷിതാവ് ചില കല്‍പനകള്‍കൊണ്ട് പരീക്ഷിക്കുകയും, അദ്ദേഹമത് നിറവേറ്റുകയും ചെയ്ത കാര്യവും (നിങ്ങള്‍ അനുസ്മരിക്കുക.) അല്ലാഹു (അപ്പോള്‍) അദ്ദേഹത്തോട് പറഞ്ഞു: ഞാന്‍ നിന്നെ മനുഷ്യര്‍ക്ക് നേതാവാക്കുകയാണ്‌. ഇബ്രാഹീം പറഞ്ഞു: എന്റെ സന്തതികളില്‍പ്പെട്ടവരെയും (നേതാക്കളാക്കണമേ.) അല്ലാഹു പറഞ്ഞു: (ശരി; പക്ഷെ) എന്റെ ഈ നിശ്ചയം അതിക്രമകാരികള്‍ക്ക് ബാധകമായിരിക്കുകയില്ല
\end{malayalam}}
\flushright{\begin{Arabic}
\quranayah[2][125]
\end{Arabic}}
\flushleft{\begin{malayalam}
ആ ഭവനത്തെ (കഅ്ബയെ) ജനങ്ങള്‍ സമ്മേളിക്കുന്ന സ്ഥലവും ഒരു സുരക്ഷിത കേന്ദ്രവുമായി നാം നിശ്ചയിച്ചതും (ഓര്‍ക്കുക.) ഇബ്രാഹീം നിന്ന് പ്രാര്‍ത്ഥിച്ച സ്ഥാനത്തെ നിങ്ങളും നമസ്കാര (പ്രാര്‍ത്ഥന) വേദിയായി സ്വീകരിക്കുക. ഇബ്രാഹീമിന്നും ഇസ്മാഈലിന്നും, നാം കല്‍പന നല്‍കിയത്‌, ത്വവാഫ് (പ്രദക്ഷിണം) ചെയ്യുന്നവര്‍ക്കും, ഇഅ്തികാഫ് (ഭജന) ഇരിക്കുന്നവര്‍ക്കും തലകുനിച്ചും സാഷ്ടാംഗം ചെയ്തും നമസ്കരിക്കുന്ന (പ്രാര്‍ത്ഥിക്കുന്ന) വര്‍ക്കും വേണ്ടി എന്റെ ഭവനത്തെ നിങ്ങള്‍ ഇരുവരും ശുദ്ധമാക്കിവെക്കുക എന്നായിരുന്നു.
\end{malayalam}}
\flushright{\begin{Arabic}
\quranayah[2][126]
\end{Arabic}}
\flushleft{\begin{malayalam}
എന്റെ രക്ഷിതാവേ, നീ ഇതൊരു നിര്‍ഭയമായ നാടാക്കുകയും ഇവിടത്തെ താമസക്കാരില്‍ നിന്ന് അല്ലാഹുവിലും അന്ത്യദിനത്തിലും വിശ്വസിക്കുന്നവര്‍ക്ക് കായ്കനികള്‍ ആഹാരമായി നല്‍കുകയും ചെയ്യേണമേ എന്ന് ഇബ്രാഹീം പ്രാര്‍ത്ഥിച്ച സന്ദര്‍ഭവും (ഓര്‍ക്കുക) അല്ലാഹു പറഞ്ഞു: അവിശ്വസിച്ചവന്നും (ഞാന്‍ ആഹാരം നല്‍കുന്നതാണ്‌.) പക്ഷെ, അല്‍പകാലത്തെ ജീവിതസുഖം മാത്രമാണ് അവന്ന് ഞാന്‍ നല്‍കുക. പിന്നീട് നരകശിക്ഷ ഏല്‍ക്കാന്‍ ഞാന്‍ അവനെ നിര്‍ബന്ധിതനാക്കുന്നതാണ്‌. (അവന്ന്‌) ചെന്നു ചേരാനുള്ള ആ സ്ഥലം വളരെ ചീത്ത തന്നെ.
\end{malayalam}}
\flushright{\begin{Arabic}
\quranayah[2][127]
\end{Arabic}}
\flushleft{\begin{malayalam}
ഇബ്രാഹീമും ഇസ്മാഈലും കൂടി ആ ഭവനത്തിന്റെ (കഅ്ബയുടെ) അടിത്തറ കെട്ടി ഉയര്‍ത്തിക്കൊണ്ടിരുന്ന സന്ദര്‍ഭവും (അനുസ്മരിക്കുക.) (അവര്‍ ഇപ്രകാരം പ്രാര്‍ത്ഥിച്ചിരുന്നു:) ഞങ്ങളുടെ രക്ഷിതാവേ, ഞങ്ങളില്‍ നിന്ന് നീയിത് സ്വീകരിക്കേണമേ. തീര്‍ച്ചയായും നീ എല്ലാം കേള്‍ക്കുന്നവനും അറിയുന്നവനുമാകുന്നു.
\end{malayalam}}
\flushright{\begin{Arabic}
\quranayah[2][128]
\end{Arabic}}
\flushleft{\begin{malayalam}
ഞങ്ങളുടെ രക്ഷിതാവേ, ഞങ്ങള്‍ ഇരുവരെയും നിനക്ക് കീഴ്പെടുന്നവരാക്കുകയും, ഞങ്ങളുടെ സന്തതികളില്‍ നിന്ന് നിനക്ക് കീഴ്പെടുന്ന ഒരു സമുദായത്തെ ഉണ്ടാക്കുകയും, ഞങ്ങളുടെ ആരാധനാ ക്രമങ്ങള്‍ ഞങ്ങള്‍ക്ക് കാണിച്ചുതരികയും, ഞങ്ങളുടെ പശ്ചാത്താപം സ്വീകരിക്കുകയും ചെയ്യേണമേ. തീര്‍ച്ചയായും നീ അത്യധികം പശ്ചാത്താപം സ്വീകരിക്കുന്നവനും കരുണാനിധിയുമാകുന്നു
\end{malayalam}}
\flushright{\begin{Arabic}
\quranayah[2][129]
\end{Arabic}}
\flushleft{\begin{malayalam}
ഞങ്ങളുടെ രക്ഷിതാവേ, അവര്‍ക്ക് (ഞങ്ങളുടെ സന്താനങ്ങള്‍ക്ക്‌) നിന്റെ ദൃഷ്ടാന്തങ്ങള്‍ ഓതികേള്‍പിച്ചു കൊടുക്കുകയും, വേദവും വിജ്ഞാനവും അഭ്യസിപ്പിക്കുകയും, അവരെ സംസ്കരിക്കുകയും ചെയ്യുന്ന ഒരു ദൂതനെ അവരില്‍ നിന്നു തന്നെ നീ നിയോഗിക്കുകയും ചെയ്യേണമേ. തീര്‍ച്ചയായും നീ പ്രതാപവാനും അഗാധജ്ഞാനിയുമാകുന്നു.
\end{malayalam}}
\flushright{\begin{Arabic}
\quranayah[2][130]
\end{Arabic}}
\flushleft{\begin{malayalam}
സ്വന്തം ആത്മാവിനെ മൂഢമാക്കിയവനല്ലാതെ മറ്റാരാണ് ഇബ്രാഹീമിന്റെ മാര്‍ഗത്തോട് വിമുഖത കാണിക്കുക? ഇഹലോകത്തില്‍ അദ്ദേഹത്തെ നാം വിശിഷ്ടനായി തെരഞ്ഞെടുത്തിരിക്കുന്നു. പരലോകത്ത് അദ്ദേഹം സജ്ജനങ്ങളുടെ കൂട്ടത്തില്‍ തന്നെയായിരിക്കും.
\end{malayalam}}
\flushright{\begin{Arabic}
\quranayah[2][131]
\end{Arabic}}
\flushleft{\begin{malayalam}
നീ കീഴ്‌പെടുക എന്ന് അദ്ദേഹത്തിന്റെ രക്ഷിതാവ് അദ്ദേഹത്തോട് പറഞ്ഞപ്പോള്‍ സര്‍വ്വലോകരക്ഷിതാവിന്ന് ഞാനിതാ കീഴ്‌പെട്ടിരിക്കുന്നു എന്ന് അദ്ദേഹം പറഞ്ഞു.
\end{malayalam}}
\flushright{\begin{Arabic}
\quranayah[2][132]
\end{Arabic}}
\flushleft{\begin{malayalam}
ഇബ്രാഹീമും യഅ്ഖൂബും അവരുടെ സന്തതികളോട് ഇത് (കീഴ്‌വണക്കം) ഉപദേശിക്കുക കൂടി ചെയ്തു. എന്റെ മക്കളേ, അല്ലാഹു നിങ്ങള്‍ക്ക് ഈ മതത്തെ വിശിഷ്ടമായി തെരഞ്ഞെടുത്തിരിക്കുന്നു. അതിനാല്‍ അല്ലാഹുവിന്ന് കീഴ്പെടുന്നവരായി (മുസ്ലിംകളായി) ക്കൊണ്ടല്ലാതെ നിങ്ങള്‍ മരിക്കാനിടയാകരുത്‌. (ഇങ്ങനെയാണ് അവര്‍ ഓരോരുത്തരും ഉപദേശിച്ചത്‌)
\end{malayalam}}
\flushright{\begin{Arabic}
\quranayah[2][133]
\end{Arabic}}
\flushleft{\begin{malayalam}
എനിക്ക് ശേഷം ഏതൊരു ദൈവത്തെയാണ് നിങ്ങള്‍ ആരാധിക്കുക ? എന്ന് യഅ്ഖൂബ് മരണം ആസന്നമായ സന്ദര്‍ഭത്തില്‍ തന്റെ സന്തതികളോട് ചോദിച്ചപ്പോള്‍ നിങ്ങളവിടെ സന്നിഹിതരായിരുന്നോ ? അവര്‍ പറഞ്ഞു: താങ്കളുടെ ആരാധ്യനായ, താങ്കളുടെ പിതാക്കളായ ഇബ്രാഹീമിന്റേയും ഇസ്മാഈലിന്റേയും ഇഷാഖിന്റേയും ആരാധ്യനായ ഏകദൈവത്തെ മാത്രം ഞങ്ങള്‍ ആരാധിക്കും. ഞങ്ങള്‍ അവന്ന് കീഴ്‌പെട്ട് ജീവിക്കുന്നവരുമായിരിക്കും
\end{malayalam}}
\flushright{\begin{Arabic}
\quranayah[2][134]
\end{Arabic}}
\flushleft{\begin{malayalam}
അത് കഴിഞ്ഞുപോയ ഒരു സമുദായമാകുന്നു. അവര്‍ പ്രവര്‍ത്തിച്ചതിന്റെ ഫലം അവര്‍ക്കാകുന്നു. നിങ്ങള്‍ പ്രവര്‍ത്തിച്ചതിന്റെ ഫലം നിങ്ങള്‍ക്കും. അവര്‍ പ്രവര്‍ത്തിച്ചിരുന്നതിനെപ്പറ്റി നിങ്ങള്‍ ചോദ്യം ചെയ്യപ്പെടുന്നതല്ല.
\end{malayalam}}
\flushright{\begin{Arabic}
\quranayah[2][135]
\end{Arabic}}
\flushleft{\begin{malayalam}
നിങ്ങള്‍ യഹൂദരോ ക്രൈസ്തവരോ ആയാലേ നേര്‍വഴിയിലാകൂ എന്നാണവര്‍ പറയുന്നത്‌. എന്നാല്‍ നീ പറയുക: അതല്ല വക്രതയില്ലാത്ത ശുദ്ധമനസ്കനായിരുന്ന ഇബ്രാഹീമിന്റെ മാര്‍ഗമാണ് (പിന്‍പറ്റേണ്ടത്‌.) അദ്ദേഹം ബഹുദൈവാരാധകരില്‍ പെട്ടവനായിരുന്നില്ല.
\end{malayalam}}
\flushright{\begin{Arabic}
\quranayah[2][136]
\end{Arabic}}
\flushleft{\begin{malayalam}
നിങ്ങള്‍ പറയുക: അല്ലാഹുവിലും, അവങ്കല്‍ നിന്ന് ഞങ്ങള്‍ക്ക് അവതരിപ്പിച്ചു കിട്ടിയതിലും, ഇബ്രാഹീമിനും ഇസ്മാഈലിനും ഇഷാഖിനും യഅ്ഖൂബിനും യഅ്ഖൂബ് സന്തതികള്‍ക്കും അവതരിപ്പിച്ച് കൊടുത്തതിലും, മൂസാ, ഈസാ എന്നിവര്‍ക്ക് നല്‍കപ്പെട്ടതിലും, സര്‍വ്വ പ്രവാചകന്‍മാര്‍ക്കും അവരുടെ രക്ഷിതാവിങ്കല്‍ നിന്ന് നല്‍കപ്പെട്ടതി (സന്ദേശങ്ങളി)ലും ഞങ്ങള്‍ വിശ്വസിച്ചിരിക്കുന്നു. അവരില്‍ ആര്‍ക്കിടയിലും ഞങ്ങള്‍ വിവേചനം കല്‍പിക്കുന്നില്ല. ഞങ്ങള്‍ അവന്ന് (അല്ലാഹുവിന്ന്‌) കീഴ്‌പെട്ട് ജീവിക്കുന്നവരുമാകുന്നു.
\end{malayalam}}
\flushright{\begin{Arabic}
\quranayah[2][137]
\end{Arabic}}
\flushleft{\begin{malayalam}
നിങ്ങള്‍ ഈ വിശ്വസിച്ചത് പോലെ അവരും വിശ്വസിച്ചിരുന്നാല്‍ അവര്‍ നേര്‍മാര്‍ഗത്തിലായിക്കഴിഞ്ഞു. അവര്‍ പിന്തിരിഞ്ഞ് കളയുകയാണെങ്കിലോ അവരുടെ നിലപാട് കക്ഷിമാത്സര്യം മാത്രമാകുന്നു. അവരില്‍ നിന്ന് നിന്നെ സംരക്ഷിക്കാന്‍ അല്ലാഹു മതി, അവന്‍ എല്ലാം കേള്‍ക്കുന്നവനും എല്ലാം അറിയുന്നവനുമത്രെ.
\end{malayalam}}
\flushright{\begin{Arabic}
\quranayah[2][138]
\end{Arabic}}
\flushleft{\begin{malayalam}
അല്ലാഹു നല്‍കിയ വര്‍ണമാകുന്നു (നമ്മുടെത്‌.) അല്ലാഹുവെക്കാള്‍ നന്നായി വര്‍ണം നല്‍കുന്നവന്‍ ആരുണ്ട് ? അവനെയാകുന്നു ഞങ്ങള്‍ ആരാധിക്കുന്നത്‌.
\end{malayalam}}
\flushright{\begin{Arabic}
\quranayah[2][139]
\end{Arabic}}
\flushleft{\begin{malayalam}
(നബിയേ,) പറയുക: അല്ലാഹുവിന്റെ കാര്യത്തില്‍ നിങ്ങള്‍ ഞങ്ങളോട് തര്‍ക്കിക്കുകയാണോ ? അവന്‍ ഞങ്ങളുടെയും നിങ്ങളുടെയും രക്ഷിതാവാണല്ലോ ? ഞങ്ങള്‍ക്കുള്ളത് ഞങ്ങളുടെ കര്‍മ്മ (ഫല) ങ്ങളാണ്‌. നിങ്ങള്‍ക്കുള്ളത് നിങ്ങളുടെ കര്‍മ്മ (ഫല) ങ്ങളും. ഞങ്ങള്‍ അവനോട് ആത്മാര്‍ത്ഥത പുലര്‍ത്തുന്നവരുമാകുന്നു.
\end{malayalam}}
\flushright{\begin{Arabic}
\quranayah[2][140]
\end{Arabic}}
\flushleft{\begin{malayalam}
അതല്ല, ഇബ്രാഹീമും ഇസ്മാഈലും, ഇഷാഖും, യഅ്ഖൂബും, യഅ്ഖൂബ് സന്തതികളുമെല്ലാം തന്നെ യഹൂദരോ ക്രൈസ്തവരോ ആയിരുന്നു എന്നാണോ നിങ്ങള്‍ പറയുന്നത്‌? (നബിയേ,) ചോദിക്കുക: നിങ്ങള്‍ക്കാണോ കൂടുതല്‍ അറിവുള്ളത് ? അതോ അല്ലാഹുവിനോ? അല്ലാഹുവിങ്കല്‍ നിന്ന് ലഭിച്ചതും, തന്റെ പക്കലുള്ളതുമായ സാക്ഷ്യം മറച്ചു വെച്ചവനേക്കാള്‍ വലിയ അതിക്രമകാരി ആരുണ്ട് ? നിങ്ങള്‍ പ്രവര്‍ത്തിക്കുന്നതിനെ പറ്റിയൊന്നും അല്ലാഹു അശ്രദ്ധനല്ല.
\end{malayalam}}
\flushright{\begin{Arabic}
\quranayah[2][141]
\end{Arabic}}
\flushleft{\begin{malayalam}
അത് കഴിഞ്ഞുപോയ ഒരു സമുദായമാകുന്നു. അവര്‍ പ്രവര്‍ത്തിച്ചതിന്റെ ഫലം അവര്‍ക്കാകുന്നു. നിങ്ങള്‍ പ്രവര്‍ത്തിച്ചതിന്റെ ഫലം നിങ്ങള്‍ക്കും. അവര്‍ പ്രവര്‍ത്തിച്ചിരുന്നതിനെപ്പറ്റി നിങ്ങള്‍ ചോദ്യം ചെയ്യപ്പെടുന്നതുമല്ല.
\end{malayalam}}
\flushright{\begin{Arabic}
\quranayah[2][142]
\end{Arabic}}
\flushleft{\begin{malayalam}
ഇവര്‍ ഇതുവരെ (പ്രാര്‍ത്ഥനാവേളയില്‍) തിരിഞ്ഞുനിന്നിരുന്ന ഭാഗത്ത് നിന്ന് ഇവരെ തിരിച്ചുവിട്ട കാരണമെന്താണെന്ന് മൂഢന്‍മാരായ ആളുകള്‍ ചോദിച്ചേക്കും. (നബിയേ,) പറയുക : അല്ലാഹുവിന്റേത് തന്നെയാണ് കിഴക്കും പടിഞ്ഞാറുമെല്ലാം. അവന്‍ ഉദ്ദേശിക്കുന്നവരെ അവന്‍ നേരായ മാര്‍ഗത്തിലേക്ക് നയിക്കുന്നു.
\end{malayalam}}
\flushright{\begin{Arabic}
\quranayah[2][143]
\end{Arabic}}
\flushleft{\begin{malayalam}
അപ്രകാരം നാം നിങ്ങളെ ഒരു ഉത്തമ സമുദായമാക്കിയിരിക്കുന്നു. നിങ്ങള്‍ ജനങ്ങള്‍ക്ക് സാക്ഷികളായിരിക്കുവാനും റസൂല്‍ നിങ്ങള്‍ക്ക് സാക്ഷിയായിരിക്കുവാനും വേണ്ടി. റസൂലിനെ പിന്‍പറ്റുന്നതാരൊക്കെയെന്നും, പിന്‍മാറിക്കളയുന്നതാരൊക്കെയെന്നും തിരിച്ചറിയുവാന്‍ വേണ്ടി മാത്രമായിരുന്നു നീ ഇതുവരെ തിരിഞ്ഞു നിന്നിരുന്ന ഭാഗത്തെ നാം ഖിബ് ലയായി നിശ്ചയിച്ചത്‌. അല്ലാഹു നേര്‍വഴിയിലാക്കിയവരൊഴിച്ച് മറ്റെല്ലാവര്‍ക്കും അത് (ഖിബ് ല മാറ്റം) ഒരു വലിയ പ്രശ്നമായിത്തീര്‍ന്നിരിക്കുന്നു. അല്ലാഹു നിങ്ങളുടെ വിശ്വാസത്തെ പാഴാക്കിക്കളയുന്നതല്ല. തീര്‍ച്ചയായും അല്ലാഹു മനുഷ്യരോട് അത്യധികം ദയയുള്ളവനും കരുണാനിധിയുമാകുന്നു.
\end{malayalam}}
\flushright{\begin{Arabic}
\quranayah[2][144]
\end{Arabic}}
\flushleft{\begin{malayalam}
(നബിയേ,) നിന്റെ മുഖം ആകാശത്തേക്ക് തിരിഞ്ഞുകൊണ്ടിരിക്കുന്നത് നാം കാണുന്നുണ്ട്‌. അതിനാല്‍ നിനക്ക് ഇഷ്ടമാകുന്ന ഒരു ഖിബ് ലയിലേക്ക് നിന്നെ നാം തിരിക്കുകയാണ്‌. ഇനി മേല്‍ നീ നിന്റെ മുഖം മസ്ജിദുല്‍ ഹറാമിന്റെ നേര്‍ക്ക് തിരിക്കുക. നിങ്ങള്‍ എവിടെയായിരുന്നാലും അതിന്റെ നേര്‍ക്കാണ് നിങ്ങള്‍ മുഖം തിരിക്കേണ്ടത്‌. വേദം നല്‍കപ്പെട്ടവര്‍ക്ക് ഇത് തങ്ങളുടെ രക്ഷിതാവിങ്കല്‍ നിന്നുള്ള സത്യമാണെന്ന് നന്നായി അറിയാം. അവര്‍ പ്രവര്‍ത്തിക്കുന്നതിനെപ്പറ്റിയൊന്നും അല്ലാഹു അശ്രദ്ധനല്ല.
\end{malayalam}}
\flushright{\begin{Arabic}
\quranayah[2][145]
\end{Arabic}}
\flushleft{\begin{malayalam}
വേദം നല്‍കപ്പെട്ടവരുടെ അടുക്കല്‍ നീ എല്ലാവിധ ദൃഷ്ടാന്തവും കൊണ്ട് ചെന്നാലും അവര്‍ നിന്റെ ഖിബ് ലയെ പിന്തുടരുന്നതല്ല. അവരുടെ ഖിബ് ലയെ നീയും പിന്തുടരുന്നതല്ല. അവരില്‍ ഒരു വിഭാഗം മറ്റൊരു വിഭാഗത്തിന്‍റെഖിബ് ലയെ പിന്തുടരുകയുമില്ല. നിനക്ക് ശരിയായ അറിവ് വന്നുകിട്ടിയ ശേഷം നീയെങ്ങാനും അവരുടെ ഇച്ഛകളെ പിന്‍പറ്റിയാല്‍ നീയും അതിക്രമകാരികളുടെ കൂട്ടത്തില്‍ തന്നെയായിരിക്കും.
\end{malayalam}}
\flushright{\begin{Arabic}
\quranayah[2][146]
\end{Arabic}}
\flushleft{\begin{malayalam}
നാം വേദം നല്‍കിയിട്ടുള്ളവര്‍ക്ക് സ്വന്തം മക്കളെ അറിയാവുന്നത് പോലെ അദ്ദേഹത്തെ (റസൂലിനെ) അറിയാവുന്നതാണ്‌. തീര്‍ച്ചയായും അവരില്‍ ഒരു വിഭാഗം അറിഞ്ഞുകൊണ്ട് തന്നെ സത്യം മറച്ചുവെക്കുകയാകുന്നു.
\end{malayalam}}
\flushright{\begin{Arabic}
\quranayah[2][147]
\end{Arabic}}
\flushleft{\begin{malayalam}
(നബിയേ, ഈ സന്ദേശം) നിന്റെ നാഥന്റെപക്കല്‍ നിന്നുള്ള സത്യമാകുന്നു. അതിനാല്‍ നീ സംശയാലുക്കളുടെ കൂട്ടത്തില്‍ പെട്ടുപോകരുത്‌.
\end{malayalam}}
\flushright{\begin{Arabic}
\quranayah[2][148]
\end{Arabic}}
\flushleft{\begin{malayalam}
ഓരോ വിഭാഗക്കാര്‍ക്കും അവര്‍ (പ്രാര്‍ത്ഥനാവേളയില്‍) തിരിഞ്ഞുനില്‍ക്കുന്ന ഓരോ ഭാഗമുണ്ട്‌. എന്നാല്‍ നിങ്ങള്‍ ചെയ്യേണ്ടത് സല്‍പ്രവര്‍ത്തനങ്ങള്‍ക്കായി മുന്നോട്ട് വരികയാണ്‌. നിങ്ങള്‍ എവിടെയൊക്കെയായിരുന്നാലും അല്ലാഹു നിങ്ങളെയെല്ലാം ഒന്നിച്ചു കൊണ്ടുവരുന്നതാണ്‌. തീര്‍ച്ചയായും അല്ലാഹു ഏത് കാര്യത്തിനും കഴിവുള്ളവനാകുന്നു.
\end{malayalam}}
\flushright{\begin{Arabic}
\quranayah[2][149]
\end{Arabic}}
\flushleft{\begin{malayalam}
ഏതൊരിടത്ത് നിന്ന് നീ പുറപ്പെടുകയാണെങ്കിലും മസ്ജിദുല്‍ ഹറാമിന്റെ നേര്‍ക്ക് (പ്രാര്‍ത്ഥനാവേളയില്‍) നിന്റെ മുഖം തിരിക്കേണ്ടതാണ്‌. തീര്‍ച്ചയായും അത് നിന്റെ രക്ഷിതാവിങ്കല്‍നിന്നുള്ള യഥാര്‍ത്ഥ (നിര്‍ദേശ) മാകുന്നു. നിങ്ങള്‍ പ്രവര്‍ത്തിക്കുന്നതിനെപ്പറ്റിയൊന്നും അല്ലാഹു അശ്രദ്ധനല്ല.
\end{malayalam}}
\flushright{\begin{Arabic}
\quranayah[2][150]
\end{Arabic}}
\flushleft{\begin{malayalam}
ഏതൊരിടത്ത് നിന്ന് നീ പുറപ്പെടുകയാണെങ്കിലും മസ്ജിദുല്‍ ഹറാമിന്റെ നേര്‍ക്ക് നിന്റെ മുഖം തിരിക്കേണ്ടതാണ്‌. (സത്യവിശ്വാസികളേ,) നിങ്ങള്‍ എവിടെയൊക്കെയായിരുന്നാലും അതിന്റെ നേര്‍ക്കാണ് നിങ്ങളുടെ മുഖം തിരിക്കേണ്ടത്‌. നിങ്ങള്‍ക്കെതിരായി ജനങ്ങള്‍ക്ക് ഇനി യാതൊരു ന്യായവും ഇല്ലാതിരിക്കുവാന്‍ വേണ്ടിയാണിത്‌. അവരില്‍ പെട്ട ചില അതിക്രമകാരികള്‍ (തര്‍ക്കിച്ചേക്കാമെന്നത്‌) അല്ലാതെ. എന്നാല്‍ നിങ്ങള്‍ അവരെ ഭയപ്പെടാതെ എന്നെ ഭയപ്പെടുക. എന്റെ അനുഗ്രഹം ഞാന്‍ നിങ്ങള്‍ക്ക് പൂര്‍ത്തിയാക്കിത്തരുവാനും, നിങ്ങള്‍ സന്‍മാര്‍ഗം പ്രാപിക്കുവാനും വേണ്ടിയാണിതെല്ലാം.
\end{malayalam}}
\flushright{\begin{Arabic}
\quranayah[2][151]
\end{Arabic}}
\flushleft{\begin{malayalam}
നമ്മുടെ ദൃഷ്ടാന്തങ്ങള്‍ നിങ്ങള്‍ക്ക് ഓതികേള്‍പിച്ച് തരികയും, നിങ്ങളെ സംസ്കരിക്കുകയും, നിങ്ങള്‍ക്ക് വേദവും വിജ്ഞാനവും പഠിപ്പിച്ചുതരികയും, നിങ്ങള്‍ക്ക് അറിവില്ലാത്തത് നിങ്ങള്‍ക്ക് അറിയിച്ചുതരികയും ചെയ്യുന്ന, നിങ്ങളുടെ കൂട്ടത്തില്‍ നിന്നു തന്നെയുള്ള ഒരു ദൂതനെ നിങ്ങളിലേക്ക് നാം നിയോഗിച്ചത് (വഴി നിങ്ങള്‍ക്ക് ചെയ്ത അനുഗ്രഹം) പോലെത്തന്നെയാകുന്നു ഇതും.
\end{malayalam}}
\flushright{\begin{Arabic}
\quranayah[2][152]
\end{Arabic}}
\flushleft{\begin{malayalam}
ആകയാല്‍ എന്നെ നിങ്ങള്‍ ഓര്‍ക്കുക. നിങ്ങളെ ഞാനും ഓര്‍ക്കുന്നതാണ്‌. എന്നോട് നിങ്ങള്‍ നന്ദികാണിക്കുക. നിങ്ങളെന്നോട് നന്ദികേട് കാണിക്കരുത്‌.
\end{malayalam}}
\flushright{\begin{Arabic}
\quranayah[2][153]
\end{Arabic}}
\flushleft{\begin{malayalam}
സത്യവിശ്വാസികളെ, നിങ്ങള്‍ സഹനവും നമസ്കാരവും മുഖേന (അല്ലാഹുവിനോട്‌) സഹായം തേടുക. തീര്‍ച്ചയായും ക്ഷമിക്കുന്നവരോടൊപ്പമാകുന്നു അല്ലാഹു.
\end{malayalam}}
\flushright{\begin{Arabic}
\quranayah[2][154]
\end{Arabic}}
\flushleft{\begin{malayalam}
അല്ലാഹുവിന്റെ മാര്‍ഗത്തില്‍ കൊല്ലപ്പെട്ടവരെപ്പറ്റി മരണപ്പെട്ടവര്‍ എന്ന് നിങ്ങള്‍ പറയേണ്ട. എന്നാല്‍ അവരാകുന്നു ജീവിക്കുന്നവര്‍. പക്ഷെ, നിങ്ങള്‍ (അതിനെപ്പറ്റി) ബോധവാന്‍മാരാകുന്നില്ല.
\end{malayalam}}
\flushright{\begin{Arabic}
\quranayah[2][155]
\end{Arabic}}
\flushleft{\begin{malayalam}
കുറച്ചൊക്കെ ഭയം, പട്ടിണി, ധനനഷ്ടം, ജീവ നഷ്ടം, വിഭവ നഷ്ടം എന്നിവ മുഖേന നിങ്ങളെ നാം പരീക്ഷിക്കുക തന്നെ ചെയ്യും. (അത്തരം സന്ദര്‍ഭങ്ങളില്‍) ക്ഷമിക്കുന്നവര്‍ക്ക് സന്തോഷവാര്‍ത്ത അറിയിക്കുക.
\end{malayalam}}
\flushright{\begin{Arabic}
\quranayah[2][156]
\end{Arabic}}
\flushleft{\begin{malayalam}
തങ്ങള്‍ക്ക് വല്ല ആപത്തും ബാധിച്ചാല്‍ അവര്‍ (ആ ക്ഷമാശീലര്‍) പറയുന്നത്‌; ഞങ്ങള്‍ അല്ലാഹുവിന്റെ അധീനത്തിലാണ്‌. അവങ്കലേക്ക് തന്നെ മടങ്ങേണ്ടവരുമാണ് എന്നായിരിക്കും.
\end{malayalam}}
\flushright{\begin{Arabic}
\quranayah[2][157]
\end{Arabic}}
\flushleft{\begin{malayalam}
അവര്‍ക്കത്രെ തങ്ങളുടെ രക്ഷിതാവിങ്കല്‍ നിന്ന് അനുഗ്രഹങ്ങളും കാരുണ്യവും ലഭിക്കുന്നത്‌. അവരത്രെ സന്‍മാര്‍ഗം പ്രാപിച്ചവര്‍.
\end{malayalam}}
\flushright{\begin{Arabic}
\quranayah[2][158]
\end{Arabic}}
\flushleft{\begin{malayalam}
തീര്‍ച്ചയായും സഫായും മര്‍വയും മതചിഹ്നങ്ങളായി അല്ലാഹു നിശ്ചയിച്ചതില്‍ പെട്ടതാകുന്നു. കഅ്ബാ മന്ദിരത്തില്‍ ചെന്ന് ഹജ്ജോ ഉംറഃയോ നിര്‍വഹിക്കുന്ന ഏതൊരാളും അവയിലൂടെ പ്രദക്ഷിണം നടത്തുന്നതില്‍ കുറ്റമൊന്നുമില്ല. ആരെങ്കിലും സല്‍കര്‍മ്മം സ്വയം സന്നദ്ധനായി ചെയ്യുകയാണെങ്കില്‍ തീര്‍ച്ചയായും അല്ലാഹു കൃതജ്ഞനും സര്‍വ്വജ്ഞനുമാകുന്നു.
\end{malayalam}}
\flushright{\begin{Arabic}
\quranayah[2][159]
\end{Arabic}}
\flushleft{\begin{malayalam}
നാമവതരിപ്പിച്ച തെളിവുകളും മാര്‍ഗദര്‍ശനവും വേദഗ്രന്ഥത്തിലൂടെ ജനങ്ങള്‍ക്ക് നാം വിശദമാക്കികൊടുത്തതിന് ശേഷം മറച്ചുവെക്കുന്നവരാരോ അവരെ അല്ലാഹു ശപിക്കുന്നതാണ്‌. ശപിക്കുന്നവരൊക്കെയും അവരെ ശപിക്കുന്നതാണ്‌.
\end{malayalam}}
\flushright{\begin{Arabic}
\quranayah[2][160]
\end{Arabic}}
\flushleft{\begin{malayalam}
എന്നാല്‍ പശ്ചാത്തപിക്കുകയും, നിലപാട് നന്നാക്കിത്തീര്‍ക്കുകയും, (സത്യം ജനങ്ങള്‍ക്ക്‌) വിവരിച്ചുകൊടുക്കുകയും ചെയ്തവര്‍ ഇതില്‍ നിന്നൊഴിവാകുന്നു. അങ്ങനെയുള്ളവരുടെ പശ്ചാത്താപം ഞാന്‍ സ്വീകരിക്കുന്നതാണ്‌. ഞാന്‍ അത്യധികം പശ്ചാത്താപം സ്വീകരിക്കുന്നവനും കരുണാനിധിയുമത്രെ.
\end{malayalam}}
\flushright{\begin{Arabic}
\quranayah[2][161]
\end{Arabic}}
\flushleft{\begin{malayalam}
സത്യം നിഷേധിക്കുകയും, നിഷേധികളായിത്തന്നെ മരിക്കുകയും ചെയ്തവരാരോ അവരുടെ മേല്‍ അല്ലാഹുവിന്‍റെയും മലക്കുകളുടെയും മനുഷ്യരുടെയും ഒന്നടങ്കം ശാപമുണ്ടായിരിക്കുന്നതാണ്‌.
\end{malayalam}}
\flushright{\begin{Arabic}
\quranayah[2][162]
\end{Arabic}}
\flushleft{\begin{malayalam}
അതവര്‍ ശാശ്വതമായി അനുഭവിക്കുന്നതാണ്‌. അവര്‍ക്ക് ശിക്ഷ ഇളവ് ചെയ്യപ്പെടുകയില്ല. അവര്‍ക്ക് ഇടകൊടുക്കപ്പെടുകയുമില്ല.
\end{malayalam}}
\flushright{\begin{Arabic}
\quranayah[2][163]
\end{Arabic}}
\flushleft{\begin{malayalam}
നിങ്ങളുടെ ദൈവം ഏകദൈവം മാത്രമാകുന്നു. അവനല്ലാതെ യാതൊരു ദൈവവുമില്ല. അവന്‍ പരമകാരുണികനും കരുണാനിധിയുമത്രെ.
\end{malayalam}}
\flushright{\begin{Arabic}
\quranayah[2][164]
\end{Arabic}}
\flushleft{\begin{malayalam}
ആകാശഭൂമികളുടെ സൃഷ്ടിപ്പിലും, രാപകലുകളുടെ മാറ്റത്തിലും, മനുഷ്യര്‍ക്ക് ഉപകാരമുള്ള വസ്തുക്കളുമായി കടലിലൂടെ സഞ്ചരിക്കുന്ന കപ്പലിലും, ആകാശത്ത് നിന്ന് അല്ലാഹു മഴ ചൊരിഞ്ഞുതന്നിട്ട് നിര്‍ജീവാവസ്ഥയ്ക്കു ശേഷം ഭൂമിക്ക് അതു മുഖേന ജീവന്‍ നല്‍കിയതിലും, ഭൂമിയില്‍ എല്ലാതരം ജന്തുവര്‍ഗങ്ങളെയും വ്യാപിപ്പിച്ചതിലും, കാറ്റുകളുടെ ഗതിക്രമത്തിലും, ആകാശഭൂമികള്‍ക്കിടയിലൂടെ നിയന്ത്രിച്ച് നയിക്കപ്പെടുന്ന മേഘത്തിലും ചിന്തിക്കുന്ന ജനങ്ങള്‍ക്ക് പല ദൃഷ്ടാന്തങ്ങളുമുണ്ട്‌; തീര്‍ച്ച.
\end{malayalam}}
\flushright{\begin{Arabic}
\quranayah[2][165]
\end{Arabic}}
\flushleft{\begin{malayalam}
അല്ലാഹുവിന് പുറമെയുള്ളവരെ അവന് സമന്‍മാരാക്കുന്ന ചില ആളുകളുണ്ട്‌. അല്ലാഹുവെ സ്നേഹിക്കുന്നത് പോലെ ഈ ആളുകള്‍ അവരെയും സ്നേഹിക്കുന്നു. എന്നാല്‍ സത്യവിശ്വാസികള്‍ അല്ലാഹുവോട് അതിശക്തമായ സ്നേഹമുള്ളവരത്രെ. ഈ അക്രമികള്‍ പരലോകശിക്ഷ കണ്‍മുമ്പില്‍ കാണുന്ന സമയത്ത് ശക്തി മുഴുവന്‍ അല്ലാഹുവിനാണെന്നും അല്ലാഹു കഠിനമായി ശിക്ഷിക്കുന്നവനാണെന്നും അവര്‍ കണ്ടറിഞ്ഞിരുന്നുവെങ്കില്‍ (അതവര്‍ക്ക് എത്ര ഗുണകരമാകുമായിരുന്നു!)
\end{malayalam}}
\flushright{\begin{Arabic}
\quranayah[2][166]
\end{Arabic}}
\flushleft{\begin{malayalam}
പിന്തുടരപ്പെട്ടവര്‍ (നേതാക്കള്‍) പിന്തുടര്‍ന്നവരെ (അനുയായികളെ) വിട്ട് ഒഴിഞ്ഞ് മാറുകയും, ശിക്ഷ നേരില്‍ കാണുകയും, അവര്‍ (ഇരുവിഭാഗവും) തമ്മിലുള്ള ബന്ധങ്ങള്‍ അറ്റുപോകുകയും ചെയ്യുന്ന സന്ദര്‍ഭമത്രെ (അത്‌.)
\end{malayalam}}
\flushright{\begin{Arabic}
\quranayah[2][167]
\end{Arabic}}
\flushleft{\begin{malayalam}
പിന്തുടര്‍ന്നവര്‍ (അനുയായികള്‍) അന്നു പറയും : ഞങ്ങള്‍ക്ക് (ഇഹലോകത്തേക്ക്‌) ഒരു തിരിച്ചുപോക്കിന്നവസരം കിട്ടിയിരുന്നെങ്കില്‍ ഇവര്‍ ഞങ്ങളെ വിട്ടൊഴിഞ്ഞ് മാറിയത് പോലെ ഞങ്ങള്‍ ഇവരെ വിട്ടും ഒഴിഞ്ഞു മാറുമായിരുന്നു. അപ്രകാരം അവരുടെ കര്‍മ്മങ്ങളെല്ലാം അവര്‍ക്ക് ഖേദത്തിന് കാരണമായി ഭവിച്ചത് അല്ലാഹു അവര്‍ക്ക് കാണിച്ചുകൊടുക്കും. നരകാഗ്നിയില്‍ നിന്ന് അവര്‍ക്ക് പുറത്ത് കടക്കാനാകുകയുമില്ല.
\end{malayalam}}
\flushright{\begin{Arabic}
\quranayah[2][168]
\end{Arabic}}
\flushleft{\begin{malayalam}
മനുഷ്യരേ, ഭൂമിയിലുള്ളതില്‍ നിന്ന് അനുവദനീയവും, വിശിഷ്ടവുമായത് നിങ്ങള്‍ ഭക്ഷിച്ച് കൊള്ളുക. പിശാചിന്‍റെകാലടികളെ നിങ്ങള്‍ പിന്തുടരാതിരിക്കുകയും ചെയ്യുക. അവന്‍ നിങ്ങളുടെ പ്രത്യക്ഷ ശത്രു തന്നെയാകുന്നു.
\end{malayalam}}
\flushright{\begin{Arabic}
\quranayah[2][169]
\end{Arabic}}
\flushleft{\begin{malayalam}
ദുഷ്കൃത്യങ്ങളിലും നീചവൃത്തികളിലും ഏര്‍പെടുവാനും, അല്ലാഹുവിന്റെ പേരില്‍ നിങ്ങള്‍ക്കറിഞ്ഞുകൂടാത്തത് പറഞ്ഞുണ്ടാക്കുവാനുമാണ് അവന്‍ നിങ്ങളോട് കല്‍പിക്കുന്നത്‌.
\end{malayalam}}
\flushright{\begin{Arabic}
\quranayah[2][170]
\end{Arabic}}
\flushleft{\begin{malayalam}
അല്ലാഹു അവതരിപ്പിച്ചത് നിങ്ങള്‍ പിന്‍ പറ്റി ജീവിക്കുക എന്ന് അവരോട് ആരെങ്കിലും പറഞ്ഞാല്‍, അല്ല, ഞങ്ങളുടെ പിതാക്കള്‍ സ്വീകരിച്ചതായി കണ്ടതേ ഞങ്ങള്‍ പിന്‍ പറ്റുകയുള്ളൂ എന്നായിരിക്കും അവര്‍ പറയുന്നത്‌. അവരുടെ പിതാക്കള്‍ യാതൊന്നും ചിന്തിച്ച് മനസ്സിലാക്കാത്തവരും നേര്‍വഴി കണ്ടെത്താത്തവരുമായിരുന്നെങ്കില്‍ പോലും (അവരെ പിന്‍ പറ്റുകയാണോ?)
\end{malayalam}}
\flushright{\begin{Arabic}
\quranayah[2][171]
\end{Arabic}}
\flushleft{\begin{malayalam}
സത്യനിഷേധികളെ ഉപമിക്കാവുന്നത് വിളിയും തെളിയുമല്ലാതെ മറ്റൊന്നും കേള്‍ക്കാത്ത ജന്തുവിനോട് ഒച്ചയിടുന്നവനോടാകുന്നു. അവര്‍ ബധിരരും ഊമകളും അന്ധരുമാകുന്നു. അതിനാല്‍ അവര്‍ (യാതൊന്നും) ചിന്തിച്ചു ഗ്രഹിക്കുകയില്ല.
\end{malayalam}}
\flushright{\begin{Arabic}
\quranayah[2][172]
\end{Arabic}}
\flushleft{\begin{malayalam}
സത്യവിശ്വാസികളേ, നിങ്ങള്‍ക്ക് നാം നല്‍കിയ വസ്തുക്കളില്‍ നിന്ന് വിശിഷ്ടമായത് ഭക്ഷിച്ചു കൊള്ളുക. അല്ലാഹുവോട് നിങ്ങള്‍ നന്ദികാണിക്കുകയും ചെയ്യുക; അവനെ മാത്രമാണ് നിങ്ങള്‍ ആരാധിക്കുന്നതെങ്കില്‍.
\end{malayalam}}
\flushright{\begin{Arabic}
\quranayah[2][173]
\end{Arabic}}
\flushleft{\begin{malayalam}
ശവം, രക്തം, പന്നിമാംസം, അല്ലാഹു അല്ലാത്തവര്‍ക്കായി പ്രഖ്യാപിക്കപ്പെട്ടത് എന്നിവ മാത്രമേ അവന്‍ നിങ്ങള്‍ക്ക് നിഷിദ്ധമാക്കിയിട്ടുള്ളൂ. ഇനി ആരെങ്കിലും (നിഷിദ്ധമായത് ഭക്ഷിക്കുവാന്‍) നിര്‍ബന്ധിതനായാല്‍ അവന്റെ മേല്‍ കുറ്റമില്ല. (എന്നാല്‍) അവന്‍ നിയമലംഘനത്തിനു മുതിരാതിരിക്കുകയും (അനിവാര്യതയുടെ) പരിധി കവിയാതിരിക്കുകയും വേണം. തീര്‍ച്ചയായും അല്ലാഹു ഏറെ പൊറുക്കുന്നവനും കരുണാനിധിയുമാകുന്നു.
\end{malayalam}}
\flushright{\begin{Arabic}
\quranayah[2][174]
\end{Arabic}}
\flushleft{\begin{malayalam}
അല്ലാഹു അവതരിപ്പിച്ച, വേദഗ്രന്ഥത്തിലുള്ള കാര്യങ്ങള്‍ മറച്ചുവെക്കുകയും, അതിന്നു വിലയായി തുച്ഛമായ നേട്ടങ്ങള്‍ നേടിയെടുക്കുകയും ചെയ്യുന്നവരാരോ അവര്‍ തങ്ങളുടെ വയറുകളില്‍ തിന്നു നിറക്കുന്നത് നരകാഗ്നിയല്ലാതെ മറ്റൊന്നുമല്ല. ഉയിര്‍ത്തെഴുന്നേല്‍പിന്റെ നാളില്‍ അല്ലാഹു അവരോട് സംസാരിക്കുകയോ (പാപങ്ങളില്‍ നിന്ന്‌) അവരെ സംശുദ്ധരാക്കുകയോ ചെയ്യുകയില്ല. അവര്‍ക്ക് വേദനയേറിയ ശിക്ഷയുണ്ടായിരിക്കുകയും ചെയ്യും.
\end{malayalam}}
\flushright{\begin{Arabic}
\quranayah[2][175]
\end{Arabic}}
\flushleft{\begin{malayalam}
സന്‍മാര്‍ഗത്തിനു പകരം ദുര്‍മാര്‍ഗവും, പാപമോചനത്തിനു പകരം ശിക്ഷയും വാങ്ങിയവരാകുന്നു അവര്‍. നരകശിക്ഷ അനുഭവിക്കുന്നതില്‍ അവര്‍ക്കെന്തൊരു ക്ഷമയാണ്‌!
\end{malayalam}}
\flushright{\begin{Arabic}
\quranayah[2][176]
\end{Arabic}}
\flushleft{\begin{malayalam}
സത്യം വ്യക്തമാക്കിക്കൊണ്ട് അല്ലാഹു വേദഗ്രന്ഥം അവതരിപ്പിച്ചു കഴിഞ്ഞിട്ടുള്ളതിനാലാണത്‌. വേദഗ്രന്ഥത്തിന്‍റെകാര്യത്തില്‍ ഭിന്നിച്ചവര്‍ (സത്യത്തില്‍ നിന്ന്‌) അകന്ന മാത്സര്യത്തിലാകുന്നു തീര്‍ച്ച.
\end{malayalam}}
\flushright{\begin{Arabic}
\quranayah[2][177]
\end{Arabic}}
\flushleft{\begin{malayalam}
നിങ്ങളുടെ മുഖങ്ങള്‍ കിഴക്കോട്ടോ പടിഞ്ഞാറോട്ടോ തിരിക്കുക എന്നതല്ല പുണ്യം എന്നാല്‍ അല്ലാഹുവിലും, അന്ത്യദിനത്തിലും, മലക്കുകളിലും, വേദഗ്രന്ഥത്തിലും, പ്രവാചകന്‍മാരിലും വിശ്വസിക്കുകയും, സ്വത്തിനോട് പ്രിയമുണ്ടായിട്ടും അത് ബന്ധുക്കള്‍ക്കും, അനാഥകള്‍ക്കും, അഗതികള്‍ക്കും, വഴിപോക്കന്നും, ചോദിച്ചു വരുന്നവര്‍ക്കും, അടിമമോചനത്തിന്നും നല്‍കുകയും, പ്രാര്‍ത്ഥന (നമസ്കാരം) മുറപ്രകാരം നിര്‍വഹിക്കുകയും, സകാത്ത് നല്‍കുകയും, കരാറില്‍ ഏര്‍പെട്ടാല്‍ അത് നിറവേറ്റുകയും, വിഷമതകളും ദുരിതങ്ങളും നേരിടുമ്പോഴും, യുദ്ധരംഗത്തും ക്ഷമ കൈക്കൊള്ളുകയും ചെയ്തവരാരോ അവരാകുന്നു പുണ്യവാന്‍മാര്‍. അവരാകുന്നു സത്യം പാലിച്ചവര്‍. അവര്‍ തന്നെയാകുന്നു (ദോഷബാധയെ) സൂക്ഷിച്ചവര്‍.
\end{malayalam}}
\flushright{\begin{Arabic}
\quranayah[2][178]
\end{Arabic}}
\flushleft{\begin{malayalam}
സത്യവിശ്വാസികളേ, കൊലചെയ്യപ്പെടുന്നവരുടെ കാര്യത്തില്‍ തുല്യശിക്ഷ നടപ്പാക്കുക എന്നത് നിങ്ങള്‍ക്ക് നിയമമാക്കപ്പെട്ടിരിക്കുന്നു. സ്വതന്ത്രനു പകരം സ്വതന്ത്രനും, അടിമയ്ക്കു പകരം അടിമയും, സ്ത്രീക്കു പകരം സ്ത്രീയും (കൊല്ലപ്പെടേണ്ടതാണ്‌.) ഇനി അവന്ന് (കൊലയാളിക്ക്‌) തന്റെ സഹോദരന്റെ പക്ഷത്ത് നിന്ന് വല്ല ഇളവും ലഭിക്കുകയാണെങ്കില്‍ അവന്‍ മര്യാദ പാലിക്കുകയും, നല്ല നിലയില്‍ (നഷ്ടപരിഹാരം) കൊടുത്തു വീട്ടുകയും ചെയ്യേണ്ടതാകുന്നു. നിങ്ങളുടെ രക്ഷിതാവിങ്കല്‍ നിന്നുള്ള ഒരു വിട്ടുവീഴ്ചയും കാരുണ്യവുമാകുന്നു അത്‌. ഇനി അതിനു ശേഷവും ആരെങ്കിലും അതിക്രമം പ്രവര്‍ത്തിക്കുകയാണെങ്കില്‍ അവന് വേദനയേറിയ ശിക്ഷയുണ്ടായിരിക്കും.
\end{malayalam}}
\flushright{\begin{Arabic}
\quranayah[2][179]
\end{Arabic}}
\flushleft{\begin{malayalam}
ബുദ്ധിമാന്‍മാരേ, (അങ്ങനെ) തുല്യശിക്ഷ നല്‍കുന്നതിലാണ് നിങ്ങളുടെ ജീവിതത്തിന്റെ നിലനില്‍പ്‌. നിങ്ങള്‍ സൂക്ഷ്മത പാലിക്കുന്നതിനു വേണ്ടിയത്രെ (ഈ നിയമനിര്‍ദേശങ്ങള്‍).
\end{malayalam}}
\flushright{\begin{Arabic}
\quranayah[2][180]
\end{Arabic}}
\flushleft{\begin{malayalam}
നിങ്ങളിലാര്‍ക്കെങ്കിലും മരണം ആസന്നമാവുമ്പോള്‍, അയാള്‍ ധനം വിട്ടുപോകുന്നുണ്ടെങ്കില്‍ മാതാപിതാക്കള്‍ക്കും, അടുത്ത ബന്ധുക്കള്‍ക്കും വേണ്ടി ന്യായപ്രകാരം വസ്വിയ്യത്ത് ചെയ്യുവാന്‍ നിങ്ങള്‍ നിര്‍ബന്ധമായി കല്‍പിക്കപ്പെട്ടിരിക്കുന്നു. സൂക്ഷ്മത പുലര്‍ത്തുന്നവര്‍ക്ക് ഒരു കടമയത്രെ അത്‌.
\end{malayalam}}
\flushright{\begin{Arabic}
\quranayah[2][181]
\end{Arabic}}
\flushleft{\begin{malayalam}
അത് (വസ്വിയ്യത്ത്‌) കേട്ടതിനു ശേഷം ആരെങ്കിലും അത് മാറ്റിമറിക്കുകയാണെങ്കില്‍ അതിന്റെ കുറ്റം മാറ്റിമറിക്കുന്നവര്‍ക്ക് മാത്രമാകുന്നു. തീര്‍ച്ചയായും അല്ലാഹു എല്ലാം കേള്‍ക്കുന്നവനും അറിയുന്നവനുമാകുന്നു.
\end{malayalam}}
\flushright{\begin{Arabic}
\quranayah[2][182]
\end{Arabic}}
\flushleft{\begin{malayalam}
ഇനി വസ്വിയ്യത്ത് ചെയ്യുന്ന ആളുടെ ഭാഗത്തു നിന്നു തന്നെ അനീതിയോ കുറ്റമോ സംഭവിച്ചതായി ആര്‍ക്കെങ്കിലും ആശങ്ക തോന്നുകയും, അവര്‍ക്കിടയില്‍ (ബന്ധപ്പെട്ട കക്ഷികള്‍ക്കിടയില്‍) രഞ്ജിപ്പുണ്ടാക്കുകയുമാണെങ്കില്‍ അതില്‍ തെറ്റില്ല. തീര്‍ച്ചയായും അല്ലാഹു ഏറെ പൊറുക്കുന്നവനും കരുണാനിധിയുമാകുന്നു.
\end{malayalam}}
\flushright{\begin{Arabic}
\quranayah[2][183]
\end{Arabic}}
\flushleft{\begin{malayalam}
സത്യവിശ്വാസികളേ, നിങ്ങളുടെ മുമ്പുള്ളവരോട് കല്‍പിച്ചിരുന്നത് പോലെത്തന്നെ നിങ്ങള്‍ക്കും നോമ്പ് നിര്‍ബന്ധമായി കല്‍പിക്കപ്പെട്ടിരിക്കുന്നു. നിങ്ങള്‍ ദോഷബാധയെ സൂക്ഷിക്കുവാന്‍ വേണ്ടിയത്രെ അത്‌.
\end{malayalam}}
\flushright{\begin{Arabic}
\quranayah[2][184]
\end{Arabic}}
\flushleft{\begin{malayalam}
എണ്ണപ്പെട്ട ഏതാനും ദിവസങ്ങളില്‍ മാത്രം. നിങ്ങളിലാരെങ്കിലും രോഗിയാവുകയോ യാത്രയിലാവുകയോ ചെയ്താല്‍ മറ്റു ദിവസങ്ങളില്‍ നിന്ന് അത്രയും എണ്ണം (നോമ്പെടുക്കേണ്ടതാണ്‌.) (ഞെരുങ്ങിക്കൊണ്ട് മാത്രം) അതിന്നു സാധിക്കുന്നവര്‍ (പകരം) ഒരു പാവപ്പെട്ടവന്നുള്ള ഭക്ഷണം പ്രായശ്ചിത്തമായി നല്‍കേണ്ടതാണ്‌. എന്നാല്‍ ആരെങ്കിലും സ്വയം സന്നദ്ധനായി കൂടുതല്‍ നന്‍മചെയ്താല്‍ അതവന്ന് ഗുണകരമാകുന്നു. നിങ്ങള്‍ കാര്യം ഗ്രഹിക്കുന്നവരാണെങ്കില്‍ നോമ്പനുഷ്ഠിക്കുന്നതാകുന്നു നിങ്ങള്‍ക്ക് കൂടുതല്‍ ഉത്തമം.
\end{malayalam}}
\flushright{\begin{Arabic}
\quranayah[2][185]
\end{Arabic}}
\flushleft{\begin{malayalam}
ജനങ്ങള്‍ക്ക് മാര്‍ഗദര്‍ശനമായിക്കൊണ്ടും, നേര്‍വഴി കാട്ടുന്നതും സത്യവും അസത്യവും വേര്‍തിരിച്ചു കാണിക്കുന്നതുമായ സുവ്യക്ത തെളിവുകളായിക്കൊണ്ടും വിശുദ്ധ ഖുര്‍ആന്‍ അവതരിപ്പിക്കപ്പെട്ട മാസമാകുന്നു റമളാന്‍. അതു കൊണ്ട് നിങ്ങളില്‍ ആര്‍ ആ മാസത്തില്‍ സന്നിഹിതരാണോ അവര്‍ ആ മാസം വ്രതമനുഷ്ഠിക്കേണ്ടതാണ്‌. ആരെങ്കിലും രോഗിയാവുകയോ, യാത്രയിലാവുകയോ ചെയ്താല്‍ പകരം അത്രയും എണ്ണം (നോമ്പെടുക്കേണ്ടതാണ്‌.) നിങ്ങള്‍ക്ക് ആശ്വാസം വരുത്താനാണ് അല്ലാഹു ഉദ്ദേശിക്കുന്നത്‌. നിങ്ങള്‍ക്ക് ഞെരുക്കം ഉണ്ടാക്കാന്‍ അവന്‍ ഉദ്ദേശിക്കുന്നില്ല. നിങ്ങള്‍ ആ എണ്ണം പൂര്‍ത്തിയാക്കുവാനും, നിങ്ങള്‍ക്ക് നേര്‍വഴി കാണിച്ചുതന്നതിന്റെപേരില്‍ അല്ലാഹുവിന്റെ മഹത്വം നിങ്ങള്‍ പ്രകീര്‍ത്തിക്കുവാനും നിങ്ങള്‍ നന്ദിയുള്ളവരായിരിക്കുവാനും വേണ്ടിയത്രെ (ഇങ്ങനെ കല്‍പിച്ചിട്ടുള്ളത്‌.)
\end{malayalam}}
\flushright{\begin{Arabic}
\quranayah[2][186]
\end{Arabic}}
\flushleft{\begin{malayalam}
നിന്നോട് എന്റെ ദാസന്‍മാര്‍ എന്നെപ്പറ്റി ചോദിച്ചാല്‍ ഞാന്‍ (അവര്‍ക്ക് ഏറ്റവും) അടുത്തുള്ളവനാകുന്നു (എന്ന് പറയുക.) പ്രാര്‍ത്ഥിക്കുന്നവന്‍ എന്നെ വിളിച്ച് പ്രാര്‍ത്ഥിച്ചാല്‍ ഞാന്‍ ആ പ്രാര്‍ത്ഥനയ്ക്ക് ഉത്തരം നല്‍കുന്നതാണ്‌. അതുകൊണ്ട് എന്റെ ആഹ്വാനം അവര്‍ സ്വീകരിക്കുകയും, എന്നില്‍ അവര്‍ വിശ്വസിക്കുകയും ചെയ്യട്ടെ. അവര്‍ നേര്‍വഴി പ്രാപിക്കുവാന്‍ വേണ്ടിയാണിത്‌.
\end{malayalam}}
\flushright{\begin{Arabic}
\quranayah[2][187]
\end{Arabic}}
\flushleft{\begin{malayalam}
നോമ്പിന്റെ രാത്രിയില്‍ നിങ്ങളുടെ ഭാര്യമാരുമായുള്ള സംസര്‍ഗം നിങ്ങള്‍ക്ക് അനുവദിക്കപ്പെട്ടിരിക്കുന്നു. അവര്‍ നിങ്ങള്‍ക്കൊരു വസ്ത്രമാകുന്നു. നിങ്ങള്‍ അവര്‍ക്കും ഒരു വസ്ത്രമാകുന്നു. (ഭാര്യാസമ്പര്‍ക്കം നിഷിദ്ധമായി കരുതിക്കൊണ്ട്‌) നിങ്ങള്‍ ആത്മവഞ്ചനയില്‍ അകപ്പെടുകയായിരുന്നുവെന്ന് അല്ലാഹു അറിഞ്ഞിരിക്കുന്നു. എന്നാല്‍ അല്ലാഹു നിങ്ങളുടെ പശ്ചാത്താപം സ്വീകരിക്കുകയും പൊറുക്കുകയും ചെയ്തിരിക്കുന്നു. അതിനാല്‍ ഇനി മേല്‍ നിങ്ങള്‍ അവരുമായി സഹവസിക്കുകയും, (വൈവാഹിക ജീവിതത്തില്‍) അല്ലാഹു നിങ്ങള്‍ക്ക് നിശ്ചയിച്ചത് തേടുകയും ചെയ്തുകൊള്ളുക. നിങ്ങള്‍ തിന്നുകയും കുടിക്കുകയും ചെയ്തുകൊള്ളുക; പുലരിയുടെ വെളുത്ത ഇഴകള്‍ കറുത്ത ഇഴകളില്‍ നിന്ന് തെളിഞ്ഞ് കാണുമാറാകുന്നത് വരെ. എന്നിട്ട് രാത്രിയാകും വരെ നിങ്ങള്‍ വ്രതം പൂര്‍ണ്ണമായി അനുഷ്ഠിക്കുകയും ചെയ്യുക. എന്നാല്‍ നിങ്ങള്‍ പള്ളികളില്‍ ഇഅ്തികാഫ് (ഭജനം) ഇരിക്കുമ്പോള്‍ അവരു (ഭാര്യമാരു) മായി സഹവസിക്കരുത്‌. അല്ലാഹുവിന്റെ അതിര്‍വരമ്പുകളാകുന്നു അവയൊക്കെ. നിങ്ങള്‍ അവയെ അതിലംഘിക്കുവാനടുക്കരുത്‌. ജനങ്ങള്‍ ദോഷബാധയെ സൂക്ഷിക്കുവാനായി അല്ലാഹു അപ്രകാരം അവന്റെ ദൃഷ്ടാന്തങ്ങള്‍ അവര്‍ക്ക് വ്യക്തമാക്കികൊടുക്കുന്നു.
\end{malayalam}}
\flushright{\begin{Arabic}
\quranayah[2][188]
\end{Arabic}}
\flushleft{\begin{malayalam}
അന്യായമായി നിങ്ങള്‍ അന്യോന്യം സ്വത്തുക്കള്‍ തിന്നരുത്‌. അറിഞ്ഞുകൊണ്ടു തന്നെ, ആളുകളുടെ സ്വത്തുക്കളില്‍ നിന്ന് വല്ലതും അധാര്‍മ്മികമായി നേടിയെടുത്തു തിന്നുവാന്‍ വേണ്ടി നിങ്ങളതുമായി വിധികര്‍ത്താക്കളെ സമീപിക്കുകയും ചെയ്യരുത്‌.
\end{malayalam}}
\flushright{\begin{Arabic}
\quranayah[2][189]
\end{Arabic}}
\flushleft{\begin{malayalam}
(നബിയേ,) നിന്നോടവര്‍ ചന്ദ്രക്കലകളെപ്പറ്റി ചോദിക്കുന്നു. പറയുക: മനുഷ്യരുടെ ആവശ്യങ്ങള്‍ക്കും ഹജ്ജ് തീര്‍ത്ഥാടനത്തിനും കാല നിര്‍ണയത്തിനുള്ള ഉപാധികളാകുന്നു അവ. നിങ്ങള്‍ വീടുകളിലേക്ക് പിന്‍വശങ്ങളിലൂടെ ചെല്ലുന്നതിലല്ല പുണ്യം കുടികൊള്ളുന്നത്‌. പ്രത്യുത, ദോഷബാധയെ കാത്തുസൂക്ഷിച്ചവനത്രെ പുണ്യവാന്‍. നിങ്ങള്‍ വീടുകളില്‍ അവയുടെ വാതിലുകളിലൂടെ പ്രവേശിക്കുക. മോക്ഷം കൈവരിക്കുവാന്‍ നിങ്ങള്‍ അല്ലാഹുവെ സൂക്ഷിക്കുകയും ചെയ്യുക.
\end{malayalam}}
\flushright{\begin{Arabic}
\quranayah[2][190]
\end{Arabic}}
\flushleft{\begin{malayalam}
നിങ്ങളോട് യുദ്ധം ചെയ്യുന്നവരുമായി അല്ലാഹുവിന്റെ മാര്‍ഗത്തില്‍ നിങ്ങളും യുദ്ധം ചെയ്യുക. എന്നാല്‍ നിങ്ങള്‍ പരിധിവിട്ട് പ്രവര്‍ത്തിക്കരുത്‌. പരിധിവിട്ട് പ്രവര്‍ത്തിക്കുന്നവരെ അല്ലാഹു ഇഷ്ടപ്പെടുകയില്ല തന്നെ.
\end{malayalam}}
\flushright{\begin{Arabic}
\quranayah[2][191]
\end{Arabic}}
\flushleft{\begin{malayalam}
അവരെ കണ്ടുമുട്ടുന്നേടത്ത് വെച്ച് നിങ്ങളവരെ കൊന്നുകളയുകയും, അവര്‍ നിങ്ങളെ പുറത്താക്കിയേടത്ത് നിന്ന് നിങ്ങള്‍ അവരെ പുറത്താക്കുകയും ചെയ്യുക. (കാരണം, അവര്‍ നടത്തുന്ന) മര്‍ദ്ദനം കൊലയേക്കാള്‍ നിഷ്ഠൂരമാകുന്നു. മസ്ജിദുല്‍ ഹറാമിന്നടുത്ത് വെച്ച് നിങ്ങള്‍ അവരോട് യുദ്ധം ചെയ്യരുത്‌; അവര്‍ നിങ്ങളോട് അവിടെ വെച്ച് യുദ്ധം ചെയ്യുന്നത് വരെ. ഇനി അവര്‍ നിങ്ങളോട് (അവിടെ വെച്ച്‌) യുദ്ധത്തില്‍ ഏര്‍പെടുകയാണെങ്കില്‍ അവരെ കൊന്നുകളയുക. അപ്രകാരമാണ് സത്യനിഷേധികള്‍ക്കുള്ള പ്രതിഫലം.
\end{malayalam}}
\flushright{\begin{Arabic}
\quranayah[2][192]
\end{Arabic}}
\flushleft{\begin{malayalam}
ഇനി അവര്‍ (പശ്ചാത്തപിച്ച്‌, എതിര്‍പ്പില്‍ നിന്ന്‌) വിരമിക്കുകയാണെങ്കിലോ തീര്‍ച്ചയായും ഏറെ പൊറുക്കുന്നവനും കരുണാനിധിയുമാണ് അല്ലാഹു.
\end{malayalam}}
\flushright{\begin{Arabic}
\quranayah[2][193]
\end{Arabic}}
\flushleft{\begin{malayalam}
മര്‍ദ്ദനം ഇല്ലാതാവുകയും, മതം അല്ലാഹുവിന് വേണ്ടിയാവുകയും ചെയ്യുന്നത് വരെ നിങ്ങളവരോട് യുദ്ധം നടത്തിക്കൊള്ളുക. എന്നാല്‍ അവര്‍ (യുദ്ധത്തില്‍ നിന്ന്‌) വിരമിക്കുകയാണെങ്കില്‍ (അവരിലെ) അക്രമികള്‍ക്കെതിരിലല്ലാതെ പിന്നീട് യാതൊരു കയ്യേറ്റവും പാടുള്ളതല്ല.
\end{malayalam}}
\flushright{\begin{Arabic}
\quranayah[2][194]
\end{Arabic}}
\flushleft{\begin{malayalam}
വിലക്കപ്പെട്ടമാസത്തി (ലെ യുദ്ധത്തി) ന് വിലക്കപ്പെട്ടമാസത്തില്‍ തന്നെ (തിരിച്ചടിക്കുക.) വിലക്കപ്പെട്ട മറ്റു കാര്യങ്ങള്‍ ലംഘിക്കുമ്പോഴും (അങ്ങനെത്തന്നെ) പ്രതിക്രിയ ചെയ്യേണ്ടതാണ്‌. അപ്രകാരം നിങ്ങള്‍ക്കെതിരെ ആര്‍ അതിക്രമം കാണിച്ചാലും അവന്‍ നിങ്ങളുടെ നേര്‍ക്ക് കാണിച്ച അതിക്രമത്തിന് തുല്യമായി അവന്റെ നേരെയും അതിക്രമം കാണിച്ചുകൊള്ളുക. നിങ്ങള്‍ അല്ലാഹുവെ സൂക്ഷിക്കുകയും, അല്ലാഹു സൂക്ഷ്മത പാലിക്കുന്നവരോടൊപ്പമാണെന്ന് മനസ്സിലാക്കുകയും ചെയ്യുക.
\end{malayalam}}
\flushright{\begin{Arabic}
\quranayah[2][195]
\end{Arabic}}
\flushleft{\begin{malayalam}
അല്ലാഹുവിന്റെ മാര്‍ഗത്തില്‍ നിങ്ങള്‍ ചെലവ് ചെയ്യുക. (പിശുക്കും ഉദാസീനതയും മൂലം) നിങ്ങളുടെ കൈകളെ നിങ്ങള്‍ തന്നെ നാശത്തില്‍ തള്ളിക്കളയരുത്‌. നിങ്ങള്‍ നല്ലത് പ്രവര്‍ത്തിക്കുക. നന്‍മ ചെയ്യുന്നവരെ അല്ലാഹു ഇഷ്ടപെടുക തന്നെ ചെയ്യും.
\end{malayalam}}
\flushright{\begin{Arabic}
\quranayah[2][196]
\end{Arabic}}
\flushleft{\begin{malayalam}
നിങ്ങള്‍ അല്ലാഹുവിന് വേണ്ടി ഹജ്ജും ഉംറഃയും പൂര്‍ണ്ണമായി നിര്‍വഹിക്കുക. ഇനി നിങ്ങള്‍ക്ക് (ഹജ്ജ് നിര്‍വഹിക്കുന്നതിന്‌) തടസ്സം സൃഷ്ടിക്കപ്പെട്ടാല്‍ നിങ്ങള്‍ക്ക് സൗകര്യപ്പെടുന്ന ഒരു ബലിമൃഗത്തെ (ബലിയര്‍പ്പിക്കേണ്ടതാണ്‌.) ബലിമൃഗം എത്തേണ്ട സ്ഥാനത്ത് എത്തുന്നത് വരെ നിങ്ങള്‍ തല മുണ്ഡനം ചെയ്യാവുന്നതല്ല. നിങ്ങളിലാരെങ്കിലും രോഗിയാവുകയോ, തലയില്‍ വല്ല ശല്യവും അനുഭവപ്പെടുകയോ ആണെങ്കില്‍ (മുടി നീക്കുന്നതിന്‌) പ്രായശ്ചിത്തമായി നോമ്പോ, ദാനധര്‍മ്മമോ, ബലികര്‍മ്മമോ നിര്‍വഹിച്ചാല്‍ മതിയാകും. ഇനി നിങ്ങള്‍ നിര്‍ഭയാവസ്ഥയിലാണെങ്കിലോ, അപ്പോള്‍ ഒരാള്‍ ഉംറഃ നിര്‍വഹിച്ചിട്ട് ഹജ്ജ് വരെ സുഖമെടുക്കുന്ന പക്ഷം സൗകര്യപ്പെടുന്ന ഒരു ബലിമൃഗത്തെ (ഹജ്ജിനിടയില്‍ ബലികഴിക്കേണ്ടതാണ്‌.) ഇനി ആര്‍ക്കെങ്കിലും അത് കിട്ടാത്ത പക്ഷം ഹജ്ജിനിടയില്‍ മൂന്നു ദിവസവും, നിങ്ങള്‍ (നാട്ടില്‍) തിരിച്ചെത്തിയിട്ട് ഏഴു ദിവസവും ചേര്‍ത്ത് ആകെ പത്ത് ദിവസം നോമ്പനുഷ്ഠിക്കേണ്ടതാണ്‌. കുടുംബസമേതം മസ്ജിദുല്‍ ഹറാമില്‍ താമസിക്കുന്നവര്‍ക്കല്ലാത്തവര്‍ക്കാകുന്നു ഈ വിധി. നിങ്ങള്‍ അല്ലാഹുവെ സൂക്ഷിക്കുകയും, അല്ലാഹു കഠിനമായി ശിക്ഷിക്കുന്നവനാണെന്ന് മനസ്സിലാക്കുകയും ചെയ്യുക.
\end{malayalam}}
\flushright{\begin{Arabic}
\quranayah[2][197]
\end{Arabic}}
\flushleft{\begin{malayalam}
ഹജ്ജ് കാലം അറിയപ്പെട്ട മാസങ്ങളാകുന്നു. ആ മാസങ്ങളില്‍ ആരെങ്കിലും ഹജ്ജ് കര്‍മ്മത്തില്‍ പ്രവേശിച്ചാല്‍ പിന്നീട് സ്ത്രീ-പുരുഷ സംസര്‍ഗമോ ദുര്‍വൃത്തിയോ വഴക്കോ ഹജ്ജിനിടയില്‍ പാടുള്ളതല്ല. നിങ്ങള്‍ ഏതൊരു സല്‍പ്രവൃത്തി ചെയ്തിരുന്നാലും അല്ലാഹു അതറിയുന്നതാണ്‌. (ഹജ്ജിനു പോകുമ്പോള്‍) നിങ്ങള്‍ യാത്രയ്ക്കുവേണ്ട വിഭവങ്ങള്‍ ഒരുക്കിപ്പോകുക. എന്നാല്‍ യാത്രയ്ക്കു വേണ്ട വിഭവങ്ങളില്‍ ഏറ്റവും ഉത്തമമായത് സൂക്ഷ്മതയാകുന്നു. ബുദ്ധിശാലികളേ, നിങ്ങളെന്നെ സൂക്ഷിച്ച് ജീവിക്കുക.
\end{malayalam}}
\flushright{\begin{Arabic}
\quranayah[2][198]
\end{Arabic}}
\flushleft{\begin{malayalam}
(ഹജ്ജിനിടയില്‍) നിങ്ങളുടെ രക്ഷിതാവിങ്കല്‍ നിന്നുള്ള ഭൌതികാനുഗ്രഹങ്ങള്‍ നിങ്ങള്‍ തേടുന്നതില്‍ കുറ്റമൊന്നുമില്ല. അറഫാത്തില്‍ നിന്ന് നിങ്ങള്‍ പുറപ്പെട്ടുകഴിഞ്ഞാല്‍ മശ്‌അറുല്‍ ഹറാമിനടുത്തുവെച്ച് നിങ്ങള്‍ അല്ലാഹുവിനെ പ്രകീര്‍ത്തിക്കുവിന്‍. അവന്‍ നിങ്ങള്‍ക്ക് വഴി കാണിച്ച പ്രകാരം നിങ്ങളവനെ ഓര്‍ക്കുവിന്‍. ഇതിനു മുമ്പ് നിങ്ങള്‍ പിഴച്ചവരില്‍ പെട്ടവരായിരുന്നാലും.
\end{malayalam}}
\flushright{\begin{Arabic}
\quranayah[2][199]
\end{Arabic}}
\flushleft{\begin{malayalam}
എന്നിട്ട് ആളുകള്‍ (സാധാരണ തീര്‍ത്ഥാടകര്‍) എവിടെ നിന്ന് പുറപ്പെടുന്നുവോ അവിടെ നിന്നു തന്നെ നിങ്ങളും പുറപ്പെടുക. നിങ്ങള്‍ അല്ലാഹുവോട് പാപമോചനം തേടുകയും ചെയ്യുക. അല്ലാഹു ഏറെ പൊറുക്കുന്നവനും കരുണാനിധിയുമാകുന്നു.
\end{malayalam}}
\flushright{\begin{Arabic}
\quranayah[2][200]
\end{Arabic}}
\flushleft{\begin{malayalam}
അങ്ങനെ നിങ്ങള്‍ ഹജ്ജ് കര്‍മ്മം നിര്‍വഹിച്ചു കഴിഞ്ഞാല്‍ നിങ്ങളുടെ പിതാക്കളെ നിങ്ങള്‍ പ്രകീര്‍ത്തിച്ചിരുന്നത് പോലെയോ അതിനെക്കാള്‍ ശക്തമായ നിലയിലോ അല്ലാഹുവെ നിങ്ങള്‍ പ്രകീര്‍ത്തിക്കുക. മനുഷ്യരില്‍ ചിലര്‍ പറയും; ഞങ്ങളുടെ രക്ഷിതാവേ, ഇഹലോകത്ത് ഞങ്ങള്‍ക്ക് നീ (അനുഗ്രഹം) നല്‍കേണമേ എന്ന്‌. എന്നാല്‍ പരലോകത്ത് അത്തരക്കാര്‍ക്ക് ഒരു ഓഹരിയും ഉണ്ടായിരിക്കുന്നതല്ല.
\end{malayalam}}
\flushright{\begin{Arabic}
\quranayah[2][201]
\end{Arabic}}
\flushleft{\begin{malayalam}
മറ്റു ചിലര്‍ പറയും; ഞങ്ങളുടെ രക്ഷിതാവേ, ഞങ്ങള്‍ക്ക് ഇഹലോകത്ത് നീ നല്ലത് തരേണമേ; പരലോകത്തും നീ നല്ലത് തരേണമേ. നരകശിക്ഷയില്‍ നിന്ന് ഞങ്ങളെ കാത്തുരക്ഷിക്കുകയും ചെയ്യേണമേ എന്ന്‌.
\end{malayalam}}
\flushright{\begin{Arabic}
\quranayah[2][202]
\end{Arabic}}
\flushleft{\begin{malayalam}
അവര്‍ സമ്പാദിച്ചതിന്റെ ഫലമായി അവര്‍ക്ക് വലിയൊരു വിഹിതമുണ്ട്‌. അല്ലാഹു അതിവേഗത്തില്‍ കണക്ക് നോക്കുന്നവനാകുന്നു.
\end{malayalam}}
\flushright{\begin{Arabic}
\quranayah[2][203]
\end{Arabic}}
\flushleft{\begin{malayalam}
എണ്ണപ്പെട്ട ദിവസങ്ങളില്‍ നിങ്ങള്‍ അല്ലാഹുവെ സ്മരിക്കുക. (അവയില്‍) രണ്ടു ദിവസം കൊണ്ട് മതിയാക്കി ആരെങ്കിലും ധൃതിപ്പെട്ട് പോരുന്ന പക്ഷം അവന് കുറ്റമില്ല. (ഒരു ദിവസവും കൂടി) താമസിച്ചു പോരുന്നവന്നും കുറ്റമില്ല. സൂക്ഷ്മത പാലിക്കുന്നവന്ന് (അതാണ് ഉത്തമം). നിങ്ങള്‍ അല്ലാഹുവെ സൂക്ഷിക്കുകയും അവങ്കലേക്ക് നിങ്ങള്‍ ഒരുമിച്ചുകൂട്ടപ്പെടുമെന്ന് മനസ്സിലാക്കുകയും ചെയ്യുക.
\end{malayalam}}
\flushright{\begin{Arabic}
\quranayah[2][204]
\end{Arabic}}
\flushleft{\begin{malayalam}
ചില ആളുകളുണ്ട്‌. ഐഹികജീവിത കാര്യത്തില്‍ അവരുടെ സംസാരം നിനക്ക് കൗതുകം തോന്നിക്കും. അവരുടെ ഹൃദയശുദ്ധിക്ക് അവര്‍ അല്ലാഹുവെ സാക്ഷി നിര്‍ത്തുകയും ചെയ്യും. വാസ്തവത്തില്‍ അവര്‍ (സത്യത്തിന്റെ) കഠിനവൈരികളത്രെ.
\end{malayalam}}
\flushright{\begin{Arabic}
\quranayah[2][205]
\end{Arabic}}
\flushleft{\begin{malayalam}
അവര്‍ തിരിച്ചുപോയാല്‍ ഭൂമിയില്‍ കുഴപ്പമുണ്ടാക്കാനും, വിള നശിപ്പിക്കാനും, ജീവനൊടുക്കാനുമായിരിക്കും ശ്രമിക്കുക. നശീകരണം അല്ലാഹു ഇഷ്ടപ്പെടുന്നതല്ല.
\end{malayalam}}
\flushright{\begin{Arabic}
\quranayah[2][206]
\end{Arabic}}
\flushleft{\begin{malayalam}
അല്ലാഹുവെ സൂക്ഷിക്കുക എന്ന് അവരോട് ആരെങ്കിലും പറഞ്ഞാല്‍ ദുരഭിമാനം അവരെ പാപത്തില്‍ പിടിച്ച് നിര്‍ത്തുന്നു. അവര്‍ക്ക് നരകം തന്നെ മതി. അത് എത്ര മോശമായ മെത്ത!
\end{malayalam}}
\flushright{\begin{Arabic}
\quranayah[2][207]
\end{Arabic}}
\flushleft{\begin{malayalam}
വേറെ ചില ആളുകളുണ്ട്‌. അല്ലാഹുവിന്റെ പൊരുത്തം തേടിക്കൊണ്ട് സ്വന്തം ജീവിതം തന്നെ അവര്‍ വില്‍ക്കുന്നു. അല്ലാഹു തന്റെ ദാസന്‍മാരോട് അത്യധികം ദയയുള്ളവനാകുന്നു.
\end{malayalam}}
\flushright{\begin{Arabic}
\quranayah[2][208]
\end{Arabic}}
\flushleft{\begin{malayalam}
സത്യവിശ്വാസികളേ, നിങ്ങള്‍ പരിപൂര്‍ണ്ണമായി കീഴ്‌വണക്കത്തില്‍ പ്രവേശിക്കുക. പിശാചിന്റെ കാലടികളെ നിങ്ങള്‍ പിന്തുടരാതിരിക്കുകയും ചെയ്യുക. തീര്‍ച്ചയായും അവന്‍ നിങ്ങളുടെ പ്രത്യക്ഷ ശത്രുവാകുന്നു.
\end{malayalam}}
\flushright{\begin{Arabic}
\quranayah[2][209]
\end{Arabic}}
\flushleft{\begin{malayalam}
നിങ്ങള്‍ക്ക് വ്യക്തമായ തെളിവുകള്‍ വന്നുകിട്ടിയതിനു ശേഷവും നിങ്ങള്‍ വഴുതിപ്പോകുകയാണെങ്കില്‍ നിങ്ങള്‍ മനസ്സിലാക്കണം; അല്ലാഹു പ്രതാപിയും യുക്തിമാനുമാണെന്ന്‌.
\end{malayalam}}
\flushright{\begin{Arabic}
\quranayah[2][210]
\end{Arabic}}
\flushleft{\begin{malayalam}
മേഘമേലാപ്പില്‍ അല്ലാഹുവും മലക്കുകളും അവരുടെയടുത്ത് വരുകയും, കാര്യം തീരുമാനിക്കപ്പെടുകയും ചെയ്യുന്നത് മാത്രമാണോ അവര്‍ കാത്തിരിക്കുന്നത്‌? എന്നാല്‍ കാര്യങ്ങളെല്ലാം അല്ലാഹുവിങ്കലേക്കാകുന്നു മടക്കപ്പെടുന്നത്‌.
\end{malayalam}}
\flushright{\begin{Arabic}
\quranayah[2][211]
\end{Arabic}}
\flushleft{\begin{malayalam}
ഇസ്രായീല്യരോട് നീ ചോദിച്ച് നോക്കുക; വ്യക്തമായ എത്ര ദൃഷ്ടാന്തമാണ് നാം അവര്‍ക്ക് നല്‍കിയിട്ടുള്ളതെന്ന്‌. തനിക്ക് അല്ലാഹുവിന്റെ അനുഗ്രഹം വന്നുകിട്ടിയതിനു ശേഷം വല്ലവനും അതിന് വിപരീതം പ്രവര്‍ത്തിക്കുകയാണെങ്കില്‍ തീര്‍ച്ചയായും അല്ലാഹു കഠിനമായി ശിക്ഷിക്കുന്നവനാകുന്നു.
\end{malayalam}}
\flushright{\begin{Arabic}
\quranayah[2][212]
\end{Arabic}}
\flushleft{\begin{malayalam}
സത്യനിഷേധികള്‍ക്ക് ഐഹികജീവിതം അലംകൃതമായി തോന്നിയിരിക്കുന്നു. സത്യവിശ്വാസികളെ അവര്‍ പരിഹസിക്കുകയും ചെയ്യുന്നു. എന്നാല്‍ സൂക്ഷ്മത പാലിച്ചവരായിരിക്കും ഉയിര്‍ത്തെഴുന്നേല്‍പിന്റെ നാളില്‍ അവരെക്കാള്‍ ഉന്നതന്‍മാര്‍. അല്ലാഹു അവന്‍ ഉദ്ദേശിക്കുന്നവര്‍ക്ക് കണക്ക് നോക്കാതെ തന്നെ കൊടുക്കുന്നതാണ്‌.
\end{malayalam}}
\flushright{\begin{Arabic}
\quranayah[2][213]
\end{Arabic}}
\flushleft{\begin{malayalam}
മനുഷ്യര്‍ ഒരൊറ്റ സമുദായമായിരുന്നു. അനന്തരം (അവര്‍ ഭിന്നിച്ചപ്പോള്‍ വിശ്വാസികള്‍ക്ക്‌) സന്തോഷവാര്‍ത്ത അറിയിക്കുവാനും, (നിഷേധികള്‍ക്ക്‌) താക്കീത് നല്‍കുവാനുമായി അല്ലാഹു പ്രവാചകന്‍മാരെ നിയോഗിച്ചു. അവര്‍ (ജനങ്ങള്‍) ഭിന്നിച്ച വിഷയത്തില്‍ തീര്‍പ്പുകല്‍പിക്കുവാനായി അവരുടെ കൂടെ സത്യവേദവും അവന്‍ അയച്ചുകൊടുത്തു. എന്നാല്‍ വേദം നല്‍കപ്പെട്ടവര്‍ തന്നെ വ്യക്തമായ തെളിവുകള്‍ വന്നുകിട്ടിയതിനു ശേഷം അതില്‍ (വേദവിഷയത്തില്‍) ഭിന്നിച്ചിട്ടുള്ളത് അവര്‍ തമ്മിലുള്ള മാത്സര്യം മൂലമല്ലാതെ മറ്റൊന്നുകൊണ്ടുമല്ല. എന്നാല്‍ ഏതൊരു സത്യത്തില്‍ നിന്ന് അവര്‍ ഭിന്നിച്ചകന്നുവോ ആ സത്യത്തിലേക്ക് അല്ലാഹു തന്റെ താല്‍പര്യപ്രകാരം സത്യവിശ്വാസികള്‍ക്ക് വഴി കാണിച്ചു. താന്‍ ഉദ്ദേശിക്കുന്നവരെ അല്ലാഹു ശരിയായ പാതയിലേക്ക് നയിക്കുന്നു.
\end{malayalam}}
\flushright{\begin{Arabic}
\quranayah[2][214]
\end{Arabic}}
\flushleft{\begin{malayalam}
അല്ല, നിങ്ങളുടെ മുമ്പ് കഴിഞ്ഞുപോയവര്‍ (വിശ്വാസികള്‍) ക്കുണ്ടായതു പോലുള്ള അനുഭവങ്ങള്‍ നിങ്ങള്‍ക്കും വന്നെത്താതെ നിങ്ങള്‍ക്ക് സ്വര്‍ഗത്തില്‍ പ്രവേശിക്കാനാകുമെന്ന് നിങ്ങള്‍ ധരിച്ചിരിക്കയാണോ ? പ്രയാസങ്ങളും ദുരിതങ്ങളും അവരെ ബാധിക്കുകയുണ്ടായി. അല്ലാഹുവിന്റെ സഹായം എപ്പോഴായിരിക്കും എന്ന് അവരിലെ ദൈവദൂതനും അദ്ദേഹത്തോടൊപ്പം വിശ്വസിച്ചവരും പറഞ്ഞുപോകുമാറ് അവര്‍ വിറപ്പിക്കപ്പെടുകയും ചെയ്തു. എന്നാല്‍ അല്ലാഹുവിന്റെ സഹായം അടുത്തു തന്നെയുണ്ട്‌.
\end{malayalam}}
\flushright{\begin{Arabic}
\quranayah[2][215]
\end{Arabic}}
\flushleft{\begin{malayalam}
(നബിയേ,) അവര്‍ നിന്നോട് ചോദിക്കുന്നു; അവരെന്താണ് ചെലവഴിക്കേണ്ടതെന്ന്‌. നീ പറയുക: നിങ്ങള്‍ നല്ലതെന്ത് ചെലവഴിക്കുകയാണെങ്കിലും മാതാപിതാക്കള്‍ക്കും അടുത്ത ബന്ധുക്കള്‍ക്കും അനാഥര്‍ക്കും അഗതികള്‍ക്കും വഴിപോക്കന്‍മാര്‍ക്കും വേണ്ടിയാണത് ചെയ്യേണ്ടത്‌. നല്ലതെന്ത് നിങ്ങള്‍ ചെയ്യുകയാണെങ്കിലും തീര്‍ച്ചയായും അല്ലാഹു അതറിയുന്നവനാകുന്നു.
\end{malayalam}}
\flushright{\begin{Arabic}
\quranayah[2][216]
\end{Arabic}}
\flushleft{\begin{malayalam}
യുദ്ധം ചെയ്യാന്‍ നിങ്ങള്‍ക്കിതാ നിര്‍ബന്ധ കല്‍പന നല്‍കപ്പെട്ടിരിക്കുന്നു. അതാകട്ടെ നിങ്ങള്‍ക്ക് അനിഷ്ടകരമാകുന്നു. എന്നാല്‍ ഒരു കാര്യം നിങ്ങള്‍ വെറുക്കുകയും (യഥാര്‍ത്ഥത്തില്‍) അത് നിങ്ങള്‍ക്ക് ഗുണകരമായിരിക്കുകയും ചെയ്യാം. നിങ്ങളൊരു കാര്യം ഇഷ്ടപ്പെടുകയും (യഥാര്‍ത്ഥത്തില്‍) നിങ്ങള്‍ക്കത് ദോഷകരമായിരിക്കുകയും ചെയ്തെന്നും വരാം. അല്ലാഹു അറിയുന്നു. നിങ്ങള്‍ അറിയുന്നില്ല.
\end{malayalam}}
\flushright{\begin{Arabic}
\quranayah[2][217]
\end{Arabic}}
\flushleft{\begin{malayalam}
വിലക്കപ്പെട്ടമാസത്തില്‍ യുദ്ധം ചെയ്യുന്നതിനെപ്പറ്റി അവര്‍ നിന്നോട് ചോദിക്കുന്നു. പറയുക: ആ മാസത്തില്‍ യുദ്ധം ചെയ്യുന്നത് വലിയ അപരാധം തന്നെയാകുന്നു. എന്നാല്‍ അല്ലാഹുവിന്റെ മാര്‍ഗത്തില്‍ നിന്ന് (ജനങ്ങളെ) തടയുന്നതും, അവനില്‍ അവിശ്വസിക്കുന്നതും, മസ്ജിദുല്‍ ഹറാമില്‍ നിന്നു (ജനങ്ങളെ) തടയുന്നതും, അതിന്റെ അവകാശികളെ അവിടെ നിന്ന് പുറത്താക്കുന്നതും അല്ലാഹുവിന്റെ അടുക്കല്‍ കൂടുതല്‍ ഗൗരവമുള്ളതാകുന്നു. കുഴപ്പം കൊലയേക്കാള്‍ ഗുരുതരമാകുന്നു. അവര്‍ക്ക് സാധിക്കുകയാണെങ്കില്‍ നിങ്ങളുടെ മതത്തില്‍ നിന്ന് നിങ്ങളെ പിന്തിരിപ്പിക്കുന്നത് വരെ അവര്‍ നിങ്ങളോട് യുദ്ധം ചെയ്തുകൊണ്ടിരിക്കും. നിങ്ങളില്‍ നിന്നാരെങ്കിലും തന്റെ മതത്തില്‍ നിന്ന് പിന്‍മാറി സത്യനിഷേധിയായിക്കൊണ്ട് മരണപ്പെടുന്ന പക്ഷം, അത്തരക്കാരുടെ കര്‍മ്മങ്ങള്‍ ഇഹത്തിലും പരത്തിലും നിഷ്ഫലമായിത്തീരുന്നതാണ്‌. അവരാകുന്നു നരകാവകാശികള്‍. അവരതില്‍ നിത്യവാസികളായിരിക്കും.
\end{malayalam}}
\flushright{\begin{Arabic}
\quranayah[2][218]
\end{Arabic}}
\flushleft{\begin{malayalam}
വിശ്വസിക്കുകയും, സ്വദേശം വെടിയുകയും, അല്ലാഹുവിന്റെ മാര്‍ഗത്തില്‍ ജിഹാദില്‍ ഏര്‍പെടുകയും ചെയ്തവരാരോ അവര്‍ അല്ലാഹുവിന്റെ കാരുണ്യം പ്രതീക്ഷിക്കുന്നവരാകുന്നു. അല്ലാഹു ഏറെ പൊറുക്കുന്നവനും കരുണാനിധിയുമത്രെ.
\end{malayalam}}
\flushright{\begin{Arabic}
\quranayah[2][219]
\end{Arabic}}
\flushleft{\begin{malayalam}
(നബിയേ,) നിന്നോടവര്‍ മദ്യത്തെയും ചൂതാട്ടത്തെയും പറ്റി ചോദിക്കുന്നു. പറയുക: അവ രണ്ടിലും ഗുരുതരമായ പാപമുണ്ട്‌. ജനങ്ങള്‍ക്ക് ചില പ്രയോജനങ്ങളുമുണ്ട്‌. എന്നാല്‍ അവയിലെ പാപത്തിന്റെ അംശമാണ് പ്രയോജനത്തിന്റെ അംശത്തേക്കാള്‍ വലുത്‌. എന്തൊന്നാണവര്‍ ചെലവ് ചെയ്യേണ്ടതെന്നും അവര്‍ നിന്നോട് ചോദിക്കുന്നു. നീ പറയുക: (അത്യാവശ്യം കഴിച്ച്‌) മിച്ചമുള്ളത്‌. അങ്ങനെ ഇഹപര ജീവിതങ്ങളെപ്പറ്റി നിങ്ങള്‍ ചിന്തിക്കുവാന്‍ വേണ്ടി അല്ലാഹു നിങ്ങള്‍ക്ക് തെളിവുകള്‍ വിവരിച്ചുതരുന്നു.
\end{malayalam}}
\flushright{\begin{Arabic}
\quranayah[2][220]
\end{Arabic}}
\flushleft{\begin{malayalam}
അനാഥകളെപ്പറ്റിയും അവര്‍ നിന്നോട് ചോദിക്കുന്നു. പറയുക: അവര്‍ക്ക് നന്‍മ വരുത്തുന്നതെന്തും നല്ലതാകുന്നു. അവരോടൊപ്പം നിങ്ങള്‍ കൂട്ടു ജീവിതം നയിക്കുകയാണെങ്കില്‍ (അതില്‍ തെറ്റില്ല.) അവര്‍ നിങ്ങളുടെ സഹോദരങ്ങളാണല്ലോ ? നാശമുണ്ടാക്കുന്നവനെയും നന്‍മവരുത്തുന്നവനെയും അല്ലാഹു വേര്‍തിരിച്ചറിയുന്നതാണ്‌. അല്ലാഹു ഉദ്ദേശിച്ചിരുന്നുവെങ്കില്‍ അവന്‍ നിങ്ങള്‍ക്ക് ക്ലേശമുണ്ടാക്കുമായിരുന്നു. തീര്‍ച്ചയായും അല്ലാഹു പ്രതാപശാലിയും യുക്തിമാനുമാകുന്നു.
\end{malayalam}}
\flushright{\begin{Arabic}
\quranayah[2][221]
\end{Arabic}}
\flushleft{\begin{malayalam}
ബഹുദൈവവിശ്വാസിനികളെ - അവര്‍ വിശ്വസിക്കുന്നത് വരെ നിങ്ങള്‍ വിവാഹം കഴിക്കരുത്‌. സത്യവിശ്വാസിനിയായ ഒരു അടിമസ്ത്രീയാണ് ബഹുദൈവവിശ്വാസിനിയെക്കാള്‍ നല്ലത്‌. അവള്‍ നിങ്ങള്‍ക്ക് കൗതുകം ജനിപ്പിച്ചാലും ശരി. ബഹുദൈവവിശ്വാസികള്‍ക്ക് അവര്‍ വിശ്വസിക്കുന്നത് വരെ നിങ്ങള്‍ വിവാഹം കഴിപ്പിച്ച് കൊടുക്കുകയും ചെയ്യരുത്‌. സത്യവിശ്വാസിയായ ഒരു അടിമയാണ് ബഹുദൈവവിശ്വാസിയെക്കാള്‍ നല്ലത്‌. അവന്‍ നിങ്ങള്‍ക്ക് കൗതുകം ജനിപ്പിച്ചാലും ശരി. അക്കൂട്ടര്‍ നരകത്തിലേക്കാണ് ക്ഷണിക്കുന്നത്‌. അല്ലാഹുവാകട്ടെ അവന്റെ ഹിതമനുസരിച്ച് സ്വര്‍ഗത്തിലേക്കും, പാപമോചനത്തിലേക്കും ക്ഷണിക്കുന്നു. ജനങ്ങള്‍ ശ്രദ്ധിച്ച് മനസ്സിലാക്കുവാന്‍ വേണ്ടി തന്റെ തെളിവുകള്‍ അവര്‍ക്ക് വിവരിച്ചുകൊടുക്കുകയും ചെയ്യുന്നു.
\end{malayalam}}
\flushright{\begin{Arabic}
\quranayah[2][222]
\end{Arabic}}
\flushleft{\begin{malayalam}
ആര്‍ത്തവത്തെപ്പറ്റി അവര്‍ നിന്നോട് ചോദിക്കുന്നു. പറയുക; അതൊരു മാലിന്യമാകുന്നു. അതിനാല്‍ ആര്‍ത്തവഘട്ടത്തില്‍ നിങ്ങള്‍ സ്ത്രീകളില്‍ നിന്ന് അകന്നു നില്‍ക്കേണ്ടതാണ്‌. അവര്‍ ശുദ്ധിയാകുന്നത് വരെ അവരെ സമീപിക്കുവാന്‍ പാടില്ല. എന്നാല്‍ അവര്‍ ശുചീകരിച്ചു കഴിഞ്ഞാല്‍ അല്ലാഹു നിങ്ങളോട് കല്‍പിച്ച വിധത്തില്‍ നിങ്ങള്‍ അവരുടെ അടുത്ത് ചെന്നുകൊള്ളുക. തീര്‍ച്ചയായും അല്ലാഹു പശ്ചാത്തപിക്കുന്നവരെ ഇഷ്ടപ്പെടുന്നു. ശുചിത്വം പാലിക്കുന്നവരെയും ഇഷ്ടപ്പെടുന്നു.
\end{malayalam}}
\flushright{\begin{Arabic}
\quranayah[2][223]
\end{Arabic}}
\flushleft{\begin{malayalam}
നിങ്ങളുടെ ഭാര്യമാര്‍ നിങ്ങളുടെ കൃഷിയിടമാകുന്നു. അതിനാല്‍ നിങ്ങള്‍ ഇച്ഛിക്കും വിധം നിങ്ങള്‍ക്ക് നിങ്ങളുടെ കൃഷിയിടത്തില്‍ ചെല്ലാവുന്നതാണ്‌. നിങ്ങളുടെ നന്‍മയ്ക്ക് വേണ്ടത് നിങ്ങള്‍ മുന്‍കൂട്ടി ചെയ്തു വെക്കേണ്ടതുമാണ്‌. നിങ്ങള്‍ അല്ലാഹുവെ സൂക്ഷിക്കുകയും അവനുമായി നിങ്ങള്‍ കണ്ടുമുട്ടേണ്ടതുണ്ടെന്ന് അറിഞ്ഞിരിക്കുകയും ചെയ്യുക. സത്യവിശ്വാസികള്‍ക്ക് നീ സന്തോഷവാര്‍ത്ത അറിയിക്കുക.
\end{malayalam}}
\flushright{\begin{Arabic}
\quranayah[2][224]
\end{Arabic}}
\flushleft{\begin{malayalam}
അല്ലാഹുവെ - അവന്റെപേരില്‍ നിങ്ങള്‍ ശപഥം ചെയ്തു പോയി എന്ന കാരണത്താല്‍ - നന്‍മ ചെയ്യുന്നതിനോ ധര്‍മ്മം പാലിക്കുന്നതിനോ ജനങ്ങള്‍ക്കിടയില്‍ രഞ്ജിപ്പുണ്ടാക്കുന്നതിനോ നിങ്ങള്‍ ഒരു തടസ്സമാക്കി വെക്കരുത്‌. അല്ലാഹു എല്ലാം കേള്‍ക്കുന്നവനും അറിയുന്നവനുമാകുന്നു.
\end{malayalam}}
\flushright{\begin{Arabic}
\quranayah[2][225]
\end{Arabic}}
\flushleft{\begin{malayalam}
(ബോധപൂര്‍വ്വമല്ലാതെ) വെറുതെ പറഞ്ഞുപോകുന്ന ശപഥവാക്കുകള്‍ മൂലം അല്ലാഹു നിങ്ങളെ പിടികൂടുന്നതല്ല. പക്ഷെ, നിങ്ങള്‍ മനസ്സറിഞ്ഞ് പ്രവര്‍ത്തിച്ചതിന്റെപേരില്‍ അല്ലാഹു നിങ്ങളെ പിടികൂടുന്നതാണ്‌. അല്ലാഹു ഏറെ പൊറുക്കുന്നവനും സഹനശീലനുമാകുന്നു.
\end{malayalam}}
\flushright{\begin{Arabic}
\quranayah[2][226]
\end{Arabic}}
\flushleft{\begin{malayalam}
തങ്ങളുടെ ഭാര്യമാരുമായി (ബന്ധപ്പെടുകയില്ലെന്ന്‌) ശപഥം ചെയ്ത് അകന്നു നില്‍ക്കുന്നവര്‍ക്ക് (അന്തിമ തീരുമാനത്തിന്‌) നാലുമാസം വരെ കാത്തിരിക്കാവുന്നതാണ്‌. അതിനിടയില്‍ അവര്‍ (ശപഥം വിട്ട് ദാമ്പത്യത്തിലേക്ക്‌) മടങ്ങുകയാണെങ്കില്‍ അല്ലാഹു ഏറെ പൊറുക്കുന്നവനും കരുണാനിധിയുമത്രെ.
\end{malayalam}}
\flushright{\begin{Arabic}
\quranayah[2][227]
\end{Arabic}}
\flushleft{\begin{malayalam}
ഇനി അവര്‍ വിവാഹമോചനം ചെയ്യാന്‍ തന്നെ തീര്‍ച്ചപ്പെടുത്തുകയാണെങ്കിലോ അല്ലാഹു എല്ലാം കേള്‍ക്കുകയും അറിയുകയും ചെയ്യുന്നവനാണല്ലോ ?
\end{malayalam}}
\flushright{\begin{Arabic}
\quranayah[2][228]
\end{Arabic}}
\flushleft{\begin{malayalam}
വിവാഹമോചനം ചെയ്യപ്പെട്ട സ്ത്രീകള്‍ തങ്ങളുടെ സ്വന്തം കാര്യത്തില്‍ മൂന്നു മാസമുറകള്‍ (കഴിയും വരെ) കാത്തിരിക്കേണ്ടതാണ്‌. അവര്‍ അല്ലാഹുവിലും അന്ത്യദിനത്തിലും വിശ്വസിക്കുന്നവരാണെങ്കില്‍ തങ്ങളുടെ ഗര്‍ഭാശയങ്ങളില്‍ അല്ലാഹു സൃഷ്ടിച്ചിട്ടുള്ളതിനെ അവര്‍ ഒളിച്ചു വെക്കാന്‍ പാടുള്ളതല്ല. അതിനകം (പ്രസ്തുത അവധിക്കകം) അവരെ തിരിച്ചെടുക്കാന്‍ അവരുടെ ഭര്‍ത്താക്കന്‍മാര്‍ ഏറ്റവും അര്‍ഹതയുള്ളവരാകുന്നു; അവര്‍ (ഭര്‍ത്താക്കന്‍മാര്‍) നിലപാട് നന്നാക്കിത്തീര്‍ക്കാന്‍ ഉദ്ദേശിച്ചിട്ടുണ്ടെങ്കില്‍. സ്ത്രീകള്‍ക്ക് (ഭര്‍ത്താക്കന്‍മാരോട്‌) ബാധ്യതകള്‍ ഉള്ളതുപോലെ തന്നെ ന്യായപ്രകാരം അവര്‍ക്ക് അവകാശങ്ങള്‍ കിട്ടേണ്ടതുമുണ്ട്‌. എന്നാല്‍ പുരുഷന്‍മാര്‍ക്ക് അവരെക്കാള്‍ ഉപരി ഒരു പദവിയുണ്ട്‌. അല്ലാഹു പ്രതാപശാലിയും യുക്തിമാനുമാകുന്നു.
\end{malayalam}}
\flushright{\begin{Arabic}
\quranayah[2][229]
\end{Arabic}}
\flushleft{\begin{malayalam}
(മടക്കിയെടുക്കാന്‍ അനുമതിയുള്ള) വിവാഹമോചനം രണ്ടു പ്രാവശ്യം മാത്രമാകുന്നു. പിന്നെ ഒന്നുകില്‍ മര്യാദയനുസരിച്ച് കൂടെ നിര്‍ത്തുകയോ, അല്ലെങ്കില്‍ നല്ല നിലയില്‍ പിരിച്ചയക്കുകയോ ആണ് വേണ്ടത്‌. നിങ്ങള്‍ അവര്‍ക്ക് (ഭാര്യമാര്‍ക്ക്‌) നല്‍കിയിട്ടുള്ളതില്‍ നിന്നു യാതൊന്നും തിരിച്ചുവാങ്ങാന്‍ നിങ്ങള്‍ക്ക് അനുവാദമില്ല. അവര്‍ ഇരുവര്‍ക്കും അല്ലാഹുവിന്റെ നിയമപരിധികള്‍ പാലിച്ചു പോരാന്‍ കഴിയില്ലെന്ന് ആശങ്ക തോന്നുന്നുവെങ്കിലല്ലാതെ. അങ്ങനെ അവര്‍ക്ക് (ദമ്പതിമാര്‍ക്ക്‌) അല്ലാഹുവിന്റെ നിയമപരിധികള്‍ പാലിക്കുവാന്‍ കഴിയില്ലെന്ന് നിങ്ങള്‍ക്ക് ഉല്‍ക്കണ്ഠ തോന്നുകയാണെങ്കില്‍ അവള്‍ വല്ലതും വിട്ടുകൊടുത്തുകൊണ്ട് സ്വയം മോചനം നേടുന്നതില്‍ അവര്‍ ഇരുവര്‍ക്കും കുറ്റമില്ല. അല്ലാഹുവിന്റെ നിയമപരിധികളത്രെ അവ. അതിനാല്‍ അവയെ നിങ്ങള്‍ ലംഘിക്കരുത്‌. അല്ലാഹുവിന്റെ നിയമപരിധികള്‍ ആര്‍ ലംഘിക്കുന്നുവോ അവര്‍ തന്നെയാകുന്നു അക്രമികള്‍.
\end{malayalam}}
\flushright{\begin{Arabic}
\quranayah[2][230]
\end{Arabic}}
\flushleft{\begin{malayalam}
ഇനിയും (മൂന്നാമതും) അവന്‍ അവളെ വിവാഹമോചനം ചെയ്യുകയാണെങ്കില്‍ അതിന് ശേഷം അവളുമായി ബന്ധപ്പെടല്‍ അവന് അനുവദനീയമാവില്ല; അവള്‍ മറ്റൊരു ഭര്‍ത്താവിനെ സ്വീകരിക്കുന്നത് വരേക്കും. എന്നിട്ട് അവന്‍ (പുതിയ ഭര്‍ത്താവ്‌) അവളെ വിവാഹമോചനം ചെയ്യുകയാണെങ്കില്‍ (പഴയ ദാമ്പത്യത്തിലേക്ക്‌) തിരിച്ചുപോകുന്നതില്‍ അവരിരുവര്‍ക്കും കുറ്റമില്ല; അല്ലാഹുവിന്റെ നിയമപരിധികള്‍ പാലിക്കാമെന്ന് അവരിരുവരും വിചാരിക്കുന്നുണ്ടെങ്കില്‍. അല്ലാഹുവിന്റെ നിയമപരിധികളത്രെ അവ. മനസ്സിലാക്കുന്ന ആളുകള്‍ക്ക് വേണ്ടി അല്ലാഹു അത് വിവരിച്ചുതരുന്നു.
\end{malayalam}}
\flushright{\begin{Arabic}
\quranayah[2][231]
\end{Arabic}}
\flushleft{\begin{malayalam}
നിങ്ങള്‍ സ്ത്രീകളെ വിവാഹമോചനം ചെയ്തിട്ട് അവരുടെ അവധി പ്രാപിച്ചാല്‍ ഒന്നുകില്‍ നിങ്ങളവരെ മര്യാദയനുസരിച്ച് കൂടെ നിര്‍ത്തുകയോ, അല്ലെങ്കില്‍ മര്യാദയനുസരിച്ച് തന്നെ പിരിച്ചയക്കുകയോ ആണ് വേണ്ടത്‌. ദ്രോഹിക്കുവാന്‍ വേണ്ടി അന്യായമായി നിങ്ങളവരെ പിടിച്ചു നിര്‍ത്തരുത്‌. അപ്രകാരം വല്ലവനും പ്രവര്‍ത്തിക്കുന്ന പക്ഷം അവന്‍ തനിക്ക് തന്നെയാണ് ദ്രോഹം വരുത്തിവെക്കുന്നത്‌. അല്ലാഹുവിന്റെ തെളിവുകളെ നിങ്ങള്‍ തമാശയാക്കിക്കളയരുത്‌. അല്ലാഹു നിങ്ങള്‍ക്ക് ചെയ്ത അനുഗ്രഹം നിങ്ങള്‍ ഓര്‍ക്കുക. നിങ്ങള്‍ക്ക് സാരോപദേശം നല്‍കിക്കൊണ്ട് അവനവതരിപ്പിച്ച വേദവും വിജ്ഞാനവും ഓര്‍മിക്കുക. അല്ലാഹുവെ നിങ്ങള്‍ സൂക്ഷിക്കുക. അല്ലാഹു എല്ലാ കാര്യവും അറിയുന്നവനാണെന്ന് മനസ്സിലാക്കുകയും ചെയ്യുക.
\end{malayalam}}
\flushright{\begin{Arabic}
\quranayah[2][232]
\end{Arabic}}
\flushleft{\begin{malayalam}
നിങ്ങള്‍ സ്ത്രീകളെ വിവാഹമോചനം ചെയ്തിട്ട് അവരുടെ അവധി പ്രാപിച്ചാല്‍ അവര്‍ തങ്ങളുടെ ഭര്‍ത്താക്കന്‍മാരുമായി വിവാഹത്തില്‍ ഏര്‍പെടുന്നതിന് നിങ്ങള്‍ തടസ്സമുണ്ടാക്കരുത്‌; മര്യാദയനുസരിച്ച് അവര്‍ അന്യോന്യം തൃപ്തിപ്പെട്ടിട്ടുണ്ടെങ്കില്‍. നിങ്ങളില്‍ നിന്ന് അല്ലാഹുവിലും അന്ത്യദിനത്തിലും വിശ്വസിക്കുന്നവര്‍ക്കുള്ള ഉപദേശമാണത്‌. അതാണ് നിങ്ങള്‍ക്ക് ഏറ്റവും ഗുണകരവും സംശുദ്ധവുമായിട്ടുള്ളത്‌. അല്ലാഹു അറിയുന്നു. നിങ്ങള്‍ അറിയുന്നില്ല.
\end{malayalam}}
\flushright{\begin{Arabic}
\quranayah[2][233]
\end{Arabic}}
\flushleft{\begin{malayalam}
മാതാക്കള്‍ തങ്ങളുടെ സന്താനങ്ങള്‍ക്ക് പൂര്‍ണ്ണമായ രണ്ടു കൊല്ലം മുലകൊടുക്കേണ്ടതാണ്‌. (കുട്ടിയുടെ) മുലകുടി പൂര്‍ണ്ണമാക്കണം എന്ന് ഉദ്ദേശിക്കുന്നവര്‍ക്കത്രെ ഇത്‌. അവര്‍ക്ക് (മുലകൊടുക്കുന്ന മാതാക്കള്‍ക്ക്‌) മര്യാദയനുസരിച്ച് ഭക്ഷണവും വസ്ത്രവും നല്‍കേണ്ടത് കുട്ടിയുടെ പിതാവിന്റെ ബാധ്യതയാകുന്നു. എന്നാല്‍ ഒരാളെയും അയാളുടെ കഴിവിലുപരി നല്‍കാന്‍ നിര്‍ബന്ധിക്കരുത്‌. ഒരു മാതാവും തന്റെ കുട്ടിയുടെ പേരില്‍ ദ്രോഹിക്കപ്പെടാന്‍ ഇടയാകരുത്‌. അതു പോലെ തന്നെ സ്വന്തം കുട്ടിയുടെ പേരില്‍ ഒരു പിതാവിന്നും ദ്രോഹം നേരിടരുത്‌. (പിതാവിന്റെ അഭാവത്തില്‍ അയാളുടെ) അവകാശികള്‍ക്കും (കുട്ടിയുടെ കാര്യത്തില്‍) അതു പോലെയുള്ള ബാധ്യതകളുണ്ട്‌. ഇനി അവര്‍ ഇരുവരും തമ്മില്‍ കൂടിയാലോചിച്ച് തൃപ്തിപ്പെട്ടുകൊണ്ട് (കുട്ടിയുടെ) മുലകുടി നിര്‍ത്താന്‍ ഉദ്ദേശിക്കുകയാണെങ്കില്‍ അവര്‍ ഇരുവര്‍ക്കും കുറ്റമില്ല; ഇനി നിങ്ങളുടെ കുട്ടികള്‍ക്ക് (മറ്റാരെക്കൊണ്ടെങ്കിലും) മുലകൊടുപ്പിക്കാനാണ് നിങ്ങള്‍ ഉദ്ദേശിക്കുന്നതെങ്കില്‍ അതിലും നിങ്ങള്‍ക്ക് കുറ്റമില്ല; (ആ പോറ്റമ്മമാര്‍ക്ക്‌) നിങ്ങള്‍ നല്‍കേണ്ടത് മര്യാദയനുസരിച്ച് കൊടുത്തു തീര്‍ക്കുകയാണെങ്കില്‍. നിങ്ങള്‍ അല്ലാഹുവെ സൂക്ഷിക്കുകയും, നിങ്ങള്‍ പ്രവര്‍ത്തിക്കുന്നതെല്ലാം അല്ലാഹു കണ്ടറിയുന്നുണ്ടെന്ന് മനസ്സിലാക്കുകയും ചെയ്യുക.
\end{malayalam}}
\flushright{\begin{Arabic}
\quranayah[2][234]
\end{Arabic}}
\flushleft{\begin{malayalam}
നിങ്ങളില്‍ ആരെങ്കിലും തങ്ങളുടെ ഭാര്യമാരെ വിട്ടേച്ചു കൊണ്ട് മരണപ്പെടുകയാണെങ്കില്‍ അവര്‍ (ഭാര്യമാര്‍) തങ്ങളുടെ കാര്യത്തില്‍ നാലുമാസവും പത്തു ദിവസവും കാത്തിരിക്കേണ്ടതാണ്‌. എന്നിട്ട് അവരുടെ ആ അവധിയെത്തിയാല്‍ തങ്ങളുടെ കാര്യത്തിലവര്‍ മര്യാദയനുസരിച്ച് പ്രവര്‍ത്തിക്കുന്നതില്‍ നിങ്ങള്‍ക്ക് കുറ്റമൊന്നുമില്ല. നിങ്ങള്‍ പ്രവര്‍ത്തിക്കുന്നതെല്ലാം അല്ലാഹു സൂക്ഷ്മമായി അറിയുന്നുണ്ട്‌.
\end{malayalam}}
\flushright{\begin{Arabic}
\quranayah[2][235]
\end{Arabic}}
\flushleft{\begin{malayalam}
(ഇദ്ദഃയുടെ ഘട്ടത്തില്‍) ആ സ്ത്രീകളുമായുള്ള വിവാഹാലോചന നിങ്ങള്‍ വ്യംഗ്യമായി സൂചിപ്പിക്കുകയോ, മനസ്സില്‍ സൂക്ഷിക്കുകയോ ചെയ്യുന്നതില്‍ നിങ്ങള്‍ക്ക് കുറ്റമില്ല. അവരെ നിങ്ങള്‍ ഓര്‍ത്തേക്കുമെന്ന് അല്ലാഹുവിന്നറിയാം. പക്ഷെ നിങ്ങള്‍ അവരോട് മര്യാദയുള്ള വല്ല വാക്കും പറയുക എന്നല്ലാതെ രഹസ്യമായി അവരോട് യാതൊരു നിശ്ചയവും ചെയ്തു പോകരുത്‌. നിയമപ്രകാരമുള്ള അവധി (ഇദ്ദഃ) പൂര്‍ത്തിയാകുന്നത് വരെ (വിവാഹമുക്തകളുമായി) വിവാഹബന്ധം സ്ഥാപിക്കാന്‍ നിങ്ങള്‍ തീരുമാനമെടുക്കരുത്‌. നിങ്ങളുടെ മനസ്സുകളിലുള്ളത് അല്ലാഹു അറിയുന്നുണ്ടെന്ന് നിങ്ങള്‍ മനസ്സിലാക്കുകയും, അവനെ നിങ്ങള്‍ ഭയപ്പെടുകയും ചെയ്യുക. അല്ലാഹു ഏറെ പൊറുക്കുന്നവനും സഹനശീലനുമാണെന്നും നിങ്ങള്‍ മനസ്സിലാക്കുക.
\end{malayalam}}
\flushright{\begin{Arabic}
\quranayah[2][236]
\end{Arabic}}
\flushleft{\begin{malayalam}
നിങ്ങള്‍ ഭാര്യമാരെ സ്പര്‍ശിക്കുകയോ, അവരുടെ വിവാഹമൂല്യം നിശ്ചയിക്കുകയോ ചെയ്യുന്നതിനു മുമ്പായി നിങ്ങളവരുമായുള്ള ബന്ധം വേര്‍പെടുത്തിയാല്‍ (മഹ്‌റ് നല്‍കാത്തതിന്റെ പേരില്‍) നിങ്ങള്‍ക്ക് കുറ്റമില്ല. എന്നാല്‍ അവര്‍ക്ക് നിങ്ങള്‍ മര്യാദയനുസരിച്ച് ജീവിതവിഭവമായി എന്തെങ്കിലും നല്‍കേണ്ടതാണ്‌. കഴിവുള്ളവന്‍ തന്റെ കഴിവനുസരിച്ചും, ഞെരുക്കമുള്ളവന്‍ തന്റെ സ്ഥിതിക്കനുസരിച്ചും. സദ്‌വൃത്തരായ ആളുകള്‍ക്ക് ഇതൊരു ബാധ്യതയത്രെ.
\end{malayalam}}
\flushright{\begin{Arabic}
\quranayah[2][237]
\end{Arabic}}
\flushleft{\begin{malayalam}
ഇനി നിങ്ങള്‍ അവരെ സ്പര്‍ശിക്കുന്നതിനു മുമ്പ് തന്നെ വിവാഹബന്ധം വേര്‍പെടുത്തുകയും, അവരുടെ വിവാഹമൂല്യം നിങ്ങള്‍ നിശ്ചയിച്ച് കഴിഞ്ഞിരിക്കുകയും ആണെങ്കില്‍ നിങ്ങള്‍ നിശ്ചയിച്ചതിന്റെ പകുതി (നിങ്ങള്‍ നല്‍കേണ്ടതാണ്‌.) അവര്‍ (ഭാര്യമാര്‍) വിട്ടുവീഴ്ച ചെയ്യുന്നുവെങ്കിലല്ലാതെ. അല്ലെങ്കില്‍ വിവാഹക്കരാര്‍ കൈവശം വെച്ചിരിക്കുന്നവന്‍ (ഭര്‍ത്താവ്‌) (മഹ്ര് പൂര്‍ണ്ണമായി നല്‍കിക്കൊണ്ട്‌) വിട്ടുവീഴ്ച ചെയ്യുന്നുവെങ്കിലല്ലാതെ. എന്നാല്‍ (ഭര്‍ത്താക്കന്‍മാരേ,) നിങ്ങള്‍ വിട്ടുവീഴ്ച ചെയ്യുന്നതാണ് ധര്‍മ്മനിഷ്ഠയ്ക്ക് കൂടുതല്‍ യോജിച്ചത്‌. നിങ്ങള്‍ അന്യോന്യം ഔദാര്യം കാണിക്കാന്‍ മറക്കരുത്‌. തീര്‍ച്ചയായും നിങ്ങള്‍ പ്രവര്‍ത്തിക്കുന്നതെല്ലാം അല്ലാഹു കണ്ടറിയുന്നവനാകുന്നു.
\end{malayalam}}
\flushright{\begin{Arabic}
\quranayah[2][238]
\end{Arabic}}
\flushleft{\begin{malayalam}
പ്രാര്‍ത്ഥനകള്‍ (അഥവാ നമസ്കാരങ്ങള്‍) നിങ്ങള്‍ സൂക്ഷ്മതയോടെ നിര്‍വഹിച്ചു പോരേണ്ടതാണ്‌. പ്രത്യേകിച്ചും ഉല്‍കൃഷ്ടമായ നമസ്കാരം. അല്ലാഹുവിന്റെ മുമ്പില്‍ ഭയഭക്തിയോടു കൂടി നിന്നുകൊണ്ടാകണം നിങ്ങള്‍ പ്രാര്‍ത്ഥിക്കുന്നത്‌.
\end{malayalam}}
\flushright{\begin{Arabic}
\quranayah[2][239]
\end{Arabic}}
\flushleft{\begin{malayalam}
നിങ്ങള്‍ (ശത്രുവിന്റെ ആക്രമണം) ഭയപ്പെടുകയാണെങ്കില്‍ കാല്‍നടയായോ വാഹനങ്ങളിലായോ (നിങ്ങള്‍ക്ക് നമസ്കരിക്കാം.) എന്നാല്‍ നിങ്ങള്‍ സുരക്ഷിതാവസ്ഥയിലായാല്‍ നിങ്ങള്‍ക്ക് അറിവില്ലാതിരുന്നത് അല്ലാഹു പഠിപ്പിച്ചുതന്ന പ്രകാരം നിങ്ങള്‍ അവനെ സ്മരിക്കേണ്ടതാണ്‌.
\end{malayalam}}
\flushright{\begin{Arabic}
\quranayah[2][240]
\end{Arabic}}
\flushleft{\begin{malayalam}
നിങ്ങളില്‍ നിന്ന് ഭാര്യമാരെ വിട്ടേച്ചു കൊണ്ട് മരണപ്പെടുന്നവര്‍ തങ്ങളുടെ ഭാര്യമാര്‍ക്ക് ഒരു കൊല്ലത്തേക്ക് (വീട്ടില്‍ നിന്ന്‌) പുറത്താക്കാതെ ജീവിതവിഭവം നല്‍കാന്‍ വസ്വിയ്യത്ത് ചെയ്യേണ്ടതാണ്‌. എന്നാല്‍ അവര്‍ (സ്വയം) പുറത്ത് പോകുന്ന പക്ഷം തങ്ങളുടെ സ്വന്തം കാര്യത്തില്‍ മര്യാദയനുസരിച്ച് അവര്‍ പ്രവര്‍ത്തിക്കുന്നതില്‍ നിങ്ങള്‍ക്ക് കുറ്റമില്ല. അല്ലാഹു പ്രതാപവാനും സൂക്ഷ്മജ്ഞാനമുള്ളവനുമാകുന്നു.
\end{malayalam}}
\flushright{\begin{Arabic}
\quranayah[2][241]
\end{Arabic}}
\flushleft{\begin{malayalam}
വിവാഹമോചിതരായ സ്ത്രീകള്‍ക്ക് ന്യായപ്രകാരം എന്തെങ്കിലും ജീവിതവിഭവമായി നല്‍കേണ്ടതാണ്‌. ഭയഭക്തിയുള്ളവര്‍ക്ക് അതൊരു ബാധ്യതയത്രെ.
\end{malayalam}}
\flushright{\begin{Arabic}
\quranayah[2][242]
\end{Arabic}}
\flushleft{\begin{malayalam}
നിങ്ങള്‍ ഗ്രഹിക്കേണ്ടതിനു വേണ്ടി അപ്രകാരം അല്ലാഹു അവന്റെ തെളിവുകള്‍ വിവരിച്ചുതരുന്നു.
\end{malayalam}}
\flushright{\begin{Arabic}
\quranayah[2][243]
\end{Arabic}}
\flushleft{\begin{malayalam}
ആയിരക്കണക്കിന് ആളുകളുണ്ടായിട്ടും മരണഭയം കൊണ്ട് സ്വന്തം വീട് വിട്ട് ഇറങ്ങിപ്പോയ ഒരു ജനതയെപ്പറ്റി നീ അറിഞ്ഞില്ലേ? അപ്പോള്‍ അല്ലാഹു അവരോട് പറഞ്ഞു: നിങ്ങള്‍ മരിച്ചു കൊള്ളുക. പിന്നീട് അല്ലാഹു അവര്‍ക്ക് ജീവന്‍ നല്‍കി. തീര്‍ച്ചയായും അല്ലാഹു മനുഷ്യരോട് ഔദാര്യം കാണിക്കുന്നവനാകുന്നു. പക്ഷെ മനുഷ്യരില്‍ അധികപേരും നന്ദികാണിക്കുന്നില്ല.
\end{malayalam}}
\flushright{\begin{Arabic}
\quranayah[2][244]
\end{Arabic}}
\flushleft{\begin{malayalam}
അല്ലാഹുവിന്റെ മാര്‍ഗത്തില്‍ നിങ്ങള്‍ യുദ്ധം ചെയ്യുക. അല്ലാഹു (എല്ലാം) കേള്‍ക്കുന്നവനും അറിയുന്നവനുമാണെന്ന് മനസ്സിലാക്കുകയും ചെയ്യുക.
\end{malayalam}}
\flushright{\begin{Arabic}
\quranayah[2][245]
\end{Arabic}}
\flushleft{\begin{malayalam}
അല്ലാഹുവിന് ഉത്തമമായ കടം നല്‍കുവാനാരുണ്ട്‌? എങ്കില്‍ അല്ലാഹു അതവന്ന് അനേകം ഇരട്ടികളായി വര്‍ദ്ധിപ്പിച്ച് കൊടുക്കുന്നതാണ്‌. (ധനം) പിടിച്ചു വെക്കുന്നതും വിട്ടുകൊടുക്കുന്നതും അല്ലാഹുവാകുന്നു. അവങ്കലേക്ക് തന്നെയാകുന്നു നിങ്ങള്‍ മടക്കപ്പെടുന്നതും.
\end{malayalam}}
\flushright{\begin{Arabic}
\quranayah[2][246]
\end{Arabic}}
\flushleft{\begin{malayalam}
മൂസായുടെ ശേഷം ഉണ്ടായിരുന്ന ചില ഇസ്രായീലീ പ്രമുഖര്‍ തങ്ങളുടെ പ്രവാചകനോട്‌, ഞങ്ങള്‍ക്കൊരു രാജാവിനെ നിയോഗിച്ച് തരൂ. (അദ്ദേഹത്തിന്റെ നേതൃത്വത്തില്‍) ഞങ്ങള്‍ അല്ലാഹുവിന്റെ മാര്‍ഗത്തില്‍ യുദ്ധം ചെയ്തുകൊള്ളാം എന്ന് പറഞ്ഞ സന്ദര്‍ഭം നീ അറിഞ്ഞില്ലേ? അദ്ദേഹം (പ്രവാചകന്‍) ചോദിച്ചു: നിങ്ങള്‍ക്ക് യുദ്ധത്തിന്ന് കല്‍പന കിട്ടിയാല്‍ നിങ്ങള്‍ യുദ്ധം ചെയ്യാതിരുന്നേക്കുമോ ? അവര്‍ പറഞ്ഞു: ഞങ്ങളുടെ താമസസ്ഥലങ്ങളില്‍ നിന്നും സന്തതികള്‍ക്കിടയില്‍ നിന്നും ഞങ്ങള്‍ പുറം തള്ളപ്പെട്ട സ്ഥിതിക്ക് ഞങ്ങള്‍ക്കെങ്ങനെ അല്ലാഹുവിന്റെ മാര്‍ഗത്തില്‍ യുദ്ധം ചെയ്യാതിരിക്കാന്‍ കഴിയും ? എന്നാല്‍ അവര്‍ക്ക് യുദ്ധത്തിന് കല്‍പന നല്‍കപ്പെട്ടപ്പോഴാകട്ടെ അല്‍പം പേരൊഴിച്ച് (എല്ലാവരും) പിന്‍മാറുകയാണുണ്ടായത്‌. അല്ലാഹു അക്രമകാരികളെപ്പറ്റി (നല്ലവണ്ണം) അറിയുന്നവനാകുന്നു.
\end{malayalam}}
\flushright{\begin{Arabic}
\quranayah[2][247]
\end{Arabic}}
\flushleft{\begin{malayalam}
അവരോട് അവരുടെ പ്രവാചകന്‍ പറഞ്ഞു: അല്ലാഹു നിങ്ങള്‍ക്ക് ത്വാലൂതിനെ രാജാവായി നിയോഗിച്ചു തന്നിരിക്കുന്നു. അവര്‍ പറഞ്ഞു: അയാള്‍ക്കെങ്ങനെ ഞങ്ങളുടെ രാജാവാകാന്‍ പറ്റും? രാജാധികാരത്തിന് അയാളെക്കാള്‍ കൂടുതല്‍ അര്‍ഹതയുള്ളത് ഞങ്ങള്‍ക്കാണല്ലോ. അയാള്‍ സാമ്പത്തിക സമൃദ്ധി ലഭിച്ച ആളുമല്ലല്ലോ. അദ്ദേഹം (പ്രവാചകന്‍) പറഞ്ഞു: അല്ലാഹു അദ്ദേഹത്തെ നിങ്ങളെക്കാള്‍ ഉല്‍കൃഷ്ടനായി തെരഞ്ഞെടുത്തിരിക്കുന്നു. കൂടുതല്‍ വിപുലമായ ജ്ഞാനവും ശരീര ശക്തിയും നല്‍കുകയും ചെയ്തിരിക്കുന്നു. അല്ലാഹു അവന്റെവകയായുള്ള ആധിപത്യം അവന്‍ ഉദ്ദേശിക്കുന്നവര്‍ക്ക് കൊടുക്കുന്നു. അല്ലാഹു വിപുലമായ കഴിവുള്ളവനും എല്ലാം അറിയുന്നവനുമാകുന്നു.
\end{malayalam}}
\flushright{\begin{Arabic}
\quranayah[2][248]
\end{Arabic}}
\flushleft{\begin{malayalam}
അവരോട് അവരുടെ പ്രവാചകന്‍ പറഞ്ഞു: ത്വാലൂതിന്റെ രാജാധികാരത്തിനുള്ള തെളിവ് ആ പെട്ടി നിങ്ങളുടെ അടുത്ത് വന്നെത്തുക എന്നതാണ്‌. അതില്‍ നിങ്ങളുടെ രക്ഷിതാവിങ്കല്‍ നിന്നുള്ള മനഃശാന്തിയും മൂസായുടെയും ഹാറൂന്‍റെയും കുടുംബങ്ങള്‍ വിട്ടേച്ചുപോയ അവശിഷ്ടങ്ങളുമുണ്ട്‌. മലക്കുകള്‍ അത് വഹിച്ച് കൊണ്ടുവരുന്നതാണ്‌. നിങ്ങള്‍ വിശ്വാസികളാണെങ്കില്‍ നിസ്സംശയം നിങ്ങള്‍ക്കതില്‍ മഹത്തായ ദൃഷ്ടാന്തമുണ്ട്‌.
\end{malayalam}}
\flushright{\begin{Arabic}
\quranayah[2][249]
\end{Arabic}}
\flushleft{\begin{malayalam}
അങ്ങനെ സൈന്യവുമായി പുറപ്പെട്ടപ്പോള്‍ ത്വാലൂത് പറഞ്ഞു: അല്ലാഹു ഒരു നദി മുഖേന നിങ്ങളെ പരീക്ഷിക്കുന്നതാണ്‌. അപ്പോള്‍ ആര്‍ അതില്‍ നിന്ന് കുടിച്ചുവോ അവന്‍ എന്റെകൂട്ടത്തില്‍ പെട്ടവനല്ല. ആരതു രുചിച്ച് നോക്കാതിരുന്നുവോ അവന്‍ എന്റെകൂട്ടത്തില്‍ പെട്ടവനാകുന്നു. എന്നാല്‍ തന്റെകൈകൊണ്ട് ഒരിക്കല്‍ മാത്രം കോരിയവന്‍ ഇതില്‍ നിന്ന് ഒഴിവാണ്‌. അവരില്‍ നിന്ന് ചുരുക്കം പേരൊഴികെ അതില്‍ നിന്ന് കുടിച്ചു. അങ്ങനെ അദ്ദേഹവും കൂടെയുള്ള വിശ്വാസികളും ആ നദി കടന്നു കഴിഞ്ഞപ്പോള്‍ അവര്‍ പറഞ്ഞു: ജാലൂതി (ഗോലിയത്ത്‌) നെയും അവന്റെ സൈന്യങ്ങളെയും നേരിടാന്‍ മാത്രമുള്ള കഴിവ് ഇന്ന് നമുക്കില്ല. തങ്ങള്‍ അല്ലാഹുവുമായി കണ്ടുമുട്ടേണ്ടവരാണ് എന്ന വിചാരമുള്ളവര്‍ പറഞ്ഞു: എത്രയെത്ര ചെറിയ സംഘങ്ങളാണ് അല്ലാഹുവിന്റെഅനുമതിയോടെ വലിയ സംഘങ്ങളെ കീഴ്പെടുത്തിയിട്ടുള്ളത്‌! അല്ലാഹു ക്ഷമിക്കുന്നവരുടെ കൂടെയാകുന്നു.
\end{malayalam}}
\flushright{\begin{Arabic}
\quranayah[2][250]
\end{Arabic}}
\flushleft{\begin{malayalam}
അങ്ങനെ അവര്‍ ജാലൂതിനും സൈന്യങ്ങള്‍ക്കുമെതിരെ പോരിനിറങ്ങിയപ്പോള്‍ അവര്‍ പ്രാര്‍ത്ഥിച്ചു: ഞങ്ങളുടെ നാഥാ! ഞങ്ങളുടെ മേല്‍ നീ ക്ഷമ ചൊരിഞ്ഞുതരികയും ഞങ്ങളുടെ പാദങ്ങളെ നീ ഉറപ്പിച്ചു നിര്‍ത്തുകയും, സത്യനിഷേധികളായ ജനങ്ങള്‍ക്കെതിരില്‍ ഞങ്ങളെ നീ സഹായിക്കുകയും ചെയ്യേണമേ.
\end{malayalam}}
\flushright{\begin{Arabic}
\quranayah[2][251]
\end{Arabic}}
\flushleft{\begin{malayalam}
അങ്ങനെ അല്ലാഹുവിന്റെഅനുമതി പ്രകാരം അവരെ (ശത്രുക്കളെ) അവര്‍ പരാജയപ്പെടുത്തി. ദാവൂദ് ജാലൂതിനെ കൊലപ്പെടുത്തി. അദ്ദേഹത്തിന് അല്ലാഹു ആധിപത്യവും ജ്ഞാനവും നല്‍കുകയും, താന്‍ ഉദ്ദേശിക്കുന്ന പലതും പഠിപ്പിക്കുകയും ചെയ്തു. മനുഷ്യരില്‍ ചിലരെ മറ്റു ചിലര്‍ മുഖേന അല്ലാഹു തടുക്കുന്നില്ലായിരുന്നുവെങ്കില്‍ ഭൂലോകം കുഴപ്പത്തിലാകുമായിരുന്നു. പക്ഷെ അല്ലാഹു ലോകരോട് വളരെ ഉദാരനത്രെ.
\end{malayalam}}
\flushright{\begin{Arabic}
\quranayah[2][252]
\end{Arabic}}
\flushleft{\begin{malayalam}
അല്ലാഹുവിന്റെ ദൃഷ്ടാന്തങ്ങളാകുന്നു അവയൊക്കെ. സത്യപ്രകാരം നിനക്ക് നാം അവ ഓതികേള്‍പിച്ച് തരുന്നു. തീര്‍ച്ചയായും നീ (നമ്മുടെ ദൌത്യവുമായി) നിയോഗിക്കപ്പെട്ടവരില്‍ ഒരാളാകുന്നു.
\end{malayalam}}
\flushright{\begin{Arabic}
\quranayah[2][253]
\end{Arabic}}
\flushleft{\begin{malayalam}
ആ ദൂതന്‍മാരില്‍ ചിലര്‍ക്ക് നാം മറ്റു ചിലരെക്കാള്‍ ശ്രേഷ്ഠത നല്‍കിയിരിക്കുന്നു. അല്ലാഹു (നേരില്‍) സംസാരിച്ചിട്ടുള്ളവര്‍ അവരിലുണ്ട്‌. അവരില്‍ ചിലരെ അവന്‍ പല പദവികളിലേക്ക് ഉയര്‍ത്തിയിട്ടുമുണ്ട്‌. മര്‍യമിന്റെമകന്‍ ഈസായ്ക്ക് നാം വ്യക്തമായ ദൃഷ്ടാന്തങ്ങള്‍ നല്‍കുകയും, പരിശുദ്ധാത്മാവ് മുഖേന അദ്ദേഹത്തിന് നാം പിന്‍ബലം നല്‍കുകയും ചെയ്തിട്ടുണ്ട്‌. അല്ലാഹു ഉദ്ദേശിച്ചിരുന്നുവെങ്കില്‍ അവരുടെ (ദൂതന്‍മാരുടെ) പിന്‍ഗാമികള്‍ വ്യക്തമായ തെളിവ് വന്നുകിട്ടിയതിനു ശേഷവും (അന്യോന്യം) പോരടിക്കുമായിരുന്നില്ല. എന്നാല്‍ അവര്‍ ഭിന്നിച്ചു. അങ്ങനെ അവരില്‍ വിശ്വസിച്ചവരും നിഷേധിച്ചവരുമുണ്ടായി. അല്ലാഹു ഉദ്ദേശിച്ചിരുന്നുവെങ്കില്‍ അവര്‍ പോരടിക്കുമായിരുന്നില്ല. പക്ഷെ അല്ലാഹു താന്‍ ഉദ്ദേശിക്കുന്നത് ചെയ്യുന്നു.
\end{malayalam}}
\flushright{\begin{Arabic}
\quranayah[2][254]
\end{Arabic}}
\flushleft{\begin{malayalam}
സത്യവിശ്വാസികളേ, ക്രയവിക്രയമോ സ്നേഹബന്ധമോ ശുപാര്‍ശയോ നടക്കാത്ത ഒരു ദിവസം വന്നെത്തുന്നതിനു മുമ്പായി, നിങ്ങള്‍ക്ക് നാം നല്‍കിയിട്ടുള്ളതില്‍ നിന്ന് നിങ്ങള്‍ ചെലവഴിക്കുവിന്‍. സത്യനിഷേധികള്‍ തന്നെയാകുന്നു അക്രമികള്‍.
\end{malayalam}}
\flushright{\begin{Arabic}
\quranayah[2][255]
\end{Arabic}}
\flushleft{\begin{malayalam}
അല്ലാഹു - അവനല്ലാതെ ദൈവമില്ല. എന്നെന്നും ജീവിച്ചിരിക്കുന്നവന്‍. എല്ലാം നിയന്ത്രിക്കുന്നവന്‍. മയക്കമോ ഉറക്കമോ അവനെ ബാധിക്കുകയില്ല. അവന്റേതാണ് ആകാശഭൂമികളിലുള്ളതെല്ലാം. അവന്റെ അനുവാദപ്രകാരമല്ലാതെ അവന്‍റെയടുക്കല്‍ ശുപാര്‍ശ നടത്താനാരുണ്ട് ? അവരുടെ മുമ്പിലുള്ളതും അവര്‍ക്ക് പിന്നിലുള്ളതും അവന്‍ അറിയുന്നു. അവന്റെ അറിവില്‍ നിന്ന് അവന്‍ ഇച്ഛിക്കുന്നതല്ലാതെ (മറ്റൊന്നും) അവര്‍ക്ക് സൂക്ഷ്മമായി അറിയാന്‍ കഴിയില്ല. അവന്റെ അധികാരപീഠം ആകാശഭൂമികളെ മുഴുവന്‍ ഉള്‍കൊള്ളുന്നതാകുന്നു. അവയുടെ സംരക്ഷണം അവന്ന് ഒട്ടും ഭാരമുള്ളതല്ല. അവന്‍ ഉന്നതനും മഹാനുമത്രെ.
\end{malayalam}}
\flushright{\begin{Arabic}
\quranayah[2][256]
\end{Arabic}}
\flushleft{\begin{malayalam}
മതത്തിന്റെ കാര്യത്തില്‍ ബലപ്രയോഗമേ ഇല്ല. സന്‍മാര്‍ഗം ദുര്‍മാര്‍ഗത്തില്‍ നിന്ന് വ്യക്തമായി വേര്‍തിരിഞ്ഞ് കഴിഞ്ഞിരിക്കുന്നു. ആകയാല്‍ ഏതൊരാള്‍ ദുര്‍മൂര്‍ത്തികളെ അവിശ്വസിക്കുകയും അല്ലാഹുവില്‍ വിശ്വസിക്കുകയും ചെയ്യുന്നുവോ അവന്‍ പിടിച്ചിട്ടുള്ളത് ബലമുള്ള ഒരു കയറിലാകുന്നു. അത് പൊട്ടി പോകുകയേ ഇല്ല. അല്ലാഹു (എല്ലാം) കേള്‍ക്കുന്നവനും അറിയുന്നവനുമാകുന്നു.
\end{malayalam}}
\flushright{\begin{Arabic}
\quranayah[2][257]
\end{Arabic}}
\flushleft{\begin{malayalam}
വിശ്വസിച്ചവരുടെ രക്ഷാധികാരിയാകുന്നു അല്ലാഹു. അവന്‍ അവരെ ഇരുട്ടുകളില്‍ നിന്ന് വെളിച്ചത്തിലേക്ക് കൊണ്ടു വരുന്നു. എന്നാല്‍ സത്യനിഷേധികളുടെ രക്ഷാധികാരികള്‍ ദുര്‍മൂര്‍ത്തികളാകുന്നു. വെളിച്ചത്തില്‍ നിന്ന് ഇരുട്ടുകളിലേക്കാണ് ആ ദുര്‍മൂര്‍ത്തികള്‍ അവരെ നയിക്കുന്നത്‌. അവരത്രെ നരകാവകാശികള്‍. അവരതില്‍ നിത്യവാസികളാകുന്നു.
\end{malayalam}}
\flushright{\begin{Arabic}
\quranayah[2][258]
\end{Arabic}}
\flushleft{\begin{malayalam}
ഇബ്രാഹീമിനോട് അദ്ദേഹത്തിന്റെ നാഥന്റെ കാര്യത്തില്‍ തര്‍ക്കിച്ചവനെപ്പറ്റി നീയറിഞ്ഞില്ലേ ? അല്ലാഹു അവന്ന് ആധിപത്യം നല്‍കിയതിനാലാണ് (അവനതിന് മുതിര്‍ന്നത്‌.) എന്റെ നാഥന്‍ ജീവിപ്പിക്കുകയും മരിപ്പിക്കുകയും ചെയ്യുന്നവനാകുന്നു എന്ന് ഇബ്രാഹീം പറഞ്ഞപ്പോള്‍ ഞാനും ജീവിപ്പിക്കുകയും മരിപ്പിക്കുകയും ചെയ്യുന്നുവല്ലോ എന്നാണവന്‍ പറഞ്ഞത്‌. ഇബ്രാഹീം പറഞ്ഞു: എന്നാല്‍ അല്ലാഹു സൂര്യനെ കിഴക്കു നിന്ന് കൊണ്ടു വരുന്നു. നീയതിനെ പടിഞ്ഞാറ് നിന്ന് കൊണ്ടു വരിക. അപ്പോള്‍ ആ സത്യനിഷേധിക്ക് ഉത്തരം മുട്ടിപ്പോയി. അക്രമികളായ ജനതയെ അല്ലാഹു നേര്‍വഴിയിലാക്കുകയില്ല.
\end{malayalam}}
\flushright{\begin{Arabic}
\quranayah[2][259]
\end{Arabic}}
\flushleft{\begin{malayalam}
അല്ലെങ്കിലിതാ, മറ്റൊരാളുടെ ഉദാഹരണം. മേല്‍ക്കൂരകളോടെ വീണടിഞ്ഞ് കിടക്കുകയായിരുന്ന ഒരു പട്ടണത്തിലൂടെ അദ്ദേഹം സഞ്ചരിക്കുകയായിരുന്നു. (അപ്പോള്‍) അദ്ദേഹം പറഞ്ഞു: നിര്‍ജീവമായിപ്പോയതിനു ശേഷം ഇതിനെ എങ്ങനെയായിരിക്കും അല്ലാഹു ജീവിപ്പിക്കുന്നത്‌. തുടര്‍ന്ന് അല്ലാഹു അദ്ദേഹത്തെ നൂറു വര്‍ഷം നിര്‍ജീവാവസ്ഥയിലാക്കുകയും പിന്നീട് അദ്ദേഹത്തെ ഉയിര്‍ത്തെഴുന്നേല്‍പിക്കുകയും ചെയ്തു. അനന്തരം അല്ലാഹു ചോദിച്ചു: നീ എത്രകാലം (നിര്‍ജീവാവസ്ഥയില്‍) കഴിച്ചുകൂട്ടി? ഒരു ദിവസമോ, ഒരു ദിവസത്തിന്റെ അല്‍പഭാഗമോ (ആണ് ഞാന്‍ കഴിച്ചുകൂട്ടിയത്‌); അദ്ദേഹം മറുപടി പറഞ്ഞു. അല്ല, നീ നൂറു വര്‍ഷം കഴിച്ചുകൂട്ടിയിരിക്കുന്നു. നിന്റെ ആഹാരപാനീയങ്ങള്‍ നോക്കൂ അവയ്ക്ക് മാറ്റം വന്നിട്ടില്ല. നിന്റെ കഴുതയുടെ നേര്‍ക്ക് നോക്കൂ. (അതെങ്ങനെയുണ്ടെന്ന്‌). നിന്നെ മനുഷ്യര്‍ക്കൊരു ദൃഷ്ടാന്തമാക്കുവാന്‍ വേണ്ടിയാകുന്നു നാമിത് ചെയ്തത്‌. ആ എല്ലുകള്‍ നാം എങ്ങനെ കൂട്ടിയിണക്കുകയും എന്നിട്ടവയെ മാംസത്തില്‍ പൊതിയുകയും ചെയ്യുന്നു വെന്നും നീ നോക്കുക എന്ന് അവന്‍ (അല്ലാഹു) പറഞ്ഞു. അങ്ങനെ അദ്ദേഹത്തിന് (കാര്യം) വ്യക്തമായപ്പോള്‍ അദ്ദേഹം പറഞ്ഞു: തീര്‍ച്ചയായും അല്ലാഹു എല്ലാകാര്യങ്ങള്‍ക്കും കഴിവുള്ളവനാണ് എന്ന് ഞാന്‍ മനസ്സിലാക്കുന്നു.
\end{malayalam}}
\flushright{\begin{Arabic}
\quranayah[2][260]
\end{Arabic}}
\flushleft{\begin{malayalam}
എന്റെനാഥാ! മരണപ്പെട്ടവരെ നീ എങ്ങനെ ജീവിപ്പിക്കുന്നു വെന്ന് എനിക്ക് നീ കാണിച്ചുതരേണമേ എന്ന് ഇബ്രാഹീം പറഞ്ഞ സന്ദര്‍ഭവും (ശ്രദ്ധേയമാകുന്നു.) അല്ലാഹു ചോദിച്ചു: നീ വിശ്വസിച്ചിട്ടില്ലേ? ഇബ്രാഹീം പറഞ്ഞു: അതെ. പക്ഷെ, എന്റെ മനസ്സിന് സമാധാനം ലഭിക്കാന്‍ വേണ്ടിയാകുന്നു . അല്ലാഹു പറഞ്ഞു: എന്നാല്‍ നീ നാലു പക്ഷികളെ പിടിക്കുകയും അവയെ നിന്നിലേക്ക് അടുപ്പിക്കുകയും (അവയെ കഷ്ണിച്ചിട്ട്‌) അവയുടെ ഓരോ അംശം ഓരോ മലയിലും വെക്കുകയും ചെയ്യുക. എന്നിട്ടവയെ നീ വിളിക്കുക. അവ നിന്റെ അടുക്കല്‍ ഓടിവരുന്നതാണ്‌. അല്ലാഹു പ്രതാപവാനും യുക്തിമാനുമാണ് എന്ന് നീ മനസ്സിലാക്കുകയും ചെയ്യുക.
\end{malayalam}}
\flushright{\begin{Arabic}
\quranayah[2][261]
\end{Arabic}}
\flushleft{\begin{malayalam}
അല്ലാഹുവിന്റെ മാര്‍ഗത്തില്‍ തങ്ങളുടെ ധനം ചെലവഴിക്കുന്നവരെ ഉപമിക്കാവുന്നത് ഒരു ധാന്യമണിയോടാകുന്നു. അത് ഏഴ് കതിരുകള്‍ ഉല്‍പാദിപ്പിച്ചു. ഓരോ കതിരിലും നൂറ് ധാന്യമണിയും. അല്ലാഹു താന്‍ ഉദ്ദേശിക്കുന്നവര്‍ക്ക് ഇരട്ടിയായി നല്‍കുന്നു. അല്ലാഹു വിപുലമായ കഴിവുകളുള്ളവനും (എല്ലാം) അറിയുന്നവനുമാണ്‌.
\end{malayalam}}
\flushright{\begin{Arabic}
\quranayah[2][262]
\end{Arabic}}
\flushleft{\begin{malayalam}
അല്ലാഹുവിന്റെ മാര്‍ഗത്തില്‍ തങ്ങളുടെ ധനം ചെലവഴിക്കുകയും എന്നിട്ടതിനെ തുടര്‍ന്ന്‌, ചെലവ് ചെയ്തത് എടുത്തുപറയുകയോ ശല്യപ്പെടുത്തുകയോ ചെയ്യാതിരിക്കുകയും ചെയ്യുന്നവര്‍ ആരോ അവര്‍ക്ക് തങ്ങളുടെ രക്ഷിതാവിങ്കല്‍ അവര്‍ അര്‍ഹിക്കുന്ന പ്രതിഫലമുണ്ടായിരിക്കും. അവര്‍ക്ക് യാതൊന്നും ഭയപ്പെടേണ്ടതില്ല. അവര്‍ ദുഃഖിക്കേണ്ടി വരികയുമില്ല.
\end{malayalam}}
\flushright{\begin{Arabic}
\quranayah[2][263]
\end{Arabic}}
\flushleft{\begin{malayalam}
കൊടുത്തതിനെത്തുടര്‍ന്ന് മനഃക്ലേശം വരുത്തുന്ന ദാനധര്‍മ്മത്തെക്കാള്‍ ഉത്തമമായിട്ടുള്ളത് നല്ല വാക്കും വിട്ടുവീഴ്ചയുമാകുന്നു. അല്ലാഹു പരാശ്രയം ആവശ്യമില്ലാത്തവനും സഹനശീലനുമാകുന്നു.
\end{malayalam}}
\flushright{\begin{Arabic}
\quranayah[2][264]
\end{Arabic}}
\flushleft{\begin{malayalam}
സത്യവിശ്വാസികളേ, (കൊടുത്തത്‌) എടുത്തുപറഞ്ഞ് കൊണ്ടും, ശല്യമുണ്ടാക്കിക്കൊണ്ടും നിങ്ങള്‍ നിങ്ങളുടെ ദാനധര്‍മ്മങ്ങളെ നിഷ്ഫലമാക്കിക്കളയരുത്‌. അല്ലാഹുവിലും പരലോകത്തിലും വിശ്വാസമില്ലാതെ, ജനങ്ങളെ കാണിക്കുവാന്‍ വേണ്ടി ധനം ചെലവ് ചെയ്യുന്നവനെപ്പോലെ നിങ്ങളാകരുത്‌. അവനെ ഉപമിക്കാവുന്നത് മുകളില്‍ അല്‍പം മണ്ണ് മാത്രമുള്ള മിനുസമുള്ള ഒരു പാറയോടാകുന്നു. ആ പാറ മേല്‍ ഒരു കനത്ത മഴ പതിച്ചു. ആ മഴ അതിനെ ഒരു മൊട്ടപ്പാറയാക്കി മാറ്റിക്കളഞ്ഞു. അവര്‍ അദ്ധ്വാനിച്ചതിന്റെ യാതൊരു ഫലവും കരസ്ഥമാക്കാന്‍ അവര്‍ക്ക് കഴിയില്ല. അല്ലാഹു സത്യനിഷേധികളായ ജനതയെ നേര്‍വഴിയിലാക്കുകയില്ല.
\end{malayalam}}
\flushright{\begin{Arabic}
\quranayah[2][265]
\end{Arabic}}
\flushleft{\begin{malayalam}
അല്ലാഹുവിന്റെ പ്രീതി തേടിക്കൊണ്ടും, തങ്ങളുടെ മനസ്സുകളില്‍ (സത്യവിശ്വാസം) ഉറപ്പിച്ചു കൊണ്ടും ധനം ചെലവഴിക്കുന്നവരെ ഉപമിക്കാവുന്നത് ഒരു ഉയര്‍ന്ന സ്ഥലത്ത് സ്ഥിതി ചെയ്യുന്ന തോട്ടത്തോടാകുന്നു. അതിന്നൊരു കനത്ത മഴ ലഭിച്ചപ്പോള്‍ അത് രണ്ടിരട്ടി കായ്കനികള്‍ നല്‍കി. ഇനി അതിന്ന് കനത്ത മഴയൊന്നും കിട്ടിയില്ല, ഒരു ചാറല്‍ മഴയേ ലഭിച്ചുള്ളൂ എങ്കില്‍ അതും മതിയാകുന്നതാണ്‌. അല്ലാഹു നിങ്ങള്‍ പ്രവര്‍ത്തിക്കുന്നതെല്ലാം കണ്ടറിയുന്നവനാകുന്നു.
\end{malayalam}}
\flushright{\begin{Arabic}
\quranayah[2][266]
\end{Arabic}}
\flushleft{\begin{malayalam}
നിങ്ങളില്‍ ഒരാള്‍ക്ക് ഈന്തപ്പനകളും മുന്തിരി വള്ളികളുമുള്ള ഒരു തോട്ടമുണ്ടെന്ന് കരുതുക. അവയുടെ താഴ്ഭാഗത്തുകൂടി അരുവികള്‍ ഒഴുകിക്കൊണ്ടിരിക്കുന്നു. എല്ലാതരം കായ്കനികളും അയാള്‍ക്കതിലുണ്ട്‌. അയാള്‍ക്കാകട്ടെ വാര്‍ദ്ധക്യം ബാധിച്ചിരിക്കുകയാണ്‌. അയാള്‍ക്ക് ദുര്‍ബലരായ കുറെ സന്താനങ്ങളുണ്ട്‌. അപ്പോഴതാ തീയോടു കൂടിയ ഒരു ചുഴലിക്കാറ്റ് അതിന്നു ബാധിച്ച് അത് കരിഞ്ഞു പോകുന്നു. ഇത്തരം ഒരു സ്ഥിതിയിലാകാന്‍ നിങ്ങളാരെങ്കിലും ആഗ്രഹിക്കുമോ ? നിങ്ങള്‍ ചിന്തിക്കുന്നതിനു വേണ്ടി ഇപ്രകാരം അല്ലാഹു തെളിവുകള്‍ വിവരിച്ചുതരുന്നു.
\end{malayalam}}
\flushright{\begin{Arabic}
\quranayah[2][267]
\end{Arabic}}
\flushleft{\begin{malayalam}
സത്യവിശ്വാസികളേ, നിങ്ങള്‍ സമ്പാദിച്ചുണ്ടാക്കിയ നല്ല വസ്തുക്കളില്‍ നിന്നും, ഭൂമിയില്‍ നിന്ന് നിങ്ങള്‍ക്ക് നാം ഉല്‍പാദിപ്പിച്ച് തന്നതില്‍ നിന്നും നിങ്ങള്‍ ചെലവഴിക്കുവിന്‍. കണ്ണടച്ചുകൊണ്ടല്ലാതെ നിങ്ങള്‍ സ്വീകരിക്കാത്ത മോശമായ സാധനങ്ങള്‍ (ദാനധര്‍മ്മങ്ങളില്‍) ചെലവഴിക്കുവാനായി കരുതി വെക്കരുത്‌. അല്ലാഹു ആരുടെയും ആശ്രയമില്ലാത്തവനും സ്തുത്യര്‍ഹനുമാണെന്ന് നിങ്ങള്‍ അറിഞ്ഞു കൊള്ളുക.
\end{malayalam}}
\flushright{\begin{Arabic}
\quranayah[2][268]
\end{Arabic}}
\flushleft{\begin{malayalam}
പിശാച് ദാരിദ്യത്തെപ്പറ്റി നിങ്ങളെ പേടിപ്പെടുത്തുകയും, നീചവൃത്തികള്‍ക്ക് നിങ്ങളെ പ്രേരിപ്പിക്കുകയും ചെയ്യുന്നു. അല്ലാഹുവാകട്ടെ അവന്റെ പക്കല്‍ നിന്നുള്ള മാപ്പും അനുഗ്രഹവും നിങ്ങള്‍ക്ക് വാഗ്ദാനം ചെയ്യുന്നു. അല്ലാഹു വിപുലമായ കഴിവുകളുള്ളവനും (എല്ലാം) അറിയുന്നവനുമാകുന്നു.
\end{malayalam}}
\flushright{\begin{Arabic}
\quranayah[2][269]
\end{Arabic}}
\flushleft{\begin{malayalam}
താന്‍ ഉദ്ദേശിക്കുന്നവര്‍ക്ക് അല്ലാഹു (യഥാര്‍ത്ഥ) ജ്ഞാനം നല്‍കുന്നു. ഏതൊരുവന്ന് (യഥാര്‍ത്ഥ) ജ്ഞാനം നല്‍കപ്പെടുന്നുവോ അവന്ന് (അതു വഴി) അത്യധികമായ നേട്ടമാണ് നല്‍കപ്പെടുന്നത്‌. എന്നാല്‍ ബുദ്ധിശാലികള്‍ മാത്രമേ ശ്രദ്ധിച്ച് മനസ്സിലാക്കുകയുള്ളൂ.
\end{malayalam}}
\flushright{\begin{Arabic}
\quranayah[2][270]
\end{Arabic}}
\flushleft{\begin{malayalam}
നിങ്ങളെന്തൊന്ന് ചെലവഴിച്ചാലും ഏതൊന്ന് നേര്‍ച്ച നേര്‍ന്നാലും തീര്‍ച്ചയായും അല്ലാഹു അതറിയുന്നതാണ്‌. അക്രമകാരികള്‍ക്ക സഹായികളായി ആരും തന്നെ ഉണ്ടായിരിക്കുന്നതല്ല.
\end{malayalam}}
\flushright{\begin{Arabic}
\quranayah[2][271]
\end{Arabic}}
\flushleft{\begin{malayalam}
നിങ്ങള്‍ ദാനധര്‍മ്മങ്ങള്‍ പരസ്യമായി ചെയ്യുന്നുവെങ്കില്‍ അത് നല്ലതു തന്നെ. എന്നാല്‍ നിങ്ങളത് രഹസ്യമാക്കുകയും ദരിദ്രര്‍ക്ക് കൊടുക്കുകയുമാണെങ്കില്‍ അതാണ് നിങ്ങള്‍ക്ക് കൂടുതല്‍ ഉത്തമം. നിങ്ങളുടെ പല തിന്‍മകളെയും അത് മായ്ച്ചുകളയുകയും ചെയ്യും. അല്ലാഹു നിങ്ങള്‍ പ്രവര്‍ത്തിക്കുന്ന കാര്യങ്ങള്‍ സൂക്ഷ്മമായി അറിയുന്നവനാകുന്നു.
\end{malayalam}}
\flushright{\begin{Arabic}
\quranayah[2][272]
\end{Arabic}}
\flushleft{\begin{malayalam}
അവരെ നേര്‍വഴിയിലാക്കാന്‍ നീ ബാധ്യസ്ഥനല്ല. എന്നാല്‍ അല്ലാഹു അവന്‍ ഉദ്ദേശിക്കുന്നവരെ നേര്‍വഴിയിലാക്കുന്നു. നല്ലതായ എന്തെങ്കിലും നിങ്ങള്‍ ചെലവഴിക്കുകയാണെങ്കില്‍ അത് നിങ്ങളുടെ നന്‍മയ്ക്ക് വേണ്ടി തന്നെയാണ്‌. അല്ലാഹുവിന്റെ പ്രീതി തേടിക്കൊണ്ട് മാത്രമാണ് നിങ്ങള്‍ ചെലവഴിക്കേണ്ടത്‌. നല്ലതെന്ത് നിങ്ങള്‍ ചെലവഴിച്ചാലും അതിന്നുള്ള പ്രതിഫലം നിങ്ങള്‍ക്ക് പൂര്‍ണ്ണമായി നല്‍കപ്പെടുന്നതാണ്‌. നിങ്ങളോട് ഒട്ടും അനീതി കാണിക്കപ്പെടുകയില്ല.
\end{malayalam}}
\flushright{\begin{Arabic}
\quranayah[2][273]
\end{Arabic}}
\flushleft{\begin{malayalam}
ഭൂമിയില്‍ സഞ്ചരിച്ച് ഉപജീവനം തേടാന്‍ സൗകര്യപ്പെടാത്ത വിധം അല്ലാഹുവിന്റെ മാര്‍ഗത്തില്‍ വ്യാപൃതരായിട്ടുള്ള ദരിദ്രന്‍മാര്‍ക്ക് വേണ്ടി (നിങ്ങള്‍ ചെലവ് ചെയ്യുക.) (അവരെപ്പറ്റി) അറിവില്ലാത്തവന്‍ (അവരുടെ) മാന്യത കണ്ട് അവര്‍ ധനികരാണെന്ന് ധരിച്ചേക്കും. എന്നാല്‍ അവരുടെ ലക്ഷണം കൊണ്ട് നിനക്കവരെ തിരിച്ചറിയാം. അവര്‍ ജനങ്ങളോട് ചോദിച്ച് വിഷമിപ്പിക്കുകയില്ല. നല്ലതായ എന്തൊന്ന് നിങ്ങള്‍ ചെലവഴിക്കുകയാണെങ്കിലും അല്ലാഹു അത് നല്ലത് പോലെ അറിയുന്നവനാണ്‌.
\end{malayalam}}
\flushright{\begin{Arabic}
\quranayah[2][274]
\end{Arabic}}
\flushleft{\begin{malayalam}
രാത്രിയും പകലും രഹസ്യമായും പരസ്യമായും തങ്ങളുടെ സ്വത്തുക്കള്‍ ചെലവഴിച്ചു കൊണ്ടിരിക്കുന്നവര്‍ക്ക് അവരുടെ രക്ഷിതാവിങ്കല്‍ അവര്‍ അര്‍ഹിക്കുന്ന പ്രതിഫലമുണ്ടായിരിക്കുന്നതാണ്‌. അവര്‍ യാതൊന്നും ഭയപ്പെടേണ്ടതില്ല. അവര്‍ ദുഃഖിക്കേണ്ടി വരികയുമില്ല.
\end{malayalam}}
\flushright{\begin{Arabic}
\quranayah[2][275]
\end{Arabic}}
\flushleft{\begin{malayalam}
പലിശ തിന്നുന്നവര്‍ പിശാച് ബാധ നിമിത്തം മറിഞ്ഞുവീഴുന്നവന്‍ എഴുന്നേല്‍ക്കുന്നത് പോലെയല്ലാതെ എഴുന്നേല്‍ക്കുകയില്ല. കച്ചവടവും പലിശ പോലെത്തന്നെയാണ് എന്ന് അവര്‍ പറഞ്ഞതിന്റെ ഫലമത്രെ അത്‌. എന്നാല്‍ കച്ചവടം അല്ലാഹു അനുവദിക്കുകയും പലിശ നിഷിദ്ധമാക്കുകയുമാണ് ചെയ്തിട്ടുള്ളത്‌. അതിനാല്‍ അല്ലാഹുവിന്റെ ഉപദേശം വന്നുകിട്ടിയിട്ട് (അതനുസരിച്ച്‌) വല്ലവനും (പലിശയില്‍ നിന്ന്‌) വിരമിച്ചാല്‍ അവന്‍ മുമ്പ് വാങ്ങിയത് അവന്നുള്ളത് തന്നെ. അവന്റെ കാര്യം അല്ലാഹുവിന്റെ തീരുമാനത്തിന്ന് വിധേയമായിരിക്കുകയും ചെയ്യും. ഇനി ആരെങ്കിലും (പലിശയിടപാടുകളിലേക്ക് തന്നെ) മടങ്ങുകയാണെങ്കില്‍ അവരത്രെ നരകാവകാശികള്‍. അവരതില്‍ നിത്യവാസികളായിരിക്കും.
\end{malayalam}}
\flushright{\begin{Arabic}
\quranayah[2][276]
\end{Arabic}}
\flushleft{\begin{malayalam}
അല്ലാഹു പലിശയെ ക്ഷയിപ്പിക്കുകയും ദാനധര്‍മ്മങ്ങളെ പോഷിപ്പിക്കുകയും ചെയ്യും. യാതൊരു നന്ദികെട്ട ദുര്‍വൃത്തനെയും അല്ലാഹു ഇഷ്ടപ്പെടുന്നതല്ല.
\end{malayalam}}
\flushright{\begin{Arabic}
\quranayah[2][277]
\end{Arabic}}
\flushleft{\begin{malayalam}
വിശ്വസിക്കുകയും സല്‍കര്‍മ്മങ്ങള്‍ പ്രവര്‍ത്തിക്കുകയും, നമസ്കാരം മുറപോലെ നിര്‍വഹിക്കുകയും, സകാത്ത് കൊടുക്കുകയും ചെയ്യുന്നവര്‍ക്ക് അവരുടെ രക്ഷിതാവിങ്കല്‍ അവര്‍ അര്‍ഹിക്കുന്ന പ്രതിഫലമുണ്ടായിരിക്കുന്നതാണ്‌. അവര്‍ക്ക് യാതൊന്നും ഭയപ്പെടേണ്ടതില്ല. അവര്‍ ദുഃഖിക്കേണ്ടി വരികയുമില്ല.
\end{malayalam}}
\flushright{\begin{Arabic}
\quranayah[2][278]
\end{Arabic}}
\flushleft{\begin{malayalam}
സത്യവിശ്വാസികളേ, നിങ്ങള്‍ അല്ലാഹുവെ സൂക്ഷിക്കുകയും, പലിശവകയില്‍ ബാക്കി കിട്ടാനുള്ളത് വിട്ടുകളയുകയും ചെയ്യേണ്ടതാണ്‌. നിങ്ങള്‍ (യഥാര്‍ത്ഥ) വിശ്വാസികളാണെങ്കില്‍.
\end{malayalam}}
\flushright{\begin{Arabic}
\quranayah[2][279]
\end{Arabic}}
\flushleft{\begin{malayalam}
നിങ്ങള്‍ അങ്ങനെ ചെയ്യുന്നില്ലെങ്കില്‍ അല്ലാഹുവിന്റേയും റസൂലിന്റേയും പക്ഷത്തു നിന്ന് (നിങ്ങള്‍ക്കെതിരിലുള്ള) സമര പ്രഖ്യാപനത്തെപ്പറ്റി അറിഞ്ഞുകൊള്ളുക. നിങ്ങള്‍ പശ്ചാത്തപിച്ചു മടങ്ങുകയാണെങ്കില്‍ നിങ്ങളുടെ മൂലധനം നിങ്ങള്‍ക്കു തന്നെ കിട്ടുന്നതാണ്‌. നിങ്ങള്‍ അക്രമം ചെയ്യരുത്‌. നിങ്ങള്‍ അക്രമിക്കപ്പെടുകയും അരുത്‌.
\end{malayalam}}
\flushright{\begin{Arabic}
\quranayah[2][280]
\end{Arabic}}
\flushleft{\begin{malayalam}
ഇനി (കടം വാങ്ങിയവരില്‍) വല്ല ഞെരുക്കക്കാരനും ഉണ്ടായിരുന്നാല്‍ (അവന്ന്‌) ആശ്വാസമുണ്ടാകുന്നത് വരെ ഇടകൊടുക്കേണ്ടതാണ്‌. എന്നാല്‍ നിങ്ങള്‍ ദാനമായി (വിട്ടു) കൊടുക്കുന്നതാണ് നിങ്ങള്‍ക്ക് കൂടുതല്‍ ഉത്തമം; നിങ്ങള്‍ അറിവുള്ളവരാണെങ്കില്‍.
\end{malayalam}}
\flushright{\begin{Arabic}
\quranayah[2][281]
\end{Arabic}}
\flushleft{\begin{malayalam}
നിങ്ങള്‍ അല്ലാഹുവിങ്കലേക്ക് മടക്കപ്പെടുന്ന ഒരു ദിവസത്തെ സൂക്ഷിച്ചുകൊള്ളുക. എന്നിട്ട് ഓരോരുത്തര്‍ക്കും അവരവര്‍ പ്രവര്‍ത്തിച്ചതിന്റെ ഫലം പൂര്‍ണ്ണമായി നല്‍കപ്പെടുന്നതാണ്‌. അവരോട് (ഒട്ടും) അനീതി കാണിക്കപ്പെടുകയില്ല.
\end{malayalam}}
\flushright{\begin{Arabic}
\quranayah[2][282]
\end{Arabic}}
\flushleft{\begin{malayalam}
സത്യവിശ്വാസികളേ, ഒരു നിശ്ചിത അവധിവെച്ചു കൊണ്ട് നിങ്ങള്‍ അന്യോന്യം വല്ല കടമിടപാടും നടത്തിയാല്‍ നിങ്ങള്‍ അത് എഴുതി വെക്കേണ്ടതാണ്‌. ഒരു എഴുത്തുകാരന്‍ നിങ്ങള്‍ക്കിടയില്‍ നീതിയോടെ അത് രേഖപ്പെടുത്തട്ടെ. ഒരു എഴുത്തുകാരനും അല്ലാഹു അവന്ന് പഠിപ്പിച്ചുകൊടുത്ത പ്രകാരം എഴുതാന്‍ വിസമ്മതിക്കരുത്‌. അവനത് എഴുതുകയും, കടബാധ്യതയുള്ളവന്‍ (എഴുതേണ്ട വാചകം) പറഞ്ഞുകൊടുക്കുകയും ചെയ്യട്ടെ. തന്‍റെരക്ഷിതാവായ അല്ലാഹുവെ അവന്‍ സൂക്ഷിക്കുകയും (ബാധ്യതയില്‍) അവന്‍ യാതൊന്നും കുറവ് വരുത്താതിരിക്കുകയും ചെയ്യേണ്ടതാണ്‌. ഇനി കടബാധ്യതയുള്ള ആള്‍ വിവേകമില്ലാത്തവനോ, കാര്യശേഷിയില്ലാത്തവനോ, (വാചകം) പറഞ്ഞുകൊടുക്കാന്‍ കഴിവില്ലാത്തവനോ ആണെങ്കില്‍ അയാളുടെ രക്ഷാധികാരി അയാള്‍ക്കു വേണ്ടി നീതിപൂര്‍വ്വം (വാചകം) പറഞ്ഞു കൊടുക്കേണ്ടതാണ്‌. നിങ്ങളില്‍ പെട്ട രണ്ടുപുരുഷന്‍മാരെ നിങ്ങള്‍ സാക്ഷി നിര്‍ത്തുകയും ചെയ്യുക. ഇനി ഇരുവരും പുരുഷന്‍മാരായില്ലെങ്കില്‍ നിങ്ങള്‍ ഇഷ്ടപെടുന്ന സാക്ഷികളില്‍ നിന്ന് ഒരു പുരുഷനും രണ്ട് സ്ത്രീകളും ആയാലും മതി. അവരില്‍ ഒരുവള്‍ക്ക് തെറ്റ് പറ്റിയാല്‍ മറ്റവള്‍ അവളെ ഓര്‍മിപ്പിക്കാന്‍ വേണ്ടി. (തെളിവ് നല്‍കാന്‍) വിളിക്കപ്പെട്ടാല്‍ സാക്ഷികള്‍ വിസമ്മതിക്കരുത്‌. ഇടപാട് ചെറുതായാലും വലുതായാലും അതിന്റെ അവധി കാണിച്ച് അത് രേഖപ്പെടുത്തി വെക്കാന്‍ നിങ്ങള്‍ മടിക്കരുത്‌. അതാണ് അല്ലാഹുവിങ്കല്‍ ഏറ്റവും നീതിപൂര്‍വ്വകമായതും, സാക്ഷ്യത്തിന് കൂടുതല്‍ ബലം നല്‍കുന്നതും, നിങ്ങള്‍ക്ക് സംശയം ജനിക്കാതിരിക്കാന്‍ കൂടുതല്‍ അനുയോജ്യമായിട്ടുള്ളതും. എന്നാല്‍ നിങ്ങള്‍ അന്യോന്യം റൊക്കമായി നടത്തിക്കൊണ്ടിരിക്കുന്ന കച്ചവട ഇടപാടുകള്‍ ഇതില്‍ നിന്നൊഴിവാകുന്നു. അതെഴുതി വെക്കാതിരിക്കുന്നതില്‍ നിങ്ങള്‍ക്ക് കുറ്റമില്ല. എന്നാല്‍ നിങ്ങള്‍ ക്രയവിക്രയം ചെയ്യുമ്പോള്‍ സാക്ഷി നിര്‍ത്തേണ്ടതാണ്‌. ഒരു എഴുത്തുകാരനോ സാക്ഷിയോ ദ്രോഹിക്കപ്പെടാന്‍ പാടില്ല. നിങ്ങളങ്ങനെ ചെയ്യുകയാണെങ്കില്‍ അത് നിങ്ങളുടെ ധിക്കാരമാകുന്നു. നിങ്ങള്‍ അല്ലാഹുവെ സൂക്ഷിക്കുക. അല്ലാഹു നിങ്ങള്‍ക്ക് പഠിപ്പിച്ചു തരികയാകുന്നു. അല്ലാഹു ഏത് കാര്യത്തെപ്പറ്റിയും അറിവുള്ളവനാകുന്നു.
\end{malayalam}}
\flushright{\begin{Arabic}
\quranayah[2][283]
\end{Arabic}}
\flushleft{\begin{malayalam}
ഇനി നിങ്ങള്‍ യാത്രയിലാവുകയും ഒരു എഴുത്തുകാരനെ കിട്ടാതിരിക്കുകയുമാണെങ്കില്‍ പണയ വസ്തുക്കള്‍ കൈവശം കൊടുത്താല്‍ മതി. ഇനി നിങ്ങളിലൊരാള്‍ മറ്റൊരാളെ (വല്ലതും) വിശ്വസിച്ചേല്‍പിച്ചാല്‍ ആ വിശ്വാസമര്‍പ്പിക്കപ്പെട്ടവന്‍ തന്റെ വിശ്വസ്തത നിറവേറ്റുകയും, തന്റെ രക്ഷിതാവിനെ സൂക്ഷിക്കുകയും ചെയ്യട്ടെ. നിങ്ങള്‍ സാക്ഷ്യം മറച്ചു വെക്കരുത്‌. ആരത് മറച്ചു വെക്കുന്നുവോ അവന്റെ മനസ്സ് പാപപങ്കിലമാകുന്നു. അല്ലാഹു നിങ്ങള്‍ ചെയ്യുന്നതെല്ലാം അറിയുന്നവനാകുന്നു.
\end{malayalam}}
\flushright{\begin{Arabic}
\quranayah[2][284]
\end{Arabic}}
\flushleft{\begin{malayalam}
ആകാശഭൂമികളിലുള്ളതെല്ലാം അല്ലാഹുവിന്റേതാകുന്നു. നിങ്ങളുടെ മനസ്സുകളിലുള്ളത് നിങ്ങള്‍ വെളിപ്പെടുത്തിയാലും മറച്ചു വെച്ചാലും അല്ലാഹു അതിന്റെ പേരില്‍ നിങ്ങളോട് കണക്ക് ചോദിക്കുക തന്നെ ചെയ്യും. എന്നിട്ടവന്‍ ഉദ്ദേശിക്കുന്നവര്‍ക്ക് അവന്‍ പൊറുത്തുകൊടുക്കുകയും അവന്‍ ഉദ്ദേശിക്കുന്നവരെ അവന്‍ ശിക്ഷിക്കുകയും ചെയ്യും. അല്ലാഹു ഏത് കാര്യത്തിനും കഴിവുള്ളവനാകുന്നു.
\end{malayalam}}
\flushright{\begin{Arabic}
\quranayah[2][285]
\end{Arabic}}
\flushleft{\begin{malayalam}
തന്റെ രക്ഷിതാവിങ്കല്‍ നിന്ന് തനിക്ക് അവതരിപ്പിക്കപ്പെട്ടതില്‍ റസൂല്‍ വിശ്വസിച്ചിരിക്കുന്നു. (അതിനെ തുടര്‍ന്ന്‌) സത്യവിശ്വാസികളും. അവരെല്ലാം അല്ലാഹുവിലും, അവന്റെ മലക്കുകളിലും അവന്റെ വേദഗ്രന്ഥങ്ങളിലും, അവന്റെ ദൂതന്‍മാരിലും വിശ്വസിച്ചിരിക്കുന്നു. അവന്റെ ദൂതന്‍മാരില്‍ ആര്‍ക്കുമിടയില്‍ ഒരു വിവേചനവും ഞങ്ങള്‍ കല്‍പിക്കുന്നില്ല. (എന്നതാണ് അവരുടെ നിലപാട്‌.) അവര്‍ പറയുകയും ചെയ്തു: ഞങ്ങളിതാ കേള്‍ക്കുകയും അനുസരിക്കുകയും ചെയ്തിരിക്കുന്നു. ഞങ്ങളുടെ നാഥാ! ഞങ്ങളോട് പൊറുക്കേണമേ. നിന്നിലേക്കാകുന്നു (ഞങ്ങളുടെ) മടക്കം.
\end{malayalam}}
\flushright{\begin{Arabic}
\quranayah[2][286]
\end{Arabic}}
\flushleft{\begin{malayalam}
അല്ലാഹു ഒരാളോടും അയാളുടെ കഴിവില്‍ പെട്ടതല്ലാതെ ചെയ്യാന്‍ നിര്‍ബന്ധിക്കുകയില്ല. ഓരോരുത്തര്‍ പ്രവര്‍ത്തിച്ചതിന്റെ സത്‍ഫലം അവരവര്‍ക്കുതന്നെ. ഓരോരുത്തര്‍ പ്രവര്‍ത്തിച്ചതിന്റെ ദുഷ്‌ഫലവും അവരവരുടെ മേല്‍ തന്നെ. ഞങ്ങളുടെ നാഥാ, ഞങ്ങള്‍ മറന്നുപോകുകയോ, ഞങ്ങള്‍ക്ക് തെറ്റുപറ്റുകയോ ചെയ്തുവെങ്കില്‍ ഞങ്ങളെ നീ ശിക്ഷിക്കരുതേ. ഞങ്ങളുടെ നാഥാ, ഞങ്ങളുടെ മുന്‍ഗാമികളുടെ മേല്‍ നീ ചുമത്തിയതു പോലുള്ള ഭാരം ഞങ്ങളുടെ മേല്‍ നീ ചുമത്തരുതേ. ഞങ്ങളുടെ നാഥാ, ഞങ്ങള്‍ക്ക് കഴിവില്ലാത്തത് ഞങ്ങളെ നീ വഹിപ്പിക്കരുതേ. ഞങ്ങള്‍ക്ക് നീ മാപ്പുനല്‍കുകയും ഞങ്ങളോട് പൊറുക്കുകയും, കരുണ കാണിക്കുകയും ചെയ്യേണമേ. നീയാണ് ഞങ്ങളുടെ രക്ഷാധികാരി. അതുകൊണ്ട് സത്യനിഷേധികളായ ജനതയ്ക്കെതിരായി നീ ഞങ്ങളെ സഹായിക്കേണമേ.
\end{malayalam}}
\chapter{\textmalayalam{ആലു ഇംറാന്‍ ( ഇംറാന്‍ കുടുംബം )}}
\begin{Arabic}
\Huge{\centerline{\basmalah}}\end{Arabic}
\flushright{\begin{Arabic}
\quranayah[3][1]
\end{Arabic}}
\flushleft{\begin{malayalam}
അലിഫ് ലാം മീം.
\end{malayalam}}
\flushright{\begin{Arabic}
\quranayah[3][2]
\end{Arabic}}
\flushleft{\begin{malayalam}
അല്ലാഹു - അവനല്ലാതെ ഒരു ദൈവവുമില്ല. എന്നെന്നും ജീവിച്ചിരിക്കുന്നവന്‍. എല്ലാം നിയന്ത്രിക്കുന്നവന്‍.
\end{malayalam}}
\flushright{\begin{Arabic}
\quranayah[3][3]
\end{Arabic}}
\flushleft{\begin{malayalam}
അവന്‍ ഈ വേദഗ്രന്ഥത്തെ മുന്‍ വേദങ്ങളെ ശരിവെക്കുന്നതായിക്കൊണ്ട് സത്യവുമായി നിനക്ക് അവതരിപ്പിച്ചു തന്നിരിക്കുന്നു. അവന്‍ തൌറാത്തും ഇന്‍ജീലും അവതരിപ്പിച്ചു.
\end{malayalam}}
\flushright{\begin{Arabic}
\quranayah[3][4]
\end{Arabic}}
\flushleft{\begin{malayalam}
ഇതിനു മുമ്പ്‌; മനുഷ്യര്‍ക്ക് മാര്‍ഗദര്‍ശനത്തിനായിട്ട് സത്യാസത്യവിവേചനത്തിനുള്ള പ്രമാണവും അവന്‍ അവതരിപ്പിച്ചിരിക്കുന്നു. തീര്‍ച്ചയായും അല്ലാഹുവിന്‍റെ ദൃഷ്ടാന്തങ്ങള്‍ നിഷേധിച്ചവരാരോ അവര്‍ക്ക് കഠിനമായ ശിക്ഷയാണുള്ളത്‌. അല്ലാഹു പ്രതാപിയും ശിക്ഷാനടപടി സ്വീകരിക്കുന്നവനുമാകുന്നു.
\end{malayalam}}
\flushright{\begin{Arabic}
\quranayah[3][5]
\end{Arabic}}
\flushleft{\begin{malayalam}
ഭൂമിയിലോ ആകാശത്തോ ഉള്ള യാതൊരു കാര്യവും അല്ലാഹുവിന്ന് അവ്യക്തമായിപ്പോകുകയില്ല; തീര്‍ച്ച.
\end{malayalam}}
\flushright{\begin{Arabic}
\quranayah[3][6]
\end{Arabic}}
\flushleft{\begin{malayalam}
ഗര്‍ഭാശയങ്ങളില്‍ താന്‍ ഉദ്ദേശിക്കുന്ന വിധത്തില്‍ നിങ്ങളെ രൂപപ്പെടുത്തുന്നത് അവനത്രെ. അവനല്ലാതെ ഒരു ദൈവവുമില്ല. അവന്‍ പ്രതാപിയും യുക്തിമാനുമത്രെ.
\end{malayalam}}
\flushright{\begin{Arabic}
\quranayah[3][7]
\end{Arabic}}
\flushleft{\begin{malayalam}
(നബിയേ,) നിനക്ക് വേദഗ്രന്ഥം അവതരിപ്പിച്ചു തന്നിരിക്കുന്നത് അവനത്രെ. അതില്‍ സുവ്യക്തവും ഖണ്ഡിതവുമായ വചനങ്ങളുണ്ട്‌. അവയത്രെ വേദഗ്രന്ഥത്തിന്‍റെ മൌലികഭാഗം. ആശയത്തില്‍ സാദൃശ്യമുള്ള ചില വചനങ്ങളുമുണ്ട്‌. എന്നാല്‍ മനസ്സുകളില്‍ വക്രതയുള്ളവര്‍ കുഴപ്പമുണ്ടാക്കാന്‍ ഉദ്ദേശിച്ചുകൊണ്ടും, ദുര്‍വ്യാഖ്യാനം നടത്താന്‍ ആഗ്രഹിച്ചു കൊണ്ടും ആശയത്തില്‍ സാദൃശ്യമുള്ള വചനങ്ങളെ പിന്തുടരുന്നു. അതിന്‍റെ സാക്ഷാല്‍ വ്യാഖ്യാനം അല്ലാഹുവിന് മാത്രമേ അറിയുകയുള്ളൂ. അറിവില്‍ അടിയുറച്ചവാരാകട്ടെ, അവര്‍ പറയും: ഞങ്ങളതില്‍ വിശ്വസിച്ചിരിക്കുന്നു. എല്ലാം ഞങ്ങളുടെ രക്ഷിതാവിങ്കല്‍ നിന്നുള്ളതാകുന്നു. ബുദ്ധിശാലികള്‍ മാത്രമേ ആലോചിച്ച് മനസ്സിലാക്കുകയുള്ളൂ.
\end{malayalam}}
\flushright{\begin{Arabic}
\quranayah[3][8]
\end{Arabic}}
\flushleft{\begin{malayalam}
അവര്‍ പ്രാര്‍ത്ഥിക്കും:) ഞങ്ങളുടെ നാഥാ, ഞങ്ങളെ നീ സന്‍മാര്‍ഗത്തിലാക്കിയതിനു ശേഷം ഞങ്ങളുടെ മനസ്സുകളെ നീ തെറ്റിക്കരുതേ. നിന്‍റെ അടുക്കല്‍ നിന്നുള്ള കാരുണ്യം ഞങ്ങള്‍ക്ക് നീ പ്രദാനം ചെയ്യേണമേ. തീര്‍ച്ചയായും നീ അത്യധികം ഔദാര്യവാനാകുന്നു
\end{malayalam}}
\flushright{\begin{Arabic}
\quranayah[3][9]
\end{Arabic}}
\flushleft{\begin{malayalam}
ഞങ്ങളുടെ നാഥാ, തീര്‍ച്ചയായും നീ ജനങ്ങളെയെല്ലാം ഒരു ദിവസം ഒരുമിച്ചുകൂട്ടുന്നതാകുന്നു. അതില്‍ യാതൊരു സംശയവുമില്ല. തീര്‍ച്ചയായും അല്ലാഹു വാഗ്ദാനം ലംഘിക്കുന്നതല്ല.
\end{malayalam}}
\flushright{\begin{Arabic}
\quranayah[3][10]
\end{Arabic}}
\flushleft{\begin{malayalam}
സത്യനിഷേധം കൈക്കൊണ്ടവര്‍ക്ക് അവരുടെ സ്വത്തുക്കളോ സന്താനങ്ങളോ അല്ലാഹുവിങ്കല്‍ യാതൊരു പ്രയോജനവും ചെയ്യുകയില്ല; തീര്‍ച്ച. അവരാകുന്നു നരകത്തിലെ ഇന്ധനമായിത്തീരുന്നവര്‍.
\end{malayalam}}
\flushright{\begin{Arabic}
\quranayah[3][11]
\end{Arabic}}
\flushleft{\begin{malayalam}
ഫിര്‍ഔന്‍റെ ആള്‍ക്കാരുടെയും അവരുടെ മുന്‍ഗാമികളുടെയും അവസ്ഥ പോലെത്തന്നെ. അവരൊക്കെ നമ്മുടെ ദൃഷ്ടാന്തങ്ങളെ തള്ളിക്കളഞ്ഞു. അപ്പോള്‍ അവരുടെ പാപങ്ങള്‍ കാരണമായി അല്ലാഹു അവരെ പിടികൂടി. അല്ലാഹു കഠിനമായി ശിക്ഷിക്കുന്നവനാകുന്നു.
\end{malayalam}}
\flushright{\begin{Arabic}
\quranayah[3][12]
\end{Arabic}}
\flushleft{\begin{malayalam}
(നബിയേ,) നീ സത്യനിഷേധികളോട് പറയുക: നിങ്ങള്‍ കീഴടക്കപ്പെടുന്നതും നരകത്തിലേക്ക് കൂട്ടത്തോടെ നയിക്കപ്പെടുന്നതുമാണ്‌. അതെത്ര ചീത്തയായ വിശ്രമസ്ഥലം!
\end{malayalam}}
\flushright{\begin{Arabic}
\quranayah[3][13]
\end{Arabic}}
\flushleft{\begin{malayalam}
(ബദ്‌റില്‍) ഏറ്റുമുട്ടിയ ആ രണ്ട് വിഭാഗങ്ങളില്‍ തീര്‍ച്ചയായും നിങ്ങള്‍ക്കൊരു ദൃഷ്ടാന്തമുണ്ട്‌. ഒരു വിഭാഗം അല്ലാഹുവിന്‍റെ മാര്‍ഗത്തില്‍ യുദ്ധം ചെയ്യുന്നു. മറുവിഭാഗമാകട്ടെ സത്യനിഷേധികളും. (അവിശ്വാസികള്‍ക്ക്‌) തങ്ങളുടെ ദൃഷ്ടിയില്‍ അവര്‍ (വിശ്വാസികള്‍) തങ്ങളുടെ ഇരട്ടിയുണ്ടെന്നാണ് തോന്നിയിരുന്നത്‌. അല്ലാഹു താന്‍ ഉദ്ദേശിക്കുന്നവര്‍ക്ക് തന്‍റെ സഹായം കൊണ്ട് പിന്‍ബലം നല്‍കുന്നു. തീര്‍ച്ചയായും കണ്ണുള്ളവര്‍ക്ക് അതില്‍ ഒരു ഗുണപാഠമുണ്ട്‌.
\end{malayalam}}
\flushright{\begin{Arabic}
\quranayah[3][14]
\end{Arabic}}
\flushleft{\begin{malayalam}
ഭാര്യമാര്‍, പുത്രന്‍മാര്‍, കൂമ്പാരമായിക്കൂട്ടിയ സ്വര്‍ണം, വെള്ളി, മേത്തരം കുതിരകള്‍, നാല്‍കാലി വര്‍ഗങ്ങള്‍, കൃഷിയിടം എന്നിങ്ങനെ ഇഷ്ടപെട്ട വസ്തുക്കളോടുള്ള പ്രേമം മനുഷ്യര്‍ക്ക് അലങ്കാരമായി തോന്നിക്കപ്പെട്ടിരിക്കുന്നു. അതൊക്കെ ഇഹലോകജീവിതത്തിലെ വിഭവങ്ങളാകുന്നു. അല്ലാഹുവിന്‍റെ അടുക്കലാകുന്നു (മനുഷ്യര്‍ക്ക്‌) ചെന്നുചേരാനുള്ള ഉത്തമ സങ്കേതം.
\end{malayalam}}
\flushright{\begin{Arabic}
\quranayah[3][15]
\end{Arabic}}
\flushleft{\begin{malayalam}
(നബിയേ,) പറയുക: അതിനെക്കാള്‍ (ആ ഇഹലോക സുഖങ്ങളെക്കാള്‍) നിങ്ങള്‍ക്ക് ഗുണകരമായിട്ടുള്ളത് ഞാന്‍ പറഞ്ഞുതരട്ടെയോ? സൂക്ഷ്മത പാലിച്ചവര്‍ക്ക് തങ്ങളുടെ രക്ഷിതാവിന്‍റെ അടുക്കല്‍ താഴ്ഭാഗത്തു കൂടി അരുവികള്‍ ഒഴുകിക്കൊണ്ടിരിക്കുന്ന സ്വര്‍ഗത്തോപ്പുകളുണ്ട്‌. അവര്‍ അവിടെ നിത്യവാസികളായിരിക്കും. പരിശുദ്ധരായ ഇണകളും (അവര്‍ക്കുണ്ടായിരിക്കും.) കൂടാതെ അല്ലാഹുവിന്‍റെ പ്രീതിയും. അല്ലാഹു തന്‍റെ ദാസന്‍മാരുടെ കാര്യങ്ങള്‍ കണ്ടറിയുന്നവനാകുന്നു.
\end{malayalam}}
\flushright{\begin{Arabic}
\quranayah[3][16]
\end{Arabic}}
\flushleft{\begin{malayalam}
ഞങ്ങളുടെ നാഥാ, ഞങ്ങളിതാ വിശ്വസിച്ചിരിക്കുന്നു. അതിനാല്‍ ഞങ്ങളുടെ പാപങ്ങള്‍ പൊറുത്തുതരികയും, നരക ശിക്ഷയില്‍ നിന്ന് ഞങ്ങളെ രക്ഷിക്കുകയും ചെയ്യേണമേ എന്ന് പ്രാര്‍ത്ഥിക്കുന്നവരും,
\end{malayalam}}
\flushright{\begin{Arabic}
\quranayah[3][17]
\end{Arabic}}
\flushleft{\begin{malayalam}
ക്ഷമ കൈക്കൊള്ളുന്നവരും, സത്യം പാലിക്കുന്നവരും, ഭക്തിയുള്ളവരും ചെലവഴിക്കുന്നവരും, രാത്രിയുടെ അന്ത്യയാമങ്ങളില്‍ പാപമോചനം തേടുന്നവരുമാകുന്നു അവര്‍ (അല്ലാഹുവിന്‍റെ ദാസന്‍മാര്‍.)
\end{malayalam}}
\flushright{\begin{Arabic}
\quranayah[3][18]
\end{Arabic}}
\flushleft{\begin{malayalam}
താനല്ലാതെ ഒരു ദൈവവുമില്ലെന്നതിന് അല്ലാഹു സാക്ഷ്യം വഹിച്ചിരിക്കുന്നു. മലക്കുകളും അറിവുള്ളവരും (അതിന്ന് സാക്ഷികളാകുന്നു.) അവന്‍ നീതി നിര്‍വഹിക്കുന്നവനത്രെ. അവനല്ലാതെ ദൈവമില്ല. പ്രതാപിയും യുക്തിമാനുമത്രെ അവന്‍.
\end{malayalam}}
\flushright{\begin{Arabic}
\quranayah[3][19]
\end{Arabic}}
\flushleft{\begin{malayalam}
തീര്‍ച്ചയായും അല്ലാഹുവിങ്കല്‍ മതം എന്നാല്‍ ഇസ്ലാമാകുന്നു. വേദഗ്രന്ഥം നല്‍കപ്പെട്ടവര്‍ തങ്ങള്‍ക്ക് (മതപരമായ) അറിവ് വന്നുകിട്ടിയ ശേഷം തന്നെയാണ് ഭിന്നിച്ചത്‌. അവര്‍ തമ്മിലുള്ള കക്ഷിമാത്സര്യം നിമിത്തമത്രെ അത്‌. വല്ലവരും അല്ലാഹുവിന്‍റെ തെളിവുകള്‍ നിഷേധിക്കുന്നുവെങ്കില്‍ അല്ലാഹു അതിവേഗം കണക്ക് ചോദിക്കുന്നവനാകുന്നു.
\end{malayalam}}
\flushright{\begin{Arabic}
\quranayah[3][20]
\end{Arabic}}
\flushleft{\begin{malayalam}
ഇനി അവര്‍ നിന്നോട് തര്‍ക്കിക്കുകയാണെങ്കില്‍ നീ പറഞ്ഞേക്കുക: ഞാന്‍ എന്നെത്തന്നെ പൂര്‍ണ്ണമായി അല്ലാഹുവിന്ന് കീഴ്പെടുത്തിയിരിക്കുന്നു. എന്നെ പിന്‍ പറ്റിയവരും (അങ്ങനെ തന്നെ) . വേദഗ്രന്ഥം നല്‍കപ്പെട്ടവരോടും അക്ഷരജ്ഞാനമില്ലാത്തവരോടും (ബഹുദൈവാരാധകരായ അറബികളോട്‌) നീ ചോദിക്കുക: നിങ്ങള്‍ (അല്ലാഹുവിന്ന്‌) കീഴ്പെട്ടുവോ? അങ്ങനെ അവര്‍ കീഴ്പെട്ടു കഴിഞ്ഞാല്‍ അവര്‍ നേര്‍വഴിയിലായിക്കഴിഞ്ഞു. അവര്‍ പിന്തിരിഞ്ഞു കളഞ്ഞാലോ അവര്‍ക്ക് (ദിവ്യ സന്ദേശം) എത്തിക്കേണ്ട ബാധ്യത മാത്രമേ നിനക്കുള്ളൂ. അല്ലാഹു (തന്‍റെ) ദാസന്‍മാരുടെ കാര്യങ്ങള്‍ കണ്ടറിയുന്നവനാകുന്നു.
\end{malayalam}}
\flushright{\begin{Arabic}
\quranayah[3][21]
\end{Arabic}}
\flushleft{\begin{malayalam}
അല്ലാഹുവിന്‍റെ തെളിവുകള്‍ നിഷേധിച്ച് തള്ളുകയും, ഒരു ന്യായവുമില്ലാതെ പ്രവാചകന്‍മാരെ കൊലപ്പെടുത്തുകയും, നീതി പാലിക്കാന്‍ കല്‍പിക്കുന്ന ആളുകളെ കൊലപ്പെടുത്തുകയും ചെയ്യുന്നവരാരോ അവര്‍ക്ക് വേദനയേറിയ ശിക്ഷയെപ്പറ്റി നീ സന്തോഷവാര്‍ത്ത അറിയിക്കുക.
\end{malayalam}}
\flushright{\begin{Arabic}
\quranayah[3][22]
\end{Arabic}}
\flushleft{\begin{malayalam}
തങ്ങളുടെ പ്രവര്‍ത്തനങ്ങള്‍ ഇഹത്തിലും പരത്തിലും നിഷ്ഫലമായിപ്പോയ വിഭാഗമത്രെ അവര്‍. അവര്‍ക്ക് സഹായികളായി ആരും ഉണ്ടായിരിക്കുകയില്ല.
\end{malayalam}}
\flushright{\begin{Arabic}
\quranayah[3][23]
\end{Arabic}}
\flushleft{\begin{malayalam}
വേദഗ്രന്ഥത്തില്‍ നിന്നും ഒരു പങ്ക് നല്‍കപ്പെട്ട ഒരു വിഭാഗത്തെപ്പറ്റി നീ അറിഞ്ഞില്ലേ? അവര്‍ക്കിടയില്‍ തീര്‍പ്പുകല്‍പിക്കുവാനായി അല്ലാഹുവിന്‍റെ ഗ്രന്ഥത്തിലേക്ക് അവര്‍ വിളിക്കപ്പെടുന്നു. എന്നിട്ടതാ അവരില്‍ ഒരു കക്ഷി അവഗണിച്ചു കൊണ്ട് പിന്തിരിഞ്ഞു കളയുന്നു.
\end{malayalam}}
\flushright{\begin{Arabic}
\quranayah[3][24]
\end{Arabic}}
\flushleft{\begin{malayalam}
എണ്ണപ്പെട്ട ഏതാനും ദിവസം മാത്രമേ തങ്ങളെ നരകാഗ്നി സ്പര്‍ശിക്കുകയുള്ളൂ എന്ന് അവര്‍ പറഞ്ഞ് കൊണ്ടിരിക്കുന്ന കാരണത്താലാണ് അവരങ്ങനെയായത്‌. അവര്‍ കെട്ടിച്ചമച്ചുണ്ടാക്കിയിരുന്ന വാദങ്ങള്‍ അവരുടെ മതകാര്യത്തില്‍ അവരെ വഞ്ചിതരാക്കിക്കളഞ്ഞു.
\end{malayalam}}
\flushright{\begin{Arabic}
\quranayah[3][25]
\end{Arabic}}
\flushleft{\begin{malayalam}
എന്നാല്‍ യാതൊരു സംശയത്തിനും ഇടയില്ലാത്ത ഒരു ദിവസത്തിനായി നാമവരെ ഒരുമിച്ചുകൂട്ടിയാല്‍ (അവരുടെ സ്ഥിതി) എങ്ങനെയായിരിക്കും? അന്ന് ഓരോ വ്യക്തിക്കും താന്‍ സമ്പാദിച്ചതിന്‍റെ ഫലം പൂര്‍ണ്ണമായി കൊടുക്കപ്പെടുന്നതാണ്‌. ഒരു അനീതിയും അവരോട് കാണിക്കപ്പെടുന്നതല്ല.
\end{malayalam}}
\flushright{\begin{Arabic}
\quranayah[3][26]
\end{Arabic}}
\flushleft{\begin{malayalam}
പറയുക: ആധിപത്യത്തിന്‍റെ ഉടമസ്ഥനായ അല്ലാഹുവേ, നീ ഉദ്ദേശിക്കുന്നവര്‍ക്ക് നീ ആധിപത്യം നല്‍കുന്നു. നീ ഉദ്ദേശിക്കുന്നവരില്‍ നിന്ന് നീ ആധിപത്യം എടുത്തുനീക്കുകയും ചെയ്യുന്നു. നീ ഉദ്ദേശിക്കുന്നവര്‍ക്ക് നീ പ്രതാപം നല്‍കുന്നു. നീ ഉദ്ദേശിക്കുന്നവര്‍ക്ക് നീ നിന്ദ്യത വരുത്തുകയും ചെയ്യുന്നു. നിന്‍റെ കൈവശമത്രെ നന്‍മയുള്ളത്‌. നിശ്ചയമായും നീ എല്ലാ കാര്യത്തിനും കഴിവുള്ളവനാകുന്നു.
\end{malayalam}}
\flushright{\begin{Arabic}
\quranayah[3][27]
\end{Arabic}}
\flushleft{\begin{malayalam}
രാവിനെ നീ പകലില്‍ പ്രവേശിപ്പിക്കുന്നു. പകലിനെ നീ രാവിലും പ്രവേശിപ്പിക്കുന്നു. ജീവനില്ലാത്തതില്‍ നിന്ന് നീ ജീവിയെ പുറത്ത് വരുത്തുന്നു. ജീവിയില്‍ നിന്ന് ജീവനില്ലാത്തതിനെയും നീ പുറത്തു വരുത്തുന്നു. നീ ഉദ്ദേശിക്കുന്നവര്‍ക്ക് കണക്ക് നോക്കാതെ നീ നല്‍കുകയും ചെയ്യുന്നു.
\end{malayalam}}
\flushright{\begin{Arabic}
\quranayah[3][28]
\end{Arabic}}
\flushleft{\begin{malayalam}
സത്യവിശ്വാസികള്‍ സത്യവിശ്വാസികളെയല്ലാതെ സത്യനിഷേധികളെ മിത്രങ്ങളാക്കിവെക്കരുത്‌. - അങ്ങനെ വല്ലവനും ചെയ്യുന്ന പക്ഷം അല്ലാഹുവുമായി അവന്ന് യാതൊരു ബന്ധവുമില്ല- നിങ്ങള്‍ അവരോട് കരുതലോടെ വര്‍ത്തിക്കുകയാണെങ്കിലല്ലാതെ. അല്ലാഹു അവനെപ്പറ്റി നിങ്ങള്‍ക്ക് താക്കീത് നല്‍കുന്നു. അല്ലാഹുവിങ്കലേക്കത്രെ (നിങ്ങള്‍) തിരിച്ചുചെല്ലേണ്ടത്‌.
\end{malayalam}}
\flushright{\begin{Arabic}
\quranayah[3][29]
\end{Arabic}}
\flushleft{\begin{malayalam}
(നബിയേ,) പറയുക: നിങ്ങളുടെ ഹൃദയങ്ങളിലുള്ളത് നിങ്ങള്‍ മറച്ചു വെച്ചാലും വെളിപ്പെടുത്തിയാലും അല്ലാഹു അറിയുന്നതാണ്‌. ആകാശങ്ങളിലുള്ളതും ഭൂമിയിലുള്ളതും അവനറിയുന്നു. അല്ലാഹു ഏതു കാര്യത്തിനും കഴിവുള്ളവനാകുന്നു.
\end{malayalam}}
\flushright{\begin{Arabic}
\quranayah[3][30]
\end{Arabic}}
\flushleft{\begin{malayalam}
നന്‍മയായും തിന്‍മയായും താന്‍ പ്രവര്‍ത്തിച്ച ഓരോ കാര്യവും (തന്‍റെ മുമ്പില്‍) ഹാജരാക്കപ്പെട്ടതായി ഓരോ വ്യക്തിയും കണ്ടെത്തുന്ന ദിവസത്തെക്കുറിച്ച് (ഓര്‍ക്കുക) . തന്‍റെയും അതിന്‍റെ (ദുഷ്പ്രവൃത്തിയുടെ) യും ഇടയില്‍ വലിയ ദൂരമുണ്ടായിരുന്നെങ്കില്‍ എന്ന് ഓരോ വ്യക്തിയും അന്ന് കൊതിച്ചു പോകും. അല്ലാഹു തന്നെപ്പറ്റി നിങ്ങള്‍ക്ക് താക്കീത് നല്‍കുന്നു. അല്ലാഹു (തന്‍റെ) ദാസന്‍മാരോട് വളരെ ദയയുള്ളവനാകുന്നു.
\end{malayalam}}
\flushright{\begin{Arabic}
\quranayah[3][31]
\end{Arabic}}
\flushleft{\begin{malayalam}
(നബിയേ,) പറയുക: നിങ്ങള്‍ അല്ലാഹുവെ സ്നേഹിക്കുന്നുണ്ടെങ്കില്‍ എന്നെ നിങ്ങള്‍ പിന്തുടരുക. എങ്കില്‍ അല്ലാഹു നിങ്ങളെ സ്നേഹിക്കുകയും നിങ്ങളുടെ പാപങ്ങള്‍ പൊറുത്തുതരികയും ചെയ്യുന്നതാണ്‌. അല്ലാഹു ഏറെ പൊറുക്കുന്നവനും കരുണാനിധിയുമത്രെ.
\end{malayalam}}
\flushright{\begin{Arabic}
\quranayah[3][32]
\end{Arabic}}
\flushleft{\begin{malayalam}
പറയുക: നിങ്ങള്‍ അല്ലാഹുവെയും റസൂലിനെയും അനുസരിക്കുവിന്‍. ഇനി അവര്‍ പിന്തിരിഞ്ഞുകളയുന്ന പക്ഷം അല്ലാഹു സത്യനിഷേധികളെ സ്നേഹിക്കുന്നതല്ല; തീര്‍ച്ച.
\end{malayalam}}
\flushright{\begin{Arabic}
\quranayah[3][33]
\end{Arabic}}
\flushleft{\begin{malayalam}
തീര്‍ച്ചയായും ആദമിനെയും നൂഹിനെയും ഇബ്രാഹീം കുടുംബത്തേയും ഇംറാന്‍ കുടുംബത്തേയും ലോകരില്‍ ഉല്‍കൃഷ്ടരായി അല്ലാഹു തെരഞ്ഞെടുത്തിരിക്കുന്നു.
\end{malayalam}}
\flushright{\begin{Arabic}
\quranayah[3][34]
\end{Arabic}}
\flushleft{\begin{malayalam}
ചിലര്‍ ചിലരുടെ സന്തതികളായിക്കൊണ്ട്‌. അല്ലാഹു (എല്ലാം) കേള്‍ക്കുന്നവനും അറിയുന്നവനുമത്രെ.
\end{malayalam}}
\flushright{\begin{Arabic}
\quranayah[3][35]
\end{Arabic}}
\flushleft{\begin{malayalam}
ഇംറാന്‍റെ ഭാര്യ പറഞ്ഞ സന്ദര്‍ഭം (ശ്രദ്ധിക്കുക:) എന്‍റെ രക്ഷിതാവേ, എന്‍റെ വയറ്റിലുള്ള കുഞ്ഞിനെ നിനക്കായ് ഉഴിഞ്ഞുവെക്കാന്‍ ഞാന്‍ നേര്‍ച്ച നേര്‍ന്നിരിക്കുന്നു. ആകയാല്‍ എന്നില്‍ നിന്ന് നീ അത് സ്വീകരിക്കേണമേ. തീര്‍ച്ചയായും നീ (എല്ലാം) കേള്‍ക്കുന്നവനും അറിയുന്നവനുമത്രെ.
\end{malayalam}}
\flushright{\begin{Arabic}
\quranayah[3][36]
\end{Arabic}}
\flushleft{\begin{malayalam}
എന്നിട്ട് പ്രസവിച്ചപ്പോള്‍ അവള്‍ പറഞ്ഞു: എന്‍റെ രക്ഷിതാവേ, ഞാന്‍ പ്രസവിച്ച കുട്ടി പെണ്ണാണല്ലോ.- എന്നാല്‍ അല്ലാഹു അവള്‍ പ്രസവിച്ചതിനെപ്പറ്റി കൂടുതല്‍ അറിവുള്ളവനത്രെ -ആണ് പെണ്ണിനെപ്പോലെയല്ല. ആ കുട്ടിക്ക് ഞാന്‍ മര്‍യം എന്ന് പേരിട്ടിരിക്കുന്നു. ശപിക്കപ്പെട്ട പിശാചില്‍ നിന്നും അവളെയും അവളുടെ സന്തതികളെയും രക്ഷിക്കുവാനായി ഞാന്‍ നിന്നില്‍ ശരണം പ്രാപിക്കുകയും ചെയ്യുന്നു.
\end{malayalam}}
\flushright{\begin{Arabic}
\quranayah[3][37]
\end{Arabic}}
\flushleft{\begin{malayalam}
അങ്ങനെ അവളുടെ (മര്‍യമിന്‍റെ) രക്ഷിതാവ് അവളെ നല്ല നിലയില്‍ സ്വീകരിക്കുകയും, നല്ല നിലയില്‍ വളര്‍ത്തിക്കൊണ്ടു വരികയും, അവളുടെ സംരക്ഷണച്ചുമതല അവന്‍ സകരിയ്യായെ ഏല്‍പിക്കുകയും ചെയ്തു. മിഹ്‌റാബില്‍ (പ്രാര്‍ത്ഥനാവേദിയില്‍) അവളുടെ അടുക്കല്‍ സകരിയ്യാ കടന്നു ചെല്ലുമ്പോഴെല്ലാം അവളുടെ അടുത്ത് എന്തെങ്കിലും ആഹാരം കണ്ടെത്തുമായിരുന്നു. അദ്ദേഹം ചോദിച്ചു: മര്‍യമേ, നിനക്ക് എവിടെ നിന്നാണിത് കിട്ടിയത്‌? അവള്‍ മറുപടി പറഞ്ഞു. അത് അല്ലാഹുവിങ്കല്‍ നിന്ന് ലഭിക്കുന്നതാകുന്നു. തീര്‍ച്ചയായും അല്ലാഹു താന്‍ ഉദ്ദേശിക്കുന്നവര്‍ക്ക് കണക്ക് നോക്കാതെ നല്‍കുന്നു.
\end{malayalam}}
\flushright{\begin{Arabic}
\quranayah[3][38]
\end{Arabic}}
\flushleft{\begin{malayalam}
അവിടെ വെച്ച് സകരിയ്യ തന്‍റെ രക്ഷിതാവിനോട് പ്രാര്‍ത്ഥിച്ചു: എന്‍റെ രക്ഷിതാവേ, എനിക്ക് നീ നിന്‍റെ പക്കല്‍ നിന്ന് ഒരു ഉത്തമ സന്താനത്തെ നല്‍കേണമേ. തീര്‍ച്ചയായും നീ പ്രാര്‍ത്ഥന കേള്‍ക്കുന്നവനാണല്ലോ എന്ന് അദ്ദേഹം പറഞ്ഞു.
\end{malayalam}}
\flushright{\begin{Arabic}
\quranayah[3][39]
\end{Arabic}}
\flushleft{\begin{malayalam}
അങ്ങനെ അദ്ദേഹം മിഹ്‌റാബില്‍ പ്രാര്‍ത്ഥിച്ചു കൊണ്ട് നില്‍ക്കുമ്പോള്‍ മലക്കുകള്‍ അദ്ദേഹത്തെ വിളിച്ചുകൊണ്ടു പറഞ്ഞു: യഹ്‌യാ (എന്ന കുട്ടി) യെപ്പറ്റി അല്ലാഹു നിനക്ക് സന്തോഷവാര്‍ത്ത അറിയിക്കുന്നു. അല്ലാഹുവിങ്കല്‍ നിന്നുള്ള ഒരു വചനത്തെ ശരിവെക്കുന്നവനും നേതാവും ആത്മനിയന്ത്രണമുള്ളവനും സദ്‌വൃത്തരില്‍ പെട്ട ഒരു പ്രവാചകനും ആയിരിക്കും അവന്‍.
\end{malayalam}}
\flushright{\begin{Arabic}
\quranayah[3][40]
\end{Arabic}}
\flushleft{\begin{malayalam}
അദ്ദേഹം പറഞ്ഞു: എന്‍റെ രക്ഷിതാവേ, എനിക്കെങ്ങനെയാണ് ഒരു ആണ്‍കുട്ടിയുണ്ടാവുക? എനിക്ക് വാര്‍ദ്ധക്യമെത്തിക്കഴിഞ്ഞു. എന്‍റെ ഭാര്യയാണെങ്കില്‍ വന്ധ്യയാണു താനും. അല്ലാഹു പറഞ്ഞു: അങ്ങനെതന്നെയാകുന്നു; അല്ലാഹു താന്‍ ഉദ്ദേശിക്കുന്നത് ചെയ്യുന്നു.
\end{malayalam}}
\flushright{\begin{Arabic}
\quranayah[3][41]
\end{Arabic}}
\flushleft{\begin{malayalam}
അദ്ദേഹം പറഞ്ഞു: എന്‍റെ രക്ഷിതാവേ, എനിക്ക് ഒരു അടയാളം ഏര്‍പെടുത്തിത്തരേണമേ. അല്ലാഹു പറഞ്ഞു: നിനക്കുള്ള അടയാളം ആംഗ്യരൂപത്തിലല്ലാതെ മൂന്നു ദിവസം നീ മനുഷ്യരോട് സംസാരിക്കാതിരിക്കലാകുന്നു. നിന്‍റെ രക്ഷിതാവിനെ നീ ധാരാളം ഓര്‍മിക്കുകയും, വൈകുന്നേരവും രാവിലെയും അവന്‍റെ പരിശുദ്ധിയെ നീ പ്രകീര്‍ത്തിക്കുകയും ചെയ്യുക.
\end{malayalam}}
\flushright{\begin{Arabic}
\quranayah[3][42]
\end{Arabic}}
\flushleft{\begin{malayalam}
മലക്കുകള്‍ പറഞ്ഞ സന്ദര്‍ഭവും (ശ്രദ്ധിക്കുക:) മര്‍യമേ, തീര്‍ച്ചയായും അല്ലാഹു നിന്നെ പ്രത്യേകം തെരഞ്ഞെടുക്കുകയും, നിനക്ക് പരിശുദ്ധി നല്‍കുകയും, ലോകത്തുള്ള സ്ത്രീകളില്‍ വെച്ച് ഉല്‍കൃഷ്ടയായി നിന്നെ തെരഞ്ഞെടുക്കുകയും ചെയ്തിരിക്കുന്നു.
\end{malayalam}}
\flushright{\begin{Arabic}
\quranayah[3][43]
\end{Arabic}}
\flushleft{\begin{malayalam}
മര്‍യമേ, നിന്‍റെ രക്ഷിതാവിനോട് നീ ഭയഭക്തി കാണിക്കുകയും, സാഷ്ടാംഗം ചെയ്യുകയും, തലകുനിക്കുന്നവരോടൊപ്പം തലകുനിക്കുകയും ചെയ്യുക.
\end{malayalam}}
\flushright{\begin{Arabic}
\quranayah[3][44]
\end{Arabic}}
\flushleft{\begin{malayalam}
(നബിയേ,) നാം നിനക്ക് ബോധനം നല്‍കുന്ന അദൃശ്യവാര്‍ത്തകളില്‍ പെട്ടതാകുന്നു അവയൊക്കെ. അവരില്‍ ആരാണ് മര്‍യമിന്‍റെ സംരക്ഷണം ഏറ്റെടുക്കേണ്ടതെന്ന് തീരുമാനിക്കുവാനായി അവര്‍ തങ്ങളുടെ അമ്പുകളിട്ടു കൊണ്ട് നറുക്കെടുപ്പ് നടത്തിയിരുന്ന സമയത്ത് നീ അവരുടെ അടുത്തുണ്ടായിരുന്നില്ലല്ലോ. അവര്‍ തര്‍ക്കത്തില്‍ ഏര്‍പെട്ടുകൊണ്ടിരുന്നപ്പോഴും നീ അവരുടെ അടുത്തുണ്ടായിരുന്നില്ല.
\end{malayalam}}
\flushright{\begin{Arabic}
\quranayah[3][45]
\end{Arabic}}
\flushleft{\begin{malayalam}
മലക്കുകള്‍ പറഞ്ഞ സന്ദര്‍ഭം ശ്രദ്ധിക്കുക: മര്‍യമേ, തീര്‍ച്ചയായും അല്ലാഹു നിനക്ക് അവന്‍റെ പക്കല്‍ നിന്നുള്ള ഒരു വചനത്തെപ്പറ്റി സന്തോഷവാര്‍ത്ത അറിയിക്കുന്നു. അവന്‍റെ പേര്‍ മര്‍യമിന്‍റെ മകന്‍ മസീഹ് ഈസാ എന്നാകുന്നു. അവന്‍ ഇഹത്തിലും പരത്തിലും മഹത്വമുള്ളവനും സാമീപ്യം സിദ്ധിച്ചവരില്‍ പെട്ടവനുമായിരിക്കും.
\end{malayalam}}
\flushright{\begin{Arabic}
\quranayah[3][46]
\end{Arabic}}
\flushleft{\begin{malayalam}
തൊട്ടിലിലായിരിക്കുമ്പോഴും മദ്ധ്യവയസ്കനായിരിക്കുമ്പോഴും അവന്‍ ജനങ്ങളോട് സംസാരിക്കുന്നതാണ്‌. അവന്‍ സദ്‌വൃത്തരില്‍ പെട്ടവനുമായിരിക്കും.
\end{malayalam}}
\flushright{\begin{Arabic}
\quranayah[3][47]
\end{Arabic}}
\flushleft{\begin{malayalam}
അവള്‍ (മര്‍യം) പറഞ്ഞു: എന്‍റെ രക്ഷിതാവേ, എനിക്ക് എങ്ങനെയാണ് കുട്ടിയുണ്ടാവുക? എന്നെ ഒരു മനുഷ്യനും സ്പര്‍ശിച്ചിട്ടില്ലല്ലോ. അല്ലാഹു പറഞ്ഞു: അങ്ങനെ ത്തന്നെയാകുന്നു. താന്‍ ഉദ്ദേശിക്കുന്നത് അല്ലാഹു സൃഷ്ടിക്കുന്നു. അവന്‍ ഒരു കാര്യം തീരുമാനിച്ചു കഴിഞ്ഞാല്‍ അതിനോട് ഉണ്ടാകൂ എന്ന് പറയുക മാത്രം ചെയ്യുന്നു. അപ്പോള്‍ അതുണ്ടാകുന്നു.
\end{malayalam}}
\flushright{\begin{Arabic}
\quranayah[3][48]
\end{Arabic}}
\flushleft{\begin{malayalam}
അവന് (ഈസാക്ക്‌) അല്ലാഹു ഗ്രന്ഥവും ജ്ഞാനവും തൌറാത്തും ഇന്‍ജീലും പഠിപ്പിക്കുകയും ചെയ്യും.
\end{malayalam}}
\flushright{\begin{Arabic}
\quranayah[3][49]
\end{Arabic}}
\flushleft{\begin{malayalam}
ഇസ്രായീല്‍ സന്തതികളിലേക്ക് (അവനെ) ദൂതനായി നിയോഗിക്കുകയും ചെയ്യും. അവന്‍ അവരോട് പറയും:) നിങ്ങളുടെ രക്ഷിതാവിങ്കല്‍ നിന്നുള്ള ദൃഷ്ടാന്തവും കൊണ്ടാണ് ഞാന്‍ നിങ്ങളുടെ അടുത്ത് വന്നിരിക്കുന്നത്‌. പക്ഷിയുടെ ആകൃതിയില്‍ ഒരു കളിമണ്‍ രൂപം നിങ്ങള്‍ക്കു വേണ്ടി ഞാന്‍ ഉണ്ടാക്കുകയും, എന്നിട്ട് ഞാനതില്‍ ഊതുമ്പോള്‍ അല്ലാഹുവിന്‍റെ അനുവാദപ്രകാരം അതൊരു പക്ഷിയായി തീരുകയും ചെയ്യും. അല്ലാഹുവിന്‍റെ അനുവാദപ്രകാരം ജന്‍മനാ കാഴ്ചയില്ലാത്തവനെയും പാണ്ഡുരോഗിയെയും ഞാന്‍ സുഖപ്പെടുത്തുകയും, മരിച്ചവരെ ഞാന്‍ ജീവിപ്പിക്കുകയും ചെയ്യും. നിങ്ങള്‍ തിന്നുതിനെപ്പറ്റിയും, നിങ്ങള്‍ നിങ്ങളുടെ വീടുകളില്‍ സൂക്ഷിച്ചു വെക്കുന്നതിനെപ്പറ്റിയും ഞാന്‍ നിങ്ങള്‍ക്ക് പറഞ്ഞറിയിച്ചു തരികയും ചെയ്യും. തീര്‍ച്ചയായും അതില്‍ നിങ്ങള്‍ക്ക് ദൃഷ്ടാന്തമുണ്ട്‌; നിങ്ങള്‍ വിശ്വസിക്കുന്നവരാണെങ്കില്‍.
\end{malayalam}}
\flushright{\begin{Arabic}
\quranayah[3][50]
\end{Arabic}}
\flushleft{\begin{malayalam}
എന്‍റെ മുമ്പിലുള്ള തൌറാത്തിനെ സത്യപ്പെടുത്തുന്നവനായിക്കൊണ്ടും നിങ്ങളുടെ മേല്‍ നിഷിദ്ധമാക്കപ്പെട്ട കാര്യങ്ങളില്‍ ചിലത് നിങ്ങള്‍ക്ക് അനുവദിച്ചു തരുവാന്‍ വേണ്ടിയുമാകുന്നു (ഞാന്‍ നിയോഗിക്കപ്പെട്ടിട്ടുള്ളത്‌). നിങ്ങളുടെ രക്ഷിതാവിങ്കല്‍ നിന്നുള്ള ദൃഷ്ടാന്തവും നിങ്ങള്‍ക്ക് ഞാന്‍ കൊണ്ടു വന്നിരിക്കുന്നു. ആകയാല്‍ നിങ്ങള്‍ അല്ലാഹുവെ സൂക്ഷിക്കുകയും എന്നെ അനുസരിക്കുകയും ചെയ്യുവിന്‍.
\end{malayalam}}
\flushright{\begin{Arabic}
\quranayah[3][51]
\end{Arabic}}
\flushleft{\begin{malayalam}
തീര്‍ച്ചയായും അല്ലാഹു എന്‍റെയും നിങ്ങളുടെയും രക്ഷിതാവാകുന്നു. അതിനാല്‍ അവനെ നിങ്ങള്‍ ആരാധിക്കുക. ഇതാകുന്നു നേരായ മാര്‍ഗം.
\end{malayalam}}
\flushright{\begin{Arabic}
\quranayah[3][52]
\end{Arabic}}
\flushleft{\begin{malayalam}
എന്നിട്ട് ഈസായ്ക്ക് അവരുടെ നിഷേധസ്വഭാവം ബോധ്യമായപ്പോള്‍ അദ്ദേഹം പറഞ്ഞു: അല്ലാഹുവിങ്കലേക്ക് എന്‍റെ സഹായികളായി ആരുണ്ട്‌? ഹവാരികള്‍ പറഞ്ഞു: ഞങ്ങള്‍ അല്ലാഹുവിന്‍റെ സഹായികളാകുന്നു. ഞങ്ങള്‍ അല്ലാഹുവില്‍ വിശ്വസിച്ചിരിക്കുന്നു. ഞങ്ങള്‍ (അല്ലാഹുവിന്ന്‌) കീഴ്പെട്ടവരാണ് എന്നതിന് താങ്കള്‍ സാക്ഷ്യം വഹിക്കുകയും ചെയ്യണം.
\end{malayalam}}
\flushright{\begin{Arabic}
\quranayah[3][53]
\end{Arabic}}
\flushleft{\begin{malayalam}
(തുടര്‍ന്ന് അവര്‍ പ്രാര്‍ത്ഥിച്ചു:) ഞങ്ങളുടെ നാഥാ, നീ അവതരിപ്പിച്ചു തന്നതില്‍ ഞങ്ങള്‍ വിശ്വസിക്കുകയും, (നിന്‍റെ) ദൂതനെ ഞങ്ങള്‍ പിന്‍പറ്റുകയും ചെയ്തിരിക്കുന്നു. ആകയാല്‍ സാക്ഷ്യം വഹിച്ചവരോടൊപ്പം ഞങ്ങളെ നീ രേഖപ്പെട്ടുത്തേണമേ.
\end{malayalam}}
\flushright{\begin{Arabic}
\quranayah[3][54]
\end{Arabic}}
\flushleft{\begin{malayalam}
അവര്‍ (സത്യനിഷേധികള്‍) തന്ത്രം പ്രയോഗിച്ചു. അല്ലാഹുവും തന്ത്രം പ്രയോഗിച്ചു. അല്ലാഹു നന്നായി തന്ത്രം പ്രയോഗിക്കുന്നവനാകുന്നു.
\end{malayalam}}
\flushright{\begin{Arabic}
\quranayah[3][55]
\end{Arabic}}
\flushleft{\begin{malayalam}
അല്ലാഹു പറഞ്ഞ സന്ദര്‍ഭം (ശ്രദ്ധിക്കുക:) ഹേ; ഈസാ, തീര്‍ച്ചയായും നിന്നെ നാം പൂര്‍ണ്ണമായി ഏറ്റെടുക്കുകയും, എന്‍റെ അടുക്കലേക്ക് നിന്നെ ഉയര്‍ത്തുകയും, സത്യനിഷേധികളില്‍ നിന്ന് നിന്നെ നാം ശുദ്ധമാക്കുകയും, നിന്നെ പിന്തുടര്‍ന്നവരെ ഉയിര്‍ത്തെഴുന്നേല്‍പിന്‍റെ നാള്‍ വരേക്കും സത്യനിഷേധികളെക്കാള്‍ ഉന്നതന്‍മാരാക്കുകയും ചെയ്യുന്നതാണ്‌. പിന്നെ എന്‍റെ അടുത്തേക്കാണ് നിങ്ങളുടെ മടക്കം. നിങ്ങള്‍ ഭിന്നിച്ചു കൊണ്ടിരിക്കുന്ന കാര്യത്തില്‍ അപ്പോള്‍ ഞാന്‍ നിങ്ങള്‍ക്കിടയില്‍ തീര്‍പ്പുകല്‍പിക്കുന്നതാണ്‌.
\end{malayalam}}
\flushright{\begin{Arabic}
\quranayah[3][56]
\end{Arabic}}
\flushleft{\begin{malayalam}
എന്നാല്‍ (സത്യം) നിഷേധിച്ചവര്‍ക്ക് ഇഹത്തിലും പരത്തിലും ഞാന്‍ കഠിനമായ ശിക്ഷ നല്‍കുന്നതാണ്‌. അവര്‍ക്ക് സഹായികളായി ആരുമുണ്ടായിരിക്കുന്നതല്ല.
\end{malayalam}}
\flushright{\begin{Arabic}
\quranayah[3][57]
\end{Arabic}}
\flushleft{\begin{malayalam}
എന്നാല്‍ വിശ്വസിക്കുകയും സല്‍കര്‍മ്മങ്ങള്‍ പ്രവര്‍ത്തിക്കുകയും ചെയ്തവര്‍ക്ക് അവര്‍ അര്‍ഹിക്കുന്ന പ്രതിഫലം അല്ലാഹു പൂര്‍ണ്ണമായി നല്‍കുന്നതാണ്‌. അക്രമികളെ അല്ലാഹു ഇഷ്ടപെടുകയില്ല.
\end{malayalam}}
\flushright{\begin{Arabic}
\quranayah[3][58]
\end{Arabic}}
\flushleft{\begin{malayalam}
നിനക്ക് നാം ഓതികേള്‍പിക്കുന്ന ആ കാര്യങ്ങള്‍ (അല്ലാഹുവിന്‍റെ) ദൃഷ്ടാന്തങ്ങളിലും യുക്തിമത്തായ ഉല്‍ബോധനത്തിലും പെട്ടതാകുന്നു.
\end{malayalam}}
\flushright{\begin{Arabic}
\quranayah[3][59]
\end{Arabic}}
\flushleft{\begin{malayalam}
അല്ലാഹുവെ സംബന്ധിച്ചിടത്തോളം ഈസായെ ഉപമിക്കാവുന്നത് ആദമിനോടാകുന്നു. അവനെ (അവന്‍റെ രൂപം) മണ്ണില്‍ നിന്നും അവന്‍ സൃഷ്ടിച്ചു. പിന്നീട് അതിനോട് ഉണ്ടാകൂ എന്ന് പറഞ്ഞപ്പോള്‍ അവന്‍ (ആദം) അതാ ഉണ്ടാകുന്നു.
\end{malayalam}}
\flushright{\begin{Arabic}
\quranayah[3][60]
\end{Arabic}}
\flushleft{\begin{malayalam}
സത്യം നിന്‍റെ രക്ഷിതാവിങ്കല്‍ നിന്നുള്ളതാകുന്നു. ആകയാല്‍ നീ സംശയാലുക്കളില്‍ പെട്ടുപോകരുത്‌.
\end{malayalam}}
\flushright{\begin{Arabic}
\quranayah[3][61]
\end{Arabic}}
\flushleft{\begin{malayalam}
ഇനി നിനക്ക് അറിവ് വന്നുകിട്ടിയതിനു ശേഷം അവന്‍റെ (ഈസായുടെ) കാര്യത്തില്‍ നിന്നോട് ആരെങ്കിലും തര്‍ക്കിക്കുകയാണെങ്കില്‍ നീ പറയുക: നിങ്ങള്‍ വരൂ. ഞങ്ങളുടെ മക്കളെയും, നിങ്ങളുടെ മക്കളെയും, ഞങ്ങളുടെ സ്ത്രീകളെയും നിങ്ങളുടെ സ്ത്രീകളെയും നമുക്ക് വിളിച്ചുകൂട്ടാം. ഞങ്ങളും നിങ്ങളും (കൂടുകയും ചെയ്യാം.) എന്നിട്ട് കള്ളം പറയുന്ന കക്ഷിയുടെ മേല്‍ അല്ലാഹുവിന്‍റെ ശാപമുണ്ടായിരിക്കാന്‍ നമുക്ക് ഉള്ളഴിഞ്ഞ് പ്രാര്‍ത്ഥിക്കാം.
\end{malayalam}}
\flushright{\begin{Arabic}
\quranayah[3][62]
\end{Arabic}}
\flushleft{\begin{malayalam}
തീര്‍ച്ചയായും ഇത് യഥാര്‍ത്ഥമായ സംഭവ വിവരണമാകുന്നു. അല്ലാഹുവല്ലാതെ യാതൊരു ദൈവവും ഇല്ല തന്നെ. തീര്‍ച്ചയായും അല്ലാഹു തന്നെയാകുന്നു പ്രതാപവാനും യുക്തിമാനും.
\end{malayalam}}
\flushright{\begin{Arabic}
\quranayah[3][63]
\end{Arabic}}
\flushleft{\begin{malayalam}
എന്നിട്ടവര്‍ പിന്തിരിഞ്ഞുകളയുകയാണെങ്കില്‍ തീര്‍ച്ചയായും അല്ലാഹു കുഴപ്പക്കാരെപ്പറ്റി അറിവുള്ളവനാകുന്നു.
\end{malayalam}}
\flushright{\begin{Arabic}
\quranayah[3][64]
\end{Arabic}}
\flushleft{\begin{malayalam}
(നബിയേ,) പറയുക: വേദക്കാരേ, ഞങ്ങള്‍ക്കും നിങ്ങള്‍ക്കുമിടയില്‍ സമമായുള്ള ഒരു വാക്യത്തിലേക്ക നിങ്ങള്‍ വരുവിന്‍. അതായത് അല്ലാഹുവെയല്ലാതെ നാം ആരാധിക്കാതിരിക്കുകയും, അവനോട് യാതൊന്നിനെയും പങ്കുചേര്‍ക്കാതിരിക്കുകയും നമ്മളില്‍ ചിലര്‍ ചിലരെ അല്ലാഹുവിനു പുറമെ രക്ഷിതാക്കളാക്കാതിരിക്കുകയും ചെയ്യുക (എന്ന തത്വത്തിലേക്ക്‌) . എന്നിട്ട് അവര്‍ പിന്തിരിഞ്ഞുകളയുന്ന പക്ഷം നിങ്ങള്‍ പറയുക: ഞങ്ങള്‍ (അല്ലാഹുവിന്ന്‌) കീഴ്പെട്ടവരാണ് എന്നതിന്ന് നിങ്ങള്‍ സാക്ഷ്യം വഹിച്ചു കൊള്ളുക.
\end{malayalam}}
\flushright{\begin{Arabic}
\quranayah[3][65]
\end{Arabic}}
\flushleft{\begin{malayalam}
വേദക്കാരേ, ഇബ്രാഹീമിന്‍റെ കാര്യത്തില്‍ നിങ്ങളെന്തിനാണ് തര്‍ക്കിക്കുന്നത്‌? തൌറാത്തും ഇന്‍ജീലും അവതരിപ്പിക്കപ്പെട്ടത് അദ്ദേഹത്തിനു ശേഷം മാത്രമാണല്ലോ. നിങ്ങളെന്താണ് ചിന്തിക്കാത്തത്‌?
\end{malayalam}}
\flushright{\begin{Arabic}
\quranayah[3][66]
\end{Arabic}}
\flushleft{\begin{malayalam}
ഹേ; കൂട്ടരേ, നിങ്ങള്‍ക്ക് അറിവുള്ള കാര്യത്തെപ്പറ്റി നിങ്ങള്‍ തര്‍ക്കിച്ചു. ഇനി നിങ്ങള്‍ക്ക് അറിവില്ലാത്ത വിഷയത്തില്‍ നിങ്ങളെന്തിന്ന് തര്‍ക്കിക്കുന്നു? അല്ലാഹു അറിയുന്നു നിങ്ങള്‍ അറിയുന്നില്ല.
\end{malayalam}}
\flushright{\begin{Arabic}
\quranayah[3][67]
\end{Arabic}}
\flushleft{\begin{malayalam}
ഇബ്രാഹീം യഹൂദനോ ക്രിസ്ത്യനോ ആയിരുന്നില്ല. എന്നാല്‍ അദ്ദേഹം ശുദ്ധമനസ്ഥിതിക്കാരനും (അല്ലാഹുവിന്ന്‌) കീഴ്പെട്ടവനും ആയിരുന്നു. അദ്ദേഹം ബഹുദൈവാരാധകരില്‍പെട്ടവനായിരുന്നിട്ടുമില്ല.
\end{malayalam}}
\flushright{\begin{Arabic}
\quranayah[3][68]
\end{Arabic}}
\flushleft{\begin{malayalam}
തീര്‍ച്ചയായും ജനങ്ങളില്‍ ഇബ്രാഹീമിനോട് കൂടുതല്‍ അടുപ്പമുള്ളവര്‍ അദ്ദേഹത്തെ പിന്തുടര്‍ന്നവരും, ഈ പ്രവാചകനും (അദ്ദേഹത്തില്‍) വിശ്വസിച്ചവരുമാകുന്നു. അല്ലാഹു സത്യവിശ്വാസികളുടെ രക്ഷാധികാരിയാകുന്നു.
\end{malayalam}}
\flushright{\begin{Arabic}
\quranayah[3][69]
\end{Arabic}}
\flushleft{\begin{malayalam}
വേദക്കാരില്‍ ഒരു വിഭാഗം, നിങ്ങളെ വഴിതെറ്റിക്കാന്‍ കഴിഞ്ഞിരുന്നെങ്കില്‍ എന്ന് കൊതിക്കുകയാണ്‌. യഥാര്‍ത്ഥത്തില്‍ അവര്‍ വഴിതെറ്റിക്കുന്നത് അവരെത്തന്നെയാണ്‌. അവരത് മനസ്സിലാക്കുന്നില്ല.
\end{malayalam}}
\flushright{\begin{Arabic}
\quranayah[3][70]
\end{Arabic}}
\flushleft{\begin{malayalam}
വേദക്കാരേ, നിങ്ങളെന്തിനാണ് അല്ലാഹുവിന്‍റെ തെളിവുകളില്‍ അവിശ്വസിക്കുന്നത്‌? നിങ്ങള്‍ തന്നെ (അവയ്ക്ക്‌) സാക്ഷ്യം വഹിക്കുന്നവരാണല്ലോ.
\end{malayalam}}
\flushright{\begin{Arabic}
\quranayah[3][71]
\end{Arabic}}
\flushleft{\begin{malayalam}
വേദക്കാരേ, നിങ്ങളെന്തിനാണ് സത്യത്തെ അസത്യവുമായി കൂട്ടികലര്‍ത്തുകയും, അറിഞ്ഞുകൊണ്ട് സത്യം മറച്ചു വെക്കുകയും ചെയ്യുന്നത്‌?
\end{malayalam}}
\flushright{\begin{Arabic}
\quranayah[3][72]
\end{Arabic}}
\flushleft{\begin{malayalam}
വേദക്കാരില്‍ ഒരു വിഭാഗം (സ്വന്തം അനുയായികളോട്‌) പറഞ്ഞു: ഈ വിശ്വാസികള്‍ക്ക് അവതരിപ്പിക്കപ്പെട്ടതില്‍ പകലിന്‍റെ ആരംഭത്തില്‍ നിങ്ങള്‍ വിശ്വസിച്ചുകൊള്ളുക. പകലിന്‍റെ അവസാനത്തില്‍ നിങ്ങളത് അവിശ്വസിക്കുകയും ചെയ്യുക. (അത് കണ്ട്‌) അവര്‍ (വിശ്വാസികള്‍) പിന്‍മാറിയേക്കാം.
\end{malayalam}}
\flushright{\begin{Arabic}
\quranayah[3][73]
\end{Arabic}}
\flushleft{\begin{malayalam}
നിങ്ങളുടെ മതത്തെ പിന്‍പറ്റിയവരെയല്ലാതെ നിങ്ങള്‍ വിശ്വസിച്ചു പോകരുത്‌- (നബിയേ,) പറയുക: (ശരിയായ) മാര്‍ഗദര്‍ശനം അല്ലാഹുവിന്‍റെ മാര്‍ഗദര്‍ശനമത്രെ-(വേദക്കാരായ) നിങ്ങള്‍ക്ക് നല്‍കപ്പെട്ടതു പോലുള്ളത് (വേദഗ്രന്ഥം) മറ്റാര്‍ക്കെങ്കിലും നല്‍കപ്പെടുമെന്നോ നിങ്ങളുടെ രക്ഷിതാവിന്‍റെ അടുക്കല്‍ അവരാരെങ്കിലും നിങ്ങളോട് ന്യായവാദം നടത്തുമെന്നോ (നിങ്ങള്‍ വിശ്വസിക്കരുത് എന്നും ആ വേദക്കാര്‍ പറഞ്ഞു) . (നബിയേ,) പറയുക: തീര്‍ച്ചയായും അനുഗ്രഹം അല്ലാഹുവിന്‍റെ കയ്യിലാകുന്നു. അവന്‍ ഉദ്ദേശിക്കുന്നവര്‍ക്ക് അത് നല്‍കുന്നു. അല്ലാഹു വിപുലമായ കഴിവുള്ളവനും എല്ലാം അറിയുന്നവനുമാകുന്നു.
\end{malayalam}}
\flushright{\begin{Arabic}
\quranayah[3][74]
\end{Arabic}}
\flushleft{\begin{malayalam}
അവന്‍ ഉദ്ദേശിക്കുന്നവരോട് അവന്‍ പ്രത്യേകം കരുണ കാണിക്കുന്നു. അല്ലാഹു മഹത്തായ അനുഗ്രഹം ചെയ്യുന്നവനാകുന്നു.
\end{malayalam}}
\flushright{\begin{Arabic}
\quranayah[3][75]
\end{Arabic}}
\flushleft{\begin{malayalam}
ഒരു സ്വര്‍ണക്കൂമ്പാരം തന്നെ വിശ്വസിച്ചേല്‍പിച്ചാലും അത് നിനക്ക് തിരിച്ചുനല്‍കുന്ന ചിലര്‍ വേദക്കാരിലുണ്ട്‌. അവരില്‍ തന്നെ മറ്റൊരു തരക്കാരുമുണ്ട്‌. അവരെ ഒരു ദീനാര്‍ നീ വിശ്വസിച്ചേല്‍പിച്ചാല്‍ പോലും നിരന്തരം (ചോദിച്ചു കൊണ്ട്‌) നിന്നെങ്കിലല്ലാതെ അവരത് നിനക്ക് തിരിച്ചുതരികയില്ല. അക്ഷരജ്ഞാനമില്ലാത്ത ആളുകളുടെ കാര്യത്തില്‍ (അവരെ വഞ്ചിക്കുന്നതില്‍) ഞങ്ങള്‍ക്ക് കുറ്റമുണ്ടാകാന്‍ വഴിയില്ലെന്ന് അവര്‍ പറഞ്ഞതിനാലത്രെ അത്‌. അവര്‍ അല്ലാഹുവിന്‍റെ പേരില്‍ അറിഞ്ഞ് കൊണ്ട് കള്ളം പറയുകയാകുന്നു.
\end{malayalam}}
\flushright{\begin{Arabic}
\quranayah[3][76]
\end{Arabic}}
\flushleft{\begin{malayalam}
അല്ല, വല്ലവനും തന്‍റെ കരാര്‍ നിറവേറ്റുകയും ധര്‍മ്മനിഷ്ഠപാലിക്കുകയും ചെയ്യുന്ന പക്ഷം തീര്‍ച്ചയായും അല്ലാഹു ധര്‍മ്മനിഷ്ഠപാലിക്കുന്നവരെ ഇഷ്ടപ്പെടുന്നു.
\end{malayalam}}
\flushright{\begin{Arabic}
\quranayah[3][77]
\end{Arabic}}
\flushleft{\begin{malayalam}
അല്ലാഹുവോടുള്ള കരാറും സ്വന്തം ശപഥങ്ങളും തുച്ഛവിലയ്ക്ക് വില്‍ക്കുന്നവരാരോ അവര്‍ക്ക് പരലോകത്തില്‍ യാതൊരു ഓഹരിയുമില്ല. ഉയിര്‍ത്തെഴുന്നേല്‍പിന്‍റെ നാളില്‍ അല്ലാഹു അവരോട് സംസാരിക്കുകയോ, അവരുടെ നേര്‍ക്ക് (കാരുണ്യപൂര്‍വ്വം) നോക്കുകയോ ചെയ്യുന്നതല്ല. അവന്‍ അവര്‍ക്ക് വിശുദ്ധി നല്‍കുന്നതുമല്ല. അവര്‍ക്ക് വേദനയേറിയ ശിക്ഷയുണ്ടായിരിക്കുന്നതുമാണ്‌.
\end{malayalam}}
\flushright{\begin{Arabic}
\quranayah[3][78]
\end{Arabic}}
\flushleft{\begin{malayalam}
വേദഗ്രന്ഥത്തിലെ വാചകശൈലികള്‍ വളച്ചൊടിക്കുന്ന ചിലരും അവരുടെ കൂട്ടത്തിലുണ്ട്‌. അത് വേദഗ്രന്ഥത്തില്‍ പെട്ടതാണെന്ന് നിങ്ങള്‍ ധരിക്കുവാന്‍ വേണ്ടിയാണത്‌. അത് വേദഗ്രന്ഥത്തിലുള്ളതല്ല. അവര്‍ പറയും; അത് അല്ലാഹുവിന്‍റെ പക്കല്‍ നിന്നുള്ളതാണെന്ന്‌. എന്നാല്‍ അത് അല്ലാഹുവിങ്കല്‍ നിന്നുള്ളതല്ല. അവര്‍ അറിഞ്ഞുകൊണ്ട് അല്ലാഹുവിന്‍റെ പേരില്‍ കള്ളം പറയുകയാണ്‌.
\end{malayalam}}
\flushright{\begin{Arabic}
\quranayah[3][79]
\end{Arabic}}
\flushleft{\begin{malayalam}
അല്ലാഹു ഒരു മനുഷ്യന് വേദവും മതവിജ്ഞാനവും പ്രവാചകത്വവും നല്‍കുകയും, എന്നിട്ട് അദ്ദേഹം ജനങ്ങളോട് നിങ്ങള്‍ അല്ലാഹുവെ വിട്ട് എന്‍റെ ദാസന്‍മാരായിരിക്കുവിന്‍ എന്ന് പറയുകയും ചെയ്യുക എന്നത് ഉണ്ടാകാവുന്നതല്ല. എന്നാല്‍ നിങ്ങള്‍ വേദഗ്രന്ഥം പഠിപ്പിച്ചുകൊണ്ടിരിക്കുന്നതിലൂടെയും, പഠിച്ച് കൊണ്ടിരിക്കുന്നതിലൂടെയും ദൈവത്തിന്‍റെ നിഷ്കളങ്ക ദാസന്‍മാരായിരിക്കണം (എന്നായിരിക്കും അദ്ദേഹം പറയുന്നത്‌.)
\end{malayalam}}
\flushright{\begin{Arabic}
\quranayah[3][80]
\end{Arabic}}
\flushleft{\begin{malayalam}
മലക്കുകളെയും പ്രവാചകന്‍മാരെയും നിങ്ങള്‍ രക്ഷിതാക്കളായി സ്വീകരിക്കണമെന്ന് അദ്ദേഹം നിങ്ങളോട് കല്‍പക്കുകയുമില്ല. നിങ്ങള്‍ മുസ്ലിംകളായിക്കഴിഞ്ഞതിന് ശേഷം അവിശ്വാസം സ്വീകരിക്കാന്‍ അദ്ദേഹം നിങ്ങളോട് കല്‍പിക്കുമെന്നാണോ (നിങ്ങള്‍ കരുതുന്നത്‌?)
\end{malayalam}}
\flushright{\begin{Arabic}
\quranayah[3][81]
\end{Arabic}}
\flushleft{\begin{malayalam}
അല്ലാഹു പ്രവാചകന്‍മാരോട് കരാര്‍ വാങ്ങിയ സന്ദര്‍ഭം (ശ്രദ്ധിക്കുക) : ഞാന്‍ നിങ്ങള്‍ക്ക് വേദഗ്രന്ഥവും വിജ്ഞാനവും നല്‍കുകയും, അനന്തരം നിങ്ങളുടെ പക്കലുള്ളതിനെ ശരിവെച്ചുകൊണ്ട് ഒരു ദൂതന്‍ നിങ്ങളുടെ അടുത്ത് വരികയുമാണെങ്കില്‍ തീര്‍ച്ചയായും നിങ്ങള്‍ അദ്ദേഹത്തില്‍ വിശ്വസിക്കുകയും, അദ്ദേഹത്തെ സഹായിക്കുകയും ചെയ്യേണ്ടതാണ് എന്ന്‌. (തുടര്‍ന്ന്‌) അവന്‍ (അവരോട്‌) ചോദിച്ചു: നിങ്ങളത് സമ്മതിക്കുകയും അക്കാര്യത്തില്‍ എന്നോടുള്ള ബാധ്യത ഏറ്റെടുക്കുകയും ചെയ്തുവോ? അവര്‍ പറഞ്ഞു: അതെ, ഞങ്ങള്‍ സമ്മതിച്ചിരിക്കുന്നു. അവന്‍ പറഞ്ഞു: എങ്കില്‍ നിങ്ങള്‍ അതിന് സാക്ഷികളായിരിക്കുക. ഞാനും നിങ്ങളോടൊപ്പം സാക്ഷിയായിരിക്കുന്നതാണ്‌.
\end{malayalam}}
\flushright{\begin{Arabic}
\quranayah[3][82]
\end{Arabic}}
\flushleft{\begin{malayalam}
അതിന് ശേഷവും ആരെങ്കിലും പിന്തിരിയുകയാണെങ്കില്‍ അവര്‍ തന്നെയാകുന്നു ധിക്കാരികള്‍.
\end{malayalam}}
\flushright{\begin{Arabic}
\quranayah[3][83]
\end{Arabic}}
\flushleft{\begin{malayalam}
അപ്പോള്‍ അല്ലാഹുവിന്‍റെ മതമല്ലാത്ത മറ്റു വല്ല മതവുമാണോ അവര്‍ ആഗ്രഹിക്കുന്നത്‌? (വാസ്തവത്തില്‍) ആകാശങ്ങളിലും ഭൂമിയിലും ഉള്ളവരെല്ലാം അനുസരണയോടെയോ നിര്‍ബന്ധിതമായോ അവന്ന് കീഴ്പെട്ടിരിക്കുകയാണ്‌. അവനിലേക്ക് തന്നെയാണ് അവര്‍ മടക്കപ്പെടുന്നതും.
\end{malayalam}}
\flushright{\begin{Arabic}
\quranayah[3][84]
\end{Arabic}}
\flushleft{\begin{malayalam}
(നബിയേ,) പറയുക: അല്ലാഹുവിലും ഞങ്ങള്‍ക്ക് അവതരിപ്പിക്കപ്പെട്ടതി (ഖുര്‍ആന്‍) ലും, ഇബ്രാഹീം, ഇസ്മാഈല്‍, ഇഷാഖ്‌, യഅ്ഖൂബ്‌, യഅ്ഖൂബ് സന്തതികള്‍ എന്നിവര്‍ക്ക് അവതരിപ്പിക്കപ്പെട്ട (ദിവ്യസന്ദേശം) തിലും, മൂസായ്ക്കും ഈസായ്ക്കും മറ്റു പ്രവാചകന്‍മാര്‍ക്കും തങ്ങളുടെ രക്ഷിതാവിങ്കല്‍ നിന്ന് നല്‍കപ്പെട്ടതിലും ഞങ്ങള്‍ വിശ്വസിച്ചിരിക്കുന്നു. അവരില്‍ ആര്‍ക്കിടയിലും ഞങ്ങള്‍ വിവേചനം കല്‍പിക്കുന്നില്ല. ഞങ്ങള്‍ അല്ലാഹുവിന് കീഴ്പെട്ടവരാകുന്നു.
\end{malayalam}}
\flushright{\begin{Arabic}
\quranayah[3][85]
\end{Arabic}}
\flushleft{\begin{malayalam}
ഇസ്ലാം (ദൈവത്തിനുള്ള ആത്മാര്‍പ്പണം) അല്ലാത്തതിനെ ആരെങ്കിലും മതമായി ആഗ്രഹിക്കുന്ന പക്ഷം അത് അവനില്‍ നിന്ന് ഒരിക്കലും സ്വീകരിക്കപ്പെടുന്നതല്ല. പരലോകത്തില്‍ അവന്‍ നഷ്ടക്കാരില്‍ പെട്ടവനുമായിരിക്കും.
\end{malayalam}}
\flushright{\begin{Arabic}
\quranayah[3][86]
\end{Arabic}}
\flushleft{\begin{malayalam}
വിശ്വാസത്തിന് ശേഷം അവിശ്വാസം സ്വീകരിച്ച ഒരു ജനതയെ അല്ലാഹു എങ്ങനെ നേര്‍വഴിയിലാക്കും? അവരാകട്ടെ ദൈവദൂതന്‍ സത്യവാനാണെന്ന് സാക്ഷ്യം വഹിച്ചിട്ടുണ്ട്‌. അവര്‍ക്ക് വ്യക്തമായ തെളിവുകള്‍ വന്നുകിട്ടിയിട്ടുമുണ്ട്‌. അക്രമികളായ ആ ജനവിഭാഗത്തെ അല്ലാഹു നേര്‍വഴിയിലാക്കുന്നതല്ല.
\end{malayalam}}
\flushright{\begin{Arabic}
\quranayah[3][87]
\end{Arabic}}
\flushleft{\begin{malayalam}
അല്ലാഹുവിന്‍റെയും മലക്കുകളുടെയും മനുഷ്യരുടെയും എല്ലാം ശാപം അവരുടെ മേലുണ്ടായിരിക്കുക എന്നതത്രെ അവര്‍ക്കുള്ള പ്രതിഫലം.
\end{malayalam}}
\flushright{\begin{Arabic}
\quranayah[3][88]
\end{Arabic}}
\flushleft{\begin{malayalam}
അവര്‍ അതില്‍ (ശാപഫലമായ ശിക്ഷയില്‍) സ്ഥിരവാസികളായിരിക്കുന്നതാണ്‌. അവര്‍ക്ക് ശിക്ഷ ലഘൂകരിക്കപ്പെടുന്നതല്ല. അവര്‍ക്ക് അവധി നല്‍കപ്പെടുകയുമില്ല.
\end{malayalam}}
\flushright{\begin{Arabic}
\quranayah[3][89]
\end{Arabic}}
\flushleft{\begin{malayalam}
അതിന് (അവിശ്വാസത്തിനു) ശേഷം പശ്ചാത്തപിക്കുകയും, ജീവിതം നന്നാക്കിത്തീര്‍ക്കുകയും ചെയ്തവരൊഴികെ. അപ്പോള്‍ അല്ലാഹു ഏറെ പൊറുക്കുന്നവനും അത്യധികം കരുണ കാണിക്കുന്നവനുമാകുന്നു.
\end{malayalam}}
\flushright{\begin{Arabic}
\quranayah[3][90]
\end{Arabic}}
\flushleft{\begin{malayalam}
വിശ്വസിച്ചതിന് ശേഷം അവിശ്വാസികളായി മാറുകയും, അവിശ്വാസം കൂടിക്കൂടി വരികയും ചെയ്ത വിഭാഗത്തിന്‍റെ പശ്ചാത്താപം ഒരിക്കലും സ്വീകരിക്കപ്പെടുകയില്ല. അവരത്രെ വഴിപിഴച്ചവര്‍.
\end{malayalam}}
\flushright{\begin{Arabic}
\quranayah[3][91]
\end{Arabic}}
\flushleft{\begin{malayalam}
അവിശ്വസിക്കുകയും അവിശ്വാസികളായിക്കൊണ്ട് മരിക്കുകയും ചെയ്തവരില്‍പെട്ട ഒരാള്‍ ഭൂമി നിറയെ സ്വര്‍ണം പ്രായശ്ചിത്തമായി നല്‍കിയാല്‍ പോലും അത് സ്വീകരിക്കപ്പെടുന്നതല്ല. അവര്‍ക്കാണ് വേദനയേറിയ ശിക്ഷയുള്ളത്‌. അവര്‍ക്ക് സഹായികളായി ആരുമുണ്ടായിരിക്കുന്നതുമല്ല.
\end{malayalam}}
\flushright{\begin{Arabic}
\quranayah[3][92]
\end{Arabic}}
\flushleft{\begin{malayalam}
നിങ്ങള്‍ ഇഷ്ടപ്പെടുന്നതില്‍ നിന്ന് നിങ്ങള്‍ ചെലവഴിക്കുന്നത് വരെ നിങ്ങള്‍ക്ക് പുണ്യം നേടാനാവില്ല. നിങ്ങള്‍ ഏതൊരു വസ്തു ചെലവഴിക്കുന്നതായാലും തീര്‍ച്ചയായും അല്ലാഹു അതിനെപ്പറ്റി അറിയുന്നവനാകുന്നു.
\end{malayalam}}
\flushright{\begin{Arabic}
\quranayah[3][93]
\end{Arabic}}
\flushleft{\begin{malayalam}
എല്ലാ ആഹാരപദാര്‍ത്ഥവും ഇസ്രായീല്‍ സന്തതികള്‍ക്ക് അനുവദനീയമായിരുന്നു. തൌറാത്ത് അവതരിപ്പിക്കപ്പെടുന്നതിന് മുമ്പായി ഇസ്രായീല്‍ (യഅ്ഖൂബ് നബി) തന്‍റെ കാര്യത്തില്‍ നിഷിദ്ധമാക്കിയതൊഴികെ. (നബിയേ,) പറയുക: നിങ്ങള്‍ സത്യവാന്‍മാരാണെങ്കില്‍ തൌറാത്ത് കൊണ്ടുവന്നു അതൊന്ന് വായിച്ചുകേള്‍പിക്കുക.
\end{malayalam}}
\flushright{\begin{Arabic}
\quranayah[3][94]
\end{Arabic}}
\flushleft{\begin{malayalam}
എന്നിട്ട് അതിനു ശേഷവും അല്ലാഹുവിന്‍റെ പേരില്‍ കള്ളം കെട്ടിച്ചമച്ചവരാരോ അവര്‍ തന്നെയാകുന്നു അക്രമികള്‍.
\end{malayalam}}
\flushright{\begin{Arabic}
\quranayah[3][95]
\end{Arabic}}
\flushleft{\begin{malayalam}
(നബിയേ,) പറയുക: അല്ലാഹു സത്യം പറഞ്ഞിരിക്കുന്നു. ആകയാല്‍ ശുദ്ധമനസ്കനായ ഇബ്രാഹീമിന്‍റെ മാര്‍ഗം നിങ്ങള്‍ പിന്തുടരുക. അദ്ദേഹം ബഹുദൈവാരാധകരുടെ കൂട്ടത്തിലായിരുന്നില്ല.
\end{malayalam}}
\flushright{\begin{Arabic}
\quranayah[3][96]
\end{Arabic}}
\flushleft{\begin{malayalam}
തീര്‍ച്ചയായും മനുഷ്യര്‍ക്ക് വേണ്ടി സ്ഥാപിക്കപ്പെട്ട ഒന്നാമത്തെ ആരാധനാ മന്ദിരം ബക്കയില്‍ ഉള്ളതത്രെ. (അത്‌) അനുഗൃഹീതമായും ലോകര്‍ക്ക് മാര്‍ഗദര്‍ശകമായും (നിലകൊള്ളുന്നു.)
\end{malayalam}}
\flushright{\begin{Arabic}
\quranayah[3][97]
\end{Arabic}}
\flushleft{\begin{malayalam}
അതില്‍ വ്യക്തമായ ദൃഷ്ടാന്തങ്ങള്‍- (വിശിഷ്യാ) ഇബ്രാഹീം നിന്ന സ്ഥലം -ഉണ്ട്‌. ആര്‍ അവിടെ പ്രവേശിക്കുന്നുവോ അവന്‍ നിര്‍ഭയനായിരിക്കുന്നതാണ്‌. ആ മന്ദിരത്തില്‍ എത്തിച്ചേരാന്‍ കഴിവുള്ള മനുഷ്യര്‍ അതിലേക്ക് ഹജ്ജ് തീര്‍ത്ഥാടനം നടത്തല്‍ അവര്‍ക്ക് അല്ലാഹുവോടുള്ള ബാധ്യതയാകുന്നു. വല്ലവനും അവിശ്വസിക്കുന്ന പക്ഷം അല്ലാഹു ലോകരെ ആശ്രയിക്കാത്തവനാകുന്നു.
\end{malayalam}}
\flushright{\begin{Arabic}
\quranayah[3][98]
\end{Arabic}}
\flushleft{\begin{malayalam}
(നബിയേ,) പറയുക: വേദക്കാരേ, നിങ്ങളെന്തിനാണ് അല്ലാഹുവിന്‍റെ വചനങ്ങളെ നിഷേധിക്കുന്നത്‌? നിങ്ങള്‍ പ്രവര്‍ത്തിക്കുന്നതിനെല്ലാം അല്ലാഹു സാക്ഷിയാകുന്നു.
\end{malayalam}}
\flushright{\begin{Arabic}
\quranayah[3][99]
\end{Arabic}}
\flushleft{\begin{malayalam}
(നബിയേ,) പറയുക: വേദക്കാരേ, അല്ലാഹുവിന്‍റെ മാര്‍ഗത്തില്‍ നിന്ന്‌- അതിനെ വളച്ചൊടിക്കാന്‍ ശ്രമിച്ചു കൊണ്ട്‌-നിങ്ങളെന്തിന് വിശ്വാസികളെ പിന്തിരിപ്പിച്ചുകളയുന്നു? (ആ മാര്‍ഗം ശരിയാണെന്നതിന്‌) നിങ്ങള്‍ തന്നെ സാക്ഷികളാണല്ലോ. നിങ്ങള്‍ പ്രവര്‍ത്തിച്ച് കൊണ്ടിരിക്കുന്നതിനെപ്പറ്റിയൊന്നും അല്ലാഹു അശ്രദ്ധനല്ല.
\end{malayalam}}
\flushright{\begin{Arabic}
\quranayah[3][100]
\end{Arabic}}
\flushleft{\begin{malayalam}
സത്യവിശ്വാസികളേ, വേദഗ്രന്ഥം നല്‍കപ്പെട്ടവരില്‍ ഒരു വിഭാഗത്തെ നിങ്ങള്‍ അനുസരിക്കുന്ന പക്ഷം നിങ്ങള്‍ വിശ്വാസം സ്വീകരിച്ചതിന് ശേഷം അവര്‍ നിങ്ങളെ അവിശ്വാസികളായി മാറ്റിയേക്കും.
\end{malayalam}}
\flushright{\begin{Arabic}
\quranayah[3][101]
\end{Arabic}}
\flushleft{\begin{malayalam}
നിങ്ങള്‍ക്ക് അല്ലാഹുവിന്‍റെ വചനങ്ങള്‍ വായിച്ചുകേള്‍പിക്കപ്പെട്ടുകൊണ്ടിരിക്കെ, നിങ്ങള്‍ക്കിടയില്‍ അവന്‍റെ ദൂതനുണ്ടായിരിക്കെ നിങ്ങളെങ്ങനെ അവിശ്വാസികളാകും? ആര്‍ അല്ലാഹുവെ മുറുകെപിടിക്കുന്നുവോ അവന്‍ നേര്‍മാര്‍ഗത്തിലേക്ക് നയിക്കപ്പെട്ടിരിക്കുന്നു.
\end{malayalam}}
\flushright{\begin{Arabic}
\quranayah[3][102]
\end{Arabic}}
\flushleft{\begin{malayalam}
സത്യവിശ്വാസികളേ, നിങ്ങള്‍ അല്ലാഹുവെ സൂക്ഷിക്കേണ്ട മുറപ്രകാരം സൂക്ഷിക്കുക. നിങ്ങള്‍ മുസ്ലിംകളായിക്കൊണ്ടല്ലാതെ മരിക്കാനിടയാകരുത്‌.
\end{malayalam}}
\flushright{\begin{Arabic}
\quranayah[3][103]
\end{Arabic}}
\flushleft{\begin{malayalam}
നിങ്ങളൊന്നിച്ച് അല്ലാഹുവിന്‍റെ കയറില്‍ മുറുകെപിടിക്കുക. നിങ്ങള്‍ ഭിന്നിച്ച് പോകരുത്‌. നിങ്ങള്‍ അന്യോന്യം ശത്രുക്കളായിരുന്നപ്പോള്‍ നിങ്ങള്‍ക്ക് അല്ലാഹു ചെയ്ത അനുഗ്രഹം ഓര്‍ക്കുകയും ചെയ്യുക. അവന്‍ നിങ്ങളുടെ മനസ്സുകള്‍ തമ്മില്‍ കൂട്ടിയിണക്കി. അങ്ങനെ അവന്‍റെ അനുഗ്രഹത്താല്‍ നിങ്ങള്‍ സഹോദരങ്ങളായിത്തീര്‍ന്നു. നിങ്ങള്‍ അഗ്നികുണ്ഡത്തിന്‍റെ വക്കിലായിരുന്നു. എന്നിട്ടതില്‍ നിന്ന് നിങ്ങളെ അവന്‍ രക്ഷപ്പെടുത്തി. അപ്രകാരം അല്ലാഹു അവന്‍റെ ദൃഷ്ടാന്തങ്ങള്‍ നിങ്ങള്‍ക്ക് വിവരിച്ചുതരുന്നു; നിങ്ങള്‍ നേര്‍മാര്‍ഗം പ്രാപിക്കുവാന്‍ വേണ്ടി.
\end{malayalam}}
\flushright{\begin{Arabic}
\quranayah[3][104]
\end{Arabic}}
\flushleft{\begin{malayalam}
നന്‍മയിലേക്ക് ക്ഷണിക്കുകയും, സദാചാരം കല്‍പിക്കുകയും, ദുരാചാരത്തില്‍ നിന്ന് വിലക്കുകയും ചെയ്യുന്ന ഒരു സമുദായം നിങ്ങളില്‍ നിന്ന് ഉണ്ടായിരിക്കട്ടെ. അവരത്രെ വിജയികള്‍.
\end{malayalam}}
\flushright{\begin{Arabic}
\quranayah[3][105]
\end{Arabic}}
\flushleft{\begin{malayalam}
വ്യക്തമായ തെളിവുകള്‍ വന്നുകിട്ടിയതിന് ശേഷം പല കക്ഷികളായി പിരിഞ്ഞ് ഭിന്നിച്ചവരെപ്പോലെ നിങ്ങളാകരുത്‌. അവര്‍ക്കാണ് കനത്ത ശിക്ഷയുള്ളത്‌.
\end{malayalam}}
\flushright{\begin{Arabic}
\quranayah[3][106]
\end{Arabic}}
\flushleft{\begin{malayalam}
ചില മുഖങ്ങള്‍ വെളുക്കുകയും, ചില മുഖങ്ങള്‍ കറുക്കുകയും ചെയ്യുന്ന ഒരു ദിവസത്തില്‍. എന്നാല്‍ മുഖങ്ങള്‍ കറുത്തു പോയവരോട് പറയപ്പെടും: വിശ്വാസം സ്വീകരിച്ചതിന് ശേഷം നിങ്ങള്‍ അവിശ്വസിക്കുകയാണോ ചെയ്തത്‌? എങ്കില്‍ നിങ്ങള്‍ അവിശ്വാസം സ്വീകരിച്ചതിന്‍റെ ഫലമായി ശിക്ഷ അനുഭവിച്ചു കൊള്ളുക.
\end{malayalam}}
\flushright{\begin{Arabic}
\quranayah[3][107]
\end{Arabic}}
\flushleft{\begin{malayalam}
എന്നാല്‍ മുഖങ്ങള്‍ വെളുത്തു തെളിഞ്ഞവര്‍ അല്ലാഹുവിന്‍റെ കാരുണ്യത്തിലായിരിക്കും. അവരതില്‍ സ്ഥിരവാസികളായിരിക്കുന്നതാണ്‌.
\end{malayalam}}
\flushright{\begin{Arabic}
\quranayah[3][108]
\end{Arabic}}
\flushleft{\begin{malayalam}
അല്ലാഹുവിന്‍റെ ദൃഷ്ടാന്തങ്ങളത്രെ അവ. സത്യപ്രകാരം നാം അവ നിനക്ക് ഓതികേള്‍പിച്ചു തരുന്നു. ലോകരോട് ഒരു അനീതിയും കാണിക്കാന്‍ അല്ലാഹു ഉദ്ദേശിക്കുന്നില്ല.
\end{malayalam}}
\flushright{\begin{Arabic}
\quranayah[3][109]
\end{Arabic}}
\flushleft{\begin{malayalam}
അല്ലാഹുവിന്‍റെതാകുന്നു ആകാശങ്ങളിലുള്ളതും ഭൂമിയിലുള്ളതും. അല്ലാഹുവിങ്കലേക്കാകുന്നു കാര്യങ്ങള്‍ മടക്കപ്പെടുന്നത്‌.
\end{malayalam}}
\flushright{\begin{Arabic}
\quranayah[3][110]
\end{Arabic}}
\flushleft{\begin{malayalam}
മനുഷ്യവംശത്തിനു വേണ്ടി രംഗത്ത് കൊണ്ടുവരപ്പെട്ട ഉത്തമസമുദായമാകുന്നു നിങ്ങള്‍. നിങ്ങള്‍ സദാചാരം കല്‍പിക്കുകയും, ദുരാചാരത്തില്‍ നിന്ന് വിലക്കുകയും, അല്ലാഹുവില്‍ വിശ്വസിക്കുകയും ചെയ്യുന്നു. വേദക്കാര്‍ വിശ്വസിച്ചിരുന്നുവെങ്കില്‍ അതവര്‍ക്ക് ഉത്തമമായിരുന്നു. അവരുടെ കൂട്ടത്തില്‍ വിശ്വാസമുള്ളവരുണ്ട്‌. എന്നാല്‍ അവരില്‍ അധികപേരും ധിക്കാരികളാകുന്നു.
\end{malayalam}}
\flushright{\begin{Arabic}
\quranayah[3][111]
\end{Arabic}}
\flushleft{\begin{malayalam}
ചില്ലറ ശല്യമല്ലാതെ നിങ്ങള്‍ക്ക് ഒരു ഉപദ്രവവും വരുത്താന്‍ അവര്‍ക്കാവില്ല. ഇനി അവര്‍ നിങ്ങളോട് യുദ്ധത്തില്‍ ഏര്‍പെടുകയാണെങ്കില്‍ തന്നെ അവര്‍ പിന്തിരിഞ്ഞോടുന്നതാണ്‌. പിന്നീടവര്‍ക്ക് സഹായം ലഭിക്കുകയുമില്ല.
\end{malayalam}}
\flushright{\begin{Arabic}
\quranayah[3][112]
\end{Arabic}}
\flushleft{\begin{malayalam}
നിന്ദ്യത അവരില്‍ അടിച്ചേല്‍പിക്കപ്പെട്ടിരിക്കുന്നു; അവര്‍ എവിടെ കാണപ്പെട്ടാലും. അല്ലാഹുവില്‍ നിന്നുള്ള പിടികയറോ, ജനങ്ങളില്‍ നിന്നുള്ള പിടികയറോ മുഖേനയല്ലാതെ (അവര്‍ക്ക് അതില്‍ നിന്ന് മോചനമില്ല.) അവര്‍ അല്ലാഹുവിന്‍റെ കോപത്തിന് പാത്രമാകുകയും, അവരുടെ മേല്‍ അധമത്വം അടിച്ചേല്‍പിക്കപ്പെടുകയും ചെയ്തിരിക്കുന്നു. അവര്‍ അല്ലാഹുവിന്‍റെ ദൃഷ്ടാന്തങ്ങള്‍ തള്ളിക്കളയുകയും, അന്യായമായി പ്രവാചകന്‍മാരെ കൊലപ്പെടുത്തുകയും ചെയ്തുകൊണ്ടിരുന്നതിന്‍റെ ഫലമത്രെ അത്‌. അവര്‍ അനുസരണക്കേട് കാണിക്കുകയും, അതിക്രമം പ്രവര്‍ത്തിച്ചുകൊണ്ടിരിക്കുകയും ചെയ്തതിന്‍റെ ഫലമത്രെ അത്‌.
\end{malayalam}}
\flushright{\begin{Arabic}
\quranayah[3][113]
\end{Arabic}}
\flushleft{\begin{malayalam}
അവരെല്ലാം ഒരുപോലെയല്ല. നേര്‍മാര്‍ഗത്തില്‍ നിലകൊള്ളുന്ന ഒരു സമൂഹവും വേദക്കാരിലുണ്ട്‌. രാത്രി സമയങ്ങളില്‍ സുജൂദില്‍ (അഥവാ നമസ്കാരത്തില്‍) ഏര്‍പെട്ടുകൊണ്ട് അവര്‍ അല്ലാഹുവിന്‍റെ വചനങ്ങള്‍ പാരായണം ചെയ്യുന്നു.
\end{malayalam}}
\flushright{\begin{Arabic}
\quranayah[3][114]
\end{Arabic}}
\flushleft{\begin{malayalam}
അവര്‍ അല്ലാഹുവിലും അന്ത്യദിനത്തിലും വിശ്വസിക്കുകയും, സദാചാരം കല്‍പിക്കുകയും. ദുരാചാരത്തില്‍ നിന്ന് വിലക്കുകയും, നല്ല കാര്യങ്ങളില്‍ അത്യുത്സാഹം കാണിക്കുകയും ചെയ്യും. അവര്‍ സജ്ജനങ്ങളില്‍ പെട്ടവരാകുന്നു.
\end{malayalam}}
\flushright{\begin{Arabic}
\quranayah[3][115]
\end{Arabic}}
\flushleft{\begin{malayalam}
അവര്‍ ഏതൊരു നല്ലകാര്യം ചെയ്താലും അതിന്‍റെ പ്രതിഫലം അവര്‍ക്ക് നിഷേധിക്കപ്പെടുന്നതല്ല. അല്ലാഹു സൂക്ഷ്മത പാലിക്കുന്നവരെപ്പറ്റി അറിവുള്ളവാനാകുന്നു.
\end{malayalam}}
\flushright{\begin{Arabic}
\quranayah[3][116]
\end{Arabic}}
\flushleft{\begin{malayalam}
സത്യനിഷേധികളെ സംബന്ധിച്ചിടത്തോളം അവരുടെ സ്വത്തുകളോ സന്താനങ്ങളോ അല്ലാഹുവിന്‍റെ ശിക്ഷയില്‍ നിന്ന് അവര്‍ക്ക് ഒട്ടും രക്ഷനേടികൊടുക്കുന്നതല്ല. അവരാണ് നരകാവകാശികള്‍. അവരതില്‍ നിത്യവാസികളായിരിക്കും.
\end{malayalam}}
\flushright{\begin{Arabic}
\quranayah[3][117]
\end{Arabic}}
\flushleft{\begin{malayalam}
ഈ ഐഹികജീവിതത്തില്‍ അവര്‍ ചെലവഴിക്കുന്നതിനെ ഉപമിക്കാവുന്നത് ആത്മദ്രോഹികളായ ഒരു ജനവിഭാഗത്തിന്‍റെ കൃഷിയിടത്തില്‍ ആഞ്ഞുവീശി അതിനെ നശിപ്പിച്ച് കളഞ്ഞ ഒരു ശീതകാറ്റിനോടാകുന്നു. അല്ലാഹു അവരോട് ദ്രോഹം കാണിച്ചിട്ടില്ല. പക്ഷെ, അവര്‍ സ്വന്തത്തോട് തന്നെ ദ്രോഹം ചെയ്യുകയായിരുന്നു.
\end{malayalam}}
\flushright{\begin{Arabic}
\quranayah[3][118]
\end{Arabic}}
\flushleft{\begin{malayalam}
സത്യവിശ്വാസികളേ, നിങ്ങള്‍ക്ക് പുറമെയുള്ളവരില്‍ നിന്ന് നിങ്ങള്‍ ഉള്ളുകള്ളിക്കാരെ സ്വീകരിക്കരുത്‌. നിങ്ങള്‍ക്ക് അനര്‍ത്ഥമുണ്ടാക്കുന്ന കാര്യത്തില്‍ അവര്‍ ഒരു വീഴ്ചയും വരുത്തുകയില്ല. നിങ്ങള്‍ ബുദ്ധിമുട്ടുന്നതാണ് അവര്‍ക്ക് ഇഷ്ടം. വിദ്വേഷം അവരുടെ വായില്‍ നിന്ന് വെളിപ്പെട്ടിരിക്കുന്നു. അവരുടെ മനസ്സുകള്‍ ഒളിച്ച് വെക്കുന്നത് കൂടുതല്‍ ഗുരുതരമാകുന്നു. നിങ്ങള്‍ക്കിതാ നാം തെളിവുകള്‍ വിവരിച്ചുതന്നിരിക്കുന്നു; നിങ്ങള്‍ ചിന്തിക്കുന്നവരാണെങ്കില്‍.
\end{malayalam}}
\flushright{\begin{Arabic}
\quranayah[3][119]
\end{Arabic}}
\flushleft{\begin{malayalam}
നോക്കൂ; നിങ്ങളുടെ സ്ഥിതി. നിങ്ങളവരെ സ്നേഹിക്കുന്നു. അവര്‍ നിങ്ങളെ സ്നേഹിക്കുന്നില്ല. നിങ്ങള്‍ എല്ലാ വേദഗ്രന്ഥങ്ങളിലും വിശ്വസിക്കുകയും ചെയ്യുന്നു. നിങ്ങളെ കണ്ടുമുട്ടുമ്പോള്‍ അവര്‍ പറയും; ഞങ്ങള്‍ വിശ്വസിച്ചിരിക്കുന്നു എന്ന്‌. എന്നാല്‍ അവര്‍ തനിച്ചാകുമ്പോള്‍ നിങ്ങളോടുള്ള അരിശം കൊണ്ട് അവര്‍ വിരലുകള്‍ കടിക്കുകയും ചെയ്യും. (നബിയേ,) പറയുക: നിങ്ങളുടെ അരിശം കൊണ്ട് നിങ്ങള്‍ മരിച്ചുകൊള്ളൂ. തീര്‍ച്ചയായും അല്ലാഹു മനസ്സുകളിലുള്ളത് അറിയുന്നവനാകുന്നു.
\end{malayalam}}
\flushright{\begin{Arabic}
\quranayah[3][120]
\end{Arabic}}
\flushleft{\begin{malayalam}
നിങ്ങള്‍ക്ക് വല്ല നേട്ടവും ലഭിക്കുന്ന പക്ഷം അതവര്‍ക്ക് മനഃപ്രയാസമുണ്ടാക്കും. നിങ്ങള്‍ക്ക് വല്ല ദോഷവും നേരിട്ടാല്‍ അവരതില്‍ സന്തോഷിക്കുകയും ചെയ്യും. നിങ്ങള്‍ ക്ഷമിക്കുകയും സൂക്ഷ്മത പാലിക്കുകയും ചെയ്യുന്ന പക്ഷം അവരുടെ കുതന്ത്രം നിങ്ങള്‍ക്കൊരു ഉപദ്രവവും വരുത്തുകയില്ല. തീര്‍ച്ചയായും അല്ലാഹു അവരുടെ പ്രവര്‍ത്തനങ്ങളുടെ എല്ലാവശവും അറിയുന്നവനാകുന്നു.
\end{malayalam}}
\flushright{\begin{Arabic}
\quranayah[3][121]
\end{Arabic}}
\flushleft{\begin{malayalam}
(നബിയേ,) സത്യവിശ്വാസികള്‍ക്ക് യുദ്ധത്തിനുള്ള താവളങ്ങള്‍ സൌകര്യപ്പെടുത്തികൊടുക്കുവാനായി നീ സ്വന്തം കുടുംബത്തില്‍ നിന്ന് കാലത്തു പുറപ്പെട്ടുപോയ സന്ദര്‍ഭം ഓര്‍ക്കുക. അല്ലാഹു എല്ലാം കേള്‍ക്കുന്നവനും അറിയുന്നവനുമാകുന്നു.
\end{malayalam}}
\flushright{\begin{Arabic}
\quranayah[3][122]
\end{Arabic}}
\flushleft{\begin{malayalam}
നിങ്ങളില്‍ പെട്ട രണ്ട് വിഭാഗങ്ങള്‍ ഭീരുത്വം കാണിക്കാന്‍ ഭാവിച്ച സന്ദര്‍ഭം (ശ്രദ്ധേയമാണ്‌.) എന്നാല്‍ അല്ലാഹുവാകുന്നു ആ രണ്ടു വിഭാഗത്തിന്‍റെയും രക്ഷാധികാരി. അല്ലാഹുവിന്‍റെ മേലാണ് സത്യവിശ്വാസികള്‍ ഭരമേല്‍പിക്കേണ്ടത്‌.
\end{malayalam}}
\flushright{\begin{Arabic}
\quranayah[3][123]
\end{Arabic}}
\flushleft{\begin{malayalam}
നിങ്ങള്‍ ദുര്‍ബലരായിരിക്കെ ബദ്‌റില്‍ വെച്ച് അല്ലാഹു നിങ്ങളെ സഹായിച്ചിട്ടുണ്ട്‌. അതിനാല്‍ നിങ്ങള്‍ അല്ലാഹുവെ സൂക്ഷിക്കുക. നിങ്ങള്‍ നന്ദിയുള്ളവരായേക്കാം.
\end{malayalam}}
\flushright{\begin{Arabic}
\quranayah[3][124]
\end{Arabic}}
\flushleft{\begin{malayalam}
(നബിയേ,) നിങ്ങളുടെ രക്ഷിതാവ് മുവ്വായിരം മലക്കുകളെ ഇറക്കികൊണ്ട് നിങ്ങളെ സഹായിക്കുക എന്നത് നിങ്ങള്‍ക്ക് മതിയാവുകയില്ലേ എന്ന് നീ സത്യവിശ്വാസികളോട് പറഞ്ഞിരുന്ന സന്ദര്‍ഭം (ഓര്‍ക്കുക.)
\end{malayalam}}
\flushright{\begin{Arabic}
\quranayah[3][125]
\end{Arabic}}
\flushleft{\begin{malayalam}
(പിന്നീട് അല്ലാഹു വാഗ്ദാനം ചെയ്തു:) അതെ, നിങ്ങള്‍ ക്ഷമിക്കുകയും, സൂക്ഷ്മത പാലിക്കുകയും, നിങ്ങളുടെ അടുക്കല്‍ ശത്രുക്കള്‍ ഈ നിമിഷത്തില്‍ തന്നെ വന്നെത്തുകയുമാണെങ്കില്‍ നിങ്ങളുടെ രക്ഷിതാവ് പ്രത്യേക അടയാളമുള്ള അയ്യായിരം മലക്കുകള്‍ മുഖേന നിങ്ങളെ സഹായിക്കുന്നതാണ്‌.
\end{malayalam}}
\flushright{\begin{Arabic}
\quranayah[3][126]
\end{Arabic}}
\flushleft{\begin{malayalam}
നിങ്ങള്‍ക്കൊരു സന്തോഷവാര്‍ത്തയായിക്കൊണ്ടും, നിങ്ങളുടെ മനസ്സുകള്‍ സമാധാനപ്പെടുവാന്‍ വേണ്ടിയും മാത്രമാണ് അല്ലാഹു പിന്‍ബലം നല്‍കിയത്‌. (സാക്ഷാല്‍) സഹായം പ്രതാപിയും യുക്തിമാനുമായ അല്ലാഹുവിങ്കല്‍ നിന്നുമാത്രമാകുന്നു.
\end{malayalam}}
\flushright{\begin{Arabic}
\quranayah[3][127]
\end{Arabic}}
\flushleft{\begin{malayalam}
സത്യനിഷേധികളില്‍ നിന്ന് ഒരു ഭാഗത്തെ ഉന്‍മൂലനം ചെയ്യുകയോ, അല്ലെങ്കില്‍ അവരെ കീഴൊതുക്കിയിട്ട് അവര്‍ നിരാശരായി പിന്തിരിഞ്ഞോടുകയോ ചെയ്യാന്‍ വേണ്ടിയത്രെ അത്‌.
\end{malayalam}}
\flushright{\begin{Arabic}
\quranayah[3][128]
\end{Arabic}}
\flushleft{\begin{malayalam}
(നബിയേ,) കാര്യത്തിന്‍റെ തീരുമാനത്തില്‍ നിനക്ക് യാതൊരു അവകാശവുമില്ല. അവന്‍ (അല്ലാഹു) ഒന്നുകില്‍ അവരുടെ പശ്ചാത്താപം സ്വീകരിച്ചേക്കാം. അല്ലെങ്കില്‍ അവന്‍ അവരെ ശിക്ഷിച്ചേക്കാം. തീര്‍ച്ചയായും അവര്‍ അക്രമികളാകുന്നു.
\end{malayalam}}
\flushright{\begin{Arabic}
\quranayah[3][129]
\end{Arabic}}
\flushleft{\begin{malayalam}
ആകാശങ്ങളിലും ഭൂമിയിലുമുള്ളതെല്ലാം അല്ലാഹുവിന്‍റെതാകുന്നു. അവന്‍ ഉദ്ദേശിക്കുന്നവര്‍ക്ക് പൊറുത്തുകൊടുക്കുകയും അവന്‍ ഉദ്ദേശിക്കുന്നവരെ ശിക്ഷിക്കുകയും ചെയ്യും. അല്ലാഹു വളരെ പൊറുക്കുന്നവനും കരുണാനിധിയുമാകുന്നു.
\end{malayalam}}
\flushright{\begin{Arabic}
\quranayah[3][130]
\end{Arabic}}
\flushleft{\begin{malayalam}
സത്യവിശ്വാസികളേ, നിങ്ങള്‍ ഇരട്ടിയിരട്ടിയായി പലിശ തിന്നാതിരിക്കുകയും അല്ലാഹുവെ സൂക്ഷിക്കുകയും ചെയ്യുക. നിങ്ങള്‍ വിജയികളായേക്കാം.
\end{malayalam}}
\flushright{\begin{Arabic}
\quranayah[3][131]
\end{Arabic}}
\flushleft{\begin{malayalam}
സത്യനിഷേധികള്‍ക്ക് ഒരുക്കിവെക്കപ്പെട്ട നരകാഗ്നിയെ നിങ്ങള്‍ സൂക്ഷിക്കുകയും ചെയ്യുക.
\end{malayalam}}
\flushright{\begin{Arabic}
\quranayah[3][132]
\end{Arabic}}
\flushleft{\begin{malayalam}
നിങ്ങള്‍ അല്ലാഹുവെയും റസൂലിനെയും അനുസരിക്കുക. നിങ്ങള്‍ അനുഗൃഹീതരായേക്കാം.
\end{malayalam}}
\flushright{\begin{Arabic}
\quranayah[3][133]
\end{Arabic}}
\flushleft{\begin{malayalam}
നിങ്ങളുടെ രക്ഷിതാവിങ്കല്‍ നിന്നുള്ള പാപമോചനവും, ആകാശഭൂമികളോളം വിശാലമായ സ്വര്‍ഗവും നേടിയെടുക്കാന്‍ നിങ്ങള്‍ ധൃതിപ്പെട്ട് മുന്നേറുക. ധര്‍മ്മനിഷ്ഠപാലിക്കുന്നവര്‍ക്കുവേണ്ടി ഒരുക്കിവെക്കപ്പെട്ടതത്രെ അത്‌.
\end{malayalam}}
\flushright{\begin{Arabic}
\quranayah[3][134]
\end{Arabic}}
\flushleft{\begin{malayalam}
(അതായത്‌) സന്തോഷാവസ്ഥയിലും വിഷമാവസ്ഥയിലും ദാനധര്‍മ്മങ്ങള്‍ ചെയ്യുകയും, കോപം ഒതുക്കിവെക്കുകയും, മനുഷ്യര്‍ക്ക് മാപ്പുനല്‍കുകയും ചെയ്യുന്നവര്‍ക്ക് വേണ്ടി. (അത്തരം) സല്‍കര്‍മ്മകാരികളെ അല്ലാഹു സ്നേഹിക്കുന്നു.
\end{malayalam}}
\flushright{\begin{Arabic}
\quranayah[3][135]
\end{Arabic}}
\flushleft{\begin{malayalam}
വല്ല നീചകൃത്യവും ചെയ്തുപോയാല്‍, അഥവാ സ്വന്തത്തോട് തന്നെ വല്ല ദ്രോഹവും ചെയ്തു പോയാല്‍ അല്ലാഹുവെ ഓര്‍ക്കുകയും തങ്ങളുടെ പാപങ്ങള്‍ക്ക് മാപ്പുതേടുകയും ചെയ്യുന്നവര്‍ക്ക് വേണ്ടി. -പാപങ്ങള്‍ പൊറുക്കുവാന്‍ അല്ലാഹുവല്ലാതെ ആരാണുള്ളത്‌?- ചെയ്തുപോയ (ദുഷ്‌) പ്രവൃത്തിയില്‍ അറിഞ്ഞുകൊണ്ട് ഉറച്ചുനില്‍ക്കാത്തവരുമാകുന്നു അവര്‍.
\end{malayalam}}
\flushright{\begin{Arabic}
\quranayah[3][136]
\end{Arabic}}
\flushleft{\begin{malayalam}
അത്തരക്കാര്‍ക്കുള്ള പ്രതിഫലം തങ്ങളുടെ രക്ഷിതാവിങ്കല്‍ നിന്നുള്ള പാപമോചനവും, താഴ്ഭാഗത്ത് കൂടി അരുവികള്‍ ഒഴുകുന്ന സ്വര്‍ഗത്തോപ്പുകളുമാകുന്നു. അവരതില്‍ നിത്യവാസികളായിരിക്കും. പ്രവര്‍ത്തിക്കുന്നവര്‍ക്ക് ലഭിക്കുന്ന പ്രതിഫലം എത്ര നന്നായിരിക്കുന്നു!
\end{malayalam}}
\flushright{\begin{Arabic}
\quranayah[3][137]
\end{Arabic}}
\flushleft{\begin{malayalam}
നിങ്ങള്‍ക്ക് മുമ്പ് പല (ദൈവിക) നടപടികളും കഴിഞ്ഞുപോയിട്ടുണ്ട്‌. അതിനാല്‍ നിങ്ങള്‍ ഭൂമിയിലൂടെ സഞ്ചരിച്ചിട്ട് സത്യനിഷേധികളുടെ പര്യവസാനം എങ്ങനെയായിരുന്നുവെന്ന് നോക്കുവിന്‍.
\end{malayalam}}
\flushright{\begin{Arabic}
\quranayah[3][138]
\end{Arabic}}
\flushleft{\begin{malayalam}
ഇത് മനുഷ്യര്‍ക്കായുള്ള ഒരു വിളംബരവും, ധര്‍മ്മനിഷ്ഠപാലിക്കുന്നവര്‍ക്ക് മാര്‍ഗദര്‍ശനവും, സാരോപദേശവുമാകുന്നു.
\end{malayalam}}
\flushright{\begin{Arabic}
\quranayah[3][139]
\end{Arabic}}
\flushleft{\begin{malayalam}
നിങ്ങള്‍ ദൌര്‍ബല്യം കാണിക്കുകയോ ദുഃഖിക്കുകയോ ചെയ്യരുത്‌. നിങ്ങള്‍ വിശ്വാസികളാണെങ്കില്‍ നിങ്ങള്‍ തന്നെയാണ് ഉന്നതന്‍മാര്‍.
\end{malayalam}}
\flushright{\begin{Arabic}
\quranayah[3][140]
\end{Arabic}}
\flushleft{\begin{malayalam}
നിങ്ങള്‍ക്കിപ്പോള്‍ കേടുപാടുകള്‍ പറ്റിയിട്ടുണ്ടെങ്കില്‍ (മുമ്പ്‌) അക്കൂട്ടര്‍ക്കും അതുപോലെ കേടുപാടുകള്‍ പറ്റിയിട്ടുണ്ട്‌. ആ (യുദ്ധ) ദിവസങ്ങളിലെ ജയാപജയങ്ങള്‍ ആളുകള്‍ക്കിടയില്‍ നാം മാറ്റിക്കൊണ്ടിരിക്കുന്നതാണ്‌. വിശ്വസിച്ചവരെ അല്ലാഹു തിരിച്ചറിയുവാനും, നിങ്ങളില്‍ നിന്ന് രക്തസാക്ഷികളെ ഉണ്ടാക്കിത്തീര്‍ക്കുവാനും കൂടിയാണത്‌. അല്ലാഹു അക്രമികളെ ഇഷ്ടപ്പെടുകയില്ല.
\end{malayalam}}
\flushright{\begin{Arabic}
\quranayah[3][141]
\end{Arabic}}
\flushleft{\begin{malayalam}
അല്ലാഹു സത്യവിശ്വാസികളെ ശുദ്ധീകരിച്ചെടുക്കുവാന്‍ വേണ്ടിയും, സത്യനിഷേധികളെ ക്ഷയിപ്പിക്കുവാന്‍ വേണ്ടിയും കൂടിയാണത്‌.
\end{malayalam}}
\flushright{\begin{Arabic}
\quranayah[3][142]
\end{Arabic}}
\flushleft{\begin{malayalam}
അതല്ല, നിങ്ങളില്‍ നിന്ന് ധര്‍മ്മസമരത്തില്‍ ഏര്‍പെട്ടവരെയും ക്ഷമാശീലരെയും അല്ലാഹു തിരിച്ചറിഞ്ഞിട്ടല്ലാതെ നിങ്ങള്‍ക്ക് സ്വര്‍ഗത്തില്‍ പ്രവേശിച്ചുകളയാമെന്ന് നിങ്ങള്‍ വിചാരിച്ചിരിക്കയാണോ?
\end{malayalam}}
\flushright{\begin{Arabic}
\quranayah[3][143]
\end{Arabic}}
\flushleft{\begin{malayalam}
നിങ്ങള്‍ മരണത്തെ നേരില്‍ കാണുന്നതിന് മുമ്പ് നിങ്ങളതിന് കൊതിക്കുന്നവരായിരുന്നു. ഇപ്പോളിതാ നിങ്ങള്‍ നോക്കിനില്‍ക്കെത്തന്നെ അത് നിങ്ങള്‍ കണ്ടു കഴിഞ്ഞു.
\end{malayalam}}
\flushright{\begin{Arabic}
\quranayah[3][144]
\end{Arabic}}
\flushleft{\begin{malayalam}
മുഹമ്മദ് അല്ലാഹുവിന്‍റെ ഒരു ദൂതന്‍ മാത്രമാകുന്നു. അദ്ദേഹത്തിന് മുമ്പും ദൂതന്‍മാര്‍ കഴിഞ്ഞുപോയിട്ടുണ്ട്‌. അദ്ദേഹം മരണപ്പെടുകയോ കൊല്ലപ്പെടുകയോ ചെയ്തെങ്കില്‍ നിങ്ങള്‍ പുറകോട്ട് തിരിച്ചുപോകുകയോ? ആരെങ്കിലും പുറകോട്ട് തിരിച്ചുപോകുന്ന പക്ഷം അല്ലാഹുവിന് ഒരു ദ്രോഹവും അത് വരുത്തുകയില്ല. നന്ദികാണിക്കുന്നവര്‍ക്ക് അല്ലാഹു തക്കതായ പ്രതിഫലം നല്‍കുന്നതാണ്‌.
\end{malayalam}}
\flushright{\begin{Arabic}
\quranayah[3][145]
\end{Arabic}}
\flushleft{\begin{malayalam}
അല്ലാഹുവിന്‍റെ ഉത്തരവനുസരിച്ചല്ലാതെ ഒരാള്‍ക്കും മരിക്കാനൊക്കുകയില്ല. അവധി കുറിക്കപ്പെട്ട ഒരു വിധിയാണത്‌. ആരെങ്കിലും ഇഹലോകത്തെ പ്രതിഫലമണ് ഉദ്ദേശിക്കുന്നതെങ്കില്‍ അവന്ന് ഇവിടെ നിന്ന് നാം നല്‍കും. ആരെങ്കിലും പരലോകത്തെ പ്രതിഫലമാണ് ഉദ്ദേശിക്കുന്നതെങ്കില്‍ അവന്ന് നാം അവിടെ നിന്ന് നല്‍കും. നന്ദികാണിക്കുന്നവര്‍ക്ക് നാം തക്കതായ പ്രതിഫലം നല്‍കുന്നതാണ്‌.
\end{malayalam}}
\flushright{\begin{Arabic}
\quranayah[3][146]
\end{Arabic}}
\flushleft{\begin{malayalam}
എത്രയെത്ര പ്രവാചകന്‍മാരോടൊപ്പം അനേകം ദൈവദാസന്‍മാര്‍ യുദ്ധം ചെയ്തിട്ടുണ്ട്‌. എന്നിട്ട് അല്ലാഹുവിന്‍റെ മാര്‍ഗത്തില്‍ തങ്ങള്‍ക്ക് നേരിട്ട യാതൊന്നു കൊണ്ടും അവര്‍ തളര്‍ന്നില്ല. അവര്‍ ദൌര്‍ബല്യം കാണിക്കുകയോ ഒതുങ്ങികൊടുക്കുകയോ ചെയ്തില്ല. അത്തരം ക്ഷമാശീലരെ അല്ലാഹു സ്നേഹിക്കുന്നു.
\end{malayalam}}
\flushright{\begin{Arabic}
\quranayah[3][147]
\end{Arabic}}
\flushleft{\begin{malayalam}
അവര്‍ പറഞ്ഞിരുന്നത് ഇപ്രകാരം മാത്രമായിരുന്നു: ഞങ്ങളുടെ രക്ഷിതാവേ, ഞങ്ങളുടെ പാപങ്ങളും, ഞങ്ങളുടെ കാര്യങ്ങളില്‍ വന്നുപോയ അതിക്രമങ്ങളും ഞങ്ങള്‍ക്ക് നീ പൊറുത്തുതരേണമേ. ഞങ്ങളുടെ കാലടികള്‍ നീ ഉറപ്പിച്ചു നിര്‍ത്തുകയും, സത്യനിഷേധികളായ ജനതക്കെതിരില്‍ ഞങ്ങളെ നീ സഹായിക്കുകയും ചെയ്യേണമേ.
\end{malayalam}}
\flushright{\begin{Arabic}
\quranayah[3][148]
\end{Arabic}}
\flushleft{\begin{malayalam}
തന്‍മൂലം ഇഹലോകത്തെ പ്രതിഫലവും, പരലോകത്തെ വിശിഷ്ടമായ പ്രതിഫലവും അല്ലാഹു അവര്‍ക്ക് നല്‍കി. അല്ലാഹു സല്‍കര്‍മ്മകാരികളെ സ്നേഹിക്കുന്നു.
\end{malayalam}}
\flushright{\begin{Arabic}
\quranayah[3][149]
\end{Arabic}}
\flushleft{\begin{malayalam}
സത്യവിശ്വാസികളേ, സത്യനിഷേധികളെ നിങ്ങള്‍ അനുസരിച്ച് പോയാല്‍ അവര്‍ നിങ്ങളെ പുറകോട്ട് തിരിച്ചുകൊണ്ടു പോകും. അങ്ങനെ നിങ്ങള്‍ നഷ്ടക്കാരായി മാറിപ്പോകും.
\end{malayalam}}
\flushright{\begin{Arabic}
\quranayah[3][150]
\end{Arabic}}
\flushleft{\begin{malayalam}
അല്ല, അല്ലാഹുവാകുന്നു നിങ്ങളുടെ രക്ഷാധികാരി. അവനാകുന്നു സഹായികളില്‍ ഉത്തമന്‍.
\end{malayalam}}
\flushright{\begin{Arabic}
\quranayah[3][151]
\end{Arabic}}
\flushleft{\begin{malayalam}
സത്യനിഷേധികളുടെ മനസ്സുകളില്‍ നാം ഭയം ഇട്ടുകൊടുക്കുന്നതാണ്‌. അല്ലാഹു യാതൊരു പ്രമാണവും അവതരിപ്പിച്ചിട്ടില്ലാത്ത വസ്തുക്കളെ അല്ലാഹുവോട് അവര്‍ പങ്കുചേര്‍ത്തതിന്‍റെ ഫലമാണത്‌. നരകമാകുന്നു അവരുടെ സങ്കേതം. അക്രമികളുടെ പാര്‍പ്പിടം എത്രമോശം!
\end{malayalam}}
\flushright{\begin{Arabic}
\quranayah[3][152]
\end{Arabic}}
\flushleft{\begin{malayalam}
അല്ലാഹുവിന്‍റെ അനുമതി പ്രകാരം നിങ്ങളവരെ കൊന്നൊടുക്കിക്കൊണ്ടിരുന്നപ്പോള്‍ നിങ്ങളോടുള്ള അല്ലാഹുവിന്‍റെ വാഗ്ദാനത്തില്‍ അവന്‍ സത്യം പാലിച്ചിട്ടുണ്ട്‌. എന്നാല്‍ നിങ്ങള്‍ ഭീരുത്വം കാണിക്കുകയും, കാര്യനിര്‍വഹണത്തില്‍ അന്യോന്യം പിണങ്ങുകയും, നിങ്ങള്‍ ഇഷ്ടപ്പെടുന്ന നേട്ടം അല്ലാഹു നിങ്ങള്‍ക്ക് കാണിച്ചുതന്നതിന് ശേഷം നിങ്ങള്‍ അനുസരണക്കേട് കാണിക്കുകയും ചെയ്തപ്പോഴാണ് (കാര്യങ്ങള്‍ നിങ്ങള്‍ക്കെതിരായത്‌.) നിങ്ങളില്‍ ഇഹലോകത്തെ ലക്ഷ്യമാക്കുന്നവരുണ്ട്‌. പരലോകത്തെ ലക്ഷ്യമാക്കുന്നവരും നിങ്ങളിലുണ്ട്‌. അനന്തരം നിങ്ങളെ പരീക്ഷിക്കുവാനായി അവരില്‍ (ശത്രുക്കളില്‍) നിന്ന് നിങ്ങളെ അല്ലാഹു പിന്തിരിപ്പിച്ചുകളഞ്ഞു. എന്നാല്‍ അല്ലാഹു നിങ്ങള്‍ക്ക് മാപ്പ് തന്നിരിക്കുന്നു. അല്ലാഹു സത്യവിശ്വാസികളോട് ഔദാര്യം കാണിക്കുന്നവനാകുന്നു.
\end{malayalam}}
\flushright{\begin{Arabic}
\quranayah[3][153]
\end{Arabic}}
\flushleft{\begin{malayalam}
ആരെയും തിരിഞ്ഞ് നോക്കാതെ നിങ്ങള്‍ (പടക്കളത്തില്‍നിന്നു) ഓടിക്കയറിയിരുന്ന സന്ദര്‍ഭം (ഓര്‍ക്കുക.) റസൂല്‍ പിന്നില്‍ നിന്ന് നിങ്ങളെ വിളിക്കുന്നുണ്ടായിരുന്നു. അങ്ങനെ അല്ലാഹു നിങ്ങള്‍ക്കു ദുഃഖത്തിനുമേല്‍ ദുഃഖം പ്രതിഫലമായി നല്‍കി. നഷ്ടപ്പെട്ടുപോകുന്ന നേട്ടത്തിന്‍റെ പേരിലോ, നിങ്ങളെ ബാധിക്കുന്ന ആപത്തിന്‍റെ പേരിലോ നിങ്ങള്‍ ദുഃഖിക്കുവാന്‍ ഇടവരാതിരിക്കുന്നതിനുവേണ്ടിയാണിത്‌. നിങ്ങള്‍ പ്രവര്‍ത്തിക്കുന്നതിനെപ്പറ്റി അല്ലാഹു സൂക്ഷ്മമായി അറിയുന്നവനാകുന്നു.
\end{malayalam}}
\flushright{\begin{Arabic}
\quranayah[3][154]
\end{Arabic}}
\flushleft{\begin{malayalam}
പിന്നീട് ആ ദുഃഖത്തിനു ശേഷം അല്ലാഹു നിങ്ങള്‍ക്കൊരു നിര്‍ഭയത്വം അഥവാ മയക്കം ഇറക്കിത്തന്നു. ആ മയക്കം നിങ്ങളില്‍ ഒരു വിഭാഗത്തെ പൊതിയുകയായിരുന്നു. വേറെ ഒരു വിഭാഗമാകട്ടെ സ്വന്തം ദേഹങ്ങളെപ്പറ്റിയുള്ള ചിന്തയാല്‍ അസ്വസ്ഥരായിരുന്നു. അല്ലാഹുവെ പറ്റി അവര്‍ ധരിച്ചിരുന്നത് സത്യവിരുദ്ധമായ അനിസ്ലാമിക ധാരണയായിരുന്നു. അവര്‍ പറയുന്നു: കാര്യത്തില്‍ നമുക്ക് വല്ല സ്വാധീനവുമുണ്ടോ? (നബിയേ,) പറയുക: കാര്യമെല്ലാം അല്ലാഹുവിന്‍റെ അധീനത്തിലാകുന്നു. നിന്നോടവര്‍ വെളിപ്പെടുത്തുന്നതല്ലാത്ത മറ്റെന്തോ മനസ്സുകളില്‍ അവര്‍ ഒളിച്ചു വെക്കുന്നു. അവര്‍ പറയുന്നു: കാര്യത്തില്‍ നമുക്ക് വല്ല സ്വാധീനവുമുണ്ടായിരുന്നുവെങ്കില്‍ നാം ഇവിടെ വെച്ച് കൊല്ലപ്പെടുമായിരുന്നില്ല. (നബിയേ,) പറയുക: നിങ്ങള്‍ സ്വന്തം വീടുകളില്‍ ആയിരുന്നാല്‍ പോലും കൊല്ലപ്പെടാന്‍ വിധിക്കപ്പെട്ടവര്‍ തങ്ങള്‍ മരിച്ചുവീഴുന്ന സ്ഥാനങ്ങളിലേക്ക് (സ്വയം) പുറപ്പെട്ട് വരുമായിരുന്നു. നിങ്ങളുടെ മനസ്സുകളിലുള്ളത് അല്ലാഹു പരീക്ഷിച്ചറിയുവാന്‍ വേണ്ടിയും, നിങ്ങളുടെ ഹൃദയങ്ങളിലുള്ളത് ശുദ്ധീകരിച്ചെടുക്കുവാന്‍ വേണ്ടിയുമാണിതെല്ലാം. മനസ്സുകളിലുള്ളതെല്ലാം അറിയുന്നവനാകുന്നു അല്ലാഹു.
\end{malayalam}}
\flushright{\begin{Arabic}
\quranayah[3][155]
\end{Arabic}}
\flushleft{\begin{malayalam}
രണ്ടു സംഘങ്ങള്‍ ഏറ്റുമുട്ടിയ ദിവസം നിങ്ങളുടെ കൂട്ടത്തില്‍ നിന്ന് പിന്തിരിഞ്ഞ് ഓടിയവരെ തങ്ങളുടെ ചില ചെയ്തികള്‍ കാരണമായി പിശാച് വഴിതെറ്റിക്കുകയാണുണ്ടായത്‌. അല്ലാഹു അവര്‍ക്ക് മാപ്പുനല്‍കിയിരിക്കുന്നു. തീര്‍ച്ചയായും അല്ലാഹു ഏറെ പൊറുക്കുന്നവനും സഹനശീലനുമാകുന്നു.
\end{malayalam}}
\flushright{\begin{Arabic}
\quranayah[3][156]
\end{Arabic}}
\flushleft{\begin{malayalam}
സത്യവിശ്വാസികളേ, നിങ്ങള്‍ (ചില) സത്യനിഷേധികളെപ്പോലെയാകരുത്‌. തങ്ങളുടെ സഹോദരങ്ങള്‍ യാത്രപോകുകയോ, യോദ്ധാക്കളായി പുറപ്പെടുകയോ ചെയ്തിട്ട് മരണമടയുകയാണെങ്കില്‍ അവര്‍ പറയും: ഇവര്‍ ഞങ്ങളുടെ അടുത്തായിരുന്നെങ്കില്‍ മരണപ്പെടുകയോ കൊല്ലപ്പെടുകയോ ഇല്ലായിരുന്നു. അങ്ങനെ അല്ലാഹു അത് അവരുടെ മനസ്സുകളില്‍ ഒരു ഖേദമാക്കിവെക്കുന്നു. അല്ലാഹുവാണ് ജീവിപ്പിക്കുകയും മരിപ്പിക്കുകയും ചെയ്യുന്നത്‌. അല്ലാഹു നിങ്ങള്‍ പ്രവര്‍ത്തിക്കുന്നതെല്ലാം കണ്ടറിയുന്നവനുമത്രെ.
\end{malayalam}}
\flushright{\begin{Arabic}
\quranayah[3][157]
\end{Arabic}}
\flushleft{\begin{malayalam}
നിങ്ങള്‍ അല്ലാഹുവിന്‍റെ മാര്‍ഗത്തില്‍ കൊല്ലപ്പെടുകയോ, മരണപ്പെടുകയോ ചെയ്യുന്ന പക്ഷം അല്ലാഹുവിങ്കല്‍ നിന്ന് ലഭിക്കുന്ന പാപമോചനവും കാരുണ്യവുമാണ് അവര്‍ ശേഖരിച്ച് വെക്കുന്നതിനെക്കാളെല്ലാം ഗുണകരമായിട്ടുള്ളത്‌.
\end{malayalam}}
\flushright{\begin{Arabic}
\quranayah[3][158]
\end{Arabic}}
\flushleft{\begin{malayalam}
നിങ്ങള്‍ മരണപ്പെടുകയാണെങ്കിലും കൊല്ലപ്പെടുകയാണെങ്കിലും തീര്‍ച്ചയായും അല്ലാഹുവിങ്കലേക്ക് തന്നെയാണ് നിങ്ങള്‍ ഒരുമിച്ചുകൂട്ടപ്പെടുന്നത്‌.
\end{malayalam}}
\flushright{\begin{Arabic}
\quranayah[3][159]
\end{Arabic}}
\flushleft{\begin{malayalam}
(നബിയേ,) അല്ലാഹുവിങ്കല്‍ നിന്നുള്ള കാരുണ്യം കൊണ്ടാണ് നീ അവരോട് സൌമ്യമായി പെരുമാറിയത്‌. നീ ഒരു പരുഷസ്വഭാവിയും കഠിനഹൃദയനുമായിരുന്നുവെങ്കില്‍ നിന്‍റെ ചുറ്റില്‍ നിന്നും അവര്‍ പിരിഞ്ഞ് പോയിക്കളയുമായിരുന്നു. ആകയാല്‍ നീ അവര്‍ക്ക് മാപ്പുകൊടുക്കുകയും, അവര്‍ക്ക് വേണ്ടി പാപമോചനം തേടുകയും ചെയ്യുക. കാര്യങ്ങളില്‍ നീ അവരോട് കൂടിയാലോചിക്കുകയും ചെയ്യുക. അങ്ങനെ നീ ഒരു തീരുമാനമെടുത്ത് കഴിഞ്ഞാല്‍ അല്ലാഹുവില്‍ ഭരമേല്‍പിക്കുക. തന്നില്‍ ഭരമേല്‍പിക്കുന്നവരെ തീര്‍ച്ചയായും അല്ലാഹു ഇഷ്ടപ്പെടുന്നതാണ്‌.
\end{malayalam}}
\flushright{\begin{Arabic}
\quranayah[3][160]
\end{Arabic}}
\flushleft{\begin{malayalam}
നിങ്ങളെ അല്ലാഹു സഹായിക്കുന്ന പക്ഷം നിങ്ങളെ തോല്‍പിക്കാനാരുമില്ല. അവന്‍ നിങ്ങളെ കൈവിട്ടുകളയുന്ന പക്ഷം അവന്നു പുറമെ ആരാണ് നിങ്ങളെ സഹായിക്കാനുള്ളത്‌? അതിനാല്‍ സത്യവിശ്വാസികള്‍ അല്ലാഹുവില്‍ ഭരമേല്‍പിക്കട്ടെ.
\end{malayalam}}
\flushright{\begin{Arabic}
\quranayah[3][161]
\end{Arabic}}
\flushleft{\begin{malayalam}
ഒരു പ്രവാചകനും വല്ലതും വഞ്ചിച്ചെടുക്കുക എന്നത് ഉണ്ടാകാവുന്നതല്ല. വല്ലവനും വഞ്ചിച്ചെടുത്താല്‍ താന്‍ വഞ്ചിച്ചെടുത്ത സാധനവുമായി ഉയിര്‍ത്തെഴുന്നേല്‍പിന്‍റെ നാളില്‍ അവന്‍ വരുന്നതാണ്‌. അനന്തരം ഓരോ വ്യക്തിക്കും താന്‍ സമ്പാദിച്ചുവെച്ചതിന്‍റെ ഫലം പൂര്‍ണ്ണമായി നല്‍കപ്പെടും. അവരോട് ഒരു അനീതിയും കാണിക്കപ്പെടുന്നതല്ല.
\end{malayalam}}
\flushright{\begin{Arabic}
\quranayah[3][162]
\end{Arabic}}
\flushleft{\begin{malayalam}
അല്ലാഹുവിന്‍റെ പ്രീതിയെ പിന്തുടര്‍ന്ന ഒരുവന്‍ അല്ലാഹുവിന്‍റെ കോപത്തിന് പാത്രമായ ഒരുവനെപ്പോലെയാണോ? അവന്‍റെ വാസസ്ഥലം നരകമത്രെ. അത് എത്ര ചീത്ത സങ്കേതം.
\end{malayalam}}
\flushright{\begin{Arabic}
\quranayah[3][163]
\end{Arabic}}
\flushleft{\begin{malayalam}
അവര്‍ അല്ലാഹുവിന്‍റെ അടുക്കല്‍ പല പദവികളിലാകുന്നു. അവര്‍ പ്രവര്‍ത്തിക്കുന്നതെല്ലാം അല്ലാഹു കണ്ടറിയുന്നവനാണ്‌.
\end{malayalam}}
\flushright{\begin{Arabic}
\quranayah[3][164]
\end{Arabic}}
\flushleft{\begin{malayalam}
തീര്‍ച്ചയായും സത്യവിശ്വാസികളില്‍ അവരില്‍ നിന്ന് തന്നെയുള്ള ഒരു ദൂതനെ നിയോഗിക്കുക വഴി അല്ലാഹു മഹത്തായ അനുഗ്രഹമാണ് അവര്‍ക്ക് നല്‍കിയിട്ടുള്ളത്‌. അല്ലാഹുവിന്‍റെ ദൃഷ്ടാന്തങ്ങള്‍ അവര്‍ക്ക് ഓതികേള്‍പിക്കുകയും, അവരെ സംസ്കരിക്കുകയും, അവര്‍ക്കു ഗ്രന്ഥവും ജ്ഞാനവും പഠിപ്പിക്കുകയും ചെയ്യുന്ന (ഒരു ദൂതനെ). അവരാകട്ടെ മുമ്പ് വ്യക്തമായ വഴികേടില്‍ തന്നെയായിരുന്നു.
\end{malayalam}}
\flushright{\begin{Arabic}
\quranayah[3][165]
\end{Arabic}}
\flushleft{\begin{malayalam}
നിങ്ങള്‍ക്ക് ഒരു വിപത്ത് നേരിട്ടു. അതിന്‍റെ ഇരട്ടി നിങ്ങള്‍ ശത്രുക്കള്‍ക്ക് വരുത്തിവെച്ചിട്ടുണ്ടായിരുന്നു. എന്നിട്ടും നിങ്ങള്‍ പറയുകയാണോ; ഇതെങ്ങനെയാണ് സംഭവിച്ചത് എന്ന്‌? (നബിയേ,) പറയുക: അത് നിങ്ങളുടെ പക്കല്‍ നിന്ന് തന്നെ ഉണ്ടായതാകുന്നു. തീര്‍ച്ചയായും അല്ലാഹു ഏതൊരു കാര്യത്തിനും കഴിവുള്ളവനാകുന്നു.
\end{malayalam}}
\flushright{\begin{Arabic}
\quranayah[3][166]
\end{Arabic}}
\flushleft{\begin{malayalam}
രണ്ട് സംഘങ്ങള്‍ ഏറ്റുമുട്ടിയ ആ ദിവസം നിങ്ങള്‍ക്ക് ബാധിച്ച വിപത്ത് അല്ലാഹുവിന്‍റെ അനുമതിയോടെത്തന്നെയാണുണ്ടായത്‌. സത്യവിശ്വാസികളാരെന്ന് അവന് തിരിച്ചറിയുവാന്‍ വേണ്ടിയുമാകുന്നു അത്‌.
\end{malayalam}}
\flushright{\begin{Arabic}
\quranayah[3][167]
\end{Arabic}}
\flushleft{\begin{malayalam}
നിങ്ങള്‍ വരൂ. അല്ലാഹുവിന്‍റെ മാര്‍ഗത്തില്‍ യുദ്ധം ചെയ്യൂ, അല്ലെങ്കില്‍ ചെറുത്ത് നില്‍ക്കുകയെങ്കിലും ചെയ്യൂ എന്ന് കല്‍പിക്കപ്പെട്ടാല്‍ യുദ്ധമുണ്ടാകുമെന്ന് ഞങ്ങള്‍ക്ക് ബോധ്യമുണ്ടായിരുന്നെങ്കില്‍ ഞങ്ങളും നിങ്ങളുടെ പിന്നാലെ വരുമായിരുന്നു എന്ന് പറയുന്ന കാപട്യക്കാരെ അവന്‍ തിരിച്ചറിയുവാന്‍ വേണ്ടിയുമാകുന്നു അത്‌. അന്ന് സത്യവിശ്വാസത്തോടുള്ളതിനെക്കാള്‍ കൂടുതല്‍ അടുപ്പം അവര്‍ക്ക് അവിശ്വാസത്തോടായിരുന്നു. തങ്ങളുടെ വായ്കൊണ്ട് അവര്‍ പറയുന്നത് അവരുടെ ഹൃദയങ്ങളിലില്ലാത്തതാണ്‌. അവര്‍ മൂടിവെക്കുന്നതിനെപ്പറ്റി അല്ലാഹു കൂടുതല്‍ അറിയുന്നവനാകുന്നു.
\end{malayalam}}
\flushright{\begin{Arabic}
\quranayah[3][168]
\end{Arabic}}
\flushleft{\begin{malayalam}
(യുദ്ധത്തിന് പോകാതെ) വീട്ടിലിരിക്കുകയും (യുദ്ധത്തിന് പോയ) സഹോദരങ്ങളെപ്പറ്റി, ഞങ്ങളുടെ വാക്ക് സ്വീകരിച്ചിരുന്നെങ്കില്‍ അവര്‍ കൊല്ലപ്പെടുമായിരുന്നില്ല എന്ന് പറയുകയും ചെയ്തവരാണവര്‍ (കപടന്‍മാര്‍). (നബിയേ,) പറയുക: എന്നാല്‍ നിങ്ങള്‍ സത്യവാന്‍മാരാണെങ്കില്‍ നിങ്ങളില്‍ നിന്ന് നിങ്ങള്‍ മരണത്തെ തടുത്തു നിര്‍ത്തൂ.
\end{malayalam}}
\flushright{\begin{Arabic}
\quranayah[3][169]
\end{Arabic}}
\flushleft{\begin{malayalam}
അല്ലാഹുവിന്‍റെ മാര്‍ഗത്തില്‍ കൊല്ലപ്പെട്ടവരെ മരിച്ച് പോയവരായി നീ ഗണിക്കരുത്‌. എന്നാല്‍ അവര്‍ അവരുടെ രക്ഷിതാവിന്‍റെ അടുക്കല്‍ ജീവിച്ചിരിക്കുന്നവരാണ്‌. അവര്‍ക്ക് ഉപജീവനം നല്‍കപ്പെട്ടിരിക്കുന്നു.
\end{malayalam}}
\flushright{\begin{Arabic}
\quranayah[3][170]
\end{Arabic}}
\flushleft{\begin{malayalam}
അല്ലാഹു തന്‍റെ അനുഗ്രഹത്തില്‍ നിന്ന് അവര്‍ക്കു നല്‍കിയതുകൊണ്ട് അവര്‍ സന്തുഷ്ടരായിരിക്കും. തങ്ങളോടൊപ്പം വന്നുചേര്‍ന്നിട്ടില്ലാത്ത, തങ്ങളുടെ പിന്നില്‍ (ഇഹലോകത്ത്‌) കഴിയുന്ന വിശ്വാസികളെപ്പറ്റി, അവര്‍ക്ക് യാതൊന്നും ഭയപ്പെടുവാനോ ദുഃഖിക്കാനോ ഇല്ലെന്നോര്‍ത്ത് അവര്‍ (ആ രക്തസാക്ഷികള്‍) സന്തോഷമടയുന്നു.
\end{malayalam}}
\flushright{\begin{Arabic}
\quranayah[3][171]
\end{Arabic}}
\flushleft{\begin{malayalam}
അല്ലാഹുവിന്‍റെ അനുഗ്രഹവും ഔദാര്യവും കൊണ്ട് അവര്‍ സന്തോഷമടയുന്നു. സത്യവിശ്വാസികളുടെ പ്രതിഫലം അല്ലാഹു പാഴാക്കുകയില്ല എന്നതും (അവരെ സന്തുഷ്ടരാക്കുന്നു.)
\end{malayalam}}
\flushright{\begin{Arabic}
\quranayah[3][172]
\end{Arabic}}
\flushleft{\begin{malayalam}
പരിക്ക് പറ്റിയതിന് ശേഷവും അല്ലാഹുവിന്‍റെയും റസൂലിന്‍റെയും കല്‍പനക്ക് ഉത്തരം ചെയ്തവരാരോ അവരില്‍ നിന്ന് സല്‍കര്‍മ്മകാരികളായിരിക്കുകയും സൂക്ഷ്മത പാലിക്കുകയും ചെയ്തവര്‍ക്ക് മഹത്തായ പ്രതിഫലമുണ്ട്‌.
\end{malayalam}}
\flushright{\begin{Arabic}
\quranayah[3][173]
\end{Arabic}}
\flushleft{\begin{malayalam}
ആ ജനങ്ങള്‍ നിങ്ങളെ നേരിടാന്‍ (സൈന്യത്തെ) ശേഖരിച്ചിരിക്കുന്നു; അവരെ ഭയപ്പെടണം എന്നു ആളുകള്‍ അവരോട് പറഞ്ഞപ്പോള്‍ അതവരുടെ വിശ്വാസം വര്‍ദ്ധിപ്പിക്കുകയാണ് ചെയ്തത്‌. അവര്‍ പറഞ്ഞു: ഞങ്ങള്‍ക്ക് അല്ലാഹു മതി. ഭരമേല്‍പിക്കുവാന്‍ ഏറ്റവും നല്ലത് അവനത്രെ.
\end{malayalam}}
\flushright{\begin{Arabic}
\quranayah[3][174]
\end{Arabic}}
\flushleft{\begin{malayalam}
അങ്ങനെ അല്ലാഹുവിങ്കല്‍ നിന്നുള്ള അനുഗ്രഹവും ഔദാര്യവും കൊണ്ട് യാതൊരു ദോഷവും ബാധിക്കാതെ അവര്‍ മടങ്ങി. അല്ലാഹുവിന്‍റെ പ്രീതിയെ അവര്‍ പിന്തുടരുകയും ചെയ്തു. മഹത്തായ ഔദാര്യമുള്ളവനത്രെ അല്ലാഹു.
\end{malayalam}}
\flushright{\begin{Arabic}
\quranayah[3][175]
\end{Arabic}}
\flushleft{\begin{malayalam}
അത് (നിങ്ങളെ പേടിപ്പിക്കാന്‍ ശ്രമിച്ചത്‌) പിശാചു മാത്രമാകുന്നു. അവന്‍ തന്‍റെ മിത്രങ്ങളെപ്പറ്റി (നിങ്ങളെ) പേടിപ്പെടുത്തുകയാണ്‌. അതിനാല്‍ നിങ്ങളവരെ ഭയപ്പെടാതെ എന്നെ ഭയപ്പെടുക: നിങ്ങള്‍ സത്യവിശ്വാസികളാണെങ്കില്‍.
\end{malayalam}}
\flushright{\begin{Arabic}
\quranayah[3][176]
\end{Arabic}}
\flushleft{\begin{malayalam}
സത്യനിഷേധത്തിലേക്ക് ധൃതിപ്പെട്ട് മുന്നേറിക്കൊണ്ടിരിക്കുന്നവര്‍ നിന്നെ ദുഃഖിപ്പിക്കാതിരിക്കട്ടെ. തീര്‍ച്ചയായും അവര്‍ അല്ലാഹുവിന് ഒരു ദ്രോഹവും വരുത്താന്‍ പോകുന്നില്ല. പരലോകത്തില്‍ അവര്‍ക്ക് ഒരു പങ്കും കൊടുക്കാതിരിക്കാന്‍ അല്ലാഹു ഉദ്ദേശിക്കുന്നു.കനത്ത ശിക്ഷയാണ് അവര്‍ക്കുള്ളത്‌.
\end{malayalam}}
\flushright{\begin{Arabic}
\quranayah[3][177]
\end{Arabic}}
\flushleft{\begin{malayalam}
തീര്‍ച്ചയായും സത്യവിശ്വാസം വിറ്റു സത്യനിഷേധം വാങ്ങിയവര്‍ അല്ലാഹുവിന് ഒരു ദ്രോഹവും വരുത്താന്‍ പോകുന്നില്ല. വേദനയേറിയ ശിക്ഷയാണവര്‍ക്കുള്ളത്‌.
\end{malayalam}}
\flushright{\begin{Arabic}
\quranayah[3][178]
\end{Arabic}}
\flushleft{\begin{malayalam}
സത്യനിഷേധികള്‍ക്ക് നാം സമയം നീട്ടികൊടുക്കുന്നത് അവര്‍ക്ക് ഗുണകരമാണെന്ന് അവര്‍ ഒരിക്കലും വിചാരിച്ചു പോകരുത്‌. അവരുടെ പാപം കൂടിക്കൊണ്ടിരിക്കാന്‍ വേണ്ടി മാത്രമാണ് നാമവര്‍ക്ക് സമയം നീട്ടികൊടുക്കുന്നത്‌. അപമാനകരമായ ശിക്ഷയാണ് അവര്‍ക്കുള്ളത്‌.
\end{malayalam}}
\flushright{\begin{Arabic}
\quranayah[3][179]
\end{Arabic}}
\flushleft{\begin{malayalam}
നല്ലതില്‍ നിന്ന് ദുഷിച്ചതിനെ വേര്‍തിരിച്ചു കാണിക്കാതെ, സത്യവിശ്വാസികളെ നിങ്ങളിന്നുള്ള അവസ്ഥയില്‍ അല്ലാഹു വിടാന്‍ പോകുന്നില്ല. അദൃശ്യജ്ഞാനം അല്ലാഹു നിങ്ങള്‍ക്ക് വെളിപ്പെടുത്തിത്തരാനും പോകുന്നില്ല. എന്നാല്‍ അല്ലാഹു അവന്‍റെ ദൂതന്‍മാരില്‍ നിന്ന് അവന്‍ ഉദ്ദേശിക്കുന്നവരെ (അദൃശ്യജ്ഞാനം അറിയിച്ചുകൊടുക്കുവാനായി) തെരഞ്ഞെടുക്കുന്നു. അതിനാല്‍ നിങ്ങള്‍ അല്ലാഹുവിലും അവന്‍റെ ദൂതന്‍മാരിലും വിശ്വസിക്കുവിന്‍. നിങ്ങള്‍ വിശ്വസിക്കുകയും സൂക്ഷ്മത പാലിക്കുകയും ചെയ്യുന്ന പക്ഷം നിങ്ങള്‍ക്കു മഹത്തായ പ്രതിഫലമുണ്ട്‌.
\end{malayalam}}
\flushright{\begin{Arabic}
\quranayah[3][180]
\end{Arabic}}
\flushleft{\begin{malayalam}
അല്ലാഹു അവന്‍റെ അനുഗ്രഹത്തില്‍ നിന്ന് തങ്ങള്‍ക്കു തന്നിട്ടുള്ളതില്‍ പിശുക്ക് കാണിക്കുന്നവര്‍ അതവര്‍ക്ക് ഗുണകരമാണെന്ന് ഒരിക്കലും വിചാരിക്കരുത്‌. അല്ല, അവര്‍ക്ക് ദോഷകരമാണത്‌. അവര്‍ പിശുക്ക് കാണിച്ച ധനം കൊണ്ട് ഉയിര്‍ത്തെഴുന്നേല്‍പിന്‍റെ നാളില്‍ അവരുടെ കഴുത്തില്‍ മാല ചാര്‍ത്തപ്പെടുന്നതാണ്‌. ആകാശങ്ങളുടെയും ഭൂമിയുടെയും അനന്തരാവകാശം അല്ലാഹുവിനത്രെ. അല്ലാഹു നിങ്ങള്‍ പ്രവര്‍ത്തിക്കുന്നതെല്ലാം സൂക്ഷ്മമായി അറിയുന്നവനാകുന്നു.
\end{malayalam}}
\flushright{\begin{Arabic}
\quranayah[3][181]
\end{Arabic}}
\flushleft{\begin{malayalam}
അല്ലാഹു ദരിദ്രനും നമ്മള്‍ ധനികരുമാണ് എന്ന് പറഞ്ഞവരുടെ വാക്ക് അല്ലാഹു തീര്‍ച്ചയായും കേട്ടിട്ടുണ്ട്‌. അവര്‍ ആ പറഞ്ഞതും അവര്‍ പ്രവാചകന്‍മാരെ അന്യായമായി കൊലപ്പെടുത്തിയതും നാം രേഖപ്പെടുത്തി വെക്കുന്നതാണ്‌. കത്തിഎരിയുന്ന നരകശിക്ഷ ആസ്വദിച്ചു കൊള്ളുക എന്ന് നാം (അവരോട്‌) പറയുകയും ചെയ്യും.
\end{malayalam}}
\flushright{\begin{Arabic}
\quranayah[3][182]
\end{Arabic}}
\flushleft{\begin{malayalam}
നിങ്ങളുടെ കൈകള്‍ മുന്‍കൂട്ടി ചെയ്തു വെച്ചതുകൊണ്ടും അല്ലാഹു അടിമകളോട് അനീതി കാണിക്കുന്നവനല്ല എന്നതുകൊണ്ടുമാണ് അത്‌.
\end{malayalam}}
\flushright{\begin{Arabic}
\quranayah[3][183]
\end{Arabic}}
\flushleft{\begin{malayalam}
ഞങ്ങളുടെ മുമ്പാകെ ഒരു ബലി നടത്തി അതിനെ ദിവ്യാഗ്നി തിന്നുകളയുന്നത് (ഞങ്ങള്‍ക്ക് കാണിച്ചുതരുന്നത്‌) വരെ ഒരു ദൈവദൂതനിലും ഞങ്ങള്‍ വിശ്വസിക്കരുതെന്ന് അല്ലാഹു ഞങ്ങളോട് കരാറു വാങ്ങിയിട്ടുണ്ട് എന്ന് പറഞ്ഞവരത്രെ അവര്‍. (നബിയേ,) പറയുക: വ്യക്തമായ തെളിവുകള്‍ സഹിതവും, നിങ്ങള്‍ ഈ പറഞ്ഞത് സഹിതവും എനിക്ക് മുമ്പ് പല ദൂതന്‍മാരും നിങ്ങളുടെ അടുത്ത് വന്നിട്ടുണ്ട്‌. എന്നിട്ട് നിങ്ങളുടെ വാദം സത്യമാണെങ്കില്‍ നിങ്ങളെന്തിന് അവരെ കൊന്നുകളഞ്ഞു?
\end{malayalam}}
\flushright{\begin{Arabic}
\quranayah[3][184]
\end{Arabic}}
\flushleft{\begin{malayalam}
അപ്പോള്‍ നിന്നെ അവര്‍ നിഷേധിച്ചിട്ടുണ്ടെങ്കില്‍ നിനക്ക് മുമ്പ് വ്യക്തമായ തെളിവുകളും ഏടുകളും വെളിച്ചം നല്‍കുന്ന വേദഗ്രന്ഥവുമായി വന്ന ദൂതന്‍മാരും നിഷേധിക്കപ്പെട്ടിട്ടുണ്ട്‌.
\end{malayalam}}
\flushright{\begin{Arabic}
\quranayah[3][185]
\end{Arabic}}
\flushleft{\begin{malayalam}
ഏതൊരു ദേഹവും മരണം ആസ്വദിക്കുന്നതാണ്‌. നിങ്ങളുടെ പ്രതിഫലങ്ങള്‍ ഉയിര്‍ത്തെഴുന്നേല്‍പിന്‍റെ നാളില്‍ മാത്രമേ നിങ്ങള്‍ക്ക് പൂര്‍ണ്ണമായി നല്‍കപ്പെടുകയുള്ളൂ. അപ്പോള്‍ ആര്‍ നരകത്തില്‍ നിന്ന് അകറ്റിനിര്‍ത്തപ്പെടുകയും സ്വര്‍ഗത്തില്‍ പ്രവേശിപ്പിക്കപ്പെടുകയും ചെയ്യുന്നുവോ അവനാണ് വിജയം നേടുന്നത്‌. ഐഹികജീവിതം കബളിപ്പിക്കുന്ന ഒരു വിഭവമല്ലാതെ മറ്റൊന്നുമല്ല.
\end{malayalam}}
\flushright{\begin{Arabic}
\quranayah[3][186]
\end{Arabic}}
\flushleft{\begin{malayalam}
തീര്‍ച്ചയായും നിങ്ങളുടെ സ്വത്തുകളിലും ശരീരങ്ങളിലും നിങ്ങള്‍ പരീക്ഷിക്കപ്പെടുന്നതാണ്‌. നിങ്ങള്‍ക്ക് മുമ്പ് വേദം നല്‍കപ്പെട്ടവരില്‍ നിന്നും ബഹുദൈവാരാധകരില്‍ നിന്നും നിങ്ങള്‍ ധാരാളം കുത്തുവാക്കുകള്‍ കേള്‍ക്കേണ്ടി വരികയും ചെയ്യും. നിങ്ങള്‍ ക്ഷമിക്കുകയും സൂക്ഷ്മത പാലിക്കുകയും ചെയ്യുന്നുവെങ്കില്‍ തീര്‍ച്ചയായും അത് ദൃഢനിശ്ചയം ചെയ്യേണ്ട കാര്യങ്ങളില്‍ പെട്ടതാകുന്നു.
\end{malayalam}}
\flushright{\begin{Arabic}
\quranayah[3][187]
\end{Arabic}}
\flushleft{\begin{malayalam}
വേദഗ്രന്ഥം നല്‍കപ്പെട്ടവരോട് നിങ്ങളത് ജനങ്ങള്‍ക്ക് വിവരിച്ചുകൊടുക്കണമെന്നും, നിങ്ങളത് മറച്ച് വെക്കരുതെന്നും അല്ലാഹു കരാര്‍ വാങ്ങിയ സന്ദര്‍ഭം (ശ്രദ്ധിക്കുക) എന്നിട്ട് അവരത് (വേദഗ്രന്ഥം) പുറകോട്ട് വലിച്ചെറിയുകയും, തുച്ഛമായ വിലയ്ക്ക് അത് വിറ്റുകളയുകയുമാണ് ചെയ്തത്‌. അവര്‍ പകരം വാങ്ങിയത് വളരെ ചീത്ത തന്നെ.
\end{malayalam}}
\flushright{\begin{Arabic}
\quranayah[3][188]
\end{Arabic}}
\flushleft{\begin{malayalam}
തങ്ങള്‍ ചെയ്തതില്‍ സന്തോഷം കൊള്ളുകയും ചെയ്തിട്ടില്ലാത്ത കാര്യത്തിന്‍റെ പേരില്‍ പ്രശംസിക്കപ്പെടാന്‍ ഇഷ്ടപ്പെടുകയും ചെയ്യുന്ന ആ വിഭാഗത്തെപ്പറ്റി അവര്‍ ശിക്ഷയില്‍ നിന്ന് മുക്തമായ അവസ്ഥയിലാണെന്ന് നീ വിചാരിക്കരുത്‌. അവര്‍ക്കാണ് വേദനയേറിയ ശിക്ഷയുള്ളത്‌.
\end{malayalam}}
\flushright{\begin{Arabic}
\quranayah[3][189]
\end{Arabic}}
\flushleft{\begin{malayalam}
അല്ലാഹുവിന്നാകുന്നു ആകാശങ്ങളുടെയും ഭൂമിയുടെയും ആധിപത്യം. അല്ലാഹു ഏത് കാര്യത്തിനും കഴിവുള്ളവനാകുന്നു.
\end{malayalam}}
\flushright{\begin{Arabic}
\quranayah[3][190]
\end{Arabic}}
\flushleft{\begin{malayalam}
തീര്‍ച്ചയായും ആകാശങ്ങളുടെയും ഭൂമിയുടെയും സൃഷ്ടിയിലും, രാപകലുകള്‍ മാറി മാറി വരുന്നതിലും സല്‍ബുദ്ധിയുള്ളവര്‍ക്ക് പല ദൃഷ്ടാന്തങ്ങളുമുണ്ട്‌.
\end{malayalam}}
\flushright{\begin{Arabic}
\quranayah[3][191]
\end{Arabic}}
\flushleft{\begin{malayalam}
നിന്നുകൊണ്ടും ഇരുന്നു കൊണ്ടും കിടന്നു കൊണ്ടും അല്ലാഹുവെ ഓര്‍മിക്കുകയും, ആകാശങ്ങളുടെയും ഭൂമിയുടെയും സൃഷ്ടിയെപറ്റി ചിന്തിച്ച് കൊണ്ടിരിക്കുകയും ചെയ്യുന്നവരത്രെ അവര്‍. (അവര്‍ പറയും:) ഞങ്ങളുടെ രക്ഷിതാവേ! നീ നിരര്‍ത്ഥകമായി സൃഷ്ടിച്ചതല്ല ഇത്‌. നീ എത്രയോ പരിശുദ്ധന്‍! അതിനാല്‍ നരകശിക്ഷയില്‍ നിന്ന് ഞങ്ങളെ കാത്തുരക്ഷിക്കണേ.
\end{malayalam}}
\flushright{\begin{Arabic}
\quranayah[3][192]
\end{Arabic}}
\flushleft{\begin{malayalam}
ഞങ്ങളുടെ രക്ഷിതാവേ, നീ വല്ലവനെയും നരകത്തില്‍ പ്രവേശിപ്പിച്ചാല്‍ അവനെ നിന്ദ്യനാക്കിക്കഴിഞ്ഞു. അക്രമികള്‍ക്ക് സഹായികളായി ആരുമില്ല താനും.
\end{malayalam}}
\flushright{\begin{Arabic}
\quranayah[3][193]
\end{Arabic}}
\flushleft{\begin{malayalam}
ഞങ്ങളുടെ രക്ഷിതാവേ, സത്യവിശ്വാസത്തിലേക്ക് ക്ഷണിക്കുന്ന ഒരു പ്രബോധകന്‍ നിങ്ങള്‍ നിങ്ങളുടെ രക്ഷിതാവില്‍ വിശ്വസിക്കുവിന്‍ എന്നു പറയുന്നത് ഞങ്ങള്‍ കേട്ടു. അങ്ങനെ ഞങ്ങള്‍ വിശ്വസിച്ചിരിക്കുന്നു. ഞങ്ങളുടെ രക്ഷിതാവേ, അതിനാല്‍ ഞങ്ങളുടെ പാപങ്ങള്‍ ഞങ്ങള്‍ക്ക് നീ പൊറുത്തുതരികയും ഞങ്ങളുടെ തിന്‍മകള്‍ ഞങ്ങളില്‍ നിന്ന് നീ മായ്ച്ചുകളയുകയും ചെയ്യേണമേ. പുണ്യവാന്‍മാരുടെ കൂട്ടത്തിലായി ഞങ്ങളെ നീ മരിപ്പിക്കുകയും ചെയ്യേണമേ.
\end{malayalam}}
\flushright{\begin{Arabic}
\quranayah[3][194]
\end{Arabic}}
\flushleft{\begin{malayalam}
ഞങ്ങളുടെ രക്ഷിതാവേ, നിന്‍റെ ദൂതന്‍മാര്‍ മുഖേന ഞങ്ങളോട് നീ വാഗ്ദാനം ചെയ്തത് ഞങ്ങള്‍ക്ക് നല്‍കുകയും ഉയിര്‍ത്തെഴുന്നേല്‍പിന്‍റെ നാളില്‍ ഞങ്ങള്‍ക്കു നീ നിന്ദ്യത വരുത്താതിരിക്കുകയും ചെയ്യേണമേ. നീ വാഗ്ദാനം ലംഘിക്കുകയില്ല; തീര്‍ച്ച.
\end{malayalam}}
\flushright{\begin{Arabic}
\quranayah[3][195]
\end{Arabic}}
\flushleft{\begin{malayalam}
അപ്പോള്‍ അവരുടെ രക്ഷിതാവ് അവര്‍ക്ക് ഉത്തരം നല്‍കി: പുരുഷനാകട്ടെ, സ്ത്രീയാകട്ടെ നിങ്ങളില്‍ നിന്നും പ്രവര്‍ത്തിക്കുന്ന ഒരാളുടെയും പ്രവര്‍ത്തനം ഞാന്‍ നിഷ്ഫലമാക്കുകയില്ല. നിങ്ങളില്‍ ഓരോ വിഭാഗവും മറ്റു വിഭാഗത്തില്‍ നിന്ന് ഉല്‍ഭവിച്ചവരാകുന്നു. ആകയാല്‍ സ്വന്തം നാട് വെടിയുകയും, സ്വന്തം വീടുകളില്‍ നിന്ന് പുറത്താക്കപ്പെടുകയും, എന്‍റെ മാര്‍ഗത്തില്‍ മര്‍ദ്ദിക്കപ്പെടുകയും, യുദ്ധത്തില്‍ ഏര്‍പെടുകയും, കൊല്ലപ്പെടുകയും ചെയ്തിട്ടുള്ളവരാരോ അവര്‍ക്ക് ഞാന്‍ അവരുടെ തിന്‍മകള്‍ മായ്ച്ചുകൊടുക്കുന്നതും, താഴ്ഭാഗത്ത് കൂടി അരുവികള്‍ ഒഴുകുന്ന സ്വര്‍ഗത്തോപ്പുകളില്‍ അവരെ ഞാന്‍ പ്രവേശിപ്പിക്കുന്നതുമാണ്‌. അല്ലാഹുവിങ്കല്‍ നിന്നുള്ള പ്രതിഫലമത്രെ അത്‌. അല്ലാഹുവിന്‍റെ പക്കലാണ് ഉത്തമമായ പ്രതിഫലമുള്ളത്‌.
\end{malayalam}}
\flushright{\begin{Arabic}
\quranayah[3][196]
\end{Arabic}}
\flushleft{\begin{malayalam}
സത്യനിഷേധികള്‍ നാടുകളിലെങ്ങും സ്വൈരവിഹാരം നടത്തിക്കൊണ്ടിരിക്കുന്നത് നിന്നെ ഒരിക്കലും വഞ്ചിതനാക്കിക്കളയരുത്‌.
\end{malayalam}}
\flushright{\begin{Arabic}
\quranayah[3][197]
\end{Arabic}}
\flushleft{\begin{malayalam}
തുച്ഛമായ ഒരു സുഖാനുഭവമാകുന്നു അത്‌. പിന്നീട് അവര്‍ക്കുള്ള സങ്കേതം നരകമത്രെ. അതെത്ര മോശമായ വാസസ്ഥലം!
\end{malayalam}}
\flushright{\begin{Arabic}
\quranayah[3][198]
\end{Arabic}}
\flushleft{\begin{malayalam}
എന്നാല്‍ തങ്ങളുടെ രക്ഷിതാവിനെ സൂക്ഷിച്ച് ജീവിച്ചതാരോ അവര്‍ക്കാണ് താഴ്ഭാഗത്ത് കൂടി അരുവികള്‍ ഒഴുകുന്ന സ്വര്‍ഗത്തോപ്പുകളുള്ളത്‌. അവരതില്‍ നിത്യവാസികളായിരിക്കും. അല്ലാഹുവിന്‍റെ പക്കല്‍ നിന്നുള്ള സല്‍ക്കാരം! അല്ലാഹുവിന്‍റെ അടുക്കലുള്ളതാകുന്നു പുണ്യവാന്‍മാര്‍ക്ക് ഏറ്റവും ഉത്തമം.
\end{malayalam}}
\flushright{\begin{Arabic}
\quranayah[3][199]
\end{Arabic}}
\flushleft{\begin{malayalam}
തീര്‍ച്ചയായും വേദക്കാരില്‍ ഒരു വിഭാഗമുണ്ട്‌. അല്ലാഹുവിലും, നിങ്ങള്‍ക്ക് അവതരിപ്പിക്കപ്പെട്ട വേദത്തിലും, അവര്‍ക്ക് അവതരിപ്പിക്കപ്പെട്ട വേദത്തിലും അവര്‍ വിശ്വസിക്കും. (അവര്‍) അല്ലാഹുവോട് താഴ്മയുള്ളവരായിരിക്കും. അല്ലാഹുവിന്‍റെ വചനങ്ങള്‍ വിറ്റ് അവര്‍ തുച്ഛമായ വില വാങ്ങുകയില്ല. അവര്‍ക്കാകുന്നു തങ്ങളുടെ രക്ഷിതാവിങ്കല്‍ അവര്‍ അര്‍ഹിക്കുന്ന പ്രതിഫലമുള്ളത്‌. തീര്‍ച്ചയായും അല്ലാഹു അതിവേഗം കണക്ക് നോക്കുന്നവനാകുന്നു.
\end{malayalam}}
\flushright{\begin{Arabic}
\quranayah[3][200]
\end{Arabic}}
\flushleft{\begin{malayalam}
സത്യവിശ്വാസികളേ, നിങ്ങള്‍ ക്ഷമിക്കുകയും ക്ഷമയില്‍ മികവ് കാണിക്കുകയും, പ്രതിരോധ സന്നദ്ധരായിരിക്കുകയും ചെയ്യുക. നിങ്ങള്‍ അല്ലാഹുവെ സൂക്ഷിച്ച് ജീവിക്കുക. നിങ്ങള്‍ വിജയം പ്രാപിച്ചേക്കാം.
\end{malayalam}}
\chapter{\textmalayalam{നിസാഅ് ( സ്ത്രീകള്‍ )}}
\begin{Arabic}
\Huge{\centerline{\basmalah}}\end{Arabic}
\flushright{\begin{Arabic}
\quranayah[4][1]
\end{Arabic}}
\flushleft{\begin{malayalam}
മനുഷ്യരേ, നിങ്ങളെ ഒരേ ആത്മാവില്‍ നിന്ന് സൃഷ്ടിക്കുകയും, അതില്‍ നിന്നുതന്നെ അതിന്‍റെ ഇണയെയും സൃഷ്ടിക്കുകയും, അവര്‍ ഇരുവരില്‍ നിന്നുമായി ധാരാളം പുരുഷന്‍മാരെയും സ്ത്രീകളെയും വ്യാപിപ്പിക്കുകയും ചെയ്തവനായ നിങ്ങളുടെ രക്ഷിതാവിനെ നിങ്ങള്‍ സൂക്ഷിക്കുവിന്‍. ഏതൊരു അല്ലാഹുവിന്‍റെ പേരില്‍ നിങ്ങള്‍ അന്യോന്യം ചോദിച്ചു കൊണ്ടിരിക്കുന്നുവോ അവനെ നിങ്ങള്‍ സൂക്ഷിക്കുക. കുടുംബബന്ധങ്ങളെയും (നിങ്ങള്‍ സൂക്ഷിക്കുക.) തീര്‍ച്ചയായും അല്ലാഹു നിങ്ങളെ നിരീക്ഷിച്ചു കൊണ്ടിരിക്കുന്നവനാകുന്നു.
\end{malayalam}}
\flushright{\begin{Arabic}
\quranayah[4][2]
\end{Arabic}}
\flushleft{\begin{malayalam}
അനാഥകള്‍ക്ക് അവരുടെ സ്വത്തുക്കള്‍ നിങ്ങള്‍ വിട്ടുകൊടുക്കുക. നല്ലതിനുപകരം ദുഷിച്ചത് നിങ്ങള്‍ മാറ്റിയെടുക്കരുത്‌. നിങ്ങളുടെ ധനത്തോട് കൂട്ടിചേര്‍ത്ത് അവരുടെ ധനം നിങ്ങള്‍ തിന്നുകളയുകയുമരുത്‌. തീര്‍ച്ചയായും അത് ഒരു കൊടും പാതകമാകുന്നു.
\end{malayalam}}
\flushright{\begin{Arabic}
\quranayah[4][3]
\end{Arabic}}
\flushleft{\begin{malayalam}
അനാഥകളുടെ കാര്യത്തില്‍ നിങ്ങള്‍ക്കു നീതി പാലിക്കാനാവില്ലെന്ന് നിങ്ങള്‍ ഭയപ്പെടുകയാണെങ്കില്‍ (മറ്റു) സ്ത്രീകളില്‍ നിന്ന് നിങ്ങള്‍ ഇഷ്ടപ്പെടുന്ന രണ്ടോ മൂന്നോ, നാലോ പേരെ വിവാഹം ചെയ്തുകൊള്ളുക. എന്നാല്‍ (അവര്‍ക്കിടയില്‍) നീതിപുലര്‍ത്താനാവില്ലെന്ന് നിങ്ങള്‍ ഭയപ്പെടുകയാണെങ്കില്‍ ഒരുവളെ മാത്രം (വിവാഹം കഴിക്കുക.) അല്ലെങ്കില്‍ നിങ്ങളുടെ അധീനത്തിലുള്ള അടിമസ്ത്രീയെ (ഭാര്യയെപ്പോലെ സ്വീകരിക്കുക.) നിങ്ങള്‍ അതിരുവിട്ട് പോകാതിരിക്കാന്‍ അതാണ് കൂടുതല്‍ അനുയോജ്യമായിട്ടുള്ളത്‌.
\end{malayalam}}
\flushright{\begin{Arabic}
\quranayah[4][4]
\end{Arabic}}
\flushleft{\begin{malayalam}
സ്ത്രീകള്‍ക്ക് അവരുടെ വിവാഹമൂല്യങ്ങള്‍ മനഃസംതൃപ്തിയോട് കൂടി നിങ്ങള്‍ നല്‍കുക. ഇനി അതില്‍ നിന്ന് വല്ലതും സന്‍മനസ്സോടെ അവര്‍ വിട്ടുതരുന്ന പക്ഷം നിങ്ങളത് സന്തോഷപൂര്‍വ്വം സുഖമായി ഭക്ഷിച്ചു കൊള്ളുക.
\end{malayalam}}
\flushright{\begin{Arabic}
\quranayah[4][5]
\end{Arabic}}
\flushleft{\begin{malayalam}
അല്ലാഹു നിങ്ങളുടെ നിലനില്‍പിന്നുള്ള മാര്‍ഗമായി നിശ്ചയിച്ച് തന്നിട്ടുള്ള നിങ്ങളുടെ സ്വത്തുകള്‍ നിങ്ങള്‍ വിവേകമില്ലാത്തവര്‍ക്ക് കൈവിട്ട് കൊടുക്കരുത്‌. എന്നാല്‍ അതില്‍ നിന്നും നിങ്ങള്‍ അവര്‍ക്ക് ഉപജീവനവും വസ്ത്രവും നല്‍കുകയും, അവരോട് മര്യാദയുള്ള വാക്ക് പറയുകയും ചെയ്യുക.
\end{malayalam}}
\flushright{\begin{Arabic}
\quranayah[4][6]
\end{Arabic}}
\flushleft{\begin{malayalam}
അനാഥകളെ നിങ്ങള്‍ പരീക്ഷിച്ച് നോക്കുക. അങ്ങനെ അവര്‍ക്കു വിവാഹപ്രായമെത്തിയാല്‍ നിങ്ങളവരില്‍ കാര്യബോധം കാണുന്ന പക്ഷം അവരുടെ സ്വത്തുക്കള്‍ അവര്‍ക്ക് വിട്ടുകൊടുക്കുക. അവര്‍ (അനാഥകള്‍) വലുതാകുമെന്നത് കണ്ട് അമിതമായും ധൃതിപ്പെട്ടും അത് തിന്നുതീര്‍ക്കരുത്‌. ഇനി (അനാഥരുടെ സംരക്ഷണമേല്‍ക്കുന്ന) വല്ലവനും കഴിവുള്ളവനാണെങ്കില്‍ (അതില്‍ നിന്നു എടുക്കാതെ) മാന്യത പുലര്‍ത്തുകയാണ് വേണ്ടത്‌. വല്ലവനും ദരിദ്രനാണെങ്കില്‍ മര്യാദപ്രകാരം അയാള്‍ക്കതില്‍ നിന്ന് ഭക്ഷിക്കാവുന്നതാണ്‌. എന്നിട്ട് അവരുടെ സ്വത്തുക്കള്‍ അവര്‍ക്ക് നിങ്ങള്‍ ഏല്‍പിച്ചുകൊടുക്കുമ്പോള്‍ നിങ്ങളതിന് സാക്ഷിനിര്‍ത്തേണ്ടതുമാണ്‌. കണക്കു നോക്കുന്നവനായി അല്ലാഹു തന്നെ മതി.
\end{malayalam}}
\flushright{\begin{Arabic}
\quranayah[4][7]
\end{Arabic}}
\flushleft{\begin{malayalam}
മാതാപിതാക്കളും അടുത്ത ബന്ധുക്കളും വിട്ടേച്ചു പോയ ധനത്തില്‍ പുരുഷന്‍മാര്‍ക്ക് ഓഹരിയുണ്ട്‌. മാതാപിതാക്കളും അടുത്ത ബന്ധുക്കളും വിട്ടേച്ചുപോയ ധനത്തില്‍ സ്ത്രീകള്‍ക്കും ഓഹരിയുണ്ട്‌. (ആ ധനം) കുറച്ചാകട്ടെ, കൂടുതലാകട്ടെ. അത് നിര്‍ണയിക്കപ്പെട്ട ഓഹരിയാകുന്നു.
\end{malayalam}}
\flushright{\begin{Arabic}
\quranayah[4][8]
\end{Arabic}}
\flushleft{\begin{malayalam}
(സ്വത്ത്‌) ഭാഗിക്കുന്ന സന്ദര്‍ഭത്തില്‍ (മറ്റു) ബന്ധുക്കളോ, അനാഥകളോ പാവപ്പെട്ടവരോ ഹാജറുണ്ടായാല്‍ അതില്‍ നിന്ന് അവര്‍ക്ക് നിങ്ങള്‍ വല്ലതും നല്‍കുകയും, അവരോട് മര്യാദയുള്ള വാക്ക് പറയുകയും ചെയ്യേണ്ടതാകുന്നു.
\end{malayalam}}
\flushright{\begin{Arabic}
\quranayah[4][9]
\end{Arabic}}
\flushleft{\begin{malayalam}
തങ്ങളുടെ പിന്നില്‍ ദുര്‍ബലരായ സന്താനങ്ങളെ വിട്ടേച്ചുപോയാല്‍ (അവരുടെ ഗതിയെന്താകുമെന്ന്‌) ഭയപ്പെടുന്നവര്‍ (അതേവിധം മറ്റുള്ളവരുടെ മക്കളുടെ കാര്യത്തില്‍) ഭയപ്പെടട്ടെ. അങ്ങനെ അവര്‍ അല്ലാഹുവെ സൂക്ഷിക്കുകയും, ശരിയായ വാക്ക് പറയുകയും ചെയ്യട്ടെ.
\end{malayalam}}
\flushright{\begin{Arabic}
\quranayah[4][10]
\end{Arabic}}
\flushleft{\begin{malayalam}
തീര്‍ച്ചയായും അനാഥകളുടെ സ്വത്തുകള്‍ അന്യായമായി തിന്നുന്നവര്‍ അവരുടെ വയറുകളില്‍ തിന്നു (നിറക്കു) ന്നത് തീ മാത്രമാകുന്നു. പിന്നീട് അവര്‍ നരകത്തില്‍ കത്തിഎരിയുന്നതുമാണ്‌.
\end{malayalam}}
\flushright{\begin{Arabic}
\quranayah[4][11]
\end{Arabic}}
\flushleft{\begin{malayalam}
നിങ്ങളുടെ സന്താനങ്ങളുടെ കാര്യത്തില്‍ അല്ലാഹു നിങ്ങള്‍ക്ക് നിര്‍ദേശം നല്‍കുന്നു; ആണിന് രണ്ട് പെണ്ണിന്‍റെതിന് തുല്യമായ ഓഹരിയാണുള്ളത്‌. ഇനി രണ്ടിലധികം പെണ്‍മക്കളാണുള്ളതെങ്കില്‍ (മരിച്ച ആള്‍) വിട്ടേച്ചു പോയ സ്വത്തിന്‍റെ മൂന്നില്‍ രണ്ടു ഭാഗമാണ് അവര്‍ക്കുള്ളത്‌. ഒരു മകള്‍ മാത്രമാണെങ്കില്‍ അവള്‍ക്ക് പകുതിയാണുള്ളത്‌. മരിച്ച ആള്‍ക്കു സന്താനമുണ്ടെങ്കില്‍ അയാളുടെ മാതാപിതാക്കളില്‍ ഓരോരുത്തര്‍ക്കും അയാള്‍ വിട്ടേച്ചുപോയ സ്വത്തിന്‍റെ ആറിലൊന്നുവീതം ഉണ്ടായിരിക്കുന്നതാണ്‌. ഇനി അയാള്‍ക്ക് സന്താനമില്ലാതിരിക്കുകയും, മാതാപിതാക്കള്‍ അയാളുടെ അനന്തരാവകാശികളായിരിക്കയുമാണെങ്കില്‍ അയാളുടെ മാതാവിന് മൂന്നിലൊരു ഭാഗം ഉണ്ടായിരിക്കും. ഇനി അയാള്‍ക്ക് സഹോദരങ്ങളുണ്ടായിരുന്നാല്‍ അയാളുടെ മാതാവിന് ആറിലൊന്നുണ്ടായിരിക്കും. മരിച്ച ആള്‍ ചെയ്തിട്ടുള്ള വസ്വിയ്യത്തിനും കടമുണ്ടെങ്കില്‍ അതിനും ശേഷമാണ് ഇതെല്ലാം. നിങ്ങളുടെ പിതാക്കളിലും നിങ്ങളുടെ മക്കളിലും ഉപകാരം കൊണ്ട് നിങ്ങളോട് ഏറ്റവും അടുത്തവര്‍ ആരാണെന്ന് നിങ്ങള്‍ക്കറിയില്ല. അല്ലാഹുവിന്‍റെ പക്കല്‍ നിന്നുള്ള (ഓഹരി) നിര്‍ണയമാണിത്‌. തീര്‍ച്ചയായും അല്ലാഹു എല്ലാം അറിയുന്നവനും യുക്തിമാനുമാകുന്നു.
\end{malayalam}}
\flushright{\begin{Arabic}
\quranayah[4][12]
\end{Arabic}}
\flushleft{\begin{malayalam}
നിങ്ങളുടെ ഭാര്യമാര്‍ക്ക് സന്താനമില്ലാത്ത പക്ഷം അവര്‍ വിട്ടേച്ചുപോയ ധനത്തിന്‍റെ പകുതി നിങ്ങള്‍ക്കാകുന്നു. ഇനി അവര്‍ക്ക് സന്താനമുണ്ടായിരുന്നാല്‍ അവര്‍ വിട്ടേച്ചുപോയതിന്‍റെ നാലിലൊന്ന് നിങ്ങള്‍ക്കായിരിക്കും. അവര്‍ ചെയ്യുന്ന വസ്വിയ്യത്തും കടമുണ്ടെങ്കില്‍ അതും കഴിച്ചാണിത്‌. നിങ്ങള്‍ക്ക് സന്താനമില്ലെങ്കില്‍ നിങ്ങള്‍ വിട്ടേച്ചുപോയ ധനത്തില്‍ നിന്ന് നാലിലൊന്നാണ് അവര്‍ക്ക് (ഭാര്യമാര്‍ക്ക്‌) ഉള്ളത്‌. ഇനി നിങ്ങള്‍ക്ക് സന്താനമുണ്ടായിരുന്നാല്‍ നിങ്ങള്‍ വിട്ടേച്ചു പോയതില്‍ നിന്ന് എട്ടിലൊന്നാണ് അവര്‍ക്കുള്ളത്‌. നിങ്ങള്‍ ചെയ്യുന്ന വസ്വിയ്യത്തും കടമുണ്ടെങ്കില്‍ അതും കഴിച്ചാണിത്‌. അനന്തരമെടുക്കുന്ന പുരുഷനോ സ്ത്രീയോ പിതാവും മക്കളുമില്ലാത്ത ആളായിരിക്കുകയും, അയാള്‍ക്ക് (മാതാവൊത്ത) ഒരു സഹോദരനോ സഹോദരിയോ ഉണ്ടായിരിക്കുകയും ചെയ്താല്‍ അവരില്‍ (ആ സഹോദരസഹോദരിമാരില്‍) ഓരോരുത്തര്‍ക്കും ആറില്‍ ഒരംശം ലഭിക്കുന്നതാണ്‌. ഇനി അവര്‍ അതിലധികം പേരുണ്ടെങ്കില്‍ അവര്‍ മൂന്നിലൊന്നില്‍ സമാവകാശികളായിരിക്കും. ദ്രോഹകരമല്ലാത്ത വസ്വിയ്യത്തോ കടമോ ഉണ്ടെങ്കില്‍ അതൊഴിച്ചാണിത്‌. അല്ലാഹുവിങ്കല്‍ നിന്നുള്ള നിര്‍ദേശമത്രെ ഇത്‌. അല്ലാഹു സര്‍വ്വജ്ഞനും സഹനശീലനുമാകുന്നു.
\end{malayalam}}
\flushright{\begin{Arabic}
\quranayah[4][13]
\end{Arabic}}
\flushleft{\begin{malayalam}
അല്ലാഹുവിന്‍റെ നിയമപരിധികളാകുന്നു ഇവയൊക്കെ. ഏതൊരാള്‍ അല്ലാഹുവിനെയും അവന്‍റെ ദൂതനെയും അനുസരിക്കുന്നുവോ അവനെ അല്ലാഹു താഴ്ഭാഗത്ത് കൂടി അരുവികള്‍ ഒഴുകുന്ന സ്വര്‍ഗത്തോപ്പുകളില്‍ പ്രവേശിപ്പിക്കുന്നതാണ്‌. അവരതില്‍ നിത്യവാസികളായിരിക്കും. അതത്രെ മഹത്തായ വിജയം.
\end{malayalam}}
\flushright{\begin{Arabic}
\quranayah[4][14]
\end{Arabic}}
\flushleft{\begin{malayalam}
ആര്‍ അല്ലാഹുവെയും അവന്‍റെ ദൂതനെയും ധിക്കരിക്കുകയും, അവന്‍റെ (നിയമ) പരിധികള്‍ ലംഘിക്കുകയും ചെയ്യുന്നുവോ അവനെ അല്ലാഹു നരകാഗ്നിയില്‍ പ്രവേശിപ്പിക്കും. അവനതില്‍ നിത്യവാസിയായിരിക്കും. അപമാനകരമായ ശിക്ഷയാണ് അവന്നുള്ളത്‌.
\end{malayalam}}
\flushright{\begin{Arabic}
\quranayah[4][15]
\end{Arabic}}
\flushleft{\begin{malayalam}
നിങ്ങളുടെ സ്ത്രീകളില്‍ നിന്ന് നീചവൃത്തിയില്‍ ഏര്‍പെടുന്നവരാരോ അവര്‍ക്കെതിരില്‍ സാക്ഷികളായി നിങ്ങളില്‍ നിന്ന് നാലുപേരെ നിങ്ങള്‍ കൊണ്ട് വരുവിന്‍. അങ്ങനെ അവര്‍ സാക്ഷ്യം വഹിച്ചാല്‍ അവരെ നിങ്ങള്‍ വീടുകളില്‍ തടഞ്ഞു വെച്ചുകൊണ്ടിരിക്കുക. അവരെ മരണം ഏറ്റെടുക്കുകയോ അല്ലാഹു അവര്‍ക്കൊരു മാര്‍ഗം ഉണ്ടാക്കുകയോ ചെയ്യുന്നത് വരെ.
\end{malayalam}}
\flushright{\begin{Arabic}
\quranayah[4][16]
\end{Arabic}}
\flushleft{\begin{malayalam}
നിങ്ങളുടെ കൂട്ടത്തില്‍ നിന്ന് ആ നീചവൃത്തി ചെയ്യുന്ന രണ്ടുപേരെയും നിങ്ങള്‍ പീഡിപ്പിക്കുക. എന്നാല്‍ അവര്‍ ഇരുവരും പശ്ചാത്തപിക്കുകയും നടപടി നന്നാക്കിത്തീര്‍ക്കുകയും ചെയ്യുന്ന പക്ഷം നിങ്ങള്‍ അവരെ വിട്ടേക്കുക. തീര്‍ച്ചയായും അല്ലാഹു പശ്ചാത്താപം ഏറെ സ്വീകരിക്കുന്നവനും കരുണാനിധിയുമാകുന്നു.
\end{malayalam}}
\flushright{\begin{Arabic}
\quranayah[4][17]
\end{Arabic}}
\flushleft{\begin{malayalam}
പശ്ചാത്താപം സ്വീകരിക്കാന്‍ അല്ലാഹു ബാധ്യത ഏറ്റിട്ടുള്ളത് അറിവുകേട് നിമിത്തം തിന്‍മ ചെയ്യുകയും, എന്നിട്ട് താമസിയാതെ പശ്ചാത്തപിക്കുകയും ചെയ്യുന്നവര്‍ക്ക് മാത്രമാകുന്നു. അങ്ങനെയുള്ളവരുടെ പശ്ചാത്താപം അല്ലാഹു സ്വീകരിക്കുന്നതാണ്‌. അല്ലാഹു എല്ലാം അറിയുന്നവനും യുക്തിമാനുമാകുന്നു.
\end{malayalam}}
\flushright{\begin{Arabic}
\quranayah[4][18]
\end{Arabic}}
\flushleft{\begin{malayalam}
പശ്ചാത്താപം എന്നത് തെറ്റുകള്‍ ചെയ്ത് കൊണ്ടിരിക്കുകയും, എന്നിട്ട് മരണം ആസന്നമാകുമ്പോള്‍ ഞാനിതാ പശ്ചാത്തപിച്ചിരിക്കുന്നു എന്ന് പറയുകയും ചെയ്യുന്നവര്‍ക്കുള്ളതല്ല. സത്യനിഷേധികളായിക്കൊണ്ട് മരണമടയുന്നവര്‍ക്കുമുള്ളതല്ല. അങ്ങനെയുള്ളവര്‍ക്ക് വേദനയേറിയ ശിക്ഷയാണ് നാം ഒരുക്കിവെച്ചിട്ടുള്ളത്‌.
\end{malayalam}}
\flushright{\begin{Arabic}
\quranayah[4][19]
\end{Arabic}}
\flushleft{\begin{malayalam}
സത്യവിശ്വാസികളേ, സ്ത്രീകളെ ബലാല്‍ക്കാരമായിട്ട് അനന്തരാവകാശ സ്വത്തായി എടുക്കല്‍ നിങ്ങള്‍ക്ക് അനുവദനീയമല്ല. അവര്‍ക്ക് (ഭാര്യമാര്‍ക്ക്‌) നിങ്ങള്‍ കൊടുത്തിട്ടുള്ളതില്‍ ഒരു ഭാഗം തട്ടിയെടുക്കുവാന്‍ വേണ്ടി നിങ്ങളവരെ മുടക്കിയിടുകയും ചെയ്യരുത്‌. അവര്‍ പ്രത്യക്ഷമായ വല്ല നീചവൃത്തിയും ചെയ്തെങ്കിലല്ലാതെ. അവരോട് നിങ്ങള്‍ മര്യാദയോടെ സഹവര്‍ത്തിക്കേണ്ടതുമുണ്ട്‌. ഇനി നിങ്ങള്‍ക്കവരോട് വെറുപ്പ് തോന്നുന്ന പക്ഷം (നിങ്ങള്‍ മനസ്സിലാക്കുക) നിങ്ങളൊരു കാര്യം വെറുക്കുകയും അതേകാര്യത്തില്‍ അല്ലാഹു ധാരാളം നന്‍മ നിശ്ചയിക്കുകയും ചെയ്തെന്ന് വരാം.
\end{malayalam}}
\flushright{\begin{Arabic}
\quranayah[4][20]
\end{Arabic}}
\flushleft{\begin{malayalam}
നിങ്ങള്‍ ഒരു ഭാര്യയുടെ സ്ഥാനത്ത് മറ്റൊരു ഭാര്യയെ പകരം സ്വീകരിക്കുവാന്‍ ഉദ്ദേശിക്കുന്ന പക്ഷം അവരില്‍ ഒരുവള്‍ക്ക് നിങ്ങള്‍ ഒരു കൂമ്പാരം തന്നെ കൊടുത്തിട്ടുണ്ടായിരുന്നുവെങ്കിലും അതില്‍ നിന്ന് യാതൊന്നും തന്നെ നിങ്ങള്‍ തിരിച്ചുവാങ്ങരുത്‌. കള്ള ആരോപണം ചുമത്തിക്കൊണ്ടും പ്രത്യക്ഷമായ അധര്‍മ്മം ചെയ്തുകൊണ്ടും നിങ്ങളത് മേടിക്കുകയോ?
\end{malayalam}}
\flushright{\begin{Arabic}
\quranayah[4][21]
\end{Arabic}}
\flushleft{\begin{malayalam}
നിങ്ങള്‍ അന്യോന്യം കൂടിച്ചേരുകയും അവര്‍ നിങ്ങളില്‍ നിന്ന് കനത്ത ഒരു കരാര്‍ വാങ്ങുകയും ചെയ്തുകഴിഞ്ഞിരിക്കെ നിങ്ങള്‍ അതങ്ങനെ മേടിക്കും?
\end{malayalam}}
\flushright{\begin{Arabic}
\quranayah[4][22]
\end{Arabic}}
\flushleft{\begin{malayalam}
നിങ്ങളുടെ പിതാക്കള്‍ വിവാഹം ചെയ്ത സ്ത്രീകളെ നിങ്ങള്‍ വിവാഹം കഴിക്കരുത്‌; മുമ്പ് ചെയ്തുപോയതൊഴികെ. തീര്‍ച്ചയായും അത് ഒരു നീചവൃത്തിയും വെറുക്കപ്പെട്ട കാര്യവും ദുഷിച്ച മാര്‍ഗവുമാകുന്നു.
\end{malayalam}}
\flushright{\begin{Arabic}
\quranayah[4][23]
\end{Arabic}}
\flushleft{\begin{malayalam}
നിങ്ങളുടെ മാതാക്കള്‍, പുത്രിമാര്‍, സഹോദരിമാര്‍, പിതൃസഹോദരിമാര്‍, മാതൃസഹോദരിമാര്‍, നിങ്ങളെ മുലകുടിപ്പിച്ച പോറ്റമ്മമാര്‍, മുലകുടി മുഖേനയുള്ള നിങ്ങളുടെ സഹോദരിമാര്‍, നിങ്ങളുടെ ഭാര്യാമാതാക്കള്‍ എന്നിവര്‍ (അവരെ വിവാഹം ചെയ്യല്‍) നിങ്ങള്‍ക്ക് നിഷിദ്ധമാക്കപ്പെട്ടിരിക്കുന്നു. നിങ്ങള്‍ ലൈംഗികവേഴ്ചയില്‍ ഏര്‍പെട്ടിട്ടുള്ള നിങ്ങളുടെ ഭാര്യമാരുടെ സന്താനങ്ങളായി നിങ്ങളുടെ സംരക്ഷണത്തിലുള്ള വളര്‍ത്ത് പുത്രിമാരും (അവരെ വിവാഹം ചെയ്യുന്നതും നിഷിദ്ധമാക്കപ്പെട്ടിരിക്കുന്നു). ഇനി നിങ്ങള്‍ അവരുമായി ലൈംഗികവേഴ്ചയില്‍ ഏര്‍പെട്ടിട്ടില്ലെങ്കില്‍ (അവരുടെ മക്കളെ വേള്‍ക്കുന്നതില്‍) നിങ്ങള്‍ക്കു കുറ്റമില്ല. നിങ്ങളുടെ മുതുകില്‍ നിന്ന് പിറന്ന പുത്രന്‍മാരുടെ ഭാര്യമാരും (നിങ്ങള്‍ക്ക് നിഷിദ്ധമാക്കപ്പെട്ടിരിക്കുന്നു.) രണ്ടുസഹോദരിമാരെ ഒന്നിച്ച് ഭാര്യമാരാക്കുന്നതും (നിഷിദ്ധമാകുന്നു.) മുമ്പ് ചെയ്ത് പോയതൊഴികെ. തീര്‍ച്ചയായും അല്ലാഹു ഏറെ പൊറുക്കുന്നവനും കരുണാനിധിയുമാകുന്നു.
\end{malayalam}}
\flushright{\begin{Arabic}
\quranayah[4][24]
\end{Arabic}}
\flushleft{\begin{malayalam}
(മറ്റുള്ളവരുടെ) വിവാഹബന്ധത്തിലിരിക്കുന്ന സ്ത്രീകളും (നിങ്ങള്‍ക്ക് നിഷിദ്ധമാക്കപ്പെട്ടിരിക്കുന്നു.) നിങ്ങളുടെ കൈകള്‍ ഉടമപ്പെടുത്തിയവര്‍ (അടിമസ്ത്രീകള്‍) ഒഴികെ. നിങ്ങളുടെ മേല്‍ അല്ലാഹുവിന്‍റെ നിയമമത്രെ ഇത്‌. അതിന്നപ്പുറമുള്ള സ്ത്രീകളുമായി നിങ്ങളുടെ ധനം (മഹ്‌റായി) നല്‍കിക്കൊണ്ട് നിങ്ങള്‍ (വിവാഹബന്ധം) തേടുന്നത് നിങ്ങള്‍ക്ക് അനുവദിക്കപ്പെട്ടിരിക്കുന്നു. നിങ്ങള്‍ വൈവാഹിക ജീവിതം ലക്ഷ്യമാക്കുന്നവരായിരിക്കണം. നീചവൃത്തി ആഗ്രഹിക്കുന്നവരാകരുത്‌. അങ്ങനെ അവരില്‍ നിന്ന് നിങ്ങള്‍ വല്ല സുഖവുമനുഭവിച്ചാല്‍ അവര്‍ക്കുള്ള വിവാഹമൂല്യം ഒരു ബാധ്യത എന്ന നിലയില്‍ നിങ്ങള്‍ നല്‍കേണ്ടതാണ്‌. ബാധ്യത (വിവാഹമൂല്യം) നിശ്ചയിച്ചതിനു ശേഷം നിങ്ങള്‍ അന്യോന്യം തൃപ്തിപ്പെട്ട് വല്ല വിട്ടുവീഴ്ചയും ചെയ്യുന്നതില്‍ നിങ്ങള്‍ക്ക് കുറ്റമൊന്നുമില്ല. തീര്‍ച്ചയായും അല്ലാഹു എല്ലാം അറിയുന്നവനും യുക്തിമാനുമാകുന്നു.
\end{malayalam}}
\flushright{\begin{Arabic}
\quranayah[4][25]
\end{Arabic}}
\flushleft{\begin{malayalam}
നിങ്ങളിലാര്‍ക്കെങ്കിലും സത്യവിശ്വാസിനികളായ സ്വതന്ത്രസ്ത്രീകളെ വിവാഹം കഴിക്കാന്‍ സാമ്പത്തിക ശേഷിയില്ലെങ്കില്‍ നിങ്ങളുടെ കൈകള്‍ ഉടമപ്പെടുത്തിയ സത്യവിശ്വാസിനികളായ ദാസിമാരില്‍ ആരെയെങ്കിലും (ഭാര്യമാരായി സ്വീകരിക്കാവുന്നതാണ്‌.) അല്ലാഹുവാകുന്നു നിങ്ങളുടെ വിശ്വാസത്തെപ്പറ്റി നന്നായി അറിയുന്നവന്‍. നിങ്ങളില്‍ ചിലര്‍ ചിലരില്‍ നിന്നുണ്ടായവരാണല്ലോ. അങ്ങനെ അവരെ (ആ ദാസിമാരെ) അവരുടെ രക്ഷാകര്‍ത്താക്കളുടെ അനുമതിപ്രകാരം നിങ്ങള്‍ വിവാഹം കഴിച്ച് കൊള്ളുക. അവരുടെ വിവാഹമൂല്യം മര്യാദപ്രകാരം അവര്‍ക്ക് നിങ്ങള്‍ നല്‍കുകയും ചെയ്യുക. മ്ലേച്ഛവൃത്തിയില്‍ ഏര്‍പെടാത്തവരും രഹസ്യവേഴ്ചക്കാരെ സ്വീകരിക്കാത്തവരുമായ പതിവ്രതകളായിരിക്കണം അവര്‍. അങ്ങനെ അവര്‍ വൈവാഹിക ജീവിതത്തിന്‍റെ സംരക്ഷണത്തിലായിക്കഴിഞ്ഞിട്ട് അവര്‍ മ്ലേച്ഛവൃത്തിയില്‍ ഏര്‍പെടുന്ന പക്ഷം സ്വതന്ത്രസ്ത്രീകള്‍ക്കുള്ളതിന്‍റെ പകുതി ശിക്ഷ അവര്‍ക്കുണ്ടായിരിക്കും. നിങ്ങളുടെ കൂട്ടത്തില്‍ (വിവാഹം കഴിച്ചില്ലെങ്കില്‍) വിഷമിക്കുമെന്ന് ഭയപ്പെടുന്നവര്‍ക്കാകുന്നു അത്‌. (അടിമസ്ത്രീകളെ ഭാര്യമാരായി സ്വീകരിക്കാനുള്ള അനുവാദം.) എന്നാല്‍ നിങ്ങള്‍ ക്ഷമിച്ചിരിക്കുന്നതാകുന്നു നിങ്ങള്‍ക്ക് കൂടുതല്‍ ഉത്തമം. അല്ലാഹു ഏറെ പൊറുക്കുന്നവനും കരുണാനിധിയുമാകുന്നു.
\end{malayalam}}
\flushright{\begin{Arabic}
\quranayah[4][26]
\end{Arabic}}
\flushleft{\begin{malayalam}
നിങ്ങള്‍ക്ക് (കാര്യങ്ങള്‍) വിവരിച്ചുതരുവാനും, നിങ്ങളുടെ മുന്‍ഗാമികളുടെ നല്ല നടപടികള്‍ നിങ്ങള്‍ക്ക് കാട്ടിത്തരുവാനും നിങ്ങളുടെ പശ്ചാത്താപം സ്വീകരിക്കുവാനും, അല്ലാഹു ഉദ്ദേശിക്കുന്നു. അല്ലാഹു എല്ലാം അറിയുന്നവനും യുക്തിമാനുമാകുന്നു.
\end{malayalam}}
\flushright{\begin{Arabic}
\quranayah[4][27]
\end{Arabic}}
\flushleft{\begin{malayalam}
അല്ലാഹു നിങ്ങളുടെ പശ്ചാത്താപം സ്വീകരിക്കാന്‍ ഉദ്ദേശിക്കുന്നു. എന്നാല്‍ തന്നിഷ്ടങ്ങളെ പിന്‍പറ്റി ജീവിക്കുന്നവര്‍ ഉദ്ദേശിക്കുന്നത് നിങ്ങള്‍ (നേര്‍വഴിയില്‍ നിന്ന്‌) വന്‍തോതില്‍ തെറ്റിപ്പോകണമെന്നാണ്‌.
\end{malayalam}}
\flushright{\begin{Arabic}
\quranayah[4][28]
\end{Arabic}}
\flushleft{\begin{malayalam}
നിങ്ങള്‍ക്ക് ഭാരം കുറച്ചുതരണമെന്ന് അല്ലാഹു ഉദ്ദേശിക്കുന്നു. ദുര്‍ബലനായിക്കൊണ്ടാണ് മനുഷ്യന്‍ സൃഷ്ടിക്കപ്പെട്ടിരിക്കുന്നത്‌.
\end{malayalam}}
\flushright{\begin{Arabic}
\quranayah[4][29]
\end{Arabic}}
\flushleft{\begin{malayalam}
സത്യവിശ്വാസികളേ, നിങ്ങള്‍ പരസ്പരം സംതൃപ്തിയോടുകൂടി നടത്തുന്ന കച്ചവട ഇടപാടു മുഖേനയല്ലാതെ നിങ്ങളുടെ സ്വത്തുക്കള്‍ അന്യായമായി നിങ്ങള്‍ അന്യോന്യം എടുത്ത് തിന്നരുത്‌. നിങ്ങള്‍ നിങ്ങളെത്തന്നെ കൊലപ്പെടുത്തുകയും ചെയ്യരുത്‌. തീര്‍ച്ചയായും അല്ലാഹു നിങ്ങളോട് കരുണയുള്ളവനാകുന്നു.
\end{malayalam}}
\flushright{\begin{Arabic}
\quranayah[4][30]
\end{Arabic}}
\flushleft{\begin{malayalam}
ആരെങ്കിലും അതിക്രമമായും അന്യായമായും അങ്ങനെ ചെയ്യുന്ന പക്ഷം നാമവനെ നരകാഗ്നിയിലിട്ട് കരിക്കുന്നതാണ്‌. അത് അല്ലാഹുവിന് എളുപ്പമുള്ള കാര്യമാകുന്നു.
\end{malayalam}}
\flushright{\begin{Arabic}
\quranayah[4][31]
\end{Arabic}}
\flushleft{\begin{malayalam}
നിങ്ങളോട് നിരോധിക്കപ്പെടുന്ന വന്‍പാപങ്ങള്‍ നിങ്ങള്‍ വര്‍ജ്ജിക്കുന്ന പക്ഷം നിങ്ങളുടെ തിന്‍മകളെ നിങ്ങളില്‍ നിന്ന് നാം മായ്ച്ചുകളയുകയും, മാന്യമായ ഒരു സ്ഥാനത്ത് നിങ്ങളെ നാം പ്രവേശിപ്പിക്കുകയും ചെയ്യുന്നതാണ്‌.
\end{malayalam}}
\flushright{\begin{Arabic}
\quranayah[4][32]
\end{Arabic}}
\flushleft{\begin{malayalam}
നിങ്ങളില്‍ ചിലര്‍ക്ക് ചിലരെക്കാള്‍ കൂടുതലായി അല്ലാഹു നല്‍കിയ അനുഗ്രഹങ്ങളോട് നിങ്ങള്‍ക്ക് മോഹം തോന്നരുത്‌. പുരുഷന്‍മാര്‍ സമ്പാദിച്ചുണ്ടാക്കിയതിന്‍റെ ഓഹരി അവര്‍ക്കുണ്ട്‌. സ്ത്രീകള്‍ സമ്പാദിച്ചുണ്ടാക്കിയതിന്‍റെ ഓഹരി അവര്‍ക്കുമുണ്ട്‌. അല്ലാഹുവോട് അവന്‍റെ ഔദാര്യത്തില്‍ നിന്ന് നിങ്ങള്‍ ആവശ്യപ്പെട്ടുകൊള്ളുക. തീര്‍ച്ചയായും അല്ലാഹു ഏത് കാര്യത്തെപ്പറ്റിയും അറിവുള്ളവനാകുന്നു.
\end{malayalam}}
\flushright{\begin{Arabic}
\quranayah[4][33]
\end{Arabic}}
\flushleft{\begin{malayalam}
ഏതൊരാള്‍ക്കും തന്‍റെ മാതാപിതാക്കളും അടുത്ത ബന്ധുക്കളും വിട്ടേച്ച് പോയ സ്വത്തിന് നാം അവകാശികളെ നിശ്ചയിച്ചിട്ടുണ്ട്‌. നിങ്ങളുടെ വലംകൈകള്‍ ബന്ധം സ്ഥാപിച്ചിട്ടുള്ളവര്‍ക്കും അവരുടെ ഓഹരി നിങ്ങള്‍ കൊടുക്കുക. തീര്‍ച്ചയായും അല്ലാഹു എല്ലാ കാര്യങ്ങള്‍ക്കും സാക്ഷിയാകുന്നു.
\end{malayalam}}
\flushright{\begin{Arabic}
\quranayah[4][34]
\end{Arabic}}
\flushleft{\begin{malayalam}
പുരുഷന്‍മാര്‍ സ്ത്രീകളുടെ മേല്‍ നിയന്ത്രണാധികാരമുള്ളവരാകുന്നു. മനുഷ്യരില്‍ ഒരു വിഭാഗത്തിന് മറു വിഭാഗത്തേക്കാള്‍ അല്ലാഹു കൂടുതല്‍ കഴിവ് നല്‍കിയത് കൊണ്ടും, (പുരുഷന്‍മാര്‍) അവരുടെ ധനം ചെലവഴിച്ചതുകൊണ്ടുമാണത്‌. അതിനാല്‍ നല്ലവരായ സ്ത്രീകള്‍ അനുസരണശീലമുള്ളവരും, അല്ലാഹു സംരക്ഷിച്ച പ്രകാരം (പുരുഷന്‍മാരുടെ) അഭാവത്തില്‍ (സംരക്ഷിക്കേണ്ടതെല്ലാം) സംരക്ഷിക്കുന്നവരുമാണ്‌. എന്നാല്‍ അനുസരണക്കേട് കാണിക്കുമെന്ന് നിങ്ങള്‍ ആശങ്കിക്കുന്ന സ്ത്രീകളെ നിങ്ങള്‍ ഉപദേശിക്കുക. കിടപ്പറകളില്‍ അവരുമായി അകന്നു നില്‍ക്കുക. അവരെ അടിക്കുകയും ചെയ്ത് കൊള്ളുക. എന്നിട്ടവര്‍ നിങ്ങളെ അനുസരിക്കുന്ന പക്ഷം പിന്നെ നിങ്ങള്‍ അവര്‍ക്കെതിരില്‍ ഒരു മാര്‍ഗവും തേടരുത്‌. തീര്‍ച്ചയായും അല്ലാഹു ഉന്നതനും മഹാനുമാകുന്നു.
\end{malayalam}}
\flushright{\begin{Arabic}
\quranayah[4][35]
\end{Arabic}}
\flushleft{\begin{malayalam}
ഇനി, അവര്‍ (ദമ്പതിമാര്‍) തമ്മില്‍ ഭിന്നിച്ച് പോകുമെന്ന് നിങ്ങള്‍ ഭയപ്പെടുന്ന പക്ഷം അവന്‍റെ ആള്‍ക്കാരില്‍ നിന്ന് ഒരു മദ്ധ്യസ്ഥനെയും അവളുടെ ആള്‍ക്കാരില്‍ നിന്ന് ഒരു മദ്ധ്യസ്ഥനെയും നിങ്ങള്‍ നിയോഗിക്കുക. ഇരു വിഭാഗവും അനുരഞ്ജനമാണ് ഉദ്ദേശിക്കുന്നതെങ്കില്‍ അല്ലാഹു അവര്‍ക്കിടയില്‍ യോജിപ്പുണ്ടാക്കുന്നതാണ്‌. തീര്‍ച്ചയായും അല്ലാഹു സര്‍വ്വജ്ഞനും സൂക്ഷ്മജ്ഞനുമാകുന്നു.
\end{malayalam}}
\flushright{\begin{Arabic}
\quranayah[4][36]
\end{Arabic}}
\flushleft{\begin{malayalam}
നിങ്ങള്‍ അല്ലാഹുവെ ആരാധിക്കുകയും അവനോട് യാതൊന്നും പങ്കുചേര്‍ക്കാതിരിക്കുകയും മാതാപിതാക്കളോട് നല്ല നിലയില്‍ വര്‍ത്തിക്കുകയും ചെയ്യുക. ബന്ധുക്കളോടും അനാഥകളോടും പാവങ്ങളോടും കുടുംബബന്ധമുള്ള അയല്‍ക്കാരോടും അന്യരായ അയല്‍ക്കാരോടും സഹവാസിയോടും വഴിപോക്കനോടും നിങ്ങളുടെ വലതുകൈകള്‍ ഉടമപ്പെടുത്തിയ അടിമകളോടും നല്ലനിലയില്‍ വര്‍ത്തിക്കുക. പൊങ്ങച്ചക്കാരനും ദുരഭിമാനിയുമായിട്ടുള്ള ആരെയും അല്ലാഹു ഒരിക്കലും ഇഷ്ടപ്പെടുകയില്ല.
\end{malayalam}}
\flushright{\begin{Arabic}
\quranayah[4][37]
\end{Arabic}}
\flushleft{\begin{malayalam}
പിശുക്ക് കാണിക്കുകയും, പിശുക്ക് കാണിക്കാന്‍ ജനങ്ങളെ പ്രേരിപ്പിക്കുകയും, തങ്ങള്‍ക്ക് അല്ലാഹു തന്‍റെ ഔദാര്യം കൊണ്ട് നല്‍കിയ അനുഗ്രഹം മറച്ചു വെക്കുകയും ചെയ്യുന്നവരാണവര്‍. ആ നന്ദികെട്ടവര്‍ക്ക് അപമാനകരമായ ശിക്ഷയാണ് നാം ഒരുക്കിവെച്ചിരിക്കുന്നത്‌.
\end{malayalam}}
\flushright{\begin{Arabic}
\quranayah[4][38]
\end{Arabic}}
\flushleft{\begin{malayalam}
ജനങ്ങളെ കാണിക്കുവാനായി തങ്ങളുടെ സ്വത്തുക്കള്‍ ചെലവഴിക്കുന്നവരും, അല്ലാഹുവിലോ അന്ത്യദിനത്തിലോ വിശ്വാസമില്ലാത്തവരുമാണവര്‍. പിശാചാണ് ഒരാളുടെ കൂട്ടാളിയാകുന്നതെങ്കില്‍ അവന്‍ എത്ര ദുഷിച്ച ഒരു കൂട്ടുകാരന്‍!
\end{malayalam}}
\flushright{\begin{Arabic}
\quranayah[4][39]
\end{Arabic}}
\flushleft{\begin{malayalam}
അവര്‍ അല്ലാഹുവിലും അന്ത്യദിനത്തിലും വിശ്വസിക്കുകയും, അല്ലാഹു അവര്‍ക്ക് നല്‍കിയതില്‍ നിന്ന് ചെലവഴിക്കുകയും ചെയ്തു പോയാല്‍ അവര്‍ക്കെന്തൊരു ദോഷമാണുള്ളത്‌? അല്ലാഹു അവരെ പറ്റി നന്നായി അറിയുന്നവനാകുന്നു.
\end{malayalam}}
\flushright{\begin{Arabic}
\quranayah[4][40]
\end{Arabic}}
\flushleft{\begin{malayalam}
തീര്‍ച്ചയായും അല്ലാഹു ഒരു അണുവോളം അനീതി കാണിക്കുകയില്ല. വല്ല നന്‍മയുമാണുള്ളതെങ്കില്‍ അതവന്‍ ഇരട്ടിച്ച് കൊടുക്കുകയും, അവന്‍റെ പക്കല്‍ നിന്നുള്ള വമ്പിച്ച പ്രതിഫലം നല്‍കുകയും ചെയ്യുന്നതാണ്‌.
\end{malayalam}}
\flushright{\begin{Arabic}
\quranayah[4][41]
\end{Arabic}}
\flushleft{\begin{malayalam}
എന്നാല്‍ ഓരോ സമുദായത്തില്‍ നിന്നും ഓരോ സാക്ഷിയെ നാം കൊണ്ട് വരികയും ഇക്കൂട്ടര്‍ക്കെതിരില്‍ നിന്നെ നാം സാക്ഷിയായി കൊണ്ട് വരികയും ചെയ്യുമ്പോള്‍ എന്തായിരിക്കും അവസ്ഥ!
\end{malayalam}}
\flushright{\begin{Arabic}
\quranayah[4][42]
\end{Arabic}}
\flushleft{\begin{malayalam}
അവിശ്വസിക്കുകയും റസൂലിനെ ധിക്കരിക്കുകയും ചെയ്തവര്‍ ആ ദിവസം കൊതിച്ചു പോകും; അവരെ മൂടിക്കൊണ്ട് ഭൂമി നിരപ്പാക്കപ്പെട്ടിരുന്നു വെങ്കില്‍ എത്ര നന്നായിരുന്നുവെന്ന്‌. ഒരു വിവരവും അല്ലാഹുവില്‍ നിന്ന് അവര്‍ക്ക് ഒളിച്ചു വെക്കാനാവില്ല.
\end{malayalam}}
\flushright{\begin{Arabic}
\quranayah[4][43]
\end{Arabic}}
\flushleft{\begin{malayalam}
സത്യവിശ്വാസികളേ, ലഹരിബാധിച്ചവരായിക്കൊണ്ട് നിങ്ങള്‍ നമസ്കാരത്തെ സമീപിക്കരുത്‌; നിങ്ങള്‍ പറയുന്നതെന്തെന്ന് നിങ്ങള്‍ക്ക് ബോധമുണ്ടാകുന്നത് വരെ. ജനാബത്തുകാരായിരിക്കുമ്പോള്‍ നിങ്ങള്‍ കുളിക്കുന്നത് വരെയും (നമസ്കാരത്തെ സമീപിക്കരുത്‌.) നിങ്ങള്‍ വഴി കടന്ന് പോകുന്നവരായിക്കൊണ്ടല്ലാതെ. നിങ്ങള്‍ രോഗികളായിരിക്കുകയോ യാത്രയിലാവുകയോ ചെയ്താല്‍- അല്ലെങ്കില്‍ നിങ്ങളിലൊരാള്‍ മലമൂത്രവിസര്‍ജ്ജനം കഴിഞ്ഞ് വരികയോ, സ്ത്രീകളുമായി സമ്പര്‍ക്കം നടത്തുകയോ ചെയ്തുവെങ്കില്‍ -എന്നിട്ട് നിങ്ങള്‍ക്ക് വെള്ളം കിട്ടിയതുമില്ലെങ്കില്‍ നിങ്ങള്‍ ശുദ്ധിയുള്ള ഭൂമുഖം തേടിക്കൊള്ളുക. എന്നിട്ടതുകൊണ്ട് നിങ്ങളുടെ മുഖങ്ങളും കൈകളും തടവുക. തീര്‍ച്ചയായും അല്ലാഹു ഏറെ മാപ്പുനല്‍കുന്നവനും പൊറുക്കുന്നവനുമാകുന്നു.
\end{malayalam}}
\flushright{\begin{Arabic}
\quranayah[4][44]
\end{Arabic}}
\flushleft{\begin{malayalam}
വേദഗ്രന്ഥത്തില്‍ നിന്ന് ഒരു വിഹിതം നല്‍കപ്പെട്ടവരെ നീ കണ്ടില്ലേ? അവര്‍ ദുര്‍മാര്‍ഗം വിലയ്ക്ക് വാങ്ങുകയും, നിങ്ങള്‍ വഴിതെറ്റിപ്പോകണമെന്ന് ആഗ്രഹിക്കുകയും ചെയ്യുന്നു.
\end{malayalam}}
\flushright{\begin{Arabic}
\quranayah[4][45]
\end{Arabic}}
\flushleft{\begin{malayalam}
അല്ലാഹു നിങ്ങളുടെ ശത്രുക്കളെപ്പറ്റി നന്നായി അറിയുന്നവനാകുന്നു. നിങ്ങള്‍ക്ക് രക്ഷകനായി അല്ലാഹു മതി, സഹായിയായും അല്ലാഹു തന്നെ മതി.
\end{malayalam}}
\flushright{\begin{Arabic}
\quranayah[4][46]
\end{Arabic}}
\flushleft{\begin{malayalam}
യഹൂദരില്‍ പെട്ടവരത്രെ (ആ ശത്രുക്കള്‍.) വാക്കുകളെ അവര്‍ സ്ഥാനം തെറ്റിച്ച് പ്രയോഗിക്കുന്നു. തങ്ങളുടെ നാവുകള്‍ വളച്ചൊടിച്ച് കൊണ്ടും, മതത്തെ കുത്തിപ്പറഞ്ഞ് കൊണ്ടും സമിഅ്നാ വഅസൈനാ എന്നും ഇസ്മഅ് ഗൈറ മുസ്മഅ് എന്നും റാഇനാ എന്നും അവര്‍ പറയുന്നു. സമിഅ്നാ വഅത്വഅ്നാ (ഞങ്ങള്‍ കേള്‍ക്കുകയും അനുസരിക്കുകയും ചെയ്തിരിക്കുന്നു) എന്നും ഇസ്മഅ് (കേള്‍ക്കണേ) എന്നും ഉന്‍ളുര്‍നാ (ഞങ്ങളെ ഗൌനിക്കണേ) എന്നും അവര്‍ പറഞ്ഞിരുന്നെങ്കില്‍ അതവര്‍ക്ക് കൂടുതല്‍ ഉത്തമവും വക്രതയില്ലാത്തതും ആകുമായിരുന്നു. പക്ഷെ അല്ലാഹു അവരുടെ നിഷേധം കാരണമായി അവരെ ശപിച്ചിരിക്കുന്നു. അതിനാല്‍ അവര്‍ വിശ്വസിക്കുകയില്ല; ചുരുക്കത്തിലല്ലാതെ.
\end{malayalam}}
\flushright{\begin{Arabic}
\quranayah[4][47]
\end{Arabic}}
\flushleft{\begin{malayalam}
ഹേ; വേദഗ്രന്ഥം നല്‍കപ്പെട്ടവരേ, നിങ്ങളുടെ പക്കലുള്ള വേദത്തെ സത്യപ്പെടുത്തിക്കൊണ്ട് നാം അവതരിപ്പിച്ചതില്‍ നിങ്ങള്‍ വിശ്വസിക്കുവിന്‍. നാം ചില മുഖങ്ങള്‍ തുടച്ചുനീക്കിയിട്ട് അവയെ പിന്‍വശങ്ങളിലേക്ക് മാറ്റുന്നതിന് മുമ്പായി, അല്ലെങ്കില്‍ ശബ്ബത്തിന്‍റെ ആള്‍ക്കാരെ നാം ശപിച്ചത് പോലെ നിങ്ങളെയും ശപിക്കുന്നതിന്നുമുമ്പായി നിങ്ങള്‍ വിശ്വസിക്കുവിന്‍. അല്ലാഹുവിന്‍റെ കല്‍പന പ്രാവര്‍ത്തികമാക്കപ്പെടുക തന്നെ ചെയ്യും.
\end{malayalam}}
\flushright{\begin{Arabic}
\quranayah[4][48]
\end{Arabic}}
\flushleft{\begin{malayalam}
തന്നോട് പങ്കുചേര്‍ക്കപ്പെടുന്നത് അല്ലാഹു ഒരിക്കലും പൊറുക്കുകയില്ല. അതൊഴിച്ചുള്ളതെല്ലാം അവന്‍ ഉദ്ദേശിക്കുന്നവര്‍ക്ക് അവന്‍ പൊറുത്തുകൊടുക്കുന്നതാണ്‌. ആര്‍ അല്ലാഹുവോട് പങ്കുചേര്‍ത്തുവോ അവന്‍ തീര്‍ച്ചയായും ഗുരുതരമായ ഒരു കുറ്റകൃത്യമാണ് ചമച്ചുണ്ടാക്കിയിരിക്കുന്നത്‌.
\end{malayalam}}
\flushright{\begin{Arabic}
\quranayah[4][49]
\end{Arabic}}
\flushleft{\begin{malayalam}
സ്വയം പരിശുദ്ധരെന്ന് അവകാശപ്പെടുന്നവരെ നീ കണ്ടില്ലേ? എന്നാല്‍ അല്ലാഹു അവന്‍ ഉദ്ദേശിക്കുന്നവരെ പരിശുദ്ധരാക്കുന്നു. അവരോട് ഒരു തരിമ്പും അനീതി കാണിക്കപ്പെടുന്നതല്ല.
\end{malayalam}}
\flushright{\begin{Arabic}
\quranayah[4][50]
\end{Arabic}}
\flushleft{\begin{malayalam}
എങ്ങനെയാണ് അവര്‍ അല്ലാഹുവിന്‍റെ പേരില്‍ കള്ളം കെട്ടിച്ചമയ്ക്കുന്നത് എന്ന് നോക്കൂ. പ്രത്യക്ഷമായ കുറ്റമായിട്ട് അതു തന്നെ മതി.
\end{malayalam}}
\flushright{\begin{Arabic}
\quranayah[4][51]
\end{Arabic}}
\flushleft{\begin{malayalam}
വേദത്തില്‍ നിന്ന് ഒരു വിഹിതം നല്‍കപ്പെട്ടവരെ നീ നോക്കിയില്ലെ? അവര്‍ ക്ഷുദ്ര വിദ്യകളിലും ദുര്‍മൂര്‍ത്തികളിലും വിശ്വസിക്കുന്നു. സത്യനിഷേധികളെപ്പറ്റി അവര്‍ പറയുന്നു; ഇക്കൂട്ടരാണ് വിശ്വാസികളെക്കാള്‍ നേര്‍മാര്‍ഗം പ്രാപിച്ചവരെന്ന്‌.
\end{malayalam}}
\flushright{\begin{Arabic}
\quranayah[4][52]
\end{Arabic}}
\flushleft{\begin{malayalam}
എന്നാല്‍ അവരെയാണ് അല്ലാഹു ശപിച്ചിരിക്കുന്നത്‌. ഏതൊരുവനെ അല്ലാഹു ശപിച്ചിരിക്കുന്നുവോ അവന്ന് ഒരു സഹായിയെയും നീ കണ്ടെത്തുകയില്ല.
\end{malayalam}}
\flushright{\begin{Arabic}
\quranayah[4][53]
\end{Arabic}}
\flushleft{\begin{malayalam}
അതല്ല, ആധിപത്യത്തില്‍ വല്ല വിഹിതവും അവര്‍ക്കുണ്ടോ? എങ്കില്‍ ഒരു അണുവോളവും അവര്‍ മനുഷ്യര്‍ക്ക് നല്‍കുമായിരുന്നില്ല.
\end{malayalam}}
\flushright{\begin{Arabic}
\quranayah[4][54]
\end{Arabic}}
\flushleft{\begin{malayalam}
അതല്ല, അല്ലാഹു അവന്‍റെ ഔദാര്യത്തില്‍ നിന്ന് മറ്റു മനുഷ്യര്‍ക്ക് നല്‍കിയിട്ടുള്ളതിന്‍റെ പേരില്‍ അവര്‍ അസൂയപ്പെടുകയാണോ? എന്നാല്‍ ഇബ്രാഹീം കുടുംബത്തിന് നാം വേദവും ജ്ഞാനവും നല്‍കിയിട്ടുണ്ട്‌. അവര്‍ക്ക് നാം മഹത്തായ ആധിപത്യവും നല്‍കിയിട്ടുണ്ട്‌.
\end{malayalam}}
\flushright{\begin{Arabic}
\quranayah[4][55]
\end{Arabic}}
\flushleft{\begin{malayalam}
അതില്‍ വിശ്വസിച്ച ഒരു വിഭാഗം അവരുടെ കൂട്ടത്തിലുണ്ട്‌. അതില്‍ നിന്ന് പിന്തിരിഞ്ഞ വിഭാഗവും അവരിലുണ്ട്‌. (അവര്‍ക്ക്‌) കത്തിജ്വലിക്കുന്ന നരകാഗ്നി തന്നെ മതി.
\end{malayalam}}
\flushright{\begin{Arabic}
\quranayah[4][56]
\end{Arabic}}
\flushleft{\begin{malayalam}
തീര്‍ച്ചയായും നമ്മുടെ തെളിവുകള്‍ നിഷേധിച്ചവരെ നാം നരകത്തിലിട്ട് കരിക്കുന്നതാണ്‌. അവരുടെ തൊലികള്‍ വെന്തുപോകുമ്പോഴെല്ലാം അവര്‍ക്ക് നാം വേറെ തൊലികള്‍ മാറ്റികൊടുക്കുന്നതാണ്‌. അവര്‍ ശിക്ഷ ആസ്വദിച്ചു കൊണ്ടിരിക്കാന്‍ വേണ്ടിയാണത്‌. തീര്‍ച്ചയായും അല്ലാഹു പ്രതാപവാനും യുക്തിമാനുമാകുന്നു.
\end{malayalam}}
\flushright{\begin{Arabic}
\quranayah[4][57]
\end{Arabic}}
\flushleft{\begin{malayalam}
വിശ്വസിക്കുകയും സല്‍പ്രവൃത്തികളില്‍ ഏര്‍പെടുകയും ചെയ്തവരാകട്ടെ, താഴ്ഭാഗത്ത് കൂടി അരുവികള്‍ ഒഴുകുന്ന സ്വര്‍ഗത്തോട്ടങ്ങളില്‍ നാം അവരെ പ്രവേശിപ്പിക്കുന്നതാണ്‌. അവരതില്‍ നിത്യവാസികളായിരിക്കും. അവര്‍ക്കവിടെ പരിശുദ്ധരായ ഇണകളുണ്ടായിരിക്കും. സ്ഥിരവും ഇടതൂര്‍ന്നതുമായ തണലില്‍ നാമവരെ പ്രവേശിപ്പിക്കുകയും ചെയ്യും.
\end{malayalam}}
\flushright{\begin{Arabic}
\quranayah[4][58]
\end{Arabic}}
\flushleft{\begin{malayalam}
വിശ്വസിച്ചേല്‍പിക്കപ്പെട്ട അനാമത്തുകള്‍ അവയുടെ അവകാശികള്‍ക്ക് നിങ്ങള്‍ കൊടുത്തു വീട്ടണമെന്നും, ജനങ്ങള്‍ക്കിടയില്‍ നിങ്ങള്‍ തീര്‍പ്പുകല്‍പിക്കുകയാണെങ്കില്‍ നീതിയോടെ തീര്‍പ്പുകല്‍പിക്കണമെന്നും അല്ലാഹു നിങ്ങളോട് കല്‍പിക്കുന്നു. എത്രയോ നല്ല ഉപദേശമാണ് അവന്‍ നിങ്ങള്‍ക്ക് നല്‍കുന്നത്‌. തീര്‍ച്ചയായും എല്ലാം കേള്‍ക്കുന്നവനും കാണുന്നവനുമാകുന്നു അല്ലാഹു.
\end{malayalam}}
\flushright{\begin{Arabic}
\quranayah[4][59]
\end{Arabic}}
\flushleft{\begin{malayalam}
സത്യവിശ്വാസികളേ, നിങ്ങള്‍ അല്ലാഹുവെ അനുസരിക്കുക. (അല്ലാഹുവിന്‍റെ) ദൂതനെയും നിങ്ങളില്‍ നിന്നുള്ള കൈകാര്യകര്‍ത്താക്കളെയും അനുസരിക്കുക. ഇനി വല്ല കാര്യത്തിലും നിങ്ങള്‍ക്കിടയില്‍ ഭിന്നിപ്പുണ്ടാകുകയാണെങ്കില്‍ നിങ്ങളത് അല്ലാഹുവിലേക്കും റസൂലിലേക്കും മടക്കുക. നിങ്ങള്‍ അല്ലാഹുവിലും അന്ത്യദിനത്തിലും വിശ്വസിക്കുന്നുവെങ്കില്‍ (അതാണ് വേണ്ടത്‌.) അതാണ് ഉത്തമവും കൂടുതല്‍ നല്ല പര്യവസാനമുള്ളതും.
\end{malayalam}}
\flushright{\begin{Arabic}
\quranayah[4][60]
\end{Arabic}}
\flushleft{\begin{malayalam}
നിനക്ക് അവതരിപ്പിക്കപ്പെട്ടതിലും നിനക്ക് മുമ്പ് അവതരിപ്പിക്കപ്പെട്ടതിലും തങ്ങള്‍ വിശ്വസിച്ചിരിക്കുന്നു എന്ന് ജല്‍പിക്കുന്ന ഒരു കൂട്ടരെ നീ കണ്ടില്ലേ? ദുര്‍മൂര്‍ത്തികളുടെ അടുത്തേക്ക് വിധിതേടിപ്പോകാനാണ് അവര്‍ ഉദ്ദേശിക്കുന്നത്‌. വാസ്തവത്തില്‍ ദുര്‍മൂര്‍ത്തികളെ അവിശ്വസിക്കുവാനാണ് അവര്‍ കല്‍പിക്കപ്പെട്ടിട്ടുള്ളത്‌. പിശാച് അവരെ ബഹുദൂരം വഴിതെറ്റിക്കുവാന്‍ ഉദ്ദേശിക്കുന്നു.
\end{malayalam}}
\flushright{\begin{Arabic}
\quranayah[4][61]
\end{Arabic}}
\flushleft{\begin{malayalam}
അല്ലാഹു അവതരിപ്പിച്ചതിലേക്കും (അവന്‍റെ) ദൂതനിലേക്കും നിങ്ങള്‍ വരൂ എന്ന് അവരോട് പറയപ്പെട്ടാല്‍ ആ കപടവിശ്വാസികള്‍ നിന്നെ വിട്ട് പാടെ പിന്തിരിഞ്ഞ് പോകുന്നത് നിനക്ക് കാണാം.
\end{malayalam}}
\flushright{\begin{Arabic}
\quranayah[4][62]
\end{Arabic}}
\flushleft{\begin{malayalam}
എന്നാല്‍ സ്വന്തം കൈകള്‍ ചെയ്ത് വെച്ചതിന്‍റെ ഫലമായി അവര്‍ക്ക് വല്ല ആപത്തും ബാധിക്കുകയും, അനന്തരം അവര്‍ നിന്‍റെ അടുത്ത് വന്ന് അല്ലാഹുവിന്‍റെ പേരില്‍ സത്യം ചെയ്ത് കൊണ്ട് ഞങ്ങള്‍ നന്‍മയും അനുരഞ്ജനവുമല്ലാതെ മറ്റൊന്നും ഉദ്ദേശിച്ചിരുന്നില്ല എന്ന് പറയുകയും ചെയ്യുമ്പോഴുള്ള സ്ഥിതി എങ്ങനെയായിരിക്കും?
\end{malayalam}}
\flushright{\begin{Arabic}
\quranayah[4][63]
\end{Arabic}}
\flushleft{\begin{malayalam}
അത്തരക്കാരുടെ മനസ്സുകളില്‍ എന്താണുള്ളതെന്ന് അല്ലാഹുവിന്നറിയാം. ആകയാല്‍ (നബിയേ,) അവരെ വിട്ട് തിരിഞ്ഞുകളയുക. അവര്‍ക്ക് സദുപദേശം നല്‍കുകയും, അവരുടെ മനസ്സില്‍ തട്ടുന്ന വാക്ക് അവരോട് പറയുകയും ചെയ്യുക.
\end{malayalam}}
\flushright{\begin{Arabic}
\quranayah[4][64]
\end{Arabic}}
\flushleft{\begin{malayalam}
അല്ലാഹുവിന്‍റെ ഉത്തരവ് പ്രകാരം അനുസരിക്കപ്പെടുവാന്‍ വേണ്ടിയല്ലാതെ നാം ഒരു ദൂതനെയും അയച്ചിട്ടില്ല. അവര്‍ അവരോട് തന്നെ അക്രമം പ്രവര്‍ത്തിച്ചപ്പോള്‍ നിന്‍റെ അടുക്കല്‍ അവര്‍ വരികയും, എന്നിട്ടവര്‍ അല്ലാഹുവോട് പാപമോചനം തേടുകയും, അവര്‍ക്കുവേണ്ടി റസൂലും പാപമോചനം തേടുകയും ചെയ്തിരുന്നുവെങ്കില്‍ അല്ലാഹുവെ ഏറെ പശ്ചാത്താപം സ്വീകരിക്കുന്നവനും കാരുണ്യമുള്ളവനുമായി അവര്‍ കണ്ടെത്തുമായിരുന്നു.
\end{malayalam}}
\flushright{\begin{Arabic}
\quranayah[4][65]
\end{Arabic}}
\flushleft{\begin{malayalam}
ഇല്ല, നിന്‍റെ രക്ഷിതാവിനെത്തന്നെയാണ സത്യം; അവര്‍ക്കിടയില്‍ ഭിന്നതയുണ്ടായ കാര്യത്തില്‍ അവര്‍ നിന്നെ വിധികര്‍ത്താവാക്കുകയും, നീ വിധികല്‍പിച്ചതിനെപ്പറ്റി പിന്നീടവരുടെ മനസ്സുകളില്‍ ഒരു വിഷമവും തോന്നാതിരിക്കുകയും, അത് പൂര്‍ണ്ണമായി സമ്മതിച്ച് അനുസരിക്കുകയും ചെയ്യുന്നതു വരെ അവര്‍ വിശ്വാസികളാവുകയില്ല.
\end{malayalam}}
\flushright{\begin{Arabic}
\quranayah[4][66]
\end{Arabic}}
\flushleft{\begin{malayalam}
നിങ്ങള്‍ സ്വന്തം ജീവന്‍ ബലിയര്‍പ്പിക്കണമെന്നോ, വീട് വിട്ടിറങ്ങണമെന്നോ നാം അവര്‍ക്ക് കല്‍പന നല്‍കിയിരുന്നുവെങ്കില്‍ അവരില്‍ ചുരുക്കം പേരൊഴികെ അത് ചെയ്യുമായിരുന്നില്ല. അവരോട് ഉപദേശിക്കപ്പെടും പ്രകാരം അവര്‍ പ്രവര്‍ത്തിച്ചിരുന്നുവെങ്കില്‍ അതവര്‍ക്ക് ഏറ്റവും ഉത്തമവും (സന്‍മാര്‍ഗത്തില്‍) അവരെ കൂടുതല്‍ ശക്തമായി ഉറപ്പിക്കുന്നതും ആകുമായിരുന്നു.
\end{malayalam}}
\flushright{\begin{Arabic}
\quranayah[4][67]
\end{Arabic}}
\flushleft{\begin{malayalam}
എന്നാല്‍ അവര്‍ക്ക് നമ്മുടെ പക്കല്‍ നിന്നുള്ള മഹത്തായ പ്രതിഫലം നാം നല്‍കുകയും,
\end{malayalam}}
\flushright{\begin{Arabic}
\quranayah[4][68]
\end{Arabic}}
\flushleft{\begin{malayalam}
നാമവരെ നേര്‍വഴിയില്‍ ചേര്‍ക്കുകയും ചെയ്യുമായിരുന്നു.
\end{malayalam}}
\flushright{\begin{Arabic}
\quranayah[4][69]
\end{Arabic}}
\flushleft{\begin{malayalam}
ആര്‍ അല്ലാഹുവെയും അവന്‍റെ ദൂതനെയും അനുസരിക്കുന്നുവോ അവര്‍ അല്ലാഹു അനുഗ്രഹിച്ചവരായ പ്രവാചകന്‍മാര്‍, സത്യസന്ധന്‍മാര്‍, രക്തസാക്ഷികള്‍, സച്ചരിതന്‍മാര്‍ എന്നിവരോടൊപ്പമായിരിക്കും. അവര്‍ എത്ര നല്ല കൂട്ടുകാര്‍!
\end{malayalam}}
\flushright{\begin{Arabic}
\quranayah[4][70]
\end{Arabic}}
\flushleft{\begin{malayalam}
അല്ലാഹുവിങ്കല്‍ നിന്നുള്ള അനുഗ്രഹമത്രെ അത്‌. എല്ലാം അറിയുന്നവനായി അല്ലാഹു മതി.
\end{malayalam}}
\flushright{\begin{Arabic}
\quranayah[4][71]
\end{Arabic}}
\flushleft{\begin{malayalam}
സത്യവിശ്വാസികളേ, നിങ്ങള്‍ ജാഗ്രത കൈക്കൊള്ളുവിന്‍. അങ്ങനെ ചെറുസംഘങ്ങളായോ, ഒന്നിച്ചൊറ്റകൂട്ടമായോ നിങ്ങള്‍ (യുദ്ധത്തിന്‌) പുറപ്പെട്ട് കൊള്ളുക.
\end{malayalam}}
\flushright{\begin{Arabic}
\quranayah[4][72]
\end{Arabic}}
\flushleft{\begin{malayalam}
തീര്‍ച്ചയായും നിങ്ങളുടെ കൂട്ടത്തില്‍ മടിച്ച് പിന്നോക്കം നില്‍ക്കുന്നവനുണ്ട്‌. അങ്ങനെ നിങ്ങള്‍ക്ക് വല്ല വിപത്തും ബാധിച്ചുവെങ്കില്‍, ഞാന്‍ അവരോടൊപ്പം (യുദ്ധത്തിന്‌) ഹാജരാകാതിരുന്നത് വഴി അല്ലാഹു എനിക്ക് അനുഗ്രഹം ചെയ്തിരിക്കുകയാണ് എന്നായിരിക്കും അവന്‍ പറയുക.
\end{malayalam}}
\flushright{\begin{Arabic}
\quranayah[4][73]
\end{Arabic}}
\flushleft{\begin{malayalam}
നിങ്ങള്‍ക്ക് അല്ലാഹുവിങ്കല്‍ നിന്ന് വല്ല അനുഗ്രഹവും വന്നുകിട്ടുകയാണെങ്കിലോ, നിങ്ങളും അവനും തമ്മില്‍ യാതൊരു സ്നേഹബന്ധവുമുണ്ടായിരുന്നില്ല എന്ന മട്ടില്‍ അവന്‍ പറയും; (ഹാ കഷ്ടമായി!) ഞാന്‍ അവരുടെ കൂടെയുണ്ടായിരുന്നെങ്കില്‍ എത്ര നന്നായിരുന്നേനെ. എങ്കില്‍ എനിക്കൊരു വലിയ നേട്ടം നേടിയെടുക്കാമായിരുന്നു.
\end{malayalam}}
\flushright{\begin{Arabic}
\quranayah[4][74]
\end{Arabic}}
\flushleft{\begin{malayalam}
ഇഹലോകജീവിതത്തെ പരലോകജീവിതത്തിന് പകരം വില്‍ക്കാന്‍ തയ്യാറുള്ളവര്‍ അല്ലാഹുവിന്‍റെ മാര്‍ഗത്തില്‍ യുദ്ധം ചെയ്യട്ടെ. അല്ലാഹുവിന്‍റെ മാര്‍ഗത്തില്‍ വല്ലവനും യുദ്ധം ചെയ്തിട്ട് അവന്‍ കൊല്ലപ്പെട്ടാലും വിജയം നേടിയാലും നാമവന് മഹത്തായ പ്രതിഫലം നല്‍കുന്നതാണ്‌.
\end{malayalam}}
\flushright{\begin{Arabic}
\quranayah[4][75]
\end{Arabic}}
\flushleft{\begin{malayalam}
അല്ലാഹുവിന്‍റെ മാര്‍ഗത്തില്‍ നിങ്ങള്‍ക്കെന്തുകൊണ്ട് യുദ്ധം ചെയ്തു കൂടാ? ഞങ്ങളുടെ രക്ഷിതാവേ, അക്രമികളായ ആളുകള്‍ അധിവസിക്കുന്ന ഈ നാട്ടില്‍ നിന്ന് ഞങ്ങളെ നീ മോചിപ്പിക്കുകയും, നിന്‍റെ വകയായി ഒരു രക്ഷാധികാരിയെയും നിന്‍റെ വകയായി ഒരു സഹായിയെയും ഞങ്ങള്‍ക്ക് നീ നിശ്ചയിച്ച് തരികയും ചെയ്യേണമേ. എന്ന് പ്രാര്‍ത്ഥിച്ച് കൊണ്ടിരിക്കുന്ന മര്‍ദ്ദിച്ചൊതുക്കപ്പെട്ട പുരുഷന്‍മാര്‍ക്കും സ്ത്രീകള്‍ക്കും കുട്ടികള്‍ക്കും വേണ്ടിയും (നിങ്ങള്‍ക്കെന്തുകൊണ്ട് യുദ്ധം ചെയ്തു കൂടാ?)
\end{malayalam}}
\flushright{\begin{Arabic}
\quranayah[4][76]
\end{Arabic}}
\flushleft{\begin{malayalam}
വിശ്വാസികള്‍ അല്ലാഹുവിന്‍റെ മാര്‍ഗത്തില്‍ യുദ്ധം ചെയ്യുന്നു. സത്യനിഷേധികളാകട്ടെ, ദുര്‍മൂര്‍ത്തികളുടെ മാര്‍ഗത്തില്‍ യുദ്ധം ചെയ്യുന്നു. അതിനാല്‍ പിശാചിന്‍റെ മിത്രങ്ങളുമായി നിങ്ങള്‍ യുദ്ധത്തില്‍ ഏര്‍പെടുക. തീര്‍ച്ചയായും പിശാചിന്‍റെ കുതന്ത്രം ദുര്‍ബലമാകുന്നു.
\end{malayalam}}
\flushright{\begin{Arabic}
\quranayah[4][77]
\end{Arabic}}
\flushleft{\begin{malayalam}
(യുദ്ധത്തിനുപോകാതെ) നിങ്ങള്‍ കൈകള്‍ അടക്കിവെക്കുകയും, പ്രാര്‍ത്ഥന മുറപോലെ നിര്‍വഹിക്കുകയും. സകാത്ത് നല്‍കുകയും ചെയ്യുവിന്‍ എന്ന് നിര്‍ദേശിക്കപ്പെട്ടിരുന്ന ഒരു കൂട്ടരെ നീ കണ്ടില്ലേ? പിന്നീടവര്‍ക്ക് യുദ്ധം നിര്‍ബന്ധമായി നിശ്ചയിക്കപ്പെട്ടപ്പോള്‍ അവരില്‍ ഒരു വിഭാഗമതാ അല്ലാഹുവെ ഭയപ്പെടും പോലെയോ, അതിനെക്കാള്‍ ശക്തമായ നിലയിലോ ജനങ്ങളെ ഭയപ്പെടുന്നു. ഞങ്ങളുടെ രക്ഷിതാവേ, നീയെന്തിനാണ് ഞങ്ങള്‍ക്ക് യുദ്ധം നിര്‍ബന്ധമാക്കിയത്‌? അടുത്ത ഒരു അവധിവരെയെങ്കിലും ഞങ്ങള്‍ക്ക് സമയം നീട്ടിത്തന്നുകൂടായിരുന്നോ? എന്നാണ് അവര്‍ പറഞ്ഞത്‌. പറയുക: ഇഹലോകത്തെ സുഖാനുഭവം വളരെ തുച്ഛമായതാണ്‌. പരലോകമാണ് സൂക്ഷ്മത പാലിക്കുന്നവര്‍ക്ക് കൂടുതല്‍ ഗുണകരം. നിങ്ങളോട് ഒരു തരിമ്പും അനീതി കാണിക്കപ്പെടുകയുമില്ല.
\end{malayalam}}
\flushright{\begin{Arabic}
\quranayah[4][78]
\end{Arabic}}
\flushleft{\begin{malayalam}
നിങ്ങള്‍ എവിടെയായിരുന്നാലും മരണം നിങ്ങളെ പിടികൂടുന്നതാണ്‌. നിങ്ങള്‍ ഭദ്രമായി കെട്ടി ഉയര്‍ത്തപ്പെട്ട കോട്ടകള്‍ക്കുള്ളിലായാല്‍ പോലും. (നബിയേ,) അവര്‍ക്ക് വല്ല നേട്ടവും വന്നുകിട്ടിയാല്‍ അവര്‍ പറയും; ഇത് അല്ലാഹുവിങ്കല്‍ നിന്ന് ലഭിച്ചതാണ് എന്ന്‌. അവര്‍ക്ക് വല്ല ദോഷവും ബാധിച്ചാല്‍ അവര്‍ പറയും; ഇത് നീ കാരണം ഉണ്ടായതാണ് എന്ന്‌.പറയുക: എല്ലാം അല്ലാഹുവിന്‍റെ പക്കല്‍ നിന്നുള്ളതാണ്‌. അപ്പോള്‍ ഈ ആളുകള്‍ക്ക് എന്ത് പറ്റി? അവര്‍ ഒരു വിഷയവും മനസ്സിലാക്കാന്‍ ഭാവമില്ല.
\end{malayalam}}
\flushright{\begin{Arabic}
\quranayah[4][79]
\end{Arabic}}
\flushleft{\begin{malayalam}
നന്‍മയായിട്ട് നിനക്ക് എന്തൊന്ന് വന്നുകിട്ടിയാലും അത് അല്ലാഹുവിങ്കല്‍ നിന്നുള്ളതാണ്‌. നിന്നെ ബാധിക്കുന്ന ഏതൊരു ദോഷവും നിന്‍റെ പക്കല്‍ നിന്നുതന്നെ ഉണ്ടാകുന്നതാണ്‌. (നബിയേ,) നിന്നെ നാം മനുഷ്യരിലേക്കുള്ള ദൂതനായിട്ടാണ് നിയോഗിച്ചിരിക്കുന്നത്‌.(അതിന്‌) സാക്ഷിയായി അല്ലാഹു മതി.
\end{malayalam}}
\flushright{\begin{Arabic}
\quranayah[4][80]
\end{Arabic}}
\flushleft{\begin{malayalam}
(അല്ലാഹുവിന്‍റെ) ദൂതനെ ആര്‍ അനുസരിക്കുന്നുവോ തീര്‍ച്ചയായും അവന്‍ അല്ലാഹുവെ അനുസരിച്ചു. ആര്‍ പിന്തിരിഞ്ഞുവോ അവരുടെ മേല്‍ കാവല്‍ക്കാരനായി നിന്നെ നാം നിയോഗിച്ചിട്ടില്ല.
\end{malayalam}}
\flushright{\begin{Arabic}
\quranayah[4][81]
\end{Arabic}}
\flushleft{\begin{malayalam}
അവര്‍ പറയും: ഞങ്ങളിതാ അനുസരിച്ചിരിക്കുന്നു എന്ന്‌. എന്നിട്ടവര്‍ നിന്‍റെ അടുക്കല്‍ നിന്ന് പുറത്ത് പോയാല്‍ അവരില്‍ ഒരു വിഭാഗം തങ്ങള്‍ പുറത്ത് പറയുന്നതിന് വിപരീതമായി രാത്രിയില്‍ ഗൂഢാലോചന നടത്തുന്നു. അവര്‍ രാത്രി ഗൂഢാലോചന നടത്തുന്നതെല്ലാം അല്ലാഹു രേഖപ്പെടുത്തിക്കൊണ്ടിരിക്കുന്നു. ആകയാല്‍ നീ അവരെ വിട്ട് തിരിഞ്ഞുകളയുക. എന്നിട്ട് അല്ലാഹുവെ ഭരമേല്‍പിക്കുക. ഭരമേല്‍പിക്കപ്പെടുന്നവനായി അല്ലാഹു മതി.
\end{malayalam}}
\flushright{\begin{Arabic}
\quranayah[4][82]
\end{Arabic}}
\flushleft{\begin{malayalam}
അവര്‍ ഖുര്‍ആനിനെപ്പറ്റി ചിന്തിക്കുന്നില്ലേ? അത് അല്ലാഹു അല്ലാത്തവരുടെ പക്കല്‍ നിന്നുള്ളതായിരുന്നെങ്കില്‍ അവരതില്‍ ധാരാളം വൈരുദ്ധ്യം കണ്ടെത്തുമായിരുന്നു.
\end{malayalam}}
\flushright{\begin{Arabic}
\quranayah[4][83]
\end{Arabic}}
\flushleft{\begin{malayalam}
സമാധാനവുമായോ (യുദ്ധ) ഭീതിയുമായോ ബന്ധപ്പെട്ട വല്ല വാര്‍ത്തയും അവര്‍ക്ക് വന്നുകിട്ടിയാല്‍ അവരത് പ്രചരിപ്പിക്കുകയായി. അവരത് റസൂലിന്‍റെയും അവരിലെ കാര്യവിവരമുള്ളവരുടെയും തീരുമാനത്തിന് വിട്ടിരുന്നുവെങ്കില്‍ അവരുടെ കൂട്ടത്തില്‍ നിന്ന് നിരീക്ഷിച്ച് മനസ്സിലാക്കാന്‍ കഴിവുള്ളവര്‍ അതിന്‍റെ യാഥാര്‍ത്ഥ്യം മനസ്സിലാക്കിക്കൊള്ളുമായിരുന്നു. നിങ്ങളുടെ മേല്‍ അല്ലാഹുവിന്‍റെ അനുഗ്രഹവും കാരുണ്യവും ഇല്ലായിരുന്നുവെങ്കില്‍ നിങ്ങളില്‍ അല്‍പം ചിലരൊഴികെ പിശാചിനെ പിന്‍പറ്റുമായിരുന്നു.
\end{malayalam}}
\flushright{\begin{Arabic}
\quranayah[4][84]
\end{Arabic}}
\flushleft{\begin{malayalam}
എന്നാല്‍(നബിയേ,) നീ അല്ലാഹുവിന്‍റെ മാര്‍ഗത്തില്‍ യുദ്ധം ചെയ്തു കൊള്ളുക. നിന്‍റെ സ്വന്തം കാര്യമല്ലാതെ നിന്നോട് ശാസിക്കപ്പെടുന്നതല്ല. സത്യവിശ്വാസികളില്‍ നീ പ്രേരണ ചെലുത്തുകയും ചെയ്യുക. സത്യനിഷേധികളുടെ ആക്രമണശക്തിയെ അല്ലാഹു തടുത്തുതന്നേക്കും. അല്ലാഹു ഏറ്റവും കൂടുതല്‍ ആക്രമണശക്തിയുള്ളവനും അതികഠിനമായി ശിക്ഷിക്കുന്നവനുമാകുന്നു.
\end{malayalam}}
\flushright{\begin{Arabic}
\quranayah[4][85]
\end{Arabic}}
\flushleft{\begin{malayalam}
വല്ലവനും ഒരു നല്ല ശുപാര്‍ശ ചെയ്താല്‍ ആ നന്‍മയില്‍ ഒരു പങ്ക് അവന്നുണ്ടായിരിക്കും. വല്ലവനും ഒരു ചീത്ത ശുപാര്‍ശ ചെയ്താല്‍ ആ തിന്‍മയില്‍ നിന്ന് ഒരു പങ്കും അവന്നുണ്ടായിരിക്കും. അല്ലാഹു എല്ലാകാര്യങ്ങളുടെയും മേല്‍നോട്ടം വഹിക്കുന്നവനാകുന്നു.
\end{malayalam}}
\flushright{\begin{Arabic}
\quranayah[4][86]
\end{Arabic}}
\flushleft{\begin{malayalam}
നിങ്ങള്‍ക്ക് അഭിവാദ്യം അര്‍പ്പിക്കപ്പെട്ടാല്‍ അതിനെക്കാള്‍ മെച്ചമായി (അങ്ങോട്ട്‌) അഭിവാദ്യം അര്‍പ്പിക്കുക. അല്ലെങ്കില്‍ അതുതന്നെ തിരിച്ചുനല്‍കുക. തീര്‍ച്ചയായും അല്ലാഹു ഏതൊരു കാര്യത്തിന്‍റെയും കണക്ക് നോക്കുന്നവനാകുന്നു.
\end{malayalam}}
\flushright{\begin{Arabic}
\quranayah[4][87]
\end{Arabic}}
\flushleft{\begin{malayalam}
അല്ലാഹു- അവനല്ലാതെ യാതൊരു ദൈവവുമില്ല. ഉയിര്‍ത്തെഴുന്നേല്‍പിന്‍റെ ദിവസത്തേക്ക് അവന്‍ നിങ്ങളെയെല്ലാം ഒരുമിച്ചുകൂട്ടുക തന്നെ ചെയ്യും. അതില്‍ സംശയമേ ഇല്ല. അല്ലാഹുവെക്കാള്‍ സത്യസന്ധമായി വിവരം നല്‍കുന്നവന്‍ ആരുണ്ട്‌?
\end{malayalam}}
\flushright{\begin{Arabic}
\quranayah[4][88]
\end{Arabic}}
\flushleft{\begin{malayalam}
എന്നാല്‍ കപടവിശ്വാസികളുടെ കാര്യത്തില്‍ നിങ്ങളെന്താണ് രണ്ട് കക്ഷികളാകുന്നത്‌? അവര്‍ സമ്പാദിച്ചുണ്ടാക്കിയത് (തിന്‍മ) കാരണം അല്ലാഹു അവരെ തലതിരിച്ചു വിട്ടിരിക്കുകയാണ്‌. അല്ലാഹു പിഴപ്പിച്ചവരെ നിങ്ങള്‍ നേര്‍വഴിയിലാക്കാന്‍ ഉദ്ദേശിച്ചിരിക്കുകയാണോ? അല്ലാഹു ഒരുവനെ പിഴപ്പിച്ചാല്‍ പിന്നെ അവന്ന് ഒരു വഴിയും നീ കണ്ടെത്തുന്നതല്ല.
\end{malayalam}}
\flushright{\begin{Arabic}
\quranayah[4][89]
\end{Arabic}}
\flushleft{\begin{malayalam}
അവര്‍ അവിശ്വസിച്ചത് പോലെ നിങ്ങളും അവിശ്വസിക്കുകയും, അങ്ങനെ നിങ്ങളെല്ലാം ഒരുപോലെയായിത്തീരുകയും ചെയ്യാനാണ് അവര്‍ കൊതിക്കുന്നത്‌. അതിനാല്‍ അവര്‍ അല്ലാഹുവിന്‍റെ മാര്‍ഗത്തില്‍ സ്വന്തം നാട് വിട്ടുവരുന്നതു വരെ അവരില്‍ നിന്ന് നിങ്ങള്‍ മിത്രങ്ങളെ സ്വീകരിച്ച് പോകരുത്‌. എന്നാല്‍ അവര്‍ പിന്തിരിഞ്ഞ് കളയുകയാണെങ്കില്‍ നിങ്ങളവരെ പിടികൂടുകയും, അവരെ കണ്ടുമുട്ടിയേടത്തുവെച്ച് നിങ്ങളവരെ കൊലപ്പെടുത്തുകയും ചെയ്യുക. അവരില്‍ നിന്ന് യാതൊരു മിത്രത്തെയും സഹായിയെയും നിങ്ങള്‍ സ്വീകരിച്ചു പോകരുത്‌.
\end{malayalam}}
\flushright{\begin{Arabic}
\quranayah[4][90]
\end{Arabic}}
\flushleft{\begin{malayalam}
നിങ്ങളുമായി സഖ്യത്തില്‍ കഴിയുന്ന ഒരു ജനവിഭാഗത്തോട് ചേര്‍ന്ന് നില്‍ക്കുന്നവരൊഴികെ. നിങ്ങളോട് യുദ്ധം ചെയ്യാനോ, സ്വന്തം ആള്‍ക്കാരോട് യുദ്ധം ചെയ്യാനോ മനഃപ്രയാസമുള്ളവരായി നിങ്ങളുടെ അടുത്ത് വരുന്നവരും ഒഴികെ. അല്ലാഹു ഉദ്ദേശിച്ചിരുന്നെങ്കില്‍ നിങ്ങളുടെ മേല്‍ അവര്‍ക്കവന്‍ ശക്തി നല്‍കുകയും, നിങ്ങളോടവര്‍ യുദ്ധത്തില്‍ ഏര്‍പെടുകയും ചെയ്യുമായിരുന്നു. എന്നാല്‍ നിങ്ങളോട് യുദ്ധം ചെയ്യാതെ അവര്‍ വിട്ടൊഴിഞ്ഞ് നില്‍ക്കുകയും, നിങ്ങളുടെ മുമ്പാകെ സമാധാനനിര്‍ദേശം വെക്കുകയും ചെയ്തിട്ടുണ്ടെങ്കില്‍ അവര്‍ക്കെതിരായി യാതൊരു മാര്‍ഗവും അല്ലാഹു നിങ്ങള്‍ക്ക് അനുവദിച്ചിട്ടില്ല.
\end{malayalam}}
\flushright{\begin{Arabic}
\quranayah[4][91]
\end{Arabic}}
\flushleft{\begin{malayalam}
വേറെ ഒരു വിഭാഗത്തെയും നിങ്ങള്‍ കണ്ടെത്തിയേക്കും. നിങ്ങളില്‍ നിന്നും സ്വന്തം ജനതയില്‍ നിന്നും ഒരുപോലെ സുരക്ഷിതരായിക്കഴിയാന്‍ അവര്‍ ആഗ്രഹിക്കുന്നു. കുഴപ്പത്തിലേക്ക് അവര്‍ തിരിച്ചുവിളിക്കപ്പെടുമ്പോഴെല്ലാം അതിലവര്‍ തലകുത്തി വീഴുന്നു. എന്നാല്‍ അവര്‍ നിങ്ങളെ വിട്ട് ഒഴിഞ്ഞ് നില്‍ക്കുകയും, നിങ്ങളുടെ മുമ്പാകെ സമാധാന നിര്‍ദേശം വെക്കുകയും, സ്വന്തം കൈകള്‍ അടക്കിവെക്കുകയും ചെയ്യാത്ത പക്ഷം അവരെ നിങ്ങള്‍ പിടികൂടുകയും, അവരെ കണ്ടുമുട്ടുന്നേടത്ത് വെച്ച് നിങ്ങള്‍ കൊലപ്പെടുത്തുകയും ചെയ്യുക. അത്തരക്കാര്‍ക്കെതിരില്‍ നാം നിങ്ങള്‍ക്ക് വ്യക്തമായ ന്യായം നല്‍കിയിരിക്കുന്നു.
\end{malayalam}}
\flushright{\begin{Arabic}
\quranayah[4][92]
\end{Arabic}}
\flushleft{\begin{malayalam}
യാതൊരു വിശ്വാസിക്കും മറ്റൊരു വിശ്വാസിയെ കൊല്ലാന്‍ പാടുള്ളതല്ല; അബദ്ധത്തില്‍ വന്നുപോകുന്നതല്ലാതെ. എന്നാല്‍ വല്ലവനും ഒരു വിശ്വാസിയെ അബദ്ധത്തില്‍ കൊന്നുപോയാല്‍ (പ്രായശ്ചിത്തമായി) ഒരു വിശ്വാസിയായ അടിമയെ മോചിപ്പിക്കുകയും, അവന്‍റെ (കൊല്ലപ്പെട്ടവന്‍റെ) അവകാശികള്‍ക്ക് നഷ്ടപരിഹാരം നല്‍കുകയുമാണ് വേണ്ടത്‌. അവര്‍ (ആ അവകാശികള്‍) അത് ഉദാരമായി വിട്ടുതന്നെങ്കിലൊഴികെ. ഇനി അവന്‍ (കൊല്ലപ്പെട്ടവന്‍) നിങ്ങളോട് ശത്രുതയുള്ള ജനവിഭാഗത്തില്‍ പെട്ടവനാണ്‌; അവനാണെങ്കില്‍ സത്യവിശ്വാസിയുമാണ് എങ്കില്‍ സത്യവിശ്വാസിയായ ഒരു അടിമയെ മോചിപ്പിക്കുക മാത്രമാണ് വേണ്ടത്‌. ഇനി അവന്‍ (കൊല്ലപ്പെട്ടവന്‍) നിങ്ങളുമായി സഖ്യത്തിലിരിക്കുന്ന ഒരു ജനവിഭാഗത്തില്‍ പെട്ടവനാണെങ്കില്‍ അവന്‍റെ അവകാശികള്‍ക്ക് നഷ്ടപരിഹാരം നല്‍കുകയും വിശ്വാസിയായ ഒരു അടിമയെ മോചിപ്പിക്കുകയും ചെയ്യേണ്ടതാണ്‌. വല്ലവന്നും അത് സാധിച്ച് കിട്ടിയില്ലെങ്കില്‍ തുടര്‍ച്ചയായി രണ്ടുമാസം നോമ്പനുഷ്ഠിക്കേണ്ടതാണ്‌. അല്ലാഹു നിശ്ചയിച്ച പശ്ചാത്താപ (മാര്‍ഗ) മാണത്‌. അല്ലാഹു എല്ലാം അറിയുന്നവനും യുക്തിമാനുമാകുന്നു.
\end{malayalam}}
\flushright{\begin{Arabic}
\quranayah[4][93]
\end{Arabic}}
\flushleft{\begin{malayalam}
ആരെങ്കിലും ഒരു സത്യവിശ്വാസിയെ മനഃപൂര്‍വ്വം കൊലപ്പെടുത്തുന്ന പക്ഷം അവന്നുള്ള പ്രതിഫലം നരകമാകുന്നു. അവനതില്‍ നിത്യവാസിയായിരിക്കും. അവന്‍റെ നേരെ അല്ലാഹു കോപിക്കുകയും, അവനെ ശപിക്കുകയും ചെയ്തിരിക്കുന്നു. കനത്ത ശിക്ഷയാണ് അവന്നുവേണ്ടി അല്ലാഹു ഒരുക്കിവെച്ചുട്ടുള്ളത്‌.
\end{malayalam}}
\flushright{\begin{Arabic}
\quranayah[4][94]
\end{Arabic}}
\flushleft{\begin{malayalam}
സത്യവിശ്വാസികളേ, നിങ്ങള്‍ അല്ലാഹുവിന്‍റെ മാര്‍ഗത്തില്‍ യുദ്ധത്തിനുപോയാല്‍ (ശത്രു ആരെന്നും മിത്രം ആരെന്നും) നിങ്ങള്‍ വ്യക്തമായി മനസ്സിലാക്കണം. നിങ്ങള്‍ക്ക് സലാം അര്‍പ്പിച്ചവനോട് നീ വിശ്വാസിയല്ല എന്ന് നിങ്ങള്‍ പറയരുത്‌. ഇഹലോകജീവിതത്തിലെ നേട്ടം കൊതിച്ചുകൊണ്ടാണ് (നിങ്ങളങ്ങനെ പറയുന്നത്‌.) എന്നാല്‍ നേടിയെടുക്കാവുന്ന ധാരാളം സ്വത്തുകള്‍ അല്ലാഹുവിന്‍റെ അടുക്കലുണ്ട്‌. മുമ്പ് നിങ്ങളും അത് പോലെ (അവിശ്വാസത്തില്‍) ആയിരുന്നല്ലോ. അനന്തരം അല്ലാഹു നിങ്ങള്‍ക്ക് അനുഗ്രഹം ചെയ്തു. അതിനാല്‍ നിങ്ങള്‍ (കാര്യങ്ങള്‍) വ്യക്തമായി (അന്വേഷിച്ച്‌) മനസ്സിലാക്കുക. തീര്‍ച്ചയായും അല്ലാഹു നിങ്ങള്‍ പ്രവര്‍ത്തിക്കുന്നതിനെപ്പറ്റിയെല്ലാം സൂക്ഷ്മമായി അറിയുന്നവനാകുന്നു.
\end{malayalam}}
\flushright{\begin{Arabic}
\quranayah[4][95]
\end{Arabic}}
\flushleft{\begin{malayalam}
ന്യായമായ വിഷമമില്ലാതെ (യുദ്ധത്തിന് പോകാതെ) ഒഴിഞ്ഞിരിക്കുന്ന വിശ്വാസികളും, തങ്ങളുടെ ധനം കൊണ്ടും ദേഹം കൊണ്ടും അല്ലാഹുവിന്‍റെ മാര്‍ഗത്തില്‍ സമരം ചെയ്യുന്നവരും തുല്യരാകുകയില്ല. തങ്ങളുടെ ധനം കൊണ്ടും ദേഹംകൊണ്ടും സമരം ചെയ്യുന്നവരെ ഒഴിഞ്ഞിരിക്കുന്നവരേക്കാള്‍ അല്ലാഹു പദവിയില്‍ ഉയര്‍ത്തിയിരിക്കുന്നു. എല്ലാവര്‍ക്കും അല്ലാഹു നല്ല പ്രതിഫലം വാഗ്ദാനം ചെയ്തിരിക്കുന്നു. എന്നാല്‍ സമരത്തില്‍ ഏര്‍പെടുന്നവര്‍ക്ക് ഒഴിഞ്ഞിരിക്കുന്നവരേക്കാളും കൂടുതലായി അല്ലാഹു മഹത്തായ പ്രതിഫലം നല്‍കുന്നതാണ്‌.
\end{malayalam}}
\flushright{\begin{Arabic}
\quranayah[4][96]
\end{Arabic}}
\flushleft{\begin{malayalam}
അവന്‍റെ പക്കല്‍ നിന്നുള്ള പല പദവികളും പാപമോചനവും കാരുണ്യവുമത്രെ (അവര്‍ക്കുള്ളത്‌.) അല്ലാഹു ഏറെ പൊറുക്കുന്നവനും കരുണാനിധിയുമാകുന്നു.
\end{malayalam}}
\flushright{\begin{Arabic}
\quranayah[4][97]
\end{Arabic}}
\flushleft{\begin{malayalam}
അവിശ്വാസികളുടെ ഇടയില്‍ തന്നെ ജീവിച്ചുകൊണ്ട്‌) സ്വന്തത്തോട് അന്യായം ചെയ്തവരെ മരിപ്പിക്കുമ്പോള്‍ മലക്കുകള്‍ അവരോട് ചോദിക്കും: നിങ്ങളെന്തൊരു നിലപാടിലായിരുന്നു? അവര്‍ പറയും: ഞങ്ങള്‍ നാട്ടില്‍ അടിച്ചൊതുക്കപ്പെട്ടവരായിരുന്നു. അവര്‍ (മലക്കുകള്‍) ചോദിക്കും: അല്ലാഹുവിന്‍റെ ഭൂമി വിശാലമായിരുന്നില്ലേ? നിങ്ങള്‍ക്ക് സ്വദേശം വിട്ട് അതില്‍ എവിടെയെങ്കിലും പോകാമായിരുന്നല്ലോ. എന്നാല്‍ അത്തരക്കാരുടെ വാസസ്ഥലം നരകമത്രെ. അതെത്ര ചീത്ത സങ്കേതം!
\end{malayalam}}
\flushright{\begin{Arabic}
\quranayah[4][98]
\end{Arabic}}
\flushleft{\begin{malayalam}
എന്നാല്‍ യാതൊരു ഉപായവും സ്വീകരിക്കാന്‍ കഴിവില്ലാതെ, ഒരു രക്ഷാമാര്‍ഗവും കണ്ടെത്താനാകാതെ അടിച്ചൊതുക്കപ്പെട്ടവരായിക്കഴിയുന്ന പുരുഷന്‍മാരും സ്ത്രീകളും കുട്ടികളും ഇതില്‍ നിന്നൊഴിവാകുന്നു.
\end{malayalam}}
\flushright{\begin{Arabic}
\quranayah[4][99]
\end{Arabic}}
\flushleft{\begin{malayalam}
അത്തരക്കാര്‍ക്ക് അല്ലാഹു മാപ്പുനല്‍കിയേക്കാം. അല്ലാഹു അത്യധികം മാപ്പ് നല്‍കുന്നവനും ഏറെ പൊറുക്കുന്നവനുമാകുന്നു.
\end{malayalam}}
\flushright{\begin{Arabic}
\quranayah[4][100]
\end{Arabic}}
\flushleft{\begin{malayalam}
അല്ലാഹുവിന്‍റെ മാര്‍ഗത്തില്‍ വല്ലവനും സ്വദേശം വെടിഞ്ഞ് പോകുന്ന പക്ഷം ഭൂമിയില്‍ ധാരാളം അഭയസ്ഥാനങ്ങളും ജീവിതവിശാലതയും അവന്‍ കണ്ടെത്തുന്നതാണ്‌. വല്ലവനും തന്‍റെ വീട്ടില്‍ നിന്ന് - സ്വദേശം വെടിഞ്ഞ് കൊണ്ട് - അല്ലാഹുവിലേക്കും അവന്‍റെ ദൂതനിലേക്കും ഇറങ്ങി പുറപ്പെടുകയും, അനന്തരം (വഴി മദ്ധ്യേ) മരണമവനെ പിടികൂടുകയും ചെയ്യുന്ന പക്ഷം അവന്നുള്ള പ്രതിഫലം അല്ലാഹുവിങ്കല്‍ സ്ഥിരപ്പെട്ടു കഴിഞ്ഞു. അല്ലാഹു ഏറെ പൊറുക്കുന്നവനും കരുണാനിധിയുമാകുന്നു.
\end{malayalam}}
\flushright{\begin{Arabic}
\quranayah[4][101]
\end{Arabic}}
\flushleft{\begin{malayalam}
നിങ്ങള്‍ ഭൂമിയില്‍ യാത്രചെയ്യുകയാണെങ്കില്‍ സത്യനിഷേധികള്‍ നിങ്ങള്‍ക്ക് നാശം വരുത്തുമെന്ന് നിങ്ങള്‍ ഭയപ്പെടുന്ന പക്ഷം നമസ്കാരം ചുരുക്കി നിര്‍വഹിക്കുന്നതില്‍ നിങ്ങള്‍ക്ക് കുറ്റമില്ല. തീര്‍ച്ചയായും സത്യനിഷേധികള്‍ നിങ്ങളുടെ പ്രത്യക്ഷ ശത്രുക്കളാകുന്നു.
\end{malayalam}}
\flushright{\begin{Arabic}
\quranayah[4][102]
\end{Arabic}}
\flushleft{\begin{malayalam}
(നബിയേ,) നീ അവരുടെ കൂട്ടത്തിലുണ്ടായിരിക്കുകയും, അവര്‍ക്ക് നേതൃത്വം നല്‍കിക്കൊണ്ട് നമസ്കാരം നിര്‍വഹിക്കുകയുമാണെങ്കില്‍ അവരില്‍ ഒരു വിഭാഗം നിന്‍റെ കൂടെ നില്‍ക്കട്ടെ. അവര്‍ അവരുടെ ആയുധങ്ങള്‍ എടുക്കുകയും ചെയ്യട്ടെ. അങ്ങനെ അവര്‍ സുജൂദ് ചെയ്ത് കഴിഞ്ഞാല്‍ അവര്‍ നിങ്ങളുടെ പിന്നിലേക്ക് മാറിനില്‍ക്കുകയും, നമസ്കരിച്ചിട്ടില്ലാത്ത മറ്റെ വിഭാഗം വന്ന് നിന്‍റെ കൂടെ നമസ്കരിക്കുകയും ചെയ്യട്ടെ. അവര്‍ ജാഗ്രത കൈക്കൊള്ളുകയും, തങ്ങളുടെ ആയുധങ്ങള്‍ എടുക്കുകയും ചെയ്യേണ്ടതാണ്‌. നിങ്ങളുടെ ആയുധങ്ങളെപ്പറ്റിയും, നിങ്ങളുടെ സാധനങ്ങളെപ്പറ്റിയും നിങ്ങള്‍ അശ്രദ്ധരായെങ്കില്‍, നിങ്ങളുടെ നേരെ തിരിഞ്ഞ് ഒരൊറ്റ ആഞ്ഞടി നടത്താമായിരുന്നുവെന്ന് സത്യനിഷേധികള്‍ മോഹിക്കുകയാണ്‌. എന്നാല്‍ മഴ കാരണം നിങ്ങള്‍ക്ക് ശല്യമുണ്ടാകുകയോ, നിങ്ങള്‍ രോഗബാധിതരാകുകയോ ചെയ്താല്‍ നിങ്ങളുടെ ആയുധങ്ങള്‍ താഴെ വെക്കുന്നതിന് കുറ്റമില്ല. എന്നാല്‍ നിങ്ങള്‍ ജാഗ്രത പുലര്‍ത്തുക തന്നെ വേണം. തീര്‍ച്ചയായും അല്ലാഹു സത്യനിഷേധികള്‍ക്ക് അപമാനകരമായ ശിക്ഷ ഒരുക്കിവെച്ചിട്ടുണ്ട്‌.
\end{malayalam}}
\flushright{\begin{Arabic}
\quranayah[4][103]
\end{Arabic}}
\flushleft{\begin{malayalam}
അങ്ങനെ നമസ്കാരം നിര്‍വഹിച്ചു കഴിഞ്ഞാല്‍ നിങ്ങള്‍ നിന്നു കൊണ്ടും ഇരുന്ന് കൊണ്ടും കിടന്ന് കൊണ്ടും അല്ലാഹുവെ ഓര്‍മിക്കുക. സമാധാനാവസ്ഥയിലായാല്‍ നിങ്ങള്‍ നമസ്കാരം മുറപ്രകാരം തന്നെ നിര്‍വഹിക്കുക. തീര്‍ച്ചയായും നമസ്കാരം സത്യവിശ്വാസികള്‍ക്ക് സമയം നിര്‍ണയിക്കപ്പെട്ട ഒരു നിര്‍ബന്ധബാധ്യതയാകുന്നു.
\end{malayalam}}
\flushright{\begin{Arabic}
\quranayah[4][104]
\end{Arabic}}
\flushleft{\begin{malayalam}
ശത്രുജനതയെ തേടിപ്പിടിക്കുന്ന കാര്യത്തില്‍ നിങ്ങള്‍ ദൌര്‍ബല്യം കാണിക്കരുത്‌. നിങ്ങള്‍ വേദന അനുഭവിക്കുന്നുണ്ടെങ്കില്‍, നിങ്ങള്‍ വേദന അനുഭവിക്കുന്നത് പോലെത്തന്നെ അവരും വേദന അനുഭവിക്കുന്നുണ്ട്‌. നിങ്ങളാകട്ടെ അവര്‍ക്ക് പ്രതീക്ഷിക്കാനില്ലാത്തത് (അനുഗ്രഹം) അല്ലാഹുവിങ്കല്‍ നിന്ന് പ്രതീക്ഷിക്കുന്നുമുണ്ട്‌. അല്ലാഹു അറിവുള്ളവനും യുക്തിയുള്ളവനുമാകുന്നു.
\end{malayalam}}
\flushright{\begin{Arabic}
\quranayah[4][105]
\end{Arabic}}
\flushleft{\begin{malayalam}
നിനക്ക് അല്ലാഹു കാണിച്ചുതന്നതനുസരിച്ച് ജനങ്ങള്‍ക്കിടയില്‍ നീ വിധികല്‍പിക്കുവാന്‍ വേണ്ടിയാണ് സത്യപ്രകാരം നാം നിനക്ക് വേദഗ്രന്ഥം അവതരിപ്പിച്ചു തന്നിട്ടുള്ളത്‌. നീ വഞ്ചകന്‍മാര്‍ക്ക് വേണ്ടി വാദിക്കുന്നവനാകരുത്‌.
\end{malayalam}}
\flushright{\begin{Arabic}
\quranayah[4][106]
\end{Arabic}}
\flushleft{\begin{malayalam}
അല്ലാഹുവോട് പാപമോചനം തേടുക. തീര്‍ച്ചയായും അല്ലാഹു ഏറെ പൊറുക്കുന്നവനും കരുണാനിധിയുമാകുന്നു.
\end{malayalam}}
\flushright{\begin{Arabic}
\quranayah[4][107]
\end{Arabic}}
\flushleft{\begin{malayalam}
ആത്മവഞ്ചന നടത്തിക്കൊണ്ടിരിക്കുന്ന ആളുകള്‍ക്ക് വേണ്ടി നീ തര്‍ക്കിക്കരുത്‌. മഹാവഞ്ചകനും അധര്‍മ്മകാരിയുമായ ഒരാളെയും അല്ലാഹു ഇഷ്ടപ്പെടുകയേ ഇല്ല.
\end{malayalam}}
\flushright{\begin{Arabic}
\quranayah[4][108]
\end{Arabic}}
\flushleft{\begin{malayalam}
അവര്‍ ജനങ്ങളില്‍ നിന്ന് (കാര്യങ്ങള്‍) ഒളിച്ചു വെക്കുന്നു. എന്നാല്‍ അല്ലാഹുവില്‍ നിന്ന് (ഒന്നും) ഒളിച്ചുവെക്കാന്‍ അവര്‍ക്ക് കഴിയില്ല. അല്ലാഹു ഇഷ്ടപ്പെടാത്ത വാക്കുകളിലൂടെ അവര്‍ രാത്രിയില്‍ ഗൂഢാലോചന നടത്തിക്കൊണ്ടിരിക്കുമ്പോള്‍ അവര്‍ അവരുടെ കൂടെത്തന്നെയുണ്ട്‌. അവര്‍ പ്രവര്‍ത്തിക്കുന്നതെല്ലാം സമ്പൂര്‍ണ്ണമായി അറിയുന്നവനാകുന്നു അല്ലാഹു.
\end{malayalam}}
\flushright{\begin{Arabic}
\quranayah[4][109]
\end{Arabic}}
\flushleft{\begin{malayalam}
ഹേ! കൂട്ടരേ, ഐഹികജീവിതത്തില്‍ നിങ്ങളവര്‍ക്ക് വേണ്ടി തര്‍ക്കിച്ചു. എന്നാല്‍ ഉയിര്‍ത്തെഴുന്നേല്‍പിന്‍റെ നാളില്‍ അവര്‍ക്ക് വേണ്ടി അല്ലാഹുവോട് തര്‍ക്കിക്കാന്‍ ആരാണുള്ളത്‌? അല്ലെങ്കില്‍ അവരുടെ കാര്യം ഏറ്റെടുക്കാന്‍ ആരാണുണ്ടായിരിക്കുക?
\end{malayalam}}
\flushright{\begin{Arabic}
\quranayah[4][110]
\end{Arabic}}
\flushleft{\begin{malayalam}
ആരെങ്കിലും വല്ല തിന്‍മയും ചെയ്യുകയോ, സ്വന്തത്തോട് തന്നെ അക്രമം പ്രവര്‍ത്തിക്കുകയോ ചെയ്തിട്ട് അല്ലാഹുവോട് പാപമോചനം തേടുന്ന പക്ഷം ഏറെ പൊറുക്കുന്നവനും കരുണാനിധിയുമായി അല്ലാഹുവെ അവന്‍ കണ്ടെത്തുന്നതാണ്‌.
\end{malayalam}}
\flushright{\begin{Arabic}
\quranayah[4][111]
\end{Arabic}}
\flushleft{\begin{malayalam}
വല്ലവനും പാപം സമ്പാദിച്ച് വെക്കുന്ന പക്ഷം അവന്‍റെ തന്നെ ദോഷത്തിനായിട്ടാണ് അവനത് സമ്പാദിച്ച് വെക്കുന്നത്‌. അല്ലാഹു സര്‍വ്വജ്ഞനും യുക്തിമാനുമാകുന്നു.
\end{malayalam}}
\flushright{\begin{Arabic}
\quranayah[4][112]
\end{Arabic}}
\flushleft{\begin{malayalam}
ആരെങ്കിലും വല്ല തെറ്റോ കുറ്റമോ പ്രവര്‍ത്തിക്കുകയും, എന്നിട്ട് അത് ഒരു നിരപരാധിയുടെ പേരില്‍ ആരോപിക്കുകയും ചെയ്യുന്ന പക്ഷം തീര്‍ച്ചയായും അവന്‍ ഒരു കള്ളആരോപണവും പ്രത്യക്ഷമായ ഒരു പാപവും പേറുകയാണ് ചെയ്തിരിക്കുന്നത്‌.
\end{malayalam}}
\flushright{\begin{Arabic}
\quranayah[4][113]
\end{Arabic}}
\flushleft{\begin{malayalam}
നിന്‍റെ മേല്‍ അല്ലാഹുവിന്‍റെ അനുഗ്രഹവും കാരുണ്യവുമില്ലായിരുന്നുവെങ്കില്‍ അവരില്‍ ഒരു വിഭാഗം നിന്നെ പിഴപ്പിച്ച് കളയുവാന്‍ തുനിഞ്ഞിരിക്കുകയായിരുന്നു. (വാസ്തവത്തില്‍) അവര്‍ അവരെ തന്നെയാണ് പിഴപ്പിക്കുന്നത്‌. നിനക്ക് അവര്‍ ഒരു ഉപദ്രവവും വരുത്തുന്നതല്ല. അല്ലാഹു നിനക്ക് വേദവും ജ്ഞാനവും അവതരിപ്പിച്ച് തരികയും, നിനക്ക് അറിവില്ലാതിരുന്നത് പഠിപ്പിക്കുകയും ചെയ്തിരിക്കുന്നു. നിന്‍റെ മേലുള്ള അല്ലാഹുവിന്‍റെ അനുഗ്രഹം മഹത്തായതാകുന്നു.
\end{malayalam}}
\flushright{\begin{Arabic}
\quranayah[4][114]
\end{Arabic}}
\flushleft{\begin{malayalam}
അവരുടെ രഹസ്യാലോചനകളില്‍ മിക്കതിലും യാതൊരു നന്‍മയുമില്ല. വല്ല ദാനധര്‍മ്മവും ചെയ്യാനോ , സദാചാരം കൈക്കൊള്ളാനോ, ജനങ്ങള്‍ക്കിടയില്‍ രഞ്ജിപ്പുണ്ടാക്കാനോ കല്‍പിക്കുന്ന ആളുകളുടെ വാക്കുകളിലൊഴികെ. വല്ലവനും അല്ലാഹുവിന്‍റെ പൊരുത്തം തേടിക്കൊണ്ട് അപ്രകാരം ചെയ്യുന്ന പക്ഷം അവന് നാം മഹത്തായ പ്രതിഫലം നല്‍കുന്നതാണ്‌.
\end{malayalam}}
\flushright{\begin{Arabic}
\quranayah[4][115]
\end{Arabic}}
\flushleft{\begin{malayalam}
തനിക്ക് സന്‍മാര്‍ഗം വ്യക്തമായിക്കഴിഞ്ഞ ശേഷവും ആരെങ്കിലും ദൈവദൂതനുമായി എതിര്‍ത്ത് നില്‍ക്കുകയും, സത്യവിശ്വാസികളുടെതല്ലാത്ത മാര്‍ഗം പിന്തുടരുകയും ചെയ്യുന്ന പക്ഷം അവന്‍ തിരിഞ്ഞ വഴിക്ക് തന്നെ നാം അവനെ തിരിച്ചുവിടുന്നതും, നരകത്തിലിട്ട് നാമവനെ കരിക്കുന്നതുമാണ്‌. അതെത്ര മോശമായ പര്യവസാനം!
\end{malayalam}}
\flushright{\begin{Arabic}
\quranayah[4][116]
\end{Arabic}}
\flushleft{\begin{malayalam}
തന്നോട് പങ്കുചേര്‍ക്കപ്പെടുക എന്നത് അല്ലാഹു പൊറുക്കുകയില്ല; തീര്‍ച്ച. അതൊഴിച്ചുള്ളത് അവന്‍ ഉദ്ദേശിക്കുന്നവര്‍ക്ക് അവന്‍ പൊറുത്തുകൊടുക്കുന്നതാണ്‌. ആര്‍ അല്ലാഹുവോട് പങ്കുചേര്‍ക്കുന്നുവോ അവന്‍ ബഹുദൂരം പിഴച്ചു പോയിരിക്കുന്നു.
\end{malayalam}}
\flushright{\begin{Arabic}
\quranayah[4][117]
\end{Arabic}}
\flushleft{\begin{malayalam}
അല്ലാഹുവിന് പുറമെ അവര്‍ വിളിച്ച് പ്രാര്‍ത്ഥിക്കുന്നത് ചില പെണ്‍ദൈവങ്ങളെ മാത്രമാകുന്നു. (വാസ്തവത്തില്‍) ധിക്കാരിയായ പിശാചിനെ മാത്രമാണ് അവര്‍ വിളിച്ച് പ്രാര്‍ത്ഥിക്കുന്നത്‌.
\end{malayalam}}
\flushright{\begin{Arabic}
\quranayah[4][118]
\end{Arabic}}
\flushleft{\begin{malayalam}
അവനെ (പിശാചിനെ) അല്ലാഹു ശപിച്ചിരിക്കുന്നു. അവന്‍ (അല്ലാഹുവോട്‌) പറയുകയുണ്ടായി: നിന്‍റെ ദാസന്‍മാരില്‍ നിന്ന് ഒരു നിശ്ചിത വിഹിതം (എന്‍റെതായി) ഞാന്‍ ഉണ്ടാക്കിത്തീര്‍ക്കുന്നതാണ്‌.
\end{malayalam}}
\flushright{\begin{Arabic}
\quranayah[4][119]
\end{Arabic}}
\flushleft{\begin{malayalam}
അവരെ ഞാന്‍ വഴിപിഴപ്പിക്കുകയും വ്യാമോഹിപ്പിക്കുകയും ചെയ്യും. ഞാനവരോട് കല്‍പിക്കുമ്പോള്‍ അവര്‍ കാലികളുടെ കാതുകള്‍ കീറിമുറിക്കും. ഞാനവരോട് കല്‍പിക്കുമ്പോള്‍ അവര്‍ അല്ലാഹുവിന്‍റെ സൃഷ്ടി (പ്രകൃതി) അലങ്കോലപ്പെടുത്തും. വല്ലവനും അല്ലാഹുവിന് പുറമെ പിശാചിനെ രക്ഷാധികാരിയായി സ്വീകരിക്കുന്ന പക്ഷം തീര്‍ച്ചയായും അവന്‍ പ്രത്യക്ഷമായ നഷ്ടം പറ്റിയവനാകുന്നു.
\end{malayalam}}
\flushright{\begin{Arabic}
\quranayah[4][120]
\end{Arabic}}
\flushleft{\begin{malayalam}
അവന്‍ (പിശാച്‌) അവര്‍ക്ക് വാഗ്ദാനങ്ങള്‍ നല്‍കുകയും, അവരെ വ്യാമോഹിപ്പിക്കുകയും ചെയ്യുന്നു. പിശാച് അവര്‍ക്ക് നല്‍കുന്ന വാഗ്ദാനം വഞ്ചനയല്ലാതെ മറ്റൊന്നുമല്ല.
\end{malayalam}}
\flushright{\begin{Arabic}
\quranayah[4][121]
\end{Arabic}}
\flushleft{\begin{malayalam}
അത്തരക്കാര്‍ക്കുള്ള സങ്കേതം നരകമാകുന്നു. അതില്‍ നിന്ന് ഓടിരക്ഷപ്പെടുവാന്‍ ഒരിടവും അവര്‍ കണ്ടെത്തുകയില്ല.
\end{malayalam}}
\flushright{\begin{Arabic}
\quranayah[4][122]
\end{Arabic}}
\flushleft{\begin{malayalam}
എന്നാല്‍ വിശ്വസിക്കുകയും, സല്‍കര്‍മ്മങ്ങള്‍ പ്രവര്‍ത്തിക്കുകയും ചെയ്തവരാരോ അവരെ നാം കീഴ്ഭാഗത്ത് കൂടി അരുവികള്‍ ഒഴുകുന്ന സ്വര്‍ഗത്തോപ്പുകളില്‍ പ്രവേശിപ്പിക്കുന്നതാണ്‌. അവരതില്‍ നിത്യവാസികളായിരിക്കും. അല്ലാഹുവിന്‍റെ സത്യമായ വാഗ്ദാനമാണത്‌. അല്ലാഹുവേക്കാള്‍ സത്യസന്ധമായി സംസാരിക്കുന്നവന്‍ ആരുണ്ട്‌?
\end{malayalam}}
\flushright{\begin{Arabic}
\quranayah[4][123]
\end{Arabic}}
\flushleft{\begin{malayalam}
കാര്യം നിങ്ങളുടെ വ്യാമോഹങ്ങളനുസരിച്ചല്ല. വേദക്കാരുടെ വ്യാമോഹങ്ങളനുസരിച്ചുമല്ല. ആര് തിന്‍മ പ്രവര്‍ത്തിച്ചാലും അതിന്നുള്ള പ്രതിഫലം അവന്ന് നല്‍കപ്പെടും. അല്ലാഹുവിന് പുറമെ തനിക്ക് ഒരു മിത്രത്തെയും സഹായിയെയും അവന്‍ കണ്ടെത്തുകയുമില്ല.
\end{malayalam}}
\flushright{\begin{Arabic}
\quranayah[4][124]
\end{Arabic}}
\flushleft{\begin{malayalam}
ആണാകട്ടെ പെണ്ണാകട്ടെ , ആര്‍ സത്യവിശ്വാസിയായിക്കൊണ്ട് സല്‍പ്രവൃത്തികള്‍ ചെയ്യുന്നുവോ അവര്‍ സ്വര്‍ഗത്തില്‍ പ്രവേശിക്കുന്നതാണ്‌. അവരോട് ഒരു തരിമ്പും അനീതി കാണിക്കപ്പെടുന്നതല്ല.
\end{malayalam}}
\flushright{\begin{Arabic}
\quranayah[4][125]
\end{Arabic}}
\flushleft{\begin{malayalam}
സദ്‌വൃത്തനായിക്കൊണ്ട് തന്‍റെ മുഖത്തെ അല്ലാഹുവിന് കീഴ്പെടുത്തുകയും, നേര്‍മാര്‍ഗത്തിലുറച്ച് നിന്ന് കൊണ്ട് ഇബ്രാഹീമിന്‍റെ മാര്‍ഗത്തെ പിന്തുടരുകയും ചെയ്തവനേക്കാള്‍ ഉത്തമ മതക്കാരന്‍ ആരുണ്ട്‌? അല്ലാഹു ഇബ്രാഹീമിനെ സുഹൃത്തായി സ്വീകരിച്ചിരിക്കുന്നു.
\end{malayalam}}
\flushright{\begin{Arabic}
\quranayah[4][126]
\end{Arabic}}
\flushleft{\begin{malayalam}
ആകാശങ്ങളിലും ഭൂമിയിലുമുള്ളതെല്ലാം അല്ലാഹുവിന്‍റെതാകുന്നു. അല്ലാഹു ഏത് കാര്യത്തെപ്പറ്റിയും പൂര്‍ണ്ണമായ അറിവുള്ളവനാകുന്നു.
\end{malayalam}}
\flushright{\begin{Arabic}
\quranayah[4][127]
\end{Arabic}}
\flushleft{\begin{malayalam}
സ്ത്രീകളുടെ കാര്യത്തില്‍ അവര്‍ നിന്നോട് വിധി തേടുന്നു. പറയുക: അവരുടെ കാര്യത്തില്‍ അല്ലാഹു നിങ്ങള്‍ക്ക് വിധി നല്‍കുന്നു. സ്ത്രീകള്‍ക്ക് നിശ്ചയിക്കപ്പെട്ട അവകാശം നിങ്ങള്‍ നല്‍കാതിരിക്കുകയും, എന്നാല്‍ നിങ്ങള്‍ വിവാഹം കഴിക്കാന്‍ മോഹിക്കുകയും ചെയ്യുന്ന അനാഥ സ്ത്രീകളുടെ കാര്യത്തിലും, ബലഹീനരായ കുട്ടികളുടെ കാര്യത്തിലും ഈ ഗ്രന്ഥത്തില്‍ നിങ്ങള്‍ക്ക് വായിച്ചുകേള്‍പിക്കപ്പെടുന്നത് (നിങ്ങള്‍ ശ്രദ്ധിക്കുകയും ചെയ്യുക.) അനാഥകളോട് നിങ്ങള്‍ നീതിയോടെ വര്‍ത്തിക്കണമെന്ന കല്‍പനയും (ശ്രദ്ധിക്കുക.) നിങ്ങള്‍ ചെയ്യുന്ന ഏതൊരു നല്ലകാര്യവും അല്ലാഹു (പൂര്‍ണ്ണമായി) അറിയുന്നവനാകുന്നു.
\end{malayalam}}
\flushright{\begin{Arabic}
\quranayah[4][128]
\end{Arabic}}
\flushleft{\begin{malayalam}
ഒരു സ്ത്രീ തന്‍റെ ഭര്‍ത്താവില്‍ നിന്ന് പിണക്കമോ അവഗണനയോ ഭയപ്പെടുന്നുവെങ്കില്‍ അവര്‍ പരസ്പരം വല്ല ഒത്തുതീര്‍പ്പും ഉണ്ടാക്കുന്നതില്‍ അവര്‍ക്ക് കുറ്റമില്ല. ഒത്തുതീര്‍പ്പില്‍ എത്തുന്നതാണ് കൂടുതല്‍ നല്ലത്‌. പിശുക്ക് മനസ്സുകളില്‍ നിന്ന് വിട്ട് മാറാത്തതാകുന്നു. നിങ്ങള്‍ നല്ല നിലയില്‍ വര്‍ത്തിക്കുകയും, സൂക്ഷ്മത പാലിക്കുകയുമാണെങ്കില്‍ തീര്‍ച്ചയായും അല്ലാഹു നിങ്ങള്‍ ചെയ്തു കൊണ്ടിരിക്കുന്നതെല്ലാം സൂക്ഷ്മമായി അറിയുന്നവനാകുന്നു.
\end{malayalam}}
\flushright{\begin{Arabic}
\quranayah[4][129]
\end{Arabic}}
\flushleft{\begin{malayalam}
നിങ്ങള്‍ എത്രതന്നെ ആഗ്രഹിച്ചാലും ഭാര്യമാര്‍ക്കിടയില്‍ തുല്യനീതി പാലിക്കാന്‍ നിങ്ങള്‍ക്കൊരിക്കലും സാധിക്കുകയില്ല. അതിനാല്‍ നിങ്ങള്‍ (ഒരാളിലേക്ക്‌) പൂര്‍ണ്ണമായി തിരിഞ്ഞുകൊണ്ട് മറ്റവളെ കെട്ടിയിട്ടവളെപ്പോലെ വിട്ടേക്കരുത്‌. നിങ്ങള്‍ (പെരുമാറ്റം) നന്നാക്കിത്തീര്‍ക്കുകയും, സൂക്ഷ്മത പാലിക്കുകയും ചെയ്യുന്ന പക്ഷം അല്ലാഹു ഏറെ പൊറുക്കുന്നവനും കരുണാനിധിയുമാകുന്നു.
\end{malayalam}}
\flushright{\begin{Arabic}
\quranayah[4][130]
\end{Arabic}}
\flushleft{\begin{malayalam}
ഇനി അവര്‍ ഇരുവരും വേര്‍പിരിയുകയാണെങ്കില്‍ അല്ലാഹു അവന്‍റെ വിശാലമായ കഴിവില്‍ നിന്ന് അവര്‍ ഓരോരുത്തര്‍ക്കും സ്വാശ്രയത്വം നല്‍കുന്നതാണ്‌. അല്ലാഹു വിപുലമായ കഴിവുള്ളവനും യുക്തിമാനുമാകുന്നു.
\end{malayalam}}
\flushright{\begin{Arabic}
\quranayah[4][131]
\end{Arabic}}
\flushleft{\begin{malayalam}
ആകാശങ്ങളിലുള്ളതും ഭൂമിയിലുള്ളതുമെല്ലാം അല്ലാഹുവിന്‍റെതാകുന്നു. നിങ്ങള്‍ അല്ലാഹുവെ സൂക്ഷിക്കുക എന്ന് നിങ്ങള്‍ക്ക് മുമ്പ് വേദം നല്‍കപ്പെട്ടവരോടും, നിങ്ങളോടും നാം വസ്വിയ്യത്ത് ചെയ്തിരിക്കുന്നു. നിങ്ങള്‍ അവിശ്വസിക്കുന്ന പക്ഷം (അല്ലാഹുവിന് ഒരു നഷ്ടവുമില്ല. കാരണം) ആകാശങ്ങളിലും ഭൂമിയിലുമുള്ളതെല്ലാം അല്ലാഹുവിന്‍റെതാകുന്നു. അല്ലാഹു പരാശ്രയമുക്തനും സ്തുത്യര്‍ഹനുമാകുന്നു.
\end{malayalam}}
\flushright{\begin{Arabic}
\quranayah[4][132]
\end{Arabic}}
\flushleft{\begin{malayalam}
ആകാശങ്ങളിലും ഭൂമിയിലുമുള്ളതെല്ലാം അല്ലാഹുവിന്‍റെതാകുന്നു. കൈകാര്യകര്‍ത്താവായി അല്ലാഹു മതി.
\end{malayalam}}
\flushright{\begin{Arabic}
\quranayah[4][133]
\end{Arabic}}
\flushleft{\begin{malayalam}
ജനങ്ങളേ, അവന്‍ ഉദ്ദേശിക്കുന്നുവെങ്കില്‍ നിങ്ങളെ അവന്‍ നീക്കം ചെയ്യുകയും, മറ്റൊരു വിഭാഗത്തെ അവന്‍ കൊണ്ട് വരികയും ചെയ്യുന്നതാണ്‌. അല്ലാഹു അതിന് കഴിവുള്ളവനത്രെ.
\end{malayalam}}
\flushright{\begin{Arabic}
\quranayah[4][134]
\end{Arabic}}
\flushleft{\begin{malayalam}
വല്ലവനും ഇഹലോകത്തെ പ്രതിഫലമാണ് ലക്ഷ്യമാക്കുന്നതെങ്കില്‍ (അവന്‍ മനസ്സിലാക്കട്ടെ) അല്ലാഹുവിന്‍റെ പക്കല്‍ തന്നെയാണ് ഇഹലോകത്തെ പ്രതിഫലവും പരലോകത്ത പ്രതിഫലവും.അല്ലാഹു എല്ലാം കേള്‍ക്കുന്നവനും കാണുന്നവനുമാകുന്നു.
\end{malayalam}}
\flushright{\begin{Arabic}
\quranayah[4][135]
\end{Arabic}}
\flushleft{\begin{malayalam}
സത്യവിശ്വാസികളേ, നിങ്ങള്‍ അല്ലാഹുവിന് വേണ്ടി സാക്ഷ്യം വഹിക്കുന്നവരെന്ന നിലയില്‍ കണിശമായി നീതി നിലനിര്‍ത്തുന്നവരായിരിക്കണം. അത് നിങ്ങള്‍ക്ക് തന്നെയോ, നിങ്ങളുടെ മാതാപിതാക്കള്‍, അടുത്ത ബന്ധുക്കള്‍ എന്നിവര്‍ക്കോ പ്രതികൂലമായിത്തീര്‍ന്നാലും ശരി. (കക്ഷി) ധനികനോ, ദരിദ്രനോ ആകട്ടെ, ആ രണ്ട് വിഭാഗത്തോടും കൂടുതല്‍ ബന്ധപ്പെട്ടവന്‍ അല്ലാഹുവാകുന്നു. അതിനാല്‍ നിങ്ങള്‍ നീതി പാലിക്കാതെ തന്നിഷ്ടങ്ങളെ പിന്‍പറ്റരുത്‌. നിങ്ങള്‍ വളച്ചൊടിക്കുകയോ ഒഴിഞ്ഞ് മാറുകയോ ചെയ്യുന്ന പക്ഷം തീര്‍ച്ചയായും നിങ്ങള്‍ പ്രവര്‍ത്തിക്കുന്നതിനെപ്പറ്റിയെല്ലാം സൂക്ഷ്മമായി അറിയുന്നവനാകുന്നു അല്ലാഹു.
\end{malayalam}}
\flushright{\begin{Arabic}
\quranayah[4][136]
\end{Arabic}}
\flushleft{\begin{malayalam}
സത്യവിശ്വാസികളേ, അല്ലാഹുവിലും, അവന്‍റെ ദൂതനിലും, അവന്‍റെ ദൂതന്ന് അവന്‍ അവതരിപ്പിച്ച ഗ്രന്ഥത്തിലും മുമ്പ് അവന്‍ അവതരിപ്പിച്ച ഗ്രന്ഥത്തിലും നിങ്ങള്‍ വിശ്വസിക്കുവിന്‍. അല്ലാഹുവിലും, അവന്‍റെ മലക്കുകളിലും, അവന്‍റെ ഗ്രന്ഥങ്ങളിലും അവന്‍റെ ദൂതന്‍മാരിലും, അന്ത്യദിനത്തിലും വല്ലവനും അവിശ്വസിക്കുന്ന പക്ഷം തീര്‍ച്ചയായും അവന്‍ ബഹുദൂരം പിഴച്ചു പോയിരിക്കുന്നു.
\end{malayalam}}
\flushright{\begin{Arabic}
\quranayah[4][137]
\end{Arabic}}
\flushleft{\begin{malayalam}
(ഒരിക്കല്‍) വിശ്വസിക്കുകയും പിന്നീട് അവിശ്വസിക്കുകയും, വീണ്ടും വിശ്വസിച്ചിട്ട് പിന്നെയും അവിശ്വസിക്കുകയും, അനന്തരം അവിശ്വാസം കൂടിക്കൂടി വരുകയും ചെയ്തവരാരോ അവര്‍ക്ക് അല്ലാഹു പൊറുത്തുകൊടുക്കുകയേ ഇല്ല. അവരെ അവന്‍ നേര്‍വഴിയിലേക്ക് നയിക്കുന്നതുമല്ല.
\end{malayalam}}
\flushright{\begin{Arabic}
\quranayah[4][138]
\end{Arabic}}
\flushleft{\begin{malayalam}
കപടവിശ്വാസികള്‍ക്ക് വേദനയേറിയ ശിക്ഷയുണ്ട് എന്ന സന്തോഷവാര്‍ത്ത നീ അവരെ അറിയിക്കുക.
\end{malayalam}}
\flushright{\begin{Arabic}
\quranayah[4][139]
\end{Arabic}}
\flushleft{\begin{malayalam}
സത്യവിശ്വാസികളെ വിട്ട് സത്യനിഷേധികളെ ഉറ്റമിത്രങ്ങളായി സ്വീകരിക്കുന്നവരാകുന്നു അവര്‍. അവരുടെ (സത്യനിഷേധികളുടെ) അടുക്കല്‍ പ്രതാപം തേടിപ്പോകുകയാണോ അവര്‍? എന്നാല്‍ തീര്‍ച്ചയായും പ്രതാപം മുഴുവന്‍ അല്ലാഹുവിന്‍റെ അധീനത്തിലാകുന്നു.
\end{malayalam}}
\flushright{\begin{Arabic}
\quranayah[4][140]
\end{Arabic}}
\flushleft{\begin{malayalam}
അല്ലാഹുവിന്‍റെ വചനങ്ങള്‍ നിഷേധിക്കപ്പെടുന്നതും പരിഹസിക്കപ്പെടുന്നതും നിങ്ങള്‍ കേട്ടാല്‍ അത്തരക്കാര്‍ മറ്റുവല്ല വര്‍ത്തമാനത്തിലും പ്രവേശിക്കുന്നത് വരെ നിങ്ങള്‍ അവരോടൊപ്പം ഇരിക്കരുതെന്നും, അങ്ങനെ ഇരിക്കുന്ന പക്ഷം നിങ്ങളും അവരെപ്പോലെത്തന്നെ ആയിരിക്കുമെന്നും ഈ ഗ്രന്ഥത്തില്‍ അല്ലാഹു നിങ്ങള്‍ക്ക് അവതരിപ്പിച്ചു തന്നിട്ടുണ്ടല്ലോ. കപടവിശ്വാസികളെയും അവിശ്വാസികളെയും ഒന്നിച്ച് അല്ലാഹു നരകത്തില്‍ ഒരുമിച്ചുകൂട്ടുക തന്നെചെയ്യും.
\end{malayalam}}
\flushright{\begin{Arabic}
\quranayah[4][141]
\end{Arabic}}
\flushleft{\begin{malayalam}
നിങ്ങളുടെ സ്ഥിതിഗതികള്‍ ഉറ്റുനോക്കിക്കൊണ്ടിരിക്കുന്നവരത്രെ അവര്‍ (കപടവിശ്വാസികള്‍) നിങ്ങള്‍ക്ക് അല്ലാഹുവിങ്കല്‍ നിന്ന് ഒരു വിജയം കൈവന്നാല്‍ അവര്‍ പറയും; ഞങ്ങള്‍ നിങ്ങളുടെ കൂടെയായിരുന്നില്ലേ എന്ന്‌. ഇനി അവിശ്വാസികള്‍ക്കാണ് വല്ല നേട്ടവുമുണ്ടാകുന്നതെങ്കില്‍ അവര്‍ പറയും; നിങ്ങളുടെ മേല്‍ ഞങ്ങള്‍ വിജയ സാധ്യത നേടിയിട്ടും വിശ്വാസികളില്‍ നിന്ന് നിങ്ങളെ ഞങ്ങള്‍ രക്ഷിച്ചില്ലേ എന്ന്‌. എന്നാല്‍ ഉയിര്‍ത്തെഴുന്നേല്‍പിന്‍റെ നാളില്‍ നിങ്ങള്‍ക്കിടയില്‍ അല്ലാഹു വിധി കല്‍പിക്കുന്നതാണ്‌. വിശ്വാസികള്‍ക്കെതിരില്‍ അല്ലാഹു ഒരിക്കലും സത്യനിഷേധികള്‍ക്ക് വഴി തുറന്നുകൊടുക്കുന്നതല്ല.
\end{malayalam}}
\flushright{\begin{Arabic}
\quranayah[4][142]
\end{Arabic}}
\flushleft{\begin{malayalam}
തീര്‍ച്ചയായും കപടവിശ്വാസികള്‍ അല്ലാഹുവെ വഞ്ചിക്കാന്‍ നോക്കുകയാണ്‌. യഥാര്‍ത്ഥത്തില്‍ അല്ലാഹു അവരെയാണ് വഞ്ചിക്കുന്നത്‌. അവര്‍ നമസ്കാരത്തിന് നിന്നാല്‍ ഉദാസീനരായിക്കൊണ്ടും, ആളുകളെ കാണിക്കാന്‍ വേണ്ടിയുമാണ് നില്‍ക്കുന്നത്‌. കുറച്ച് മാത്രമേ അവര്‍ അല്ലാഹുവെ ഓര്‍മിക്കുകയുള്ളൂ.
\end{malayalam}}
\flushright{\begin{Arabic}
\quranayah[4][143]
\end{Arabic}}
\flushleft{\begin{malayalam}
ഈ കക്ഷിയിലേക്കോ, ആ കക്ഷിയിലേക്കോ ചേരാതെ അതിനിടയില്‍ ആടിക്കൊണ്ടിരിക്കുന്നവരാണവര്‍. വല്ലവനെയും അല്ലാഹു വഴിപിഴപ്പിച്ചാല്‍ അവന്ന് പിന്നെ ഒരു മാര്‍ഗവും നീ കണ്ടെത്തുകയില്ല.
\end{malayalam}}
\flushright{\begin{Arabic}
\quranayah[4][144]
\end{Arabic}}
\flushleft{\begin{malayalam}
സത്യവിശ്വാസികളേ, നിങ്ങള്‍ സത്യവിശ്വാസികളെയല്ലാതെ ഉറ്റമിത്രങ്ങളായി സ്വീകരിക്കരുത്‌. അല്ലാഹുവിന് നിങ്ങള്‍ക്കെതിരില്‍ വ്യക്തമായ തെളിവുണ്ടാക്കിവെക്കാന്‍ നിങ്ങള്‍ ആഗ്രഹിക്കുന്നുവോ?
\end{malayalam}}
\flushright{\begin{Arabic}
\quranayah[4][145]
\end{Arabic}}
\flushleft{\begin{malayalam}
തീര്‍ച്ചയായും കപടവിശ്വാസികള്‍ നരകത്തിന്‍റെ അടിത്തട്ടിലാകുന്നു. അവര്‍ക്കൊരു സഹായിയെയും നീ കണ്ടെത്തുന്നതല്ല.
\end{malayalam}}
\flushright{\begin{Arabic}
\quranayah[4][146]
\end{Arabic}}
\flushleft{\begin{malayalam}
എന്നാല്‍ പശ്ചാത്തപിച്ച് മടങ്ങുകയും, നിലപാട് നന്നാക്കുകയും, അല്ലാഹുവെ മുറുകെപിടിക്കുകയും, തങ്ങളുടെ മതത്തെ നിഷ്കളങ്കമായി അല്ലാഹുവിനു വേണ്ടി ആക്കുകയും ചെയ്തവര്‍ ഇതില്‍ നിന്നൊഴിവാകുന്നു, അവര്‍ സത്യവിശ്വാസികളോടൊപ്പമാകുന്നു. സത്യവിശ്വാസികള്‍ക്ക് അല്ലാഹു മഹത്തായ പ്രതിഫലം നല്‍കുന്നതാണ്‌.
\end{malayalam}}
\flushright{\begin{Arabic}
\quranayah[4][147]
\end{Arabic}}
\flushleft{\begin{malayalam}
നിങ്ങള്‍ നന്ദികാണിക്കുകയും, വിശ്വസിക്കുകയും ചെയ്യുന്ന പക്ഷം നിങ്ങളെ ശിക്ഷിച്ചിട്ട് അല്ലാഹുവിന് എന്ത് കിട്ടാനാണ് ? അല്ലാഹു കൃതജ്ഞനും സര്‍വ്വജ്ഞനുമാകുന്നു.
\end{malayalam}}
\flushright{\begin{Arabic}
\quranayah[4][148]
\end{Arabic}}
\flushleft{\begin{malayalam}
ചീത്തവാക്ക് പരസ്യമാക്കുന്നത് അല്ലാഹു ഇഷ്ടപ്പെടുകയില്ല. ദ്രോഹിക്കപ്പെട്ടവന്ന് ഒഴികെ. അല്ലാഹു എല്ലാം കേള്‍ക്കുന്നവനും അറിയുന്നവനുമാകുന്നു.
\end{malayalam}}
\flushright{\begin{Arabic}
\quranayah[4][149]
\end{Arabic}}
\flushleft{\begin{malayalam}
നിങ്ങള്‍ ഒരു നല്ല കാര്യം രഹസ്യമായോ പരസ്യമായോ ചെയ്യുകയാണെങ്കില്‍, അഥവാ, ഒരു ദുഷ്പ്രവൃത്തി മാപ്പ് ചെയ്ത് കൊടുക്കുകയാണെങ്കില്‍ തീര്‍ച്ചയായും അല്ലാഹു ഏറെ മാപ്പുനല്‍കുന്നവനും സര്‍വ്വശക്തനുമാകുന്നു.
\end{malayalam}}
\flushright{\begin{Arabic}
\quranayah[4][150]
\end{Arabic}}
\flushleft{\begin{malayalam}
അല്ലാഹുവിലും അവന്‍റെ ദൂതന്‍മാരിലും അവിശ്വസിക്കുകയും, (വിശ്വാസകാര്യത്തില്‍) അല്ലാഹുവിനും അവന്‍റെ ദൂതന്‍മാര്‍ക്കുമിടയില്‍ വിവേചനം കല്‍പിക്കാന്‍ ആഗ്രഹിക്കുകയും ഞങ്ങള്‍ ചിലരില്‍ വിശ്വസിക്കുകയും, ചിലരെ നിഷേധിക്കുകയും ചെയ്യുന്നു എന്ന് പറയുകയും, അങ്ങനെ അതിന്നിടയില്‍ (വിശ്വാസത്തിനും അവിശ്വാസത്തിനുമിടയില്‍) മറ്റൊരു മാര്‍ഗം സ്വീകരിക്കാന്‍ ഉദ്ദേശിക്കുകയും ചെയ്യുന്നവരാരോ,
\end{malayalam}}
\flushright{\begin{Arabic}
\quranayah[4][151]
\end{Arabic}}
\flushleft{\begin{malayalam}
അവര്‍ തന്നെയാകുന്നു യഥാര്‍ത്ഥത്തില്‍ സത്യനിഷേധികള്‍. സത്യനിഷേധികള്‍ക്ക് അപമാനകരമായ ശിക്ഷ നാം ഒരുക്കിവെച്ചിട്ടുണ്ട്‌.
\end{malayalam}}
\flushright{\begin{Arabic}
\quranayah[4][152]
\end{Arabic}}
\flushleft{\begin{malayalam}
അല്ലാഹുവിലും അവന്‍റെ ദൂതന്‍മാരിലും വിശ്വസിക്കുകയും, അവരില്‍ ആര്‍ക്കിടയിലും വിവേചനം കാണിക്കാതിരിക്കുകയും ചെയ്തവരാരോ അവര്‍ അര്‍ഹിക്കുന്ന പ്രതിഫലം അവര്‍ക്ക് അല്ലാഹു നല്‍കുന്നതാണ്‌. അല്ലാഹു ഏറെ പൊറുക്കുന്നവനും കരുണാനിധിയുമാകുന്നു.
\end{malayalam}}
\flushright{\begin{Arabic}
\quranayah[4][153]
\end{Arabic}}
\flushleft{\begin{malayalam}
വേദക്കാര്‍ നിന്നോട് ആവശ്യപ്പെടുന്നു; നീ അവര്‍ക്ക് ആകാശത്ത് നിന്ന് ഒരു ഗ്രന്ഥം ഇറക്കികൊടുക്കണമെന്ന്‌. എന്നാല്‍ അതിനെക്കാള്‍ ഗുരുതരമായത് അവര്‍ മൂസായോട് ആവശ്യപ്പെട്ടിട്ടുണ്ട് (അതായത്‌) അല്ലാഹുവെ ഞങ്ങള്‍ക്ക് പ്രത്യക്ഷത്തില്‍ കാണിച്ചുതരണം എന്നവര്‍ പറയുകയുണ്ടായി. അപ്പോള്‍ അവരുടെ അക്രമം കാരണം ഇടിത്തീ അവരെ പിടികൂടി. പിന്നെ വ്യക്തമായ തെളിവുകള്‍ വന്നുകിട്ടിയതിന് ശേഷം അവര്‍ കാളക്കുട്ടിയെ (ദൈവമായി) സ്വീകരിച്ചു. എന്നിട്ട് നാം അത് പൊറുത്തുകൊടുത്തു. മൂസായ്ക്ക് നം വ്യക്തമായ ന്യായപ്രമാണം നല്‍കുകയും ചെയ്തു.
\end{malayalam}}
\flushright{\begin{Arabic}
\quranayah[4][154]
\end{Arabic}}
\flushleft{\begin{malayalam}
അവരോട് കരാര്‍ വാങ്ങുവാന്‍ വേണ്ടി നാം അവര്‍ക്ക് മീതെ പര്‍വ്വതത്തെ ഉയര്‍ത്തുകയും ചെയ്തു. നിങ്ങള്‍ (പട്ടണ) വാതില്‍ കടക്കുന്നത് തലകുനിച്ച് കൊണ്ടാകണം എന്ന് നാം അവരോട് പറയുകയും ചെയ്തു. നിങ്ങള്‍ ശബ്ബത്ത് നാളില്‍ അതിക്രമം കാണിക്കരുത് എന്നും നാം അവരോട് പറഞ്ഞു. ഉറപ്പേറിയ ഒരു കരാര്‍ നാമവരോട് വാങ്ങുകയും ചെയ്തു.
\end{malayalam}}
\flushright{\begin{Arabic}
\quranayah[4][155]
\end{Arabic}}
\flushleft{\begin{malayalam}
എന്നിട്ട് അവര്‍ കരാര്‍ ലംഘിച്ചതിനാലും, അല്ലാഹുവിന്‍റെ ദൃഷ്ടാന്തങ്ങള്‍ നിഷേധിച്ചതിനാലും, അന്യായമായി പ്രവാചകന്‍മാരെ കൊലപ്പെടുത്തിയതിനാലും, തങ്ങളുടെ മനസ്സുകള്‍ അടഞ്ഞുകിടക്കുകയാണ് എന്ന് അവര്‍ പറഞ്ഞതിനാലും (അവര്‍ ശപിക്കപ്പെട്ടിരിക്കുന്നു.) തന്നെയുമല്ല, അവരുടെ സത്യനിഷേധം കാരണമായി അല്ലാഹു ആ മനസ്സുകളുടെ മേല്‍ മുദ്രകുത്തിയിരിക്കുകയാണ്‌. ആകയാല്‍ ചുരുക്കത്തിലല്ലാതെ അവര്‍ വിശ്വസിക്കുകയില്ല.
\end{malayalam}}
\flushright{\begin{Arabic}
\quranayah[4][156]
\end{Arabic}}
\flushleft{\begin{malayalam}
അവരുടെ സത്യനിഷേധം കാരണമായും മര്‍യമിന്‍റെ പേരില്‍ അവര്‍ ഗുരുതരമായ അപവാദം പറഞ്ഞതിനാലും
\end{malayalam}}
\flushright{\begin{Arabic}
\quranayah[4][157]
\end{Arabic}}
\flushleft{\begin{malayalam}
അല്ലാഹുവിന്‍റെ ദൂതനായ, മര്‍യമിന്‍റെ മകന്‍ മസീഹ് ഈസായെ ഞങ്ങള്‍ കോന്നിരിക്കുന്നു എന്നവര്‍ പറഞ്ഞതിനാലും (അവര്‍ ശപിക്കപ്പെട്ടിരിക്കുന്നു.) വാസ്തവത്തില്‍ അദ്ദേഹത്തെ അവര്‍ കൊലപ്പെടുത്തിയിട്ടുമില്ല, ക്രൂശിച്ചിട്ടുമില്ല. പക്ഷെ (യാഥാര്‍ത്ഥ്യം) അവര്‍ക്ക് തിരിച്ചറിയാതാവുകയാണുണ്ടായത്‌. തീര്‍ച്ചയായും അദ്ദേഹത്തിന്‍റെ (ഈസായുടെ) കാര്യത്തില്‍ ഭിന്നിച്ചവര്‍ അതിനെപ്പറ്റി സംശയത്തില്‍ തന്നെയാകുന്നു. ഊഹാപോഹത്തെ പിന്തുടരുന്നതല്ലാതെ അവര്‍ക്ക് അക്കാര്യത്തെപ്പറ്റി യാതൊരു അറിവുമില്ല. ഉറപ്പായും അദ്ദേഹത്തെ അവര്‍ കൊലപ്പെടുത്തിയിട്ടില്ല.
\end{malayalam}}
\flushright{\begin{Arabic}
\quranayah[4][158]
\end{Arabic}}
\flushleft{\begin{malayalam}
എന്നാല്‍ അദ്ദേഹത്തെ അല്ലാഹു അവങ്കലേക്ക് ഉയര്‍ത്തുകയത്രെ ചെയ്തത്‌. അല്ലാഹു പ്രതാപിയും യുക്തിമാനുമാകുന്നു.
\end{malayalam}}
\flushright{\begin{Arabic}
\quranayah[4][159]
\end{Arabic}}
\flushleft{\begin{malayalam}
വേദക്കാരില്‍ ആരും തന്നെ അദ്ദേഹത്തിന്‍റെ (ഈസായുടെ) മരണത്തിനുമുമ്പ് അദ്ദേഹത്തില്‍ വിശ്വസിക്കാത്തവരായി ഉണ്ടാവുകയില്ല. ഉയിര്‍ത്തെഴുന്നേല്‍പിന്‍റെ നാളിലാകട്ടെ അദ്ദേഹം അവര്‍ക്കെതിരില്‍ സാക്ഷിയാകുകയും ചെയ്യും.
\end{malayalam}}
\flushright{\begin{Arabic}
\quranayah[4][160]
\end{Arabic}}
\flushleft{\begin{malayalam}
അങ്ങനെ യഹൂദമതം സ്വീകരിച്ചവരുടെ അക്രമം കാരണമായി അവര്‍ക്ക് അനുവദിക്കപ്പെട്ടിരുന്ന പല നല്ല വസ്തുക്കളും നാമവര്‍ക്ക് നിഷിദ്ധമാക്കി. അല്ലാഹുവിന്‍റെ മാര്‍ഗത്തില്‍ നിന്ന് അവര്‍ ജനങ്ങളെ ധാരാളമായി തടഞ്ഞതുകൊണ്ടും.
\end{malayalam}}
\flushright{\begin{Arabic}
\quranayah[4][161]
\end{Arabic}}
\flushleft{\begin{malayalam}
പലിശ അവര്‍ക്ക് നിരോധിക്കപ്പെട്ടതായിട്ടും, അവരത് വാങ്ങിയതുകൊണ്ടും, ജനങ്ങളുടെ സ്വത്തുകള്‍ അവര്‍ അന്യായമായി തിന്നതുകൊണ്ടും കൂടിയാണ് (അത് നിഷിദ്ധമാക്കപ്പെട്ടത്‌.) അവരില്‍ നിന്നുള്ള സത്യനിഷേധികള്‍ക്ക് നാം വേദനയേറിയ ശിക്ഷ ഒരുക്കിവെച്ചിട്ടുണ്ട്‌.
\end{malayalam}}
\flushright{\begin{Arabic}
\quranayah[4][162]
\end{Arabic}}
\flushleft{\begin{malayalam}
എന്നാല്‍ അവരില്‍ നിന്ന് അടിയുറച്ച അറിവുള്ളവരും, സത്യവിശ്വാസികളുമായിട്ടുള്ളവര്‍ നിനക്ക് അവതരിപ്പിക്കപ്പെട്ടതിലും നിനക്ക് മുമ്പ് അവതരിപ്പിക്കപ്പെട്ടതിലും വിശ്വസിക്കുന്നു. പ്രാര്‍ത്ഥന മുറപോലെ നിര്‍വഹിക്കുന്നവരും, സകാത്ത് നല്‍കുന്നവരും, അല്ലാഹുവിലും അന്ത്യദിനത്തിലും വിശ്വസിക്കുന്നവരുമത്രെ അവര്‍. അങ്ങനെയുള്ളവര്‍ക്ക് നാം മഹത്തായ പ്രതിഫലം നല്‍കുന്നതാണ്‌.
\end{malayalam}}
\flushright{\begin{Arabic}
\quranayah[4][163]
\end{Arabic}}
\flushleft{\begin{malayalam}
(നബിയേ,) നൂഹിനും അദ്ദേഹത്തിന്‍റെ ശേഷമുള്ള പ്രവാചകന്‍മാര്‍ക്കും നാം സന്ദേശം നല്‍കിയത് പോലെ തന്നെ നിനക്കും നാം സന്ദേശം നല്‍കിയിരിക്കുന്നു. ഇബ്രാഹീം, ഇസ്മാഈല്‍, ഇഷാഖ്‌, യഅ്ഖൂബ്‌. യഅ്ഖൂബ് സന്തതികള്‍, ഈസാ, അയ്യൂബ്‌, യൂനുസ്‌, ഹാറൂന്‍, സുലൈമാന്‍ എന്നിവര്‍ക്കും നാം സന്ദേശം നല്‍കിയിരിക്കുന്നു. ദാവൂദിന് നാം സബൂര്‍ (സങ്കീര്‍ത്തനം) നല്‍കി.
\end{malayalam}}
\flushright{\begin{Arabic}
\quranayah[4][164]
\end{Arabic}}
\flushleft{\begin{malayalam}
നിനക്ക് നാം മുമ്പ് വിവരിച്ചുതന്നിട്ടുള്ള ദൂതന്‍മാരെയും, നിനക്ക് നാം വിവരിച്ചുതന്നിട്ടില്ലാത്ത ദൂതന്‍മാരെയും (നാം നിയോഗിക്കുകയുണ്ടായി.) മൂസായോട് അല്ലാഹു നേരിട്ട് സംസാരിക്കുകയും ചെയ്തു.
\end{malayalam}}
\flushright{\begin{Arabic}
\quranayah[4][165]
\end{Arabic}}
\flushleft{\begin{malayalam}
സന്തോഷവാര്‍ത്ത അറിയിക്കുന്നവരും, താക്കീത് നല്‍കുന്നവരുമായ ദൂതന്‍മാരായിരുന്നു അവര്‍. ആ ദൂതന്‍മാര്‍ക്ക് ശേഷം ജനങ്ങള്‍ക്ക് അല്ലാഹുവിനെതിരില്‍ ഒരു ന്യായവും ഇല്ലാതിരിക്കാന്‍ വേണ്ടിയാണത്‌. അല്ലാഹു പ്രതാപിയും യുക്തിമാനുമാകുന്നു.
\end{malayalam}}
\flushright{\begin{Arabic}
\quranayah[4][166]
\end{Arabic}}
\flushleft{\begin{malayalam}
എന്നാല്‍ അല്ലാഹു നിനക്ക് അവതരിപ്പിച്ചുതന്നതിന്‍റെ കാര്യത്തില്‍ അവന്‍ തന്നെ സാക്ഷ്യം വഹിക്കുന്നു. അവന്‍റെ അറിവിന്‍റെ അടിസ്ഥാനത്തില്‍ തന്നെയാണ് അവനത് അവതരിപ്പിച്ചിട്ടുള്ളത്‌. മലക്കുകളും (അതിന്‌) സാക്ഷ്യം വഹിക്കുന്നു. സാക്ഷിയായി അല്ലാഹു മതി.
\end{malayalam}}
\flushright{\begin{Arabic}
\quranayah[4][167]
\end{Arabic}}
\flushleft{\begin{malayalam}
അവിശ്വസിക്കുകയും, അല്ലാഹുവിന്‍റെ മാര്‍ഗത്തില്‍ നിന്ന് (ജനങ്ങളെ) തടയുകയും ചെയ്തവര്‍ തീര്‍ച്ചയായും ബഹുദൂരം പിഴച്ച് പോയിരിക്കുന്നു.
\end{malayalam}}
\flushright{\begin{Arabic}
\quranayah[4][168]
\end{Arabic}}
\flushleft{\begin{malayalam}
അവിശ്വസിക്കുകയും, അന്യായം പ്രവര്‍ത്തിക്കുകയും ചെയ്തവരാരോ അവര്‍ക്ക് അല്ലാഹു ഒരിക്കലും പൊറുത്തുകൊടുക്കുന്നതല്ല.
\end{malayalam}}
\flushright{\begin{Arabic}
\quranayah[4][169]
\end{Arabic}}
\flushleft{\begin{malayalam}
നരകത്തിന്‍റെ മാര്‍ഗത്തിലേക്കല്ലാതെ മറ്റൊരു മാര്‍ഗത്തിലേക്കും അവന്‍ അവരെ നയിക്കുന്നതുമല്ല. എന്നെന്നേക്കുമായി അവരതില്‍ സ്ഥിരവാസികളായിരിക്കും. അല്ലാഹുവിന് അത് എളുപ്പമുള്ള കാര്യമാകുന്നു.
\end{malayalam}}
\flushright{\begin{Arabic}
\quranayah[4][170]
\end{Arabic}}
\flushleft{\begin{malayalam}
ജനങ്ങളേ, നിങ്ങളുടെ രക്ഷിതാവിങ്കല്‍ നിന്നുള്ള സത്യവുമായി നിങ്ങളുടെ അടുക്കലിതാ റസൂല്‍ വന്നിരിക്കുന്നു. അതിനാല്‍ നിങ്ങളുടെ നന്‍മയ്ക്കായി നിങ്ങള്‍ വിശ്വസിക്കുക. നിങ്ങള്‍ നിഷേധിക്കുകയാണെങ്കിലോ, ആകാശങ്ങളിലും ഭൂമിയിലുമുള്ളതെല്ലാം അല്ലാഹുവിന്‍റെതാണ്‌. (എന്ന് നിങ്ങള്‍ ഓര്‍ത്തു കൊള്ളുക.) അല്ലാഹു എല്ലാം അറിയുന്നവനും യുക്തിമാനുമാകുന്നു.
\end{malayalam}}
\flushright{\begin{Arabic}
\quranayah[4][171]
\end{Arabic}}
\flushleft{\begin{malayalam}
വേദക്കാരേ, നിങ്ങള്‍ മതകാര്യത്തില്‍ അതിരുകവിയരുത്‌. അല്ലാഹുവിന്‍റെ പേരില്‍ വാസ്തവമല്ലാതെ നിങ്ങള്‍ പറയുകയും ചെയ്യരുത്‌. മര്‍യമിന്‍റെ മകനായ മസീഹ് ഈസാ അല്ലാഹുവിന്‍റെ ദൂതനും, മര്‍യമിലേക്ക് അവന്‍ ഇട്ടുകൊടുത്ത അവന്‍റെ വചനവും, അവങ്കല്‍ നിന്നുള്ള ഒരു ആത്മാവും മാത്രമാകുന്നു. അത് കൊണ്ട് നിങ്ങള്‍ അല്ലാഹുവിലും അവന്‍റെ ദൂതന്‍മാരിലും വിശ്വസിക്കുക. ത്രിത്വം എന്ന വാക്ക് നിങ്ങള്‍ പറയരുത്‌. നിങ്ങളുടെ നന്‍മയ്ക്കായി നിങ്ങള്‍ (ഇതില്‍ നിന്ന്‌) വിരമിക്കുക. അല്ലാഹു ഏക ആരാധ്യന്‍ മാത്രമാകുന്നു. തനിക്ക് ഒരു സന്താനമുണ്ടായിരിക്കുക എന്നതില്‍ നിന്ന് അവനെത്രയോ പരിശുദ്ധനത്രെ. ആകാശങ്ങളിലുള്ളതും ഭൂമിയിലുള്ളതുമെല്ലാം അവന്‍റെതാകുന്നു. കൈകാര്യകര്‍ത്താവായി അല്ലാഹു തന്നെ മതി.
\end{malayalam}}
\flushright{\begin{Arabic}
\quranayah[4][172]
\end{Arabic}}
\flushleft{\begin{malayalam}
അല്ലാഹുവിന്‍റെ അടിമയായിരിക്കുന്നതില്‍ മസീഹ് ഒരിക്കലും വൈമനസ്യം കാണിക്കുന്നതല്ല. (അല്ലാഹുവിന്‍റെ) സാമീപ്യം സിദ്ധിച്ച മലക്കുകളും (വൈമനസ്യം കാണിക്കുന്നതല്ല.) അവനെ (അല്ലാഹുവെ) ആരാധിക്കുന്നതില്‍ ആര്‍ വൈമനസ്യം കാണിക്കുകയും, അഹംഭാവം നടിക്കുകയും ചെയ്യുന്നുവോ അവരെ മുഴുവനും അവന്‍ തന്‍റെ അടുക്കലേക്ക് ഒരുമിച്ചുകൂട്ടുന്നതാണ്‌.
\end{malayalam}}
\flushright{\begin{Arabic}
\quranayah[4][173]
\end{Arabic}}
\flushleft{\begin{malayalam}
എന്നാല്‍ വിശ്വസിക്കുകയും, സല്‍കര്‍മ്മങ്ങള്‍ പ്രവര്‍ത്തിക്കുകയും ചെയ്തത് ആരോ അവരുടെതായ പ്രതിഫലം അവര്‍ക്കവന്‍ നിറവേറ്റികൊടുക്കുകയും, അവന്‍റെ അനുഗ്രഹത്തില്‍ നിന്ന് കൂടുതലായി അവര്‍ക്ക് നല്‍കുകയും ചെയ്യുന്നതാണ്‌. എന്നാല്‍, വൈമനസ്യം കാണിക്കുകയും, അഹങ്കരിക്കുകയും ചെയ്തവരാരോ അവര്‍ക്കവന്‍ വേദനയേറിയ ശിക്ഷ നല്‍കുന്നതാണ്‌. അല്ലാഹുവെ കൂടാതെ തങ്ങള്‍ക്ക് ഒരു ഉറ്റമിത്രത്തെയോ സഹായിയെയോ അവര്‍ കണ്ടെത്തുകയുമില്ല.
\end{malayalam}}
\flushright{\begin{Arabic}
\quranayah[4][174]
\end{Arabic}}
\flushleft{\begin{malayalam}
മനുഷ്യരേ, നിങ്ങള്‍ക്കിതാ നിങ്ങളുടെ രക്ഷിതാവിങ്കല്‍ നിന്നുള്ള ന്യായപ്രമാണം വന്നുകിട്ടിയിരിക്കുന്നു. വ്യക്തമായ ഒരു പ്രകാശം നാമിതാ നിങ്ങള്‍ക്ക് ഇറക്കിത്തന്നിരിക്കുന്നു.
\end{malayalam}}
\flushright{\begin{Arabic}
\quranayah[4][175]
\end{Arabic}}
\flushleft{\begin{malayalam}
അതുകൊണ്ട് ആര്‍ അല്ലാഹുവില്‍ വിശ്വസിക്കുകയും, അവനെ മുറുകെപിടിക്കുകയും ചെയ്തുവോ, അവരെ തന്‍റെ കാരുണ്യത്തിലും അനുഗ്രഹത്തിലും അവന്‍ പ്രവേശിപ്പിക്കുന്നതാണ്‌. അവങ്കലേക്ക് അവരെ നേര്‍വഴിയിലൂടെ അവന്‍ നയിക്കുന്നതുമാണ്‌.
\end{malayalam}}
\flushright{\begin{Arabic}
\quranayah[4][176]
\end{Arabic}}
\flushleft{\begin{malayalam}
(നബിയേ,) അവര്‍ നിന്നോട് മതവിധി അന്വേഷിക്കുന്നു. പറയുക: കലാലത്തിന്‍റെ പ്രശ്നത്തില്‍ അല്ലാഹു നിങ്ങള്‍ക്കിതാ മതവിധി പറഞ്ഞുതരുന്നു. അതായത് ഒരാള്‍ മരിച്ചു; അയാള്‍ക്ക് സന്താനമില്ല; ഒരു സഹോദരിയുണ്ട്‌. എങ്കില്‍ അയാള്‍ വിട്ടേച്ചു പോയതിന്‍റെ പകുതി അവള്‍ക്കുള്ളതാണ്‌. ഇനി (സഹോദരി മരിക്കുകയും) അവള്‍ക്ക് സന്താനമില്ലാതിരിക്കുകയുമാണെങ്കില്‍ സഹോദരന്‍ അവളുടെ (പൂര്‍ണ്ണ) അവകാശിയായിരിക്കും. ഇനി രണ്ട് സഹോദരികളാണുള്ളതെങ്കില്‍, അവന്‍ (സഹോദരന്‍) വിട്ടേച്ചുപോയ സ്വത്തിന്‍റെ മൂന്നില്‍ രണ്ടു ഭാഗം അവര്‍ക്കുള്ളതാണ്‌. ഇനി സഹോദരന്‍മാരും സഹോദരിമാരും കൂടിയാണുള്ളതെങ്കില്‍, ആണിന് രണ്ട് പെണ്ണിന്‍റെതിന് തുല്യമായ ഓഹരിയാണുള്ളത്‌. നിങ്ങള്‍ പിഴച്ച് പോകുമെന്ന് കരുതി അല്ലാഹു നിങ്ങള്‍ക്ക് കാര്യങ്ങള്‍ വിവരിച്ചുതരുന്നു. അല്ലാഹു ഏത് കാര്യത്തെപ്പറ്റിയും അറിവുള്ളവനാകുന്നു.
\end{malayalam}}
\chapter{\textmalayalam{മാഇദ ( ഭക്ഷണ തളിക )}}
\begin{Arabic}
\Huge{\centerline{\basmalah}}\end{Arabic}
\flushright{\begin{Arabic}
\quranayah[5][1]
\end{Arabic}}
\flushleft{\begin{malayalam}
സത്യവിശ്വാസികളേ, നിങ്ങള്‍ കരാറുകള്‍ നിറവേറ്റുക. (പിന്നീട്‌) നിങ്ങള്‍ക്ക് വിവരിച്ചുതരുന്നതൊഴിച്ചുള്ള ആട്‌, മാട്‌, ഒട്ടകം എന്നീ ഇനങ്ങളില്‍ പെട്ട മൃഗങ്ങള്‍ നിങ്ങള്‍ക്ക് അനുവദിക്കപ്പെട്ടിരിക്കുന്നു. എന്നാല്‍ നിങ്ങള്‍ ഇഹ്‌റാമില്‍ പ്രവേശിച്ചവരായിരിക്കെ വേട്ടയാടുന്നത് അനുവദനീയമാക്കരുത്‌. തീര്‍ച്ചയായും അല്ലാഹു അവന്‍ ഉദ്ദേശിക്കുന്നത് വിധിക്കുന്നു.
\end{malayalam}}
\flushright{\begin{Arabic}
\quranayah[5][2]
\end{Arabic}}
\flushleft{\begin{malayalam}
സത്യവിശ്വാസികളേ, അല്ലാഹുവിന്‍റെ മതചിഹ്നങ്ങളെ നിങ്ങള്‍ അനാദരിക്കരുത്‌. പവിത്രമായ മാസത്തെയും (കഅ്ബത്തിങ്കലേക്ക് കൊണ്ടുപോകുന്ന) ബലിമൃഗങ്ങളെയും, (അവയുടെ കഴുത്തിലെ) അടയാളത്താലികളെയും നിങ്ങളുടെ രക്ഷിതാവിന്‍റെ അനുഗ്രഹവും പൊരുത്തവും തേടിക്കൊണ്ട് വിശുദ്ധ മന്ദിരത്തെ ലക്ഷ്യമാക്കിപ്പോകുന്ന തീര്‍ത്ഥാടകരെയും (നിങ്ങള്‍ അനാദരിക്കരുത്‌.) എന്നാല്‍ ഇഹ്‌റാമില്‍ നിന്ന് നിങ്ങള്‍ ഒഴിവായാല്‍ നിങ്ങള്‍ക്ക് വേട്ടയാടാവുന്നതാണ്‌. മസ്ജിദുല്‍ ഹറാമില്‍ നിന്ന് നിങ്ങളെ തടഞ്ഞു എന്നതിന്‍റെ പേരില്‍ ഒരു ജനവിഭാഗത്തോട് നിങ്ങള്‍ക്കുള്ള അമര്‍ഷം അതിക്രമം പ്രവര്‍ത്തിക്കുന്നതിന്ന് നിങ്ങള്‍ക്കൊരിക്കലും പ്രേരകമാകരുത്‌. പുണ്യത്തിലും ധര്‍മ്മനിഷ്ഠയിലും നിങ്ങള്‍ അന്യോന്യം സഹായിക്കുക. പാപത്തിലും അതിക്രമത്തിലും നിങ്ങള്‍ അന്യോന്യം സഹായിക്കരുത്‌. നിങ്ങള്‍ അല്ലാഹുവെ സൂക്ഷിക്കുക. തീര്‍ച്ചയായും അല്ലാഹു കഠിനമായി ശിക്ഷിക്കുന്നവനാകുന്നു.
\end{malayalam}}
\flushright{\begin{Arabic}
\quranayah[5][3]
\end{Arabic}}
\flushleft{\begin{malayalam}
ശവം, രക്തം, പന്നിമാംസം, അല്ലാഹു അല്ലാത്തവരുടെ പേരില്‍ അറുക്കപ്പെട്ടത്‌, ശ്വാസം മുട്ടി ചത്തത്‌, അടിച്ചുകൊന്നത്‌, വീണുചത്തത്‌, കുത്തേറ്റ് ചത്തത്‌, വന്യമൃഗം കടിച്ചുതിന്നത് എന്നിവ നിങ്ങള്‍ക്ക് നിഷിദ്ധമാക്കപ്പെട്ടിരിക്കുന്നു. എന്നാല്‍ (ജീവനോടെ) നിങ്ങള്‍ അറുത്തത് ഇതില്‍ നിന്നൊഴിവാകുന്നു. പ്രതിഷ്ഠകള്‍ക്കുമുമ്പില്‍ ബലിയര്‍പ്പിക്കപ്പെട്ടതും (നിങ്ങള്‍ക്ക്‌) നിഷിദ്ധമാകുന്നു. അമ്പുകളുപയോഗിച്ച് ഭാഗ്യം നോക്കലും (നിങ്ങള്‍ക്ക് നിഷിദ്ധമാക്കപ്പെട്ടിരിക്കുന്നു.) അതൊക്കെ അധര്‍മ്മമാകുന്നു. ഇന്ന് സത്യനിഷേധികള്‍ നിങ്ങളുടെ മതത്തെ നേരിടുന്ന കാര്യത്തില്‍ നിരാശപ്പെട്ടിരിക്കുകയാണ്‌. അതിനാല്‍ അവരെ നിങ്ങള്‍ പേടിക്കേണ്ടതില്ല. എന്നെ നിങ്ങള്‍ പേടിക്കുക. ഇന്ന് ഞാന്‍ നിങ്ങള്‍ക്ക് നിങ്ങളുടെ മതം പൂര്‍ത്തിയാക്കി തന്നിരിക്കുന്നു. എന്‍റെ അനുഗ്രഹം നിങ്ങള്‍ക്ക് ഞാന്‍ നിറവേറ്റിത്തരികയും ചെയ്തിരിക്കുന്നു. മതമായി ഇസ്ലാമിനെ ഞാന്‍ നിങ്ങള്‍ക്ക് തൃപ്തിപ്പെട്ട് തന്നിരിക്കുന്നു. വല്ലവനും പട്ടിണി കാരണം (നിഷിദ്ധമായത്‌) തിന്നുവാന്‍ നിര്‍ബന്ധിതനാകുന്ന പക്ഷം അവന്‍ അധര്‍മ്മത്തിലേക്ക് ചായ്‌വുള്ളവനല്ലെങ്കില്‍ തീര്‍ച്ചയായും അല്ലാഹു ഏറെ പൊറുക്കുന്നവനും കരുണ കാണിക്കുന്നവനുമാകുന്നു.
\end{malayalam}}
\flushright{\begin{Arabic}
\quranayah[5][4]
\end{Arabic}}
\flushleft{\begin{malayalam}
തങ്ങള്‍ക്ക് അനുവദിക്കപ്പെട്ടിട്ടുള്ളത് എന്തൊക്കെയാണെന്ന് അവര്‍ നിന്നോട് ചോദിക്കും. പറയുക: നല്ല വസ്തുക്കളെല്ലാം നിങ്ങള്‍ക്ക് അനുവദിക്കപ്പെട്ടിരിക്കുന്നു. അല്ലാഹു നിങ്ങള്‍ക്ക് നല്‍കിയ വിദ്യ ഉപയോഗിച്ച് നായാട്ട് പരിശീലിപ്പിക്കാറുള്ള രീതിയില്‍ നിങ്ങള്‍ പഠിപ്പിച്ചെടുത്ത ഏതെങ്കിലും വേട്ടമൃഗം നിങ്ങള്‍ക്ക് വേണ്ടി പിടിച്ച് കൊണ്ടുവന്നതില്‍ നിന്ന് നിങ്ങള്‍ തിന്നുകൊള്ളുക. ആ ഉരുവിന്‍റെ മേല്‍ നിങ്ങള്‍ അല്ലാഹുവിന്‍റെ നാമം ഉച്ചരിക്കുകയും ചെയ്യുക. നിങ്ങള്‍ അല്ലാഹുവെ സൂക്ഷിക്കുക. തീര്‍ച്ചയായും അല്ലാഹു അതിവേഗം കണക്ക് നോക്കുന്നവനാകുന്നു.
\end{malayalam}}
\flushright{\begin{Arabic}
\quranayah[5][5]
\end{Arabic}}
\flushleft{\begin{malayalam}
എല്ലാ നല്ല വസ്തുക്കളും ഇന്ന് നിങ്ങള്‍ക്ക് അനുവദിക്കപ്പെട്ടിരിക്കുന്നു. വേദം നല്‍കപ്പെട്ടവരുടെ ഭക്ഷണം നിങ്ങള്‍ക്ക് അനുവദനീയമാണ്‌. നിങ്ങളുടെ ഭക്ഷണം അവര്‍ക്കും അനുവദനീയമാണ്‌. സത്യവിശ്വാസിനികളില്‍ നിന്നുള്ള പതിവ്രതകളായ സ്ത്രീകളും, നിങ്ങള്‍ക്ക് മുമ്പ് വേദം നല്‍കപ്പെട്ടവരില്‍ നിന്നുള്ള പതിവ്രതകളായ സ്ത്രീകളും - നിങ്ങള വര്‍ക്ക് വിവാഹമൂല്യം നല്‍കിക്കഴിഞ്ഞിട്ടുണ്ടെങ്കില്‍ - (നിങ്ങള്‍ക്ക് അനുവദിക്കപ്പെട്ടിരിക്കുന്നു.) നിങ്ങള്‍ വൈവാഹിക ജീവിതത്തില്‍ ഒതുങ്ങി നില്‍ക്കുന്നവരായിരിക്കണം. വ്യഭിചാരത്തില്‍ ഏര്‍പെടുന്നവരാകരുത്‌. രഹസ്യവേഴ്ചക്കാരെ സ്വീകരിക്കുന്നവരുമാകരുത്‌. സത്യവിശ്വാസത്തെ ആരെങ്കിലും തള്ളിക്കളയുന്ന പക്ഷം അവന്‍റെ കര്‍മ്മം നിഷ്ഫലമായിക്കഴിഞ്ഞു. പരലോകത്ത് അവന്‍ നഷ്ടം പറ്റിയവരുടെ കൂട്ടത്തിലായിരിക്കുകയും ചെയ്യും.
\end{malayalam}}
\flushright{\begin{Arabic}
\quranayah[5][6]
\end{Arabic}}
\flushleft{\begin{malayalam}
സത്യവിശ്വാസികളേ, നിങ്ങള്‍ നമസ്കാരത്തിന് ഒരുങ്ങിയാല്‍, നിങ്ങളുടെ മുഖങ്ങളും, മുട്ടുവരെ രണ്ടുകൈകളും കഴുകുകയും, നിങ്ങളുടെ തല തടവുകയും നെരിയാണിവരെ രണ്ട് കാലുകള്‍ കഴുകുകയും ചെയ്യുക. നിങ്ങള്‍ ജനാബത്ത് (വലിയ അശുദ്ധി) ബാധിച്ചവരായാല്‍ നിങ്ങള്‍ (കുളിച്ച്‌) ശുദ്ധിയാകുക. നിങ്ങള്‍ രോഗികളാകുകയോ യാത്രയിലാകുകയോ ചെയ്താല്‍, അല്ലെങ്കില്‍ നിങ്ങളിലൊരാള്‍ മലമൂത്രവിസര്‍ജ്ജനം കഴിഞ്ഞ് വരികയോ, നിങ്ങള്‍ സ്ത്രീകളുമായി സംസര്‍ഗം നടത്തുകയോ ചെയ്തിട്ട് നിങ്ങള്‍ക്ക് വെള്ളം കിട്ടിയില്ലെങ്കില്‍ ശുദ്ധമായ ഭൂമുഖം തേടിക്കൊള്ളുക. എന്നിട്ട് അതുകൊണ്ട് നിങ്ങളുടെ മുഖവും കൈകളും തടവുക. നിങ്ങള്‍ക്ക് ഒരു ബുദ്ധിമുട്ടും വരുത്തിവെക്കണമെന്ന് അല്ലാഹു ഉദ്ദേശിക്കുന്നില്ല. എന്നാല്‍ നിങ്ങളെ ശുദ്ധീകരിക്കണമെന്നും, തന്‍റെ അനുഗ്രഹം നിങ്ങള്‍ക്ക് പൂര്‍ത്തിയാക്കിത്തരണമെന്നും അവന്‍ ഉദ്ദേശിക്കുന്നു. നിങ്ങള്‍ നന്ദിയുള്ളവരായേക്കാം.
\end{malayalam}}
\flushright{\begin{Arabic}
\quranayah[5][7]
\end{Arabic}}
\flushleft{\begin{malayalam}
അല്ലാഹു നിങ്ങള്‍ക്ക് ചെയ്തു തന്ന അനുഗ്രഹം നിങ്ങള്‍ ഓര്‍ക്കുക. ഞങ്ങളിതാ കേള്‍ക്കുകയും അനുസരിക്കുകയും ചെയ്തിരിക്കുന്നു എന്ന് നിങ്ങള്‍ പറഞ്ഞ സന്ദര്‍ഭത്തില്‍ അല്ലാഹു നിങ്ങളോട് ഉറപ്പേറിയ കരാര്‍ വാങ്ങിയതും (ഓര്‍ക്കുക) നിങ്ങള്‍ അല്ലാഹുവെ സൂക്ഷിക്കുക. തീര്‍ച്ചയായും അല്ലാഹു മനസ്സുകളിലുള്ളത് അറിയുന്നവനാകുന്നു.
\end{malayalam}}
\flushright{\begin{Arabic}
\quranayah[5][8]
\end{Arabic}}
\flushleft{\begin{malayalam}
സത്യവിശ്വാസികളേ, നിങ്ങള്‍ അല്ലാഹുവിന്ന് വേണ്ടി നിലകൊള്ളുന്നവരും, നീതിക്ക് സാക്ഷ്യം വഹിക്കുന്നവരുമായിരിക്കുക. ഒരു ജനതയോടുള്ള അമര്‍ഷം നീതി പാലിക്കാതിരിക്കാന്‍ നിങ്ങള്‍ക്ക് പ്രേരകമാകരുത്‌. നിങ്ങള്‍ നീതി പാലിക്കുക. അതാണ് ധര്‍മ്മനിഷ്ഠയോട് ഏറ്റവും അടുത്തത്‌. നിങ്ങള്‍ അല്ലാഹുവെ സൂക്ഷിക്കുക. തീര്‍ച്ചയായും നിങ്ങള്‍ പ്രവര്‍ത്തിക്കുന്നതിനെ കുറിച്ചെല്ലാം അല്ലാഹു സൂക്ഷ്മമായി അറിയുന്നവനാകുന്നു.
\end{malayalam}}
\flushright{\begin{Arabic}
\quranayah[5][9]
\end{Arabic}}
\flushleft{\begin{malayalam}
വിശ്വസിക്കുകയും, സല്‍കര്‍മ്മങ്ങള്‍ പ്രവര്‍ത്തിക്കുകയും ചെയ്തവരോട് അല്ലാഹു വാഗ്ദാനം ചെയ്തിരിക്കുന്നു. അവര്‍ക്ക് പാപമോചനവും മഹത്തായ പ്രതിഫലവും ഉണ്ടെന്ന്‌.
\end{malayalam}}
\flushright{\begin{Arabic}
\quranayah[5][10]
\end{Arabic}}
\flushleft{\begin{malayalam}
അവിശ്വസിക്കുകയും നമ്മുടെ ദൃഷ്ടാന്തങ്ങളെ നിഷേധിക്കുകയും ചെയ്തവരാരോ അവരാകുന്നു നരകാവകാശികള്‍.
\end{malayalam}}
\flushright{\begin{Arabic}
\quranayah[5][11]
\end{Arabic}}
\flushleft{\begin{malayalam}
സത്യവിശ്വാസികളേ, ഒരു ജനവിഭാഗം നിങ്ങളുടെ നേരെ (ആക്രമണാര്‍ത്ഥം) അവരുടെ കൈകള്‍ നീട്ടുവാന്‍ മുതിര്‍ന്നപ്പോള്‍, അവരുടെ കൈകളെ നിങ്ങളില്‍ നിന്ന് തട്ടിമാറ്റിക്കൊണ്ട് അല്ലാഹു നിങ്ങള്‍ക്ക് ചെയ്തു തന്ന അനുഗ്രഹം നിങ്ങള്‍ ഓര്‍ക്കുവിന്‍. നിങ്ങള്‍ അല്ലാഹുവെ സൂക്ഷിക്കുക. സത്യവിശ്വാസികള്‍ അല്ലാഹുവില്‍ മാത്രം ഭരമേല്‍പിക്കട്ടെ.
\end{malayalam}}
\flushright{\begin{Arabic}
\quranayah[5][12]
\end{Arabic}}
\flushleft{\begin{malayalam}
അല്ലാഹു ഇസ്രായീല്‍ സന്തതികളോട് കരാര്‍ വാങ്ങുകയും, അവരില്‍ നിന്ന് പന്ത്രണ്ട് നേതാക്കന്‍മാരെ നിയോഗിക്കുകയുമുണ്ടായി. അല്ലാഹു (അവരോട്‌) പറഞ്ഞു: തീര്‍ച്ചയായും ഞാന്‍ നിങ്ങളുടെ കൂടെയുണ്ട്‌. നിങ്ങള്‍ പ്രാര്‍ത്ഥന മുറപോലെ നിര്‍വഹിക്കുകയും, സകാത്ത് നല്‍കുകയും, എന്‍റെ ദൂതന്‍മാരില്‍ വിശ്വസിക്കുകയും, അവരെ സഹായിക്കുകയും, അല്ലാഹുവിന്ന് ഉത്തമമായ കടം കൊടുക്കുകയും ചെയ്തു കൊണ്ടിരിക്കുന്ന പക്ഷം തീര്‍ച്ചയായും നിങ്ങളുടെ തിന്‍മകള്‍ നിങ്ങളില്‍ നിന്ന് ഞാന്‍ മായ്ച്ചുകളയുകയും, താഴ്ഭാഗത്ത് കൂടി അരുവികള്‍ ഒഴുകുന്ന സ്വര്‍ഗത്തോപ്പുകളില്‍ നിങ്ങളെ ഞാന്‍ പ്രവേശിപ്പിക്കുകയും ചെയ്യുന്നതാണ്‌. എന്നാല്‍ അതിനു ശേഷം നിങ്ങളില്‍ നിന്ന് ആര്‍ അവിശ്വസിച്ചുവോ അവന്‍ നേര്‍മാര്‍ഗത്തില്‍ നിന്ന് തെറ്റിപ്പോയിരിക്കുന്നു.
\end{malayalam}}
\flushright{\begin{Arabic}
\quranayah[5][13]
\end{Arabic}}
\flushleft{\begin{malayalam}
അങ്ങനെ അവര്‍ കരാര്‍ ലംഘിച്ചതിന്‍റെ ഫലമായി നാം അവരെ ശപിക്കുകയും, അവരുടെ മനസ്സുകളെ നാം കടുത്തതാക്കിത്തീര്‍ക്കുകയും ചെയ്തു. വേദവാക്യങ്ങളെ അവയുടെ സ്ഥാനങ്ങളില്‍ നിന്ന് അവര്‍ തെറ്റിക്കുന്നു. അവര്‍ക്ക് ഉല്‍ബോധനം നല്‍കപ്പെട്ടതില്‍ ഒരു ഭാഗം അവര്‍ മറന്നുകളയുകയും ചെയ്തു. അവര്‍ - അല്‍പം ചിലരൊഴികെ - നടത്തിക്കൊണ്ടിരിക്കുന്ന വഞ്ചന (മേലിലും) നീ കണ്ടുകൊണ്ടിരിക്കും. എന്നാല്‍ അവര്‍ക്ക് നീ മാപ്പുനല്‍കുകയും അവരോട് വിട്ടുവീഴ്ച കാണിക്കുകയും ചെയ്യുക. നല്ല നിലയില്‍ വര്‍ത്തിക്കുന്നവരെ തീര്‍ച്ചയായും അല്ലാഹു ഇഷ്ടപ്പെടും.
\end{malayalam}}
\flushright{\begin{Arabic}
\quranayah[5][14]
\end{Arabic}}
\flushleft{\begin{malayalam}
ഞങ്ങള്‍ ക്രിസ്ത്യാനികളാണ് എന്ന് പറഞ്ഞവരില്‍ നിന്നും നാം കരാര്‍ വാങ്ങുകയുണ്ടായി. എന്നിട്ട് അവര്‍ക്ക് ഉല്‍ബോധനം നല്‍കപ്പെട്ടതില്‍ നിന്ന് ഒരു ഭാഗം അവര്‍ മറന്നുകളഞ്ഞു. അതിനാല്‍ അവര്‍ക്കിടയില്‍ ഉയിര്‍ത്തെഴുന്നേല്‍പിന്‍റെ നാളുവരേക്കും ശത്രുതയും വിദ്വേഷവും നാം ഇളക്കിവിട്ടു. അവര്‍ ചെയ്ത്കൊണ്ടിരുന്നതിനെപ്പറ്റിയെല്ലാം അല്ലാഹു പിന്നീടവരെ പറഞ്ഞറിയിക്കുന്നതാണ്‌.
\end{malayalam}}
\flushright{\begin{Arabic}
\quranayah[5][15]
\end{Arabic}}
\flushleft{\begin{malayalam}
വേദക്കാരേ, വേദഗ്രന്ഥത്തില്‍ നിന്ന് നിങ്ങള്‍ മറച്ച് വെച്ചുകൊണ്ടിരുന്ന പലതും നിങ്ങള്‍ക്ക് വെളിപ്പെടുത്തിത്തന്നുകൊണ്ട് നമ്മുടെ ദൂതന്‍ (ഇതാ) നിങ്ങളുടെ അടുത്ത് വന്നിരിക്കുന്നു. പലതും അദ്ദേഹം മാപ്പാക്കുകയും ചെയ്യുന്നു. നിങ്ങള്‍ക്കിതാ അല്ലാഹുവിങ്കല്‍ നിന്ന് ഒരു പ്രകാശവും വ്യക്തമായ ഒരു ഗ്രന്ഥവും വന്നിരിക്കുന്നു.
\end{malayalam}}
\flushright{\begin{Arabic}
\quranayah[5][16]
\end{Arabic}}
\flushleft{\begin{malayalam}
അല്ലാഹു തന്‍റെ പൊരുത്തം തേടിയവരെ അത് മുഖേന സമാധാനത്തിന്‍റെ വഴികളിലേക്ക് നയിക്കുന്നു. തന്‍റെ ഉത്തരവ് മുഖേന അവരെ അന്ധകാരങ്ങളില്‍ നിന്ന് അവന്‍ പ്രകാശത്തിലേക്ക് കൊണ്ടുവരുകയും, നേരായ പാതയിലേക്ക് അവരെ നയിക്കുകയും ചെയ്യുന്നു.
\end{malayalam}}
\flushright{\begin{Arabic}
\quranayah[5][17]
\end{Arabic}}
\flushleft{\begin{malayalam}
മര്‍യമിന്‍റെ മകന്‍ മസീഹ് തന്നെയാണ് അല്ലാഹു എന്ന് പറഞ്ഞവര്‍ തീര്‍ച്ചയായും അവിശ്വാസികളായിരിക്കുന്നു. (നബിയേ,) പറയുക: മര്‍യമിന്‍റെ മകന്‍ മസീഹിനെയും അദ്ദേഹത്തിന്‍റെ മാതാവിനെയും, ഭൂമിയിലുള്ള മുഴുവന്‍ പേരെയും അല്ലാഹു നശിപ്പിക്കാന്‍ ഉദ്ദേശിക്കുകയാണെങ്കില്‍ അവന്‍റെ വല്ല നടപടിയിലും സ്വാധീനം ചെലുത്താന്‍ ആര്‍ക്കാണ് കഴിയുക? ആകാശങ്ങളുടെയും, ഭൂമിയുടെയും, അവയ്ക്കിടയിലുള്ളതിന്‍റെയും എല്ലാം ആധിപത്യം അല്ലാഹുവിന്നത്രെ. അവന്‍ ഉദ്ദേശിക്കുന്നത് അവന്‍ സൃഷ്ടിക്കുന്നു. അല്ലാഹു ഏത് കാര്യത്തിനും കഴിവുള്ളവനത്രെ.
\end{malayalam}}
\flushright{\begin{Arabic}
\quranayah[5][18]
\end{Arabic}}
\flushleft{\begin{malayalam}
യഹൂദരും ക്രിസ്ത്യാനികളും പറഞ്ഞു: ഞങ്ങള്‍ അല്ലാഹുവിന്‍റെ മക്കളും അവന്ന് പ്രിയപ്പെട്ടവരുമാകുന്നു എന്ന്‌. (നബിയേ,) പറയുക: പിന്നെ എന്തിനാണ് നിങ്ങളുടെ കുറ്റങ്ങള്‍ക്ക് അല്ലാഹു നിങ്ങളെ ശിക്ഷിക്കുന്നത്‌? അങ്ങനെയല്ല; അവന്‍റെ സൃഷ്ടികളില്‍ പെട്ട മനുഷ്യര്‍ മാത്രമാകുന്നു നിങ്ങള്‍. അവന്‍ ഉദ്ദേശിക്കുന്നവര്‍ക്ക് അവന്‍ പൊറുത്തുകൊടുക്കുകയും, അവന്‍ ഉദ്ദേശിക്കുന്നവരെ അവന്‍ ശിക്ഷിക്കുകയും ചെയ്യും. ആകാശങ്ങളുടെയും, ഭൂമിയുടെയും, അവയ്ക്കിടയിലുള്ളതിന്‍റെയും എല്ലാം ആധിപത്യം അല്ലാഹുവിനത്രെ. അവങ്കലേക്ക് തന്നെയാണ് മടക്കം.
\end{malayalam}}
\flushright{\begin{Arabic}
\quranayah[5][19]
\end{Arabic}}
\flushleft{\begin{malayalam}
വേദക്കാരേ, ദൈവദൂതന്‍മാര്‍ വരാതെ ഒരു ഇടവേള കഴിഞ്ഞ ശേഷം നിങ്ങള്‍ക്ക് (കാര്യങ്ങള്‍) വിവരിച്ചുതന്നു കൊണ്ട് നമ്മുടെ ദൂതന്‍ ഇതാ നിങ്ങളുടെ അടുത്ത് വന്നിരിക്കുന്നു. ഞങ്ങളുടെ അടുത്ത് ഒരു സന്തോഷവാര്‍ത്തക്കാരനോ, താക്കീതുകാരനോ വന്നില്ല എന്ന് നിങ്ങള്‍ പറയാതിരിക്കാന്‍ വേണ്ടിയാണിത്‌. അതെ, നിങ്ങള്‍ക്ക് സന്തോഷവാര്‍ത്ത അറിയിക്കുകയും, താക്കീത് നല്‍കുകയും ചെയ്യുന്ന ആള്‍ (ഇതാ) വന്നു കഴിഞ്ഞിരിക്കുന്നു. അല്ലാഹു ഏത് കാര്യത്തിനും കഴിവുള്ളവനത്രെ.
\end{malayalam}}
\flushright{\begin{Arabic}
\quranayah[5][20]
\end{Arabic}}
\flushleft{\begin{malayalam}
മൂസാ തന്‍റെ ജനതയോട് പറഞ്ഞ സന്ദര്‍ഭം (ഓര്‍ക്കുക:) എന്‍റെ ജനങ്ങളേ, നിങ്ങളില്‍ പ്രവാചകന്‍മാരെ നിയോഗിക്കുകയും, നിങ്ങളെ രാജാക്കന്‍മാരാക്കുകയും, മനുഷ്യരില്‍ നിന്ന് മറ്റാര്‍ക്കും നല്‍കിയിട്ടില്ലാത്ത പലതും നിങ്ങള്‍ക്ക് നല്‍കുകയും ചെയ്ത്കൊണ്ട് അല്ലാഹു നിങ്ങളെ അനുഗ്രഹിച്ചത് നിങ്ങള്‍ ഓര്‍ക്കുക.
\end{malayalam}}
\flushright{\begin{Arabic}
\quranayah[5][21]
\end{Arabic}}
\flushleft{\begin{malayalam}
എന്‍റെ ജനങ്ങളേ, അല്ലാഹു നിങ്ങള്‍ക്ക് വിധിച്ചിട്ടുള്ള പവിത്രഭൂമിയില്‍ നിങ്ങള്‍ പ്രവേശിക്കുവിന്‍. നിങ്ങള്‍ പിന്നോക്കം മടങ്ങരുത്‌. എങ്കില്‍ നിങ്ങള്‍ നഷ്ടക്കാരായി മാറും.
\end{malayalam}}
\flushright{\begin{Arabic}
\quranayah[5][22]
\end{Arabic}}
\flushleft{\begin{malayalam}
അവര്‍ പറഞ്ഞു: ഓ; മൂസാ, പരാക്രമശാലികളായ ഒരു ജനതയാണ് അവിടെയുള്ളത്‌. അവര്‍ അവിടെ നിന്ന് പുറത്ത് പോകുന്നത് വരെ ഞങ്ങള്‍ അവിടെ പ്രവേശിക്കുകയേയില്ല. അവര്‍ അവിടെ നിന്ന് പുറത്ത് പോകുന്ന പക്ഷം തീര്‍ച്ചയായും ഞങ്ങള്‍ (അവിടെ) പ്രവേശിച്ചുകൊള്ളാം.
\end{malayalam}}
\flushright{\begin{Arabic}
\quranayah[5][23]
\end{Arabic}}
\flushleft{\begin{malayalam}
ദൈവഭയമുള്ളവരില്‍ പെട്ട, അല്ലാഹു അനുഗ്രഹിച്ച രണ്ടുപേര്‍ പറഞ്ഞു: നിങ്ങള്‍ അവരുടെ നേര്‍ക്ക് കവാടം കടന്നങ്ങ് ചെല്ലുക. അങ്ങനെ നിങ്ങള്‍ കടന്ന് ചെന്നാല്‍ തീര്‍ച്ചയായും നിങ്ങള്‍ തന്നെയായിരിക്കും ജയിക്കുന്നത്‌. നിങ്ങള്‍ വിശ്വാസികളാണെങ്കില്‍ അല്ലാഹുവില്‍ നിങ്ങള്‍ ഭരമേല്‍പിക്കുക.
\end{malayalam}}
\flushright{\begin{Arabic}
\quranayah[5][24]
\end{Arabic}}
\flushleft{\begin{malayalam}
അപ്പോള്‍ അവര്‍ പറഞ്ഞു: ഓ; മൂസാ, അവരവിടെ ഉണ്ടായിരിക്കുന്ന കാലത്തോളം ഞങ്ങളൊരിക്കലും അവിടെ പ്രവേശിക്കുകയില്ല. അതിനാല്‍ താങ്കളും താങ്കളുടെ രക്ഷിതാവും കൂടിപ്പോയി യുദ്ധം ചെയ്ത് കൊള്ളുക. ഞങ്ങള്‍ ഇവിടെ ഇരിക്കുകയാണ്‌.
\end{malayalam}}
\flushright{\begin{Arabic}
\quranayah[5][25]
\end{Arabic}}
\flushleft{\begin{malayalam}
അദ്ദേഹം (മൂസാ) പറഞ്ഞു: എന്‍റെ രക്ഷിതാവേ, എന്‍റെയും എന്‍റെ സഹോദരന്‍റെയും കാര്യമല്ലാതെ എന്‍റെ അധീനത്തിലില്ല. ആകയാല്‍ ഞങ്ങളെയും ഈ ധിക്കാരികളായ ജനങ്ങളെയും തമ്മില്‍ വേര്‍പിരിക്കേണമേ.
\end{malayalam}}
\flushright{\begin{Arabic}
\quranayah[5][26]
\end{Arabic}}
\flushleft{\begin{malayalam}
അവന്‍ (അല്ലാഹു) പറഞ്ഞു: എന്നാല്‍ ആ നാട് നാല്‍പത് കൊല്ലത്തേക്ക് അവര്‍ക്ക് വിലക്കപ്പെട്ടിരിക്കുകയാണ്‌; തീര്‍ച്ച. (അക്കാലമത്രയും) അവര്‍ ഭൂമിയില്‍ അന്തം വിട്ട് അലഞ്ഞ് നടക്കുന്നതാണ്‌. ആകയാല്‍ ആ ധിക്കാരികളായ ജനങ്ങളുടെ പേരില്‍ നീ ദുഃഖിക്കരുത്‌.
\end{malayalam}}
\flushright{\begin{Arabic}
\quranayah[5][27]
\end{Arabic}}
\flushleft{\begin{malayalam}
(നബിയേ,) നീ അവര്‍ക്ക് ആദമിന്‍റെ രണ്ടുപുത്രന്‍മാരുടെ വൃത്താന്തം സത്യപ്രകാരം പറഞ്ഞുകേള്‍പിക്കുക: അവര്‍ ഇരുവരും ഓരോ ബലിയര്‍പ്പിച്ച സന്ദര്‍ഭം, ഒരാളില്‍ നിന്ന് ബലി സ്വീകരിക്കപ്പെട്ടു. മറ്റവനില്‍ നിന്ന് സ്വീകരിക്കപ്പെട്ടില്ല. മറ്റവന്‍ പറഞ്ഞു: ഞാന്‍ നിന്നെ കൊലപ്പെടുത്തുക തന്നെ ചെയ്യും. അവന്‍ (ബലിസ്വീകരിക്കപ്പെട്ടവന്‍) പറഞ്ഞു: ധര്‍മ്മനിഷ്ഠയുള്ളവരില്‍ നിന്നു മാത്രമേ അല്ലാഹു സ്വീകരിക്കുകയുള്ളൂ
\end{malayalam}}
\flushright{\begin{Arabic}
\quranayah[5][28]
\end{Arabic}}
\flushleft{\begin{malayalam}
എന്നെ കൊല്ലുവാന്‍ വേണ്ടി നീ എന്‍റെ നേരെ കൈനീട്ടിയാല്‍ തന്നെയും, നിന്നെ കൊല്ലുവാന്‍ വേണ്ടി ഞാന്‍ നിന്‍റെ നേരെ കൈനീട്ടുന്നതല്ല. തീര്‍ച്ചയായും ഞാന്‍ ലോകരക്ഷിതാവായ അല്ലാഹുവെ ഭയപ്പെടുന്നു.
\end{malayalam}}
\flushright{\begin{Arabic}
\quranayah[5][29]
\end{Arabic}}
\flushleft{\begin{malayalam}
എന്‍റെ കുറ്റത്തിനും, നിന്‍റെ കുറ്റത്തിനും നീ അര്‍ഹനായിത്തീരുവാനും, അങ്ങനെ നീ നരകാവകാശികളുടെ കൂട്ടത്തിലാകുവാനുമാണ് ഞാന്‍ ആഗ്രഹിക്കുന്നത്‌. അതാണ് അക്രമികള്‍ക്കുള്ള പ്രതിഫലം.
\end{malayalam}}
\flushright{\begin{Arabic}
\quranayah[5][30]
\end{Arabic}}
\flushleft{\begin{malayalam}
എന്നിട്ട് തന്‍റെ സഹോദരനെ കൊല്ലുവാന്‍ അവന്‍റെ മനസ്സ് അവന്ന് പ്രേരണ നല്‍കി. അങ്ങനെ അവനെ കൊലപ്പെടുത്തി. അതിനാല്‍ അവന്‍ നഷ്ടക്കാരില്‍പെട്ടവനായിത്തീര്‍ന്നു.
\end{malayalam}}
\flushright{\begin{Arabic}
\quranayah[5][31]
\end{Arabic}}
\flushleft{\begin{malayalam}
അപ്പോള്‍ തന്‍റെ സഹോദരന്‍റെ മൃതദേഹം മറവു ചെയ്യേണ്ടത് എങ്ങനെയെന്ന് അവന്ന് കാണിച്ചുകൊടുക്കുവാനായി നിലത്ത് മാന്തികുഴിയുണ്ടാക്കുന്ന ഒരു കാക്കയെ അല്ലാഹു അയച്ചു. അവന്‍ പറഞ്ഞു: എന്തൊരു കഷ്ടം! എന്‍റെ സഹോദരന്‍റെ മൃതദേഹം മറവുചെയ്യുന്ന കാര്യത്തില്‍ ഈ കാക്കയെപ്പോലെ ആകാന്‍ പോലും എനിക്ക് കഴിയാതെ പോയല്ലോ. അങ്ങനെ അവന്‍ ഖേദക്കാരുടെ കൂട്ടത്തിലായിത്തീര്‍ന്നു.
\end{malayalam}}
\flushright{\begin{Arabic}
\quranayah[5][32]
\end{Arabic}}
\flushleft{\begin{malayalam}
അക്കാരണത്താല്‍ ഇസ്രായീല്‍ സന്തതികള്‍ക്ക് നാം ഇപ്രകാരം വിധിനല്‍കുകയുണ്ടായി: മറ്റൊരാളെ കൊന്നതിന് പകരമായോ, ഭൂമിയില്‍ കുഴപ്പമുണ്ടാക്കിയതിന്‍റെ പേരിലോ അല്ലാതെ വല്ലവനും ഒരാളെ കൊലപ്പെടുത്തിയാല്‍, അത് മനുഷ്യരെ മുഴുവന്‍ കൊലപ്പെടുത്തിയതിന് തുല്യമാകുന്നു. ഒരാളുടെ ജീവന്‍ വല്ലവനും രക്ഷിച്ചാല്‍, അത് മനുഷ്യരുടെ മുഴുവന്‍ ജീവന്‍ രക്ഷിച്ചതിന് തുല്യമാകുന്നു. നമ്മുടെ ദൂതന്‍മാര്‍ വ്യക്തമായ തെളിവുകളുമായി അവരുടെ (ഇസ്രായീല്യരുടെ) അടുത്ത് ചെന്നിട്ടുണ്ട്‌. എന്നിട്ട് അതിനു ശേഷം അവരില്‍ ധാരാളം പേര്‍ ഭൂമിയില്‍ അതിക്രമം പ്രവര്‍ത്തിച്ചുകൊണ്ടിരിക്കുകയാണ്‌.
\end{malayalam}}
\flushright{\begin{Arabic}
\quranayah[5][33]
\end{Arabic}}
\flushleft{\begin{malayalam}
അല്ലാഹുവോടും അവന്‍റെ ദൂതനോടും പോരാടുകയും, ഭൂമിയില്‍ കുഴപ്പമുണ്ടാക്കാന്‍ ശ്രമിക്കുകയും ചെയ്യുന്നവര്‍ക്കുള്ള പ്രതിഫലം അവര്‍ കൊന്നൊടുക്കപ്പെടുകയോ, ക്രൂശിക്കപ്പെടുകയോ, അവരുടെ കൈകളും കാലുകളും എതിര്‍വശങ്ങളില്‍ നിന്നായി മുറിച്ചുകളയപ്പെടുകയോ, നാടുകടത്തപ്പെടുകയോ ചെയ്യുക മാത്രമാകുന്നു. അതവര്‍ക്ക് ഇഹലോകത്തുള്ള അപമാനമാകുന്നു. പരലോകത്ത് അവര്‍ക്ക് കനത്ത ശിക്ഷയുമുണ്ടായിരിക്കും.
\end{malayalam}}
\flushright{\begin{Arabic}
\quranayah[5][34]
\end{Arabic}}
\flushleft{\begin{malayalam}
എന്നാല്‍, അവര്‍ക്കെതിരില്‍ നടപടിയെടുക്കാന്‍ നിങ്ങള്‍ക്ക് കഴിയുന്നതിന്‍റെ മുമ്പായി പശ്ചാത്തപിച്ച് മടങ്ങിയവര്‍ ഇതില്‍ നിന്നൊഴിവാണ്‌. അല്ലാഹു ഏറെ പൊറുക്കുന്നവനും കരുണാനിധിയുമാണെന്ന് നിങ്ങള്‍ മനസ്സിലാക്കുക.
\end{malayalam}}
\flushright{\begin{Arabic}
\quranayah[5][35]
\end{Arabic}}
\flushleft{\begin{malayalam}
സത്യവിശ്വാസികളേ, നിങ്ങള്‍ അല്ലാഹുവെ സൂക്ഷിക്കുകയും അവനിലേക്ക് അടുക്കുവാനുള്ള മാര്‍ഗം തേടുകയും, അവന്‍റെ മാര്‍ഗത്തില്‍ സമരത്തില്‍ ഏര്‍പെടുകയും ചെയ്യുക. നിങ്ങള്‍ക്ക് (അത് വഴി) വിജയം പ്രാപിക്കാം.
\end{malayalam}}
\flushright{\begin{Arabic}
\quranayah[5][36]
\end{Arabic}}
\flushleft{\begin{malayalam}
ഉയിര്‍ത്തെഴുന്നേല്‍പിന്‍റെ നാളിലെ ശിക്ഷ ഒഴിവായിക്കിട്ടുവാന്‍ വേണ്ടി പ്രായശ്ചിത്തം നല്‍കുന്നതിനായി സത്യനിഷേധികളുടെ കൈവശം ഭൂമിയിലുള്ളത് മുഴുക്കെയും, അത്രതന്നെ വേറെയും ഉണ്ടായിരുന്നാല്‍ പോലും അവരില്‍ നിന്ന് അത് സ്വീകരിക്കപ്പെടുകയില്ല തന്നെ. അവര്‍ക്ക് വേദനയേറിയ ശിക്ഷയാണുള്ളത്‌.
\end{malayalam}}
\flushright{\begin{Arabic}
\quranayah[5][37]
\end{Arabic}}
\flushleft{\begin{malayalam}
നരകത്തില്‍ നിന്ന് പുറത്ത് കടക്കാന്‍ അവര്‍ ആഗ്രഹിക്കും. അതില്‍ നിന്ന് പുറത്തുപോകാന്‍ അവര്‍ക്ക് സാധ്യമാവുകയേയില്ല. നിരന്തരമായ ശിക്ഷയാണ് അവര്‍ക്കുള്ളത്‌.
\end{malayalam}}
\flushright{\begin{Arabic}
\quranayah[5][38]
\end{Arabic}}
\flushleft{\begin{malayalam}
മോഷ്ടിക്കുന്നവന്‍റെയും മോഷ്ടിക്കുന്നവളുടെയും കൈകള്‍ നിങ്ങള്‍ മുറിച്ചുകളയുക. അവര്‍ സമ്പാദിച്ചതിന്നുള്ള പ്രതിഫലവും, അല്ലാഹുവിങ്കല്‍ നിന്നുള്ള മാതൃകാപരമായ ശിക്ഷയുമാണത്‌. അല്ലാഹു പ്രതാപിയും യുക്തിമാനുമാകുന്നു.
\end{malayalam}}
\flushright{\begin{Arabic}
\quranayah[5][39]
\end{Arabic}}
\flushleft{\begin{malayalam}
എന്നാല്‍, അക്രമം ചെയ്ത് പോയതിനു ശേഷം വല്ലവനും പശ്ചാത്തപിക്കുകയും, നിലപാട് നന്നാക്കിത്തീര്‍ക്കുകയും ചെയ്താല്‍ തീര്‍ച്ചയായും അല്ലാഹു അവന്‍റെ പശ്ചാത്താപം സ്വീകരിക്കുന്നതാണ്‌. തീര്‍ച്ചയായും അല്ലാഹു ഏറെ പൊറുക്കുന്നവനും കരുണ കാണിക്കുന്നവനുമത്രെ.
\end{malayalam}}
\flushright{\begin{Arabic}
\quranayah[5][40]
\end{Arabic}}
\flushleft{\begin{malayalam}
ആകാശങ്ങളുടെയും ഭൂമിയുടെയും ആധിപത്യം അല്ലാഹുവിനാണെന്ന് നിനക്കറിഞ്ഞ് കൂടെ? അവന്‍ ഉദ്ദേശിക്കുന്നവരെ അവന്‍ ശിക്ഷിക്കുകയും, അവന്‍ ഉദ്ദേശിക്കുന്നവര്‍ക്ക് അവന്‍ പൊറുത്തുകൊടുക്കുകയും ചെയ്യുന്നു. അല്ലാഹു ഏതു കാര്യത്തിനും കഴിവുള്ളവനത്രെ.
\end{malayalam}}
\flushright{\begin{Arabic}
\quranayah[5][41]
\end{Arabic}}
\flushleft{\begin{malayalam}
ഓ, റസൂലേ; സത്യനിഷേധത്തിലേക്ക് കുതിച്ചുചെല്ലുന്നവര്‍ (അവരുടെ പ്രവൃത്തി) നിനക്ക് ദുഃഖമുണ്ടാക്കാതിരിക്കട്ടെ. അവര്‍ മനസ്സില്‍ വിശ്വാസം കടക്കാതെ ഞങ്ങള്‍ വിശ്വസിച്ചിരിക്കുന്നു. എന്ന് വായകൊണ്ട് പറയുന്നവരില്‍ പെട്ടവരാകട്ടെ, യഹൂദമതക്കാരില്‍ പെട്ടവരാകട്ടെ, കള്ളം ചെവിയോര്‍ത്ത് കേള്‍ക്കുന്നവരും, നിന്‍റെ അടുത്ത് വരാത്ത മറ്റു ആളുകളുടെ വാക്കുകള്‍ ചെവിയോര്‍ത്തുകേള്‍ക്കുന്നവരുമാണവര്‍. വേദവാക്യങ്ങളെ അവയുടെ സന്ദര്‍ഭങ്ങളില്‍ നിന്നു അവര്‍ മാറ്റിക്കളയുന്നു. അവര്‍ പറയും: ഇതേ വിധി തന്നെയാണ് (നബിയുടെ പക്കല്‍ നിന്ന്‌) നിങ്ങള്‍ക്ക് നല്‍കപ്പെടുന്നതെങ്കില്‍ അത് സ്വീകരിക്കുക. അതല്ല നല്‍കപ്പെടുന്നതെങ്കില്‍ നിങ്ങള്‍ സൂക്ഷിച്ച് കൊള്ളുക; വല്ലവന്നും നാശം വരുത്താന്‍ അല്ലാഹു ഉദ്ദേശിക്കുന്ന പക്ഷം അവന്നു വേണ്ടി അല്ലാഹുവില്‍ നിന്ന് യാതൊന്നും നേടിയെടുക്കാന്‍ നിനക്ക് സാധിക്കുന്നതല്ല. അത്തരക്കാരുടെ മനസ്സുകളെ ശുദ്ധീകരിക്കുവാന്‍ അല്ലാഹു ഉദ്ദേശിച്ചിട്ടില്ല. അവര്‍ക്ക് ഇഹലോകത്ത് അപമാനമാണുള്ളത്‌. പരലോകത്ത് അവര്‍ക്ക് കനത്ത ശിക്ഷയുണ്ടായിരിക്കുകയും ചെയ്യും.
\end{malayalam}}
\flushright{\begin{Arabic}
\quranayah[5][42]
\end{Arabic}}
\flushleft{\begin{malayalam}
കള്ളം ചെവിയോര്‍ത്ത് കേള്‍ക്കുന്നവരും, നിഷിദ്ധമായ സമ്പാദ്യം ധാരാളം തിന്നുന്നവരുമത്രെ അവര്‍. അവര്‍ നിന്‍റെ അടുത്ത് വരുകയാണെങ്കില്‍ അവര്‍ക്കിടയില്‍ നീ തീര്‍പ്പുകല്‍പിക്കുകയോ, അവരെ അവഗണിച്ച് കളയുകയോ ചെയ്യുക. നീ അവരെ അവഗണിച്ച് കളയുന്ന പക്ഷം അവര്‍ നിനക്ക് ഒരു ദോഷവും വരുത്തുകയില്ല. എന്നാല്‍ നീ തീര്‍പ്പുകല്‍പിക്കുകയാണെങ്കില്‍ അവര്‍ക്കിടയില്‍ നീതിപൂര്‍വ്വം തീര്‍പ്പുകല്‍പിക്കുക. നീതിപാലിക്കുന്നവരെ തീര്‍ച്ചയായും അല്ലാഹു സ്നേഹിക്കുന്നു.
\end{malayalam}}
\flushright{\begin{Arabic}
\quranayah[5][43]
\end{Arabic}}
\flushleft{\begin{malayalam}
എന്നാല്‍ അവര്‍ എങ്ങനെയാണ് നിന്നെ വിധികര്‍ത്താവാക്കുന്നത്‌? അവരുടെ പക്കല്‍ തൌറാത്തുണ്ട്‌. അതിലാകട്ടെ അല്ലാഹുവിന്‍റെ വിധിവിലക്കുകളുണ്ട്‌. എന്നിട്ടതിന് ശേഷവും അവര്‍ പിന്തിരിഞ്ഞ് കളയുകയാണ്‌. യഥാര്‍ത്ഥത്തില്‍ അവര്‍ വിശ്വാസികളേ അല്ല.
\end{malayalam}}
\flushright{\begin{Arabic}
\quranayah[5][44]
\end{Arabic}}
\flushleft{\begin{malayalam}
തീര്‍ച്ചയായും നാം തന്നെയാണ് തൌറാത്ത് അവതരിപ്പിച്ചിരിക്കുന്നത്‌. അതില്‍ മാര്‍ഗദര്‍ശനവും പ്രകാശവുമുണ്ട്‌. (അല്ലാഹുവിന്‌) കീഴ്പെട്ട പ്രവാചകന്‍മാര്‍ യഹൂദമതക്കാര്‍ക്ക് അതിനനുസരിച്ച് വിധികല്‍പിച്ച് പോന്നു. പുണ്യവാന്‍മാരും പണ്ഡിതന്‍മാരും (അതേ പ്രകാരം തന്നെ വിധികല്‍പിച്ചിരുന്നു.) കാരണം അല്ലാഹുവിന്‍റെ ഗ്രന്ഥത്തിന്‍റെ സംരക്ഷണം അവര്‍ക്ക് ഏല്‍പിക്കപ്പെട്ടിരുന്നു. അവരതിന് സാക്ഷികളുമായിരുന്നു. അതിനാല്‍ നിങ്ങള്‍ ജനങ്ങളെ പേടിക്കാതെ എന്നെ മാത്രം ഭയപ്പെടുക. എന്‍റെ വചനങ്ങള്‍ നിങ്ങള്‍ തുച്ഛമായ വിലയ്ക്ക് വിറ്റുകളയാതിരിക്കുക. അല്ലാഹു അവതരിപ്പിച്ച് തന്നതനുസരിച്ച് ആര്‍ വിധിക്കുന്നില്ലയോ അവര്‍ തന്നെയാകുന്നു അവിശ്വാസികള്‍ .
\end{malayalam}}
\flushright{\begin{Arabic}
\quranayah[5][45]
\end{Arabic}}
\flushleft{\begin{malayalam}
ജീവന് ജീവന്‍, കണ്ണിന് കണ്ണ്‌, മൂക്കിന് മൂക്ക്‌, ചെവിക്ക് ചെവി, പല്ലിന് പല്ല്‌, മുറിവുകള്‍ക്ക് തത്തുല്യമായ പ്രതിക്രിയ എന്നിങ്ങിനെയാണ് അതില്‍ (തൌറാത്തില്‍) നാം അവര്‍ക്ക് നിയമമായി വെച്ചിട്ടുള്ളത്‌. വല്ലവനും (പ്രതിക്രിയ ചെയ്യാതെ) മാപ്പുനല്‍കുന്ന പക്ഷം അത് അവന്ന് പാപമോചന (ത്തിന് ഉതകുന്ന ഒരു പുണ്യകര്‍മ്മ) മാകുന്നു. ആര്‍ അല്ലാഹു അവതരിപ്പിച്ചതനുസരിച്ച് വിധിക്കുന്നില്ലയോ അവര്‍ തന്നെയാണ് അക്രമികള്‍.
\end{malayalam}}
\flushright{\begin{Arabic}
\quranayah[5][46]
\end{Arabic}}
\flushleft{\begin{malayalam}
അവരെ (ആ പ്രവാചകന്‍മാരെ) ത്തുടര്‍ന്ന് അവരുടെ കാല്‍പാടുകളിലായിക്കൊണ്ട് മര്‍യമിന്‍റെ മകന്‍ ഈസായെ തന്‍റെ മുമ്പിലുള്ള തൌറാത്തിനെ ശരിവെക്കുന്നവനായിക്കൊണ്ട് നാം നിയോഗിച്ചു. സന്‍മാര്‍ഗനിര്‍ദേശവും, സത്യപ്രകാശവും അടങ്ങിയ ഇന്‍ജീലും അദ്ദേഹത്തിന് നാം നല്‍കി. അതിന്‍റെ മുമ്പിലുള്ള തൌറാത്തിനെ ശരിവെക്കുന്നതും, സൂക്ഷ്മത പാലിക്കുന്നവര്‍ക്ക് സദുപദേശവുമത്രെ അത്‌.
\end{malayalam}}
\flushright{\begin{Arabic}
\quranayah[5][47]
\end{Arabic}}
\flushleft{\begin{malayalam}
ഇന്‍ജീലിന്‍റെ അനുയായികള്‍, അല്ലാഹു അവതരിപ്പിച്ചതനുസരിച്ച് വിധികല്‍പിക്കട്ടെ. അല്ലാഹു അവതരിപ്പിച്ചതനുസരിച്ച് ആര്‍ വിധിക്കുന്നില്ലയോ അവര്‍ തന്നെയാകുന്നു ധിക്കാരികള്‍.
\end{malayalam}}
\flushright{\begin{Arabic}
\quranayah[5][48]
\end{Arabic}}
\flushleft{\begin{malayalam}
(നബിയേ,) നിനക്കിതാ സത്യപ്രകാരം വേദഗ്രന്ഥം അവതരിപ്പിച്ച് തന്നിരിക്കുന്നു. അതിന്‍റെ മുമ്പിലുള്ള വേദഗ്രന്ഥങ്ങളെ ശരിവെക്കുന്നതും അവയെ കാത്തുരക്ഷിക്കുന്നതുമത്രെ അത്‌. അതിനാല്‍ നീ അവര്‍ക്കിടയില്‍ നാം അവതരിപ്പിച്ച് തന്നതനുസരിച്ച് വിധികല്‍പിക്കുക. നിനക്ക് വന്നുകിട്ടിയ സത്യത്തെ വിട്ട് നീ അവരുടെ തന്നിഷ്ടങ്ങളെ പിന്‍പറ്റിപോകരുത്‌. നിങ്ങളില്‍ ഓരോ വിഭാഗത്തിനും ഓരോ നിയമക്രമവും കര്‍മ്മമാര്‍ഗവും നാം നിശ്ചയിച്ച് തന്നിരിക്കുന്നു. അല്ലാഹു ഉദ്ദേശിച്ചിരുന്നെങ്കില്‍ നിങ്ങളെ അവന്‍ ഒരൊറ്റ സമുദായമാക്കുമായിരുന്നു. പക്ഷെ നിങ്ങള്‍ക്കവന്‍ നല്‍കിയിട്ടുള്ളതില്‍ നിങ്ങളെ പരീക്ഷിക്കുവാന്‍ (അവന്‍ ഉദ്ദേശിക്കുന്നു.) അതിനാല്‍ നല്ല കാര്യങ്ങളിലേക്ക് നിങ്ങള്‍ മത്സരിച്ച് മുന്നേറുക. അല്ലാഹുവിങ്കലേക്കത്രെ നിങ്ങളുടെയെല്ലാം മടക്കം. നിങ്ങള്‍ ഭിന്നിച്ചിരുന്ന വിഷയങ്ങളെപ്പറ്റി അപ്പോളവന്‍ നിങ്ങള്‍ക്ക് അറിയിച്ച് തരുന്നതാണ്‌.
\end{malayalam}}
\flushright{\begin{Arabic}
\quranayah[5][49]
\end{Arabic}}
\flushleft{\begin{malayalam}
അല്ലാഹു അവതരിപ്പിച്ചതനുസരിച്ച് അവര്‍ക്കിടയില്‍ നീ വിധികല്‍പിക്കുകയും, അവരുടെ തന്നിഷ്ടങ്ങളെ പിന്‍പറ്റാതിരിക്കുകയും, അല്ലാഹു നിനക്ക് അവതരിപ്പിച്ച് തന്ന വല്ല നിര്‍ദേശത്തില്‍ നിന്നും അവര്‍ നിന്നെ തെറ്റിച്ചുകളയുന്നതിനെപ്പറ്റി നീ ജാഗ്രത പുലര്‍ത്തുകയും ചെയ്യണമെന്നും (നാം കല്‍പിക്കുന്നു.) ഇനി അവര്‍ പിന്തിരിഞ്ഞ് കളയുകയാണെങ്കില്‍ നീ മനസ്സിലാക്കണം; അവരുടെ ചില പാപങ്ങള്‍ കാരണമായി അവര്‍ക്ക് നാശം വരുത്തണമെന്നാണ് അല്ലാഹു ഉദ്ദേശിക്കുന്നതെന്ന്‌. തീര്‍ച്ചയായും മനുഷ്യരില്‍ അധികപേരും ധിക്കാരികളാകുന്നു.
\end{malayalam}}
\flushright{\begin{Arabic}
\quranayah[5][50]
\end{Arabic}}
\flushleft{\begin{malayalam}
ജാഹിലിയ്യത്തിന്‍റെ (അനിസ്ലാമിക മാര്‍ഗത്തിന്‍റെ) വിധിയാണോ അവര്‍ തേടുന്നത്‌? ദൃഢവിശ്വാസികളായ ജനങ്ങള്‍ക്ക് അല്ലാഹുവെക്കാള്‍ നല്ല വിധികര്‍ത്താവ് ആരാണുള്ളത്‌?
\end{malayalam}}
\flushright{\begin{Arabic}
\quranayah[5][51]
\end{Arabic}}
\flushleft{\begin{malayalam}
സത്യവിശ്വാസികളേ, യഹൂദരെയും ക്രൈസ്തവരേയും നിങ്ങള്‍ ഉറ്റമിത്രങ്ങളായി സ്വീകരിക്കരുത്‌. അവരാകട്ടെ, അന്യോന്യം ഉറ്റമിത്രങ്ങളാണ് താനും. നിങ്ങളില്‍ നിന്നാരെങ്കിലും അവരെ ഉറ്റമിത്രങ്ങളായി സ്വീകരിക്കുന്ന പക്ഷം അവനും അവരില്‍ പെട്ടവന്‍ തന്നെയാണ്‌. അക്രമികളായ ആളുകളെ അല്ലാഹു നേര്‍വഴിയിലാക്കുകയില്ല; തീര്‍ച്ച.
\end{malayalam}}
\flushright{\begin{Arabic}
\quranayah[5][52]
\end{Arabic}}
\flushleft{\begin{malayalam}
എന്നാല്‍, മനസ്സുകള്‍ക്ക് രോഗം ബാധിച്ച ചില ആളുകള്‍ അവരുടെ കാര്യത്തില്‍ (അവരുമായി സ്നേഹബന്ധം സ്ഥാപിക്കുന്നതില്‍) തിടുക്കം കൂട്ടുന്നതായി നിനക്ക് കാണാം. ഞങ്ങള്‍ക്ക് വല്ല ആപത്തും സംഭവിച്ചേക്കുമോ എന്ന് ഞങ്ങള്‍ ഭയപ്പെടുന്നു. എന്നായിരിക്കും അവര്‍ പറയുന്നത്‌. എന്നാല്‍ അല്ലാഹു (നിങ്ങള്‍ക്ക്‌) പൂര്‍ണ്ണവിജയം നല്‍കുകയോ, അല്ലെങ്കില്‍ അവന്‍റെ പക്കല്‍ നിന്ന് മറ്റുവല്ല നടപടിയും ഉണ്ടാകുകയോ ചെയ്തേക്കാം. അപ്പോള്‍ തങ്ങളുടെ മനസ്സുകളില്‍ രഹസ്യമാക്കിവെച്ചതിനെപ്പറ്റി ഈ കൂട്ടര്‍ ഖേദിക്കുന്നവരായിത്തീരും.
\end{malayalam}}
\flushright{\begin{Arabic}
\quranayah[5][53]
\end{Arabic}}
\flushleft{\begin{malayalam}
(അന്ന്‌) സത്യവിശ്വാസികള്‍ പറയും; ഞങ്ങള്‍ നിങ്ങളുടെ കൂടെത്തന്നെയാണ്‌, എന്ന് അല്ലാഹുവിന്‍റെ പേരില്‍ ബലമായി സത്യം ചെയ്ത് പറഞ്ഞിരുന്നവര്‍ ഇക്കൂട്ടര്‍ തന്നെയാണോ? എന്ന്‌. അവരുടെ കര്‍മ്മങ്ങള്‍ നിഷ്ഫലമാകുകയും, അങ്ങനെ അവര്‍ നഷ്ടക്കാരായി മാറുകയും ചെയ്തിരിക്കുന്നു.
\end{malayalam}}
\flushright{\begin{Arabic}
\quranayah[5][54]
\end{Arabic}}
\flushleft{\begin{malayalam}
സത്യവിശ്വാസികളേ, നിങ്ങളില്‍ ആരെങ്കിലും തന്‍റെ മതത്തില്‍ നിന്ന് പിന്തിരിഞ്ഞ് കളയുന്ന പക്ഷം അല്ലാഹു ഇഷ്ടപ്പെടുന്നവരും, അല്ലാഹുവെ ഇഷ്ടപ്പെടുന്നവരുമായ മറ്റൊരു ജനവിഭാഗത്തെ അല്ലാഹു പകരം കൊണ്ട് വരുന്നതാണ്‌. അവര്‍ വിശ്വാസികളോട് വിനയം കാണിക്കുന്നവരും, സത്യനിഷേധികളോട് പ്രതാപം പ്രകടിപ്പിക്കുന്നവരുമായിരിക്കും. അവര്‍ അല്ലാഹുവിന്‍റെ മാര്‍ഗത്തില്‍ സമരത്തില്‍ ഏര്‍പെടും. ഒരു ആക്ഷേപകന്‍റെ ആക്ഷേപവും അവര്‍ ഭയപ്പെടുകയില്ല. അത് അല്ലാഹുവിന്‍റെ അനുഗ്രഹമത്രെ. അവന്‍ ഉദ്ദേശിക്കുന്നവര്‍ക്ക് അത് നല്‍കുന്നു. അല്ലാഹു വിപുലമായ കഴിവുള്ളവനും സര്‍വ്വജ്ഞനുമത്രെ.
\end{malayalam}}
\flushright{\begin{Arabic}
\quranayah[5][55]
\end{Arabic}}
\flushleft{\begin{malayalam}
അല്ലാഹുവും അവന്‍റെ ദൂതനും, താഴ്മയുള്ളവരായിക്കൊണ്ട് നമസ്കാരം മുറപോലെ നിര്‍വഹിക്കുകയും സകാത്ത് നല്‍കുകയും ചെയ്യുന്ന സത്യവിശ്വാസികളും മാത്രമാകുന്നു നിങ്ങളുടെ ഉറ്റമിത്രങ്ങള്‍.
\end{malayalam}}
\flushright{\begin{Arabic}
\quranayah[5][56]
\end{Arabic}}
\flushleft{\begin{malayalam}
വല്ലവനും അല്ലാഹുവെയും അവന്‍റെ ദൂതനെയും, സത്യവിശ്വാസികളെയും മിത്രങ്ങളായി സ്വീകരിക്കുന്നുവെങ്കില്‍ തീര്‍ച്ചയായും അല്ലാഹുവിന്‍റെ കക്ഷി തന്നെയാണ് വിജയം നേടുന്നവര്‍.
\end{malayalam}}
\flushright{\begin{Arabic}
\quranayah[5][57]
\end{Arabic}}
\flushleft{\begin{malayalam}
സത്യവിശ്വാസികളേ, നിങ്ങള്‍ക്ക് മുമ്പ് വേദഗ്രന്ഥം നല്‍കപ്പെട്ടവരില്‍ നിന്ന് നിങ്ങളുടെ മതത്തെ തമാശയും വിനോദവിഷയവുമാക്കി തീര്‍ത്തവരെയും, സത്യനിഷേധികളെയും നിങ്ങള്‍ ഉറ്റമിത്രങ്ങളായി സ്വീകരിക്കരുത്‌. നിങ്ങള്‍ സത്യവിശ്വാസികളാണെങ്കില്‍ അല്ലാഹുവെ സൂക്ഷിക്കുവിന്‍.
\end{malayalam}}
\flushright{\begin{Arabic}
\quranayah[5][58]
\end{Arabic}}
\flushleft{\begin{malayalam}
നിങ്ങള്‍ നമസ്കാരത്തിന്നായി വിളിച്ചാല്‍, അവരതിനെ ഒരു തമാശയും വിനോദവിഷയവുമാക്കിത്തീര്‍ക്കുന്നു. അവര്‍ ചിന്തിച്ചുമനസ്സിലാക്കാത്ത ഒരു ജനവിഭാഗമായത് കൊണ്ടത്രെ അത്‌.
\end{malayalam}}
\flushright{\begin{Arabic}
\quranayah[5][59]
\end{Arabic}}
\flushleft{\begin{malayalam}
(നബിയേ,) പറയുക: വേദക്കാരേ, അല്ലാഹുവിലും (അവങ്കല്‍ നിന്ന്‌) ഞങ്ങള്‍ക്ക് അവതരിപ്പിക്കപ്പെട്ട വേദത്തിലും, മുമ്പ് അവതരിപ്പിക്കപ്പെട്ട വേദത്തിലും ഞങ്ങള്‍ വിശ്വസിച്ചിരിക്കുന്നു എന്നത് കൊണ്ടും, നിങ്ങളില്‍ അധികപേരും ധിക്കാരികളാണ് എന്നത് കൊണ്ടും മാത്രമല്ലേ നിങ്ങള്‍ ഞങ്ങളെ കുറ്റപ്പെടുത്തുന്നത്‌?
\end{malayalam}}
\flushright{\begin{Arabic}
\quranayah[5][60]
\end{Arabic}}
\flushleft{\begin{malayalam}
പറയുക: എന്നാല്‍ അല്ലാഹുവിന്‍റെ അടുക്കല്‍ അതിനെക്കാള്‍ മോശമായ പ്രതിഫലമുള്ളവരെ പറ്റി ഞാന്‍ നിങ്ങള്‍ക്ക് അറിയിച്ചുതരട്ടെയോ? ഏതൊരു വിഭാഗത്തെ അല്ലാഹു ശപിക്കുകയും അവരോടവന്‍ കോപിക്കുകയും ചെയ്തുവോ, ഏത് വിഭാഗത്തില്‍ പെട്ടവരെ അല്ലാഹു കുരങ്ങുകളും പന്നികളുമാക്കിത്തീര്‍ത്തുവോ, ഏതൊരു വിഭാഗം ദുര്‍മൂര്‍ത്തികളെ ആരാധിച്ചുവോ അവരത്രെ ഏറ്റവും മോശമായ സ്ഥാനമുള്ളവരും നേര്‍മാര്‍ഗത്തില്‍ നിന്ന് ഏറെ പിഴച്ച് പോയവരും.
\end{malayalam}}
\flushright{\begin{Arabic}
\quranayah[5][61]
\end{Arabic}}
\flushleft{\begin{malayalam}
നിങ്ങളുടെ അടുത്ത് വരുമ്പോള്‍ അവര്‍ പറയും, ഞങ്ങള്‍ വിശ്വസിച്ചിരിക്കുന്നു എന്ന്‌. വാസ്തവത്തില്‍ അവര്‍ അവിശ്വാസത്തോടെയാണ് കടന്നുവന്നിട്ടുള്ളത്‌. അവിശ്വാസത്തോട് കൂടിത്തന്നെയാണ് അവര്‍ പുറത്ത് പോയിട്ടുള്ളതും. അവര്‍ ഒളിച്ചുവെച്ചുകൊണ്ടിരിക്കുന്നതിനെപ്പറ്റി അല്ലാഹു നല്ലവണ്ണം അറിയുന്നവനാകുന്നു.
\end{malayalam}}
\flushright{\begin{Arabic}
\quranayah[5][62]
\end{Arabic}}
\flushleft{\begin{malayalam}
അവരിലധികം പേരും പാപകൃത്യങ്ങളിലും, അതിക്രമത്തിലും, നിഷിദ്ധസമ്പാദ്യം ഭുജിക്കുന്നതിലും മത്സരിച്ച് മുന്നേറുന്നതായി നിനക്ക് കാണാം. അവര്‍ പ്രവര്‍ത്തിച്ചു കൊണ്ടിരിക്കുന്നത് വളരെ ചീത്ത തന്നെ.
\end{malayalam}}
\flushright{\begin{Arabic}
\quranayah[5][63]
\end{Arabic}}
\flushleft{\begin{malayalam}
കുറ്റകരമായത് അവര്‍ പറയുന്നതില്‍ നിന്നും നിഷിദ്ധമായ സമ്പാദ്യം അവര്‍ തിന്നുതില്‍ നിന്നും പുണ്യപുരുഷന്‍മാരും പണ്ഡിതന്‍മാരും അവരെ തടയാതിരുന്നത് എന്ത്കൊണ്ടാണ്‌? അവര്‍ ചെയ്ത് കൊണ്ടിരിക്കുന്നത് വളരെ ചീത്ത തന്നെ.
\end{malayalam}}
\flushright{\begin{Arabic}
\quranayah[5][64]
\end{Arabic}}
\flushleft{\begin{malayalam}
അല്ലാഹുവിന്‍റെ കൈകള്‍ ബന്ധിക്കപ്പെട്ടിരിക്കുകയാണ് എന്ന് യഹൂദന്‍മാര്‍ പറഞ്ഞു അവരുടെ കൈകള്‍ ബന്ധിതമാകട്ടെ. അവര്‍ പറഞ്ഞ വാക്ക് കാരണം അവര്‍ ശപിക്കപ്പെട്ടിരിക്കുന്നു. അല്ല, അവന്‍റെ ഇരു കൈകളും നിവര്‍ത്തപ്പെട്ടവയാകുന്നു. അവന്‍ എങ്ങനെ ഉദ്ദേശിക്കുന്നുവോ അങ്ങനെ ചെലവഴിക്കുന്നു. നിനക്ക് നിന്‍റെ രക്ഷിതാവിങ്കല്‍ നിന്ന് അവതരിപ്പിക്കപ്പെട്ട സന്ദേശം അവരില്‍ അധികം പേര്‍ക്കും ധിക്കാരവും അവിശ്വാസവും വര്‍ദ്ധിപ്പിക്കുക തന്നെ ചെയ്യും. അവര്‍ക്കിടയില്‍ ഉയിര്‍ത്തെഴുന്നേല്‍പിന്‍റെ നാളുവരെ ശത്രുതയും വിദ്വേഷവും നാം ഇട്ടുകൊടുത്തിരിക്കുകയാണ്‌. അവര്‍ യുദ്ധത്തിന് തീ കൊളുത്തുമ്പോഴെല്ലാം അല്ലാഹു അത് കെടുത്തിക്കളയുന്നു. അവര്‍ നാട്ടില്‍ കുഴപ്പമുണ്ടാക്കുവാന്‍ വേണ്ടി ശ്രമിക്കുകയാണ്‌. കുഴപ്പക്കാരെ അല്ലാഹു ഇഷ്ടപ്പെടുന്നില്ല.
\end{malayalam}}
\flushright{\begin{Arabic}
\quranayah[5][65]
\end{Arabic}}
\flushleft{\begin{malayalam}
വേദക്കാര്‍ വിശ്വസിക്കുകയും സൂക്ഷ്മത പാലിക്കുകയും ചെയ്തിരുന്നുവെങ്കില്‍ അവരില്‍ നിന്ന് അവരുടെ തിന്‍മകള്‍ നാം മായ്ച്ചുകളയുകയും അനുഗ്രഹപൂര്‍ണ്ണമായ സ്വര്‍ഗത്തോപ്പുകളില്‍ നാം അവരെ പ്രവേശിപ്പിക്കുകയും ചെയ്യുമായിരുന്നു.
\end{malayalam}}
\flushright{\begin{Arabic}
\quranayah[5][66]
\end{Arabic}}
\flushleft{\begin{malayalam}
തൌറാത്തും, ഇന്‍ജീലും, അവര്‍ക്ക് അവരുടെ രക്ഷിതാവിങ്കല്‍ നിന്ന് അവതരിപ്പിക്കപ്പെട്ട സന്ദേശങ്ങളും അവര്‍ നേരാംവണ്ണം നിലനിര്‍ത്തിയിരുന്നെങ്കില്‍ തങ്ങളുടെ മുകള്‍ഭാഗത്ത് നിന്നും, കാലുകള്‍ക്ക് ചുവട്ടില്‍ നിന്നും അവര്‍ക്ക് ആഹാരം ലഭിക്കുമായിരുന്നു. അവരില്‍ തന്നെ മിതത്വം പാലിക്കുന്ന ഒരു സമൂഹമുണ്ട്‌. എന്നാല്‍ അവരില്‍ അധികം പേരുടെയും പ്രവര്‍ത്തനങ്ങള്‍ വളരെ ചീത്ത തന്നെ.
\end{malayalam}}
\flushright{\begin{Arabic}
\quranayah[5][67]
\end{Arabic}}
\flushleft{\begin{malayalam}
ഹേ; റസൂലേ, നിന്‍റെ രക്ഷിതാവിങ്കല്‍ നിന്ന് നിനക്ക് അവതരിപ്പിക്കപ്പെട്ടത് നീ (ജനങ്ങള്‍ക്ക്‌) എത്തിച്ചുകൊടുക്കുക. അങ്ങനെ ചെയ്യാത്ത പക്ഷം നീ അവന്‍റെ ദൌത്യം നിറവേറ്റിയിട്ടില്ല. ജനങ്ങളില്‍ നിന്ന് അല്ലാഹു നിന്നെ രക്ഷിക്കുന്നതാണ്‌. സത്യനിഷേധികളായ ആളുകളെ തീര്‍ച്ചയായും അല്ലാഹു നേര്‍വഴിയിലാക്കുകയില്ല.
\end{malayalam}}
\flushright{\begin{Arabic}
\quranayah[5][68]
\end{Arabic}}
\flushleft{\begin{malayalam}
പറയുക: വേദക്കാരേ, തൌറാത്തും ഇന്‍ജീലും നിങ്ങള്‍ക്ക് നിങ്ങളുടെ രക്ഷിതാവിങ്കല്‍ നിന്ന് അവതരിപ്പിക്കപ്പെട്ട സന്ദേശങ്ങളും നിങ്ങള്‍ (നേരാംവണ്ണം) നിലനിര്‍ത്തുന്നത് വരെ നിങ്ങള്‍ യാതൊരു അടിസ്ഥാനത്തിലുമല്ല. എന്നാല്‍ നിനക്ക് നിന്‍റെ രക്ഷിതാവിങ്കല്‍ നിന്ന് അവതരിപ്പിക്കപ്പെട്ട സന്ദേശം അവരില്‍ അധികപേര്‍ക്കും ധിക്കാരവും അവിശ്വാസവും വര്‍ദ്ധിപ്പിക്കുക തന്നെ ചെയ്യും. അതിനാല്‍ സത്യനിഷേധികളായ ജനങ്ങളെപ്പറ്റി നീ ദുഃഖിക്കേണ്ടതില്ല.
\end{malayalam}}
\flushright{\begin{Arabic}
\quranayah[5][69]
\end{Arabic}}
\flushleft{\begin{malayalam}
സത്യവിശ്വാസികളോ, യഹൂദരോ, സാബികളോ, ക്രൈസ്തവരോ ആരാകട്ടെ, അവരില്‍ നിന്ന് അല്ലാഹുവിലും അന്ത്യദിനത്തിലും വിശ്വസിക്കുകയും, സല്‍കര്‍മ്മങ്ങള്‍ പ്രവര്‍ത്തിക്കുകയും ചെയ്തവര്‍ക്ക് യാതൊന്നും ഭയപ്പെടേണ്ടതില്ല. അവര്‍ ദുഃഖിക്കേണ്ടി വരികയുമില്ല.
\end{malayalam}}
\flushright{\begin{Arabic}
\quranayah[5][70]
\end{Arabic}}
\flushleft{\begin{malayalam}
ഇസ്രായീല്‍ സന്തതികളോട് നാം കരാര്‍ വാങ്ങുകയും, അവരിലേക്ക് നാം ദൂതന്‍മാരെ അയക്കുകയും ചെയ്തിട്ടുണ്ട്‌. അവരുടെ മനസ്സിന് പിടിക്കാത്ത കാര്യങ്ങളുമായി അവരുടെ അടുത്ത് ഏതെങ്കിലുമൊരു ദൂതന്‍ ചെന്നപ്പോളൊക്കെ ദൂതന്‍മാരില്‍ ഒരു വിഭാഗത്തെ അവര്‍ നിഷേധിച്ച് തള്ളുകയും, മറ്റൊരു വിഭാഗത്തെ അവര്‍ കൊലപ്പെടുത്തുകയുമാണ് ചെയ്തത്‌.
\end{malayalam}}
\flushright{\begin{Arabic}
\quranayah[5][71]
\end{Arabic}}
\flushleft{\begin{malayalam}
ഒരു കുഴപ്പവുമുണ്ടാകുകയില്ലെന്ന് അവര്‍ കണക്ക് കൂട്ടുകയും, അങ്ങനെ അവര്‍ അന്ധരും ബധിരരുമായികഴിയുകയും ചെയ്തു. പിന്നീട് അല്ലാഹു അവരുടെ പശ്ചാത്താപം സ്വീകരിച്ചു. വീണ്ടും അവരില്‍ അധികപേരും അന്ധരും ബധിരരുമായിക്കഴിഞ്ഞു. എന്നാല്‍ അല്ലാഹു അവര്‍ പ്രവര്‍ത്തിക്കുന്നതെല്ലാം കണ്ടറിയുന്നവനാകുന്നു.
\end{malayalam}}
\flushright{\begin{Arabic}
\quranayah[5][72]
\end{Arabic}}
\flushleft{\begin{malayalam}
മര്‍യമിന്‍റെ മകന്‍ മസീഹ് തന്നെയാണ് അല്ലാഹു എന്ന് പറഞ്ഞവര്‍ തീര്‍ച്ചയായും അവിശ്വാസികളായിരിക്കുന്നു. എന്നാല്‍ മസീഹ് പറഞ്ഞത്‌; ഇസ്രായീല്‍ സന്തതികളേ, എന്‍റെയും നിങ്ങളുടെയും രക്ഷിതാവായ അല്ലാഹുവെ നിങ്ങള്‍ ആരാധിക്കുവിന്‍. അല്ലാഹുവോട് വല്ലവനും പങ്കുചേര്‍ക്കുന്ന പക്ഷം തീര്‍ച്ചയായും അല്ലാഹു അവന്ന് സ്വര്‍ഗം നിഷിദ്ധമാക്കുന്നതാണ്‌. നരകം അവന്‍റെ വാസസ്ഥലമായിരിക്കുകയും ചെയ്യും. അക്രമികള്‍ക്ക് സഹായികളായി ആരും തന്നെയില്ല. എന്നാണ്‌.
\end{malayalam}}
\flushright{\begin{Arabic}
\quranayah[5][73]
\end{Arabic}}
\flushleft{\begin{malayalam}
അല്ലാഹു മൂവരില്‍ ഒരാളാണ് എന്ന് പറഞ്ഞവര്‍ തീര്‍ച്ചയായും അവിശ്വാസികളാണ്‌. ഏക ആരാധ്യനല്ലാതെ യാതൊരു ആരാധ്യനും ഇല്ല തന്നെ. അവര്‍ ആ പറയുന്നതില്‍ നിന്ന് വിരമിച്ചില്ലെങ്കില്‍ അവരില്‍ നിന്ന് അവിശ്വസിച്ചവര്‍ക്ക് വേദനയേറിയ ശിക്ഷ ബാധിക്കുക തന്നെ ചെയ്യും.
\end{malayalam}}
\flushright{\begin{Arabic}
\quranayah[5][74]
\end{Arabic}}
\flushleft{\begin{malayalam}
ആകയാല്‍ അവര്‍ അല്ലാഹുവിലേക്ക് ഖേദിച്ചുമടങ്ങുകയും, അവനോട് പാപമോചനം തേടുകയും ചെയ്യുന്നില്ലേ? അല്ലാഹു ഏറെ പൊറുക്കുന്നവനും കരുണാനിധിയുമത്രെ.
\end{malayalam}}
\flushright{\begin{Arabic}
\quranayah[5][75]
\end{Arabic}}
\flushleft{\begin{malayalam}
മര്‍യമിന്‍റെ മകന്‍ മസീഹ് ഒരു ദൈവദൂതന്‍ മാത്രമാകുന്നു. അദ്ദേഹത്തിന് മുമ്പ് ദൂതന്‍മാര്‍ കഴിഞ്ഞുപോയിട്ടുണ്ട്‌. അദ്ദേഹത്തിന്‍റെ മാതാവ് സത്യവതിയുമാകുന്നു. അവര്‍ ഇരുവരും ഭക്ഷണംകഴിക്കുന്നവരായിരുന്നു. നോക്കൂ; എന്നിട്ടും അവര്‍ എങ്ങനെയാണ് (സത്യത്തില്‍ നിന്ന്‌) തെറ്റിക്കപ്പെടുന്നതെന്ന്‌.
\end{malayalam}}
\flushright{\begin{Arabic}
\quranayah[5][76]
\end{Arabic}}
\flushleft{\begin{malayalam}
(നബിയേ,) പറയുക: അല്ലാഹുവെ കൂടാതെ നിങ്ങള്‍ക്ക് ഉപകാരമോ ഉപദ്രവമോ ചെയ്യാന്‍ കഴിയാത്ത വസ്തുക്കളെയാണോ നിങ്ങള്‍ ആരാധിക്കുന്നത്‌? അല്ലാഹുവാകട്ടെ എല്ലാം കേള്‍ക്കുന്നവനും അറിയുന്നവനുമാകുന്നു.
\end{malayalam}}
\flushright{\begin{Arabic}
\quranayah[5][77]
\end{Arabic}}
\flushleft{\begin{malayalam}
പറയുക: വേദക്കാരേ, സത്യത്തിനെതിരായിക്കൊണ്ട് നിങ്ങളുടെ മതകാര്യത്തില്‍ നിങ്ങള്‍ അതിരുകവിയരുത്‌. മുമ്പേപിഴച്ച് പോകുകയും, ധാരാളം പേരെ വഴിപിഴപ്പിക്കുകയും നേര്‍മാര്‍ഗത്തില്‍ നിന്ന് തെറ്റിപ്പോകുകയും ചെയ്ത ഒരു ജനവിഭാഗത്തിന്‍റെ തന്നിഷ്ടങ്ങളെ നിങ്ങള്‍ പിന്‍പറ്റുകയും ചെയ്യരുത്‌.
\end{malayalam}}
\flushright{\begin{Arabic}
\quranayah[5][78]
\end{Arabic}}
\flushleft{\begin{malayalam}
ഇസ്രായീല്‍ സന്തതികളിലെ സത്യനിഷേധികള്‍ ദാവൂദിന്‍റെയും, മര്‍യമിന്‍റെ മകന്‍ ഈസായുടെയും നാവിലൂടെ ശപിക്കപ്പെട്ടിരിക്കുന്നു. അവര്‍ അനുസരണക്കേട് കാണിക്കുകയും, അതിക്രമം കൈക്കൊള്ളുകയും ചെയ്തതിന്‍റെ ഫലമത്രെ അത്‌.
\end{malayalam}}
\flushright{\begin{Arabic}
\quranayah[5][79]
\end{Arabic}}
\flushleft{\begin{malayalam}
അവര്‍ ചെയ്തിരുന്ന ദുരാചാരത്തെ അവര്‍ അന്യോന്യം തടയുമായിരുന്നില്ല.അവര്‍ ചെയ്ത് കൊണ്ടിരുന്നത് വളരെ ചീത്ത തന്നെ.
\end{malayalam}}
\flushright{\begin{Arabic}
\quranayah[5][80]
\end{Arabic}}
\flushleft{\begin{malayalam}
അവരിലധികപേരും സത്യനിഷേധികളെ ഉറ്റമിത്രങ്ങളായി സ്വീകരിക്കുന്നത് നിനക്ക് കാണാം. സ്വന്തത്തിനു വേണ്ടി അവര്‍ മുന്‍കൂട്ടി ഒരുക്കിവെച്ചിട്ടുള്ളത് വളരെ ചീത്ത തന്നെ. (അതായത്‌) അല്ലാഹു അവരുടെ നേരെ കോപിച്ചിരിക്കുന്നു എന്നത്‌. ശിക്ഷയില്‍ അവര്‍ നിത്യവാസികളായിരിക്കുന്നതുമാണ്‌.
\end{malayalam}}
\flushright{\begin{Arabic}
\quranayah[5][81]
\end{Arabic}}
\flushleft{\begin{malayalam}
അവര്‍ അല്ലാഹുവിലും പ്രവാചകനിലും, അദ്ദേഹത്തിന് അവതരിപ്പിക്കപ്പെട്ടതിലും വിശ്വസിച്ചിരുന്നുവെങ്കില്‍ അവരെ (അവിശ്വാസികളെ) ഉറ്റമിത്രങ്ങളായി സ്വീകരിക്കുമായിരുന്നില്ല. പക്ഷെ, അവരില്‍ അധികപേരും ധിക്കാരികളാകുന്നു.
\end{malayalam}}
\flushright{\begin{Arabic}
\quranayah[5][82]
\end{Arabic}}
\flushleft{\begin{malayalam}
ജനങ്ങളില്‍ സത്യവിശ്വാസികളോട് ഏറ്റവും കടുത്ത ശത്രുതയുള്ളവര്‍ യഹൂദരും, ബഹുദൈവാരാധകരുമാണ് എന്ന് തീര്‍ച്ചയായും നിനക്ക് കാണാം. ഞങ്ങള്‍ ക്രിസ്ത്യാനികളാകുന്നു. എന്ന് പറഞ്ഞവരാണ് ജനങ്ങളില്‍ വെച്ച് സത്യവിശ്വാസികളോട് ഏറ്റവും അടുത്ത സൌഹൃദമുള്ളവര്‍ എന്നും നിനക്ക് കാണാം. അവരില്‍ മതപണ്ഡിതന്‍മാരും സന്യാസികളും ഉണ്ടെന്നതും, അവര്‍ അഹംഭാവം നടിക്കുന്നില്ല എന്നതുമാണതിന് കാരണം.
\end{malayalam}}
\flushright{\begin{Arabic}
\quranayah[5][83]
\end{Arabic}}
\flushleft{\begin{malayalam}
റസൂലിന് അവതരിപ്പിക്കപ്പെട്ടത് അവര്‍ കേട്ടാല്‍ സത്യം മനസ്സിലാക്കിയതിന്‍റെ ഫലമായി അവരുടെ കണ്ണുകളില്‍ നിന്ന് കണ്ണുനീര്‍ ഒഴുകുന്നതായി നിനക്ക് കാണാം. അവര്‍ പറയും: ഞങ്ങളുടെ രക്ഷിതാവേ! ഞങ്ങള്‍ വിശ്വസിച്ചിരിക്കുന്നു. അതിനാല്‍ സത്യസാക്ഷികളോടൊപ്പം ഞങ്ങളെയും നീ രേഖപ്പെടുത്തേണമേ.
\end{malayalam}}
\flushright{\begin{Arabic}
\quranayah[5][84]
\end{Arabic}}
\flushleft{\begin{malayalam}
ഞങ്ങളുടെ രക്ഷിതാവ് സജ്ജനങ്ങളോടൊപ്പം ഞങ്ങളെ പ്രവേശിപ്പിക്കുവാന്‍ ഞങ്ങള്‍ മോഹിച്ച് കൊണ്ടിരിക്കെ, ഞങ്ങള്‍ക്കെങ്ങനെ അല്ലാഹുവിലും ഞങ്ങള്‍ക്ക് വന്നുകിട്ടിയ സത്യത്തിലും വിശ്വസിക്കാതിരിക്കാന്‍ കഴിയും?
\end{malayalam}}
\flushright{\begin{Arabic}
\quranayah[5][85]
\end{Arabic}}
\flushleft{\begin{malayalam}
അങ്ങനെ അവരീ പറഞ്ഞത് നിമിത്തം അല്ലാഹു അവര്‍ക്ക് താഴ്ഭാഗത്ത് കൂടി അരുവികള്‍ ഒഴുകുന്ന സ്വര്‍ഗത്തോപ്പുകള്‍ പ്രതിഫലമായി നല്‍കി. അവരതില്‍ നിത്യവാസികളായിരിക്കും. സദ്‌വൃത്തര്‍ക്കുള്ള പ്രതിഫലമത്രെ അത്‌.
\end{malayalam}}
\flushright{\begin{Arabic}
\quranayah[5][86]
\end{Arabic}}
\flushleft{\begin{malayalam}
അവിശ്വസിക്കുകയും, നമ്മുടെ തെളിവുകളെ തള്ളിക്കളയുകയും ചെയ്തവരാരോ അവരാകുന്നു നരകാവകാശികള്‍.
\end{malayalam}}
\flushright{\begin{Arabic}
\quranayah[5][87]
\end{Arabic}}
\flushleft{\begin{malayalam}
സത്യവിശ്വാസികളേ, അല്ലാഹു നിങ്ങള്‍ക്ക് അനുവദിച്ച് തന്ന വിശിഷ്ടവസ്തുക്കളെ നിങ്ങള്‍ നിഷിദ്ധമാക്കരുത്‌. നിങ്ങള്‍ പരിധി ലംഘിക്കുകയും ചെയ്യരുത്‌. പരിധി ലംഘിക്കുന്നവരെ അല്ലാഹു ഒട്ടും ഇഷ്ടപ്പെടുകയില്ല.
\end{malayalam}}
\flushright{\begin{Arabic}
\quranayah[5][88]
\end{Arabic}}
\flushleft{\begin{malayalam}
അല്ലാഹു നിങ്ങള്‍ക്ക് നല്‍കിയതില്‍ നിന്ന് അനുവദനീയവും വിശിഷ്ടവും ആയത് നിങ്ങള്‍ തിന്നുകൊള്ളുക. ഏതൊരുവനിലാണോ നിങ്ങള്‍ വിശ്വസിക്കുന്നത് ആ അല്ലാഹുവിനെ നിങ്ങള്‍ സൂക്ഷിക്കുകയും ചെയ്യുക.
\end{malayalam}}
\flushright{\begin{Arabic}
\quranayah[5][89]
\end{Arabic}}
\flushleft{\begin{malayalam}
ബോധപൂര്‍വ്വമല്ലാത്ത നിങ്ങളുടെ ശപഥങ്ങളുടെ പേരില്‍ അവന്‍ നിങ്ങളെ പിടികൂടുകയില്ല. എന്നാല്‍ നിങ്ങള്‍ ഉറപ്പിച്ചു ചെയ്ത ശപഥങ്ങളുടെ പേരില്‍ അവന്‍ നിങ്ങളെ പിടികൂടുന്നതാണ്‌. അപ്പോള്‍ അതിന്‍റെ (അത് ലംഘിക്കുന്നതിന്‍റെ) പ്രായശ്ചിത്തം നിങ്ങള്‍ നിങ്ങളുടെ വീട്ടുകാര്‍ക്ക് നല്‍കാറുള്ള മദ്ധ്യനിലയിലുള്ള ഭക്ഷണത്തില്‍ നിന്ന് പത്തു സാധുക്കള്‍ക്ക് ഭക്ഷിക്കാന്‍ കൊടുക്കുകയോ, അല്ലെങ്കില്‍ അവര്‍ക്ക് വസ്ത്രം നല്‍കുകയോ, അല്ലെങ്കില്‍ ഒരു അടിമയെ മോചിപ്പിക്കുകയോ ആകുന്നു. ഇനി വല്ലവന്നും (അതൊന്നും) കിട്ടിയില്ലെങ്കില്‍ മൂന്നു ദിവസം നോമ്പെടുക്കുകയാണ് വേണ്ടത്‌. നിങ്ങള്‍ സത്യം ചെയ്തു പറഞ്ഞാല്‍, നിങ്ങളുടെ ശപഥങ്ങള്‍ ലംഘിക്കുന്നതിനുള്ള പ്രായശ്ചിത്തമാകുന്നു അത്‌. നിങ്ങളുടെ ശപഥങ്ങളെ നിങ്ങള്‍ സൂക്ഷിച്ച് കൊള്ളുക. അപ്രകാരം അല്ലാഹു അവന്‍റെ വചനങ്ങള്‍ നിങ്ങള്‍ക്ക് വിവരിച്ചുതരുന്നു; നിങ്ങള്‍ നന്ദിയുള്ളവരായിരിക്കാന്‍ വേണ്ടി.
\end{malayalam}}
\flushright{\begin{Arabic}
\quranayah[5][90]
\end{Arabic}}
\flushleft{\begin{malayalam}
സത്യവിശ്വാസികളേ, മദ്യവും ചൂതാട്ടവും പ്രതിഷ്ഠകളും പ്രശ്നം വെച്ച് നോക്കാനുള്ള അമ്പുകളും പൈശാചികമായ മ്ലേച്ഛവൃത്തി മാത്രമാകുന്നു. അതിനാല്‍ നിങ്ങള്‍ അതൊക്കെ വര്‍ജ്ജിക്കുക. നിങ്ങള്‍ക്ക് വിജയം പ്രാപിക്കാം.
\end{malayalam}}
\flushright{\begin{Arabic}
\quranayah[5][91]
\end{Arabic}}
\flushleft{\begin{malayalam}
പിശാച് ഉദ്ദേശിക്കുന്നത് മദ്യത്തിലൂടെയും, ചൂതാട്ടത്തിലൂടെയും നിങ്ങള്‍ക്കിടയില്‍ ശത്രുതയും വിദ്വേഷവും ഉളവാക്കുവാനും, അല്ലാഹുവെ ഓര്‍മിക്കുന്നതില്‍ നിന്നും നമസ്കാരത്തില്‍ നിന്നും നിങ്ങളെ തടയുവാനും മാത്രമാകുന്നു. അതിനാല്‍ നിങ്ങള്‍ (അവയില്‍ നിന്ന്‌) വിരമിക്കുവാനൊരുക്കമുണ്ടോ?
\end{malayalam}}
\flushright{\begin{Arabic}
\quranayah[5][92]
\end{Arabic}}
\flushleft{\begin{malayalam}
നിങ്ങള്‍ അല്ലാഹുവെയും റസൂലിനെയും അനുസരിക്കുകയും, (ധിക്കാരം വന്നു പോകാതെ) സൂക്ഷിക്കുകയും ചെയ്യുക. ഇനി നിങ്ങള്‍ പിന്തിരിഞ്ഞ് കളയുകയാണെങ്കില്‍ നമ്മുടെ ദൂതന്‍റെ ബാധ്യത വ്യക്തമായ രീതിയില്‍ സന്ദേശമെത്തിക്കുക മാത്രമാണെന്ന് നിങ്ങള്‍ മനസ്സിലാക്കുക.
\end{malayalam}}
\flushright{\begin{Arabic}
\quranayah[5][93]
\end{Arabic}}
\flushleft{\begin{malayalam}
വിശ്വസിക്കുകയും സല്‍കര്‍മ്മങ്ങള്‍ പ്രവര്‍ത്തിക്കുകയും ചെയ്തവര്‍ക്ക് അവര്‍ (മുമ്പ്‌) കഴിച്ചു പോയതില്‍ കുറ്റമില്ല. അവര്‍ (അല്ലാഹുവെ) സൂക്ഷിക്കുകയും വിശ്വസിക്കുകയും സല്‍പ്രവൃത്തികളില്‍ ഏര്‍പെടുകയും ചെയ്തിട്ടുണ്ടെങ്കില്‍. അതിനു ശേഷവും അവര്‍ സൂക്ഷ്മത പാലിക്കുകയും, നല്ല നിലയില്‍ വര്‍ത്തിക്കുകയും ചെയ്തിട്ടുണ്ടെങ്കില്‍. സദ്‌വൃത്തരെ അല്ലാഹു ഇഷ്ടപ്പെടുന്നു.
\end{malayalam}}
\flushright{\begin{Arabic}
\quranayah[5][94]
\end{Arabic}}
\flushleft{\begin{malayalam}
സത്യവിശ്വാസികളേ, നിങ്ങളുടെ കൈകള്‍കൊണ്ടും ശൂലങ്ങള്‍ കൊണ്ടും വേട്ടയാടിപ്പിടിക്കാവുന്ന വിധത്തിലുള്ള വല്ല ജന്തുക്കളും മുഖേന അല്ലാഹു നിങ്ങളെ പരീക്ഷിക്കുക തന്നെ ചെയ്യും. അദൃശ്യമായ നിലയില്‍ അല്ലാഹുവെ ഭയപ്പെടുന്നവരെ അവന്‍ വേര്‍തിരിച്ചറിയാന്‍ വേണ്ടിയത്രെ അത്‌. വല്ലവനും അതിന് ശേഷം അതിക്രമം കാണിച്ചാല്‍ അവന്ന് വേദനയേറിയ ശിക്ഷയുണ്ടായിരിക്കും.
\end{malayalam}}
\flushright{\begin{Arabic}
\quranayah[5][95]
\end{Arabic}}
\flushleft{\begin{malayalam}
സത്യവിശ്വാസികളേ, നിങ്ങള്‍ ഇഹ്‌റാമിലായിരിക്കെ വേട്ടമൃഗത്തെ കൊല്ലരുത്‌. നിങ്ങളിലൊരാള്‍ മനഃപൂര്‍വ്വം അതിനെ കൊല്ലുന്ന പക്ഷം, അവന്‍ കൊന്നതിന് തുല്യമെന്ന് നിങ്ങളില്‍ രണ്ടുപേര്‍ തീര്‍പ്പുകല്‍പിക്കുന്ന കാലിയെ (അഥവാ കാലികളെ) കഅ്ബത്തിങ്കല്‍ എത്തിച്ചേരേണ്ട ബലിമൃഗമായി നല്‍കേണ്ടതാണ്‌. അല്ലെങ്കില്‍ പ്രായശ്ചിത്തമായി ഏതാനും അഗതികള്‍ക്ക് ആഹാരം നല്‍കുകയോ, അല്ലെങ്കില്‍ അതിന് തുല്യമായി നോമ്പെടുക്കുകയോ ചെയ്യേണ്ടതാണ്‌. അവന്‍ ചെയ്തതിന്‍റെ ഭവിഷ്യത്ത് അവന്‍ അനുഭവിക്കാന്‍ വേണ്ടിയാണിത്‌. മുമ്പ് ചെയ്തു പോയതിന് അല്ലാഹു മാപ്പുനല്‍കിയിരിക്കുന്നു. വല്ലവനും അത് ആവര്‍ത്തിക്കുന്ന പക്ഷം അല്ലാഹു അവന്‍റെ നേരെ ശിക്ഷാനടപടി സ്വീകരിക്കുന്നതാണ്‌. അല്ലാഹു പ്രതാപിയും ശിക്ഷാനടപടി കൈക്കൊള്ളുന്നവനുമാകുന്നു.
\end{malayalam}}
\flushright{\begin{Arabic}
\quranayah[5][96]
\end{Arabic}}
\flushleft{\begin{malayalam}
നിങ്ങള്‍ക്കും യാത്രാസംഘങ്ങള്‍ക്കും ജീവിതവിഭവമായിക്കൊണ്ട് കടലിലെ വേട്ട ജന്തുക്കളും സമുദ്രാഹാരവും നിങ്ങള്‍ക്ക് അനുവദിക്കപ്പെട്ടിരിക്കുന്നു. നിങ്ങള്‍ ഇഹ്‌റാമിലായിരിക്കുമ്പോഴൊക്കെയും കരയിലെ വേട്ട ജന്തുക്കള്‍ നിങ്ങള്‍ക്ക് നിഷിദ്ധമാക്കപ്പെടുകയും ചെയ്തിരിക്കുന്നു. എതൊരുവനിലേക്കാണോ നിങ്ങള്‍ ഒരുമിച്ചുകൂട്ടപ്പെടുന്നത് ആ അല്ലാഹുവെ നിങ്ങള്‍ സൂക്ഷിക്കുക.
\end{malayalam}}
\flushright{\begin{Arabic}
\quranayah[5][97]
\end{Arabic}}
\flushleft{\begin{malayalam}
പവിത്രഭവനമായ കഅ്ബയെയും, യുദ്ധം നിഷിദ്ധമായ മാസത്തെയും അല്ലാഹു ജനങ്ങളുടെ നിലനില്‍പിന് ആധാരമാക്കിയിരിക്കുന്നു. (അതുപോലെതന്നെ കഅ്ബത്തിങ്കലേക്ക് കൊണ്ടുപോകുന്ന) ബലിമൃഗത്തെയും (അവയുടെ കഴുത്തിലെ) അടയാളത്താലികളെയും (അല്ലാഹു നിശ്ചയിച്ചിരിക്കുന്നു.) ആകാശത്തിലുള്ളതും ഭൂമിയിലുള്ളതും അല്ലാഹു അറിയുന്നുണ്ടെന്നും, അല്ലാഹു ഏത് കാര്യത്തെപ്പറ്റിയും അറിവുള്ളവനാണെന്നും നിങ്ങള്‍ മനസ്സിലാക്കുവാന്‍ വേണ്ടിയത്രെ അത്‌.
\end{malayalam}}
\flushright{\begin{Arabic}
\quranayah[5][98]
\end{Arabic}}
\flushleft{\begin{malayalam}
അല്ലാഹു കഠിനമായി ശിക്ഷിക്കുന്നവനാണെന്നും അല്ലാഹു ഏറെ പൊറുക്കുന്നവനും കരുണാനിധിയുമാണെന്നും നിങ്ങള്‍ മനസ്സിലാക്കുക.
\end{malayalam}}
\flushright{\begin{Arabic}
\quranayah[5][99]
\end{Arabic}}
\flushleft{\begin{malayalam}
റസൂലിന്‍റെ മേല്‍ പ്രബോധന ബാധ്യത മാത്രമേയുള്ളൂ. നിങ്ങള്‍ വെളിപ്പെടുത്തുന്നതും ഒളിച്ചുവെക്കുന്നതുമെല്ലാം അല്ലാഹു അറിയുന്നു.
\end{malayalam}}
\flushright{\begin{Arabic}
\quranayah[5][100]
\end{Arabic}}
\flushleft{\begin{malayalam}
(നബിയേ,) പറയുക: ദുഷിച്ചതും നല്ലതും സമമാകുകയില്ല. ദുഷിച്ചതിന്‍റെ വര്‍ദ്ധനവ് നിന്നെ അത്ഭുതപ്പെടുത്തിയാലും ശരി. അതിനാല്‍ ബുദ്ധിമാന്‍മാരേ, നിങ്ങള്‍ അല്ലാഹുവെ സൂക്ഷിക്കുക. നിങ്ങള്‍ വിജയം പ്രാപിച്ചേക്കാം.
\end{malayalam}}
\flushright{\begin{Arabic}
\quranayah[5][101]
\end{Arabic}}
\flushleft{\begin{malayalam}
സത്യവിശ്വാസികളേ, ചിലകാര്യങ്ങളെപ്പറ്റി നിങ്ങള്‍ ചോദിക്കരുത്‌. നിങ്ങള്‍ക്ക് അവ വെളിപ്പെടുത്തപ്പെട്ടാല്‍ നിങ്ങള്‍ക്കത് മനഃപ്രയാസമുണ്ടാക്കും. ഖുര്‍ആന്‍ അവതരിപ്പിക്കപ്പെടുന്ന സമയത്ത് നിങ്ങളവയെപ്പറ്റി ചോദിക്കുകയാണെങ്കില്‍ നിങ്ങള്‍ക്കവ വെളിപ്പെടുത്തുക തന്നെ ചെയ്യും. (നിങ്ങള്‍ ചോദിച്ച് കഴിഞ്ഞതിന്‌) അല്ലാഹു (നിങ്ങള്‍ക്ക്‌) മാപ്പുനല്‍കിയിരിക്കുന്നു. അല്ലാഹു ഏറെ പൊറുക്കുന്നവനും സഹനശീലനുമാകുന്നു.
\end{malayalam}}
\flushright{\begin{Arabic}
\quranayah[5][102]
\end{Arabic}}
\flushleft{\begin{malayalam}
നിങ്ങള്‍ക്ക് മുമ്പ് ഒരു ജനവിഭാഗം അത്തരം ചോദ്യങ്ങള്‍ ചോദിക്കുകയുണ്ടായി. പിന്നെ അവയില്‍ അവര്‍ അവിശ്വസിക്കുന്നവരായിത്തീരുകയും ചെയ്തു.
\end{malayalam}}
\flushright{\begin{Arabic}
\quranayah[5][103]
\end{Arabic}}
\flushleft{\begin{malayalam}
ബഹീറഃ, സാഇബഃ, വസ്വീലഃ, ഹാം എന്നീ നേര്‍ച്ചമൃഗങ്ങളെയൊന്നും അല്ലാഹു നിശ്ചയിച്ചതല്ല. പക്ഷെ, സത്യനിഷേധികള്‍ അല്ലാഹുവിന്‍റെ പേരില്‍ കള്ളം കെട്ടിച്ചമയ്ക്കുകയാണ്‌. അവരില്‍ അധികപേരും ചിന്തിച്ച് മനസ്സിലാക്കുന്നില്ല.
\end{malayalam}}
\flushright{\begin{Arabic}
\quranayah[5][104]
\end{Arabic}}
\flushleft{\begin{malayalam}
അല്ലാഹു അവതരിപ്പിച്ചതിലേക്കും, റസൂലിലേക്കും വരുവിന്‍ എന്ന് അവരോട് പറയപ്പെട്ടാല്‍, ഞങ്ങളുടെ പിതാക്കളെ ഏതൊരു നിലപാടിലാണോ ഞങ്ങള്‍ കണ്ടെത്തിയത് അതു മതി ഞങ്ങള്‍ക്ക്‌. എന്നായിരിക്കും അവര്‍ പറയുക: അവരുടെ പിതാക്കള്‍ യാതൊന്നുമറിയാത്തവരും, സന്‍മാര്‍ഗം പ്രാപിക്കാത്തവരും ആയിരുന്നാല്‍ പോലും (അത് മതിയെന്നോ?)
\end{malayalam}}
\flushright{\begin{Arabic}
\quranayah[5][105]
\end{Arabic}}
\flushleft{\begin{malayalam}
സത്യവിശ്വാസികളേ, നിങ്ങള്‍ നിങ്ങളുടെ കാര്യങ്ങള്‍ ശ്രദ്ധിച്ച് കൊള്ളുക. നിങ്ങള്‍ സന്‍മാര്‍ഗം പ്രാപിച്ചിട്ടുണ്ടെങ്കില്‍ വഴിപിഴച്ചവര്‍ നിങ്ങള്‍ക്കൊരു ദ്രോഹവും വരുത്തുകയില്ല. അല്ലാഹുവിങ്കലേക്കത്രെ നിങ്ങളുടെയെല്ലാം മടക്കം. നിങ്ങള്‍ ചെയ്ത് കൊണ്ടിരിക്കുന്നതിനെപ്പറ്റിയെല്ലാം അപ്പോള്‍ അവന്‍ നിങ്ങളെ വിവരമറിയിക്കുന്നതാണ്‌.
\end{malayalam}}
\flushright{\begin{Arabic}
\quranayah[5][106]
\end{Arabic}}
\flushleft{\begin{malayalam}
സത്യവിശ്വാസികളേ, നിങ്ങളിലൊരാള്‍ക്ക് മരണമാസന്നമായാല്‍ വസ്വിയ്യത്തിന്‍റെ സമയത്ത് നിങ്ങളില്‍ നിന്നുള്ള നീതിമാന്‍മാരായ രണ്ടുപേര്‍ നിങ്ങള്‍ക്കിടയില്‍ സാക്ഷ്യം വഹിക്കേണ്ടതാണ്‌. ഇനി നിങ്ങള്‍ ഭൂമിയിലൂടെ യാത്രചെയ്യുന്ന സമയത്താണ് മരണവിപത്ത് നിങ്ങള്‍ക്ക് വന്നെത്തുന്നതെങ്കില്‍ (വസ്വിയ്യത്തിന് സാക്ഷികളായി) നിങ്ങളല്ലാത്തവരില്‍ പെട്ട രണ്ടുപേരായാലും മതി. നിങ്ങള്‍ക്ക് സംശയം തോന്നുകയാണെങ്കില്‍ അവരെ രണ്ടുപേരെയും നമസ്കാരം കഴിഞ്ഞതിന് ശേഷം നിങ്ങള്‍ തടഞ്ഞ് നിര്‍ത്തണം. എന്നിട്ടവര്‍ അല്ലാഹുവിന്‍റെ പേരില്‍ ഇപ്രകാരം സത്യം ചെയ്ത് പറയണം: ഇതിന് (ഈ സത്യം മറച്ചു വെക്കുന്നതിന്‌) പകരം യാതൊരു വിലയും ഞങ്ങള്‍ വാങ്ങുകയില്ല. അത് അടുത്ത ഒരു ബന്ധുവെ ബാധിക്കുന്ന കാര്യമായാല്‍ പോലും. അല്ലാഹുവിനുവേണ്ടി ഏറ്റെടുത്ത സാക്ഷ്യം ഞങ്ങള്‍ മറച്ച് വെക്കുകയില്ല. അങ്ങനെ ചെയ്താല്‍ തീര്‍ച്ചയായും ഞങ്ങള്‍ കുറ്റക്കാരില്‍ പെട്ടവരായിരിക്കും.
\end{malayalam}}
\flushright{\begin{Arabic}
\quranayah[5][107]
\end{Arabic}}
\flushleft{\begin{malayalam}
ഇനി അവര്‍ (രണ്ടു സാക്ഷികള്‍) കുറ്റത്തിന് അവകാശികളായിട്ടുണ്ട് എന്ന് തെളിയുന്ന പക്ഷം കുറ്റം ചെയ്തിട്ടുള്ളത് ആര്‍ക്കെതിരിലാണോ അവരില്‍ പെട്ട (പരേതനോട്‌) കൂടുതല്‍ ബന്ധമുള്ള മറ്റ് രണ്ടുപേര്‍ അവരുടെ സ്ഥാനത്ത് (സാക്ഷികളായി) നില്‍ക്കണം. എന്നിട്ട് അവര്‍ രണ്ടുപേരും അല്ലാഹുവിന്‍റെ പേരില്‍ ഇപ്രകാരം സത്യം ചെയ്ത് പറയണം: തീര്‍ച്ചയായും ഞങ്ങളുടെ സാക്ഷ്യമാകുന്നു ഇവരുടെ സാക്ഷ്യത്തേക്കാള്‍ സത്യസന്ധമായിട്ടുള്ളത്‌. ഞങ്ങള്‍ ഒരു അന്യായവും ചെയ്തിട്ടില്ല. അങ്ങനെ ചെയ്താല്‍ തീര്‍ച്ചയായും ഞങ്ങള്‍ അക്രമികളില്‍ പെട്ടവരായിരിക്കും.
\end{malayalam}}
\flushright{\begin{Arabic}
\quranayah[5][108]
\end{Arabic}}
\flushleft{\begin{malayalam}
അവര്‍ (സാക്ഷികള്‍) മുറപോലെ സാക്ഷ്യം വഹിക്കുന്നതിന് അതാണ് കൂടുതല്‍ അനുയോജ്യമായിട്ടുള്ളത്‌. തങ്ങള്‍ സത്യം ചെയ്തതിന് ശേഷം (അനന്തരാവകാശികള്‍ക്ക്‌) സത്യം ചെയ്യാന്‍ അവസരം നല്‍കപ്പെടുമെന്ന് അവര്‍ക്ക് (സാക്ഷികള്‍ക്ക്‌) പേടിയുണ്ടാകുവാനും (അതാണ് കൂടുതല്‍ ഉപകരിക്കുക.) നിങ്ങള്‍ അല്ലാഹുവെ സൂക്ഷിക്കുകയും (അവന്‍റെ കല്‍പനകള്‍) ശ്രദ്ധിക്കുകയും ചെയ്യുക. ധിക്കാരികളായ ആളുകളെ അല്ലാഹു നേര്‍വഴിയിലാക്കുകയില്ല.
\end{malayalam}}
\flushright{\begin{Arabic}
\quranayah[5][109]
\end{Arabic}}
\flushleft{\begin{malayalam}
അല്ലാഹു ദൂതന്‍മാരെ ഒരുമിച്ചുകൂട്ടുകയും, നിങ്ങള്‍ക്ക് എന്ത് മറുപടിയാണ് കിട്ടിയത് എന്ന് ചോദിക്കുകയും ചെയ്യുന്ന ദിവസം അവര്‍ പറയും: ഞങ്ങള്‍ക്ക് യാതൊരു അറിവുമില്ല. നീയാണ് അദൃശ്യകാര്യങ്ങള്‍ നന്നായി അറിയുന്നവന്‍.
\end{malayalam}}
\flushright{\begin{Arabic}
\quranayah[5][110]
\end{Arabic}}
\flushleft{\begin{malayalam}
(ഈസായോട്‌) അല്ലാഹു പറഞ്ഞ സന്ദര്‍ഭം (ശ്രദ്ധേയമാകുന്നു.) മര്‍യമിന്‍റെ മകനായ ഈസാ! തൊട്ടിലില്‍ വെച്ചും, മദ്ധ്യവയസ്കനായിരിക്കെയും നീ ജനങ്ങളോട് സംസാരിക്കവെ, പരിശുദ്ധാത്മാവ് മുഖേന നിനക്ക് ഞാന്‍ പിന്‍ബലം നല്‍കിയ സന്ദര്‍ഭത്തിലും, ഗ്രന്ഥവും ജ്ഞാനവും തൌറാത്തും ഇന്‍ജീലും നിനക്ക് ഞാന്‍ പഠിപ്പിച്ചുതന്ന സന്ദര്‍ഭത്തിലും, എന്‍റെ അനുമതി പ്രകാരം കളിമണ്ണ് കൊണ്ട് നീ പക്ഷിയുടെ മാതൃകയില്‍ രൂപപ്പെടുത്തുകയും, എന്നിട്ട് നീ അതില്‍ ഊതുമ്പോള്‍ എന്‍റെ അനുമതി പ്രകാരം അത് പക്ഷിയായിത്തീരുകയും ചെയ്യുന്ന സന്ദര്‍ഭത്തിലും, എന്‍റെ അനുമതി പ്രകാരം ജന്‍മനാ കാഴ്ചയില്ലാത്തവനെയും, പാണ്ഡുരോഗിയെയും നീ സുഖപ്പെടുത്തുന്ന സന്ദര്‍ഭത്തിലും, എന്‍റെ അനുമതി പ്രകാരം നീ മരണപ്പെട്ടവരെ പുറത്ത് കൊണ്ട് വരുന്ന സന്ദര്‍ഭത്തിലും, നീ ഇസ്രായീല്‍ സന്തതികളുടെ അടുത്ത് വ്യക്തമായ തെളിവുകളുമായി ചെന്നിട്ട് അവരിലെ സത്യനിഷേധികള്‍ ഇത് പ്രത്യക്ഷമായ മാരണം മാത്രമാകുന്നു. എന്ന് പറഞ്ഞ അവസരത്തില്‍ നിന്നെ അപകടപ്പെടുത്തുന്നതില്‍ നിന്ന് അവരെ ഞാന്‍ തടഞ്ഞ സന്ദര്‍ഭത്തിലും ഞാന്‍ നിനക്കും നിന്‍റെ മാതാവിനും ചെയ്ത് തന്ന അനുഗ്രഹം ഓര്‍ക്കുക.
\end{malayalam}}
\flushright{\begin{Arabic}
\quranayah[5][111]
\end{Arabic}}
\flushleft{\begin{malayalam}
നിങ്ങള്‍ എന്നിലും എന്‍റെ ദൂതനിലും വിശ്വസിക്കൂ എന്ന് ഞാന്‍ ഹവാരികള്‍ക്ക് ബോധനം നല്‍കിയ സന്ദര്‍ഭത്തിലും. അവര്‍ പറഞ്ഞു: ഞങ്ങള്‍ വിശ്വസിച്ചിരിക്കുന്നു. ഞങ്ങള്‍ മുസ്ലിംകളാണെന്നതിന് നീ സാക്ഷ്യം വഹിച്ച് കൊള്ളുക.
\end{malayalam}}
\flushright{\begin{Arabic}
\quranayah[5][112]
\end{Arabic}}
\flushleft{\begin{malayalam}
ഹവാരികള്‍ പറഞ്ഞ സന്ദര്‍ഭം ശ്രദ്ധിക്കുക: മര്‍യമിന്‍റെ മകനായ ഈസാ, ആകാശത്തുനിന്ന് ഞങ്ങള്‍ക്ക് ഒരു ഭക്ഷണത്തളിക ഇറക്കിത്തരുവാന്‍ നിന്‍റെ രക്ഷിതാവിന് സാധിക്കുമോ? അദ്ദേഹം പറഞ്ഞു: നിങ്ങള്‍ വിശ്വാസികളാണെങ്കില്‍ അല്ലാഹുവെ സൂക്ഷിക്കുക.
\end{malayalam}}
\flushright{\begin{Arabic}
\quranayah[5][113]
\end{Arabic}}
\flushleft{\begin{malayalam}
അവര്‍ പറഞ്ഞു: ഞങ്ങള്‍ക്കതില്‍ നിന്ന് ഭക്ഷിക്കുവാനും അങ്ങനെ ഞങ്ങള്‍ക്ക് മനസ്സമാധാനമുണ്ടാകുവാനും, താങ്കള്‍ ഞങ്ങളോട് പറഞ്ഞത് സത്യമാണെന്ന് ഞങ്ങള്‍ക്ക് ബോധ്യമാകുവാനും, ഞങ്ങള്‍ അതിന് ദൃക്സാക്ഷികളായിത്തീരുവാനുമാണ് ഞങ്ങള്‍ ആഗ്രഹിക്കുന്നത്‌.
\end{malayalam}}
\flushright{\begin{Arabic}
\quranayah[5][114]
\end{Arabic}}
\flushleft{\begin{malayalam}
മര്‍യമിന്‍റെ മകന്‍ ഈസാ പറഞ്ഞു: ഞങ്ങളുടെ രക്ഷിതാവായ അല്ലാഹുവേ, ഞങ്ങള്‍ക്ക് നീ ആകാശത്ത് നിന്ന് ഒരു ഭക്ഷണത്തളിക ഇറക്കിത്തരേണമേ. ഞങ്ങള്‍ക്ക്‌, ഞങ്ങളിലെ ആദ്യത്തിലുള്ളവന്നും, അവസാനത്തിലുള്ളവന്നും ഒരു പെരുന്നാളും, നിന്‍റെ പക്കല്‍ നിന്നുള്ള ഒരു ദൃഷ്ടാന്തവുമായിരിക്കണം അത്‌. ഞങ്ങള്‍ക്ക് നീ ഉപജീവനം നല്‍കുകയും ചെയ്യേണമേ. നീ ഉപജീവനം നല്‍കുന്നവരില്‍ ഏറ്റവും ഉത്തമനാണല്ലോ.
\end{malayalam}}
\flushright{\begin{Arabic}
\quranayah[5][115]
\end{Arabic}}
\flushleft{\begin{malayalam}
അല്ലാഹു പറഞ്ഞു: ഞാന്‍ നിങ്ങള്‍ക്കത് ഇറക്കിത്തരാം. എന്നാല്‍ അതിന് ശേഷം നിങ്ങളില്‍ ആരെങ്കിലും അവിശ്വസിക്കുന്ന പക്ഷം ലോകരില്‍ ഒരാള്‍ക്കും ഞാന്‍ നല്‍കാത്ത വിധമുള്ള (കടുത്ത) ശിക്ഷ അവന്ന് നല്‍കുന്നതാണ്‌.
\end{malayalam}}
\flushright{\begin{Arabic}
\quranayah[5][116]
\end{Arabic}}
\flushleft{\begin{malayalam}
അല്ലാഹു പറയുന്ന സന്ദര്‍ഭവും (ശ്രദ്ധിക്കുക.) മര്‍യമിന്‍റെ മകന്‍ ഈസാ, അല്ലാഹുവിന് പുറമെ എന്നെയും, എന്‍റെ മാതാവിനെയും ദൈവങ്ങളാക്കിക്കൊള്ളുവിന്‍. എന്ന് നീയാണോ ജനങ്ങളോട് പറഞ്ഞത്‌? അദ്ദേഹം പറയും: നീയെത്ര പരിശുദ്ധന്‍! എനിക്ക് (പറയാന്‍) യാതൊരു അവകാശവുമില്ലാത്തത് ഞാന്‍ പറയാവതല്ലല്ലോ? ഞാനത് പറഞ്ഞിരുന്നെങ്കില്‍ തീര്‍ച്ചയായും നീയത് അറിഞ്ഞിരിക്കുമല്ലോ. എന്‍റെ മനസ്സിലുള്ളത് നീ അറിയും. നിന്‍റെ മനസ്സിലുള്ളത് ഞാനറിയില്ല. തീര്‍ച്ചയായും നീ തന്നെയാണ് അദൃശ്യകാര്യങ്ങള്‍ അറിയുന്നവന്‍.
\end{malayalam}}
\flushright{\begin{Arabic}
\quranayah[5][117]
\end{Arabic}}
\flushleft{\begin{malayalam}
നീ എന്നോട് കല്‍പിച്ച കാര്യം അഥവാ എന്‍റെയും നിങ്ങളുടെയും രക്ഷിതാവായ അല്ലാഹുവെ നിങ്ങള്‍ ആരാധിക്കണം എന്ന കാര്യം മാത്രമേ ഞാനവരോട് പറഞ്ഞിട്ടുള്ളൂ. ഞാന്‍ അവര്‍ക്കിടയില്‍ ഉണ്ടായിരുന്നപ്പോഴൊക്കെ ഞാന്‍ അവരുടെ മേല്‍ സാക്ഷിയായിരുന്നു. പിന്നീട് നീ എന്നെ പൂര്‍ണ്ണമായി ഏറ്റെടുത്തപ്പോള്‍ നീ തന്നെയായിരുന്നു അവരെ നിരീക്ഷിച്ചിരുന്നവന്‍. നീ എല്ലാകാര്യത്തിനും സാക്ഷിയാകുന്നു.
\end{malayalam}}
\flushright{\begin{Arabic}
\quranayah[5][118]
\end{Arabic}}
\flushleft{\begin{malayalam}
നീ അവരെ ശിക്ഷിക്കുകയാണെങ്കില്‍ തീര്‍ച്ചയായും അവര്‍ നിന്‍റെ ദാസന്‍മാരാണല്ലോ. നീ അവര്‍ക്ക് പൊറുത്തുകൊടുക്കുകയാണെങ്കില്‍ നീ തന്നെയാണല്ലോ പ്രതാപിയും യുക്തിമാനും.
\end{malayalam}}
\flushright{\begin{Arabic}
\quranayah[5][119]
\end{Arabic}}
\flushleft{\begin{malayalam}
അല്ലാഹു പറയും: ഇത് സത്യവാന്‍മാര്‍ക്ക് തങ്ങളുടെ സത്യസന്ധത പ്രയോജനപ്പെടുന്ന ദിവസമാകുന്നു. അവര്‍ക്ക് താഴ്ഭാഗത്ത് കൂടി അരുവികള്‍ ഒഴുകുന്ന സ്വര്‍ഗത്തോപ്പുകളുണ്ട്‌. അവരതില്‍ നിത്യവാസികളായിരിക്കും. അവരെപ്പറ്റി അല്ലാഹു തൃപ്തിപ്പെട്ടിരിക്കുന്നു. അവര്‍ അവനെപ്പറ്റിയും തൃപ്തിപ്പെട്ടിരിക്കുന്നു. അതത്രെ മഹത്തായ വിജയം.
\end{malayalam}}
\flushright{\begin{Arabic}
\quranayah[5][120]
\end{Arabic}}
\flushleft{\begin{malayalam}
ആകാശങ്ങളുടെയും ഭൂമിയുടെയും അവയിലുള്ളതിന്‍റെയും ആധിപത്യം അല്ലാഹുവിന്നത്രെ. അവന്‍ ഏത് കാര്യത്തിനും കഴിവുള്ളവനാകുന്നു.
\end{malayalam}}
\chapter{\textmalayalam{അന്‍ആം ( കാലികള്‍ )}}
\begin{Arabic}
\Huge{\centerline{\basmalah}}\end{Arabic}
\flushright{\begin{Arabic}
\quranayah[6][1]
\end{Arabic}}
\flushleft{\begin{malayalam}
ആകാശങ്ങളും ഭൂമിയും സൃഷ്ടിക്കുകയും, ഇരുട്ടുകളും വെളിച്ചവും ഉണ്ടാക്കുകയും ചെയ്ത അല്ലാഹുവിന്നാകുന്നു സ്തുതി. എന്നിട്ടുമതാ സത്യനിഷേധികള്‍ തങ്ങളുടെ രക്ഷിതാവിന് സമന്‍മാരെ വെക്കുന്നു.
\end{malayalam}}
\flushright{\begin{Arabic}
\quranayah[6][2]
\end{Arabic}}
\flushleft{\begin{malayalam}
അവനത്രെ കളിമണ്ണില്‍ നിന്നും നിങ്ങളെ സൃഷ്ടിച്ചത്‌. എന്നിട്ടവന്‍ ഒരു അവധി നിശ്ചയിച്ചിരിക്കുന്നു. അവങ്കല്‍ നിര്‍ണിതമായ മറ്റൊരവധിയുമുണ്ട്‌. എന്നിട്ടും നിങ്ങള്‍ സംശയിച്ചു കൊണ്ടിരിക്കുന്നു.
\end{malayalam}}
\flushright{\begin{Arabic}
\quranayah[6][3]
\end{Arabic}}
\flushleft{\begin{malayalam}
അവന്‍ തന്നെയാണ് ആകാശങ്ങളിലും ഭൂമിയിലും സാക്ഷാല്‍ ദൈവം. നിങ്ങളുടെ രഹസ്യവും നിങ്ങളുടെ പരസ്യവും അവന്‍ അറിയുന്നു. നിങ്ങള്‍ നേടിയെടുക്കുന്നതും അവന്‍ അറിയുന്നു.
\end{malayalam}}
\flushright{\begin{Arabic}
\quranayah[6][4]
\end{Arabic}}
\flushleft{\begin{malayalam}
അവരുടെ രക്ഷിതാവിങ്കല്‍ നിന്നുള്ള ഏതൊരു ദൃഷ്ടാന്തം അവര്‍ക്ക് വന്നുകിട്ടുമ്പോഴും അവരതിനെ അവഗണിച്ച് കളയുക തന്നെയാകുന്നു.
\end{malayalam}}
\flushright{\begin{Arabic}
\quranayah[6][5]
\end{Arabic}}
\flushleft{\begin{malayalam}
അങ്ങനെ ഈ സത്യം അവര്‍ക്ക് വന്ന് കിട്ടിയപ്പോഴും അവരതിനെ നിഷേധിച്ചു കളഞ്ഞു. എന്നാല്‍ അവര്‍ ഏതൊന്നിനെ പരിഹസിച്ച് കൊണ്ടിരുന്നുവോ അതിന്‍റെ വൃത്താന്തങ്ങള്‍ വഴിയെ അവര്‍ക്ക് വന്നെത്തുന്നതാണ്‌.
\end{malayalam}}
\flushright{\begin{Arabic}
\quranayah[6][6]
\end{Arabic}}
\flushleft{\begin{malayalam}
അവര്‍ കണ്ടില്ലേ; അവര്‍ക്ക് മുമ്പ് നാം എത്ര തലമുറകളെ നശിപ്പിച്ചിട്ടുണ്ടെന്ന്‌? നിങ്ങള്‍ക്ക് നാം ചെയ്ത് തന്നിട്ടില്ലാത്ത സൌകര്യം ഭൂമിയില്‍ അവര്‍ക്ക് നാം ചെയ്ത് കൊടുത്തിരുന്നു. നാം അവര്‍ക്ക് ധാരാളമായി മഴ വര്‍ഷിപ്പിച്ച് കൊടുക്കുകയും, അവരുടെ താഴ്ഭാഗത്ത് കൂടി നദികള്‍ ഒഴുക്കുകയും ചെയ്തിരുന്നു. എന്നിട്ട് അവരുടെ പാപങ്ങള്‍ കാരണം നാം അവരെ നശിപ്പിക്കുകയും, അവര്‍ക്ക് ശേഷം നാം വേറെ തലമുറകളെ ഉണ്ടാക്കുകയും ചെയ്തു.
\end{malayalam}}
\flushright{\begin{Arabic}
\quranayah[6][7]
\end{Arabic}}
\flushleft{\begin{malayalam}
(നബിയേ,) നിനക്ക് നാം കടലാസില്‍ എഴുതിയ ഒരു ഗ്രന്ഥം ഇറക്കിത്തരികയും, എന്നിട്ടവരത് സ്വന്തം കൈകള്‍കൊണ്ട് തൊട്ടുനോക്കുകയും ചെയ്താല്‍ പോലും ഇത് വ്യക്തമായ മായാജാലമല്ലാതെ മറ്റൊന്നുമല്ല എന്നായിരിക്കും സത്യനിഷേധികള്‍ പറയുക.
\end{malayalam}}
\flushright{\begin{Arabic}
\quranayah[6][8]
\end{Arabic}}
\flushleft{\begin{malayalam}
ഇയാളുടെ (നബി (സ) യുടെ) മേല്‍ ഒരു മലക്ക് ഇറക്കപ്പെടാത്തത് എന്താണ് എന്നും അവര്‍ പറയുകയുണ്ടായി. എന്നാല്‍ നാം മലക്കിനെ ഇറക്കിയിരുന്നെങ്കില്‍ കാര്യം (അന്തിമമായി) തീരുമാനിക്കപ്പെടുമായിരുന്നു. പിന്നീടവര്‍ക്ക് സമയം നീട്ടിക്കിട്ടുമായിരുന്നില്ല.
\end{malayalam}}
\flushright{\begin{Arabic}
\quranayah[6][9]
\end{Arabic}}
\flushleft{\begin{malayalam}
ഇനി നാം ഒരു മലക്കിനെ (ദൂതനായി) നിശ്ചയിക്കുകയാണെങ്കില്‍ തന്നെ ആ മലക്കിനെയും നാം പുരുഷരൂപത്തിലാക്കുമായിരുന്നു. അങ്ങനെ (ഇന്ന്‌) അവര്‍ ആശയക്കുഴപ്പമുണ്ടാക്കുന്ന വിഷയത്തില്‍ (അപ്പോഴും) നാം അവര്‍ക്ക് സംശയമുണ്ടാക്കുന്നതാണ്‌.
\end{malayalam}}
\flushright{\begin{Arabic}
\quranayah[6][10]
\end{Arabic}}
\flushleft{\begin{malayalam}
നിനക്ക് മുമ്പ് പല ദൂതന്‍മാരും പരിഹസിക്കപ്പെട്ടിട്ടുണ്ട്‌. എന്നിട്ട് അവരെ കളിയാക്കിയിരുന്നവര്‍ക്ക് അവര്‍ പരിഹസിച്ചു കൊണ്ടിരുന്നതെന്തോ അത് വന്നുഭവിക്കുക തന്നെ ചെയ്തു.
\end{malayalam}}
\flushright{\begin{Arabic}
\quranayah[6][11]
\end{Arabic}}
\flushleft{\begin{malayalam}
(നബിയേ,) പറയുക: നിങ്ങള്‍ ഭൂമിയിലൂടെ സഞ്ചരിക്കൂ. എന്നിട്ട് സത്യനിഷേധികളുടെ പര്യവസാനം എങ്ങനെയായിരുന്നുവെന്ന് നോക്കൂ.
\end{malayalam}}
\flushright{\begin{Arabic}
\quranayah[6][12]
\end{Arabic}}
\flushleft{\begin{malayalam}
ചോദിക്കുക: ആകാശങ്ങളിലും ഭൂമിയിലുമുള്ളതെല്ലാം ആരുടെതാകുന്നു? പറയുക: അല്ലാഹുവിന്‍റെതത്രെ. അവന്‍ കാരുണ്യത്തെ സ്വന്തം പേരില്‍ (ബാധ്യതയായി) രേഖപ്പെടുത്തിയിരിക്കുന്നു. ഉയിര്‍ത്തെഴുന്നേല്‍പിന്‍റെ നാളിലേക്ക് നിങ്ങളെ അവന്‍ ഒരുമിച്ചുകൂട്ടുക തന്നെ ചെയ്യും. അതില്‍ യാതൊരു സംശയവുമില്ല. എന്നാല്‍ സ്വദേഹങ്ങളെത്തന്നെ നഷ്ടത്തിലാക്കിയവരത്രെ അവര്‍. അതിനാല്‍ അവര്‍ വിശ്വസിക്കുകയില്ല.
\end{malayalam}}
\flushright{\begin{Arabic}
\quranayah[6][13]
\end{Arabic}}
\flushleft{\begin{malayalam}
അവന്‍റെതാകുന്നു രാത്രിയിലും പകലിലും അടങ്ങിയതെല്ലാം. അവന്‍ എല്ലാം കേള്‍ക്കുന്നവനും അറിയുന്നവനുമത്രെ.
\end{malayalam}}
\flushright{\begin{Arabic}
\quranayah[6][14]
\end{Arabic}}
\flushleft{\begin{malayalam}
പറയുക: ആകാശങ്ങളുടെയും ഭൂമിയുടെയും സ്രഷ്ടാവായ അല്ലാഹുവെയല്ലാതെ ഞാന്‍ രക്ഷാധികാരിയായി സ്വീകരിക്കുകയോ? അവനാകട്ടെ ആഹാരം നല്‍കുന്നു. അവന്ന് ആഹാരം നല്‍കപ്പെടുകയില്ല. പറയുക: തീര്‍ച്ചയായും അല്ലാഹുവിന് കീഴ്പെട്ടവരില്‍ ഒന്നാമനായിരിക്കുവാനാണ് ഞാന്‍ കല്‍പിക്കപ്പെട്ടിട്ടുള്ളത്‌. നീ ഒരിക്കലും ബഹുദൈവാരാധകരില്‍ പെട്ടുപോകരുത്‌.
\end{malayalam}}
\flushright{\begin{Arabic}
\quranayah[6][15]
\end{Arabic}}
\flushleft{\begin{malayalam}
പറയുക: ഞാന്‍ എന്‍റെ രക്ഷിതാവിനോട് അനുസരണക്കേട് കാണിക്കുന്ന പക്ഷം ഭയങ്കരമായ ഒരു ദിവസത്തെ ശിക്ഷയെപ്പറ്റി തീര്‍ച്ചയായും ഞാന്‍ ഭയപ്പെടുന്നു.
\end{malayalam}}
\flushright{\begin{Arabic}
\quranayah[6][16]
\end{Arabic}}
\flushleft{\begin{malayalam}
അന്നേ ദിവസം ആരില്‍ നിന്ന് അത് (ശിക്ഷ) ഒഴിവാക്കപ്പെടുന്നുവോ അവനെ അല്ലാഹു തീര്‍ച്ചയായും അനുഗ്രഹിച്ചിരിക്കുന്നു. അതത്രെ വ്യക്തമായ വിജയം.
\end{malayalam}}
\flushright{\begin{Arabic}
\quranayah[6][17]
\end{Arabic}}
\flushleft{\begin{malayalam}
(നബിയേ,) നിനക്ക് അല്ലാഹു വല്ല ദോഷവും വരുത്തിവെക്കുകയാണെങ്കില്‍ അത് നീക്കം ചെയ്യുവാന്‍ അവനല്ലാതെ മറ്റാരുമില്ല. നിനക്ക് അവന്‍ വല്ല ഗുണവും വരുത്തുകയാണെങ്കിലോ അവന്‍ ഏതൊരു കാര്യത്തിനും കഴിവുള്ളവനത്രെ.
\end{malayalam}}
\flushright{\begin{Arabic}
\quranayah[6][18]
\end{Arabic}}
\flushleft{\begin{malayalam}
അവന്‍ തന്‍റെ ദാസന്‍മാരുടെ മേല്‍ പരമാധികാരമുള്ളവനാണ്‌. യുക്തിമാനും സൂക്ഷ്മമായി അറിയുന്നവനുമത്രെ അവന്‍.
\end{malayalam}}
\flushright{\begin{Arabic}
\quranayah[6][19]
\end{Arabic}}
\flushleft{\begin{malayalam}
(നബിയേ,) ചോദിക്കുക: സാക്ഷ്യത്തില്‍ വെച്ച് ഏറ്റവും വലിയത് ഏതാകുന്നു? പറയുക: അല്ലാഹുവാണ് എനിക്കും നിങ്ങള്‍ക്കും ഇടയില്‍ സാക്ഷി. ഈ ഖുര്‍ആന്‍ എനിക്ക് ദിവ്യബോധനമായി നല്‍കപ്പെട്ടിട്ടുള്ളത്‌, അത് മുഖേന നിങ്ങള്‍ക്കും അത് (അതിന്‍റെ സന്ദേശം) ചെന്നെത്തുന്ന എല്ലാവര്‍ക്കും ഞാന്‍ മുന്നറിയിപ്പ് നല്‍കുന്നതിന് വേണ്ടിയാകുന്നു. അല്ലാഹുവോടൊപ്പം വേറെ ദൈവങ്ങളുണ്ടെന്നതിന് യഥാര്‍ത്ഥത്തില്‍ നിങ്ങള്‍ സാക്ഷ്യം വഹിക്കുമോ? പറയുക: ഞാന്‍ സാക്ഷ്യം വഹിക്കുകയില്ല. പറയുക: അവന്‍ ഏകദൈവം മാത്രമാകുന്നു. നിങ്ങള്‍ അവനോട് പങ്കുചേര്‍ക്കുന്നതുമായി എനിക്ക് യാതൊരു ബന്ധവുമില്ല.
\end{malayalam}}
\flushright{\begin{Arabic}
\quranayah[6][20]
\end{Arabic}}
\flushleft{\begin{malayalam}
നാം വേദഗ്രന്ഥം നല്‍കിയിട്ടുള്ളവര്‍ സ്വന്തം മക്കളെ അറിയുന്നത് പോലെ അത് അറിയുന്നുണ്ട്‌. സ്വദേഹങ്ങളെ നഷ്ടത്തിലാക്കിയവരത്രെ അവര്‍. അതിനാല്‍ അവര്‍ വിശ്വസിക്കുകയില്ല.
\end{malayalam}}
\flushright{\begin{Arabic}
\quranayah[6][21]
\end{Arabic}}
\flushleft{\begin{malayalam}
അല്ലാഹുവിന്‍റെ പേരില്‍ കള്ളം കെട്ടിച്ചമയ്ക്കുകയോ, അവന്‍റെ ദൃഷ്ടാന്തങ്ങള്‍ തള്ളിക്കളയുകയോ ചെയ്തവനേക്കാള്‍ കടുത്ത അക്രമി ആരുണ്ട്‌? അക്രമികള്‍ വിജയം വരിക്കുകയില്ല; തീര്‍ച്ച.
\end{malayalam}}
\flushright{\begin{Arabic}
\quranayah[6][22]
\end{Arabic}}
\flushleft{\begin{malayalam}
നാം അവരെ മുഴുവന്‍ ഒരുമിച്ചുകൂട്ടുകയും, പിന്നീട് ബഹുദൈവാരാധകരോട് നിങ്ങള്‍ ജല്‍പിച്ച് കൊണ്ടിരുന്ന നിങ്ങളുടെ വകയായുള്ള ആ പങ്കാളികള്‍ എവിടെയെന്ന് നാം ചോദിക്കുകയും ചെയ്യുന്ന ദിവസം (ഓര്‍ക്കുക.)
\end{malayalam}}
\flushright{\begin{Arabic}
\quranayah[6][23]
\end{Arabic}}
\flushleft{\begin{malayalam}
അനന്തരം, അവരുടെ ഗതികേട് ഞങ്ങളുടെ രക്ഷിതാവായ അല്ലാഹുവിനെത്തന്നെയാണ സത്യം, ഞങ്ങള്‍ പങ്കുചേര്‍ക്കുന്നവരായിരുന്നില്ല എന്ന് പറയുന്നതല്ലാതെ മറ്റൊന്നുമായിരിക്കില്ല.
\end{malayalam}}
\flushright{\begin{Arabic}
\quranayah[6][24]
\end{Arabic}}
\flushleft{\begin{malayalam}
നോക്കൂ; അവര്‍ സ്വന്തം പേരില്‍ തന്നെ എങ്ങനെ കള്ളം പറഞ്ഞു എന്ന്‌. അവര്‍ എന്തൊന്ന് കെട്ടിച്ചമച്ചിരുന്നുവോ അതവര്‍ക്ക് ഉപകരിക്കാതെ പോയിരിക്കുന്നു.
\end{malayalam}}
\flushright{\begin{Arabic}
\quranayah[6][25]
\end{Arabic}}
\flushleft{\begin{malayalam}
നീ പറയുന്നത് ശ്രദ്ധിച്ച് കേള്‍ക്കുന്ന ചിലരും അവരുടെ കൂട്ടത്തിലുണ്ട്‌. എന്നാല്‍ അത് അവര്‍ ഗ്രഹിക്കാത്ത വിധം അവരുടെ ഹൃദയങ്ങളിന്‍മേല്‍ നാം മൂടികള്‍ ഇടുകയും, അവരുടെ കാതുകളില്‍ അടപ്പ് വെക്കുകയും ചെയ്തിരിക്കുന്നു. എന്തെല്ലാം ദൃഷ്ടാന്തങ്ങള്‍ കണ്ടാലും അവരതില്‍ വിശ്വസിക്കുകയില്ല. അങ്ങനെ അവര്‍ നിന്‍റെ അടുക്കല്‍ നിന്നോട് തര്‍ക്കിക്കുവാനായി വന്നാല്‍ ആ സത്യനിഷേധികള്‍ പറയും; ഇത് പൂര്‍വ്വികന്‍മാരുടെ കെട്ടുകഥകളല്ലാതെ മറ്റൊന്നുമല്ല എന്ന്‌.
\end{malayalam}}
\flushright{\begin{Arabic}
\quranayah[6][26]
\end{Arabic}}
\flushleft{\begin{malayalam}
അവര്‍ അതില്‍ നിന്ന് മറ്റുള്ളവരെ തടയുകയും, അതില്‍ നിന്ന് (സ്വയം) അകന്നു നില്‍ക്കുകയും ചെയ്യുന്നു. (വാസ്തവത്തില്‍) അവര്‍ സ്വദേഹങ്ങള്‍ക്ക് തന്നെ നാശമുണ്ടാക്കുക മാത്രമാണ് ചെയ്യുന്നത്‌. അവര്‍ (അതിനെപ്പറ്റി) ബോധവാന്‍മാരാകുന്നില്ല.
\end{malayalam}}
\flushright{\begin{Arabic}
\quranayah[6][27]
\end{Arabic}}
\flushleft{\begin{malayalam}
അവര്‍ നരകത്തിങ്കല്‍ നിര്‍ത്തപ്പെടുന്ന രംഗം നീ കണ്ടിരുന്നുവെങ്കില്‍! അപ്പോള്‍ അവര്‍ പറയും: ഞങ്ങള്‍ (ഇഹലോകത്തേക്ക്‌) ഒന്നു തിരിച്ചയക്കപ്പെട്ടിരുന്നുവെങ്കില്‍ എത്ര നന്നായിരുന്നേനെ. എങ്കില്‍ ഞങ്ങള്‍ ഞങ്ങളുടെ രക്ഷിതാവിന്‍റെ ദൃഷ്ടാന്തങ്ങള്‍ തള്ളിക്കളയാതിരിക്കുകയും, ഞങ്ങള്‍ സത്യവിശ്വാസികളുടെ കൂട്ടത്തിലാവുകയും ചെയ്യുമായിരുന്നു.
\end{malayalam}}
\flushright{\begin{Arabic}
\quranayah[6][28]
\end{Arabic}}
\flushleft{\begin{malayalam}
അല്ല; അവര്‍ മുമ്പ് മറച്ചുവെച്ചുകൊണ്ടിരുന്നത് (ഇപ്പോള്‍) അവര്‍ക്ക് വെളിപ്പെട്ടിരിക്കുന്നു. തിരിച്ചയക്കപ്പെട്ടാല്‍ തന്നെയും അവര്‍ എന്തില്‍ നിന്നൊക്കെ വിലക്കപ്പെട്ടുവോ അതിലേക്ക് തന്നെ അവര്‍ മടങ്ങിപ്പോകുന്നതാണ്‌. തീര്‍ച്ചയായും അവര്‍ കള്ളം പറയുന്നവരാകുന്നു.
\end{malayalam}}
\flushright{\begin{Arabic}
\quranayah[6][29]
\end{Arabic}}
\flushleft{\begin{malayalam}
അവര്‍ പറഞ്ഞിരുന്നു; ഞങ്ങളുടെ ഐഹികജീവിതമല്ലാതെ യാതൊന്നുമില്ല. ഞങ്ങള്‍ ഉയിര്‍ത്തെഴുന്നേല്‍പിക്കപ്പെടുന്നവരുമല്ല എന്ന്‌.
\end{malayalam}}
\flushright{\begin{Arabic}
\quranayah[6][30]
\end{Arabic}}
\flushleft{\begin{malayalam}
അവര്‍ അവരുടെ രക്ഷിതാവിന്‍റെ മുമ്പില്‍ നിര്‍ത്തപ്പെടുന്ന രംഗം നീ കണ്ടിരുന്നെങ്കില്‍! അവന്‍ ചോദിക്കും.: ഇത് യഥാര്‍ത്ഥം തന്നെയല്ലേ? അവര്‍ പറയും: അതെ; ഞങ്ങളുടെ രക്ഷിതാവിനെ തന്നെയാണ സത്യം. അവന്‍ പറയും: എന്നാല്‍ നിങ്ങള്‍ അവിശ്വസിച്ചു കൊണ്ടിരുന്നത് നിമിത്തം ശിക്ഷ ആസ്വദിച്ച് കൊള്ളുക.
\end{malayalam}}
\flushright{\begin{Arabic}
\quranayah[6][31]
\end{Arabic}}
\flushleft{\begin{malayalam}
അല്ലാഹുവുമായുള്ള കൂടിക്കാഴ്ചയെ നിഷേധിച്ചു തള്ളിയവര്‍ തീര്‍ച്ചയായും നഷ്ടത്തില്‍ പെട്ടിരിക്കുന്നു. അങ്ങനെ പെട്ടെന്ന് ആ സമയം വന്നെത്തുമ്പോള്‍ അവര്‍ പറയും: ഞങ്ങള്‍ ഇത് സംബന്ധിച്ച കാര്യത്തില്‍ വീഴ്ച വരുത്തിയതിനാല്‍ ഹോ! ഞങ്ങള്‍ക്ക് കഷ്ടം! അവര്‍ അവരുടെ പാപഭാരങ്ങള്‍ അവരുടെ മുതുകുകളില്‍ വഹിക്കുന്നുണ്ടായിരിക്കും. അവര്‍ പേറുന്ന ഭാരം എത്രയോ ചീത്ത!
\end{malayalam}}
\flushright{\begin{Arabic}
\quranayah[6][32]
\end{Arabic}}
\flushleft{\begin{malayalam}
ഐഹികജീവിതമെന്നത് കളിയും വിനോദവുമല്ലാതെ മറ്റൊന്നുമല്ല. പാരത്രിക ലോകമാണ് സൂക്ഷ്മത പാലിക്കുന്നവര്‍ക്ക് ഉത്തമമായിട്ടുള്ളത്‌. നിങ്ങളെന്താണ് ചിന്തിക്കാത്തത്‌?
\end{malayalam}}
\flushright{\begin{Arabic}
\quranayah[6][33]
\end{Arabic}}
\flushleft{\begin{malayalam}
(നബിയേ,) അവര്‍ പറയുന്നത് നിനക്ക് വ്യസനമുണ്ടാക്കുന്നുണ്ട് എന്ന് തീര്‍ച്ചയായും നമുക്ക് അറിയാം. എന്നാല്‍ (യഥാര്‍ത്ഥത്തില്‍) നിന്നെയല്ല അവര്‍ നിഷേധിച്ചു തള്ളുന്നത്‌, പ്രത്യുത,അല്ലാഹുവിന്‍റെ ദൃഷ്ടാന്തങ്ങളെയാണ് ആ അക്രമികള്‍ നിഷേധിക്കുന്നത്‌.
\end{malayalam}}
\flushright{\begin{Arabic}
\quranayah[6][34]
\end{Arabic}}
\flushleft{\begin{malayalam}
നിനക്ക് മുമ്പും ദൂതന്‍മാര്‍ നിഷേധിക്കപ്പെട്ടിട്ടുണ്ട്‌. എന്നിട്ട് തങ്ങള്‍ നിഷേധിക്കപ്പെടുകയും, മര്‍ദ്ദിക്കപ്പെടുകയും ചെയ്തത് നമ്മുടെ സഹായം അവര്‍ക്ക് വന്നെത്തുന്നത് വരെ അവര്‍ സഹിച്ചു. അല്ലാഹുവിന്‍റെ വചനങ്ങള്‍ക്ക് (കല്‍പനകള്‍ക്ക്‌) മാറ്റം വരുത്താന്‍ ആരും തന്നെയില്ല. ദൈവദൂതന്‍മാരുടെ വൃത്താന്തങ്ങളില്‍ ചിലത് നിനക്ക് വന്നുകിട്ടിയിട്ടുണ്ടല്ലോ.
\end{malayalam}}
\flushright{\begin{Arabic}
\quranayah[6][35]
\end{Arabic}}
\flushleft{\begin{malayalam}
അവര്‍ പിന്തിരിഞ്ഞ് കളയുന്നത് നിനക്ക് ദുസ്സഹമായി തോന്നുന്നുവെങ്കില്‍ ഭൂമിയില്‍ (ഇറങ്ങിപ്പോകുവാന്‍) ഒരു തുരങ്കമോ, ആകാശത്ത് (കയറിപ്പോകുവാന്‍) ഒരു കോണിയോ തേടിപ്പിടിച്ചിട്ട് അവര്‍ക്കൊരു ദൃഷ്ടാന്തം കൊണ്ടു വന്നുകൊടുക്കാന്‍ നിനക്ക് സാധിക്കുന്ന പക്ഷം (അതങ്ങ് ചെയ്തേക്കുക.) അല്ലാഹു ഉദ്ദേശിച്ചിരുന്നുവെങ്കില്‍ അവരെയൊക്കെ അവന്‍ സന്‍മാര്‍ഗത്തില്‍ ഒരുമിച്ചുകൂട്ടുക തന്നെ ചെയ്യുമായിരുന്നു. അതിനാല്‍ നീ ഒരിക്കലും അവിവേകികളില്‍ പെട്ടുപോകരുത്‌.
\end{malayalam}}
\flushright{\begin{Arabic}
\quranayah[6][36]
\end{Arabic}}
\flushleft{\begin{malayalam}
കേള്‍ക്കുന്നവര്‍ മാത്രമേ ഉത്തരം നല്‍കുകയുള്ളൂ. മരിച്ചവരെയാകട്ടെ അല്ലാഹു ഉയിര്‍ത്തെഴുന്നേല്‍പിക്കുന്നതാണ്‌. എന്നിട്ട് അവങ്കലേക്ക് അവര്‍ മടക്കപ്പെടുകയും ചെയ്യും.
\end{malayalam}}
\flushright{\begin{Arabic}
\quranayah[6][37]
\end{Arabic}}
\flushleft{\begin{malayalam}
ഇവന്‍റെ മേല്‍ ഇവന്‍റെ രക്ഷിതാവിങ്കല്‍ നിന്ന് ഏതെങ്കിലും ദൃഷ്ടാന്തം ഇറക്കപ്പെടാത്തതെന്താണ് എന്നവര്‍ ചോദിക്കുന്നു. പറയുക: തീര്‍ച്ചയായും അല്ലാഹു ദൃഷ്ടാന്തം ഇറക്കുവാന്‍ കഴിവുള്ളവനാണ്‌. പക്ഷെ, അവരില്‍ അധികപേരും (യാഥാര്‍ത്ഥ്യം) അറിയുന്നില്ല.
\end{malayalam}}
\flushright{\begin{Arabic}
\quranayah[6][38]
\end{Arabic}}
\flushleft{\begin{malayalam}
ഭൂമിയിലുള്ള ഏതൊരു ജന്തുവും, രണ്ട് ചിറകുകള്‍ കൊണ്ട് പറക്കുന്ന ഏതൊരു പക്ഷിയും നിങ്ങളെപ്പോലെയുള്ള ചില സമൂഹങ്ങള്‍ മാത്രമാകുന്നു. ഗ്രന്ഥത്തില്‍ നാം യാതൊന്നും വീഴ്ച വരുത്തിയിട്ടില്ല. പിന്നീട് തങ്ങളുടെ രക്ഷിതാവിങ്കലേക്ക് അവര്‍ ഒരുമിച്ചുകൂട്ടപ്പെടുന്നതാണ്‌.
\end{malayalam}}
\flushright{\begin{Arabic}
\quranayah[6][39]
\end{Arabic}}
\flushleft{\begin{malayalam}
നമ്മുടെ ദൃഷ്ടാന്തങ്ങളെ നിഷേധിച്ചവര്‍ ബധിരരും ഊമകളും ഇരുട്ടുകളില്‍ അകപ്പെട്ടവരുമത്രെ. താന്‍ ഉദ്ദേശിക്കുന്നവരെ അല്ലാഹു വഴികേടിലാക്കും. താന്‍ ഉദ്ദേശിക്കുന്നവരെ അവന്‍ നേര്‍മാര്‍ഗത്തിലാക്കുകയും ചെയ്യും.
\end{malayalam}}
\flushright{\begin{Arabic}
\quranayah[6][40]
\end{Arabic}}
\flushleft{\begin{malayalam}
(നബിയേ,) പറയുക: നിങ്ങളൊന്ന് പറഞ്ഞുതരൂ; അല്ലാഹുവിന്‍റെ ശിക്ഷ നിങ്ങള്‍ക്ക് വന്നുഭവിച്ചാല്‍, അല്ലെങ്കില്‍ അന്ത്യസമയം നിങ്ങള്‍ക്ക് വന്നെത്തിയാല്‍ അല്ലാഹുവല്ലാത്തവരെ നിങ്ങള്‍ വിളിച്ച് പ്രാര്‍ത്ഥിക്കുമോ ? (പറയൂ;) നിങ്ങള്‍ സത്യസന്ധരാണെങ്കില്‍.
\end{malayalam}}
\flushright{\begin{Arabic}
\quranayah[6][41]
\end{Arabic}}
\flushleft{\begin{malayalam}
ഇല്ല, അവനെ മാത്രമേ നിങ്ങള്‍ വിളിച്ച് പ്രാര്‍ത്ഥിക്കുകയുള്ളൂ. അപ്പോള്‍ അവന്‍ ഉദ്ദേശിക്കുന്ന പക്ഷം ഏതൊരു വിഷമത്തിന്‍റെ പേരില്‍ നിങ്ങളവനെ വിളിച്ച് പ്രാര്‍ത്ഥിക്കുന്നുവോ അതവന്‍ ദൂരീകരിച്ച് തരുന്നതാണ്‌. നിങ്ങള്‍ (അവനോട്‌) പങ്കുചേര്‍ക്കുന്നവയെ നിങ്ങള്‍ (അപ്പോള്‍) മറന്നുകളയും.
\end{malayalam}}
\flushright{\begin{Arabic}
\quranayah[6][42]
\end{Arabic}}
\flushleft{\begin{malayalam}
നിനക്ക് മുമ്പ് നാം പല സമൂഹങ്ങളിലേക്കും (ദൂതന്‍മാരെ) അയച്ചിട്ടുണ്ട്‌. അനന്തരം അവരെ (ആ സമൂഹങ്ങളെ) കഷ്ടപ്പാടും ദുരിതവും കൊണ്ട് നാം പിടികൂടി; അവര്‍ വിനയശീലരായിത്തീരുവാന്‍ വേണ്ടി.
\end{malayalam}}
\flushright{\begin{Arabic}
\quranayah[6][43]
\end{Arabic}}
\flushleft{\begin{malayalam}
അങ്ങനെ അവര്‍ക്ക് നമ്മുടെ ശിക്ഷ വന്നെത്തിയപ്പോള്‍ അവരെന്താണ് താഴ്മയുള്ളവരാകാതിരുന്നത് ? എന്നാല്‍ അവരുടെ ഹൃദയങ്ങള്‍ കടുത്തുപോകുകയാണുണ്ടായത്‌. അവര്‍ ചെയ്ത് കൊണ്ടിരുന്നത് പിശാച് അവര്‍ക്ക് ഭംഗിയായി തോന്നിക്കുകയും ചെയ്തു.
\end{malayalam}}
\flushright{\begin{Arabic}
\quranayah[6][44]
\end{Arabic}}
\flushleft{\begin{malayalam}
അങ്ങനെ അവരോട് ഉല്‍ബോധിപ്പിക്കപ്പെട്ട കാര്യങ്ങള്‍ അവര്‍ മറന്നുകളഞ്ഞപ്പോള്‍ എല്ലാ കാര്യങ്ങളുടെയും വാതിലുകള്‍ നാം അവര്‍ക്ക് തുറന്നുകൊടുത്തു. അങ്ങനെ അവര്‍ക്ക് നല്‍കപ്പെട്ടതില്‍ അവര്‍ ആഹ്ലാദം കൊണ്ടപ്പോള്‍ പെട്ടെന്ന് നാം അവരെ പിടികൂടി. അപ്പോള്‍ അവരതാ നിരാശപ്പെട്ടവരായിത്തീരുന്നു.
\end{malayalam}}
\flushright{\begin{Arabic}
\quranayah[6][45]
\end{Arabic}}
\flushleft{\begin{malayalam}
അങ്ങനെ ആ അക്രമികളായ ജനത നിശ്ശേഷം നശിപ്പിക്കപ്പെട്ടു. ലോകരക്ഷിതാവായ അല്ലാഹുവിന്ന് സ്തുതി.
\end{malayalam}}
\flushright{\begin{Arabic}
\quranayah[6][46]
\end{Arabic}}
\flushleft{\begin{malayalam}
(നബിയേ,) പറയുക: നിങ്ങള്‍ ചിന്തിച്ച് നോക്കിയോ? അല്ലാഹു നിങ്ങളുടെ കേള്‍വിയും കാഴ്ചകളും പിടിച്ചെടുക്കുകയും, നിങ്ങളുടെ ഹൃദയങ്ങളിന്‍മേല്‍ അവന്‍ മുദ്രവെക്കുകയും ചെയ്യുന്ന പക്ഷം അല്ലാഹുവല്ലാതെ ഏതൊരു ദൈവമാണ് നിങ്ങള്‍ക്കത് കൊണ്ടുവന്ന് തരാനുള്ളത്‌? നോക്കൂ: ഏതെല്ലാം വിധത്തില്‍ നാം തെളിവുകള്‍ വിവരിച്ചുകൊടുക്കുന്നു. എന്നിട്ടും അവര്‍ പിന്തിരിഞ്ഞ് കളയുന്നു.
\end{malayalam}}
\flushright{\begin{Arabic}
\quranayah[6][47]
\end{Arabic}}
\flushleft{\begin{malayalam}
(നബിയേ,) പറയുക: നിങ്ങളൊന്ന് പറഞ്ഞുതരൂ; നിങ്ങള്‍ക്ക് അവിചാരിതമായിട്ടോ പ്രത്യക്ഷമായിട്ടോ അല്ലാഹുവിന്‍റെ ശിക്ഷ വന്നെത്തുന്ന പക്ഷം അക്രമികളായ ജനവിഭാഗമല്ലാതെ നശിപ്പിക്കപ്പെടുമോ ?
\end{malayalam}}
\flushright{\begin{Arabic}
\quranayah[6][48]
\end{Arabic}}
\flushleft{\begin{malayalam}
സന്തോഷവാര്‍ത്ത അറിയിക്കുന്നവരും, താക്കീത് നല്‍കുന്നവരും ആയിട്ടല്ലാതെ നാം ദൂതന്‍മാരെ അയക്കുന്നില്ല. എന്നിട്ട് ആര്‍ വിശ്വസിക്കുകയും, നിലപാട് നന്നാക്കിത്തീര്‍ക്കുകയും ചെയ്തുവോ അവര്‍ക്ക് യാതൊന്നും ഭയപ്പെടാനില്ല.അവര്‍ ദുഃഖിക്കേണ്ടി വരികയുമില്ല.
\end{malayalam}}
\flushright{\begin{Arabic}
\quranayah[6][49]
\end{Arabic}}
\flushleft{\begin{malayalam}
എന്നാല്‍ നമ്മുടെ ദൃഷ്ടാന്തങ്ങളെ നിഷേധിച്ച് കളഞ്ഞവരാരോ അവര്‍ക്ക് ശിക്ഷ ബാധിക്കുന്നതാണ്‌; അവര്‍ ധിക്കാരികളായതിന്‍റെ ഫലമായിട്ട്‌.
\end{malayalam}}
\flushright{\begin{Arabic}
\quranayah[6][50]
\end{Arabic}}
\flushleft{\begin{malayalam}
പറയുക: അല്ലാഹുവിന്‍റെ ഖജനാവുകള്‍ എന്‍റെ പക്കലുണ്ടെന്ന് ഞാന്‍ നിങ്ങളോട് പറയുന്നില്ല. അദൃശ്യകാര്യം ഞാന്‍ അറിയുകയുമില്ല. ഞാന്‍ ഒരു മലക്കാണ് എന്നും നിങ്ങളോട് പറയുന്നില്ല. എനിക്ക് ബോധനം നല്‍കപ്പെടുന്നതിനെയല്ലാതെ ഞാന്‍ പിന്തുടരുന്നില്ല. പറയുക: അന്ധനും കാഴ്ചയുള്ളവനും സമമാകുമോ ? നിങ്ങളെന്താണ് ചിന്തിച്ച് നോക്കാത്തത്‌?
\end{malayalam}}
\flushright{\begin{Arabic}
\quranayah[6][51]
\end{Arabic}}
\flushleft{\begin{malayalam}
തങ്ങളുടെ രക്ഷിതാവിങ്കലേക്ക് ഒരുമിച്ചുകൂട്ടപ്പെടുമെന്ന് ഭയപ്പെടുന്നവര്‍ക്ക് ഇത് (ദിവ്യബോധനം) മുഖേന നീ താക്കീത് നല്‍കുക. അവന്നു പുറമെ യാതൊരു രക്ഷാധികാരിയും ശുപാര്‍ശകനും അവര്‍ക്കില്ല. അവര്‍ സൂക്ഷ്മത പാലിക്കുന്നവരായേക്കാം.
\end{malayalam}}
\flushright{\begin{Arabic}
\quranayah[6][52]
\end{Arabic}}
\flushleft{\begin{malayalam}
തങ്ങളുടെ രക്ഷിതാവിന്‍റെ അനുഗ്രഹം ലക്ഷ്യമാക്കിക്കൊണ്ട് രാവിലെയും വൈകുന്നേരവും അവനോട് പ്രാര്‍ത്ഥിച്ചു കൊണ്ടിരിക്കുന്നവരെ നീ ആട്ടിയകറ്റരുത്‌. അവരുടെ കണക്ക് നോക്കേണ്ട യാതൊരു ബാധ്യതയും നിനക്കില്ല. നിന്‍റെ കണക്ക് നോക്കേണ്ട യാതൊരു ബാധ്യതയും അവര്‍ക്കുമില്ല. എങ്കിലല്ലേ നീ അവരെ ആട്ടിയകറ്റേണ്ടി വരുന്നത് ? അങ്ങനെ ചെയ്യുന്ന പക്ഷം നീ അക്രമികളില്‍ പെട്ടവനായിരിക്കും.
\end{malayalam}}
\flushright{\begin{Arabic}
\quranayah[6][53]
\end{Arabic}}
\flushleft{\begin{malayalam}
അപ്രകാരം അവരില്‍ ചിലരെ മറ്റു ചിലരെക്കൊണ്ട് നാം പരീക്ഷണവിധേയരാക്കിയിരിക്കുന്നു. ഞങ്ങളുടെ ഇടയില്‍ നിന്ന് അല്ലാഹു അനുഗ്രഹിച്ചിട്ടുള്ളത് ഇക്കൂട്ടരെയാണോ എന്ന് അവര്‍ പറയുവാന്‍ വേണ്ടിയത്രെ അത്‌. നന്ദികാണിക്കുന്നവരെപ്പറ്റി അല്ലാഹു നല്ലവണ്ണം അറിയുന്നവനല്ലയോ ?
\end{malayalam}}
\flushright{\begin{Arabic}
\quranayah[6][54]
\end{Arabic}}
\flushleft{\begin{malayalam}
നമ്മുടെ ദൃഷ്ടാന്തങ്ങളില്‍ വിശ്വസിക്കുന്നവര്‍ നിന്‍റെ അടുക്കല്‍ വന്നാല്‍ നീ പറയുക: നിങ്ങള്‍ക്ക് സമാധാനമുണ്ടായിരിക്കട്ടെ. നിങ്ങളുടെ രക്ഷിതാവ് കാരുണ്യത്തെ തന്‍റെ മേല്‍ (ബാധ്യതയായി) നിശ്ചയിച്ചിരിക്കുന്നു. അതായത് നിങ്ങളില്‍ നിന്നാരെങ്കിലും അവിവേകത്താല്‍ വല്ല തിന്‍മയും ചെയ്തു പോകുകയും എന്നിട്ടതിന് ശേഷം പശ്ചാത്തപിക്കുകയും, നിലപാട് നന്നാക്കിത്തീര്‍ക്കുകയും ചെയ്യുന്ന പക്ഷം അവന്‍ ഏറെ പൊറുക്കുന്നവനും കരുണാനിധിയുമാകുന്നു.
\end{malayalam}}
\flushright{\begin{Arabic}
\quranayah[6][55]
\end{Arabic}}
\flushleft{\begin{malayalam}
അപ്രകാരം നാം തെളിവുകള്‍ വിശദീകരിച്ച് തരുന്നു. കുറ്റവാളികളുടെ മാര്‍ഗം വ്യക്തമായി വേര്‍ തിരിഞ്ഞ് കാണുവാന്‍ വേണ്ടിയുമാകുന്നു അത്‌.
\end{malayalam}}
\flushright{\begin{Arabic}
\quranayah[6][56]
\end{Arabic}}
\flushleft{\begin{malayalam}
(നബിയേ,) പറയുക: അല്ലാഹുവിന് പുറമെ നിങ്ങള്‍ വിളിച്ചു പ്രാര്‍ത്ഥിക്കുന്നവരെ ആരാധിക്കുന്നതില്‍ നിന്ന് തീര്‍ച്ചയായും ഞാന്‍ വിലക്കപ്പെട്ടിരിക്കുന്നു. പറയുക: നിങ്ങളുടെ തന്നിഷ്ടങ്ങളെ ഞാന്‍ പിന്തുടരുകയില്ല. അങ്ങനെ ചെയ്യുന്ന പക്ഷം ഞാന്‍ പിഴച്ചു കഴിഞ്ഞു; സന്‍മാര്‍ഗം പ്രാപിച്ചവരുടെ കൂട്ടത്തില്‍ ഞാന്‍ ആയിരിക്കുകയുമില്ല.
\end{malayalam}}
\flushright{\begin{Arabic}
\quranayah[6][57]
\end{Arabic}}
\flushleft{\begin{malayalam}
പറയുക: തീര്‍ച്ചയായും എന്‍റെ രക്ഷിതാവിങ്കല്‍ നിന്നുള്ള വ്യക്തമായ പ്രമാണത്തിന്‍മേലാണ് ഞാന്‍. നിങ്ങളാകട്ടെ, അതിനെ നിഷേധിച്ച് കളഞ്ഞിരിക്കുന്നു. നിങ്ങള്‍ എന്തൊന്നിന് വേണ്ടി തിടുക്കം കൂട്ടുന്നുവോ അത് (ശിക്ഷ) എന്‍റെ പക്കലില്ല. (അതിന്‍റെ) തീരുമാനാധികാരം അല്ലാഹുവിന് മാത്രമാണ്‌. അവന്‍ സത്യം വിവരിച്ചുതരുന്നു. അവനത്രെ തീര്‍പ്പുകല്‍പിക്കുന്നവരില്‍ ഉത്തമന്‍.
\end{malayalam}}
\flushright{\begin{Arabic}
\quranayah[6][58]
\end{Arabic}}
\flushleft{\begin{malayalam}
പറയുക: നിങ്ങള്‍ തിടുക്കം കൂട്ടിക്കൊണ്ടിരിക്കുന്ന കാര്യം എന്‍റെ പക്കലുണ്ടായിരുന്നുവെങ്കില്‍ എന്‍റെയും നിങ്ങളുടെയും ഇടയില്‍ കാര്യം തീരുമാനിക്കപ്പെട്ടുകഴിഞ്ഞിരുന്നു. അല്ലാഹു അക്രമികളെപ്പറ്റി നല്ലവണ്ണം അറിയുന്നവനത്രെ.
\end{malayalam}}
\flushright{\begin{Arabic}
\quranayah[6][59]
\end{Arabic}}
\flushleft{\begin{malayalam}
അവന്‍റെ പക്കലാകുന്നു അദൃശ്യകാര്യത്തിന്‍റെ ഖജനാവുകള്‍. അവനല്ലാതെ അവ അറിയുകയില്ല. കരയിലും കടലിലുമുള്ളത് അവന്‍ അറിയുന്നു. അവനറിയാതെ ഒരു ഇല പോലും വീഴുന്നില്ല. ഭൂമിയിലെ ഇരുട്ടുകള്‍ക്കുള്ളിലിരിക്കുന്ന ഒരു ധാന്യമണിയാകട്ടെ, പച്ചയോ, ഉണങ്ങിയതോ ആയ ഏതൊരു വസ്തുവാകട്ടെ, വ്യക്തമായ ഒരു രേഖയില്‍ എഴുതപ്പെട്ടതായിട്ടല്ലാതെ ഉണ്ടാവില്ല.
\end{malayalam}}
\flushright{\begin{Arabic}
\quranayah[6][60]
\end{Arabic}}
\flushleft{\begin{malayalam}
അവനത്രെ രാത്രിയില്‍ (ഉറങ്ങുമ്പോള്‍) നിങ്ങളെ പൂര്‍ണ്ണമായി ഏറ്റെടുക്കുന്നവന്‍. പകലില്‍ നിങ്ങള്‍ പ്രവര്‍ത്തിച്ചതെല്ലാം അവന്‍ അറിയുകയും ചെയ്യുന്നു. പിന്നീട് നിര്‍ണിതമായ ജീവിതാവധി പൂര്‍ത്തിയാക്കപ്പെടുവാന്‍ വേണ്ടി പകലില്‍ നിങ്ങളെ അവന്‍ എഴുന്നേല്‍പിക്കുന്നു. പിന്നീട് അവങ്കലേക്കാണ് നിങ്ങളുടെ മടക്കം. അനന്തരം നിങ്ങള്‍ ചെയ്ത്കൊണ്ടിരിക്കുന്നതിനെപ്പറ്റിയെല്ലാം അവന്‍ നിങ്ങളെ അറിയിക്കുകയും ചെയ്യും.
\end{malayalam}}
\flushright{\begin{Arabic}
\quranayah[6][61]
\end{Arabic}}
\flushleft{\begin{malayalam}
അവനത്രെ തന്‍റെ ദാസന്‍മാരുടെ മേല്‍ പരമാധികാരമുള്ളവന്‍. നിങ്ങളുടെ മേല്‍നോട്ടത്തിനായി അവന്‍ കാവല്‍ക്കാരെ അയക്കുകയും ചെയ്യുന്നു. അങ്ങനെ അവരില്‍ ഒരാള്‍ക്ക് മരണം വന്നെത്തുമ്പോള്‍ നമ്മുടെ ദൂതന്‍മാര്‍ (മലക്കുകള്‍) അവനെ പൂര്‍ണ്ണമായി ഏറ്റെടുക്കുന്നു. (അക്കാര്യത്തില്‍) അവര്‍ ഒരു വീഴ്ചയും വരുത്തുകയില്ല.
\end{malayalam}}
\flushright{\begin{Arabic}
\quranayah[6][62]
\end{Arabic}}
\flushleft{\begin{malayalam}
എന്നിട്ട് അവര്‍ യഥാര്‍ത്ഥ രക്ഷാധികാരിയായ അല്ലാഹുവിലേക്ക് തിരിച്ചയക്കപ്പെടും. അറിയുക: അവന്നത്രെ തീരുമാനാധികാരം. അവന്‍ അതിവേഗം കണക്ക് നോക്കുന്നവനത്രെ.
\end{malayalam}}
\flushright{\begin{Arabic}
\quranayah[6][63]
\end{Arabic}}
\flushleft{\begin{malayalam}
പറയുക: ഇതില്‍ നിന്ന് (ഈ വിപത്തുകളില്‍ നിന്ന്‌) അല്ലാഹു ഞങ്ങളെ രക്ഷപ്പെടുത്തുകയാണെങ്കില്‍ തീര്‍ച്ചയായും ഞങ്ങള്‍ നന്ദിയുള്ളവരുടെ കൂട്ടത്തില്‍ ആയിക്കൊള്ളാം. എന്ന് പറഞ്ഞുകൊണ്ട് അവനോട് നിങ്ങള്‍ താഴ്മയോടെയും രഹസ്യമായും പ്രാര്‍ത്ഥിക്കുന്ന സമയത്ത് കരയിലെയും കടലിലെയും അന്ധകാരങ്ങളില്‍ നിന്ന് നിങ്ങളെ രക്ഷിക്കുന്നത് ആരാണ്‌?
\end{malayalam}}
\flushright{\begin{Arabic}
\quranayah[6][64]
\end{Arabic}}
\flushleft{\begin{malayalam}
പറയുക: അല്ലാഹുവാണ് അവയില്‍നിന്നും മറ്റെല്ലാ ദുരിതങ്ങളില്‍ നിന്നും നിങ്ങളെ രക്ഷിക്കുന്നത്‌. എന്നിട്ടും നിങ്ങളവനോട് പങ്കുചേര്‍ക്കുന്നുവല്ലോ.
\end{malayalam}}
\flushright{\begin{Arabic}
\quranayah[6][65]
\end{Arabic}}
\flushleft{\begin{malayalam}
പറയുക: നിങ്ങളുടെ മുകള്‍ ഭാഗത്ത് നിന്നോ, നിങ്ങളുടെ കാലുകളുടെ ചുവട്ടില്‍ നിന്നോ നിങ്ങളുടെ മേല്‍ ശിക്ഷ അയക്കുവാന്‍, അല്ലെങ്കില്‍ നിങ്ങളെ ഭിന്നകക്ഷികളാക്കി ആശയക്കുഴപ്പത്തിലാക്കുകയും, നിങ്ങളില്‍ ചിലര്‍ക്ക് മറ്റു ചിലരുടെ പീഡനം അനുഭവിപ്പിക്കുകയും ചെയ്യാന്‍ കഴിവുള്ളവനത്രെ അവന്‍. നോക്കൂ; അവര്‍ ഗ്രഹിക്കുവാന്‍ വേണ്ടി നാം തെളിവുകള്‍ വിവിധ രൂപത്തില്‍ വിവരിച്ചുകൊടുക്കുന്നത് എങ്ങനെയാണെന്ന്‌!
\end{malayalam}}
\flushright{\begin{Arabic}
\quranayah[6][66]
\end{Arabic}}
\flushleft{\begin{malayalam}
(നബിയേ,) നിന്‍റെ ജനത ഇത് സത്യമായിരിക്കെ ഇതിനെ നിഷേധിച്ച് കളഞ്ഞിരിക്കുന്നു. പറയുക: ഞാന്‍ നിങ്ങളുടെ മേല്‍ (ഉത്തരവാദിത്തം) ഏല്‍പിക്കപ്പെട്ടവനൊന്നുമല്ല.
\end{malayalam}}
\flushright{\begin{Arabic}
\quranayah[6][67]
\end{Arabic}}
\flushleft{\begin{malayalam}
ഓരോ വൃത്താന്തത്തിനും അത് (സത്യമായി) പുലരുന്ന ഒരു സന്ദര്‍ഭമുണ്ട്‌. വഴിയെ നിങ്ങള്‍ അതറിഞ്ഞു കൊള്ളും.
\end{malayalam}}
\flushright{\begin{Arabic}
\quranayah[6][68]
\end{Arabic}}
\flushleft{\begin{malayalam}
നമ്മുടെ ദൃഷ്ടാന്തങ്ങളെ അപഹസിക്കുന്നതില്‍ മുഴുകിയവരെ നീ കണ്ടാല്‍ അവര്‍ മറ്റു വല്ല വര്‍ത്തമാനത്തിലും പ്രവേശിക്കുന്നത് വരെ നീ അവരില്‍ നിന്ന് തിരിഞ്ഞുകളയുക. ഇനി വല്ലപ്പോഴും നിന്നെ പിശാച് മറപ്പിച്ച് കളയുന്ന പക്ഷം ഓര്‍മ വന്നതിന് ശേഷം അക്രമികളായ ആ ആളുകളുടെ കൂടെ നീ ഇരിക്കരുത്‌.
\end{malayalam}}
\flushright{\begin{Arabic}
\quranayah[6][69]
\end{Arabic}}
\flushleft{\begin{malayalam}
സൂക്ഷ്മത പാലിക്കുന്നവര്‍ക്ക് അവരുടെ (അക്രമികളുടെ) കണക്ക് നോക്കേണ്ട യാതൊരു ബാധ്യതയുമില്ല. പക്ഷെ, ഓര്‍മിപ്പിക്കേണ്ടതുണ്ട്‌. അവര്‍ സൂക്ഷ്മതയുള്ളവരായേക്കാം.
\end{malayalam}}
\flushright{\begin{Arabic}
\quranayah[6][70]
\end{Arabic}}
\flushleft{\begin{malayalam}
തങ്ങളുടെ മതത്തെ കളിയും വിനോദവുമാക്കിത്തീര്‍ക്കുകയും, ഐഹികജീവിതം കണ്ട് വഞ്ചിതരാകുകയും ചെയ്തിട്ടുള്ളവരെ വിട്ടേക്കുക. ഏതൊരു ആത്മാവും സ്വയം ചെയ്തു വെച്ചതിന്‍റെ ഫലമായി നാശത്തിലേക്ക് തള്ളപ്പെടുമെന്നതിനാല്‍ ഇത് (ഖുര്‍ആന്‍) മുഖേന നീ ഉല്‍ബോധനം നടത്തുക. അല്ലാഹുവിന് പുറമെ ആ ആത്മാവിന് യാതൊരു രക്ഷാധികാരിയും ശുപാര്‍ശകനും ഉണ്ടായിരിക്കുന്നതല്ല. എല്ലാവിധ പ്രായശ്ചിത്തവും നല്‍കിയാലും ആ ആത്മാവില്‍ നിന്നത് സ്വീകരിക്കപ്പെടുകയില്ല. സ്വയം ചെയ്ത് വെച്ചതിന്‍റെ ഫലമായി നാശത്തിലേക്ക് തള്ളപ്പെട്ടവരത്രെ അവര്‍. അവര്‍ നിഷേധിച്ചിരുന്നതിന്‍റെ ഫലമായി ചുട്ടുപൊള്ളുന്ന കുടിനീരും വേദനാജനകമായ ശിക്ഷയുമാണ് അവര്‍ക്കുണ്ടായിരിക്കുക.
\end{malayalam}}
\flushright{\begin{Arabic}
\quranayah[6][71]
\end{Arabic}}
\flushleft{\begin{malayalam}
പറയുക: അല്ലാഹുവിന് പുറമെ ഞങ്ങള്‍ക്ക് ഉപകാരമോ, ഉപദ്രവമോ ചെയ്യാന്‍ കഴിവില്ലാത്തതിനെ ഞങ്ങള്‍ വിളിച്ച് പ്രാര്‍ത്ഥിക്കുകയോ? അല്ലാഹു ഞങ്ങളെ നേര്‍വഴിയിലാക്കിയതിനു ശേഷം ഞങ്ങള്‍ പുറകോട്ട് മടക്കപ്പെടുകയോ? (എന്നിട്ട്‌) പിശാചുക്കള്‍ തട്ടിത്തിരിച്ചു കൊണ്ടുപോയിട്ട് ഭൂമിയില്‍ അന്ധാളിച്ച് കഴിയുന്ന ഒരുത്തനെപ്പോലെ (ഞങ്ങളാവുകയോ?) ഞങ്ങളുടെ അടുത്തേക്ക് വരൂ എന്നു പറഞ്ഞുകൊണ്ട് അവനെ നേര്‍വഴിയിലേക്ക് ക്ഷണിച്ചു കൊണ്ടിരിക്കുന്ന ചില കൂട്ടുകാരുണ്ട് അവന്ന്‌. പറയുക: തീര്‍ച്ചയായും അല്ലാഹുവിന്‍റെ മാര്‍ഗദര്‍ശനമാണ് യഥാര്‍ത്ഥ മാര്‍ഗദര്‍ശനം. ലോകരക്ഷിതാവിന് കീഴ്പെടുവാനാണ് ഞങ്ങള്‍ കല്‍പിക്കപ്പെട്ടിരിക്കുന്നത്‌.
\end{malayalam}}
\flushright{\begin{Arabic}
\quranayah[6][72]
\end{Arabic}}
\flushleft{\begin{malayalam}
നിങ്ങള്‍ നമസ്കാരം മുറപ്രകാരം നിര്‍വഹിക്കണമെന്നും, അവനെ സൂക്ഷിക്കണമെന്നും കല്‍പിക്കപ്പെട്ടിരിക്കുന്നു. അവങ്കലേക്കായിരിക്കും നിങ്ങളെല്ലാം ഒരുമിച്ചുകൂട്ടപ്പെടുന്നത്‌.
\end{malayalam}}
\flushright{\begin{Arabic}
\quranayah[6][73]
\end{Arabic}}
\flushleft{\begin{malayalam}
അവനത്രെ ആകാശങ്ങളും ഭൂമിയും മുറപ്രകാരം സൃഷ്ടിച്ചവന്‍. അവന്‍ ഉണ്ടാകൂ എന്ന പറയുന്ന ദിവസം അതുണ്ടാകുക തന്നെ ചെയ്യുന്നു. അവന്‍റെ വചനം സത്യമാകുന്നു. കാഹളത്തില്‍ ഊതപ്പെടുന്ന ദിവസം അവന്ന് മാത്രമാകുന്നു ആധിപത്യം. അദൃശ്യവും ദൃശ്യവും അറിയുന്നവനാണവന്‍. അവന്‍ യുക്തിമാനും സൂക്ഷ്മജ്ഞാനമുള്ളവനുമത്രെ.
\end{malayalam}}
\flushright{\begin{Arabic}
\quranayah[6][74]
\end{Arabic}}
\flushleft{\begin{malayalam}
ഇബ്രാഹീം തന്‍റെ പിതാവായ ആസറിനോട് പറഞ്ഞ സന്ദര്‍ഭം(ഓര്‍ക്കുക.) ചില ബിംബങ്ങളെയാണോ താങ്കള്‍ ദൈവങ്ങളായി സ്വീകരിക്കുന്നത്‌? തീര്‍ച്ചയായും താങ്കളും താങ്കളുടെ ജനതയും വ്യക്തമായ വഴികേടിലാണെന്ന് ഞാന്‍ കാണുന്നു.
\end{malayalam}}
\flushright{\begin{Arabic}
\quranayah[6][75]
\end{Arabic}}
\flushleft{\begin{malayalam}
അപ്രകാരം ഇബ്രാഹീമിന് നാം ആകാശങ്ങളുടെയും ഭൂമിയുടെയും ആധിപത്യരഹസ്യങ്ങള്‍ കാണിച്ചുകൊടുക്കുന്നു. അദ്ദേഹം ദൃഢബോധ്യമുള്ളവരുടെ കൂട്ടത്തില്‍ ആയിരിക്കാന്‍ വേണ്ടിയും കൂടിയാണത്‌.
\end{malayalam}}
\flushright{\begin{Arabic}
\quranayah[6][76]
\end{Arabic}}
\flushleft{\begin{malayalam}
അങ്ങനെ രാത്രി അദ്ദേഹത്തെ (ഇരുട്ട്കൊണ്ട്‌) മൂടിയപ്പോള്‍ അദ്ദേഹം ഒരു നക്ഷത്രം കണ്ടു. അദ്ദേഹം പറഞ്ഞു: ഇതാ, എന്‍റെ രക്ഷിതാവ്‌! എന്നിട്ട് അത് അസ്തമിച്ചപ്പോള്‍ അദ്ദേഹം പറഞ്ഞു: അസ്തമിച്ച് പോകുന്നവരെ ഞാന്‍ ഇഷ്ടപ്പെടുന്നില്ല.
\end{malayalam}}
\flushright{\begin{Arabic}
\quranayah[6][77]
\end{Arabic}}
\flushleft{\begin{malayalam}
അനന്തരം ചന്ദ്രന്‍ ഉദിച്ചുയരുന്നത് കണ്ടപ്പോള്‍ അദ്ദേഹം പറഞ്ഞു: ഇതാ എന്‍റെ രക്ഷിതാവ്‌! എന്നിട്ട് അതും അസ്തമിച്ചപ്പോള്‍ അദ്ദേഹം പറഞ്ഞു: എന്‍റെ രക്ഷിതാവ് എനിക്ക് നേര്‍വഴി കാണിച്ചുതന്നില്ലെങ്കില്‍ തീര്‍ച്ചയായും ഞാന്‍ വഴിപിഴച്ച ജനവിഭാഗത്തില്‍ പെട്ടവനായിത്തീരും.
\end{malayalam}}
\flushright{\begin{Arabic}
\quranayah[6][78]
\end{Arabic}}
\flushleft{\begin{malayalam}
അനന്തരം സൂര്യന്‍ ഉദിച്ചുയരുന്നതായി കണ്ടപ്പോള്‍ അദ്ദേഹം പറഞ്ഞു: ഇതാ എന്‍റെ രക്ഷിതാവ്‌! ഇതാണ് ഏറ്റവും വലുത്‌!! അങ്ങനെ അതും അസ്തമിച്ചു പോയപ്പോള്‍ അദ്ദേഹം പറഞ്ഞു: എന്‍റെ സമുദായമേ, നിങ്ങള്‍ (ദൈവത്തോട്‌) പങ്കുചേര്‍ക്കുന്നതില്‍ നിന്നെല്ലാം തീര്‍ച്ചയായും ഞാന്‍ ഒഴിവാകുന്നു.
\end{malayalam}}
\flushright{\begin{Arabic}
\quranayah[6][79]
\end{Arabic}}
\flushleft{\begin{malayalam}
തീര്‍ച്ചയായും ഞാന്‍ നേര്‍മാര്‍ഗത്തില്‍ ഉറച്ചുനിന്നു കൊണ്ട് എന്‍റെ മുഖം ആകാശങ്ങളും ഭൂമിയും സൃഷ്ടിച്ചവനിലേക്ക് തിരിച്ചിരിക്കുന്നു. ഞാന്‍ ബഹുദൈവവാദികളില്‍ പെട്ടവനേ അല്ല.
\end{malayalam}}
\flushright{\begin{Arabic}
\quranayah[6][80]
\end{Arabic}}
\flushleft{\begin{malayalam}
അദ്ദേഹത്തിന്‍റെ ജനത അദ്ദേഹവുമായി തര്‍ക്കത്തില്‍ ഏര്‍പെടുകയുണ്ടായി. അദ്ദേഹം പറഞ്ഞു: അല്ലാഹുവിന്‍റെ കാര്യത്തില്‍ നിങ്ങളെന്നോട് തര്‍ക്കിക്കുകയാണോ? അവനാകട്ടെ എന്നെ നേര്‍വഴിയിലാക്കിയിരിക്കുകയാണ്‌. നിങ്ങള്‍ അവനോട് പങ്കുചേര്‍ക്കുന്ന യാതൊന്നിനെയും ഞാന്‍ ഭയപ്പെടുന്നില്ല. എന്‍റെ രക്ഷിതാവ് ഉദ്ദേശിക്കുന്നതെന്തോ അതല്ലാതെ (സംഭവിക്കുകയില്ല.) എന്‍റെ രക്ഷിതാവിന്‍റെ ജ്ഞാനം സര്‍വ്വകാര്യങ്ങളെയും ഉള്‍കൊള്ളാന്‍ മാത്രം വിപുലമായിരിക്കുന്നു. നിങ്ങളെന്താണ് ആലോചിച്ച് നോക്കാത്തത്‌?
\end{malayalam}}
\flushright{\begin{Arabic}
\quranayah[6][81]
\end{Arabic}}
\flushleft{\begin{malayalam}
നിങ്ങള്‍ അല്ലാഹുവിനോട് പങ്കുചേര്‍ത്തതിനെ ഞാന്‍ എങ്ങനെ ഭയപ്പെടും? നിങ്ങളാകട്ടെ, അല്ലാഹു നിങ്ങള്‍ക്ക് യാതൊരു പ്രമാണവും നല്‍കിയിട്ടില്ലാത്ത വസ്തുക്കളെ അവനോട് പങ്ക് ചേര്‍ക്കുന്നതിനെപ്പറ്റി ഭയപ്പെടുന്നുമില്ല. അപ്പോള്‍ രണ്ടു കക്ഷികളില്‍ ആരാണ് നിര്‍ഭയരായിരിക്കാന്‍ കൂടുതല്‍ അര്‍ഹതയുള്ളവര്‍ ? (പറയൂ;) നിങ്ങള്‍ക്കറിയാമെങ്കില്‍.
\end{malayalam}}
\flushright{\begin{Arabic}
\quranayah[6][82]
\end{Arabic}}
\flushleft{\begin{malayalam}
വിശ്വസിക്കുകയും, തങ്ങളുടെ വിശ്വാസത്തില്‍ അന്യായം കൂട്ടികലര്‍ത്താതിരിക്കുകയും ചെയ്തവരാരോ അവര്‍ക്കാണ് നിര്‍ഭയത്വമുള്ളത്‌. അവര്‍ തന്നെയാണ് നേര്‍മാര്‍ഗം പ്രാപിച്ചവര്‍.
\end{malayalam}}
\flushright{\begin{Arabic}
\quranayah[6][83]
\end{Arabic}}
\flushleft{\begin{malayalam}
ഇബ്രാഹീമിന് തന്‍റെ ജനതയ്ക്കെതിരായി നാം നല്‍കിയ ന്യായപ്രമാണമത്രെ അത്‌. നാം ഉദ്ദേശിക്കുന്നവര്‍ക്ക് നാം പദവികള്‍ ഉയര്‍ത്തികൊടുക്കുന്നു. തീര്‍ച്ചയായും നിന്‍റെ രക്ഷിതാവ് യുക്തിമാനും സര്‍വ്വജ്ഞനുമത്രെ.
\end{malayalam}}
\flushright{\begin{Arabic}
\quranayah[6][84]
\end{Arabic}}
\flushleft{\begin{malayalam}
അദ്ദേഹത്തിന് നാം ഇസഹാഖിനെയും യഅ്ഖൂബിനെയും നല്‍കുകയും ചെയ്തു. അവരെയെല്ലാം നാം നേര്‍വഴിയിലാക്കിയിരിക്കുന്നു. അദ്ദേഹത്തിന് മുമ്പ് നൂഹിനെയും നാം നേര്‍വഴിയിലാക്കിയിട്ടുണ്ട്‌. അദ്ദേഹത്തിന്‍റെ സന്താനങ്ങളില്‍ നിന്ന് ദാവൂദിനെയും സുലൈമാനെയും അയ്യൂബിനെയും യൂസുഫിനെയും മൂസായെയും ഹാറൂനെയും (നാം നേര്‍വഴിയിലാക്കി.) അപ്രകാരം സദ്‌വൃത്തര്‍ക്ക് നാം പ്രതിഫലം നല്‍കുന്നു.
\end{malayalam}}
\flushright{\begin{Arabic}
\quranayah[6][85]
\end{Arabic}}
\flushleft{\begin{malayalam}
സകരിയ്യാ, യഹ്‌യാ, ഈസാ, ഇല്‍യാസ് എന്നിവരെയും (നേര്‍വഴിയിലാക്കി.) അവരെല്ലാം സജ്ജനങ്ങളില്‍ പെട്ടവരത്രെ.
\end{malayalam}}
\flushright{\begin{Arabic}
\quranayah[6][86]
\end{Arabic}}
\flushleft{\begin{malayalam}
ഇസ്മാഈല്‍, അല്‍യസഅ്‌, യൂനുസ്‌, ലൂത്വ് എന്നിവരെയും (നേര്‍വഴിയിലാക്കി.) അവരെല്ലാവരെയും നാം ലോകരില്‍ വെച്ച് ശ്രേഷ്ഠരാക്കിയിരിക്കുന്നു.
\end{malayalam}}
\flushright{\begin{Arabic}
\quranayah[6][87]
\end{Arabic}}
\flushleft{\begin{malayalam}
അവരുടെ പിതാക്കളില്‍ നിന്നും സന്തതികളില്‍ നിന്നും സഹോദരങ്ങളില്‍ നിന്നും (ചിലര്‍ക്ക് നാം ശ്രേഷ്ഠത നല്‍കിയിരിക്കുന്നു.) അവരെ നാം വിശിഷ്ടരായി തെരഞ്ഞെടുക്കുകയും, നേര്‍മാര്‍ഗത്തിലേക്ക് അവരെ നയിക്കുകയും ചെയ്തിരിക്കുന്നു.
\end{malayalam}}
\flushright{\begin{Arabic}
\quranayah[6][88]
\end{Arabic}}
\flushleft{\begin{malayalam}
അതാണ് അല്ലാഹുവിന്‍റെ മാര്‍ഗദര്‍ശനം. അത് മുഖേന തന്‍റെ ദാസന്‍മാരില്‍ നിന്ന് താന്‍ ഉദ്ദേശിക്കുന്നവരെ അവന്‍ നേര്‍മാര്‍ഗത്തിലേക്ക് നയിക്കുന്നു. അവര്‍ (അല്ലാഹുവോട്‌) പങ്കുചേര്‍ത്തിരുന്നുവെങ്കില്‍ അവര്‍ പ്രവര്‍ത്തിച്ചിരുന്നതെല്ലാം അവരെ സംബന്ധിച്ചിടത്തോളം നിഷ്ഫലമായിപ്പോകുമായിരുന്നു.
\end{malayalam}}
\flushright{\begin{Arabic}
\quranayah[6][89]
\end{Arabic}}
\flushleft{\begin{malayalam}
നാം വേദവും വിജ്ഞാനവും പ്രവാചകത്വവും നല്‍കിയിട്ടുള്ളവരത്രെ അവര്‍. ഇനി ഇക്കൂട്ടര്‍ അവയൊക്കെ നിഷേധിക്കുകയാണെങ്കില്‍ അവയില്‍ അവിശ്വസിക്കുന്നവരല്ലാത്ത ഒരു ജനവിഭാഗത്തെ നാമത് ഭരമേല്‍പിച്ചിട്ടുണ്ട്‌
\end{malayalam}}
\flushright{\begin{Arabic}
\quranayah[6][90]
\end{Arabic}}
\flushleft{\begin{malayalam}
അവരെയാണ് അല്ലാഹു നേര്‍വഴിയിലാക്കിയിട്ടുള്ളത്‌. അതിനാല്‍ അവരുടെ നേര്‍മാര്‍ഗത്തെ നീ പിന്തുടര്‍ന്ന് കൊള്ളുക. (നബിയേ,) പറയുക: ഇതിന്‍റെ പേരില്‍ യാതൊരു പ്രതിഫലവും ഞാന്‍ നിങ്ങളോട് ആവശ്യപ്പെടുന്നില്ല. ഇത് ലോകര്‍ക്ക് വേണ്ടിയുള്ള ഒരു ഉല്‍ബോധനമല്ലാതെ മറ്റൊന്നുമല്ല.
\end{malayalam}}
\flushright{\begin{Arabic}
\quranayah[6][91]
\end{Arabic}}
\flushleft{\begin{malayalam}
ഒരു മനുഷ്യന്നും അല്ലാഹു യാതൊന്നും അവതരിപ്പിച്ചുകൊടുത്തിട്ടില്ല എന്നു പറഞ്ഞ സന്ദര്‍ഭത്തില്‍ അല്ലാഹുവെ വിലയിരുത്തേണ്ട മുറപ്രകാരം വിലയിരുത്താതിരിക്കുകയാണ് അവര്‍ ചെയ്തത്‌. പറയുക: എന്നാല്‍ സത്യപ്രകാശമായിക്കൊണ്ടും, മനുഷ്യര്‍ക്ക് മാര്‍ഗദര്‍ശകമായിക്കൊണ്ടും മൂസാ കൊണ്ടു വന്ന ഗ്രന്ഥം ആരാണ് അവതരിപ്പിച്ചത് ? നിങ്ങള്‍ അതിനെ കടലാസ് തുണ്ടുകളാക്കി ചില ഭാഗങ്ങള്‍ വെളിപ്പെടുത്തുകയും, (മറ്റു) പലതും ഒളിച്ച് വെക്കുകയും ചെയ്യുന്നുണ്ടല്ലോ. നിങ്ങള്‍ക്കോ നിങ്ങളുടെ പിതാക്കന്‍മാര്‍ക്കോ അറിവില്ലാതിരുന്ന പലതും (ആ ഗ്രന്ഥത്തിലൂടെ) നിങ്ങള്‍ പഠിപ്പിക്കപ്പെട്ടിട്ടുമുണ്ട്‌. അല്ലാഹുവാണ് (അത് അവതരിപ്പിച്ചത്‌) എന്ന് പറയുക. പിന്നീട് അവരുടെ കുതര്‍ക്കങ്ങളുമായി വിളയാടുവാന്‍ അവരെ വിട്ടേക്കുക.
\end{malayalam}}
\flushright{\begin{Arabic}
\quranayah[6][92]
\end{Arabic}}
\flushleft{\begin{malayalam}
ഇതാ, നാം അവതരിപ്പിച്ച, നന്‍മ നിറഞ്ഞ ഒരു ഗ്രന്ഥം! അതിന്‍റെ മുമ്പുള്ള വേദത്തെ ശരിവെക്കുന്നതത്രെ അത്‌. മാതൃനഗരി (മക്ക) യിലും അതിന്‍റെ ചുറ്റുഭാഗത്തുമുള്ളവര്‍ക്ക് നീ താക്കീത് നല്‍കുവാന്‍ വേണ്ടി ഉള്ളതുമാണ് അത്‌. പരലോകത്തില്‍ വിശ്വസിക്കുന്നവര്‍ ഈ ഗ്രന്ഥത്തില്‍ വിശ്വസിക്കുന്നതാണ്‌. തങ്ങളുടെ പ്രാര്‍ത്ഥന അവര്‍ മുറപ്രകാരം സൂക്ഷിച്ച് പോരുന്നതുമാണ്‌.
\end{malayalam}}
\flushright{\begin{Arabic}
\quranayah[6][93]
\end{Arabic}}
\flushleft{\begin{malayalam}
അല്ലാഹുവിന്‍റെ പേരില്‍ കള്ളം കെട്ടിച്ചമയ്ക്കുകയോ, തനിക്ക് യാതൊരു ബോധനവും നല്‍കപ്പെടാതെ എനിക്ക് ബോധനം ലഭിച്ചിരിക്കുന്നു എന്ന് പറയുകയോ ചെയ്തവനേക്കാളും, അല്ലാഹു അവതരിപ്പിച്ചത് പോലെയുള്ളത് ഞാനും അവതരിപ്പിക്കാമെന്ന് പറഞ്ഞവനെക്കാളും വലിയ അക്രമി ആരുണ്ട് ? ആ അക്രമികള്‍ മരണവെപ്രാളത്തിലായിരിക്കുന്ന രംഗം നീ കണ്ടിരുന്നുവെങ്കില്‍! നിങ്ങള്‍ നിങ്ങളുടെ ആത്മാക്കളെ പുറത്തിറക്കുവിന്‍ എന്ന് പറഞ്ഞ് കൊണ്ട് മലക്കുകള്‍ അവരുടെ നേരെ തങ്ങളുടെ കൈകള്‍ നീട്ടികൊണ്ടിരിക്കുകയാണ്‌. നിങ്ങള്‍ അല്ലാഹുവിന്‍റെ പേരില്‍ സത്യമല്ലാത്തത് പറഞ്ഞുകൊണ്ടിരുന്നതിന്‍റെയും, അവന്‍റെ ദൃഷ്ടാന്തങ്ങളെ നിങ്ങള്‍ അഹങ്കരിച്ച് തള്ളിക്കളഞ്ഞിരുന്നതിന്‍റെയും ഫലമായി ഇന്ന് നിങ്ങള്‍ക്ക് ഹീനമായ ശിക്ഷ നല്‍കപ്പെടുന്നതാണ്‌. (എന്ന് മലക്കുകള്‍ പറയും.)
\end{malayalam}}
\flushright{\begin{Arabic}
\quranayah[6][94]
\end{Arabic}}
\flushleft{\begin{malayalam}
(അവരോട് അല്ലാഹു പറയും:) നിങ്ങളെ നാം ആദ്യഘട്ടത്തില്‍ സൃഷ്ടിച്ചത് പോലെത്തന്നെ നിങ്ങളിതാ നമ്മുടെ അടുക്കല്‍ ഒറ്റപ്പെട്ടവരായി വന്നെത്തിയിരിക്കുന്നു. നിങ്ങള്‍ക്ക് നാം അധീനപ്പെടുത്തിതന്നതെല്ലാം നിങ്ങളുടെ പിന്നില്‍ നിങ്ങള്‍ വിട്ടേച്ച് പോന്നിരിക്കുന്നു. നിങ്ങളുടെ കാര്യത്തില്‍ (അല്ലാഹുവിന്‍റെ) പങ്കുകാരാണെന്ന് നിങ്ങള്‍ ജല്‍പിച്ചിരുന്ന നിങ്ങളുടെ ആ ശുപാര്‍ശക്കാരെ നിങ്ങളോടൊപ്പം നാം കാണുന്നില്ല. നിങ്ങള്‍ തമ്മിലുള്ള ബന്ധം അറ്റുപോകുകയും നിങ്ങള്‍ ജല്‍പിച്ചിരുന്നതെല്ലാം നിങ്ങളെ വിട്ടുപോകുകയും ചെയ്തിരിക്കുന്നു.
\end{malayalam}}
\flushright{\begin{Arabic}
\quranayah[6][95]
\end{Arabic}}
\flushleft{\begin{malayalam}
തീര്‍ച്ചയായും ധാന്യമണികളും ഈന്തപ്പഴക്കുരുവും പിളര്‍ക്കുന്നവനാകുന്നു അല്ലാഹു നിര്‍ജീവമായതില്‍ നിന്ന് ജീവനുള്ളതിനെ അവന്‍ പുറത്ത് വരുത്തുന്നു. ജീവനുള്ളതില്‍ നിന്ന് നിര്‍ജീവമായതിനെയും അവന്‍ പുറത്ത് വരുത്തുന്നതാണ്‌. അങ്ങനെയുള്ളവനത്രെ അല്ലാഹു.
\end{malayalam}}
\flushright{\begin{Arabic}
\quranayah[6][96]
\end{Arabic}}
\flushleft{\begin{malayalam}
പ്രഭാതത്തെ പിളര്‍ത്തിക്കൊണ്ട് വരുന്നവനാണവന്‍. രാത്രിയെ അവന്‍ ശാന്തമായ വിശ്രമവേളയാക്കിയിരിക്കുന്നു. സൂര്യനെയും ചന്ദ്രനെയും കണക്കുകള്‍ക്ക് അടിസ്ഥാനവും (ആക്കിയിരിക്കുന്നു.) പ്രതാപിയും സര്‍വ്വജ്ഞനുമായ അല്ലാഹുവിന്‍റെ ക്രമീകരണമത്രെ അത്‌.
\end{malayalam}}
\flushright{\begin{Arabic}
\quranayah[6][97]
\end{Arabic}}
\flushleft{\begin{malayalam}
അവനാണ് നിങ്ങള്‍ക്ക് വേണ്ടി നക്ഷത്രങ്ങളെ, കരയിലെയും കടലിലെയും അന്ധകാരങ്ങളില്‍ നിങ്ങള്‍ക്ക് അവ മുഖേന വഴിയറിയാന്‍ പാകത്തിലാക്കിത്തന്നത്‌. മനസ്സിലാക്കുന്ന ആളുകള്‍ക്ക് വേണ്ടി നാമിതാ ദൃഷ്ടാന്തങ്ങള്‍ വിശദീകരിച്ചിരിക്കുന്നു.
\end{malayalam}}
\flushright{\begin{Arabic}
\quranayah[6][98]
\end{Arabic}}
\flushleft{\begin{malayalam}
അവനാണ് ഒരേ ആത്മാവില്‍ നിന്നും നിങ്ങളെ സൃഷ്ടിച്ചുണ്ടാക്കിയവന്‍. പിന്നെ (നിങ്ങള്‍ക്ക്‌) ഒരു സ്ഥിരസങ്കേതവും സൂക്ഷിപ്പ് കേന്ദ്രവുമുണ്ട്‌. (കാര്യം) ഗ്രഹിക്കുന്ന ആളുകള്‍ക്ക് വേണ്ടി നാം ഇതാ ദൃഷ്ടാന്തങ്ങള്‍ വിശദീകരിച്ചിരിക്കുന്നു.
\end{malayalam}}
\flushright{\begin{Arabic}
\quranayah[6][99]
\end{Arabic}}
\flushleft{\begin{malayalam}
അവനാണ് ആകാശത്ത് നിന്ന് വെള്ളം ചൊരിഞ്ഞുതന്നവന്‍. എന്നിട്ട് അത് മുഖേന നാം എല്ലാ വസ്തുക്കളുടെയും മുളകള്‍ പുറത്ത് കൊണ്ടുവരികയും, അനന്തരം അതില്‍ നിന്ന് പച്ചപിടിച്ച ചെടികള്‍ വളര്‍ത്തിക്കൊണ്ട് വരികയും ചെയ്തു. ആ ചെടികളില്‍ നിന്ന് നാം തിങ്ങിനിറഞ്ഞ ധാന്യം പുറത്ത് വരുത്തുന്നു. ഈന്തപ്പനയില്‍ നിന്ന് അഥവാ അതിന്‍റെ കൂമ്പോളയില്‍ നിന്ന് തൂങ്ങി നില്‍ക്കുന്ന കുലകള്‍ പുറത്ത് വരുന്നു. (അപ്രകാരം തന്നെ) മുന്തിരിത്തോട്ടങ്ങളും , പരസ്പരം തുല്യത തോന്നുന്നതും, എന്നാല്‍ ഒരുപോലെയല്ലാത്തതുമായ ഒലീവും മാതളവും (നാം ഉല്‍പാദിപ്പിച്ചു.) അവയുടെ കായ്കള്‍ കായ്ച്ച് വരുന്നതും മൂപ്പെത്തുന്നതും നിങ്ങള്‍ നോക്കൂ. വിശ്വസിക്കുന്ന ജനങ്ങള്‍ക്ക് അതിലെല്ലാം ദൃഷ്ടാന്തങ്ങളുണ്ട്‌.
\end{malayalam}}
\flushright{\begin{Arabic}
\quranayah[6][100]
\end{Arabic}}
\flushleft{\begin{malayalam}
അവര്‍ ജിന്നുകളെ അല്ലാഹുവിന് പങ്കാളികളാക്കിയിരിക്കുന്നു. എന്നാല്‍ അവരെ അവന്‍ സൃഷ്ടിച്ചതാണ്‌. ഒരു വിവരവും കൂടാതെ അവന്ന് പുത്രന്‍മാരെയും പുത്രിമാരെയും അവര്‍ ആരോപിച്ചുണ്ടാക്കിയിരിക്കുന്നു. അവര്‍ പറഞ്ഞുണ്ടാക്കുന്നതില്‍ നിന്നെല്ലാം അല്ലാഹു എത്രയോ പരിശുദ്ധനും ഉന്നതനുമാകുന്നു.
\end{malayalam}}
\flushright{\begin{Arabic}
\quranayah[6][101]
\end{Arabic}}
\flushleft{\begin{malayalam}
ആകാശങ്ങളുടെയും ഭൂമിയുടെയും നിര്‍മാതാവാണവന്‍. അവന്ന് എങ്ങനെ ഒരു സന്താനമുണ്ടാകും? അവന്നൊരു കൂട്ടുകാരിയുമില്ലല്ലോ? എല്ലാ വസ്തുക്കളെയും അവന്‍ സൃഷ്ടിച്ചതാണ്‌. അവന്‍ എല്ലാകാര്യത്തെപ്പറ്റിയും അറിയുന്നവനുമാണ്‌.
\end{malayalam}}
\flushright{\begin{Arabic}
\quranayah[6][102]
\end{Arabic}}
\flushleft{\begin{malayalam}
അങ്ങനെയുള്ളവനാണ് നിങ്ങളുടെ രക്ഷിതാവായ അല്ലാഹു. അവനല്ലാതെ യാതൊരു ദൈവവുമില്ല. എല്ലാ വസ്തുക്കളുടെയും സ്രഷ്ടാവാണ് അവന്‍. അതിനാല്‍ അവനെ നിങ്ങള്‍ ആരാധിക്കുക. അവന്‍ സകലകാര്യങ്ങളുടെയും കൈകാര്യക്കാരനാകുന്നു.
\end{malayalam}}
\flushright{\begin{Arabic}
\quranayah[6][103]
\end{Arabic}}
\flushleft{\begin{malayalam}
കണ്ണുകള്‍ അവനെ കണ്ടെത്തുകയില്ല. കണ്ണുകളെ അവന്‍ കണ്ടെത്തുകയും ചെയ്യും. അവന്‍ സൂക്ഷ്മജ്ഞാനിയും അഭിജ്ഞനുമാകുന്നു.
\end{malayalam}}
\flushright{\begin{Arabic}
\quranayah[6][104]
\end{Arabic}}
\flushleft{\begin{malayalam}
നിങ്ങളുടെ രക്ഷിതാവിങ്കല്‍ നിന്ന് നിങ്ങള്‍ക്കിതാ കണ്ണുതുറപ്പിക്കുന്ന തെളിവുകള്‍ വന്നെത്തിയിരിക്കുന്നു. വല്ലവനും അത് കണ്ടറിഞ്ഞാല്‍ അതിന്‍റെ ഗുണം അവന്ന് തന്നെയാണ്‌. വല്ലവനും അന്ധത കൈക്കൊണ്ടാല്‍ അതിന്‍റെ ദോഷവും അവന്നു തന്നെ. ഞാന്‍ നിങ്ങളുടെ മേല്‍ ഒരു കാവല്‍ക്കാരനൊന്നുമല്ല.
\end{malayalam}}
\flushright{\begin{Arabic}
\quranayah[6][105]
\end{Arabic}}
\flushleft{\begin{malayalam}
അപ്രകാരം നാം വിവിധ രൂപത്തില്‍ ദൃഷ്ടാന്തങ്ങള്‍ വിവരിക്കുന്നു. നീ (വല്ലവരില്‍ നിന്നും) പഠിച്ചുവന്നതാണെന്ന് അവിശ്വാസികള്‍ പറയുവാനും, എന്നാല്‍ മനസ്സിലാക്കുന്ന ആളുകള്‍ക്ക് നാം കാര്യം വ്യക്തമാക്കികൊടുക്കുവാനും വേണ്ടിയാണത്‌.
\end{malayalam}}
\flushright{\begin{Arabic}
\quranayah[6][106]
\end{Arabic}}
\flushleft{\begin{malayalam}
നിനക്ക് നിന്‍റെ രക്ഷിതാവിങ്കല്‍ നിന്ന് ബോധനം നല്‍കപ്പെട്ടതിനെ നീ പിന്തുടരുക. അവനല്ലാതെ ഒരു ദൈവവുമില്ല. ബഹുദൈവവാദികളില്‍ നിന്ന് നീ തിരിഞ്ഞുകളയുക.
\end{malayalam}}
\flushright{\begin{Arabic}
\quranayah[6][107]
\end{Arabic}}
\flushleft{\begin{malayalam}
അല്ലാഹു ഉദ്ദേശിച്ചിരുന്നെങ്കില്‍ അവര്‍ (അവനോട്‌) പങ്കുചേര്‍ക്കുമായിരുന്നില്ല. നിന്നെ നാം അവരുടെ മേല്‍ ഒരു കാവല്‍ക്കാരനാക്കിയിട്ടുമില്ല. നീ അവരുടെ മേല്‍ ഉത്തരവാദിത്തം ഏല്‍പിക്കപ്പെട്ടവനുമല്ല.
\end{malayalam}}
\flushright{\begin{Arabic}
\quranayah[6][108]
\end{Arabic}}
\flushleft{\begin{malayalam}
അല്ലാഹുവിനു പുറമെ അവര്‍ വിളിച്ച് പ്രാര്‍ത്ഥിക്കുന്നവരെ നിങ്ങള്‍ ശകാരിക്കരുത്‌. അവര്‍ വിവരമില്ലാതെ അതിക്രമമായി അല്ലാഹുവെ ശകാരിക്കാന്‍ അത് കാരണമായേക്കും. അപ്രകാരം ഓരോ വിഭാഗത്തിനും അവരുടെ പ്രവര്‍ത്തനം നാം ഭംഗിയായി തോന്നിച്ചിരിക്കുന്നു. പിന്നീട് അവരുടെ രക്ഷിതാവിങ്കലേക്കാണ് അവരുടെ മടക്കം. അവര്‍ ചെയ്തുകൊണ്ടിരുന്നതിനെപ്പറ്റിയെല്ലാം അപ്പോള്‍ അവന്‍ അവരെ അറിയിക്കുന്നതാണ്‌.
\end{malayalam}}
\flushright{\begin{Arabic}
\quranayah[6][109]
\end{Arabic}}
\flushleft{\begin{malayalam}
തങ്ങള്‍ക്ക് വല്ല ദൃഷ്ടാന്തവും വന്നുകിട്ടുന്ന പക്ഷം അതില്‍ വിശ്വസിക്കുക തന്നെ ചെയ്യുമെന്ന് അവര്‍ അല്ലാഹുവിന്‍റെ പേരില്‍ തങ്ങളെകൊണ്ടാവും വിധം ഉറപ്പിച്ച് സത്യം ചെയ്ത് പറയുന്നു. പറയുക: ദൃഷ്ടാന്തങ്ങള്‍ അല്ലാഹുവിന്‍റെ അധീനത്തില്‍ മാത്രമാണുള്ളത്‌. നിങ്ങള്‍ക്കെന്തറിയാം? അത് വന്ന് കിട്ടിയാല്‍ തന്നെ അവര്‍ വിശ്വസിക്കുന്നതല്ല.
\end{malayalam}}
\flushright{\begin{Arabic}
\quranayah[6][110]
\end{Arabic}}
\flushleft{\begin{malayalam}
ഇതില്‍ (ഖുര്‍ആനില്‍) ആദ്യതവണ അവര്‍ വിശ്വസിക്കാതിരുന്നത് പോലെത്തന്നെ (ഇപ്പോഴും) നാം അവരുടെ മനസ്സുകളെയും കണ്ണുകളെയും മറിച്ചുകൊണ്ടിരിക്കും. അവരുടെ ധിക്കാരവുമായി വിഹരിച്ചുകൊള്ളുവാന്‍ നാം അവരെ വിട്ടേക്കുകയും ചെയ്യും.
\end{malayalam}}
\flushright{\begin{Arabic}
\quranayah[6][111]
\end{Arabic}}
\flushleft{\begin{malayalam}
നാം അവരിലേക്ക് മലക്കുകളെ ഇറക്കുകയും, മരിച്ചവര്‍ അവരോട് സംസാരിക്കുകയും, സര്‍വ്വവസ്തുക്കളെയും നാം അവരുടെ മുമ്പാകെ കൂട്ടം കൂട്ടമായി ശേഖരിക്കുകയും ചെയ്താലും അവര്‍ വിശ്വസിക്കാന്‍ പോകുന്നില്ല. അല്ലാഹു ഉദ്ദേശിച്ചെങ്കിലല്ലാതെ. എന്നാല്‍ അവരില്‍ അധികപേരും വിവരക്കേട് പറയുകയാകുന്നു.
\end{malayalam}}
\flushright{\begin{Arabic}
\quranayah[6][112]
\end{Arabic}}
\flushleft{\begin{malayalam}
അപ്രകാരം ഓരോ പ്രവാചകന്നും മനുഷ്യരിലും ജിന്നുകളിലും പെട്ട പിശാചുക്കളെ നാം ശത്രുക്കളാക്കിയിട്ടുണ്ട്‌. കബളിപ്പിക്കുന്ന ഭംഗിവാക്കുകള്‍ അവര്‍ അന്യോന്യം ദുര്‍ബോധനം ചെയ്യുന്നു. നിന്‍റെ രക്ഷിതാവ് ഉദ്ദേശിച്ചിരുന്നെങ്കില്‍ അവരത് ചെയ്യുമായിരുന്നില്ല. അത് കൊണ്ട് അവര്‍ കെട്ടിച്ചമയ്ക്കുന്ന കാര്യങ്ങളുമായി അവരെ നീ വിട്ടേക്കുക.
\end{malayalam}}
\flushright{\begin{Arabic}
\quranayah[6][113]
\end{Arabic}}
\flushleft{\begin{malayalam}
പരലോകത്തില്‍ വിശ്വാസമില്ലാത്തവരുടെ മനസ്സുകള്‍ അതിലേക്ക് (ആ ഭംഗിവാക്കുകളിലേക്ക്‌) ചായുവാനും, അവര്‍ അതില്‍ സംതൃപ്തരാകുവാനും, അവര്‍ ചെയ്ത് കൂട്ടുന്നതെല്ലാം ചെയ്ത് കൂട്ടുവാനും വേണ്ടിയത്രെ അത്‌.
\end{malayalam}}
\flushright{\begin{Arabic}
\quranayah[6][114]
\end{Arabic}}
\flushleft{\begin{malayalam}
(പറയുക:) അപ്പോള്‍ വിധികര്‍ത്താവായി ഞാന്‍ അന്വേഷിക്കേണ്ടത് അല്ലാഹു അല്ലാത്തവരെയാണോ? അവനാകട്ടെ, വിശദവിവരങ്ങളുള്ള വേദഗ്രന്ഥം നിങ്ങള്‍ക്കിറക്കിത്തന്നവനാകുന്നു. അത് സത്യവുമായി നിന്‍റെ രക്ഷിതാവിങ്കല്‍ നിന്ന് ഇറക്കപ്പെട്ടതാണെന്ന് നാം മുമ്പ് വേദഗ്രന്ഥം നല്‍കിയിട്ടുള്ളവര്‍ക്കറിയാം. അതിനാല്‍ നീ ഒരിക്കലും സംശയാലുക്കളില്‍ പെട്ടുപോകരുത്‌.
\end{malayalam}}
\flushright{\begin{Arabic}
\quranayah[6][115]
\end{Arabic}}
\flushleft{\begin{malayalam}
നിന്‍റെ രക്ഷിതാവിന്‍റെ വചനം സത്യത്തിലും നീതിയിലും പരിപൂര്‍ണ്ണമായിരിക്കുന്നു. അവന്‍റെ വചനങ്ങള്‍ക്ക് മാറ്റം വരുത്താനാരുമില്ല. അവന്‍ എല്ലാം കേള്‍ക്കുന്നവനും അറിയുന്നവനുമത്രെ.
\end{malayalam}}
\flushright{\begin{Arabic}
\quranayah[6][116]
\end{Arabic}}
\flushleft{\begin{malayalam}
ഭൂമിയിലുള്ളവരില്‍ അധികപേരെയും നീ അനുസരിക്കുന്ന പക്ഷം അല്ലാഹുവിന്‍റെ മാര്‍ഗത്തില്‍ നിന്നും നിന്നെ അവര്‍ തെറ്റിച്ചുകളയുന്നതാണ്‌. ഊഹത്തെ മാത്രമാണ് അവര്‍ പിന്തുടരുന്നത്‌. അവര്‍ അനുമാനിക്കുക മാത്രമാണ് ചെയ്യുന്നത്‌.
\end{malayalam}}
\flushright{\begin{Arabic}
\quranayah[6][117]
\end{Arabic}}
\flushleft{\begin{malayalam}
തന്‍റെ മാര്‍ഗത്തില്‍ നിന്ന് തെറ്റിപ്പോകുന്നവന്‍ ആരാണെന്ന് തീര്‍ച്ചയായും നിന്‍റെ രക്ഷിതാവിന് അറിയാം. നേര്‍വഴിപ്രാപിച്ചവരെപ്പറ്റി നല്ലവണ്ണം അറിയുന്നവനും അവന്‍ തന്നെയാണ്‌.
\end{malayalam}}
\flushright{\begin{Arabic}
\quranayah[6][118]
\end{Arabic}}
\flushleft{\begin{malayalam}
അതിനാല്‍ അല്ലാഹുവിന്‍റെ നാമം ഉച്ചരി(ച്ച് അറു) ക്കപ്പെട്ടതില്‍ നിന്നും നിങ്ങള്‍ തിന്നുകൊള്ളുക. നിങ്ങള്‍ അവന്‍റെ വചനങ്ങളില്‍ വിശ്വസിക്കുന്നവരാണെങ്കില്‍.
\end{malayalam}}
\flushright{\begin{Arabic}
\quranayah[6][119]
\end{Arabic}}
\flushleft{\begin{malayalam}
അല്ലാഹുവിന്‍റെ നാമം ഉച്ചരി (ച്ച് അറു) ക്കപ്പെട്ടതില്‍ നിന്ന് നിങ്ങള്‍ എന്തിന് തിന്നാതിരിക്കണം.? നിങ്ങളുടെ മേല്‍ നിഷിദ്ധമാക്കിയത് അവന്‍ നിങ്ങള്‍ക്ക് വിശദമാക്കിത്തന്നിട്ടുണ്ടല്ലോ. നിങ്ങള്‍ (തിന്നുവാന്‍) നിര്‍ബന്ധിതരായിത്തീരുന്നതൊഴികെ. ധാരാളം പേര്‍ യാതൊരു വിവരവുമില്ലാതെ തന്നിഷ്ടങ്ങള്‍ക്കനുസരിച്ച് (ആളുകളെ) പിഴപ്പിച്ചു കൊണ്ടിരിക്കുകയാണ്‌. തീര്‍ച്ചയായും നിന്‍റെ രക്ഷിതാവ് അതിക്രമകാരികളെപ്പറ്റി നല്ലവണ്ണം അറിയുന്നവനത്രെ.
\end{malayalam}}
\flushright{\begin{Arabic}
\quranayah[6][120]
\end{Arabic}}
\flushleft{\begin{malayalam}
പാപത്തില്‍ നിന്ന് പ്രത്യക്ഷമായതും പരോക്ഷമായതും നിങ്ങള്‍ വെടിയുക. പാപം സമ്പാദിച്ച് വെക്കുന്നവരാരോ അവര്‍ ചെയ്ത് കൂട്ടുന്നതിന് തക്ക പ്രതിഫലം തീര്‍ച്ചയായും അവര്‍ക്ക് നല്‍കപ്പെടുന്നതാണ്‌.
\end{malayalam}}
\flushright{\begin{Arabic}
\quranayah[6][121]
\end{Arabic}}
\flushleft{\begin{malayalam}
അല്ലാഹുവിന്‍റെ നാമം ഉച്ചരിക്കപ്പെടാത്തതില്‍ നിന്ന് നിങ്ങള്‍ തിന്നരുത്‌. തീര്‍ച്ചയായും അത് അധര്‍മ്മമാണ്‌. നിങ്ങളോട് തര്‍ക്കിക്കുവാന്‍ വേണ്ടി പിശാചുക്കള്‍ അവരുടെ മിത്രങ്ങള്‍ക്ക് തീര്‍ച്ചയായും ദുര്‍ബോധനം നല്‍കിക്കൊണ്ടിരിക്കും. നിങ്ങള്‍ അവരെ അനുസരിക്കുന്ന പക്ഷം തീര്‍ച്ചയായും നിങ്ങള്‍ (അല്ലാഹുവോട്‌) പങ്കുചേര്‍ക്കുന്നവരായിപ്പോകും.
\end{malayalam}}
\flushright{\begin{Arabic}
\quranayah[6][122]
\end{Arabic}}
\flushleft{\begin{malayalam}
നിര്‍ജീവാവസ്ഥയിലായിരിക്കെ നാം ജീവന്‍ നല്‍കുകയും, നാം ഒരു (സത്യ) പ്രകാശം നല്‍കിയിട്ട് അതുമായി ജനങ്ങള്‍ക്കിടയിലൂടെ നടന്ന് കൊണ്ടിരിക്കുകയും ചെയ്യുന്നവന്‍റെ അവസ്ഥ, പുറത്ത് കടക്കാനാകാത്ത വിധം അന്ധകാരങ്ങളില്‍ അകപ്പെട്ട അവസ്ഥയില്‍ കഴിയുന്നവന്‍റെത് പോലെയാണോ? അങ്ങനെ, സത്യനിഷേധികള്‍ക്ക് തങ്ങള്‍ ചെയ്ത് കൊണ്ടിരിക്കുന്നത് ഭംഗിയായി തോന്നിക്കപ്പെട്ടിരിക്കുന്നു.
\end{malayalam}}
\flushright{\begin{Arabic}
\quranayah[6][123]
\end{Arabic}}
\flushleft{\begin{malayalam}
അതേ പ്രകാരം തന്നെ ഓരോ നാട്ടിലും കുതന്ത്രങ്ങളുണ്ടാക്കുവാന്‍ അവിടത്തെ കുറ്റവാളികളുടെ തലവന്‍മാരെ നാം ഏര്‍പെടുത്തിയിട്ടുണ്ട്‌. എന്നാല്‍ അവര്‍ കുതന്ത്രം പ്രയോഗിക്കുന്നത് അവര്‍ക്കെതിരില്‍ തന്നെയാണ്‌. അവര്‍ (അതിനെപ്പറ്റി) ബോധവാന്‍മാരാകുന്നില്ല.
\end{malayalam}}
\flushright{\begin{Arabic}
\quranayah[6][124]
\end{Arabic}}
\flushleft{\begin{malayalam}
അവര്‍ക്ക് വല്ല ദൃഷ്ടാന്തവും വന്നാല്‍, അല്ലാഹുവിന്‍റെ ദൂതന്‍മാര്‍ക്ക് നല്‍കപ്പെട്ടത് പോലുള്ളത് ഞങ്ങള്‍ക്കും ലഭിക്കുന്നത് വരെ ഞങ്ങള്‍ വിശ്വസിക്കുകയേ ഇല്ല എന്നായിരിക്കും അവര്‍ പറയുക. എന്നാല്‍ അല്ലാഹുവിന്ന് നല്ലവണ്ണമറിയാം; തന്‍റെ ദൌത്യം എവിടെയാണ് ഏല്‍പിക്കേണ്ടതെന്ന്‌. കുറ്റകൃത്യങ്ങളില്‍ ഏര്‍പെട്ടവര്‍ക്ക് തങ്ങള്‍ പ്രയോഗിച്ചിരുന്ന കുതന്ത്രത്തിന്‍റെ ഫലമായി അല്ലാഹുവിങ്കല്‍ ഹീനതയും കഠിനമായ ശിക്ഷയും വന്നുഭവിക്കുന്നതാണ്‌.
\end{malayalam}}
\flushright{\begin{Arabic}
\quranayah[6][125]
\end{Arabic}}
\flushleft{\begin{malayalam}
ഏതൊരാളെ നേര്‍വഴിയിലേക്ക് നയിക്കുവാന്‍ അല്ലാഹു ഉദ്ദേശിക്കുന്നുവോ അവന്‍റെ ഹൃദയത്തെ ഇസ്ലാമിലേക്ക് അവന്‍ തുറന്നുകൊടുക്കുന്നതാണ്‌. ഏതൊരാളെ അല്ലാഹു പിഴവിലാക്കാന്‍ ഉദ്ദേശിക്കുന്നുവോ അവന്‍റെ ഹൃദയത്തെ ഇടുങ്ങിയതും ഞെരുങ്ങിയതുമാക്കിത്തീര്‍ക്കുന്നതാണ്‌. അവന്‍ ആകാശത്തിലൂടെ കയറിപ്പോകുന്നത് പോലെ. വിശ്വസിക്കാത്തവരുടെ മേല്‍ അപ്രകാരം അല്ലാഹു ശിക്ഷ ഏര്‍പെടുത്തുന്നു.
\end{malayalam}}
\flushright{\begin{Arabic}
\quranayah[6][126]
\end{Arabic}}
\flushleft{\begin{malayalam}
നിന്‍റെ രക്ഷിതാവിന്‍റെ നേരായ മാര്‍ഗമാണിത്‌. ശ്രദ്ധിച്ച് മനസ്സിലാക്കുന്ന ജനങ്ങള്‍ക്ക് വേണ്ടി നാമിതാ ദൃഷ്ടാന്തങ്ങള്‍ വിശദീകരിച്ചിരിക്കുന്നു.
\end{malayalam}}
\flushright{\begin{Arabic}
\quranayah[6][127]
\end{Arabic}}
\flushleft{\begin{malayalam}
അവര്‍ക്ക് അവരുടെ രക്ഷിതാവിന്‍റെ അടുക്കല്‍ സമാധാനത്തിന്‍റെ ഭവനമുണ്ട്‌. അവന്‍ അവരുടെ രക്ഷാധികാരിയായിരിക്കും. അവര്‍ പ്രവര്‍ത്തിച്ചിരുന്നതിന്‍റെ ഫലമത്രെ അത്‌.
\end{malayalam}}
\flushright{\begin{Arabic}
\quranayah[6][128]
\end{Arabic}}
\flushleft{\begin{malayalam}
അവരെയെല്ലാം അവന്‍ (അല്ലാഹു) ഒരുമിച്ചുകൂട്ടുന്ന ദിവസം. (ജിന്നുകളോട് അവന്‍ പറയും:) ജിന്നുകളുടെ സമൂഹമേ, മനുഷ്യരില്‍ നിന്ന് ധാരാളം പേരെ നിങ്ങള്‍ പിഴപ്പിച്ചിട്ടുണ്ട്‌. മനുഷ്യരില്‍ നിന്നുള്ള അവരുടെ ഉറ്റമിത്രങ്ങള്‍ പറയും: ഞങ്ങളുടെ രക്ഷിതാവേ, ഞങ്ങളില്‍ ചിലര്‍ മറ്റുചിലരെക്കൊണ്ട് സുഖമനുഭവിക്കുകയുണ്ടായി. നീ ഞങ്ങള്‍ക്ക് നിശ്ചയിച്ച അവധിയില്‍ ഞങ്ങളിതാ എത്തിയിരിക്കുന്നു. അവന്‍ പറയും: നരകമാണ് നിങ്ങളുടെ പാര്‍പ്പിടം. അല്ലാഹു ഉദ്ദേശിച്ച സമയം ഒഴികെ നിങ്ങളതില്‍ നിത്യവാസികളായിരിക്കും. തീര്‍ച്ചയായും നിന്‍റെ രക്ഷിതാവ് യുക്തിമാനും സര്‍വ്വജ്ഞനുമാകുന്നു.
\end{malayalam}}
\flushright{\begin{Arabic}
\quranayah[6][129]
\end{Arabic}}
\flushleft{\begin{malayalam}
അപ്രകാരം ആ അക്രമികളില്‍ ചിലരെ ചിലര്‍ക്ക് നാം കൂട്ടാളികളാക്കുന്നു. അവര്‍ സമ്പാദിച്ച് കൊണ്ടിരുന്നതിന്‍റെ ഫലമത്രെ അത്‌.
\end{malayalam}}
\flushright{\begin{Arabic}
\quranayah[6][130]
\end{Arabic}}
\flushleft{\begin{malayalam}
ജിന്നുകളുടെയും മനുഷ്യരുടെയും സമൂഹമേ, എന്‍റെ ദൃഷ്ടാന്തങ്ങള്‍ നിങ്ങള്‍ക്ക് വിവരിച്ചുതരികയും ഈ ദിവസത്തെ നിങ്ങള്‍ അഭിമുഖീകരിക്കേണ്ടി വരുമെന്ന് നിങ്ങള്‍ക്ക് താക്കീത് നല്‍കുകയും ചെയ്തുകൊണ്ട് നിങ്ങളില്‍ നിന്നുതന്നെയുള്ള ദൂതന്‍മാര്‍ നിങ്ങളുടെ അടുക്കല്‍ വരികയുണ്ടായില്ലേ? അവര്‍ പറഞ്ഞു: ഞങ്ങളിതാ ഞങ്ങള്‍ക്കെതിരായിത്തന്നെ സാക്ഷ്യം വഹിച്ചിരിക്കുന്നു. ഐഹികജീവിതം അവരെ വഞ്ചിച്ചു കളഞ്ഞു. തങ്ങള്‍ സത്യനിഷേധികളായിരുന്നു വെന്ന് സ്വദേഹങ്ങള്‍ക്കെതിരായി തന്നെ അവര്‍ സാക്ഷ്യം വഹിച്ചു.
\end{malayalam}}
\flushright{\begin{Arabic}
\quranayah[6][131]
\end{Arabic}}
\flushleft{\begin{malayalam}
നാട്ടുകാര്‍ (സത്യത്തെപ്പറ്റി) ബോധവാന്‍മാരല്ലാതിരിക്കെ അവര്‍ ചെയ്ത അക്രമത്തിന്‍റെ പേരില്‍ നിന്‍റെ രക്ഷിതാവ് നാടുകള്‍ നശിപ്പിക്കുന്നവനല്ല എന്നതിനാലത്രെ അത് (ദൂതന്‍മാരെ അയച്ചത്‌)
\end{malayalam}}
\flushright{\begin{Arabic}
\quranayah[6][132]
\end{Arabic}}
\flushleft{\begin{malayalam}
ഓരോരുത്തര്‍ക്കും അവരവര്‍ പ്രവര്‍ത്തിച്ചതിന്‍റെ ഫലമായി പല പദവികളുണ്ട്‌. അവര്‍ പ്രവര്‍ത്തിക്കുന്നതിനെപ്പറ്റി നിന്‍റെ രക്ഷിതാവ് ഒട്ടും അശ്രദ്ധനല്ല.
\end{malayalam}}
\flushright{\begin{Arabic}
\quranayah[6][133]
\end{Arabic}}
\flushleft{\begin{malayalam}
നിന്‍റെ രക്ഷിതാവ് പരാശ്രയമുക്തനും കാരുണ്യവാനുമാകുന്നു. അവന്‍ ഉദ്ദേശിക്കുകയാണെങ്കില്‍ നിങ്ങളെ നീക്കം ചെയ്യുകയും, നിങ്ങള്‍ക്ക് ശേഷം അവന്‍ ഉദ്ദേശിക്കുന്ന മറ്റൊരു ജനതയെ പകരം കൊണ്ടുവരുകയും ചെയ്യുന്നതാണ്‌. മറ്റൊരു ജനതയുടെ വംശപരമ്പരയില്‍ നിന്ന് നിങ്ങളെ അവന്‍ വളര്‍ത്തിയെടുത്തത് പോലെ.
\end{malayalam}}
\flushright{\begin{Arabic}
\quranayah[6][134]
\end{Arabic}}
\flushleft{\begin{malayalam}
തീര്‍ച്ചയായും നിങ്ങള്‍ക്ക് മുന്നറിയിപ്പ് നല്‍കപ്പെടുന്ന ആ കാര്യം വരിക തന്നെ ചെയ്യും. (ആ വിഷയത്തില്‍ അല്ലാഹുവെ) പരാജയപ്പെടുത്താന്‍ നിങ്ങള്‍ക്ക് കഴിയില്ല.
\end{malayalam}}
\flushright{\begin{Arabic}
\quranayah[6][135]
\end{Arabic}}
\flushleft{\begin{malayalam}
(നബിയേ,) പറയുക: എന്‍റെ ജനങ്ങളേ, നിങ്ങള്‍ നിങ്ങളുടെ നിലപാടനുസരിച്ച് പ്രവര്‍ത്തിച്ച് കൊള്ളുക. തീര്‍ച്ചയായും ഞാനും (അങ്ങനെ) പ്രവര്‍ത്തിക്കാം. ലോകത്തിന്‍റെ പര്യവസാനം ആര്‍ക്ക് അനുകൂലമായിരിക്കുമെന്ന് വഴിയെ നിങ്ങള്‍ക്കറിയാം. അക്രമികള്‍ വിജയം വരിക്കുകയില്ല; തീര്‍ച്ച.
\end{malayalam}}
\flushright{\begin{Arabic}
\quranayah[6][136]
\end{Arabic}}
\flushleft{\begin{malayalam}
അല്ലാഹു സൃഷ്ടിച്ചുണ്ടാക്കിയ കൃഷിയില്‍ നിന്നും, കന്നുകാലികളില്‍ നിന്നും അവര്‍ അവന്ന് ഒരു ഓഹരി നിശ്ചയിച്ച് കൊടുത്തിരിക്കുകയാണ്‌. എന്നിട്ട് അവരുടെ ജല്‍പനമനുസരിച്ച് ഇത് അല്ലാഹുവിനുള്ളതും, മറ്റേത് തങ്ങള്‍ പങ്കാളികളാക്കിയ ദൈവങ്ങള്‍ക്കുള്ളതുമാണെന്ന് അവര്‍ പറഞ്ഞു. എന്നാല്‍ അവരുടെ പങ്കാളികള്‍ക്കുള്ളത് അല്ലാഹുവിന്നെത്തുകയില്ല. അല്ലാഹുവിന്നുള്ളതാകട്ടെ അവരുടെ പങ്കാളികള്‍ക്കെത്തുകയും ചെയ്യും. അവര്‍ തീര്‍പ്പുകല്‍പിക്കുന്നത് എത്രമോശം!
\end{malayalam}}
\flushright{\begin{Arabic}
\quranayah[6][137]
\end{Arabic}}
\flushleft{\begin{malayalam}
അതുപോലെ തന്നെ ബഹുദൈവവാദികളില്‍പെട്ട പലര്‍ക്കും സ്വന്തം മക്കളെ കൊല്ലുന്നത് അവര്‍ പങ്കാളികളാക്കിയ ദൈവങ്ങള്‍ ഭംഗിയായി തോന്നിച്ചിരിക്കുന്നു. അവരെ നാശത്തില്‍ പെടുത്തുകയും, അവര്‍ക്ക് അവരുടെ മതം തിരിച്ചറിയാന്‍ പറ്റാതാക്കുകയുമാണ് അതുകൊണ്ടുണ്ടായിത്തീരുന്നത്‌. അല്ലാഹു ഉദ്ദേശിച്ചിരുന്നുവെങ്കില്‍ അവരത് ചെയ്യുമായിരുന്നില്ല. അതിനാല്‍ അവര്‍ കെട്ടിച്ചമച്ചുണ്ടാക്കുന്നതുമായി അവരെ വിട്ടേക്കുക.
\end{malayalam}}
\flushright{\begin{Arabic}
\quranayah[6][138]
\end{Arabic}}
\flushleft{\begin{malayalam}
അവര്‍ പറഞ്ഞു: ഇവ വിലക്കപ്പെട്ട കാലികളും കൃഷിയുമാകുന്നു. ഞങ്ങള്‍ ഉദ്ദേശിക്കുന്ന ചിലരല്ലാതെ അവ ഭക്ഷിച്ചു കൂടാ. അതവരുടെ ജല്‍പനമത്രെ. പുറത്ത് സവാരിചെയ്യുന്നത് നിഷിദ്ധമാക്കപ്പെട്ട ചില കാലികളുമുണ്ട്‌. വേറെ ചില കാലികളുമുണ്ട്‌; അവയുടെ മേല്‍ അവര്‍ അല്ലാഹുവിന്‍റെ നാമം ഉച്ചരിക്കുകയില്ല. ഇതെല്ലാം അവന്‍റെ (അല്ലാഹുവിന്‍റെ) പേരില്‍ കെട്ടിച്ചമച്ചുണ്ടാക്കിയതാണ്‌. അവര്‍ കെട്ടിച്ചമച്ച് കൊണ്ടിരുന്നതിന് തക്ക ഫലം അവന്‍ അവര്‍ക്ക് നല്‍കിക്കൊള്ളും.
\end{malayalam}}
\flushright{\begin{Arabic}
\quranayah[6][139]
\end{Arabic}}
\flushleft{\begin{malayalam}
അവര്‍ പറഞ്ഞു: ഈ കാലികളുടെ ഗര്‍ഭാശയങ്ങളിലുള്ളത് ഞങ്ങളിലെ ആണുങ്ങള്‍ക്ക് മാത്രമുള്ളതും, ഞങ്ങളുടെ ഭാര്യമാര്‍ക്ക് നിഷിദ്ധമാക്കപ്പെട്ടതുമാണ്‌. അത് ചത്തതാണെങ്കിലോ അവരെല്ലാം അതില്‍ പങ്ക് പറ്റുന്നവരായിരിക്കും. അവരുടെ ഈ ജല്‍പനത്തിന് തക്ക പ്രതിഫലം അവന്‍ (അല്ലാഹു) വഴിയെ അവര്‍ക്ക് നല്‍കുന്നതാണ്‌. തീര്‍ച്ചയായും അവന്‍ യുക്തിമാനും സര്‍വ്വജ്ഞനുമാകുന്നു.
\end{malayalam}}
\flushright{\begin{Arabic}
\quranayah[6][140]
\end{Arabic}}
\flushleft{\begin{malayalam}
ഭോഷത്വം കാരണമായി ഒരു വിവരവുമില്ലാതെ സ്വന്തം സന്താനങ്ങളെ കൊല്ലുകയും, തങ്ങള്‍ക്ക് അല്ലാഹു നല്‍കിയത് അല്ലാഹുവിന്‍റെ പേരില്‍ കള്ളം കെട്ടിച്ചമച്ചുകൊണ്ട് നിഷിദ്ധമാക്കുകയും ചെയ്തവര്‍ തീര്‍ച്ചയായും നഷ്ടത്തില്‍ പെട്ടിരിക്കുന്നു. തീര്‍ച്ചയായും അവര്‍ പിഴച്ചു പോയി. അവര്‍ നേര്‍മാര്‍ഗം പ്രാപിക്കുന്നവരായില്ല.
\end{malayalam}}
\flushright{\begin{Arabic}
\quranayah[6][141]
\end{Arabic}}
\flushleft{\begin{malayalam}
പന്തലില്‍ പടര്‍ത്തപ്പെട്ടതും അല്ലാത്തതുമായ തോട്ടങ്ങളും, ഈന്തപ്പനകളും, വിവധതരം കനികളുള്ള കൃഷികളും, പരസ്പരം തുല്യത തോന്നുന്നതും എന്നാല്‍ സാദൃശ്യമില്ലാത്തതുമായ നിലയില്‍ ഒലീവും മാതളവും എല്ലാം സൃഷ്ടിച്ചുണ്ടാക്കിയത് അവനാകുന്നു. അവയോരോന്നും കായ്ക്കുമ്പോള്‍ അതിന്‍റെ ഫലങ്ങളില്‍ നിന്ന് നിങ്ങള്‍ ഭക്ഷിച്ച് കൊള്ളുക. അതിന്‍റെ വിളവെടുപ്പ് ദിവസം അതിലുള്ള ബാധ്യത നിങ്ങള്‍ കൊടുത്ത് വീട്ടുകയും ചെയ്യുക. നിങ്ങള്‍ ദുര്‍വ്യയം ചെയ്യരുത്‌. തീര്‍ച്ചയായും ദുര്‍വ്യയം ചെയ്യുന്നവരെ അല്ലാഹു ഇഷ്ടപ്പെടുകയില്ല.
\end{malayalam}}
\flushright{\begin{Arabic}
\quranayah[6][142]
\end{Arabic}}
\flushleft{\begin{malayalam}
കാലികളില്‍ നിന്ന് ഭാരം ചുമക്കുന്നവയും, അറുത്ത് ഭക്ഷിക്കാനുള്ളവയും (അവന്‍ സൃഷ്ടിച്ചിരിക്കുന്നു.) നിങ്ങള്‍ക്ക് അല്ലാഹു നല്‍കിയതില്‍ നിന്ന് നിങ്ങള്‍ തിന്നുകൊള്ളുക. പിശാചിന്‍റെ കാലടികളെ നിങ്ങള്‍ പിന്‍പറ്റിപോകരുത്‌. തീര്‍ച്ചയായും അവന്‍ നിങ്ങളുടെ പ്രത്യക്ഷശത്രുവാകുന്നു.
\end{malayalam}}
\flushright{\begin{Arabic}
\quranayah[6][143]
\end{Arabic}}
\flushleft{\begin{malayalam}
എട്ടു ഇണകളെ (അവന്‍ സൃഷ്ടിച്ചിരിക്കുന്നു.) ചെമ്മരിയാടില്‍ നിന്ന് രണ്ടും, കോലാടില്‍ നിന്ന് രണ്ടും. പറയുക: (അവ രണ്ടിലെയും) ആണ്‍വര്‍ഗങ്ങളെയാണോ, അതല്ല, പെണ്‍വര്‍ഗങ്ങളെയാണോ, അതുമല്ല പെണ്‍വര്‍ഗങ്ങളുടെ ഗര്‍ഭാശയങ്ങള്‍ ഉള്‍കൊണ്ടതിനെയാണോ അല്ലാഹു നിഷിദ്ധമാക്കിയിട്ടുള്ളത്‌? അറിവിന്‍റെ അടിസ്ഥാനത്തില്‍ നിങ്ങള്‍ എനിക്ക് പറഞ്ഞുതരൂ; നിങ്ങള്‍ സത്യവാന്‍മാരാണെങ്കില്‍.
\end{malayalam}}
\flushright{\begin{Arabic}
\quranayah[6][144]
\end{Arabic}}
\flushleft{\begin{malayalam}
ഒട്ടകത്തില്‍ നിന്ന് രണ്ട് ഇണകളെയും, പശുവര്‍ഗത്തില്‍ നിന്ന് രണ്ട് ഇണകളെയും(അവന്‍ സൃഷ്ടിച്ചു.) പറയുക: (അവ രണ്ടിലെയും) ആണ്‍വര്‍ഗങ്ങളെയാണോ, പെണ്‍വര്‍ഗങ്ങളെയാണോ അതുമല്ല പെണ്‍വര്‍ഗങ്ങളുടെ ഗര്‍ഭാശയങ്ങള്‍ ഉള്‍കൊണ്ടതിനെയാണോ അല്ലാഹു നിഷിദ്ധമാക്കിയിട്ടുള്ളത്‌? അല്ല, അല്ലാഹു നിങ്ങളോട് ഇതൊക്കെ ഉപദേശിച്ച സന്ദര്‍ഭത്തിന് നിങ്ങള്‍ സാക്ഷികളായിട്ടുണ്ടോ? അപ്പോള്‍ ഒരു അറിവുമില്ലാതെ ജനങ്ങളെ പിഴപ്പിക്കാന്‍ വേണ്ടി അല്ലാഹുവിന്‍റെ പേരില്‍ കള്ളം കെട്ടിച്ചമച്ചവനെക്കാള്‍ വലിയ അക്രമി ആരുണ്ട്‌? അക്രമികളായ ആളുകളെ അല്ലാഹു നേര്‍വഴിയിലേക്ക് നയിക്കുകയില്ല; തീര്‍ച്ച.
\end{malayalam}}
\flushright{\begin{Arabic}
\quranayah[6][145]
\end{Arabic}}
\flushleft{\begin{malayalam}
(നബിയേ,) പറയുക: എനിക്ക് ബോധനം നല്‍കപ്പെട്ടിട്ടുള്ളതില്‍ ഒരു ഭക്ഷിക്കുന്നവന്ന് ഭക്ഷിക്കുവാന്‍ പാടില്ലാത്തതായി യാതൊന്നും ഞാന്‍ കാണുന്നില്ല; അത് ശവമോ, ഒഴുക്കപ്പെട്ട രക്തമോ, പന്നിമാംസമോ ആണെങ്കിലൊഴികെ. കാരണം അത് മ്ലേച്ഛമത്രെ. അല്ലെങ്കില്‍ അല്ലാഹുവല്ലാത്തവരുടെ പേരില്‍ (നേര്‍ച്ചയായി) പ്രഖ്യാപിക്കപ്പെട്ടതിനാല്‍ അധാര്‍മ്മികമായിത്തീര്‍ന്നിട്ടുള്ളതും ഒഴികെ. എന്നാല്‍ വല്ലവനും (ഇവ ഭക്ഷിക്കാന്‍) നിര്‍ബന്ധിതനാകുന്ന പക്ഷം അവന്‍ നിയമലംഘനം ആഗ്രഹിക്കാത്തവനും അതിരുവിട്ടുപോകാത്തവനുമാണെങ്കില്‍ നിന്‍റെ നാഥന്‍ തീര്‍ച്ചയായും ഏറെ പൊറുക്കുന്നവനും കരുണാനിധിയുമാകുന്നു.
\end{malayalam}}
\flushright{\begin{Arabic}
\quranayah[6][146]
\end{Arabic}}
\flushleft{\begin{malayalam}
നഖമുള്ള എല്ലാ ജീവികളെയും ജൂതന്‍മാര്‍ക്ക് നാം നിഷിദ്ധമാക്കുകയുണ്ടായി. പശു, ആട് എന്നീ വര്‍ഗങ്ങളില്‍ നിന്ന് അവയുടെ കൊഴുപ്പുകളും നാം അവര്‍ക്ക് നിഷിദ്ധമാക്കി. അവയുടെ മുതുകിന്‍മേലോ കുടലുകള്‍ക്ക് മീതെയോ ഉള്ളതോ, എല്ലുമായി ഒട്ടിച്ചേര്‍ന്നതോ ഒഴികെ. അവരുടെ ധിക്കാരത്തിന്ന് നാമവര്‍ക്ക് നല്‍കിയ പ്രതിഫലമത്രെ അത്‌. തീര്‍ച്ചയായും നാം സത്യം പറയുകയാകുന്നു.
\end{malayalam}}
\flushright{\begin{Arabic}
\quranayah[6][147]
\end{Arabic}}
\flushleft{\begin{malayalam}
ഇനി അവര്‍ നിന്നെ നിഷേധിച്ചുകളയുകയാണെങ്കില്‍ നീ പറഞ്ഞേക്കുക: നിങ്ങളുടെ രക്ഷിതാവ് വിശാലമായ കാരുണ്യമുള്ളവനാകുന്നു. എന്നാല്‍ കുറ്റവാളികളായ ജനങ്ങളില്‍ നിന്ന് അവന്‍റെ ശിക്ഷ ഒഴിവാക്കപ്പെടുന്നതല്ല.
\end{malayalam}}
\flushright{\begin{Arabic}
\quranayah[6][148]
\end{Arabic}}
\flushleft{\begin{malayalam}
ആ ബഹുദൈവാരാധകര്‍ പറഞ്ഞേക്കും: അല്ലാഹു ഉദ്ദേശിച്ചിരുന്നെങ്കില്‍ ഞങ്ങളോ, ഞങ്ങളുടെ പിതാക്കളോ (അല്ലാഹുവോട്‌) പങ്കുചേര്‍ക്കുമായിരുന്നില്ല; ഞങ്ങള്‍ യാതൊന്നും നിഷിദ്ധമാക്കുമായിരുന്നുമില്ല എന്ന്‌. ഇതേ പ്രകാരം അവരുടെ മുന്‍ഗാമികളും നമ്മുടെ ശിക്ഷ ആസ്വദിക്കുന്നത് വരെ നിഷേധിച്ചു കളയുകയുണ്ടായി. പറയുക: നിങ്ങളുടെ പക്കല്‍ വല്ല വിവരവുമുണ്ടോ? എങ്കില്‍ ഞങ്ങള്‍ക്ക് നിങ്ങള്‍ അതൊന്ന് വെളിപ്പെടുത്തിത്തരൂ. ഊഹത്തെ മാത്രമാണ് നിങ്ങള്‍ പിന്തുടരുന്നത്‌. നിങ്ങള്‍ അനുമാനിക്കുക മാത്രമാണ് ചെയ്യുന്നത്‌.
\end{malayalam}}
\flushright{\begin{Arabic}
\quranayah[6][149]
\end{Arabic}}
\flushleft{\begin{malayalam}
പറയുക: ആകയാല്‍ അല്ലാഹുവിനാണ് മികച്ച തെളിവുള്ളത്‌. അവന്‍ ഉദ്ദേശിച്ചിരുന്നെങ്കില്‍ നിങ്ങളെ മുഴുവന്‍ അവന്‍ നേര്‍വഴിയിലാക്കുക തന്നെ ചെയ്യുമായിരുന്നു.
\end{malayalam}}
\flushright{\begin{Arabic}
\quranayah[6][150]
\end{Arabic}}
\flushleft{\begin{malayalam}
പറയുക: അല്ലാഹു ഇതൊക്കെ നിഷിദ്ധമാക്കിയിരിക്കുന്നു എന്ന് സാക്ഷ്യപ്പെടുത്തുന്ന നിങ്ങളുടെ സാക്ഷികളെ കൊണ്ടുവരിക. ഇനി അവര്‍ (കള്ള) സാക്ഷ്യം വഹിക്കുകയാണെങ്കില്‍ നീ അവരോടൊപ്പം സാക്ഷിയാകരുത്‌. നമ്മുടെ ദൃഷ്ടാന്തങ്ങളെ നിഷേധിച്ച് തള്ളിയവരും, പരലോകത്തില്‍ വിശ്വസിക്കാത്തവരും തങ്ങളുടെ രക്ഷിതാവിന് സമന്‍മാരെവെക്കുന്നവരുമായ ആളുകളുടെ തന്നിഷ്ടങ്ങളെ നീ പിന്തുടര്‍ന്ന് പോകരുത്‌.
\end{malayalam}}
\flushright{\begin{Arabic}
\quranayah[6][151]
\end{Arabic}}
\flushleft{\begin{malayalam}
(നബിയേ,) പറയുക: നിങ്ങള്‍ വരൂ! നിങ്ങളുടെ രക്ഷിതാവ് നിങ്ങളുടെ മേല്‍ നിഷിദ്ധമാക്കിയത് നിങ്ങള്‍ക്ക് ഞാന്‍ പറഞ്ഞ് കേള്‍പിക്കാം. അവനോട് യാതൊന്നിനെയും നിങ്ങള്‍ പങ്കചേര്‍ക്കരുത്‌. മാതാപിതാക്കള്‍ക്ക് നന്‍മചെയ്യണം. ദാരിദ്ര്യം കാരണമായി സ്വന്തം മക്കളെ നിങ്ങള്‍ കൊന്നുകളയരുത്‌. നാമാണ് നിങ്ങള്‍ക്കും അവര്‍ക്കും ആഹാരം തരുന്നത്‌. പ്രത്യക്ഷവും പരോക്ഷവുമായ നീചവൃത്തികളെ നിങ്ങള്‍ സമീപിച്ച് പോകരുത്‌. അല്ലാഹു പരിപാവനമാക്കിയ ജീവനെ ന്യായപ്രകാരമല്ലാതെ നിങ്ങള്‍ ഹനിച്ചുകളയരുത്‌. നിങ്ങള്‍ ചിന്തിച്ച് മനസ്സിലാക്കുവാന്‍ വേണ്ടി. അവന്‍ (അല്ലാഹു) നിങ്ങള്‍ക്ക് നല്‍കിയ ഉപദേശമാണത്‌.
\end{malayalam}}
\flushright{\begin{Arabic}
\quranayah[6][152]
\end{Arabic}}
\flushleft{\begin{malayalam}
ഏറ്റവും ഉത്തമമായ മാര്‍ഗത്തിലൂടെയല്ലാതെ നിങ്ങള്‍ അനാഥയുടെ സ്വത്തിനെ സമീപിച്ചു പോകരുത്‌. അവന്ന് കാര്യപ്രാപ്തി എത്തുന്നത് വരെ (നിങ്ങള്‍ അവന്‍റെ രക്ഷാകര്‍ത്തൃത്വം ഏറ്റെടുക്കണം.) നിങ്ങള്‍ നീതിപൂര്‍വ്വം അളവും തൂക്കവും തികച്ചുകൊടുക്കണം. ഒരാള്‍ക്കും അയാളുടെ കഴിവിലുപരിയായി നാം ബാധ്യത ചുമത്തുന്നതല്ല. നിങ്ങള്‍ സംസാരിക്കുകയാണെങ്കില്‍ നീതി പാലിക്കുക. അതൊരു ബന്ധുവിന്‍റെ കാര്യത്തിലായിരുന്നാല്‍ പോലും. അല്ലാഹുവോടുള്ള കരാര്‍ നിങ്ങള്‍ നിറവേറ്റുക. നിങ്ങള്‍ ശ്രദ്ധിച്ച് മനസ്സിലാക്കുവാന്‍ വേണ്ടി അല്ലാഹു നിങ്ങള്‍ക്ക് നല്‍കിയ ഉപദേശമാണത്‌.
\end{malayalam}}
\flushright{\begin{Arabic}
\quranayah[6][153]
\end{Arabic}}
\flushleft{\begin{malayalam}
ഇതത്രെ എന്‍റെ നേരായ പാത. നിങ്ങള്‍ അത് പിന്തുടരുക. മറ്റുമാര്‍ഗങ്ങള്‍ പിന്‍പറ്റരുത്‌. അവയൊക്കെ അവന്‍റെ (അല്ലാഹുവിന്‍റെ) മാര്‍ഗത്തില്‍ നിന്ന് നിങ്ങളെ ചിതറിച്ച് കളയും. നിങ്ങള്‍ സൂക്ഷ്മത പാലിക്കാന്‍ വേണ്ടി അവന്‍ നിങ്ങള്‍ക്ക് നല്‍കിയ ഉപദേശമാണത്‌.
\end{malayalam}}
\flushright{\begin{Arabic}
\quranayah[6][154]
\end{Arabic}}
\flushleft{\begin{malayalam}
നന്‍മചെയ്തവന്ന് (അനുഗ്രഹത്തിന്‍റെ) പൂര്‍ത്തീകരണമായിക്കൊണ്ടും, എല്ലാകാര്യത്തിനുമുള്ള വിശദീകരണവും മാര്‍ഗദര്‍ശനവും കാരുണ്യവുമായിക്കൊണ്ടും പിന്നീട് മൂസായ്ക്ക് നാം വേദഗ്രന്ഥം നല്‍കി. തങ്ങളുടെ രക്ഷിതാവുമായുള്ള കൂടിക്കാഴ്ചയില്‍ അവര്‍ വിശ്വസിക്കുന്നവരാകാന്‍ വേണ്ടി.
\end{malayalam}}
\flushright{\begin{Arabic}
\quranayah[6][155]
\end{Arabic}}
\flushleft{\begin{malayalam}
ഇതാകട്ടെ നാം അവതരിപ്പിച്ച നന്‍മ നിറഞ്ഞ ഗ്രന്ഥമത്രെ. അതിനെ നിങ്ങള്‍ പിന്‍പറ്റുകയും സൂക്ഷ്മത പാലിക്കുകയും ചെയ്യുക. നിങ്ങള്‍ക്ക് കാരുണ്യം ലഭിച്ചേക്കാം.
\end{malayalam}}
\flushright{\begin{Arabic}
\quranayah[6][156]
\end{Arabic}}
\flushleft{\begin{malayalam}
ഞങ്ങളുടെ മുമ്പുള്ള രണ്ട് വിഭാഗങ്ങള്‍ക്ക് മാത്രമേ വേദഗ്രന്ഥം അവതരിപ്പിക്കപ്പെട്ടിട്ടുള്ളൂ. അവര്‍ വായിച്ചുപഠിച്ചു കൊണ്ടിരിക്കുന്നതിനെപ്പറ്റി ഞങ്ങള്‍ തീര്‍ത്തും ധാരണയില്ലാത്തവരായിരുന്നു എന്ന് നിങ്ങള്‍ പറഞ്ഞേക്കാം എന്നതിനാലാണ് (ഇതവതരിപ്പിച്ചത്‌.)
\end{malayalam}}
\flushright{\begin{Arabic}
\quranayah[6][157]
\end{Arabic}}
\flushleft{\begin{malayalam}
അല്ലെങ്കില്‍, ഞങ്ങള്‍ക്ക് ഒരു വേദഗ്രന്ഥം അവതരിച്ച് കിട്ടിയിരുന്നെങ്കില്‍ ഞങ്ങള്‍ അവരെക്കാള്‍ സന്‍മാര്‍ഗികളാകുമായിരുന്നു എന്ന് നിങ്ങള്‍ പറഞ്ഞേക്കാം എന്നതിനാല്‍. അങ്ങനെ നിങ്ങള്‍ക്കിതാ നിങ്ങളുടെ രക്ഷിതാവിങ്കല്‍ നിന്ന് വ്യക്തമായ പ്രമാണവും മാര്‍ഗദര്‍ശനവും കാരുണ്യവും വന്നുകിട്ടിയിരിക്കുന്നു. എന്നിട്ടും അല്ലാഹുവിന്‍റെ തെളിവുകളെ നിഷേധിച്ചുതള്ളുകയും, അവയില്‍ നിന്ന് തിരിഞ്ഞുകളയുകയും ചെയ്തവനെക്കാള്‍ കടുത്ത അക്രമി ആരുണ്ട്‌? നമ്മുടെ തെളിവുകളില്‍ നിന്ന് തിരിഞ്ഞ് കളയുന്നവര്‍ക്ക് അവര്‍ തിരിഞ്ഞ് കളഞ്ഞുകൊണ്ടിരുന്നതിന് പ്രതിഫലമായി നാം കടുത്ത ശിക്ഷ നല്‍കുന്നതാണ്‌.
\end{malayalam}}
\flushright{\begin{Arabic}
\quranayah[6][158]
\end{Arabic}}
\flushleft{\begin{malayalam}
തങ്ങളുടെ അടുക്കല്‍ മലക്കുകള്‍ വരുന്നതോ, നിന്‍റെ രക്ഷിതാവ് തന്നെ വരുന്നതോ, നിന്‍റെ രക്ഷിതാവിന്‍റെ ഏതെങ്കിലുമൊരു ദൃഷ്ടാന്തം വരുന്നതോ അല്ലാതെ മറ്റെന്താണവര്‍ കാത്തിരിക്കുന്നത്‌? നിന്‍റെ രക്ഷിതാവിന്‍റെ ഏതെങ്കിലുമൊരു ദൃഷ്ടാന്തം വരുന്ന ദിവസം, മുമ്പ് തന്നെ വിശ്വസിക്കുകയോ, വിശ്വാസത്തോട് കൂടി വല്ല നന്‍മയും ചെയ്ത് വെക്കുകയോ ചെയ്തിട്ടില്ലാത്ത യാതൊരാള്‍ക്കും തന്‍റെ വിശ്വാസം പ്രയോജനപ്പെടുന്നതല്ല.പറയുക: നിങ്ങള്‍ കാത്തിരിക്കൂ; ഞങ്ങളും കാത്തിരിക്കുകയാണ്‌.
\end{malayalam}}
\flushright{\begin{Arabic}
\quranayah[6][159]
\end{Arabic}}
\flushleft{\begin{malayalam}
തങ്ങളുടെ മതത്തില്‍ ഭിന്നതയുണ്ടാക്കുകയും, കക്ഷികളായിത്തീരുകയും ചെയ്തവരാരോ അവരുമായി നിനക്ക് യാതൊരു ബന്ധവുമില്ല. അവരുടെ കാര്യം അല്ലാഹുവിങ്കലേക്ക് തന്നെയാണ് (മടക്കപ്പെടുന്നത്‌.) അവര്‍ ചെയ്തു കൊണ്ടിരുന്നതിനെപ്പറ്റി അവന്‍ അവരെ അറിയിച്ച് കൊള്ളും.
\end{malayalam}}
\flushright{\begin{Arabic}
\quranayah[6][160]
\end{Arabic}}
\flushleft{\begin{malayalam}
വല്ലവനും ഒരു നന്‍മ കൊണ്ടു വന്നാല്‍ അവന്ന് അതിന്‍റെ പതിന്‍മടങ്ങ് ലഭിക്കുന്നതാണ്‌. വല്ലവനും ഒരു തിന്‍മകൊണ്ടു വന്നാല്‍ അതിന് തുല്യമായ പ്രതിഫലം മാത്രമേ അവന്ന് നല്‍കപ്പെടുകയുള്ളൂ. അവരോട് യാതൊരു അനീതിയും കാണിക്കപ്പെടുകയില്ല.
\end{malayalam}}
\flushright{\begin{Arabic}
\quranayah[6][161]
\end{Arabic}}
\flushleft{\begin{malayalam}
പറയുക: തീര്‍ച്ചയായും എന്‍റെ രക്ഷിതാവ് എന്നെ നേരായ പാതയിലേക്ക് നയിച്ചിരിക്കുന്നു. വക്രതയില്ലാത്ത മതത്തിലേക്ക്‌. നേര്‍മാര്‍ഗത്തില്‍ നിലകൊണ്ട ഇബ്രാഹീമിന്‍റെ ആദര്‍ശത്തിലേക്ക്‌. അദ്ദേഹം ബഹുദൈവവാദികളില്‍ പെട്ടവനായിരുന്നില്ല.
\end{malayalam}}
\flushright{\begin{Arabic}
\quranayah[6][162]
\end{Arabic}}
\flushleft{\begin{malayalam}
പറയുക: തീര്‍ച്ചയായും എന്‍റെ പ്രാര്‍ത്ഥനയും, എന്‍റെ ആരാധനാകര്‍മ്മങ്ങളും, എന്‍റെ ജീവിതവും, എന്‍റെ മരണവും ലോകരക്ഷിതാവായ അല്ലാഹുവിന്നുള്ളതാകുന്നു.
\end{malayalam}}
\flushright{\begin{Arabic}
\quranayah[6][163]
\end{Arabic}}
\flushleft{\begin{malayalam}
അവന്ന് പങ്കുകാരേയില്ല. അപ്രകാരമാണ് ഞാന്‍ കല്‍പിക്കപ്പെട്ടിരിക്കുന്നത്‌. (അവന്ന്‌) കീഴ്പെടുന്നവരില്‍ ഞാന്‍ ഒന്നാമനാണ്‌.
\end{malayalam}}
\flushright{\begin{Arabic}
\quranayah[6][164]
\end{Arabic}}
\flushleft{\begin{malayalam}
പറയുക: രക്ഷിതാവായിട്ട് അല്ലാഹുവല്ലാത്തവരെ ഞാന്‍ തേടുകയോ? അവനാകട്ടെ മുഴുവന്‍ വസ്തുക്കളുടെയും രക്ഷിതാവാണ്‌. ഏതൊരാളും ചെയ്ത് വെക്കുന്നതിന്‍റെ ഉത്തരവാദിത്തം അയാള്‍ക്ക് മാത്രമായിരിക്കും. ഭാരം ചുമക്കുന്ന യാതൊരാളും മറ്റൊരാളുടെ ഭാരം ചുമക്കുന്നതല്ല. അനന്തരം നിങ്ങളുടെ രക്ഷിതാവിങ്കലേക്കാണ് നിങ്ങളുടെ മടക്കം. ഏതൊരു കാര്യത്തില്‍ നിങ്ങള്‍ അഭിപ്രായഭിന്നത പുലര്‍ത്തിയിരുന്നുവോ അതിനെപ്പറ്റി അപ്പോള്‍ അവന്‍ നിങ്ങളെ അറിയിക്കുന്നതാണ്‌.
\end{malayalam}}
\flushright{\begin{Arabic}
\quranayah[6][165]
\end{Arabic}}
\flushleft{\begin{malayalam}
അവനാണ് നിങ്ങളെ ഭൂമിയില്‍ പിന്തുടര്‍ച്ചാവകാശികളാക്കിയത്‌. നിങ്ങളില്‍ ചിലരെ ചിലരെക്കാള്‍ പദവികളില്‍ അവന്‍ ഉയര്‍ത്തുകയും ചെയ്തിരിക്കുന്നു. നിങ്ങള്‍ക്കവന്‍ നല്‍കിയതില്‍ നിങ്ങളെ പരീക്ഷിക്കാന്‍ വേണ്ടിയത്രെ അത്‌. തീര്‍ച്ചയായും അവന്‍ ഏറെ പൊറുക്കുന്നവനും കരുണാനിധിയും കൂടിയാകുന്നു.
\end{malayalam}}
\chapter{\textmalayalam{അഅ്റാഫ് ( ഉന്നതസ്ഥലങ്ങള്‍‍ )}}
\begin{Arabic}
\Huge{\centerline{\basmalah}}\end{Arabic}
\flushright{\begin{Arabic}
\quranayah[7][1]
\end{Arabic}}
\flushleft{\begin{malayalam}
അലിഫ്‌-ലാം-മീം-സ്വാദ്‌.
\end{malayalam}}
\flushright{\begin{Arabic}
\quranayah[7][2]
\end{Arabic}}
\flushleft{\begin{malayalam}
(നബിയേ,) നിനക്ക് അവതരിപ്പിക്കപ്പെട്ട ഗ്രന്ഥമത്രെ ഇത്‌. അതിനെ സംബന്ധിച്ച് നിന്‍റെ മനസ്സില്‍ ഒരു പ്രയാസവും ഉണ്ടായിരിക്കരുത്‌. അതു മുഖേന നീ താക്കീത് നല്‍കുവാന്‍ വേണ്ടിയും സത്യവിശ്വാസികള്‍ക്ക് ഉല്‍ബോധനം നല്‍കുവാന്‍ വേണ്ടിയുമാണത്‌.
\end{malayalam}}
\flushright{\begin{Arabic}
\quranayah[7][3]
\end{Arabic}}
\flushleft{\begin{malayalam}
നിങ്ങളുടെ രക്ഷിതാവിങ്കല്‍ നിന്ന് നിങ്ങള്‍ക്കായി അവതരിപ്പിക്കപ്പെട്ടത് നിങ്ങള്‍ പിന്‍പറ്റുക. അവനു പുറമെ മറ്റു രക്ഷാധികാരികളെ നിങ്ങള്‍ പിന്‍പറ്റരുത്‌. വളരെ കുറച്ച് മാത്രമേ നിങ്ങള്‍ ആലോചിച്ച് മനസ്സിലാക്കുന്നുള്ളൂ.
\end{malayalam}}
\flushright{\begin{Arabic}
\quranayah[7][4]
\end{Arabic}}
\flushleft{\begin{malayalam}
എത്രയോ രാജ്യങ്ങളെ നാം നശിപ്പിച്ചിട്ടുണ്ട്‌. രാത്രിയിലോ, അവര്‍ ഉച്ചയുറക്കത്തിലായിരിക്കുമ്പോഴോ നമ്മുടെ ശിക്ഷ അവര്‍ക്ക് വന്നുഭവിച്ചു.
\end{malayalam}}
\flushright{\begin{Arabic}
\quranayah[7][5]
\end{Arabic}}
\flushleft{\begin{malayalam}
അവര്‍ക്ക് നമ്മുടെ ശിക്ഷ വന്നുഭവിച്ചപ്പോള്‍ ഞങ്ങള്‍ അക്രമികളായിപ്പോയല്ലോ എന്ന് പറയുക മാത്രമായിരുന്നു അവരുടെ മുറവിളി.
\end{malayalam}}
\flushright{\begin{Arabic}
\quranayah[7][6]
\end{Arabic}}
\flushleft{\begin{malayalam}
എന്നാല്‍ (നമ്മുടെ ദൂതന്‍മാര്‍) ആര്‍ക്കിടയിലേക്ക് അയക്കപ്പെട്ടുവോ അവരെ തീര്‍ച്ചയായും നാം ചോദ്യം ചെയ്യും. അയക്കപ്പെട്ട ദൂതന്‍മാരെയും തീര്‍ച്ചയായും നാം ചോദ്യം ചെയ്യും.
\end{malayalam}}
\flushright{\begin{Arabic}
\quranayah[7][7]
\end{Arabic}}
\flushleft{\begin{malayalam}
എന്നിട്ട് ശരിയായ അറിവോടുകൂടി നാം അവര്‍ക്കു (കാര്യം) വിവരിച്ചുകൊടുക്കുന്നതാണ്‌. ഒരിക്കലും നമ്മുടെ അസാന്നിദ്ധ്യം ഉണ്ടായിട്ടില്ല.
\end{malayalam}}
\flushright{\begin{Arabic}
\quranayah[7][8]
\end{Arabic}}
\flushleft{\begin{malayalam}
അന്നത്തെ ദിവസം (കര്‍മ്മങ്ങള്‍) തൂക്കികണക്കാക്കുന്നത് സത്യമായിരിക്കും. അപ്പോള്‍ ആരുടെ തുലാസുകള്‍ ഘനം തൂങ്ങിയോ അവരാണ് വിജയികള്‍.
\end{malayalam}}
\flushright{\begin{Arabic}
\quranayah[7][9]
\end{Arabic}}
\flushleft{\begin{malayalam}
ആരുടെ തുലാസുകള്‍ ഘനം കുറഞ്ഞുവോ അവരാണ് ആത്മനഷ്ടം നേരിട്ടവര്‍. നമ്മുടെ ദൃഷ്ടാന്തങ്ങളുടെ നേരെ അവര്‍ അന്യായം കൈക്കൊണ്ടിരുന്നതിന്‍റെ ഫലമത്രെ അത്‌.
\end{malayalam}}
\flushright{\begin{Arabic}
\quranayah[7][10]
\end{Arabic}}
\flushleft{\begin{malayalam}
നിങ്ങള്‍ക്കു നാം ഭൂമിയില്‍ സ്വാധീനം നല്‍കുകയും, നിങ്ങള്‍ക്കവിടെ നാം ജീവിതമാര്‍ഗങ്ങള്‍ ഏര്‍പെടുത്തുകയും ചെയ്തിരിക്കുന്നു. കുറച്ച് മാത്രമേ നിങ്ങള്‍ നന്ദികാണിക്കുന്നുള്ളൂ.
\end{malayalam}}
\flushright{\begin{Arabic}
\quranayah[7][11]
\end{Arabic}}
\flushleft{\begin{malayalam}
തീര്‍ച്ചയായും നാം നിങ്ങളെ സൃഷ്ടിക്കുകയും, നിങ്ങള്‍ക്ക് രൂപം നല്‍കുകയും ചെയ്തു. പിന്നീട് നാം മലക്കുകളോട് പറഞ്ഞു: നിങ്ങള്‍ ആദമിനെ പ്രണമിക്കുക. അവര്‍ പ്രണമിച്ചു; ഇബ്ലീസൊഴികെ. അവന്‍ പ്രണമിച്ചവരുടെ കൂട്ടത്തിലായില്ല.
\end{malayalam}}
\flushright{\begin{Arabic}
\quranayah[7][12]
\end{Arabic}}
\flushleft{\begin{malayalam}
അവന്‍ (അല്ലാഹു) പറഞ്ഞു: ഞാന്‍ നിന്നോട് കല്‍പിച്ചപ്പോള്‍ സുജൂദ് ചെയ്യാതിരിക്കാന്‍ നിനക്കെന്ത് തടസ്സമായിരുന്നു ? അവന്‍ പറഞ്ഞു: ഞാന്‍ അവനെക്കാള്‍ (ആദമിനെക്കാള്‍) ഉത്തമനാകുന്നു. എന്നെ നീ അഗ്നിയില്‍ നിന്നാണ് സൃഷ്ടിച്ചത്‌. അവനെ നീ സൃഷ്ടിച്ചത് കളിമണ്ണില്‍ നിന്നും.
\end{malayalam}}
\flushright{\begin{Arabic}
\quranayah[7][13]
\end{Arabic}}
\flushleft{\begin{malayalam}
അവന്‍ (അല്ലാഹു) പറഞ്ഞു: നീ ഇവിടെ നിന്ന് ഇറങ്ങിപ്പോകുക. ഇവിടെ നിനക്ക് അഹങ്കാരം കാണിക്കാന്‍ പറ്റുകയില്ല. തീര്‍ച്ചയായും നീ നിന്ദ്യരുടെ കൂട്ടത്തിലാകുന്നു.
\end{malayalam}}
\flushright{\begin{Arabic}
\quranayah[7][14]
\end{Arabic}}
\flushleft{\begin{malayalam}
അവന്‍ പറഞ്ഞു: മനുഷ്യര്‍ ഉയിര്‍ത്തെഴുന്നേല്‍പിക്കപ്പെടുന്ന ദിവസം വരെ നീ എനിക്ക് അവധി നല്‍കേണമേ.
\end{malayalam}}
\flushright{\begin{Arabic}
\quranayah[7][15]
\end{Arabic}}
\flushleft{\begin{malayalam}
അവന്‍ (അല്ലാഹു) പറഞ്ഞു: തീര്‍ച്ചയായും നീ അവധി നല്‍കപ്പെട്ടവരുടെ കൂട്ടത്തിലാകുന്നു.
\end{malayalam}}
\flushright{\begin{Arabic}
\quranayah[7][16]
\end{Arabic}}
\flushleft{\begin{malayalam}
അവന്‍ (ഇബ്ലീസ്‌) പറഞ്ഞു: നീ എന്നെ വഴിപിഴപ്പിച്ചതിനാല്‍ നിന്‍റെ നേരായ പാതയില്‍ അവര്‍ (മനുഷ്യര്‍) പ്രവേശിക്കുന്നത് തടയാന്‍ ഞാന്‍ കാത്തിരിക്കും.
\end{malayalam}}
\flushright{\begin{Arabic}
\quranayah[7][17]
\end{Arabic}}
\flushleft{\begin{malayalam}
പിന്നീട് അവരുടെ മുന്നിലൂടെയും, അവരുടെ പിന്നിലൂടെയും, അവരുടെ വലതുഭാഗങ്ങളിലൂടെയും, ഇടതുഭാഗങ്ങളിലൂടെയും ഞാന്‍ അവരുടെ അടുത്ത് ചെല്ലുക തന്നെ ചെയ്യും. അവരില്‍ അധികപേരെയും നന്ദിയുള്ളവരായി നീ കണ്ടെത്തുന്നതല്ല.
\end{malayalam}}
\flushright{\begin{Arabic}
\quranayah[7][18]
\end{Arabic}}
\flushleft{\begin{malayalam}
അവന്‍ (അല്ലാഹു) പറഞ്ഞു: നിന്ദ്യനും തള്ളപ്പെട്ടവനുമായിക്കൊണ്ട് നീ ഇവിടെ നിന്ന് പുറത്ത് കടക്കൂ. അവരില്‍ നിന്ന് വല്ലവരും നിന്നെ പിന്‍പറ്റുന്ന പക്ഷം നിങ്ങളെല്ലാവരെയും കൊണ്ട് ഞാന്‍ നരകം നിറക്കുക തന്നെ ചെയ്യും.
\end{malayalam}}
\flushright{\begin{Arabic}
\quranayah[7][19]
\end{Arabic}}
\flushleft{\begin{malayalam}
ആദമേ, നീയും നിന്‍റെ ഇണയും കൂടി ഈ തോട്ടത്തില്‍ താമസിക്കുകയും, നിങ്ങള്‍ക്ക് ഇഷ്ടമുള്ളേടത്ത് നിന്ന് തിന്നുകൊള്ളുകയും ചെയ്യുക. എന്നാല്‍ ഈ വൃക്ഷത്തെ നിങ്ങള്‍ സമീപിച്ചു പോകരുത്‌. എങ്കില്‍ നിങ്ങള്‍ ഇരുവരും അക്രമികളില്‍ പെട്ടവരായിരിക്കും എന്നും (അല്ലാഹു പറഞ്ഞു.)
\end{malayalam}}
\flushright{\begin{Arabic}
\quranayah[7][20]
\end{Arabic}}
\flushleft{\begin{malayalam}
അവരില്‍ നിന്ന് മറച്ചു വെക്കപ്പെട്ടിരുന്ന അവരുടെ ഗോപ്യസ്ഥാനങ്ങള്‍ അവര്‍ക്കു വെളിപ്പെടുത്തുവാനായി പിശാച് അവര്‍ ഇരുവരോടും ദുര്‍മന്ത്രണം നടത്തി. അവന്‍ പറഞ്ഞു: നിങ്ങളുടെ രക്ഷിതാവ് ഈ വൃക്ഷത്തില്‍ നിന്ന് നിങ്ങള്‍ ഇരുവരെയും വിലക്കിയിട്ടുള്ളത് നിങ്ങള്‍ ഇരുവരും മലക്കുകളായിത്തീരുമെന്നത് കൊണ്ടോ, നിങ്ങള്‍ ഇവിടെ നിത്യവാസികളായിത്തീരുമെന്നത് കൊണ്ടോ അല്ലാതെ മറ്റൊന്നുകൊണ്ടുമല്ല.
\end{malayalam}}
\flushright{\begin{Arabic}
\quranayah[7][21]
\end{Arabic}}
\flushleft{\begin{malayalam}
തീര്‍ച്ചയായും ഞാന്‍ നിങ്ങളിരുവരുടെയും ഗുണകാംക്ഷികളില്‍പ്പെട്ടവനാണ് എന്ന് അവരോട് അവന്‍ സത്യം ചെയ്ത് പറയുകയും ചെയ്തു.
\end{malayalam}}
\flushright{\begin{Arabic}
\quranayah[7][22]
\end{Arabic}}
\flushleft{\begin{malayalam}
അങ്ങനെ അവര്‍ ഇരുവരെയും വഞ്ചനയിലൂടെ അവന്‍ തരംതാഴ്ത്തിക്കളഞ്ഞു. അവര്‍ ഇരുവരും ആ വൃക്ഷത്തില്‍ നിന്ന് രുചി നോക്കിയതോടെ അവര്‍ക്ക് അവരുടെ ഗോപ്യസ്ഥാനങ്ങള്‍ വെളിപ്പെട്ടു. ആ തോട്ടത്തിലെ ഇലകള്‍ കൂട്ടിചേര്‍ത്ത് അവര്‍ ഇരുവരും തങ്ങളുടെ ശരീരം പൊതിയാന്‍ തുടങ്ങി. അവര്‍ ഇരുവരെയും വിളിച്ച് അവരുടെ രക്ഷിതാവ് പറഞ്ഞു: ആ വൃക്ഷത്തില്‍ നിന്ന് നിങ്ങളെ ഞാന്‍ വിലക്കിയിട്ടില്ലേ? തീര്‍ച്ചയായും പിശാച് നിങ്ങളുടെ പ്രത്യക്ഷശത്രുവാണെന്ന് ഞാന്‍ നിങ്ങളോട് പറഞ്ഞിട്ടുമില്ലേ?
\end{malayalam}}
\flushright{\begin{Arabic}
\quranayah[7][23]
\end{Arabic}}
\flushleft{\begin{malayalam}
അവര്‍ രണ്ടുപേരും പറഞ്ഞു: ഞങ്ങളുടെ രക്ഷിതാവേ, ഞങ്ങള്‍ ഞങ്ങളോട് തന്നെ അക്രമം ചെയ്തിരിക്കുന്നു. നീ ഞങ്ങള്‍ക്ക് പൊറുത്തുതരികയും, കരുണ കാണിക്കുകയും ചെയ്തില്ലെങ്കില്‍ തീര്‍ച്ചയായും ഞങ്ങള്‍ നഷ്ടം പറ്റിയവരുടെ കൂട്ടത്തിലായിരിക്കും.
\end{malayalam}}
\flushright{\begin{Arabic}
\quranayah[7][24]
\end{Arabic}}
\flushleft{\begin{malayalam}
അവന്‍ (അല്ലാഹു) പറഞ്ഞു: നിങ്ങള്‍ ഇറങ്ങിപ്പോകൂ. നിങ്ങളില്‍ ചിലര്‍ ചിലര്‍ക്ക് ശത്രുക്കളായിരിക്കും. നിങ്ങള്‍ക്ക് ഭൂമിയില്‍ വാസസ്ഥലമുണ്ട്‌. ഒരു നിശ്ചിതസമയം വരെ ജീവിതസൌകര്യങ്ങളുമുണ്ട്‌.
\end{malayalam}}
\flushright{\begin{Arabic}
\quranayah[7][25]
\end{Arabic}}
\flushleft{\begin{malayalam}
അവന്‍ പറഞ്ഞു: അതില്‍ (ഭൂമിയില്‍) തന്നെ നിങ്ങള്‍ ജീവിക്കും. അവിടെ തന്നെ നിങ്ങള്‍ മരിക്കും. അവിടെ നിന്ന് തന്നെ നിങ്ങള്‍ പുറത്ത് കൊണ്ട് വരപ്പെടുകയും ചെയ്യും.
\end{malayalam}}
\flushright{\begin{Arabic}
\quranayah[7][26]
\end{Arabic}}
\flushleft{\begin{malayalam}
ആദം സന്തതികളേ, നിങ്ങള്‍ക്കു നാം നിങ്ങളുടെ ഗോപ്യസ്ഥാനങ്ങള്‍ മറയ്ക്കാനുതകുന്ന വസ്ത്രവും അലങ്കാരവസ്ത്രവും നല്‍കിയിരിക്കുന്നു. ധര്‍മ്മനിഷ്ഠയാകുന്ന വസ്ത്രമാകട്ടെ അതാണു കൂടുതല്‍ ഉത്തമം. അവര്‍ ശ്രദ്ധിച്ച് മനസ്സിലാക്കാന്‍ വേണ്ടി അല്ലാഹു അവതരിപ്പിക്കുന്ന തെളിവുകളില്‍ പെട്ടതത്രെ അത്‌.
\end{malayalam}}
\flushright{\begin{Arabic}
\quranayah[7][27]
\end{Arabic}}
\flushleft{\begin{malayalam}
ആദം സന്തതികളേ, നിങ്ങളുടെ മാതാപിതാക്കളെ ആ തോട്ടത്തില്‍ നിന്ന് പുറത്താക്കിയത് പോലെ പിശാച് നിങ്ങളെ കുഴപ്പത്തിലാക്കാതിരിക്കട്ടെ. അവര്‍ ഇരുവരുടെയും ഗോപ്യസ്ഥാനങ്ങള്‍ അവര്‍ക്ക് കാണിച്ചുകൊടുക്കുവാനായി അവന്‍ അവരില്‍ നിന്ന് അവരുടെ വസ്ത്രം എടുത്തുനീക്കുകയായിരുന്നു. തീര്‍ച്ചയായും അവനും അവന്‍റെ വര്‍ഗക്കാരും നിങ്ങളെ കണ്ടുകൊണ്ടിരിക്കും; നിങ്ങള്‍ക്ക് അവരെ കാണാന്‍ പറ്റാത്ത വിധത്തില്‍. തീര്‍ച്ചയായും വിശ്വസിക്കാത്തവര്‍ക്ക് പിശാചുക്കളെ നാം മിത്രങ്ങളാക്കി കൊടുത്തിരിക്കുന്നു.
\end{malayalam}}
\flushright{\begin{Arabic}
\quranayah[7][28]
\end{Arabic}}
\flushleft{\begin{malayalam}
അവര്‍ വല്ല നീചവൃത്തിയും ചെയ്താല്‍, ഞങ്ങളുടെ പിതാക്കള്‍ അതില്‍ നിലകൊള്ളുന്നതായി ഞങ്ങള്‍ കണ്ടിട്ടുണ്ടെന്നും, അല്ലാഹു ഞങ്ങളോട് കല്‍പിച്ചതാണത് എന്നുമാണവര്‍ പറയുക. (നബിയേ,) പറയുക: നീചവൃത്തി ചെയ്യുവാന്‍ അല്ലാഹു കല്‍പിക്കുകയേയില്ല. നിങ്ങള്‍ അല്ലാഹുവിന്‍റെ പേരില്‍ നിങ്ങള്‍ക്ക് വിവരമില്ലാത്തത് പറഞ്ഞുണ്ടാക്കുകയാണോ?
\end{malayalam}}
\flushright{\begin{Arabic}
\quranayah[7][29]
\end{Arabic}}
\flushleft{\begin{malayalam}
പറയുക: എന്‍റെ രക്ഷിതാവ് നീതിപാലിക്കാനാണ് കല്‍പിച്ചിട്ടുള്ളത്‌. എല്ലാ ആരാധനാവേളയിലും (അഥവാ എല്ലാ ആരാധനാലയങ്ങളിലും) നിങ്ങളുടെ മുഖങ്ങളെ ശരിയാം വിധം (അവനിലേക്ക് തിരിച്ച്‌) നിര്‍ത്തുകയും കീഴ്‌വണക്കം അവന് മാത്രമാക്കി കൊണ്ട് അവനോട് പ്രാര്‍ത്ഥിക്കുകയും ചെയ്യുവിന്‍. നിങ്ങളെ അവന്‍ ആദ്യമായി സൃഷ്ടിച്ചുണ്ടാക്കിയതുപോലുള്ള അവസ്ഥയിലേക്ക് തന്നെ നിങ്ങള്‍ മടങ്ങുന്നതാകുന്നു.
\end{malayalam}}
\flushright{\begin{Arabic}
\quranayah[7][30]
\end{Arabic}}
\flushleft{\begin{malayalam}
ഒരു വിഭാഗത്തെ അവന്‍ നേര്‍വഴിയിലാക്കിയിരിക്കുന്നു. ഒരു വിഭാഗമാകട്ടെ വഴിപിഴക്കാന്‍ അര്‍ഹരായിരിക്കുന്നു. അല്ലാഹുവിനെ വിട്ട് പിശാചുക്കളെയാണ് അവര്‍ രക്ഷാധികാരികളാക്കി വെച്ചിരിക്കുന്നത്‌. തങ്ങള്‍ സന്‍മാര്‍ഗം പ്രാപിച്ചവരാണെന്ന് അവര്‍ വിചാരിക്കുകയും ചെയ്യുന്നു.
\end{malayalam}}
\flushright{\begin{Arabic}
\quranayah[7][31]
\end{Arabic}}
\flushleft{\begin{malayalam}
ആദം സന്തതികളേ, എല്ലാ ആരാധനാലയത്തിങ്കലും (അഥവാ എല്ലാ ആരാധനാവേളകളിലും) നിങ്ങള്‍ക്ക് അലങ്കാരമായിട്ടുള്ള വസ്ത്രങ്ങള്‍ ധരിച്ചുകൊള്ളുക നിങ്ങള്‍ തിന്നുകയും കുടിക്കുകയും ചെയ്തു കൊള്ളുക. എന്നാല്‍ നിങ്ങള്‍ ദുര്‍വ്യയം ചെയ്യരുത്‌. ദുര്‍വ്യയം ചെയ്യുന്നവരെ അല്ലാഹു ഇഷ്ടപ്പെടുകയേയില്ല.
\end{malayalam}}
\flushright{\begin{Arabic}
\quranayah[7][32]
\end{Arabic}}
\flushleft{\begin{malayalam}
(നബിയേ,) പറയുക: അല്ലാഹു അവന്‍റെ ദാസന്‍മാര്‍ക്ക് വേണ്ടി ഉല്‍പാദിപ്പിച്ചിട്ടുള്ള അലങ്കാര വസ്തുക്കളും വിശിഷ്ടമായ ആഹാരപദാര്‍ത്ഥങ്ങളും നിഷിദ്ധമാക്കിയതാരാണ്‌? പറയുക: അവ ഐഹികജീവിതത്തില്‍ സത്യവിശ്വാസികള്‍ക്ക് അവകാശപ്പെട്ടതാണ്‌. ഉയിര്‍ത്തെഴുന്നേല്‍പിന്‍റെ നാളില്‍ അവര്‍ക്കുമാത്രമുള്ളതുമാണ്‌. മനസ്സിലാക്കുന്ന ആളുകള്‍ക്ക് വേണ്ടി അപ്രകാരം നാം തെളിവുകള്‍ വിശദീകരിക്കുന്നു.
\end{malayalam}}
\flushright{\begin{Arabic}
\quranayah[7][33]
\end{Arabic}}
\flushleft{\begin{malayalam}
പറയുക: എന്‍റെ രക്ഷിതാവ് നിഷിദ്ധമാക്കിയിട്ടുള്ളത് പ്രത്യക്ഷമായതും പരോക്ഷമായതുമായ നീചവൃത്തികളും, അധര്‍മ്മവും, ന്യായം കൂടാതെയുള്ള കയ്യേറ്റവും, യാതൊരു പ്രമാണവും അല്ലാഹു ഇറക്കിത്തന്നിട്ടില്ലാത്തതിനെ അവനോട് നിങ്ങള്‍ പങ്കുചേര്‍ക്കുന്നതും, അല്ലാഹുവിന്‍റെ പേരില്‍ നിങ്ങള്‍ക്കു വിവരമില്ലാത്തത് നിങ്ങള്‍ പറഞ്ഞുണ്ടാക്കുന്നതും മാത്രമാണ്‌.
\end{malayalam}}
\flushright{\begin{Arabic}
\quranayah[7][34]
\end{Arabic}}
\flushleft{\begin{malayalam}
ഓരോ സമുദായത്തിനും ഓരോ അവധിയുണ്ട്‌. അങ്ങനെ അവരുടെ അവധി വന്നെത്തിയാല്‍ അവര്‍ ഒരു നാഴിക നേരം പോലും വൈകിക്കുകയോ, നേരത്തെ ആക്കുകയോ ഇല്ല.
\end{malayalam}}
\flushright{\begin{Arabic}
\quranayah[7][35]
\end{Arabic}}
\flushleft{\begin{malayalam}
ആദം സന്തതികളേ, നിങ്ങള്‍ക്ക് എന്‍റെ ദൃഷ്ടാന്തങ്ങള്‍ വിവരിച്ചുതന്നു കൊണ്ട് നിങ്ങളില്‍ നിന്നു തന്നെയുള്ള ദൂതന്‍മാര്‍ നിങ്ങളുടെ അടുത്ത് വരുന്ന പക്ഷം അപ്പോള്‍ സൂക്ഷ്മത പാലിക്കുകയും, നിലപാട് നന്നാക്കിത്തീര്‍ക്കുകയും ചെയ്യുന്നതാരോ അവര്‍ക്ക് യാതൊന്നും ഭയപ്പെടേണ്ടതില്ല. അവര്‍ ദുഃഖിക്കേണ്ടി വരികയുമില്ല.
\end{malayalam}}
\flushright{\begin{Arabic}
\quranayah[7][36]
\end{Arabic}}
\flushleft{\begin{malayalam}
എന്നാല്‍ നമ്മുടെ ദൃഷ്ടാന്തങ്ങളെ നിഷേധിച്ചു തള്ളുകയും, അവയുടെ നേരെ അഹങ്കാരം നടിക്കുകയും ചെയ്യുന്നതാരോ അവരാണ് നരകാവകാശികള്‍. അവര്‍ അതില്‍ നിത്യവാസികളായിരിക്കും.
\end{malayalam}}
\flushright{\begin{Arabic}
\quranayah[7][37]
\end{Arabic}}
\flushleft{\begin{malayalam}
അപ്പോള്‍ അല്ലാഹുവിന്‍റെ പേരില്‍ കള്ളം കെട്ടിച്ചമയ്ക്കുകയോ, അവന്‍റെ തെളിവുകളെ നിഷേധിച്ചുതള്ളുകയോ ചെയ്തവനേക്കാള്‍ കടുത്ത അക്രമി ആരുണ്ട്‌? (അല്ലാഹുവിന്‍റെ) രേഖയില്‍ തങ്ങള്‍ക്ക് നിശ്ചയിച്ചിട്ടുള്ള ഓഹരി അത്തരക്കാര്‍ക്കു ലഭിക്കുന്നതാണ്‌. അവസാനം അവരെ മരിപ്പിക്കുവാനായി നമ്മുടെ ദൂതന്‍മാര്‍ (മലക്കുകള്‍) അവരുടെ അടുത്ത് ചെല്ലുമ്പോള്‍ അവര്‍ പറയും: അല്ലാഹുവിന് പുറമെ നിങ്ങള്‍ വിളിച്ച് പ്രാര്‍ത്ഥിച്ച് കൊണ്ടിരുന്നവരൊക്കെ എവിടെ? അവര്‍ പറയും : അവരൊക്കെ ഞങ്ങളെ വിട്ടുപോയിക്കളഞ്ഞു. തങ്ങള്‍ സത്യനിഷേധികളായിരുന്നുവെന്ന് അവര്‍ക്കെതിരായി അവര്‍ തന്നെ സാക്ഷ്യം വഹിക്കുകയും ചെയ്യും.
\end{malayalam}}
\flushright{\begin{Arabic}
\quranayah[7][38]
\end{Arabic}}
\flushleft{\begin{malayalam}
അവന്‍ (അല്ലാഹു) പറയും: ജിന്നുകളില്‍ നിന്നും മനുഷ്യരില്‍ നിന്നുമായി നിങ്ങള്‍ക്കു മുമ്പ് കഴിഞ്ഞുപോയിട്ടുള്ള സമൂഹങ്ങളുടെ കൂട്ടത്തില്‍ നരകത്തില്‍ പ്രവേശിച്ചുകൊള്ളുക. ഓരോ സമൂഹവും (അതില്‍) പ്രവേശിക്കുമ്പോഴൊക്കെ അതിന്‍റെ സഹോദര സമൂഹത്തെ ശപിക്കും. അങ്ങനെ അവരെല്ലാവരും അവിടെ ഒരുമിച്ചുകൂടിക്കഴിഞ്ഞാല്‍ അവരിലെ പിന്‍ഗാമികള്‍ അവരുടെ മുന്‍ഗാമികളെപ്പറ്റി പറയും: ഞങ്ങളുടെ രക്ഷിതാവേ, ഇവരാണ് ഞങ്ങളെ വഴിതെറ്റിച്ചത്‌. അത് കൊണ്ട് അവര്‍ക്ക് നീ നരകത്തില്‍ നിന്ന് ഇരട്ടി ശിക്ഷ കൊടുക്കേണമേ. അവന്‍ പറയും: എല്ലാവര്‍ക്കും ഇരട്ടിയുണ്ട്‌. പക്ഷെ നിങ്ങള്‍ മനസ്സിലാക്കുന്നില്ല.
\end{malayalam}}
\flushright{\begin{Arabic}
\quranayah[7][39]
\end{Arabic}}
\flushleft{\begin{malayalam}
അവരിലെ മുന്‍ഗാമികള്‍ അവരുടെ പിന്‍ഗാമികളോട് പറയും: അപ്പോള്‍ നിങ്ങള്‍ക്ക് ഞങ്ങളെക്കാളുപരി യാതൊരു ശ്രേഷ്ഠതയുമില്ല. ആകയാല്‍ നിങ്ങള്‍ സമ്പാദിച്ചു വെച്ചിരുന്നതിന്‍റെ ഫലമായി നിങ്ങള്‍ ശിക്ഷ അനുഭവിച്ച് കൊള്ളുക.
\end{malayalam}}
\flushright{\begin{Arabic}
\quranayah[7][40]
\end{Arabic}}
\flushleft{\begin{malayalam}
നമ്മുടെ ദൃഷ്ടാന്തങ്ങളെ നിഷേധിച്ചുതള്ളുകയും, അവയുടെ നേരെ അഹങ്കാരം നടിക്കുകയും ചെയ്തവരാരോ അവര്‍ക്ക് വേണ്ടി ആകാശത്തിന്‍റെ കവാടങ്ങള്‍ തുറന്നുകൊടുക്കപ്പെടുകയേയില്ല. ഒട്ടകം സൂചിയുടെ ദ്വാരത്തിലൂടെ കടന്ന് പോകുന്നത് വരെ അവര്‍ സ്വര്‍ഗത്തില്‍ പ്രവേശിക്കുകയുമില്ല. അപ്രകാരമാണ് നാം കുറ്റവാളികള്‍ക്ക് പ്രതിഫലം നല്‍കുന്നത്‌.
\end{malayalam}}
\flushright{\begin{Arabic}
\quranayah[7][41]
\end{Arabic}}
\flushleft{\begin{malayalam}
അവര്‍ക്ക് നരകാഗ്നിയാലുള്ള മെത്തയും അവരുടെ മീതെക്കൂടി പുതപ്പുകളും ഉണ്ടായിരിക്കും. അപ്രകാരമാണ് നാം അക്രമികള്‍ക്കു പ്രതിഫലം നല്‍കുന്നത്‌.
\end{malayalam}}
\flushright{\begin{Arabic}
\quranayah[7][42]
\end{Arabic}}
\flushleft{\begin{malayalam}
വിശ്വസിക്കുകയും സല്‍കര്‍മ്മങ്ങള്‍ പ്രവര്‍ത്തിക്കുകയും ചെയ്തവരാരോ- ഒരാള്‍ക്കും അയാളുടെ കഴിവില്‍ പെട്ടതല്ലാതെ നാം ബാധ്യതയേല്‍പ്പിക്കുന്നില്ല.- അവരാണ് സ്വര്‍ഗാവകാശികള്‍. അവരതില്‍ നിത്യവാസികളായിരിക്കും.
\end{malayalam}}
\flushright{\begin{Arabic}
\quranayah[7][43]
\end{Arabic}}
\flushleft{\begin{malayalam}
അവരുടെ (വിശ്വാസികളുടെ) മനസ്സുകളിലുള്ള ഉള്‍പകയെല്ലാം നാം നീക്കികളയുന്നതാണ്‌. അവരുടെ താഴ്ഭാഗത്ത് കൂടി അരുവികള്‍ ഒഴുകിക്കൊണ്ടിരിക്കും. അവര്‍ പറയുകയും ചെയ്യും: ഞങ്ങളെ ഇതിലേക്ക് നയിച്ച അല്ലാഹുവിന് സ്തുതി. അല്ലാഹു ഞങ്ങളെ നേര്‍വഴിയിലേക്ക് നയിച്ചിരുന്നില്ലെങ്കില്‍ ഞങ്ങളൊരിക്കലും നേര്‍വഴി പ്രാപിക്കുമായിരുന്നില്ല. ഞങ്ങളുടെ രക്ഷിതാവിന്‍റെ ദൂതന്‍മാര്‍ തീര്‍ച്ചയായും സത്യവും കൊണ്ടാണ് വന്നത്‌. അവരോട് വിളിച്ചുപറയപ്പെടുകയും ചെയ്യും: അതാ, സ്വര്‍ഗം. നിങ്ങള്‍ പ്രവര്‍ത്തിച്ചിരുന്നതിന്‍റെ ഫലമായി നിങ്ങള്‍ അതിന്‍റെ അവകാശികളാക്കപ്പെട്ടിരിക്കുന്നു.
\end{malayalam}}
\flushright{\begin{Arabic}
\quranayah[7][44]
\end{Arabic}}
\flushleft{\begin{malayalam}
സ്വര്‍ഗാവകാശികള്‍ നരകാവകാശികളോട് വിളിച്ചു പറയും: ഞങ്ങളോട് ഞങ്ങളുടെ രക്ഷിതാവ് വാഗ്ദാനം ചെയ്തത് ഞങ്ങള്‍ യാഥാര്‍ത്ഥ്യമായി കണ്ടെത്തിക്കഴിഞ്ഞു. എന്നാല്‍ നിങ്ങളുടെ രക്ഷിതാവ് (നിങ്ങളോട്‌) വാഗ്ദാനം ചെയ്തത് നിങ്ങള്‍ യാഥാര്‍ത്ഥ്യമായി കണ്ടെത്തിയോ? അവര്‍ പറയും: അതെ അപ്പോള്‍ ഒരു വിളംബരക്കാരന്‍ അവര്‍ക്കിടയില്‍ വിളിച്ചുപറയും: അല്ലാഹുവിന്‍റെ ശാപം അക്രമികളുടെ മേലാകുന്നു.
\end{malayalam}}
\flushright{\begin{Arabic}
\quranayah[7][45]
\end{Arabic}}
\flushleft{\begin{malayalam}
അതായത്‌, അല്ലാഹുവിന്‍റെ മാര്‍ഗത്തില്‍ നിന്ന് തടയുകയും, അത് വക്രമാക്കാന്‍ ആഗ്രഹിക്കുകയും, പരലോകത്തില്‍ അവിശ്വസിക്കുകയും ചെയ്യുന്നവരുടെ മേല്‍.
\end{malayalam}}
\flushright{\begin{Arabic}
\quranayah[7][46]
\end{Arabic}}
\flushleft{\begin{malayalam}
ആ രണ്ടു വിഭാഗത്തിനുമിടയില്‍ ഒരു തടസ്സം ഉണ്ടായിരിക്കും. ഉന്നത സ്ഥലങ്ങളില്‍ ചില ആളുകളുണ്ടായിരിക്കും. ഓരോ വിഭാഗത്തെയും അവരുടെ ലക്ഷണം മുഖേന അവര്‍ തിരിച്ചറിയും. സ്വര്‍ഗാവകാശികളോട് അവര്‍ വിളിച്ചുപറയും: നിങ്ങള്‍ക്കു സമാധാനമുണ്ടായിരിക്കട്ടെ. അവര്‍ (ഉയരത്തുള്ളവര്‍) അതില്‍ (സ്വര്‍ഗത്തില്‍) പ്രവേശിച്ചിട്ടില്ല. അവര്‍ (അത്‌) ആശിച്ചുകൊണ്ടിരിക്കുകയാണ്‌.
\end{malayalam}}
\flushright{\begin{Arabic}
\quranayah[7][47]
\end{Arabic}}
\flushleft{\begin{malayalam}
അവരുടെ ദൃഷ്ടികള്‍ നരകാവകാശികളുടെ നേരെ തിരിക്കപ്പെട്ടാല്‍ അവര്‍ പറയും: ഞങ്ങളുടെ രക്ഷിതാവേ, ഞങ്ങളെ നീ അക്രമികളായ ജനങ്ങളുടെ കൂട്ടത്തിലാക്കരുതേ.
\end{malayalam}}
\flushright{\begin{Arabic}
\quranayah[7][48]
\end{Arabic}}
\flushleft{\begin{malayalam}
ഉയര്‍ന്ന സ്ഥലങ്ങളിലുള്ളവര്‍ ലക്ഷണം മുഖേന അവര്‍ക്ക് തിരിച്ചറിയാവുന്ന ചില ആളുകളെ വിളിച്ചുകൊണ്ട് പറയും: നിങ്ങള്‍ ശേഖരിച്ചിരുന്നതും, നിങ്ങള്‍ അഹങ്കരിച്ചിരുന്നതും നിങ്ങള്‍ക്കെന്തൊരു പ്രയോജനമാണ് ചെയ്തത്‌?
\end{malayalam}}
\flushright{\begin{Arabic}
\quranayah[7][49]
\end{Arabic}}
\flushleft{\begin{malayalam}
ഇക്കൂട്ടരെപ്പറ്റിയാണോ അല്ലാഹു അവര്‍ക്കൊരു കാരുണ്യവും നല്‍കുകയില്ലെന്ന് നിങ്ങള്‍ സത്യം ചെയ്ത് പറഞ്ഞത്‌? (എന്നാല്‍ അവരോടാണല്ലോ) നിങ്ങള്‍ സ്വര്‍ഗത്തില്‍ പ്രവേശിച്ചുകൊള്ളുക, നിങ്ങള്‍ യാതൊന്നും ഭയപ്പെടേണ്ടതില്ല, നിങ്ങള്‍ ദുഃഖിക്കേണ്ടി വരികയുമില്ല. (എന്ന് പറയപ്പെട്ടിരിക്കുന്നത്‌!)
\end{malayalam}}
\flushright{\begin{Arabic}
\quranayah[7][50]
\end{Arabic}}
\flushleft{\begin{malayalam}
നരകാവകാശികള്‍ സ്വര്‍ഗാവകാശികളെ വിളിച്ചുപറയും: ഞങ്ങള്‍ക്ക് അല്‍പം വെള്ളമോ, അല്ലാഹു നിങ്ങള്‍ക്ക് നല്‍കിയ ഉപജീവനത്തില്‍ നിന്ന് അല്‍പമോ നിങ്ങള്‍ ചൊരിഞ്ഞുതരണേ! അവര്‍ പറയും: സത്യനിഷേധികള്‍ക്കു അല്ലാഹു അത് രണ്ടും തീര്‍ത്തും വിലക്കിയിരിക്കുകയാണ്‌.
\end{malayalam}}
\flushright{\begin{Arabic}
\quranayah[7][51]
\end{Arabic}}
\flushleft{\begin{malayalam}
(അതായത്‌) തങ്ങളുടെ മതത്തെ വിനോദവും കളിയുമാക്കിത്തീര്‍ക്കുകയും, ഐഹികജീവിതം കണ്ടു വഞ്ചിതരാവുകയും ചെയ്തവര്‍ക്ക്‌. അതിനാല്‍ അവരുടെതായ ഈ ദിവസത്തെ കണ്ടുമുട്ടുമെന്നത് അവര്‍ മറന്നുകളഞ്ഞത് പോലെ, നമ്മുടെ ദൃഷ്ടാന്തങ്ങളെ അവര്‍ നിഷേധിച്ചു കളഞ്ഞിരുന്നത് പോലെ ഇന്ന് അവരെ നാം മറന്നുകളയുന്നു.
\end{malayalam}}
\flushright{\begin{Arabic}
\quranayah[7][52]
\end{Arabic}}
\flushleft{\begin{malayalam}
ജ്ഞാനത്തിന്‍റെ അടിസ്ഥാനത്തില്‍ നാം കാര്യങ്ങള്‍ വിശദമാക്കിയിട്ടുള്ള ഒരു ഗ്രന്ഥം അവര്‍ക്കു നാം കൊണ്ടുവന്നുകൊടുത്തു. വിശ്വസിക്കുന്ന ജനങ്ങള്‍ക്കു മാര്‍ഗദര്‍ശനവും കാരുണ്യവുമത്രെ അത്‌.
\end{malayalam}}
\flushright{\begin{Arabic}
\quranayah[7][53]
\end{Arabic}}
\flushleft{\begin{malayalam}
അതിലുള്ളത് പുലര്‍ന്ന് കാണുക എന്നതല്ലാതെ മറ്റുവല്ലതുമാണോ അവര്‍ നോക്കിക്കൊണ്ടിരിക്കുന്നത്‌? മുമ്പ് അതിനെ മറന്നുകളഞ്ഞവര്‍ അതിന്‍റെ പുലര്‍ച്ചവന്നെത്തുന്ന ദിവസത്തില്‍ പറയും: ഞങ്ങളുടെ രക്ഷിതാവിന്‍റെ ദൂതന്‍മാര്‍ സത്യവും കൊണ്ട് തന്നെയാണ് വന്നത്‌. ഇനി ഞങ്ങള്‍ക്കു വേണ്ടി ശുപാര്‍ശ ചെയ്യാന്‍ വല്ല ശുപാര്‍ശക്കാരുമുണ്ടോ? അതല്ല, ഞങ്ങളൊന്ന് തിരിച്ചയക്കപ്പെടുമോ? എങ്കില്‍ ഞങ്ങള്‍ മുമ്പ് ചെയ്തിരുന്നതില്‍ നിന്ന് വ്യത്യസ്തമായി പ്രവര്‍ത്തിക്കുമായിരുന്നു. തങ്ങള്‍ക്ക് തന്നെ അവര്‍ നഷ്ടം വരുത്തിവെച്ചു. അവര്‍ കെട്ടിച്ചമച്ചിരുന്നതെല്ലാം അവരെ വിട്ട് പോയിക്കളയുകയും ചെയ്തു.
\end{malayalam}}
\flushright{\begin{Arabic}
\quranayah[7][54]
\end{Arabic}}
\flushleft{\begin{malayalam}
തീര്‍ച്ചയായും നിങ്ങളുടെ രക്ഷിതാവ് ആറുദിവസങ്ങളിലായി (ഘട്ടങ്ങളിലായി) ആകാശങ്ങളും ഭൂമിയും സൃഷ്ടിച്ചവനായ അല്ലാഹുവാകുന്നു. എന്നിട്ടവന്‍ സിംഹാസനസ്ഥനായിരിക്കുന്നു. രാത്രിയെക്കൊണ്ട് അവന്‍ പകലിനെ മൂടുന്നു. ദ്രുതഗതിയില്‍ അത് പകലിനെ തേടിച്ചെല്ലുന്നു. സൂര്യനെയും ചന്ദ്രനെയും നക്ഷത്രങ്ങളെയും തന്‍റെ കല്‍പനയ്ക്കു വിധേയമാക്കപ്പെട്ട നിലയില്‍ (അവന്‍ സൃഷ്ടിച്ചിരിക്കുന്നു.) അറിയുക: സൃഷ്ടിപ്പും ശാസനാധികാരവും അവന്നുതന്നെയാണ.് ലോകരക്ഷിതാവായ അല്ലാഹു മഹത്വപൂര്‍ണ്ണനായിരിക്കുന്നു.
\end{malayalam}}
\flushright{\begin{Arabic}
\quranayah[7][55]
\end{Arabic}}
\flushleft{\begin{malayalam}
താഴ്മയോടു കൂടിയും രഹസ്യമായിക്കൊണ്ടും നിങ്ങള്‍ നിങ്ങളുടെ രക്ഷിതാവിനോട് പ്രാര്‍ത്ഥിക്കുക. പരിധി വിട്ട് പോകുന്നവരെ അല്ലാഹു ഇഷ്ടപ്പെടുക തന്നെയില്ല.
\end{malayalam}}
\flushright{\begin{Arabic}
\quranayah[7][56]
\end{Arabic}}
\flushleft{\begin{malayalam}
ഭൂമിയില്‍ നന്‍മവരുത്തിയതിനു ശേഷം നിങ്ങള്‍ അവിടെ നാശമുണ്ടാക്കരുത്‌. ഭയപ്പാടോടു കൂടിയും പ്രതീക്ഷയോടുകൂടിയും നിങ്ങള്‍ അവനെ വിളിച്ചു പ്രാര്‍ത്ഥിക്കുകയും ചെയ്യുക. തീര്‍ച്ചയായും അല്ലാഹുവിന്‍റെ കാരുണ്യം സല്‍കര്‍മ്മകാരികള്‍ക്ക് സമീപസ്ഥമാകുന്നു.
\end{malayalam}}
\flushright{\begin{Arabic}
\quranayah[7][57]
\end{Arabic}}
\flushleft{\begin{malayalam}
അവനത്രെ തന്‍റെ അനുഗ്രഹത്തിന്ന് (മഴയ്ക്കു) മുമ്പായി സന്തോഷവാര്‍ത്ത അറിയിച്ചുകൊണ്ട് കാറ്റുകളെ അയക്കുന്നവന്‍. അങ്ങനെ അവ (കാറ്റുകള്‍) ഭാരിച്ച മേഘത്തെ വഹിച്ചുകഴിഞ്ഞാല്‍ നിര്‍ജീവമായ വല്ല നാട്ടിലേക്കും നാം അതിനെ നയിച്ചുകൊണ്ട് പോകുകയും, എന്നിട്ടവിടെ വെള്ളം ചൊരിയുകയും, അത് മൂലം എല്ലാതരം കായ്കനികളും നാം പുറത്ത് കൊണ്ടുവരികയും ചെയ്യുന്നു. അത് പോലെ നാം മരണപ്പെട്ടവരെ പുറത്ത് കൊണ്ട് വരുന്നതാണ്‌. നിങ്ങള്‍ ശ്രദ്ധിച്ചു മനസ്സിലാക്കുന്നവരായേക്കാം.
\end{malayalam}}
\flushright{\begin{Arabic}
\quranayah[7][58]
\end{Arabic}}
\flushleft{\begin{malayalam}
നല്ല നാട്ടില്‍ അതിലെ സസ്യങ്ങള്‍ അതിന്‍റെ രക്ഷിതാവിന്‍റെ അനുമതിയോടെ നന്നായി മുളച്ചു വരുന്നു. എന്നാല്‍ മോശമായ നാട്ടില്‍ ശുഷ്ക്കമായിക്കൊണ്ടല്ലാതെ സസ്യങ്ങള്‍ മുളച്ച് വരികയില്ല. അപ്രകാരം, നന്ദികാണിക്കുന്ന ജനങ്ങള്‍ക്കുവേണ്ടി നാം ദൃഷ്ടാന്തങ്ങള്‍ വിവധ രൂപത്തില്‍ വിവരിക്കുന്നു.
\end{malayalam}}
\flushright{\begin{Arabic}
\quranayah[7][59]
\end{Arabic}}
\flushleft{\begin{malayalam}
നൂഹിനെ അദ്ദേഹത്തിന്‍റെ ജനതയിലേക്ക് നാം അയക്കുകയുണ്ടായി. എന്നിട്ട് അദ്ദേഹം പറഞ്ഞു: എന്‍റെ ജനങ്ങളേ, നിങ്ങള്‍ അല്ലാഹുവെ ആരാധിക്കുവിന്‍. അവനല്ലാതെ നിങ്ങള്‍ക്ക് ഒരു ദൈവവുമില്ല. തീര്‍ച്ചയായും ഭയങ്കരമായ ഒരു ദിവസത്തെ ശിക്ഷ നിങ്ങള്‍ക്കു (വന്നുഭവിക്കുമെന്ന്‌) ഞാന്‍ ഭയപ്പെടുന്നു.
\end{malayalam}}
\flushright{\begin{Arabic}
\quranayah[7][60]
\end{Arabic}}
\flushleft{\begin{malayalam}
അദ്ദേഹത്തിന്‍റെ ജനതയിലെ പ്രമാണിമാര്‍ പറഞ്ഞു: തീര്‍ച്ചയായും നീ പ്രത്യക്ഷമായ ദുര്‍മാര്‍ഗത്തിലാണെന്ന് ഞങ്ങള്‍ കാണുന്നു.
\end{malayalam}}
\flushright{\begin{Arabic}
\quranayah[7][61]
\end{Arabic}}
\flushleft{\begin{malayalam}
അദ്ദേഹം പറഞ്ഞു: എന്‍റെ ജനങ്ങളേ, എന്നില്‍ ദുര്‍മാര്‍ഗമൊന്നുമില്ല. പക്ഷെ ഞാന്‍ ലോകരക്ഷിതാവിങ്കല്‍ നിന്നുള്ള ദൂതനാകുന്നു.
\end{malayalam}}
\flushright{\begin{Arabic}
\quranayah[7][62]
\end{Arabic}}
\flushleft{\begin{malayalam}
എന്‍റെ രക്ഷിതാവിന്‍റെ സന്ദേശങ്ങള്‍ ഞാന്‍ നിങ്ങള്‍ക്കു എത്തിച്ചുതരികയും, നിങ്ങളോട് ആത്മാര്‍ത്ഥമായി ഉപദേശിക്കുകയുമാകുന്നു. നിങ്ങള്‍ക്കറിഞ്ഞ് കൂടാത്ത പലതും അല്ലാഹുവിങ്കല്‍ നിന്ന് ഞാന്‍ അറിയുന്നുമുണ്ട്‌.
\end{malayalam}}
\flushright{\begin{Arabic}
\quranayah[7][63]
\end{Arabic}}
\flushleft{\begin{malayalam}
നിങ്ങള്‍ക്കു മുന്നറിയിപ്പ് നല്‍കുന്നതിന് വേണ്ടിയും, നിങ്ങള്‍ സൂക്ഷ്മത പാലിക്കുന്നതിന് വേണ്ടിയും, നിങ്ങള്‍ക്ക് കാരുണ്യം നല്‍കപ്പെടുന്നതിന് വേണ്ടിയും നിങ്ങളുടെ രക്ഷിതാവിങ്കല്‍ നിന്നുള്ള ഒരു ഉല്‍ബോധനം നിങ്ങളില്‍ പെട്ട ഒരു പുരുഷനിലൂടെ നിങ്ങള്‍ക്ക് വന്നുകിട്ടിയതില്‍ നിങ്ങള്‍ അത്ഭുതപ്പെടുകയാണോ?
\end{malayalam}}
\flushright{\begin{Arabic}
\quranayah[7][64]
\end{Arabic}}
\flushleft{\begin{malayalam}
എന്നാല്‍ അവര്‍ അദ്ദേഹത്തെ നിഷേധിച്ചു തള്ളിക്കളഞ്ഞു. അപ്പോള്‍ അദ്ദേഹത്തെയും അദ്ദേഹത്തിന്‍റെ കൂടെയുള്ളവരെയും നാം കപ്പലില്‍ രക്ഷപ്പെടുത്തുകയും, നമ്മുടെ ദൃഷ്ടാന്തങ്ങള്‍ നിഷേധിച്ചു തള്ളിക്കളഞ്ഞവരെ നാം മുക്കിക്കൊല്ലുകയും ചെയ്തു. തീര്‍ച്ചയായും അവര്‍ അന്ധരായ ഒരു ജനതയായിരുന്നു.
\end{malayalam}}
\flushright{\begin{Arabic}
\quranayah[7][65]
\end{Arabic}}
\flushleft{\begin{malayalam}
ആദ് സമുദായത്തിലേക്ക് അവരുടെ സഹോദരനായ ഹൂദിനെയും (അയച്ചു.) അദ്ദേഹം പറഞ്ഞു: എന്‍റെ ജനങ്ങളേ, നിങ്ങള്‍ അല്ലാഹുവെ ആരാധിക്കുവിന്‍. നിങ്ങള്‍ക്ക് അവനല്ലാതെ യാതൊരു ദൈവവുമില്ല. നിങ്ങളെന്താണ് സൂക്ഷ്മത പുലര്‍ത്താത്തത്‌?
\end{malayalam}}
\flushright{\begin{Arabic}
\quranayah[7][66]
\end{Arabic}}
\flushleft{\begin{malayalam}
അദ്ദേഹത്തിന്‍റെ ജനതയിലെ സത്യനിഷേധികളായ പ്രമാണിമാര്‍ പറഞ്ഞു: തീര്‍ച്ചയായും നീ എന്തോ മൌഢ്യത്തില്‍പ്പെട്ടിരിക്കുകയാണെന്ന് ഞങ്ങള്‍ കാണുന്നു. തീര്‍ച്ചയായും നീ കള്ളം പറയുന്നവരുടെ കൂട്ടത്തിലാണെന്ന് ഞങ്ങള്‍ വിചാരിക്കുന്നു.
\end{malayalam}}
\flushright{\begin{Arabic}
\quranayah[7][67]
\end{Arabic}}
\flushleft{\begin{malayalam}
അദ്ദേഹം പറഞ്ഞു: എന്‍റെ ജനങ്ങളേ, എന്നില്‍ യാതൊരു മൌഢ്യവുമില്ല. പക്ഷെ, ഞാന്‍ ലോകരക്ഷിതാവിങ്കല്‍ നിന്നുള്ള ദൂതനാണ്‌.
\end{malayalam}}
\flushright{\begin{Arabic}
\quranayah[7][68]
\end{Arabic}}
\flushleft{\begin{malayalam}
എന്‍റെ രക്ഷിതാവിന്‍റെ സന്ദേശങ്ങള്‍ ഞാന്‍ നിങ്ങള്‍ക്കു എത്തിച്ചുതരുന്നു. ഞാന്‍ നിങ്ങളുടെ വിശ്വസ്തനായ ഗുണകാംക്ഷിയുമാകുന്നു.
\end{malayalam}}
\flushright{\begin{Arabic}
\quranayah[7][69]
\end{Arabic}}
\flushleft{\begin{malayalam}
നിങ്ങള്‍ക്കു മുന്നറിയിപ്പു നല്‍കാന്‍ വേണ്ടി നിങ്ങളില്‍ പെട്ട ഒരു പുരുഷനിലൂടെ നിങ്ങളുടെ രക്ഷിതാവിങ്കല്‍ നിന്നുള്ള ഒരു ഉല്‍ബോധനം നിങ്ങള്‍ക്കു വന്നുകിട്ടിയതിനാല്‍ നിങ്ങള്‍ അത്ഭുതപ്പെടുകയാണോ? നൂഹിന്‍റെ ജനതയ്ക്കു ശേഷം നിങ്ങളെ അവന്‍ പിന്‍ഗാമികളാക്കുകയും, സൃഷ്ടിയില്‍ അവന്‍ നിങ്ങള്‍ക്കു (ശാരീരിക) വികാസം വര്‍ദ്ധിപ്പിക്കുകയും ചെയ്തത് നിങ്ങള്‍ ഓര്‍ത്ത് നോക്കുക. അല്ലാഹുവിന്‍റെ അനുഗ്രഹങ്ങള്‍ നിങ്ങള്‍ ഓര്‍മ്മിക്കുക. നിങ്ങള്‍ക്ക് വിജയം പ്രാപിക്കാം.
\end{malayalam}}
\flushright{\begin{Arabic}
\quranayah[7][70]
\end{Arabic}}
\flushleft{\begin{malayalam}
അവര്‍ പറഞ്ഞു: ഞങ്ങള്‍ അല്ലാഹുവെ മാത്രം ആരാധിക്കുവാനും, ഞങ്ങളുടെ പിതാക്കള്‍ ആരാധിച്ചിരുന്നതിനെ ഞങ്ങള്‍ വിട്ടുകളയുവാനും വേണ്ടിയാണോ നീ ഞങ്ങളുടെ അടുത്ത് വന്നിരിക്കുന്നത്‌? എങ്കില്‍ ഞങ്ങളോട് നീ ഭീഷണിപ്പെടുത്തിക്കൊണ്ടിരിക്കുന്നത് (ശിക്ഷ) നീ ഞങ്ങള്‍ക്കു കൊണ്ടുവാ; നീ സത്യവാന്‍മാരില്‍ പെട്ടവനാണെങ്കില്‍.
\end{malayalam}}
\flushright{\begin{Arabic}
\quranayah[7][71]
\end{Arabic}}
\flushleft{\begin{malayalam}
ഹൂദ് പറഞ്ഞു: തീര്‍ച്ചയായും നിങ്ങളുടെ രക്ഷിതാവിങ്കല്‍ നിന്നുള്ള ശിക്ഷയും കോപവും (ഇതാ) നിങ്ങള്‍ക്ക് വന്നുഭവിക്കുകയായി. നിങ്ങളും നിങ്ങളുടെ പിതാക്കന്‍മാരും പേരിട്ടുവെച്ചിട്ടുള്ളതും, അല്ലാഹു യാതൊരു പ്രമാണവും അവതരിപ്പിച്ചിട്ടില്ലാത്തതുമായ ചില (ദൈവ) നാമങ്ങളുടെ പേരിലാണോ നിങ്ങളെന്നോട് തര്‍ക്കിക്കുന്നത്‌? എന്നാല്‍ നിങ്ങള്‍ കാത്തിരുന്ന് കൊള്ളുക. തീര്‍ച്ചയായും ഞാനും നിങ്ങളോടൊപ്പം കാത്തിരിക്കുകയാണ്‌.
\end{malayalam}}
\flushright{\begin{Arabic}
\quranayah[7][72]
\end{Arabic}}
\flushleft{\begin{malayalam}
അങ്ങനെ അദ്ദേഹത്തെയും അദ്ദേഹത്തിന്‍റെ കൂടെയുള്ളവരെയും നമ്മുടെ കാരുണ്യം കൊണ്ട് നാം രക്ഷപ്പെടുത്തുകയും നമ്മുടെ ദൃഷ്ടാന്തങ്ങള്‍ നിഷേധിച്ചുതള്ളുകയും, വിശ്വസിക്കാതിരിക്കുകയും ചെയ്തവരെ നാം മുരടോടെ മുറിച്ചുകളയുകയും ചെയ്തു.
\end{malayalam}}
\flushright{\begin{Arabic}
\quranayah[7][73]
\end{Arabic}}
\flushleft{\begin{malayalam}
ഥമൂദ് സമുദായത്തിലേക്ക് അവരുടെ സഹോദരന്‍ സ്വാലിഹിനെയും (നാം അയച്ചു.) അദ്ദേഹം പറഞ്ഞു: എന്‍റെ ജനങ്ങളേ, നിങ്ങള്‍ അല്ലാഹുവിനെ ആരാധിക്കുവിന്‍. അവനല്ലാതെ നിങ്ങള്‍ക്കു ഒരു ദൈവവുമില്ല. നിങ്ങളുടെ രക്ഷിതാവിങ്കല്‍ നിന്നു വ്യക്തമായ ഒരു തെളിവ് നിങ്ങള്‍ക്കു വന്നിട്ടുണ്ട്‌. നിങ്ങള്‍ക്കൊരു ദൃഷ്ടാന്തമായിട്ട് അല്ലാഹുവിന്‍റെ ഒട്ടകമാണിത്‌. ആകയാല്‍ അല്ലാഹുവിന്‍റെ ഭൂമിയില്‍ (നടന്നു) തിന്നുവാന്‍ നിങ്ങള്‍ അതിനെ വിട്ടേക്കുക. നിങ്ങളതിന് ഒരു ഉപദ്രവവും ചെയ്യരുത്‌. എങ്കില്‍ വേദനയേറിയ ശിക്ഷ നിങ്ങളെ പിടികൂടും.
\end{malayalam}}
\flushright{\begin{Arabic}
\quranayah[7][74]
\end{Arabic}}
\flushleft{\begin{malayalam}
ആദ് സമുദായത്തിനു ശേഷം അവന്‍ നിങ്ങളെ പിന്‍ഗാമികളാക്കുകയും, നിങ്ങള്‍ക്കവന്‍ ഭൂമിയില്‍ വാസസ്ഥലം ഒരുക്കിത്തരികയും ചെയ്ത സന്ദര്‍ഭം നിങ്ങള്‍ ഓര്‍ക്കുകയും ചെയ്യുക. അതിലെ സമതലങ്ങളില്‍ നിങ്ങള്‍ സൌധങ്ങളുണ്ടാക്കുന്നു. മലകള്‍ വെട്ടിയെടുത്ത് നിങ്ങള്‍ വീടുകളുണ്ടാക്കുകയും ചെയ്യുന്നു. അങ്ങനെ അല്ലാഹുവിന്‍റെ അനുഗ്രഹങ്ങള്‍ നിങ്ങള്‍ ഓര്‍ത്ത് നോക്കുക. നിങ്ങള്‍ നാശകാരികളായിക്കൊണ്ട് ഭൂമിയില്‍ കുഴപ്പം സൃഷ്ടിക്കരുത്‌.
\end{malayalam}}
\flushright{\begin{Arabic}
\quranayah[7][75]
\end{Arabic}}
\flushleft{\begin{malayalam}
അദ്ദേഹത്തിന്‍റെ ജനതയില്‍ പെട്ട അഹങ്കാരികളായ പ്രമാണിമാര്‍ ബലഹീനരായി കരുതപ്പെട്ടവരോട് (അതായത്‌) അവരില്‍ നിന്ന് വിശ്വസിച്ചവരോട് പറഞ്ഞു: സ്വാലിഹ് തന്‍റെ രക്ഷിതാവിങ്കല്‍ നിന്ന് അയക്കപ്പെട്ട ആള്‍ തന്നെയാണെന്ന് നിങ്ങള്‍ക്കറിയുമോ? അവര്‍ പറഞ്ഞു: അദ്ദേഹം ഏതൊന്നുമായി അയക്കപ്പെട്ടിരിക്കുന്നുവോ അതില്‍ ഞങ്ങള്‍ തീര്‍ച്ചയായും വിശ്വസിക്കുന്നവരാണ്‌.
\end{malayalam}}
\flushright{\begin{Arabic}
\quranayah[7][76]
\end{Arabic}}
\flushleft{\begin{malayalam}
അഹങ്കാരം കൈക്കൊണ്ടവര്‍ പറഞ്ഞു: നിങ്ങള്‍ ഏതൊന്നില്‍ വിശ്വസിക്കുന്നുവോ അതിനെ ഞങ്ങള്‍ തീര്‍ത്തും നിഷേധിക്കുന്നവരാണ്‌.
\end{malayalam}}
\flushright{\begin{Arabic}
\quranayah[7][77]
\end{Arabic}}
\flushleft{\begin{malayalam}
അങ്ങനെ അവര്‍ ആ ഒട്ടകത്തെ അറുകൊലചെയ്യുകയും, തങ്ങളുടെ രക്ഷിതാവിന്‍റെ കല്‍പനയെ ധിക്കരിക്കുകയും ചെയ്തു. അവര്‍ പറഞ്ഞു: സ്വാലിഹേ, നീ ദൈവദൂതന്‍മാരില്‍ പെട്ട ആളാണെങ്കില്‍ ഞങ്ങളോട് നീ ഭീഷണിപ്പെടുത്തിക്കൊണ്ടിരിക്കുന്നത് (ശിക്ഷ) ഞങ്ങള്‍ക്ക് നീ കൊണ്ടുവാ.
\end{malayalam}}
\flushright{\begin{Arabic}
\quranayah[7][78]
\end{Arabic}}
\flushleft{\begin{malayalam}
അപ്പോള്‍ ഭൂകമ്പം അവരെ പിടികൂടി. അങ്ങനെ നേരം പുലര്‍ന്നപ്പോള്‍ അവര്‍ തങ്ങളുടെ വീടുകളില്‍ കമിഴ്ന്ന് വീണ് കിടക്കുന്നവരായിരുന്നു.
\end{malayalam}}
\flushright{\begin{Arabic}
\quranayah[7][79]
\end{Arabic}}
\flushleft{\begin{malayalam}
അനന്തരം സ്വാലിഹ് അവരില്‍ നിന്ന് പിന്തിരിഞ്ഞു പോയി. അദ്ദേഹം പറഞ്ഞു: എന്‍റെ ജനങ്ങളേ, തീര്‍ച്ചയായും ഞാന്‍ നിങ്ങള്‍ക്കു എന്‍റെ രക്ഷിതാവിന്‍റെ സന്ദേശം എത്തിച്ചുതരികയും, ആത്മാര്‍ത്ഥമായി ഞാന്‍ നിങ്ങളോട് ഉപദേശിക്കുകയുമുണ്ടായി. പക്ഷെ, സദുപദേശികളെ നിങ്ങള്‍ ഇഷ്ടപ്പെടുന്നില്ല.
\end{malayalam}}
\flushright{\begin{Arabic}
\quranayah[7][80]
\end{Arabic}}
\flushleft{\begin{malayalam}
ലൂത്വിനെയും (നാം അയച്ചു.) അദ്ദേഹം തന്‍റെ ജനതയോട്‌, നിങ്ങള്‍ക്ക് മുമ്പ് ലോകരില്‍ ഒരാളും തന്നെ ചെയ്തിട്ടില്ലാത്ത ഈ നീചവൃത്തിക്ക് നിങ്ങള്‍ ചെല്ലുകയോ? എന്ന് പറഞ്ഞ സന്ദര്‍ഭം (ഓര്‍ക്കുക.)
\end{malayalam}}
\flushright{\begin{Arabic}
\quranayah[7][81]
\end{Arabic}}
\flushleft{\begin{malayalam}
സ്ത്രീകളെ വിട്ട് പുരുഷന്‍മാരുടെ അടുത്ത് തന്നെ നിങ്ങള്‍ കാമവികാരത്തോടെ ചെല്ലുന്നു. അല്ല, നിങ്ങള്‍ അതിരുവിട്ട് പ്രവര്‍ത്തിക്കുന്ന ഒരു ജനതയാകുന്നു.
\end{malayalam}}
\flushright{\begin{Arabic}
\quranayah[7][82]
\end{Arabic}}
\flushleft{\begin{malayalam}
ഇവരെ നിങ്ങളുടെ നാട്ടില്‍ നിന്നു പുറത്താക്കുക, ഇവര്‍ പരിശുദ്ധിപാലിക്കുന്ന ആളുകളാകുന്നു. എന്നു പറഞ്ഞത് മാത്രമായിരുന്നു അദ്ദേഹത്തിന്‍റെ ജനതയുടെ മറുപടി.
\end{malayalam}}
\flushright{\begin{Arabic}
\quranayah[7][83]
\end{Arabic}}
\flushleft{\begin{malayalam}
അപ്പോള്‍ അദ്ദേഹത്തെയും അദ്ദേഹത്തിന്‍റെ ഭാര്യ ഒഴിച്ചുള്ള കുടുംബക്കാരെയും നാം രക്ഷപ്പെടുത്തി. അവള്‍ പിന്തിരിഞ്ഞ് നിന്നവരുടെ കൂട്ടത്തിലായിരുന്നു.
\end{malayalam}}
\flushright{\begin{Arabic}
\quranayah[7][84]
\end{Arabic}}
\flushleft{\begin{malayalam}
നാം അവരുടെ മേല്‍ ഒരു തരം മഴ വര്‍ഷിപ്പിക്കുകയും ചെയ്തു. അപ്പോള്‍ ആ കുറ്റവാളികളുടെ പര്യവസാനം എങ്ങനെയായിരുന്നുവെന്ന് നോക്കുക.
\end{malayalam}}
\flushright{\begin{Arabic}
\quranayah[7][85]
\end{Arabic}}
\flushleft{\begin{malayalam}
മദ്‌യങ്കാരിലേക്ക് അവരുടെ സഹോദരനായ ശുഐബിനെയും (അയച്ചു.) അദ്ദേഹം പറഞ്ഞു: എന്‍റെ ജനങ്ങളേ, നിങ്ങള്‍ അല്ലാഹുവെ ആരാധിക്കുക. നിങ്ങള്‍ക്ക് അവനല്ലാതെ യാതൊരു ദൈവവുമില്ല. നിങ്ങള്‍ക്ക് നിങ്ങളുടെ രക്ഷിതാവിങ്കല്‍ നിന്ന് വ്യക്തമായ തെളിവ് വന്നിട്ടുണ്ട്‌. അതിനാല്‍ നിങ്ങള്‍ അളവും തൂക്കവും തികച്ചുകൊടുക്കണം. ജനങ്ങള്‍ക്കുഅവരുടെ സാധനങ്ങളില്‍ നിങ്ങള്‍ കമ്മിവരുത്തരുത്‌. ഭൂമിയില്‍ നന്‍മവരുത്തിയതിന് ശേഷം നിങ്ങള്‍ അവിടെ നാശമുണ്ടാക്കരുത്‌. നിങ്ങള്‍ വിശ്വാസികളാണെങ്കില്‍ അതാണ് നിങ്ങള്‍ക്ക് ഉത്തമം.
\end{malayalam}}
\flushright{\begin{Arabic}
\quranayah[7][86]
\end{Arabic}}
\flushleft{\begin{malayalam}
ഭീഷണിയുണ്ടാക്കിക്കൊണ്ടും, അല്ലാഹുവിന്‍റെ മാര്‍ഗത്തില്‍ നിന്ന് അതില്‍ വിശ്വസിച്ചവരെ തടഞ്ഞുകൊണ്ടും അത് (ആ മാര്‍ഗം) വക്രമായിരിക്കാന്‍ ആഗ്രഹിച്ചുകൊണ്ടും നിങ്ങള്‍ പാതകളിലെല്ലാം ഇരിക്കുകയും അരുത്‌. നിങ്ങള്‍ എണ്ണത്തില്‍ കുറവായിരുന്നിട്ടും നിങ്ങള്‍ക്ക് അവന്‍ വര്‍ദ്ധനവ് നല്‍കിയത് ഓര്‍ക്കുകയും നാശകാരികളുടെ പര്യവസാനം എങ്ങനെയായിരുന്നുവെന്ന് നോക്കുകയും ചെയ്യുക.
\end{malayalam}}
\flushright{\begin{Arabic}
\quranayah[7][87]
\end{Arabic}}
\flushleft{\begin{malayalam}
ഞാന്‍ എന്തൊന്നുമായി അയക്കപ്പെട്ടിരിക്കുന്നുവോ അതില്‍ നിങ്ങളില്‍ ഒരു വിഭാഗം വിശ്വസിച്ചിരിക്കുകയും, മറ്റൊരു വിഭാഗം വിശ്വസിക്കാതിരിക്കുകയുമാണെങ്കില്‍ നമുക്കിടയില്‍ അല്ലാഹു തീര്‍പ്പുകല്‍പിക്കുന്നത് വരെ നിങ്ങള്‍ ക്ഷമിച്ചിരിക്കുക.അവനത്രെ തീര്‍പ്പുകല്‍പിക്കുന്നവരില്‍ ഉത്തമന്‍.
\end{malayalam}}
\flushright{\begin{Arabic}
\quranayah[7][88]
\end{Arabic}}
\flushleft{\begin{malayalam}
അദ്ദേഹത്തിന്‍റെ ജനതയിലെ അഹങ്കാരികളായ പ്രമാണിമാര്‍ പറഞ്ഞു: ശുഐബേ, തീര്‍ച്ചയായും നിന്നെയും നിന്‍റെ കൂടെയുള്ള വിശ്വാസികളെയും ഞങ്ങളുടെ നാട്ടില്‍ നിന്ന് പുറത്താക്കുക തന്നെ ചെയ്യും. അല്ലെങ്കില്‍ നിങ്ങള്‍ ഞങ്ങളുടെ മാര്‍ഗത്തില്‍ മടങ്ങി വരിക തന്നെ വേണം. അദ്ദേഹം പറഞ്ഞു: ഞങ്ങള്‍ അതിനെ (ആ മാര്‍ഗത്തെ) വെറുക്കുന്നവരാണെങ്കില്‍ പോലും (ഞങ്ങള്‍ മടങ്ങണമെന്നോ?)
\end{malayalam}}
\flushright{\begin{Arabic}
\quranayah[7][89]
\end{Arabic}}
\flushleft{\begin{malayalam}
നിങ്ങളുടെ മാര്‍ഗത്തില്‍ നിന്ന് അല്ലാഹു ഞങ്ങളെ രക്ഷപ്പെടുത്തിയതിന് ശേഷം അതില്‍ തന്നെ ഞങ്ങള്‍ മടങ്ങി വരുന്ന പക്ഷം തീര്‍ച്ചയായും ഞങ്ങള്‍ അല്ലാഹുവിന്‍റെ പേരില്‍ കള്ളം കെട്ടിച്ചമയ്ക്കുകയായിരിക്കും ചെയ്യുന്നത്‌. അതില്‍ മടങ്ങി വരാന്‍ ഞങ്ങള്‍ക്കു പാടില്ലാത്തതാണ്‌; ഞങ്ങളുടെ രക്ഷിതാവായ അല്ലാഹു ഉദ്ദേശിക്കുന്നുവെങ്കിലല്ലാതെ. ഞങ്ങളുടെ രക്ഷിതാവിന്‍റെ അറിവ് എല്ലാകാര്യത്തെയും ഉള്‍കൊള്ളുന്നതായിരിക്കുന്നു. അല്ലാഹുവിന്‍റെ മേലാണ് ഞങ്ങള്‍ ഭരമേല്‍പിച്ചിരിക്കുന്നത്‌. ഞങ്ങളുടെ രക്ഷിതാവേ, ഞങ്ങള്‍ക്കും ഞങ്ങളുടെ ജനങ്ങള്‍ക്കുമിടയില്‍ നീ സത്യപ്രകാരം തീര്‍പ്പുണ്ടാക്കണമേ. നീയാണ് തീര്‍പ്പുണ്ടാക്കുന്നവരില്‍ ഉത്തമന്‍.
\end{malayalam}}
\flushright{\begin{Arabic}
\quranayah[7][90]
\end{Arabic}}
\flushleft{\begin{malayalam}
അദ്ദേഹത്തിന്‍റെ ജനതയിലെ സത്യനിഷേധികളായ പ്രമാണിമാര്‍ പറഞ്ഞു: നിങ്ങള്‍ ശുഐബിനെ പിന്‍പറ്റുന്ന പക്ഷം തീര്‍ച്ചയായും അത് മൂലം നിങ്ങള്‍ നഷ്ടക്കാരായിരിക്കും.
\end{malayalam}}
\flushright{\begin{Arabic}
\quranayah[7][91]
\end{Arabic}}
\flushleft{\begin{malayalam}
അപ്പോള്‍ അവരെ ഭൂകമ്പം പിടികൂടി. അങ്ങനെ നേരം പുലര്‍ന്നപ്പോള്‍ അവര്‍ അവരുടെ വാസസ്ഥലത്ത് കമിഴ്ന്നു വീണു കിടക്കുകയായിരുന്നു.
\end{malayalam}}
\flushright{\begin{Arabic}
\quranayah[7][92]
\end{Arabic}}
\flushleft{\begin{malayalam}
ശുഐബിനെ നിഷേധിച്ചു തള്ളിയവരുടെ സ്ഥിതി അവരവിടെ താമസിച്ചിട്ടേയില്ലാത്ത പോലെയായി. ശുഐബിനെ നിഷേധിച്ചു തള്ളിയവര്‍ തന്നെയായിരുന്നു നഷ്ടക്കാര്‍.
\end{malayalam}}
\flushright{\begin{Arabic}
\quranayah[7][93]
\end{Arabic}}
\flushleft{\begin{malayalam}
അനന്തരം അദ്ദേഹം അവരില്‍ നിന്ന് പിന്തിരിഞ്ഞ് പോയി. അദ്ദേഹം പറഞ്ഞു: എന്‍റെ ജനങ്ങളേ, തീര്‍ച്ചയായും എന്‍റെ രക്ഷിതാവിന്‍റെ സന്ദേശങ്ങള്‍ ഞാന്‍ നിങ്ങള്‍ക്ക് എത്തിച്ചുതരികയും ഞാന്‍ നിങ്ങളോട് ആത്മാര്‍ത്ഥമായി ഉപദേശിക്കുകയും ചെയ്തിട്ടുണ്ട്‌. അങ്ങനെയിരിക്കെ സത്യനിഷേധികളായ ജനതയുടെ പേരില്‍ ഞാന്‍ എന്തിനു ദുഃഖിക്കണം.?
\end{malayalam}}
\flushright{\begin{Arabic}
\quranayah[7][94]
\end{Arabic}}
\flushleft{\begin{malayalam}
ഏതൊരു നാട്ടില്‍ നാം പ്രവാചകനെ അയച്ചപ്പോഴും അവിടത്തുകാരെ ദുരിതവും കഷ്ടപ്പാടും കൊണ്ട് നാം പിടികൂടാതിരുന്നിട്ടില്ല. അവര്‍ വിനയമുള്ളവരായിത്തീരാന്‍ വേണ്ടിയത്രെ അത്‌.
\end{malayalam}}
\flushright{\begin{Arabic}
\quranayah[7][95]
\end{Arabic}}
\flushleft{\begin{malayalam}
പിന്നെ നാം വിഷമത്തിന്‍റെ സ്ഥാനത്ത് സൌഖ്യം മാറ്റിവച്ചുകൊടുത്തു. അങ്ങനെ അവര്‍ അഭിവൃദ്ധിപ്പെട്ടു വളര്‍ന്നു. ഞങ്ങളുടെ പിതാക്കന്‍മാര്‍ക്കും ദുരിതവും സന്തോഷവുമൊക്കെ വന്നുഭവിച്ചിട്ടുണ്ടല്ലോ എന്നാണ് അപ്പോള്‍ അവര്‍ പറഞ്ഞത്‌. അപ്പോള്‍ അവരറിയാതെ പെട്ടെന്ന് നാം അവരെ പിടികൂടി.
\end{malayalam}}
\flushright{\begin{Arabic}
\quranayah[7][96]
\end{Arabic}}
\flushleft{\begin{malayalam}
ആ നാടുകളിലുള്ളവര്‍ വിശ്വസിക്കുകയും, സൂക്ഷ്മത പാലിക്കുകയും ചെയ്തിരുന്നെങ്കില്‍ ആകാശത്തുനിന്നും ഭൂമിയില്‍ നിന്നും നാം അവര്‍ക്കു അനുഗ്രഹങ്ങള്‍ തുറന്നുകൊടുക്കുമായിരുന്നു. പക്ഷെ അവര്‍ നിഷേധിച്ചു തള്ളുകയാണ് ചെയ്തത്‌. അപ്പോള്‍ അവര്‍ ചെയ്ത് വെച്ചിരുന്നതിന്‍റെ ഫലമായി നാം അവരെ പിടികൂടി.
\end{malayalam}}
\flushright{\begin{Arabic}
\quranayah[7][97]
\end{Arabic}}
\flushleft{\begin{malayalam}
എന്നാല്‍ ആ നാടുകളിലുള്ളവര്‍ക്ക് അവര്‍ രാത്രിയില്‍ ഉറങ്ങിക്കൊണ്ടിരിക്കെ നമ്മുടെ ശിക്ഷ വന്നെത്തുന്നതിനെപ്പറ്റി അവര്‍ നിര്‍ഭയരായിരിക്കുകയാണോ?
\end{malayalam}}
\flushright{\begin{Arabic}
\quranayah[7][98]
\end{Arabic}}
\flushleft{\begin{malayalam}
ആ നാടുകളിലുള്ളവര്‍ക്ക് അവര്‍ പകല്‍ സമയത്ത് കളിച്ചു നടക്കുന്നതിനിടയില്‍ നമ്മുടെ ശിക്ഷ വന്നെത്തുന്നതിനെ പറ്റിയും അവര്‍ നിര്‍ഭയരായിരിക്കുകയാണോ?
\end{malayalam}}
\flushright{\begin{Arabic}
\quranayah[7][99]
\end{Arabic}}
\flushleft{\begin{malayalam}
അപ്പോള്‍ അല്ലാഹുവിന്‍റെ തന്ത്രത്തെപ്പറ്റി തന്നെ അവര്‍ നിര്‍ഭയരായിരിക്കുകയാണോ? എന്നാല്‍ നഷ്ടം പറ്റിയ ഒരു ജനവിഭാഗമല്ലാതെ അല്ലാഹുവിന്‍റെ തന്ത്രത്തെപ്പറ്റി നിര്‍ഭയരായിരിക്കുകയില്ല.
\end{malayalam}}
\flushright{\begin{Arabic}
\quranayah[7][100]
\end{Arabic}}
\flushleft{\begin{malayalam}
(പഴയ) അവകാശികള്‍ക്കു ശേഷം ഭൂമിയുടെ അനന്തരാവകാശികളായിത്തീരുന്നവര്‍ക്ക് നാം ഉദ്ദേശിക്കുകയാണെങ്കില്‍ അവരുടെ കുറ്റകൃത്യങ്ങളുടെ ഫലമായി നാം ശിക്ഷ ഏല്‍പിക്കുന്നതാണ് എന്ന ബോധം അവരെ നേര്‍വഴിക്ക് നയിക്കുന്നില്ലേ? നാം അവരുടെ ഹൃദയങ്ങളില്‍ മുദ്രവെക്കുകയും ചെയ്യും. അപ്പോള്‍ അവര്‍ (ഒന്നും) കേട്ടു മനസ്സിലാക്കാത്തവരായിത്തീരും.
\end{malayalam}}
\flushright{\begin{Arabic}
\quranayah[7][101]
\end{Arabic}}
\flushleft{\begin{malayalam}
ആ നാടുകളുടെ വൃത്താന്തങ്ങളില്‍ ചിലത് നാം നിനക്ക് വിവരിച്ചുതരികയാണ്‌. അവരിലേക്കയക്കപ്പെട്ട ദൂതന്‍മാര്‍ വ്യക്തമായ തെളിവുകളും കൊണ്ട് അവരുടെ അടുത്ത് ചെല്ലുകയുണ്ടായി. എന്നിട്ടും മുമ്പ് അവര്‍ നിഷേധിച്ചു തള്ളിയിരുന്നതില്‍ അവര്‍ വിശ്വസിക്കുകയുണ്ടായില്ല. സത്യനിഷേധികളുടെ ഹൃദയങ്ങളിന്‍മേല്‍ അപ്രകാരം അല്ലാഹു മുദ്രവെക്കും.
\end{malayalam}}
\flushright{\begin{Arabic}
\quranayah[7][102]
\end{Arabic}}
\flushleft{\begin{malayalam}
അവരില്‍ അധികപേര്‍ക്കും കരാറുപാലിക്കുന്ന സ്വഭാവം നാം കണ്ടില്ല. തീര്‍ച്ചയായും അവരില്‍ അധികപേരെയും ധിക്കാരികളായിത്തന്നെയാണ് നാം കണ്ടെത്തിയത്‌.
\end{malayalam}}
\flushright{\begin{Arabic}
\quranayah[7][103]
\end{Arabic}}
\flushleft{\begin{malayalam}
പിന്നീട് അവരുടെയൊക്കെ ശേഷം മൂസായെ നമ്മുടെ ദൃഷ്ടാന്തങ്ങളുമായി ഫിര്‍ഔന്‍റെയും അവന്‍റെ പ്രമാണിമാരുടെയും അടുക്കലേക്ക് നാം നിയോഗിച്ചു. എന്നാല്‍ അവര്‍ ആ ദൃഷ്ടാന്തങ്ങളോട് അന്യായം കാണിക്കുകയാണ് ചെയ്തത്‌. അപ്പോള്‍ നോക്കൂ; ആ നാശകാരികളുടെ പര്യവസാനം എങ്ങനെയായിരുന്നുവെന്ന്‌.
\end{malayalam}}
\flushright{\begin{Arabic}
\quranayah[7][104]
\end{Arabic}}
\flushleft{\begin{malayalam}
മൂസാ പറഞ്ഞു: ഫിര്‍ഔനേ, തീര്‍ച്ചയായും ഞാന്‍ ലോകരക്ഷിതാവിങ്കല്‍ നിന്നുള്ള ദൂതനാകുന്നു.
\end{malayalam}}
\flushright{\begin{Arabic}
\quranayah[7][105]
\end{Arabic}}
\flushleft{\begin{malayalam}
അല്ലാഹുവിന്‍റെ പേരില്‍ സത്യമല്ലാതൊന്നും പറയാതിരിക്കാന്‍ കടപ്പെട്ടവനാണ് ഞാന്‍. നിങ്ങളുടെ രക്ഷിതാവിങ്കല്‍ നിന്നുള്ള വ്യക്തമായ തെളിവും കൊണ്ടാണ് ഞാന്‍ നിങ്ങളുടെ അടുത്ത് വന്നിരിക്കുന്നത്‌. അതിനാല്‍ ഇസ്രായീല്‍ സന്തതികളെ എന്‍റെ കൂടെ അയക്കൂ.
\end{malayalam}}
\flushright{\begin{Arabic}
\quranayah[7][106]
\end{Arabic}}
\flushleft{\begin{malayalam}
ഫിര്‍ഔന്‍ പറഞ്ഞു: നീ തെളിവും കൊണ്ട് തന്നെയാണ് വന്നിട്ടുള്ളതെങ്കില്‍ അതിങ്ങ് കൊണ്ടുവാ; നീ സത്യവാന്‍മാരില്‍ പെട്ടവനാണെങ്കില്‍.
\end{malayalam}}
\flushright{\begin{Arabic}
\quranayah[7][107]
\end{Arabic}}
\flushleft{\begin{malayalam}
അപ്പോള്‍ മൂസാ തന്‍റെ വടി താഴെയിട്ടു. അപ്പോഴതാ അത് ഒരു പ്രത്യക്ഷമായ സര്‍പ്പമാകുന്നു.
\end{malayalam}}
\flushright{\begin{Arabic}
\quranayah[7][108]
\end{Arabic}}
\flushleft{\begin{malayalam}
അദ്ദേഹം തന്‍റെ കൈ പുറത്തെടുത്ത് കാണിച്ചു. അപ്പോഴതാ നിരീക്ഷിക്കുന്നവര്‍ക്കെല്ലാം അത് വെള്ളയായി കാണുന്നു.
\end{malayalam}}
\flushright{\begin{Arabic}
\quranayah[7][109]
\end{Arabic}}
\flushleft{\begin{malayalam}
ഫിര്‍ഔന്‍റെ ജനതയിലെ പ്രമാണിമാര്‍ പറഞ്ഞു: ഇവന്‍ നല്ല വിവരമുള്ള ജാലവിദ്യക്കാരന്‍ തന്നെ.
\end{malayalam}}
\flushright{\begin{Arabic}
\quranayah[7][110]
\end{Arabic}}
\flushleft{\begin{malayalam}
നിങ്ങളെ നിങ്ങളുടെ നാട്ടില്‍ നിന്ന് പുറത്താക്കാനാണ് അവന്‍ ഉദ്ദേശിക്കുന്നത്‌. അതിനാല്‍ നിങ്ങള്‍ക്കെന്താണ് നിര്‍ദേശിക്കാനുള്ളത്‌?
\end{malayalam}}
\flushright{\begin{Arabic}
\quranayah[7][111]
\end{Arabic}}
\flushleft{\begin{malayalam}
അവര്‍ (ഫിര്‍ഔനോട്‌) പറഞ്ഞു: ഇവന്നും ഇവന്‍റെ സഹോദരന്നും താങ്കള്‍ കുറച്ച് ഇടകൊടുക്കുക. നഗരങ്ങളില്‍ ചെന്ന് വിളിച്ചുകൂട്ടാന്‍ ആളുകളെ അയക്കുകയും ചെയ്യുക.
\end{malayalam}}
\flushright{\begin{Arabic}
\quranayah[7][112]
\end{Arabic}}
\flushleft{\begin{malayalam}
എല്ലാ വിവരമുള്ള ജാലവിദ്യക്കാരെയും അവര്‍ താങ്കളുടെ അടുക്കല്‍ കൊണ്ടുവരട്ടെ
\end{malayalam}}
\flushright{\begin{Arabic}
\quranayah[7][113]
\end{Arabic}}
\flushleft{\begin{malayalam}
ജാലവിദ്യക്കാര്‍ ഫിര്‍ഔന്‍റെ അടുത്ത് വന്നു. അവര്‍ പറഞ്ഞു: ഞങ്ങളാണ് ജയിക്കുന്നവരെങ്കില്‍ ഞങ്ങള്‍ക്കു നല്ല പ്രതിഫലമുണ്ടായിരിക്കുമെന്ന് തീര്‍ച്ചയാണല്ലോ?
\end{malayalam}}
\flushright{\begin{Arabic}
\quranayah[7][114]
\end{Arabic}}
\flushleft{\begin{malayalam}
ഫിര്‍ഔന്‍ പറഞ്ഞു: അതെ, തീര്‍ച്ചയായും നിങ്ങള്‍ (എന്‍റെ അടുക്കല്‍) സാമീപ്യം നല്‍കപ്പെടുന്നവരുടെ കൂട്ടത്തിലായിരിക്കുകയും ചെയ്യും.
\end{malayalam}}
\flushright{\begin{Arabic}
\quranayah[7][115]
\end{Arabic}}
\flushleft{\begin{malayalam}
അവര്‍ പറഞ്ഞു: ഹേ, മൂസാ ഒന്നുകില്‍ നീ ഇടുക. അല്ലെങ്കില്‍ ഞങ്ങളാകാം ഇടുന്നത്‌.
\end{malayalam}}
\flushright{\begin{Arabic}
\quranayah[7][116]
\end{Arabic}}
\flushleft{\begin{malayalam}
മൂസാ പറഞ്ഞു: നിങ്ങള്‍ ഇട്ടുകൊള്ളുക. അങ്ങനെ ഇട്ടപ്പോള്‍ അവര്‍ ആളുകളുടെ കണ്ണുകെട്ടുകയും അവര്‍ക്ക് ഭയമുണ്ടാക്കുകയും ചെയ്തു. വമ്പിച്ച ഒരു ജാലവിദ്യയാണ് അവര്‍ കൊണ്ടു വന്നത്‌.
\end{malayalam}}
\flushright{\begin{Arabic}
\quranayah[7][117]
\end{Arabic}}
\flushleft{\begin{malayalam}
മൂസായ്ക്ക് നാം ബോധനം നല്‍കി; നീ നിന്‍റെ വടി ഇട്ടേക്കുക എന്ന്‌. അപ്പോള്‍ ആ വടിയതാ അവര്‍ കൃത്രിമമായി ഉണ്ടാക്കിയതിനെ വിഴുങ്ങുന്നു.
\end{malayalam}}
\flushright{\begin{Arabic}
\quranayah[7][118]
\end{Arabic}}
\flushleft{\begin{malayalam}
അങ്ങനെ സത്യം സ്ഥിരപ്പെടുകയും, അവര്‍ പ്രവര്‍ത്തിച്ചിരുന്നതെല്ലാം നിഷ്ഫലമാകുകയും ചെയ്തു.
\end{malayalam}}
\flushright{\begin{Arabic}
\quranayah[7][119]
\end{Arabic}}
\flushleft{\begin{malayalam}
അങ്ങനെ അവിടെ വെച്ച് അവര്‍ പരാജയപ്പെടുകയും, അവര്‍ നിസ്സാരന്‍മാരായി മാറുകയും ചെയ്തു.
\end{malayalam}}
\flushright{\begin{Arabic}
\quranayah[7][120]
\end{Arabic}}
\flushleft{\begin{malayalam}
അവര്‍ (ആ ജാലവിദ്യക്കാര്‍) സാഷ്ടാംഗംചെയ്യുന്നവരായി വീഴുകയും ചെയ്തു.
\end{malayalam}}
\flushright{\begin{Arabic}
\quranayah[7][121]
\end{Arabic}}
\flushleft{\begin{malayalam}
അവര്‍ പറഞ്ഞു: ലോകരക്ഷിതാവില്‍ ഞങ്ങളിതാ വിശ്വസിച്ചിരിക്കുന്നു.
\end{malayalam}}
\flushright{\begin{Arabic}
\quranayah[7][122]
\end{Arabic}}
\flushleft{\begin{malayalam}
മൂസായുടെയും ഹാറൂന്‍റെയും രക്ഷിതാവില്‍.
\end{malayalam}}
\flushright{\begin{Arabic}
\quranayah[7][123]
\end{Arabic}}
\flushleft{\begin{malayalam}
ഫിര്‍ഔന്‍ പറഞ്ഞു: ഞാന്‍ നിങ്ങള്‍ക്ക് അനുവാദം നല്‍കുന്നതിന് മുമ്പ് നിങ്ങള്‍ വിശ്വസിച്ചിരിക്കുകയാണോ? ഈ നഗരത്തിലുള്ളവരെ ഇവിടെ നിന്ന് പുറത്താക്കാന്‍ വേണ്ടി നിങ്ങളെല്ലാം കൂടി ഇവിടെ വെച്ച് നടത്തിയ ഒരു ഗൂഢതന്ത്രം തന്നെയാണിത്‌. അതിനാല്‍ വഴിയെ നിങ്ങള്‍ മനസ്സിലാക്കിക്കൊള്ളും.
\end{malayalam}}
\flushright{\begin{Arabic}
\quranayah[7][124]
\end{Arabic}}
\flushleft{\begin{malayalam}
നിങ്ങളുടെ കൈകളും കാലുകളും എതിര്‍വശങ്ങളില്‍ നിന്നായി ഞാന്‍ മുറിച്ചുകളയുക തന്നെ ചെയ്യും. പിന്നെ നിങ്ങളെ മുഴുവന്‍ ഞാന്‍ ക്രൂശിക്കുകയും ചെയ്യും; തീര്‍ച്ച.
\end{malayalam}}
\flushright{\begin{Arabic}
\quranayah[7][125]
\end{Arabic}}
\flushleft{\begin{malayalam}
അവര്‍ പറഞ്ഞു: തീര്‍ച്ചയായും ഞങ്ങളുടെ രക്ഷിതാവിങ്കലേക്കാണല്ലോ ഞങ്ങള്‍ തിരിച്ചെത്തുന്നത്‌.
\end{malayalam}}
\flushright{\begin{Arabic}
\quranayah[7][126]
\end{Arabic}}
\flushleft{\begin{malayalam}
ഞങ്ങളുടെ രക്ഷിതാവിന്‍റെ ദൃഷ്ടാന്തങ്ങള്‍ ഞങ്ങള്‍ക്ക് വന്നപ്പോള്‍ ഞങ്ങള്‍ അത് വിശ്വസിച്ചു എന്നത് മാത്രമാണല്ലോ നീ ഞങ്ങളുടെ മേല്‍ കുറ്റം ചുമത്തുന്നത്‌. ഞങ്ങളുടെ രക്ഷിതാവേ, ഞങ്ങളുടെ മേല്‍ നീ ക്ഷമ ചൊരിഞ്ഞുതരികയും, ഞങ്ങളെ നീ മുസ്ലിംകളായിക്കൊണ്ട് മരിപ്പിക്കുകയും ചെയ്യേണമേ.
\end{malayalam}}
\flushright{\begin{Arabic}
\quranayah[7][127]
\end{Arabic}}
\flushleft{\begin{malayalam}
ഫിര്‍ഔന്‍റെ ജനതയിലെ പ്രമാണിമാര്‍ പറഞ്ഞു: ഭൂമിയില്‍ കുഴപ്പമുണ്ടാക്കുവാനും, താങ്കളേയും താങ്കളുടെ ദൈവങ്ങളേയും വിട്ടുകളയുവാനും താങ്കള്‍ മൂസായെയും അവന്‍റെ ആള്‍ക്കാരെയും (അനുവദിച്ച്‌) വിടുകയാണോ? അവന്‍ (ഫിര്‍ഔന്‍) പറഞ്ഞു: നാം അവരുടെ (ഇസ്രായീല്യരുടെ) ആണ്‍മക്കളെ കൊന്നൊടുക്കുകയും, അവരുടെ സ്ത്രീകളെ ജീവിക്കാന്‍ വിടുകയും ചെയ്യുന്നതാണ്‌. തീര്‍ച്ചയായും നാം അവരുടെ മേല്‍ സര്‍വ്വാധിപത്യമുള്ളവരായിരിക്കും.
\end{malayalam}}
\flushright{\begin{Arabic}
\quranayah[7][128]
\end{Arabic}}
\flushleft{\begin{malayalam}
മൂസാ തന്‍റെ ജനങ്ങളോട് പറഞ്ഞു: നിങ്ങള്‍ അല്ലാഹുവോട് സഹായം തേടുകയും ക്ഷമിക്കുകയും ചെയ്യുക. തീര്‍ച്ചയായും ഭൂമി അല്ലാഹുവിന്‍റെതാകുന്നു. അവന്‍റെ ദാസന്‍മാരില്‍ നിന്ന് അവന്‍ ഉദ്ദേശിക്കുന്നവര്‍ക്ക് അവന്‍ അത് അവകാശപ്പെടുത്തികൊടുക്കുന്നു. പര്യവസാനം ധര്‍മ്മനിഷ്ഠ പാലിക്കുന്നവര്‍ക്ക് അനുകൂലമായിരിക്കും.
\end{malayalam}}
\flushright{\begin{Arabic}
\quranayah[7][129]
\end{Arabic}}
\flushleft{\begin{malayalam}
അവര്‍ പറഞ്ഞു: താങ്കള്‍ ഞങ്ങളുടെ അടുത്ത് (ദൂതനായി) വരുന്നതിന്‍റെ മുമ്പും, താങ്കള്‍ ഞങ്ങളുടെ അടുത്ത് വന്നതിന് ശേഷവും ഞങ്ങള്‍ മര്‍ദ്ദിക്കപ്പെട്ടിരിക്കുകയാണ്‌. അദ്ദേഹം (മൂസാ) പറഞ്ഞു: നിങ്ങളുടെ രക്ഷിതാവ് നിങ്ങളുടെ ശത്രുവിനെ നശിപ്പിക്കുകയും, ഭൂമിയില്‍ നിങ്ങളെ അവന്‍ അനന്തരാവകാശികളാക്കുകയും ചെയ്തേക്കാം. എന്നിട്ട് നിങ്ങള്‍ എങ്ങനെ പ്രവര്‍ത്തിക്കുന്നുവെന്ന് അവന്‍ നോക്കുന്നതാണ്‌.
\end{malayalam}}
\flushright{\begin{Arabic}
\quranayah[7][130]
\end{Arabic}}
\flushleft{\begin{malayalam}
ഫിര്‍ഔന്‍റെ ആള്‍ക്കാരെ (വരള്‍ച്ചയുടെ) കൊല്ലങ്ങളും, വിളകളുടെ കമ്മിയും കൊണ്ട് നാം പിടികൂടുകയുണ്ടായി; അവര്‍ ചിന്തിച്ച് മനസ്സിലാക്കുവാന്‍ വേണ്ടി.
\end{malayalam}}
\flushright{\begin{Arabic}
\quranayah[7][131]
\end{Arabic}}
\flushleft{\begin{malayalam}
എന്നാല്‍ അവര്‍ക്കൊരു നന്‍മ വന്നാല്‍ അവര്‍ പറയുമായിരുന്നു: നമുക്ക് അര്‍ഹതയുള്ളത് തന്നെയാണിത്‌. ഇനി അവര്‍ക്ക് വല്ല തിന്‍മയും ബാധിച്ചുവെങ്കിലോ അത് മൂസായുടെയും കൂടെയുള്ളവരുടെയും ശകുനപ്പിഴയാണ് എന്നാണവര്‍ പറഞ്ഞിരുന്നത്‌. അല്ല, അവരുടെ ശകുനം അല്ലാഹുവിന്‍റെ പക്കല്‍ തന്നെയാകുന്നു. പക്ഷെ അവരില്‍ അധികപേരും മനസ്സിലാക്കുന്നില്ല.
\end{malayalam}}
\flushright{\begin{Arabic}
\quranayah[7][132]
\end{Arabic}}
\flushleft{\begin{malayalam}
അവര്‍ പറഞ്ഞു: ഞങ്ങളെ മായാജാലത്തില്‍ പെടുത്താന്‍ വേണ്ടി ഏതൊരു ദൃഷ്ടാന്തവുമായി നീ ഞങ്ങളുടെ അടുത്ത് വന്നാലും ഞങ്ങള്‍ നിന്നെ വിശ്വസിക്കാന്‍ പോകുന്നില്ല.
\end{malayalam}}
\flushright{\begin{Arabic}
\quranayah[7][133]
\end{Arabic}}
\flushleft{\begin{malayalam}
വെള്ളപ്പൊക്കം, വെട്ടുകിളി, പേന്‍, തവളകള്‍, രക്തം എന്നിങ്ങനെ വ്യക്തമായ ദൃഷ്ടാന്തങ്ങള്‍ അവരുടെ നേരെ നാം അയച്ചു. എന്നിട്ടും അവര്‍ അഹങ്കരിക്കുകയും കുറ്റവാളികളായ ജനതയായിരിക്കുകയും ചെയ്തു.
\end{malayalam}}
\flushright{\begin{Arabic}
\quranayah[7][134]
\end{Arabic}}
\flushleft{\begin{malayalam}
ശിക്ഷ അവരുടെ മേല്‍ വന്നുഭവിച്ചപ്പോള്‍ അവര്‍ പറഞ്ഞു: ഹേ; മൂസാ, നിന്‍റെ രക്ഷിതാവ് നിന്നോട് ചെയ്തിട്ടുള്ള കരാര്‍ മുന്‍നിര്‍ത്തി ഞങ്ങള്‍ക്ക് വേണ്ടി അവനോട് നീ പ്രാര്‍ത്ഥിക്കുക. ഞങ്ങളില്‍ നിന്ന് ഈ ശിക്ഷ അകറ്റിത്തരുന്ന പക്ഷം ഞങ്ങള്‍ നിന്നെ വിശ്വസിക്കുകയും, ഇസ്രായീല്‍ സന്തതികളെ നിന്‍റെ കൂടെ ഞങ്ങള്‍ അയച്ചു തരികയും ചെയ്യുന്നതാണ്‌; തീര്‍ച്ച.
\end{malayalam}}
\flushright{\begin{Arabic}
\quranayah[7][135]
\end{Arabic}}
\flushleft{\begin{malayalam}
എന്നാല്‍ അവര്‍ എത്തേണ്ടതായ ഒരു അവധിവരെ നാം അവരില്‍ നിന്ന് ശിക്ഷ അകറ്റികൊടുത്തപ്പോള്‍ അവരതാ വാക്ക് ലംഘിക്കുന്നു.
\end{malayalam}}
\flushright{\begin{Arabic}
\quranayah[7][136]
\end{Arabic}}
\flushleft{\begin{malayalam}
അപ്പോള്‍ നാം അവരുടെ കാര്യത്തില്‍ ശിക്ഷാനടപടി എടുത്തു. അങ്ങനെ അവരെ നാം കടലില്‍ മുക്കിക്കളഞ്ഞു. അവര്‍ നമ്മുടെ ദൃഷ്ടാന്തങ്ങളെ നിഷേധിച്ചുകളയുകയും അവയെപ്പറ്റി അശ്രദ്ധരായിരിക്കുകയും ചെയ്തതിന്‍റെ ഫലമത്രെ അത്‌.
\end{malayalam}}
\flushright{\begin{Arabic}
\quranayah[7][137]
\end{Arabic}}
\flushleft{\begin{malayalam}
അടിച്ചൊതുക്കപ്പെട്ടിരുന്ന ആ ജനതയ്ക്ക്‌, നാം അനുഗ്രഹിച്ച, കിഴക്കും പടിഞ്ഞാറുമുള്ള ഭൂപ്രദേശങ്ങള്‍ നാം അവകാശപ്പെടുത്തികൊടുക്കുകയും ചെയ്തു. ഇസ്രായീല്‍ സന്തതികളില്‍, അവര്‍ ക്ഷമിച്ചതിന്‍റെ ഫലമായി നിന്‍റെ രക്ഷിതാവിന്‍റെ ഉത്തമമായ വചനം നിറവേറുകയും, ഫിര്‍ഔനും അവന്‍റെ ജനതയും നിര്‍മിച്ചുകൊണ്ടിരുന്നതും, അവര്‍ കെട്ടി ഉയര്‍ത്തിയിരുന്നതും നാം തകര്‍ത്ത് കളയുകയും ചെയ്തു.
\end{malayalam}}
\flushright{\begin{Arabic}
\quranayah[7][138]
\end{Arabic}}
\flushleft{\begin{malayalam}
ഇസ്രായീല്‍ സന്തതികളെ നാം കടല്‍ കടത്തി (രക്ഷപ്പെടുത്തി.) എന്നിട്ട് തങ്ങളുടെ ബിംബങ്ങളുടെ മുമ്പാകെ ഭജനമിരിക്കുന്ന ഒരു ജനതയുടെ അടുക്കല്‍ അവര്‍ ചെന്നെത്തി. അവര്‍ പറഞ്ഞു: ഹേ; മൂസാ, ഇവര്‍ക്ക് ദൈവങ്ങളുള്ളത് പോലെ ഞങ്ങള്‍ക്കും ഒരു ദൈവത്തെ നീ ഏര്‍പെടുത്തിത്തരണം. അദ്ദേഹം പറഞ്ഞു: തീര്‍ച്ചയായും നിങ്ങള്‍ വിവരമില്ലാത്ത ഒരു ജനവിഭാഗമാകുന്നു.
\end{malayalam}}
\flushright{\begin{Arabic}
\quranayah[7][139]
\end{Arabic}}
\flushleft{\begin{malayalam}
തീര്‍ച്ചയായും ഈ കൂട്ടര്‍ എന്തൊന്നില്‍ നിലകൊള്ളുന്നുവോ അത് നശിപ്പിക്കപ്പെടുന്നതാണ്‌. അവര്‍ പ്രവര്‍ത്തിച്ച് കൊണ്ടിരിക്കുന്നതെല്ലാം നിഷ്ഫലവുമാകുന്നു.
\end{malayalam}}
\flushright{\begin{Arabic}
\quranayah[7][140]
\end{Arabic}}
\flushleft{\begin{malayalam}
അദ്ദേഹം പറഞ്ഞു: അല്ലാഹുവല്ലാത്തവരെയാണോ ഞാന്‍ നിങ്ങള്‍ക്ക് ദൈവമായി അന്വേഷിക്കേണ്ടത്‌? അവനാകട്ടെ നിങ്ങളെ ലോകരില്‍ വെച്ച് ഉല്‍കൃഷ്ടരാക്കിയിരിക്കുകയാണ്‌.
\end{malayalam}}
\flushright{\begin{Arabic}
\quranayah[7][141]
\end{Arabic}}
\flushleft{\begin{malayalam}
നിങ്ങള്‍ക്ക് കടുത്ത ശിക്ഷ അനുഭവിപ്പിക്കുകയും, നിങ്ങളുടെ ആണ്‍മക്കളെ കൊന്നൊടുക്കുകയും, നിങ്ങളുടെ സ്ത്രീകളെ ജീവിക്കാന്‍ വിടുകയും ചെയ്ത് കൊണ്ടിരുന്ന ഫിര്‍ഔന്‍റെ കൂട്ടരില്‍ നിന്ന് നിങ്ങളെ നാം രക്ഷപ്പെടുത്തിയ സന്ദര്‍ഭം (നിങ്ങള്‍ ഓര്‍ക്കുക.) നിങ്ങളുടെ രക്ഷിതാവിങ്കല്‍ നിന്നുള്ള ഒരു കടുത്ത പരീക്ഷണമാണ് അതിലുണ്ടായിരുന്നത്‌.
\end{malayalam}}
\flushright{\begin{Arabic}
\quranayah[7][142]
\end{Arabic}}
\flushleft{\begin{malayalam}
മൂസായ്ക്ക് നാം മുപ്പത് രാത്രി നിശ്ചയിച്ച് കൊടുക്കുകയും, പത്ത് കൂടി ചേര്‍ത്ത് അത് പൂര്‍ത്തിയാക്കുകയും ചെയ്തു. അങ്ങനെ അദ്ദേഹത്തിന്‍റെ രക്ഷിതാവ് നിശ്ചയിച്ച നാല്‍പത് രാത്രിയുടെ സമയപരിധി പൂര്‍ത്തിയായി. മൂസാ തന്‍റെ സഹോദരനായ ഹാറൂനോട് പറഞ്ഞു: എന്‍റെ ജനതയുടെ കാര്യത്തില്‍ നീ എന്‍റെ പ്രാതിനിധ്യം വഹിക്കുകയും, നല്ലത് പ്രവര്‍ത്തിക്കുകയും, കുഴപ്പക്കാരുടെ മാര്‍ഗം പിന്തുടരാതിരിക്കുകയും ചെയ്യുക.
\end{malayalam}}
\flushright{\begin{Arabic}
\quranayah[7][143]
\end{Arabic}}
\flushleft{\begin{malayalam}
നമ്മുടെ നിശ്ചിത സമയത്തിന് മൂസാ വരികയും, അദ്ദേഹത്തിന്‍റെ രക്ഷിതാവ് അദ്ദേഹത്തോട് സംസാരിക്കുകയും ചെയ്തപ്പോള്‍ മൂസാ പറഞ്ഞു: എന്‍റെ രക്ഷിതാവേ, (നിന്നെ) എനിക്കൊന്നു കാണിച്ചുതരൂ. ഞാന്‍ നിന്നെയൊന്ന് നോക്കിക്കാണട്ടെ. അവന്‍ (അല്ലാഹു) പറഞ്ഞു: നീ എന്നെ കാണുകയില്ല തന്നെ. എന്നാല്‍ നീ ആ മലയിലേക്ക് നോക്കൂ. അത് അതിന്‍റെ സ്ഥാനത്ത് ഉറച്ചുനിന്നാല്‍ വഴിയെ നിനക്കെന്നെ കാണാം. അങ്ങനെ അദ്ദേഹത്തിന്‍റെ രക്ഷിതാവ് പര്‍വ്വതത്തിന് വെളിപ്പെട്ടപ്പോള്‍ അതിനെ അവന്‍ പൊടിയാക്കി. മൂസാ ബോധരഹിതനായി വീഴുകയും ചെയ്തു. എന്നിട്ട് അദ്ദേഹത്തിന് ബോധം വന്നപ്പോള്‍ അദ്ദേഹം പറഞ്ഞു: നീയെത്ര പരിശുദ്ധന്‍! ഞാന്‍ നിന്നിലേക്ക് ഖേദിച്ചുമടങ്ങിയിരിക്കുന്നു. ഞാന്‍ വിശ്വാസികളില്‍ ഒന്നാമനാകുന്നു.
\end{malayalam}}
\flushright{\begin{Arabic}
\quranayah[7][144]
\end{Arabic}}
\flushleft{\begin{malayalam}
അവന്‍ (അല്ലാഹു) പറഞ്ഞു: ഹേ; മൂസാ, എന്‍റെ സന്ദേശങ്ങള്‍കൊണ്ടും, എന്‍റെ (നേരിട്ടുള്ള) സംസാരം കൊണ്ടും തീര്‍ച്ചയായും നിന്നെ ജനങ്ങളില്‍ ഉല്‍കൃഷ്ടനായി ഞാന്‍ തെരഞ്ഞെടുത്തിരിക്കുന്നു. അതിനാല്‍ ഞാന്‍ നിനക്ക് നല്‍കിയത് സ്വീകരിക്കുകയും നന്ദിയുള്ളവരുടെ കൂട്ടത്തിലായിരിക്കുകയും ചെയ്യുക.
\end{malayalam}}
\flushright{\begin{Arabic}
\quranayah[7][145]
\end{Arabic}}
\flushleft{\begin{malayalam}
എല്ലാകാര്യത്തെപ്പറ്റിയും നാം അദ്ദേഹത്തിന് (മൂസായ്ക്ക്‌) പലകകളില്‍ എഴുതികൊടുക്കുകയും ചെയ്തു. അതായത് സദുപദേശവും, എല്ലാ കാര്യത്തെപ്പറ്റിയുള്ള വിശദീകരണവും. (നാം പറഞ്ഞു:) അവയെ മുറുകെപിടിക്കുകയും, അവയിലെ വളരെ നല്ല കാര്യങ്ങള്‍ സ്വീകരിക്കാന്‍ നിന്‍റെ ജനതയോട് കല്‍പിക്കുകയും ചെയ്യുക. ധിക്കാരികളുടെ പാര്‍പ്പിടം വഴിയെ ഞാന്‍ നിങ്ങള്‍ക്ക് കാണിച്ചുതരുന്നതാണ്‌.
\end{malayalam}}
\flushright{\begin{Arabic}
\quranayah[7][146]
\end{Arabic}}
\flushleft{\begin{malayalam}
ന്യായം കൂടാതെ ഭൂമിയില്‍ അഹങ്കാരം നടിച്ച് കൊണ്ടിരിക്കുന്നവരെ എന്‍റെ ദൃഷ്ടാന്തങ്ങളില്‍ നിന്ന് ഞാന്‍ തിരിച്ചുകളയുന്നതാണ്‌. എല്ലാ ദൃഷ്ടാന്തവും കണ്ടാലും അവരതില്‍ വിശ്വസിക്കുകയില്ലണേര്‍മാര്‍ഗം കണ്ടാല്‍ അവര്‍ അതിനെ മാര്‍ഗമായി സ്വീകരിക്കുകയില്ല. ദുര്‍മാര്‍ഗം കണ്ടാല്‍ അവരത് മാര്‍ഗമായി സ്വീകരിക്കുകയും ചെയ്യും. നമ്മുടെ ദൃഷ്ടാന്തങ്ങള്‍ അവര്‍ നിഷേധിച്ച് തള്ളുകയും , അവയെപ്പറ്റി അവര്‍ അശ്രദ്ധരായിരിക്കുകയും ചെയ്തതിന്‍റെ ഫലമാണത്‌.
\end{malayalam}}
\flushright{\begin{Arabic}
\quranayah[7][147]
\end{Arabic}}
\flushleft{\begin{malayalam}
നമ്മുടെ ദൃഷ്ടാന്തങ്ങളെയും, പരലോകത്തെ കണ്ടുമുട്ടുന്നതിനെയും നിഷേധിച്ച് കളഞ്ഞവരാരോ അവരുടെ കര്‍മ്മങ്ങള്‍ നിഷ്ഫലമായിരിക്കുന്നു. അവര്‍ പ്രവര്‍ത്തിച്ചു കൊണ്ടിരുന്നതിന്‍റെ ഫലമല്ലാതെ അവര്‍ക്കു നല്‍കപ്പെടുമോ?
\end{malayalam}}
\flushright{\begin{Arabic}
\quranayah[7][148]
\end{Arabic}}
\flushleft{\begin{malayalam}
മൂസായുടെ ജനത അദ്ദേഹം പോയതിനു ശേഷം അവരുടെ ആഭരണങ്ങള്‍ കൊണ്ടുണ്ടാക്കിയ മുക്രയിടുന്ന ഒരു കാളക്കുട്ടിയുടെ സ്വരൂപത്തെ ദൈവമായി സ്വീകരിച്ചു. അതവരോട് സംസാരിക്കുകയില്ലെന്നും, അവര്‍ക്ക് വഴി കാണിക്കുകയില്ലെന്നും അവര്‍ കണ്ടില്ലേ? അതിനെ അവര്‍ (ദൈവമായി) സ്വീകരിക്കുകയും അതോടെ അവര്‍ അക്രമികളാവുകയും ചെയ്തിരിക്കുന്നു.
\end{malayalam}}
\flushright{\begin{Arabic}
\quranayah[7][149]
\end{Arabic}}
\flushleft{\begin{malayalam}
അവര്‍ക്കു ഖേദം തോന്നുകയും, തങ്ങള്‍ പിഴച്ച് പോയിരിക്കുന്നു എന്ന് അവര്‍ കാണുകയും ചെയ്തപ്പോള്‍ അവര്‍ പറഞ്ഞു: ഞങ്ങളുടെ രക്ഷിതാവ് ഞങ്ങളോട് കരുണ കാണിക്കുകയും, ഞങ്ങള്‍ക്ക് പൊറുത്തുതരികയും ചെയ്തിട്ടില്ലെങ്കില്‍ തീര്‍ച്ചയായും ഞങ്ങള്‍ നഷ്ടക്കാരില്‍ പെട്ടവരായിരിക്കും.
\end{malayalam}}
\flushright{\begin{Arabic}
\quranayah[7][150]
\end{Arabic}}
\flushleft{\begin{malayalam}
കുപിതനും ദുഃഖിതനുമായിക്കൊണ്ട് തന്‍റെ ജനങ്ങളിലേക്ക് മടങ്ങി വന്നിട്ട് മൂസാ പറഞ്ഞു: ഞാന്‍ പോയ ശേഷം എന്‍റെ പിന്നില്‍ നിങ്ങള്‍ പ്രവര്‍ത്തിച്ച കാര്യം വളരെ ചീത്ത തന്നെ. നിങ്ങളുടെ രക്ഷിതാവിന്‍റെ കല്‍പന കാത്തിരിക്കാതെ നിങ്ങള്‍ ധൃതികൂട്ടിയോ? അദ്ദേഹം പലകകള്‍ താഴെയിടുകയും, തന്‍റെ സഹോദരന്‍റെ തല പിടിച്ച് തന്‍റെ അടുത്തേക്ക് വലിക്കുകയും ചെയ്തു. അവന്‍ (സഹോദരന്‍) പറഞ്ഞു: എന്‍റെ ഉമ്മയുടെ മകനേ, ജനങ്ങള്‍ എന്നെ ദുര്‍ബലനായി ഗണിച്ചു. അവരെന്നെ കൊന്നേക്കുമായിരുന്നു. അതിനാല്‍ (എന്നോട് കയര്‍ത്തു കൊണ്ട്‌) നീ ശത്രുക്കള്‍ക്ക് സന്തോഷത്തിന് ഇടവരുത്തരുത്‌. അക്രമികളായ ജനങ്ങളുടെ കൂട്ടത്തില്‍ എന്നെ കണക്കാക്കുകയും ചെയ്യരുത്‌.
\end{malayalam}}
\flushright{\begin{Arabic}
\quranayah[7][151]
\end{Arabic}}
\flushleft{\begin{malayalam}
അദ്ദേഹം (മൂസാ) പറഞ്ഞു: എന്‍റെ രക്ഷിതാവേ, എനിക്കും എന്‍റെ സഹോദരന്നും നീ പൊറുത്തുതരികയും, ഞങ്ങളെ നീ നിന്‍റെ കാരുണ്യത്തില്‍ ഉള്‍പെടുത്തുകയും ചെയ്യേണമേ. നീ പരമകാരുണികനാണല്ലോ.
\end{malayalam}}
\flushright{\begin{Arabic}
\quranayah[7][152]
\end{Arabic}}
\flushleft{\begin{malayalam}
കാളക്കുട്ടിയെ ദൈവമായി സ്വീകരിച്ചവരാരോ അവര്‍ക്കു തങ്ങളുടെ രക്ഷിതാവിങ്കല്‍ നിന്നുള്ള കോപവും, ഐഹികജീവിതത്തില്‍ നിന്ദ്യതയും വന്നുഭവിക്കുന്നതാണ്‌. കള്ളം കെട്ടിച്ചമയ്ക്കുന്നവര്‍ക്കു നാം പ്രതിഫലം നല്‍കുന്നത് അപ്രകാരമത്രെ.
\end{malayalam}}
\flushright{\begin{Arabic}
\quranayah[7][153]
\end{Arabic}}
\flushleft{\begin{malayalam}
എന്നാല്‍ തിന്‍മകള്‍ പ്രവര്‍ത്തിക്കുകയും, എന്നിട്ടതിനു ശേഷം പശ്ചാത്തപിക്കുകയും, വിശ്വസിക്കുകയും ചെയ്തവര്‍ക്കു തീര്‍ച്ചയായും നിന്‍റെ രക്ഷിതാവ് അതിന് ശേഷം ഏറെ പൊറുത്തുകൊടുക്കുകയും, കരുണ കാണിക്കുകയും ചെയ്യുന്നവനാകുന്നു.
\end{malayalam}}
\flushright{\begin{Arabic}
\quranayah[7][154]
\end{Arabic}}
\flushleft{\begin{malayalam}
മൂസായുടെ കോപം അടങ്ങിയപ്പോള്‍ അദ്ദേഹം (ദിവ്യസന്ദേശമെഴുതിയ) പലകകള്‍ എടുത്തു. അവയില്‍ രേഖപ്പെടുത്തിയതില്‍ തങ്ങളുടെ രക്ഷിതാവിനെ ഭയപ്പെടുന്ന ആളുകള്‍ക്ക് മാര്‍ഗദര്‍ശനവും കാരുണ്യവുമാണുണ്ടായിരുന്നത്‌.
\end{malayalam}}
\flushright{\begin{Arabic}
\quranayah[7][155]
\end{Arabic}}
\flushleft{\begin{malayalam}
നമ്മുടെ നിശ്ചിത സമയത്തേക്ക് മൂസാ തന്‍റെ ജനങ്ങളില്‍ നിന്ന് എഴുപത് പുരുഷന്‍മാരെ തെരഞ്ഞെടുത്തു. എന്നിട്ട് ഉഗ്രമായ കുലുക്കം അവര്‍ക്ക് പിടിപെട്ടപ്പോള്‍ അദ്ദേഹം പറഞ്ഞു: എന്‍റെ രക്ഷിതാവേ, നീ ഉദ്ദേശിച്ചിരുന്നെങ്കില്‍ മുമ്പ് തന്നെ അവരെയും എന്നെയും നിനക്ക് നശിപ്പിക്കാമായിരുന്നു. ഞങ്ങളുടെ കൂട്ടത്തിലെ മൂഢന്‍മാര്‍ പ്രവര്‍ത്തിച്ചതിന്‍റെ പേരില്‍ നീ ഞങ്ങളെ നശിപ്പിക്കുകയാണോ? അത് നിന്‍റെ പരീക്ഷണമല്ലാതെ മറ്റൊന്നുമല്ല. അത് മൂലം നീ ഉദ്ദേശിക്കുന്നവരെ നീ പിഴവിലാക്കുകയും നീ ഉദ്ദേശിക്കുന്നവരെ നീ നേര്‍വഴിയിലാക്കുകയും ചെയ്യുന്നു. നീയാണ് ഞങ്ങളുടെ രക്ഷാധികാരി. അതിനാല്‍ ഞങ്ങള്‍ക്ക് നീ പൊറുത്തുതരികയും, ഞങ്ങളോട് കരുണ കാണിക്കുകയും ചെയ്യേണമേ. നീയാണ് പൊറുക്കുന്നവരില്‍ ഉത്തമന്‍.
\end{malayalam}}
\flushright{\begin{Arabic}
\quranayah[7][156]
\end{Arabic}}
\flushleft{\begin{malayalam}
ഇഹലോകത്തും പരലോകത്തും ഞങ്ങള്‍ക്ക് നീ നന്‍മ രേഖപ്പെടുത്തുകയും (അഥവാ വിധിക്കുകയും) ചെയ്യേണമേ. തീര്‍ച്ചയായും ഞങ്ങള്‍ നിന്നിലേക്ക് മടങ്ങിയിരിക്കുന്നു. അവന്‍ (അല്ലാഹു) പറഞ്ഞു: എന്‍റെ ശിക്ഷ ഞാന്‍ ഉദ്ദേശിക്കുന്നവര്‍ക്ക് ഏല്‍പിക്കുന്നതാണ്‌. എന്‍റെ കാരുണ്യമാകട്ടെ സര്‍വ്വ വസ്തുക്കളെയും ഉള്‍കൊള്ളുന്നതായിരിക്കും. എന്നാല്‍ ധര്‍മ്മനിഷ്ഠ പാലിക്കുകയും, സകാത്ത് നല്‍കുകയും, നമ്മുടെ ദൃഷ്ടാന്തങ്ങളില്‍ വിശ്വസിക്കുകയും ചെയ്യുന്നവരായ ആളുകള്‍ക്ക് (പ്രത്യേകമായി) ഞാന്‍ അത് രേഖപ്പെടുത്തുന്നതാണ്‌.
\end{malayalam}}
\flushright{\begin{Arabic}
\quranayah[7][157]
\end{Arabic}}
\flushleft{\begin{malayalam}
(അതായത്‌) തങ്ങളുടെ പക്കലുള്ള തൌറാത്തിലും ഇന്‍ജീലിലും രേഖപ്പെടുത്തപ്പെട്ടതായി അവര്‍ക്ക് കണ്ടെത്താന്‍ കഴിയുന്ന ആ അക്ഷരജ്ഞാനമില്ലാത്ത പ്രവാചകനായ ദൈവദൂതനെ (മുഹമ്മദ് നബിയെ) പിന്‍പറ്റുന്നവര്‍ക്ക് (ആ കാരുണ്യം രേഖപ്പെടുത്തുന്നതാണ്‌.) അവരോട് അദ്ദേഹം സദാചാരം കല്‍പിക്കുകയും, ദുരാചാരത്തില്‍ നിന്ന് അവരെ വിലക്കുകയും ചെയ്യുന്നു. നല്ല വസ്തുക്കള്‍ അവര്‍ക്ക് അനുവദനീയമാക്കുകയും, ചീത്ത വസ്തുക്കള്‍ അവരുടെ മേല്‍ നിഷിദ്ധമാക്കുകയും ചെയ്യുന്നു. അവരുടെ ഭാരങ്ങളും അവരുടെ മേലുണ്ടായിരുന്ന വിലങ്ങുകളും അദ്ദേഹം ഇറക്കിവെക്കുകയും ചെയ്യുന്നു. അപ്പോള്‍ അദ്ദേഹത്തില്‍ വിശ്വസിക്കുകയും അദ്ദേഹത്തെ പിന്തുണക്കുകയും സഹായിക്കുകയും അദ്ദേഹത്തോടൊപ്പം അവതരിപ്പിക്കപ്പെട്ടിട്ടുള്ള ആ പ്രകാശത്തെ പിന്‍പറ്റുകയും ചെയ്തവരാരോ, അവര്‍ തന്നെയാണ് വിജയികള്‍.
\end{malayalam}}
\flushright{\begin{Arabic}
\quranayah[7][158]
\end{Arabic}}
\flushleft{\begin{malayalam}
പറയുക: മനുഷ്യരേ, തീര്‍ച്ചയായും ഞാന്‍ നിങ്ങളിലേക്കെല്ലാമുള്ള അല്ലാഹുവിന്‍റെ ദൂതനാകുന്നു. ആകാശങ്ങളുടെയും ഭൂമിയുടെയും ആധിപത്യം ഏതൊരുവന്നാണോ അവന്‍റെ (ദൂതന്‍.) അവനല്ലാതെ ഒരു ദൈവവുമില്ല. അവന്‍ ജീവിപ്പിക്കുകയും മരിപ്പിക്കുകയും ചെയ്യുന്നു. അതിനാല്‍ നിങ്ങള്‍ അല്ലാഹുവിലും അവന്‍റെ ദൂതനിലും വിശ്വസിക്കുവിന്‍. അതെ, അല്ലാഹുവിലും അവന്‍റെ വചനങ്ങളിലും വിശ്വസിക്കുന്ന അക്ഷരജ്ഞാനമില്ലാത്ത ആ പ്രവാചകനില്‍. അദ്ദേഹത്തെ നിങ്ങള്‍ പിന്‍പറ്റുവിന്‍ നിങ്ങള്‍ നേര്‍മാര്‍ഗം പ്രാപിക്കാം.
\end{malayalam}}
\flushright{\begin{Arabic}
\quranayah[7][159]
\end{Arabic}}
\flushleft{\begin{malayalam}
മൂസായുടെ ജനതയില്‍ തന്നെ സത്യത്തിന്‍റെ അടിസ്ഥാനത്തില്‍ മാര്‍ഗദര്‍ശനം ചെയ്യുകയും അതനുസരിച്ച് തന്നെ നീതി പാലിക്കുകയും ചെയ്യുന്ന ഒരു സമൂഹമുണ്ട്‌.
\end{malayalam}}
\flushright{\begin{Arabic}
\quranayah[7][160]
\end{Arabic}}
\flushleft{\begin{malayalam}
അവരെ നാം പന്ത്രണ്ട് ഗോത്രങ്ങളായി അഥവാ സമൂഹങ്ങളായി പിരിച്ചു. മൂസായോട് അദ്ദേഹത്തിന്‍റെ ജനത കുടിനീര്‍ ആവശ്യപ്പെട്ട സമയത്ത് നിന്‍റെ വടികൊണ്ട് ആ പാറക്കല്ലില്‍ അടിക്കൂ എന്ന് അദ്ദേഹത്തിന് നാം ബോധനം നല്‍കി. അപ്പോള്‍ അതില്‍ നിന്ന് പന്ത്രണ്ടു നീര്‍ചാലുകള്‍ പൊട്ടി ഒഴുകി. ഓരോ വിഭാഗക്കാരും തങ്ങള്‍ക്ക് കുടിക്കാനുള്ള സ്ഥലം മനസ്സിലാക്കി. നാം അവര്‍ക്ക് മേഘത്തണല്‍ നല്‍കുകയും, മന്നായും കാടപക്ഷികളും നാം അവര്‍ക്ക് ഇറക്കികൊടുക്കുകയും ചെയ്തു. നിങ്ങള്‍ക്കു നാം നല്‍കിയിട്ടുള്ള വിശിഷ്ട വസ്തുക്കളില്‍ നിന്ന് തിന്നുകൊള്ളുക (എന്ന് നാം നിര്‍ദേശിക്കുകയും ചെയ്തു.) (അവരുടെ ധിക്കാരം നിമിത്തം) നമുക്ക് അവര്‍ ഒരു ദ്രോഹവും വരുത്തിയിട്ടില്ല. എന്നാല്‍ അവര്‍ ദ്രോഹം വരുത്തിവെച്ചിരുന്നത് അവര്‍ക്കു തന്നെയാണ്‌.
\end{malayalam}}
\flushright{\begin{Arabic}
\quranayah[7][161]
\end{Arabic}}
\flushleft{\begin{malayalam}
നിങ്ങള്‍ ഈ രാജ്യത്ത് താമസിക്കുകയും ഇവിടെ നിങ്ങള്‍ക്ക് ഇഷ്ടമുള്ളേടത്ത് നിന്ന് തിന്നുകയും ചെയ്ത് കൊള്ളുക. നിങ്ങള്‍ പാപമോചനത്തിന് പ്രാര്‍ത്ഥിക്കുകയും, തലകുനിച്ച് കൊണ്ട് പട്ടണവാതില്‍ കടക്കുകയും ചെയ്യുക. എങ്കില്‍ നിങ്ങളുടെ തെറ്റുകള്‍ നിങ്ങള്‍ക്കു നാം പൊറുത്തുതരുന്നതാണ്‌. സല്‍കര്‍മ്മകാരികള്‍ക്ക് വഴിയെ നാം കൂടുതല്‍ കൊടുക്കുന്നതുമാണ് എന്ന് അവരോട് പറയപ്പെട്ട സന്ദര്‍ഭവും (ഓര്‍ക്കുക.)
\end{malayalam}}
\flushright{\begin{Arabic}
\quranayah[7][162]
\end{Arabic}}
\flushleft{\begin{malayalam}
അപ്പോള്‍ അവരിലുള്ള അക്രമികള്‍ അവരോട് നിര്‍ദേശിക്കപ്പെട്ടതില്‍ നിന്ന് വ്യത്യസ്തമായിട്ട് വാക്കു മാറ്റിപ്പറയുകയാണ് ചെയ്തത്‌. അവര്‍ അക്രമം ചെയ്ത്കൊണ്ടിരുന്നതിന്‍റെ ഫലമായി നാം അവരുടെ മേല്‍ ആകാശത്ത് നിന്ന് ഒരു ശിക്ഷ അയച്ചു.
\end{malayalam}}
\flushright{\begin{Arabic}
\quranayah[7][163]
\end{Arabic}}
\flushleft{\begin{malayalam}
കടല്‍ത്തീരത്ത് സ്ഥിതിചെയ്തിരുന്ന ആ പട്ടണത്തെപ്പറ്റി നീ അവരോട് ചോദിച്ച് നോക്കൂ. (അതായത്‌) ശബ്ബത്ത് ദിനം (ശനിയാഴ്ച) ആചരിക്കുന്നതില്‍ അവര്‍ അതിക്രമം കാണിച്ചിരുന്ന സന്ദര്‍ഭത്തെപ്പറ്റി. അവരുടെ ശബ്ബത്ത് ദിനത്തില്‍ അവര്‍ക്ക് ആവശ്യമുള്ള മത്സ്യങ്ങള്‍ വെള്ളത്തിനു മീതെ തലകാണിച്ചുകൊണ്ട് അവരുടെ അടുത്ത് വരുകയും അവര്‍ക്ക് ശബ്ബത്ത് ആചരിക്കാനില്ലാത്ത ദിവസത്തില്‍ അവരുടെ അടുത്ത് അവ വരാതിരിക്കുകയും ചെയ്തിരുന്നസന്ദര്‍ഭം. അവര്‍ ധിക്കരിച്ചിരുന്നതിന്‍റെ ഫലമായി അപ്രകാരം നാം അവരെ പരീക്ഷിക്കുകയായിരുന്നു.
\end{malayalam}}
\flushright{\begin{Arabic}
\quranayah[7][164]
\end{Arabic}}
\flushleft{\begin{malayalam}
അല്ലാഹു നശിപ്പിക്കുകയോ കഠിനമായി ശിക്ഷിക്കുകയോ ചെയ്യാന്‍ പോകുന്ന ഒരു ജനവിഭാഗത്തെ നിങ്ങളെന്തിനാണ് ഉപദേശിക്കുന്നത്‌? എന്ന് അവരില്‍ പെട്ട ഒരു സമൂഹം പറഞ്ഞ സന്ദര്‍ഭം (ശ്രദ്ധിക്കുക) അവര്‍ മറുപടി പറഞ്ഞു: നിങ്ങളുടെ രക്ഷിതാവിങ്കല്‍ (ഞങ്ങള്‍) അപരാധത്തില്‍ നിന്ന് ഒഴിവാകുന്നതിന് വേണ്ടിയാണ്‌. ഒരു വേള അവര്‍ സൂക്ഷ്മത പാലിച്ചെന്നും വരാമല്ലോ.
\end{malayalam}}
\flushright{\begin{Arabic}
\quranayah[7][165]
\end{Arabic}}
\flushleft{\begin{malayalam}
എന്നാല്‍ അവരെ ഓര്‍മപ്പെടുത്തിയിരുന്നത് അവര്‍ മറന്നുകളഞ്ഞപ്പോള്‍ ദുഷ്പ്രവൃത്തിയില്‍ നിന്ന് വിലക്കിയിരുന്നവരെ നാം രക്ഷപ്പെടുത്തുകയും, അക്രമികളായ ആളുകളെ അവര്‍ ധിക്കാരം കാണിച്ചിരുന്നതിന്‍റെ ഫലമായി നാം കഠിനമായ ശിക്ഷ മുഖേന പിടികൂടുകയും ചെയ്തു.
\end{malayalam}}
\flushright{\begin{Arabic}
\quranayah[7][166]
\end{Arabic}}
\flushleft{\begin{malayalam}
അങ്ങനെ അവരോട് വിലക്കപ്പെട്ടതിന്‍റെ കാര്യത്തിലെല്ലാം അവര്‍ ധിക്കാരം പ്രവര്‍ത്തിച്ചപ്പോള്‍ നാം അവരോട് പറഞ്ഞു: നിങ്ങള്‍ നിന്ദ്യന്‍മാരായ കുരങ്ങന്‍മാരായിക്കൊള്ളുക.
\end{malayalam}}
\flushright{\begin{Arabic}
\quranayah[7][167]
\end{Arabic}}
\flushleft{\begin{malayalam}
അവരുടെ (ഇസ്രായീല്യരുടെ) മേല്‍ ഉയിര്‍ത്തെഴുന്നേല്‍പിന്‍റെ നാളുവരെ അവര്‍ക്കു ഹീനമായ ശിക്ഷ ഏല്‍പിച്ച് കൊണ്ടിരിക്കുന്നവരെ നിന്‍റെ രക്ഷിതാവ് നിയോഗിക്കുക തന്നെ ചെയ്യുമെന്ന് അവന്‍ പ്രഖ്യാപിച്ച സന്ദര്‍ഭവും ഓര്‍ക്കുക. തീര്‍ച്ചയായും നിന്‍റെ രക്ഷിതാവ് അതിവേഗം ശിക്ഷ നടപ്പാക്കുന്നവനാണ്‌. തീര്‍ച്ചയായും അവന്‍ ഏറെ പൊറുക്കുന്നവനും കരുണാനിധിയുമത്രെ.
\end{malayalam}}
\flushright{\begin{Arabic}
\quranayah[7][168]
\end{Arabic}}
\flushleft{\begin{malayalam}
ഭൂമിയില്‍ അവരെ നാം പല സമൂഹങ്ങളായി പിരിക്കുകയും ചെയ്തിരിക്കുന്നു. അവരുടെ കൂട്ടത്തില്‍ സദ്‌വൃത്തരുണ്ട്‌. അതിന് താഴെയുള്ളവരും അവരിലുണ്ട്‌. അവര്‍ മടങ്ങേണ്ടതിനായി നാം അവരെ നന്‍മകള്‍കൊണ്ടും തിന്‍മകള്‍ കൊണ്ടും പരീക്ഷിക്കുകയുണ്ടായി.
\end{malayalam}}
\flushright{\begin{Arabic}
\quranayah[7][169]
\end{Arabic}}
\flushleft{\begin{malayalam}
അനന്തരം അവര്‍ക്ക് ശേഷം അവരുടെ പിന്‍ഗാമികളായി ഒരു തലമുറ രംഗത്ത് വന്നു. അവര്‍ വേദത്തിന്‍റെ അനന്തരാവകാശമെടുത്തു. ഈ നിസ്സാരമായ ലോകത്തിലെ വിഭവങ്ങളാണ് അവര്‍ കൈപ്പറ്റുന്നത്‌. ഞങ്ങള്‍ക്ക് അതൊക്കെ പൊറുത്തുകിട്ടുന്നതാണ് എന്ന് അവര്‍ പറയുകയും ചെയ്യും. അത്തരത്തിലുള്ള മറ്റൊരു വിഭവം അവര്‍ക്ക് വന്നുകിട്ടുകയാണെങ്കിലും അവരത് സ്വീകരിച്ചേക്കും. അല്ലാഹുവെപ്പറ്റി സത്യമല്ലാതെ ഒന്നും അവര്‍ പറയുകയില്ലെന്ന് വേദഗ്രന്ഥത്തിലൂടെ അവരോട് ഉറപ്പ് മേടിക്കപ്പെടുകയും, അതിലുള്ളത് അവര്‍ വായിച്ചുപഠിക്കുകയും ചെയ്തിട്ടില്ലേ? എന്നാല്‍ പരലോകമാണ് സൂക്ഷ്മത പാലിക്കുന്നവര്‍ക്ക് ഉത്തമമായിട്ടുള്ളത്‌. നിങ്ങള്‍ ചിന്തിച്ച് മനസ്സിലാക്കുന്നില്ലേ?
\end{malayalam}}
\flushright{\begin{Arabic}
\quranayah[7][170]
\end{Arabic}}
\flushleft{\begin{malayalam}
വേദഗ്രന്ഥത്തെ മുറുകെപിടിക്കുകയും, പ്രാര്‍ത്ഥന മുറപോലെ നിര്‍വഹിക്കുകയും ചെയ്യുന്നവരാരോ ആ സല്‍കര്‍മ്മകാരികള്‍ക്കുള്ള പ്രതിഫലം നാം നഷ്ടപ്പെടുത്തിക്കളയുകയില്ല; തീര്‍ച്ച.
\end{malayalam}}
\flushright{\begin{Arabic}
\quranayah[7][171]
\end{Arabic}}
\flushleft{\begin{malayalam}
നാം പര്‍വ്വതത്തെ അവര്‍ക്കു മീതെ ഒരു കുടയെന്നോണം ഉയര്‍ത്തി നിര്‍ത്തുകയും അതവരുടെ മേല്‍ വീഴുക തന്നെ ചെയ്യുമെന്ന് അവര്‍ വിചാരിക്കുകയും ചെയ്ത സന്ദര്‍ഭം ഓര്‍ക്കുക. (നാം പറഞ്ഞു:) നാം നിങ്ങള്‍ക്ക് നല്‍കിയത് മുറുകെപിടിക്കുകയും, അതിലുള്ളത് നിങ്ങള്‍ ഓര്‍മിക്കുകയും ചെയ്യുക. നിങ്ങള്‍ സൂക്ഷ്മത പാലിക്കുന്നവരായേക്കാം.
\end{malayalam}}
\flushright{\begin{Arabic}
\quranayah[7][172]
\end{Arabic}}
\flushleft{\begin{malayalam}
നിന്‍റെ രക്ഷിതാവ് ആദം സന്തതികളില്‍ നിന്ന്‌, അവരുടെ മുതുകുകളില്‍ നിന്ന് അവരുടെ സന്താനങ്ങളെ പുറത്ത് കൊണ്ട് വരികയും, അവരുടെ കാര്യത്തില്‍ അവരെ തന്നെ അവന്‍ സാക്ഷി നിര്‍ത്തുകയും ചെയ്ത സന്ദര്‍ഭം (ഓര്‍ക്കുക.) (അവന്‍ ചോദിച്ചു:) ഞാന്‍ നിങ്ങളുടെ രക്ഷിതാവല്ലയോ? അവര്‍ പറഞ്ഞു: അതെ, ഞങ്ങള്‍ സാക്ഷ്യം വാഹിച്ചിരിക്കുന്നു. തീര്‍ച്ചയായും ഞങ്ങള്‍ ഇതിനെപ്പറ്റി ശ്രദ്ധയില്ലാത്തവരായിരുന്നു. എന്ന് ഉയിര്‍ത്തെഴുന്നേല്‍പിന്‍റെ നാളില്‍ നിങ്ങള്‍ പറഞ്ഞേക്കും എന്നതിനാലാണ് (അങ്ങനെ ചെയ്തത്‌.)
\end{malayalam}}
\flushright{\begin{Arabic}
\quranayah[7][173]
\end{Arabic}}
\flushleft{\begin{malayalam}
അല്ലെങ്കില്‍ മുമ്പ് തന്നെ ഞങ്ങളുടെ പൂര്‍വ്വപിതാക്കള്‍ അല്ലാഹുവോട് പങ്കചേര്‍ത്തിരുന്നു. ഞങ്ങള്‍ അവര്‍ക്കു ശേഷം സന്തതിപരമ്പരകളായി വന്നവര്‍ മാത്രമാണ്‌. എന്നിരിക്കെ ആ അസത്യവാദികള്‍ പ്രവര്‍ത്തിച്ചതിന്‍റെ പേരില്‍ നീ ഞങ്ങളെ നശിപ്പിക്കുകയാണോ എന്ന് നിങ്ങള്‍ പറഞ്ഞേക്കും എന്നതിനാല്‍.
\end{malayalam}}
\flushright{\begin{Arabic}
\quranayah[7][174]
\end{Arabic}}
\flushleft{\begin{malayalam}
അപ്രകാരം നാം തെളിവുകള്‍ വിശദമായി വിവരിക്കുന്നു. അവര്‍ മടങ്ങിയേക്കാം.
\end{malayalam}}
\flushright{\begin{Arabic}
\quranayah[7][175]
\end{Arabic}}
\flushleft{\begin{malayalam}
നാം നമ്മുടെ ദൃഷ്ടാന്തങ്ങള്‍ നല്‍കിയിട്ട് അതില്‍ നിന്ന് ഊരിച്ചാടുകയും, അങ്ങനെ പിശാച് പിന്നാലെ കൂടുകയും, എന്നിട്ട് ദുര്‍മാര്‍ഗികളുടെ കൂട്ടത്തിലാവുകയും ചെയ്ത ഒരുവന്‍റെ വൃത്താന്തം നീ അവര്‍ക്ക് വായിച്ചുകേള്‍പിച്ചു കൊടുക്കുക.
\end{malayalam}}
\flushright{\begin{Arabic}
\quranayah[7][176]
\end{Arabic}}
\flushleft{\begin{malayalam}
നാം ഉദ്ദേശിച്ചിരുന്നുവെങ്കില്‍ അവ (ദൃഷ്ടാന്തങ്ങള്‍) മൂലം അവന്ന് ഉയര്‍ച്ച നല്‍കുമായിരുന്നു. പക്ഷെ, അവന്‍ ഭൂമിയലേക്ക് (അത് ശാശ്വതമാണെന്ന ഭാവേന) തിരിയുകയും അവന്‍റെ തന്നിഷ്ടത്തെ പിന്‍പറ്റുകയുമാണ് ചെയ്തത്‌. അപ്പോള്‍ അവന്‍റെ ഉപമ ഒരു നായയുടെത് പോലെയാകുന്നു. നീ അതിനെ ആക്രമിച്ചാല്‍ അത് നാവ് തൂക്കിയിടും. നീ അതിനെ വെറുതെ വിട്ടാലും അത് നാവ് തൂക്കിയിടും. അതാണ് നമ്മുടെ ദൃഷ്ടാന്തങ്ങള്‍ നിഷേധിച്ച് തള്ളിയവരുടെ ഉപമ. അതിനാല്‍ (അവര്‍ക്ക്‌) ഈ കഥ വിവരിച്ചുകൊടുക്കൂ. അവര്‍ ചിന്തിച്ചെന്ന് വരാം.
\end{malayalam}}
\flushright{\begin{Arabic}
\quranayah[7][177]
\end{Arabic}}
\flushleft{\begin{malayalam}
നമ്മുടെ ദൃഷ്ടാന്തങ്ങളെ നിഷേധിച്ച് തള്ളുകയും, സ്വദേഹങ്ങള്‍ക്ക് തന്നെ ദ്രോഹം വരുത്തിക്കൊണ്ടിരിക്കുകയും ചെയ്ത ആളുകളുടെ ഉപമ വളരെ ചീത്ത തന്നെ.
\end{malayalam}}
\flushright{\begin{Arabic}
\quranayah[7][178]
\end{Arabic}}
\flushleft{\begin{malayalam}
അല്ലാഹു ഏതൊരാളെ നേര്‍വഴിയിലാക്കുന്നുവോ അവനാണ് സന്‍മാര്‍ഗം പ്രാപിക്കുന്നവന്‍. അവന്‍ ആരെ പിഴവിലാക്കുന്നുവോ അവരാണ് നഷ്ടം പറ്റിയവര്‍.
\end{malayalam}}
\flushright{\begin{Arabic}
\quranayah[7][179]
\end{Arabic}}
\flushleft{\begin{malayalam}
ജിന്നുകളില്‍ നിന്നും മനുഷ്യരില്‍ നിന്നും ധാരാളം പേരെ നാം നരകത്തിന് വേണ്ടി സൃഷ്ടിച്ചിട്ടുണ്ട്‌. അവര്‍ക്ക് മനസ്സുകളുണ്ട്‌. അതുപയോഗിച്ച് അവര്‍ കാര്യം ഗ്രഹിക്കുകയില്ല. അവര്‍ക്കു കണ്ണുകളുണ്ട്‌. അതുപയോഗിച്ച് അവര്‍ കണ്ടറിയുകയില്ല. അവര്‍ക്ക് കാതുകളുണ്ട്‌. അതുപയോഗിച്ച് അവര്‍ കേട്ടു മനസ്സിലാക്കുകയില്ല. അവര്‍ കാലികളെപ്പോലെയാകുന്നു. അല്ല; അവരാണ് കൂടുതല്‍ പിഴച്ചവര്‍. അവര്‍ തന്നെയാണ് ശ്രദ്ധയില്ലാത്തവര്‍.
\end{malayalam}}
\flushright{\begin{Arabic}
\quranayah[7][180]
\end{Arabic}}
\flushleft{\begin{malayalam}
അല്ലാഹുവിന് ഏറ്റവും നല്ല പേരുകളുണ്ട്‌. അതിനാല്‍ ആ പേരുകളില്‍ അവനെ നിങ്ങള്‍ വിളിച്ചുകൊള്ളുക, അവന്‍റെ പേരുകളില്‍ കൃത്രിമം കാണിക്കുന്നവരെ നിങ്ങള്‍ വിട്ടുകളയുക. അവര്‍ ചെയ്തു വരുന്നതിന്‍റെ ഫലം അവര്‍ക്കു വഴിയെ നല്‍കപ്പെടും.
\end{malayalam}}
\flushright{\begin{Arabic}
\quranayah[7][181]
\end{Arabic}}
\flushleft{\begin{malayalam}
സത്യത്തിന്‍റെ അടിസ്ഥാനത്തില്‍ മാര്‍ഗദര്‍ശനം നല്‍കുകയും, അതനുസരിച്ച് തന്നെ നീതി നടത്തുകയും ചെയ്യുന്ന ഒരു സമൂഹം നാം സൃഷ്ടിച്ചവരുടെ കൂട്ടത്തിലുണ്ട്‌.
\end{malayalam}}
\flushright{\begin{Arabic}
\quranayah[7][182]
\end{Arabic}}
\flushleft{\begin{malayalam}
എന്നാല്‍ നമ്മുടെ ദൃഷ്ടാന്തങ്ങളെ നിഷേധിച്ച് കളഞ്ഞവരാകട്ടെ, അവരറിയാത്ത വിധത്തില്‍ അവരെ നാം പടിപടിയായി പിടികൂടുന്നതാണ്‌.
\end{malayalam}}
\flushright{\begin{Arabic}
\quranayah[7][183]
\end{Arabic}}
\flushleft{\begin{malayalam}
അവര്‍ക്കു ഞാന്‍ ഇടകൊടുക്കുകയും ചെയ്യും. തീര്‍ച്ചയായും എന്‍റെ തന്ത്രം സുശക്തമാണ്‌.
\end{malayalam}}
\flushright{\begin{Arabic}
\quranayah[7][184]
\end{Arabic}}
\flushleft{\begin{malayalam}
അവര്‍ ചിന്തിച്ച് നോക്കിയില്ലേ: അവരുടെ കൂട്ടുകാരന് (മുഹമ്മദ് നബിക്ക്‌) ഭ്രാന്തൊന്നുമില്ല. അദ്ദേഹം വ്യക്തമായി താക്കീത് നല്‍കിക്കൊണ്ടിരിക്കുന്ന ഒരാള്‍ മാത്രമാണ്‌.
\end{malayalam}}
\flushright{\begin{Arabic}
\quranayah[7][185]
\end{Arabic}}
\flushleft{\begin{malayalam}
ആകാശഭൂമികളുടെ ആധിപത്യരഹസ്യത്തെപ്പറ്റിയും, അല്ലാഹു സൃഷ്ടിച്ച ഏതൊരു വസ്തുവെപ്പറ്റിയും, അവരുടെ അവധി അടുത്തിട്ടുണ്ടായിരിക്കാം എന്നതിനെപ്പറ്റിയും അവര്‍ ചിന്തിച്ച് നോക്കിയില്ലേ? ഇനി ഇതിന് (ഖുര്‍ആന്ന്‌) ശേഷം ഏതൊരു വൃത്താന്തത്തിലാണ് അവര്‍ വിശ്വസിക്കാന്‍ പോകുന്നത്‌?
\end{malayalam}}
\flushright{\begin{Arabic}
\quranayah[7][186]
\end{Arabic}}
\flushleft{\begin{malayalam}
ഏതൊരുവനെ അല്ലാഹു പിഴവിലാക്കുന്നുവോ അവനെ നേര്‍വഴിയിലാക്കാന്‍ പിന്നെ ആരുമില്ല. അവരുടെ ധിക്കാരത്തില്‍ അന്ധമായി വിഹരിച്ചുകൊള്ളാന്‍ അല്ലാഹു അവരെ വിട്ടേക്കുന്നതുമാണ്‌.
\end{malayalam}}
\flushright{\begin{Arabic}
\quranayah[7][187]
\end{Arabic}}
\flushleft{\begin{malayalam}
അന്ത്യസമയത്തെപ്പറ്റി അവര്‍ നിന്നോട് ചോദിക്കുന്നു; അതെപ്പോഴാണ് വന്നെത്തുന്നതെന്ന്‌. പറയുക: അതിനെപ്പറ്റിയുള്ള അറിവ് എന്‍റെ രക്ഷിതാവിങ്കല്‍ മാത്രമാണ്‌. അതിന്‍റെ സമയത്ത് അത് വെളിപ്പെടുത്തുന്നത് അവന്‍ മാത്രമാകുന്നു. ആകാശങ്ങളിലും ഭൂമിയിലും അത് ഭാരിച്ചതായിരിക്കുന്നു. പെട്ടെന്നല്ലാതെ അത് നിങ്ങള്‍ക്കു വരുകയില്ല. നീ അതിനെപ്പറ്റി ചുഴിഞ്ഞന്വേഷിച്ചു മനസ്സിലാക്കിയവനാണെന്ന മട്ടില്‍ നിന്നോടവര്‍ ചോദിക്കുന്നു. പറയുക: അതിനെപ്പറ്റിയുള്ള അറിവ് അല്ലാഹുവിങ്കല്‍ മാത്രമാണ്‌. പക്ഷെ അധികമാളുകളും (കാര്യം) മനസ്സിലാക്കുന്നില്ല.
\end{malayalam}}
\flushright{\begin{Arabic}
\quranayah[7][188]
\end{Arabic}}
\flushleft{\begin{malayalam}
(നബിയേ,) പറയുക: എന്‍റെ സ്വന്തം ദേഹത്തിന് തന്നെ ഉപകാരമോ, ഉപദ്രവമോ വരുത്തല്‍ എന്‍റെ അധീനത്തില്‍ പെട്ടതല്ല. അല്ലാഹു ഉദ്ദേശിച്ചതൊഴികെ. എനിക്ക് അദൃശ്യകാര്യമറിയാമായിരുന്നുവെങ്കില്‍ ഞാന്‍ ധാരാളം ഗുണം നേടിയെടുക്കുമായിരുന്നു. തിന്‍മ എന്നെ ബാധിക്കുകയുമില്ലായിരുന്നു. ഞാനൊരു താക്കീതുകാരനും വിശ്വസിക്കുന്ന ജനങ്ങള്‍ക്ക് സന്തോഷമറിയിക്കുന്നവനും മാത്രമാണ്‌.
\end{malayalam}}
\flushright{\begin{Arabic}
\quranayah[7][189]
\end{Arabic}}
\flushleft{\begin{malayalam}
ഒരൊറ്റ സത്തയില്‍ നിന്ന് തന്നെ നിങ്ങളെ സൃഷ്ടിച്ചുണ്ടാക്കിയവനാണവന്‍. അതില്‍ നിന്ന് തന്നെ അതിന്‍റെ ഇണയേയും അവനുണ്ടാക്കി. അവളോടൊത്ത് അവന്‍ സമാധാനമടയുവാന്‍ വേണ്ടി. അങ്ങനെ അവന്‍ അവളെ പ്രാപിച്ചപ്പോള്‍ അവള്‍ ലഘുവായ ഒരു (ഗര്‍ഭ) ഭാരം വഹിച്ചു. എന്നിട്ട് അവളതുമായി നടന്നു. തുടര്‍ന്ന് അവള്‍ക്ക് ഭാരം കൂടിയപ്പോള്‍ അവര്‍ ഇരുവരും അവരുടെ രക്ഷിതാവായ അല്ലാഹുവോട് പ്രാര്‍ത്ഥിച്ചു. ഞങ്ങള്‍ക്കു നീ ഒരു നല്ല സന്താനത്തെ തരികയാണെങ്കില്‍ തീര്‍ച്ചയായും ഞങ്ങള്‍ നന്ദിയുള്ളവരുടെ കൂട്ടത്തിലായിരിക്കും.
\end{malayalam}}
\flushright{\begin{Arabic}
\quranayah[7][190]
\end{Arabic}}
\flushleft{\begin{malayalam}
അങ്ങനെ അവന്‍ (അല്ലാഹു) അവര്‍ക്കൊരു നല്ല സന്താനത്തെ നല്‍കിയപ്പോള്‍ അവര്‍ക്കവന്‍ നല്‍കിയതില്‍ അവര്‍ അവന്ന് പങ്കുകാരെ ഏര്‍പെടുത്തി. എന്നാല്‍ അവര്‍ പങ്കുചേര്‍ക്കുന്നതില്‍ നിന്നെല്ലാം അല്ലാഹു ഉന്നതനായിരിക്കുന്നു.
\end{malayalam}}
\flushright{\begin{Arabic}
\quranayah[7][191]
\end{Arabic}}
\flushleft{\begin{malayalam}
അവര്‍ പങ്കുചേര്‍ക്കുന്നത് യാതൊന്നും സൃഷ്ടിക്കാത്തവരെയാണോ? അവര്‍ (ആ ആരാധ്യര്‍) തന്നെ സൃഷ്ടിച്ചുണ്ടാക്കപ്പെടുന്നവരുമാണ്‌.
\end{malayalam}}
\flushright{\begin{Arabic}
\quranayah[7][192]
\end{Arabic}}
\flushleft{\begin{malayalam}
അവര്‍ക്കൊരു സഹായവും ചെയ്യാന്‍ അവര്‍ക്ക് (പങ്കാളികള്‍ക്കു) സാധിക്കുകയില്ല. സ്വദേഹങ്ങള്‍ക്കു തന്നെ അവര്‍ സഹായം ചെയ്യുന്നതുമല്ല.
\end{malayalam}}
\flushright{\begin{Arabic}
\quranayah[7][193]
\end{Arabic}}
\flushleft{\begin{malayalam}
നിങ്ങള്‍ അവരെ സന്‍മാര്‍ഗത്തിലേക്ക് ക്ഷണിച്ചാല്‍ അവര്‍ നിങ്ങളെ പിന്‍പറ്റുന്നതുമല്ല. നിങ്ങള്‍ അവരെ ക്ഷണിച്ചിരുന്നാലും, നിങ്ങള്‍ നിശ്ശബ്ദത പാലിച്ചിരുന്നാലും നിങ്ങള്‍ക്ക് സമമാണ്‌.
\end{malayalam}}
\flushright{\begin{Arabic}
\quranayah[7][194]
\end{Arabic}}
\flushleft{\begin{malayalam}
തീര്‍ച്ചയായും അല്ലാഹുവിന് പുറമെ നിങ്ങള്‍ വിളിച്ച് പ്രാര്‍ത്ഥിച്ചുകൊണ്ടിരിക്കുന്നവരെല്ലാം നിങ്ങളെപ്പോലെയുള്ള ദാസന്‍മാര്‍ മാത്രമാണ്‌. എന്നാല്‍ അവരെ നിങ്ങള്‍ വിളിച്ച് പ്രാര്‍ത്ഥിക്കൂ; അവര്‍ നിങ്ങള്‍ക്ക് ഉത്തരം നല്‍കട്ടെ; നിങ്ങള്‍ സത്യവാദികളാണെങ്കില്‍.
\end{malayalam}}
\flushright{\begin{Arabic}
\quranayah[7][195]
\end{Arabic}}
\flushleft{\begin{malayalam}
അവര്‍ക്ക് നടക്കാന്‍ കാലുകളുണ്ടോ? അവര്‍ക്ക് പിടിക്കാന്‍ കൈകളുണ്ടോ? അവര്‍ക്ക് കാണാന്‍ കണ്ണുകളുണ്ടോ? അവര്‍ക്ക് കേള്‍ക്കാന്‍ കാതുകളുണ്ടോ? (നബിയേ,) പറയുക: നിങ്ങള്‍ നിങ്ങളുടെ പങ്കാളികളെ വിളിച്ചിട്ട് എനിക്കെതിരായി തന്ത്രങ്ങള്‍ പ്രയോഗിച്ച് കൊള്ളുക. എനിക്ക് നിങ്ങള്‍ ഇടതരേണ്ടതില്ല.
\end{malayalam}}
\flushright{\begin{Arabic}
\quranayah[7][196]
\end{Arabic}}
\flushleft{\begin{malayalam}
തീര്‍ച്ചയായും ഈ ഗ്രന്ഥം അവതരിപ്പിച്ചവനായ അല്ലാഹുവാകുന്നു എന്‍റെ രക്ഷാധികാരി. അവനാണ് സജ്ജനങ്ങളുടെ സംരക്ഷണമേല്‍ക്കുന്നത്‌.
\end{malayalam}}
\flushright{\begin{Arabic}
\quranayah[7][197]
\end{Arabic}}
\flushleft{\begin{malayalam}
അവന്ന് പുറമെ നിങ്ങള്‍ വിളിച്ച് പ്രാര്‍ത്ഥിക്കുന്നവര്‍ക്കൊന്നും നിങ്ങളെ സഹായിക്കാന്‍ സാധിക്കുകയില്ല. സ്വദേഹങ്ങള്‍ക്ക് തന്നെയും അവര്‍ സഹായം ചെയ്യുകയില്ല.
\end{malayalam}}
\flushright{\begin{Arabic}
\quranayah[7][198]
\end{Arabic}}
\flushleft{\begin{malayalam}
നിങ്ങള്‍ അവരെ നേര്‍വഴിയിലേക്ക് ക്ഷണിക്കുന്ന പക്ഷം അവര്‍ കേള്‍ക്കുകയില്ല. അവര്‍ നിന്‍റെ നേരെ നോക്കുന്നതായി നിനക്ക് കാണാം. എന്നാല്‍ അവര്‍ കാണുന്നില്ല താനും.
\end{malayalam}}
\flushright{\begin{Arabic}
\quranayah[7][199]
\end{Arabic}}
\flushleft{\begin{malayalam}
നീ വിട്ടുവീഴ്ച സ്വീകരിക്കുകയും സദാചാരം കല്‍പിക്കുകയും, അവിവേകികളെ വിട്ട് തിരിഞ്ഞുകളയുകയും ചെയ്യുക.
\end{malayalam}}
\flushright{\begin{Arabic}
\quranayah[7][200]
\end{Arabic}}
\flushleft{\begin{malayalam}
പിശാചില്‍ നിന്നുള്ള വല്ല ദുഷ്പ്രേരണയും നിന്നെ ബാധിക്കുകയാണെങ്കില്‍ നീ അല്ലാഹുവോട് ശരണം തേടിക്കൊള്ളുക. തീര്‍ച്ചയായും അവന്‍ എല്ലാം കേള്‍ക്കുന്നവനും അറിയുന്നവനുമാണ്‌.
\end{malayalam}}
\flushright{\begin{Arabic}
\quranayah[7][201]
\end{Arabic}}
\flushleft{\begin{malayalam}
തീര്‍ച്ചയായും സൂക്ഷ്മത പാലിക്കുന്നവരെ പിശാചില്‍ നിന്നുള്ള വല്ല ദുര്‍ബോധനവും ബാധിച്ചാല്‍ അവര്‍ക്ക് (അല്ലാഹുവെപ്പറ്റി) ഓര്‍മവരുന്നതാണ്‌. അപ്പോഴതാ അവര്‍ ഉള്‍കാഴ്ചയുള്ളവരാകുന്നു.
\end{malayalam}}
\flushright{\begin{Arabic}
\quranayah[7][202]
\end{Arabic}}
\flushleft{\begin{malayalam}
എന്നാല്‍ അവരുടെ (പിശാചുക്കളുടെ) സഹോദരങ്ങളെയാവട്ടെ, അവര്‍ ദുര്‍മാര്‍ഗത്തില്‍ അയച്ചുവിട്ടുകൊണ്ടിരിക്കുന്നു. പിന്നെ അവര്‍ (അധര്‍മ്മത്തില്‍) ഒട്ടും കമ്മിവരുത്തുകയില്ല.
\end{malayalam}}
\flushright{\begin{Arabic}
\quranayah[7][203]
\end{Arabic}}
\flushleft{\begin{malayalam}
നീ അവര്‍ക്ക് ഏതെങ്കിലും ദൃഷ്ടാന്തം കൊണ്ട് വന്ന് കൊടുത്തില്ലെങ്കില്‍ അവര്‍ പറയും: നിനക്ക് തന്നെ അത് സ്വയം കെട്ടിച്ചമച്ചുണ്ടാക്കിക്കൂടേ? (നബിയേ,) പറയുക: എന്‍റെ രക്ഷിതാവിങ്കല്‍ നിന്ന് ബോധനം നല്‍കപ്പെടുന്നതിനെ പിന്‍പറ്റുക മാത്രമാണ് ഞാന്‍ ചെയ്യുന്നത്‌. നിങ്ങളുടെ രക്ഷിതാവിങ്കല്‍ നിന്നുള്ള കണ്ണുതുറപ്പിക്കുന്ന തെളിവുകളും വിശ്വസിക്കുന്ന ജനങ്ങള്‍ക്ക് മാര്‍ഗദര്‍ശനവും കാരുണ്യവുമാണ് ഇത് (ഖുര്‍ആന്‍.)
\end{malayalam}}
\flushright{\begin{Arabic}
\quranayah[7][204]
\end{Arabic}}
\flushleft{\begin{malayalam}
ഖുര്‍ആന്‍ പാരായണം ചെയ്യപ്പെട്ടാല്‍ നിങ്ങളത് ശ്രദ്ധിച്ച് കേള്‍ക്കുകയും നിശ്ശബ്ദത പാലിക്കുകയും ചെയ്യുക. നിങ്ങള്‍ക്ക് കാരുണ്യം ലഭിച്ചേക്കാം.
\end{malayalam}}
\flushright{\begin{Arabic}
\quranayah[7][205]
\end{Arabic}}
\flushleft{\begin{malayalam}
വിനയത്തോടും ഭയപ്പാടോടും കൂടി, വാക്ക് ഉച്ചത്തിലാകാതെ രാവിലെയും വൈകുന്നേരവും നീ നിന്‍റെ രക്ഷിതാവിനെ മനസ്സില്‍ സ്മരിക്കുക. നീ ശ്രദ്ധയില്ലാത്തവരുടെ കൂട്ടത്തിലാകരുത്‌.
\end{malayalam}}
\flushright{\begin{Arabic}
\quranayah[7][206]
\end{Arabic}}
\flushleft{\begin{malayalam}
തീര്‍ച്ചയായും നിന്‍റെ രക്ഷിതാവിന്‍റെ അടുക്കലുള്ളവര്‍ (മലക്കുകള്‍) അവനെ ആരാധിക്കുന്നതിനെപ്പറ്റി അഹംഭാവം നടിക്കുകയില്ല. അവര്‍ അവന്‍റെ മഹത്വം പ്രകീര്‍ത്തിക്കുകയും അവനെ പ്രണമിക്കുകയും ചെയ്തുകൊണ്ടിരിക്കുന്നു.
\end{malayalam}}
\chapter{\textmalayalam{അന്‍ഫാല്‍ ( യുദ്ധമുതല്‍‍ )}}
\begin{Arabic}
\Huge{\centerline{\basmalah}}\end{Arabic}
\flushright{\begin{Arabic}
\quranayah[8][1]
\end{Arabic}}
\flushleft{\begin{malayalam}
(നബിയേ,) നിന്നോടവര്‍ യുദ്ധത്തില്‍ നേടിയ സ്വത്തുക്കളെപ്പറ്റി ചോദിക്കുന്നു. പറയുക: യുദ്ധത്തില്‍ നേടിയ സ്വത്തുക്കള്‍ അല്ലാഹുവിനും അവന്‍റെ റസൂലിനുമുള്ളതാകുന്നു. അതിനാല്‍ നിങ്ങള്‍ അല്ലാഹുവെ സൂക്ഷിക്കുകയും നിങ്ങള്‍ തമ്മിലുള്ള ബന്ധങ്ങള്‍ നന്നാക്കിത്തീര്‍ക്കുകയും ചെയ്യുക. നിങ്ങള്‍ വിശ്വാസികളാണെങ്കില്‍ അല്ലാഹുവെയും റസൂലിനെയും നിങ്ങള്‍ അനുസരിക്കുകയും ചെയ്യുക.
\end{malayalam}}
\flushright{\begin{Arabic}
\quranayah[8][2]
\end{Arabic}}
\flushleft{\begin{malayalam}
അല്ലാഹുവെപ്പറ്റി പറയപ്പെട്ടാല്‍ ഹൃദയങ്ങള്‍ പേടിച്ച് നടുങ്ങുകയും, അവന്‍റെ ദൃഷ്ടാന്തങ്ങള്‍ വായിച്ചുകേള്‍പിക്കപ്പെട്ടാല്‍ വിശ്വാസം വര്‍ദ്ധിക്കുകയും, തങ്ങളുടെ രക്ഷിതാവിന്‍റെ മേല്‍ ഭരമേല്‍പിക്കുകയും ചെയ്യുന്നവര്‍ മാത്രമാണ് സത്യവിശ്വാസികള്‍.
\end{malayalam}}
\flushright{\begin{Arabic}
\quranayah[8][3]
\end{Arabic}}
\flushleft{\begin{malayalam}
നമസ്കാരം മുറപോലെ നിര്‍വഹിക്കുകയും നാം നല്‍കിയിട്ടുള്ളതില്‍ നിന്ന് ചെലവഴിക്കുകയും ചെയ്യുന്നവര്‍.
\end{malayalam}}
\flushright{\begin{Arabic}
\quranayah[8][4]
\end{Arabic}}
\flushleft{\begin{malayalam}
അവര്‍ തന്നെയാണ് യഥാര്‍ത്ഥത്തില്‍ വിശ്വാസികള്‍. അവര്‍ക്ക് അവരുടെ രക്ഷിതാവിങ്കല്‍ പല പദവികളുണ്ട്‌. പാപമോചനവും ഉദാരമായ ഉപജീവനവുമുണ്ട്‌.
\end{malayalam}}
\flushright{\begin{Arabic}
\quranayah[8][5]
\end{Arabic}}
\flushleft{\begin{malayalam}
വിശ്വാസികളില്‍ ഒരു വിഭാഗം ഇഷ്ടമില്ലാത്തവരായിരിക്കെ ത്തന്നെ നിന്‍റെ വീട്ടില്‍ നിന്ന് ന്യായമായ കാര്യത്തിന് നിന്‍റെ രക്ഷിതാവ് നിന്നെ പുറത്തിറക്കിയത് പോലെത്തന്നെയാണിത്‌.
\end{malayalam}}
\flushright{\begin{Arabic}
\quranayah[8][6]
\end{Arabic}}
\flushleft{\begin{malayalam}
ന്യായമായ കാര്യത്തില്‍, അതു വ്യക്തമായതിനു ശേഷം അവര്‍ നിന്നോട് തര്‍ക്കിക്കുകയായിരുന്നു. അവര്‍ നോക്കിക്കൊണ്ടിരിക്കെ മരണത്തിലേക്ക് അവര്‍ നയിക്കപ്പെടുന്നത് പോലെ.
\end{malayalam}}
\flushright{\begin{Arabic}
\quranayah[8][7]
\end{Arabic}}
\flushleft{\begin{malayalam}
രണ്ടു സംഘങ്ങളിലൊന്ന് നിങ്ങള്‍ക്ക് അധീനമാകുമെന്ന് അല്ലാഹു നിങ്ങളോട് വാഗ്ദാനം ചെയ്തിരുന്ന സന്ദര്‍ഭം (ഓര്‍ക്കുക.) ആയുധബലമില്ലാത്ത സംഘം നിങ്ങള്‍ക്കധീനമാകണമെന്നായിരുന്നു നിങ്ങള്‍ കൊതിച്ചിരുന്നത്‌. അല്ലാഹുവാകട്ടെ തന്‍റെ കല്‍പനകള്‍ മുഖേന സത്യം പുലര്‍ത്തിക്കാണിക്കുവാനും സത്യനിഷേധികളുടെ മുരട് മുറിച്ചുകളയുവാനും ആണ് ഉദ്ദേശിച്ചിരുന്നത്‌.
\end{malayalam}}
\flushright{\begin{Arabic}
\quranayah[8][8]
\end{Arabic}}
\flushleft{\begin{malayalam}
സത്യത്തെ സത്യമായി പുലര്‍ത്തേണ്ടതിനും അസത്യത്തെ ഫലശൂന്യമാക്കിത്തീര്‍ക്കേണ്ടതിനുമത്രെ അത്‌. ദുഷ്ടന്‍മാര്‍ക്ക് അതെത്ര അനിഷ്ടകരമായാലും ശരി.
\end{malayalam}}
\flushright{\begin{Arabic}
\quranayah[8][9]
\end{Arabic}}
\flushleft{\begin{malayalam}
നിങ്ങള്‍ നിങ്ങളുടെ രക്ഷിതാവിനോട് സഹായം തേടിയിരുന്ന സന്ദര്‍ഭം (ഓര്‍ക്കുക.) തുടരെത്തുടരെയായി ആയിരം മലക്കുകളെ അയച്ചുകൊണ്ട് ഞാന്‍ നിങ്ങള്‍ക്ക് സഹായം നല്‍കുന്നതാണ് എന്ന് അവന്‍ അപ്പോള്‍ നിങ്ങള്‍ക്കു മറുപടി നല്‍കി.
\end{malayalam}}
\flushright{\begin{Arabic}
\quranayah[8][10]
\end{Arabic}}
\flushleft{\begin{malayalam}
ഒരു സന്തോഷവാര്‍ത്തയായിക്കൊണ്ടും നിങ്ങളുടെ ഹൃദയങ്ങള്‍ക്കു സമാധാനം നല്‍കുന്നതിന് വേണ്ടിയും മാത്രമാണ് അല്ലാഹു അത് ഏര്‍പെടുത്തിയത്‌. അല്ലാഹുവിങ്കല്‍ നിന്നല്ലാതെ യാതൊരു സഹായവും ഇല്ല. തീര്‍ച്ചയായും അല്ലാഹു പ്രതാപിയും യുക്തിമാനുമാകുന്നു.
\end{malayalam}}
\flushright{\begin{Arabic}
\quranayah[8][11]
\end{Arabic}}
\flushleft{\begin{malayalam}
അല്ലാഹു തന്‍റെ പക്കല്‍ നിന്നുള്ള മനഃശാന്തിയുമായി നിങ്ങളെ നിദ്രാമയക്കം കൊണ്ട് ആവരണം ചെയ്തിരുന്ന സന്ദര്‍ഭം (ഓര്‍ക്കുക.) നിങ്ങളെ ശുദ്ധീകരിക്കുന്നതിനും, നിങ്ങളില്‍ നിന്ന് പിശാചിന്‍റെ ദുര്‍ബോധനം നീക്കികളയുന്നതിനും, നിങ്ങളുടെ മനസ്സുകള്‍ക്ക് കെട്ടുറപ്പ് നല്‍കുന്നതിനും, പാദങ്ങള്‍ ഉറപ്പിച്ചു നിര്‍ത്തുന്നതിനും വേണ്ടി അവന്‍ നിങ്ങളുടെ മേല്‍ ആകാശത്തു നിന്ന് വെള്ളം ചൊരിഞ്ഞുതന്നിരുന്ന സന്ദര്‍ഭവും (ഓര്‍ക്കുക.)
\end{malayalam}}
\flushright{\begin{Arabic}
\quranayah[8][12]
\end{Arabic}}
\flushleft{\begin{malayalam}
നിന്‍റെ രക്ഷിതാവ് മലക്കുകള്‍ക്ക് ബോധനം നല്‍കിയിരുന്ന സന്ദര്‍ഭം (ഓര്‍ക്കുക.) ഞാന്‍ നിങ്ങളുടെ കൂടെയുണ്ട്‌. അതിനാല്‍ സത്യവിശ്വാസികളെ നിങ്ങള്‍ ഉറപ്പിച്ചു നിര്‍ത്തുക. സത്യനിഷേധികളുടെ മനസ്സുകളില്‍ ഞാന്‍ ഭയം ഇട്ടുകൊടുക്കുന്നതാണ്‌. അതിനാല്‍ കഴുത്തുകള്‍ക്ക് മീതെ നിങ്ങള്‍ വെട്ടിക്കൊള്ളുക. അവരുടെ വിരലുകളെല്ലാം നിങ്ങള്‍ വെട്ടിക്കളയുകയും ചെയ്യുക.
\end{malayalam}}
\flushright{\begin{Arabic}
\quranayah[8][13]
\end{Arabic}}
\flushleft{\begin{malayalam}
അവര്‍ അല്ലാഹുവോടും അവന്‍റെ ദൂതനോടും എതിര്‍ത്തു നിന്നതിന്‍റെ ഫലമത്രെ അത്‌. വല്ലവനും അല്ലാഹുവെയും അവന്‍റെ ദൂതനെയും എതിര്‍ക്കുന്ന പക്ഷം തീര്‍ച്ചയായും അല്ലാഹു കഠിനമായി ശിക്ഷിക്കുന്നവനാണ്‌.
\end{malayalam}}
\flushright{\begin{Arabic}
\quranayah[8][14]
\end{Arabic}}
\flushleft{\begin{malayalam}
അതാ അതു നിങ്ങള്‍ ആസ്വദിച്ചുകൊള്ളുക. സത്യനിഷേധികള്‍ക്ക് തന്നെയാണ് നരകശിക്ഷ എന്ന് (മനസ്സിലാക്കുകയും ചെയ്യുക.)
\end{malayalam}}
\flushright{\begin{Arabic}
\quranayah[8][15]
\end{Arabic}}
\flushleft{\begin{malayalam}
സത്യവിശ്വാസികളേ, സത്യനിഷേധികള്‍ പടയണിയായി വരുന്നതു നിങ്ങള്‍ കണ്ടാല്‍ നിങ്ങള്‍ അവരില്‍ നിന്ന് പിന്തിരിഞ്ഞ് ഓടരുത്‌.
\end{malayalam}}
\flushright{\begin{Arabic}
\quranayah[8][16]
\end{Arabic}}
\flushleft{\begin{malayalam}
യുദ്ധ (തന്ത്ര) ത്തിനായി സ്ഥാനം മാറുന്നതിനോ (സ്വന്തം) സംഘത്തോടൊപ്പം ചേരുന്നതിനോ അല്ലാതെ അന്ന് അവരില്‍ നിന്നു (ശത്രുക്കളുടെ മുമ്പില്‍ നിന്ന്‌) വല്ലവനും പിന്തിരിഞ്ഞ് കളയുന്ന പക്ഷം അവന്‍ അല്ലാഹുവില്‍നിന്നുള്ള കോപത്തിനു പാത്രമായിരിക്കുന്നതും അവന്‍റെ സങ്കേതം നരകമായിരിക്കുന്നതുമാണ്‌. ചെന്നുചേരാന്‍ കൊള്ളരുതാത്ത സ്ഥലമത്രെ അത്‌.
\end{malayalam}}
\flushright{\begin{Arabic}
\quranayah[8][17]
\end{Arabic}}
\flushleft{\begin{malayalam}
എന്നാല്‍ നിങ്ങള്‍ അവരെ കൊലപ്പെടുത്തിയിട്ടില്ല. പക്ഷെ അല്ലാഹുവാണ് അവരെ കൊലപ്പെടുത്തിയത്‌. (നബിയേ,) നീ എറിഞ്ഞ സമയത്ത് നീ എറിഞ്ഞിട്ടുമില്ല. പക്ഷെ അല്ലാഹുവാണ് എറിഞ്ഞത്‌. തന്‍റെ ഭാഗത്തു നിന്നുള്ള ഗുണകരമായ ഒരു പരീക്ഷണത്തിലൂടെ അല്ലാഹു സത്യവിശ്വാസികളെ പരീക്ഷിക്കുന്നതിനു വേണ്ടിയായിരുന്നു അത്‌. തീര്‍ച്ചയായും അല്ലാഹു എല്ലാം കേള്‍ക്കുന്നവനും അറിയുന്നവനുമാണ്‌.
\end{malayalam}}
\flushright{\begin{Arabic}
\quranayah[8][18]
\end{Arabic}}
\flushleft{\begin{malayalam}
അതാണ് (കാര്യം) സത്യനിഷേധികളുടെ തന്ത്രത്തെ അല്ലാഹു ബലഹീനമാക്കുക തന്നെ ചെയ്യുന്നതുമാണ്‌.
\end{malayalam}}
\flushright{\begin{Arabic}
\quranayah[8][19]
\end{Arabic}}
\flushleft{\begin{malayalam}
(സത്യനിഷേധികളേ,) നിങ്ങള്‍ വിജയമായിരുന്നു തേടിയിരുന്നതെങ്കില്‍ ആ വിജയമിതാ നിങ്ങള്‍ക്കു വന്നു കഴിഞ്ഞിരിക്കുന്നു. നിങ്ങള്‍ വിരമിക്കുകയാണെങ്കില്‍ അതാണ് നിങ്ങള്‍ക്ക് ഉത്തമം. നിങ്ങള്‍ ആവര്‍ത്തിക്കുകയാണെങ്കിലോ നാമും ആവര്‍ത്തിക്കുന്നതാണ്‌. നിങ്ങളുടെ സംഘം എത്ര എണ്ണക്കൂടുതലുള്ളതാണെങ്കിലും അത് നിങ്ങള്‍ക്ക് ഉപകരിക്കുകയേയില്ല. അല്ലാഹു സത്യവിശ്വാസികളുടെ കൂടെത്തന്നെയാണ്‌.
\end{malayalam}}
\flushright{\begin{Arabic}
\quranayah[8][20]
\end{Arabic}}
\flushleft{\begin{malayalam}
സത്യവിശ്വാസികളേ, നിങ്ങള്‍ അല്ലാഹുവെയും അവന്‍റെ റസൂലിനെയും അനുസരിക്കുക. (സത്യസന്ദേശം) കേട്ടുകൊണ്ടിരിക്കെ നിങ്ങള്‍ അദ്ദേഹത്തെ വിട്ട് തിരിഞ്ഞുകളയരുത്‌.
\end{malayalam}}
\flushright{\begin{Arabic}
\quranayah[8][21]
\end{Arabic}}
\flushleft{\begin{malayalam}
ഞങ്ങള്‍ കേട്ടിരിക്കുന്നു എന്ന് പറയുകയും യാതൊന്നും കേള്‍ക്കാതിരിക്കുകയും ചെയ്തവരെപോലെ നിങ്ങളാകരുത്‌.
\end{malayalam}}
\flushright{\begin{Arabic}
\quranayah[8][22]
\end{Arabic}}
\flushleft{\begin{malayalam}
തീര്‍ച്ചയായും ജന്തുക്കളുടെ കൂട്ടത്തില്‍ അല്ലാഹുവിന്‍റെ അടുക്കല്‍ ഏറ്റവും മോശമായവര്‍ ചിന്തിച്ചു മനസ്സിലാക്കാത്ത ഊമകളും ബധിരന്‍മാരുമാകുന്നു.
\end{malayalam}}
\flushright{\begin{Arabic}
\quranayah[8][23]
\end{Arabic}}
\flushleft{\begin{malayalam}
അവരില്‍ വല്ല നന്മയുമുള്ളതായി അല്ലാഹു അറിഞ്ഞിരുന്നുവെങ്കില്‍ അവരെ അവന്‍ കേള്‍പ്പിക്കുക തന്നെ ചെയ്യുമായിരുന്നു. അവരെ അവന്‍ കേള്‍പിച്ചിരുന്നെങ്കില്‍ തന്നെ അവര്‍ അവഗണിച്ചുകൊന്നു് തിരിഞ്ഞു കളയുമായിരുന്നു
\end{malayalam}}
\flushright{\begin{Arabic}
\quranayah[8][24]
\end{Arabic}}
\flushleft{\begin{malayalam}
നിങ്ങള്‍ക്ക് ജീവന്‍ നല്‍കുന്ന കാര്യത്തിലേക്ക് നിങ്ങളെ വിളിക്കുമ്പോള്‍ സത്യവിശ്വാസികളേ, നിങ്ങള്‍ അല്ലാഹുവിനും റസൂലിനും ഉത്തരം നല്‍കുക. മനുഷ്യന്നും അവന്‍റെ മനസ്സിനും ഇടയില്‍ അല്ലാഹു മറയിടുന്നതാണ് എന്നും അവങ്കലേക്ക് നിങ്ങള്‍ ഒരുമിച്ചുകൂട്ടപ്പെടുമെന്നും നിങ്ങള്‍ അറിഞ്ഞ് കൊള്ളുക.
\end{malayalam}}
\flushright{\begin{Arabic}
\quranayah[8][25]
\end{Arabic}}
\flushleft{\begin{malayalam}
ഒരു പരീക്ഷണം (ശിക്ഷ) വരുന്നത് നിങ്ങള്‍ സൂക്ഷിച്ചു കൊള്ളുക. അത് ബാധിക്കുന്നത് നിങ്ങളില്‍ നിന്നുള്ള അക്രമികള്‍ക്ക് പ്രത്യേകമായിട്ടാവുകയില്ല. അല്ലാഹു കഠിനമായി ശിക്ഷിക്കുന്നവനാണെന്ന് നിങ്ങള്‍ മനസ്സിലാക്കുകയും ചെയ്യുക.
\end{malayalam}}
\flushright{\begin{Arabic}
\quranayah[8][26]
\end{Arabic}}
\flushleft{\begin{malayalam}
നിങ്ങള്‍ ഭൂമിയില്‍ ബലഹീനരായി ഗണിക്കപ്പെട്ടിരുന്ന കുറച്ച് പേര്‍ മാത്രമായിരുന്ന സന്ദര്‍ഭം നിങ്ങള്‍ ഓര്‍ക്കുക. ജനങ്ങള്‍ നിങ്ങളെ റാഞ്ചിയെടുത്ത് കളയുമെന്ന് നിങ്ങള്‍ ഭയപ്പെട്ടിരുന്നു. എന്നിട്ട് അവന്‍ നിങ്ങള്‍ക്ക് ആശ്രയം നല്‍കുകയും അവന്‍റെ സഹായം കൊണ്ട് നിങ്ങള്‍ക്ക് പിന്‍ബലം നല്‍കുകയും വിശിഷ്ട വസ്തുക്കളില്‍ നിന്ന് നിങ്ങള്‍ക്ക് ഉപജീവനം നല്‍കുകയും ചെയ്തു. നിങ്ങള്‍ നന്ദിയുള്ളവരാകാന്‍ വേണ്ടി.
\end{malayalam}}
\flushright{\begin{Arabic}
\quranayah[8][27]
\end{Arabic}}
\flushleft{\begin{malayalam}
സത്യവിശ്വാസികളേ, നിങ്ങള്‍ അല്ലാഹുവോടും റസൂലിനോടും വഞ്ചന കാണിക്കരുത്‌. നിങ്ങള്‍ വിശ്വസിച്ചേല്‍പിക്കപ്പെട്ട കാര്യങ്ങളില്‍ അറിഞ്ഞ് കൊണ്ട് വഞ്ചന കാണിക്കുകയും ചെയ്യരുത്‌.
\end{malayalam}}
\flushright{\begin{Arabic}
\quranayah[8][28]
\end{Arabic}}
\flushleft{\begin{malayalam}
നിങ്ങളുടെ സ്വത്തുക്കളും നിങ്ങളുടെ സന്താനങ്ങളും ഒരു പരീക്ഷണമാണെന്നും അല്ലാഹുവിങ്കലാണ് മഹത്തായ പ്രതിഫലമുള്ളതെന്നും നിങ്ങള്‍ മനസ്സിലാക്കുകയും ചെയ്യുക.
\end{malayalam}}
\flushright{\begin{Arabic}
\quranayah[8][29]
\end{Arabic}}
\flushleft{\begin{malayalam}
സത്യവിശ്വാസികളേ, നിങ്ങള്‍ അല്ലാഹുവെ സൂക്ഷിച്ച് ജീവിക്കുകയാണെങ്കില്‍ നിങ്ങള്‍ക്ക് സത്യവും അസത്യവും വിവേചിക്കുവാനുള്ള കഴിവ് അവനുണ്ടാക്കിത്തരികയും, അവന്‍ നിങ്ങളുടെ തിന്‍മകള്‍ മായ്ച്ചുകളയുകയും, നിങ്ങള്‍ക്ക് പൊറുത്തുതരികയും ചെയ്യുന്നതാണ്‌. അല്ലാഹു മഹത്തായ അനുഗ്രഹമുള്ളവനാകുന്നു.
\end{malayalam}}
\flushright{\begin{Arabic}
\quranayah[8][30]
\end{Arabic}}
\flushleft{\begin{malayalam}
നിന്നെ ബന്ധനസ്ഥനാക്കുകയോ കൊല്ലുകയോ നാട്ടില്‍ നിന്ന് പുറത്താക്കുകയോ ചെയ്യാന്‍ വേണ്ടി നിനക്കെതിരായി സത്യനിഷേധികള്‍ തന്ത്രം പ്രയോഗിച്ചിരുന്ന സന്ദര്‍ഭം (ഓര്‍ക്കുക.) അവര്‍ തന്ത്രം പ്രയോഗിക്കുന്നു. അല്ലാഹുവും തന്ത്രം പ്രയോഗിക്കുന്നു. എന്നാല്‍ അല്ലാഹുവാണ് തന്ത്രം പ്രയോഗിക്കുന്നവരില്‍ മെച്ചപ്പെട്ടവന്‍.
\end{malayalam}}
\flushright{\begin{Arabic}
\quranayah[8][31]
\end{Arabic}}
\flushleft{\begin{malayalam}
നമ്മുടെ വചനങ്ങള്‍ അവര്‍ക്ക് ഓതികേള്‍പിക്കപ്പെടുമ്പോള്‍ അവര്‍ പറയും: ഞങ്ങള്‍ കേട്ടിരിക്കുന്നു. ഞങ്ങള്‍ വിചാരിച്ചിരുന്നെങ്കില്‍ ഇതു (ഖുര്‍ആന്‍) പോലെ ഞങ്ങളും പറയുമായിരുന്നു. ഇത് പൂര്‍വ്വികന്‍മാരുടെ പഴങ്കഥകളല്ലാതെ മറ്റൊന്നുമല്ല.
\end{malayalam}}
\flushright{\begin{Arabic}
\quranayah[8][32]
\end{Arabic}}
\flushleft{\begin{malayalam}
അല്ലാഹുവേ, ഇതു നിന്‍റെ പക്കല്‍ നിന്നുള്ള സത്യമാണെങ്കില്‍ നീ ഞങ്ങളുടെ മേല്‍ ആകാശത്ത് നിന്ന് കല്ല് വര്‍ഷിപ്പിക്കുകയോ, അല്ലെങ്കില്‍ ഞങ്ങള്‍ക്ക് വേദനാജനകമായ ശിക്ഷ കൊണ്ടുവരികയോ ചെയ്യുക എന്ന് അവര്‍ (അവിശ്വാസികള്‍) പറഞ്ഞ സന്ദര്‍ഭവും (ഓര്‍ക്കുക.)
\end{malayalam}}
\flushright{\begin{Arabic}
\quranayah[8][33]
\end{Arabic}}
\flushleft{\begin{malayalam}
എന്നാല്‍ നീ അവര്‍ക്കിടയില്‍ ഉണ്ടായിരിക്കെ അല്ലാഹു അവരെ ശിക്ഷിക്കുന്നതല്ല. അവര്‍ പാപമോചനം തേണ്ടിക്കൊണ്ടിരിക്കുമ്പോഴും അല്ലാഹു അവരെ ശിക്ഷിക്കുന്നതല്ല.
\end{malayalam}}
\flushright{\begin{Arabic}
\quranayah[8][34]
\end{Arabic}}
\flushleft{\begin{malayalam}
അല്ലാഹു അവരെ ശിക്ഷിക്കാതിരിക്കാന്‍ എന്ത് അര്‍ഹതയാണുള്ളത്‌? അവരാകട്ടെ മസ്ജിദുല്‍ ഹറാമില്‍ നിന്ന് ആളുകളെ തടഞ്ഞുകൊണ്ടിരിക്കുന്നു. അവരാണെങ്കില്‍ അതിന്‍റെ രക്ഷാധികാരികളല്ലതാനും. ഭയഭക്തിയുള്ളവരല്ലാതെ അതിന്‍റെ രക്ഷാധികാരികളാകാവുന്നതല്ല. പക്ഷെ അവരില്‍ അധികപേരും (കാര്യം) മനസ്സിലാക്കുന്നില്ല.
\end{malayalam}}
\flushright{\begin{Arabic}
\quranayah[8][35]
\end{Arabic}}
\flushleft{\begin{malayalam}
ആ ഭവനത്തിന്‍റെ (കഅ്ബയുടെ) അടുക്കല്‍ അവര്‍ നടത്തുന്ന പ്രാര്‍ത്ഥന ചൂളംവിളിയും കൈകൊട്ടലുമല്ലാതെ മറ്റൊന്നുമായിരുന്നില്ല. അതിനാല്‍ നിങ്ങള്‍ സത്യനിഷേധം കൈക്കൊണ്ടിരുന്നത് നിമിത്തം ശിക്ഷ ആസ്വദിച്ച് കൊള്ളുക.
\end{malayalam}}
\flushright{\begin{Arabic}
\quranayah[8][36]
\end{Arabic}}
\flushleft{\begin{malayalam}
തീര്‍ച്ചയായും സത്യനിഷേധികള്‍ തങ്ങളുടെ സ്വത്തുക്കള്‍ ചെലവഴിക്കുന്നത് അല്ലാഹുവിന്‍റെ മാര്‍ഗത്തില്‍ നിന്ന് (ജനങ്ങളെ) പിന്തിരിപ്പിക്കുവാന്‍ വേണ്ടിയത്രെ. അവര്‍ അത് ചെലവഴിക്കും. പിന്നീട് അതവര്‍ക്ക് ഖേദത്തിന് കാരണമായിത്തീരും. അനന്തരം അവര്‍ കീഴടക്കപ്പെടുകയും ചെയ്യും. സത്യനിഷേധികള്‍ നരകത്തിലേക്ക് വിളിച്ചുകൂട്ടപ്പെടുന്നതാണ്‌.
\end{malayalam}}
\flushright{\begin{Arabic}
\quranayah[8][37]
\end{Arabic}}
\flushleft{\begin{malayalam}
അല്ലാഹു നല്ലതില്‍ നിന്ന് ചീത്തയെ വേര്‍തിരിക്കാനും ചീത്തയെ ഒന്നിനുമേല്‍ മറ്റൊന്നായി ഒന്നിച്ചു കൂമ്പാരമാക്കി നരകത്തിലിടാനും വേണ്ടിയത്രെ അത്‌. അക്കൂട്ടര്‍ തന്നെയാണ് നഷ്ടം പറ്റിയവര്‍.
\end{malayalam}}
\flushright{\begin{Arabic}
\quranayah[8][38]
\end{Arabic}}
\flushleft{\begin{malayalam}
സത്യനിഷേധികളോട്‌, അവര്‍ വിരമിക്കുകയാണെങ്കില്‍ അവര്‍ മുമ്പ് ചെയ്തുപോയിട്ടുള്ളത് അവര്‍ക്കു പൊറുത്തുകൊടുക്കപ്പെടുന്നതാണ് എന്ന് നീ പറയുക. ഇനി അവര്‍ (നിഷേധത്തിലേക്കു തന്നെ) മടങ്ങുകയാണെങ്കിലോ, പൂര്‍വ്വികന്‍മാരുടെ കാര്യത്തില്‍ (അല്ലാഹുവിന്‍റെ) നടപടി കഴിഞ്ഞുപോയിട്ടുണ്ടല്ലോ.
\end{malayalam}}
\flushright{\begin{Arabic}
\quranayah[8][39]
\end{Arabic}}
\flushleft{\begin{malayalam}
കുഴപ്പം ഇല്ലാതാവുകയും മതം മുഴുവന്‍ അല്ലാഹുവിന് വേണ്ടിയാകുകയും ചെയ്യുന്നത് വരെ. നിങ്ങള്‍ അവരോട് യുദ്ധം ചെയ്യുക. ഇനി, അവര്‍ വിരമിക്കുന്ന പക്ഷം അവര്‍ പ്രവര്‍ത്തിക്കുന്നതെല്ലാം അല്ലാഹു കണ്ടറിയുന്നവനാണ്‌.
\end{malayalam}}
\flushright{\begin{Arabic}
\quranayah[8][40]
\end{Arabic}}
\flushleft{\begin{malayalam}
എന്നാല്‍ അവര്‍ പിന്തിരിഞ്ഞ് കളയുകയാണെങ്കില്‍ അല്ലാഹുവാണ് നിങ്ങളുടെ രക്ഷാധികാരിയെന്ന് നിങ്ങള്‍ മനസ്സിലാക്കുക. എത്രയോ നല്ല രക്ഷാധികാരി! എത്രയോ നല്ല സഹായി!!
\end{malayalam}}
\flushright{\begin{Arabic}
\quranayah[8][41]
\end{Arabic}}
\flushleft{\begin{malayalam}
നിങ്ങള്‍ (യുദ്ധത്തില്‍) നേടിയെടുത്ത ഏതൊരു വസ്തുവില്‍ നിന്നും അതിന്‍റെ അഞ്ചിലൊന്ന് അല്ലാഹുവിനും റസൂലിനും (റസൂലിന്‍റെ) അടുത്ത ബന്ധുക്കള്‍ക്കും അനാഥകള്‍ക്കും പാവപ്പെട്ടവര്‍ക്കും വഴിപോക്കന്‍മാര്‍ക്കും ഉള്ളതാണെന്ന് നിങ്ങള്‍ മനസ്സിലാക്കുവിന്‍. അല്ലാഹുവിലും സത്യാസത്യവിവേചനത്തിന്‍റെ ദിവസത്തില്‍ അഥവാ ആ രണ്ടു സംഘങ്ങള്‍ ഏറ്റുമുട്ടിയ ദിവസത്തില്‍ നമ്മുടെ ദാസന്‍റെ മേല്‍ നാം അവതരിപ്പിച്ചതിലും നിങ്ങള്‍ വിശ്വസിച്ചുകഴിഞ്ഞിട്ടുണ്ടെങ്കില്‍. അല്ലാഹു ഏതൊരു കാര്യത്തിനും കഴിവുള്ളവനാകുന്നു.
\end{malayalam}}
\flushright{\begin{Arabic}
\quranayah[8][42]
\end{Arabic}}
\flushleft{\begin{malayalam}
നിങ്ങള്‍ (താഴ്‌വരയില്‍ മദീനയോട്‌) അടുത്ത ഭാഗത്തും, അവര്‍ അകന്ന ഭാഗത്തും, സാര്‍ത്ഥവാഹകസംഘം നിങ്ങളെക്കാള്‍ താഴെയുമായിരുന്ന സന്ദര്‍ഭം (ഓര്‍ക്കുക.) നിങ്ങള്‍ അന്യോന്യം (പോരിന്‌) നിശ്ചയിച്ചിരുന്നുവെങ്കില്‍ നിങ്ങള്‍ ആ നിശ്ചയം നിറവേറ്റുന്നതില്‍ ഭിന്നിക്കുമായിരുന്നു. പക്ഷെ ഉണ്ടാകേണ്ട ഒരു കാര്യം അല്ലാഹു നിര്‍വഹിക്കുന്നതിന് വേണ്ടിയായിരുന്നു അത്‌. അതായത് നശിച്ചവര്‍ വ്യക്തമായ തെളിവ് കണ്ടുകൊണ്ട് നശിക്കാനും, ജീവിച്ചവര്‍ വ്യക്തമായ തെളിവ് കണ്ട് കൊണ്ട് ജീവിക്കുവാനും വേണ്ടി. തീര്‍ച്ചയായും അല്ലാഹു എല്ലാം കേള്‍ക്കുന്നവനും അറിയുന്നവനുമാകുന്നു.
\end{malayalam}}
\flushright{\begin{Arabic}
\quranayah[8][43]
\end{Arabic}}
\flushleft{\begin{malayalam}
അവരെ (ശത്രുക്കളെ) അല്ലാഹു നിനക്ക് നിന്‍റെ സ്വപ്നത്തില്‍ കുറച്ച് പേര്‍ മാത്രമായി കാണിച്ചുതന്നിരുന്ന സന്ദര്‍ഭം (ഓര്‍ക്കുക.) നിനക്ക് അവരെ അധികമായി കാണിച്ചിരുന്നെങ്കില്‍ നിങ്ങളുടെ ധൈര്യം ക്ഷയിക്കുകയും, കാര്യത്തില്‍ നിങ്ങള്‍ ഭിന്നിക്കുകയും ചെയ്യുമായിരുന്നു. പക്ഷെ അല്ലാഹു രക്ഷിച്ചു. തീര്‍ച്ചയായും അവന്‍ ഹൃദയങ്ങളിലുള്ളത് അറിയുന്നവനാകുന്നു.
\end{malayalam}}
\flushright{\begin{Arabic}
\quranayah[8][44]
\end{Arabic}}
\flushleft{\begin{malayalam}
നിങ്ങള്‍ കണ്ടുമുട്ടിയ സന്ദര്‍ഭത്തില്‍ നിങ്ങളുടെ ദൃഷ്ടിയില്‍ നിങ്ങള്‍ക്ക് അവരെ അവന്‍ കുറച്ച് മാത്രമായി കാണിക്കുകയും, അവരുടെ ദൃഷ്ടിയില്‍ നിങ്ങളെ എണ്ണം കുറച്ച് കാണിക്കുകയും ചെയ്ത സന്ദര്‍ഭം ഓര്‍ക്കുക. നടക്കേണ്ടതായ ഒരു കാര്യം അല്ലാഹു നിര്‍വഹിക്കുവാന്‍ വേണ്ടിയത്രെ അത്‌. അല്ലാഹുവിങ്കലേക്കാണ് കാര്യങ്ങള്‍ മടക്കപ്പെടുന്നത്‌.
\end{malayalam}}
\flushright{\begin{Arabic}
\quranayah[8][45]
\end{Arabic}}
\flushleft{\begin{malayalam}
സത്യവിശ്വാസികളേ, നിങ്ങള്‍ ഒരു (സൈന്യ) സംഘത്തെ കണ്ടുമുട്ടിയാല്‍ ഉറച്ചുനില്‍ക്കുകയും അല്ലാഹുവെ അധികമായി ഓര്‍മിക്കുകയും ചെയ്യുക. നിങ്ങള്‍ വിജയം പ്രാപിച്ചേക്കാം.
\end{malayalam}}
\flushright{\begin{Arabic}
\quranayah[8][46]
\end{Arabic}}
\flushleft{\begin{malayalam}
അല്ലാഹുവെയും അവന്‍റെ ദൂതനെയും നിങ്ങള്‍ അനുസരിക്കുകയും ചെയ്യുക. നിങ്ങള്‍ ഭിന്നിച്ചു പോകരുത്‌. എങ്കില്‍ നിങ്ങള്‍ക്ക് ധൈര്യക്ഷയം നേരിടുകയും നിങ്ങളുടെ വീര്യം (നശിച്ചു) പോകുകയും ചെയ്യും. നിങ്ങള്‍ ക്ഷമിക്കുക. തീര്‍ച്ചയായും അല്ലാഹു ക്ഷമാശീലരുടെ കൂടെയാകുന്നു.
\end{malayalam}}
\flushright{\begin{Arabic}
\quranayah[8][47]
\end{Arabic}}
\flushleft{\begin{malayalam}
ഗര്‍വ്വോട് കൂടിയും, ജനങ്ങളെ കാണിക്കാന്‍ വേണ്ടിയും അല്ലാഹുവിന്‍റെ മാര്‍ഗത്തില്‍ നിന്ന് (ജനങ്ങളെ) തടഞ്ഞു നിര്‍ത്താന്‍ വേണ്ടിയും തങ്ങളുടെ വീടുകളില്‍ നിന്ന് ഇറങ്ങിപ്പുറപ്പെട്ടവരെ പോലെ നിങ്ങളാകരുത്‌. അല്ലാഹു അവര്‍ പ്രവര്‍ത്തിക്കുന്നതെല്ലാം സൂക്ഷ്മമായി അറിയുന്നവനാകുന്നു.
\end{malayalam}}
\flushright{\begin{Arabic}
\quranayah[8][48]
\end{Arabic}}
\flushleft{\begin{malayalam}
ഇന്ന് ജനങ്ങളില്‍ നിങ്ങളെ തോല്‍പിക്കാന്‍ ആരും തന്നെയില്ല. തീര്‍ച്ചയായും ഞാന്‍ നിങ്ങളുടെ സംരക്ഷകനായിരിക്കും. എന്ന് പറഞ്ഞ് കൊണ്ട് പിശാച് അവര്‍ക്ക് അവരുടെ ചെയ്തികള്‍ ഭംഗിയായി തോന്നിച്ച സന്ദര്‍ഭവും (ഓര്‍ക്കുക.) അങ്ങനെ ആ രണ്ടുസംഘങ്ങള്‍ കണ്ടുമുട്ടിയപ്പോള്‍ എനിക്കു നിങ്ങളുമായി ഒരു ബന്ധവുമില്ല, തീര്‍ച്ചയായും നിങ്ങള്‍ കാണാത്ത പലതും ഞാന്‍ കാണുന്നുണ്ട്‌, തീര്‍ച്ചയായും ഞാന്‍ അല്ലാഹുവെ ഭയപ്പെടുന്നു, അല്ലാഹു കഠിനമായി ശിക്ഷിക്കുന്നവനത്രെ. എന്ന് പറഞ്ഞുകൊണ്ട് അവന്‍ (പിശാച്‌) പിന്‍മാറിക്കളഞ്ഞു.
\end{malayalam}}
\flushright{\begin{Arabic}
\quranayah[8][49]
\end{Arabic}}
\flushleft{\begin{malayalam}
ഈ കൂട്ടരെ (മുസ്ലിംകളെ) അവരുടെ മതവിശ്വാസം വഞ്ചിച്ചു കളഞ്ഞിരിക്കുന്നു എന്ന് കപടവിശ്വാസികളും, മനസ്സില്‍ രോഗമുള്ളവരും പറഞ്ഞുകൊണ്ടിരുന്ന സന്ദര്‍ഭമത്രെ അത്‌. വല്ലവനും അല്ലാഹുവിന്‍റെ മേല്‍ ഭരമേല്‍പിക്കുന്ന പക്ഷം തീര്‍ച്ചയായും അല്ലാഹു പ്രതാപിയും യുക്തിമാനുമാകുന്നു.
\end{malayalam}}
\flushright{\begin{Arabic}
\quranayah[8][50]
\end{Arabic}}
\flushleft{\begin{malayalam}
സത്യനിഷേധികളുടെ മുഖങ്ങളിലും പിന്‍വശങ്ങളിലും അടിച്ചു കൊണ്ട് മലക്കുകള്‍ അവരെ മരിപ്പിക്കുന്ന സന്ദര്‍ഭം നീ കണ്ടിരുന്നുവെങ്കില്‍! (അവര്‍ (മലക്കുകള്‍) അവരോട് പറയും:) ജ്വലിക്കുന്ന അഗ്നിയുടെ ശിക്ഷ നിങ്ങള്‍ ആസ്വദിച്ച് കൊള്ളുക.
\end{malayalam}}
\flushright{\begin{Arabic}
\quranayah[8][51]
\end{Arabic}}
\flushleft{\begin{malayalam}
നിങ്ങളുടെ കൈകള്‍ മുന്‍കൂട്ടിചെയ്തുവെച്ചത് നിമിത്തമത്രെ അത്‌. അല്ലാഹു അടിമകളോട് ഒട്ടും അനീതി കാണിക്കുന്നവനല്ല എന്നതിനാലും.
\end{malayalam}}
\flushright{\begin{Arabic}
\quranayah[8][52]
\end{Arabic}}
\flushleft{\begin{malayalam}
ഫിര്‍ഔന്‍റെ ആളുകളുടെയും അവരുടെ മുമ്പുള്ളവരുടെയും സമ്പ്രദായം പോലെത്തന്നെ. അല്ലാഹുവിന്‍റെ ദൃഷ്ടാന്തങ്ങളെ അവര്‍ നിഷേധിക്കുകയും, അപ്പോള്‍ അവരുടെ പാപങ്ങള്‍ കാരണമായി അല്ലാഹു അവരെ പിടികൂടുകയും ചെയ്തു. തീര്‍ച്ചയായും അല്ലാഹു ശക്തനും കഠിനമായി ശിക്ഷിക്കുന്നവനുമാണ്‌.
\end{malayalam}}
\flushright{\begin{Arabic}
\quranayah[8][53]
\end{Arabic}}
\flushleft{\begin{malayalam}
ഒരു ജനവിഭാഗത്തിനു് താന്‍ ചെയ്തുകൊടുത്ത അനുഗ്രഹം അവരുടെ സ്വന്തം നിലപാടില്‍ അവര്‍ മാറ്റം വരുത്തുന്നത് വരെ അല്ലാഹു മാറ്റിക്കളയുന്നതല്ല എന്നത്കൊന്നുത്രെ അത്‌. അല്ലാഹു എല്ലാം കേള്‍ ക്കുന്നവനും അറിയുന്നവനുമാണ് എന്നത്കൊന്നുു‍ം
\end{malayalam}}
\flushright{\begin{Arabic}
\quranayah[8][54]
\end{Arabic}}
\flushleft{\begin{malayalam}
ഫിര്‍ഔന്‍റെ ആളുകളുടെയും അവരുടെ മുമ്പുള്ളവരുടെയും സമ്പ്രദായം പോലെത്തന്നെ. അവര്‍ അവരുടെ രക്ഷിതാവിന്‍റെ ദൃഷ്ടാന്തങ്ങള്‍ നിഷേധിച്ചുതള്ളുകയും, അപ്പോള്‍ അവരുടെ പാപങ്ങള്‍ കാരണമായി നാം അവരെ നശിപ്പിക്കുകയും ചെയ്തു. ഫിര്‍ഔന്‍റെ ആളുകളെ നാം മുക്കിനശിപ്പിക്കുകയാണ് ചെയ്തത്‌. (അവര്‍) എല്ലാവരും അക്രമികളായിരുന്നു.
\end{malayalam}}
\flushright{\begin{Arabic}
\quranayah[8][55]
\end{Arabic}}
\flushleft{\begin{malayalam}
തീര്‍ച്ചയായും അല്ലാഹുവിന്‍റെ അടുക്കല്‍ ജന്തുക്കളില്‍ വെച്ച് ഏറ്റവും മോശപ്പെട്ടവര്‍ സത്യനിഷേധികളാകുന്നു. ആകയാല്‍ അവര്‍ വിശ്വസിക്കുകയില്ല.
\end{malayalam}}
\flushright{\begin{Arabic}
\quranayah[8][56]
\end{Arabic}}
\flushleft{\begin{malayalam}
അവരില്‍ ഒരു വിഭാഗവുമായി നീ കരാറില്‍ ഏര്‍പെടുകയുണ്ടായി. എന്നിട്ട് ഓരോ തവണയും തങ്ങളുടെ കരാര്‍ അവര്‍ ലംഘിച്ചുകൊണ്ടിരുന്നു. അവര്‍ (അല്ലാഹുവെ) സൂക്ഷിക്കുന്നുമില്ല.
\end{malayalam}}
\flushright{\begin{Arabic}
\quranayah[8][57]
\end{Arabic}}
\flushleft{\begin{malayalam}
അതിനാല്‍ നീ അവരെ യുദ്ധത്തില്‍ കണ്ടുമുട്ടിയാല്‍ അവര്‍ക്കേല്‍പിക്കുന്ന നാശം അവരുടെ പിന്നില്‍ വരുന്നവരെയും കൂടി തിരിച്ചോടിക്കും വിധമാക്കുക. അവര്‍ ശ്രദ്ധിച്ചു മനസ്സിലാക്കിയേക്കാം.
\end{malayalam}}
\flushright{\begin{Arabic}
\quranayah[8][58]
\end{Arabic}}
\flushleft{\begin{malayalam}
വല്ല ജനവിഭാഗത്തില്‍ നിന്നും വഞ്ചനയുണ്ടാകുമെന്ന് നീ ഭയപ്പെടുന്ന പക്ഷം തത്തുല്യമായി നീ അവരിലേക്ക് എറിഞ്ഞുകൊടുത്തേക്കുക. തീര്‍ച്ചയായും അല്ലാഹു വഞ്ചകന്‍മാരെ ഇഷ്ടപ്പെടുകയില്ല.
\end{malayalam}}
\flushright{\begin{Arabic}
\quranayah[8][59]
\end{Arabic}}
\flushleft{\begin{malayalam}
സത്യനിഷേധികളായ ആളുകള്‍, തങ്ങള്‍ അതിജയിച്ചുകഴിഞ്ഞിരിക്കുന്നു എന്ന് ധരിച്ചു പോകരുത്‌. തീര്‍ച്ചയായും അവര്‍ക്ക് (അല്ലാഹുവെ) തോല്‍പിക്കാനാവില്ല.
\end{malayalam}}
\flushright{\begin{Arabic}
\quranayah[8][60]
\end{Arabic}}
\flushleft{\begin{malayalam}
അവരെ നേരിടാന്‍ വേണ്ടി നിങ്ങളുടെ കഴിവില്‍ പെട്ട എല്ലാ ശക്തിയും, കെട്ടിനിര്‍ത്തിയ കുതിരകളെയും നിങ്ങള്‍ ഒരുക്കുക. അതുമുഖേന അല്ലാഹുവിന്‍റെയും നിങ്ങളുടെയും ശത്രുവെയും, അവര്‍ക്ക് പുറമെ നിങ്ങള്‍ അറിയാത്തവരും അല്ലാഹു അറിയുന്നവരുമായ മറ്റുചിലരെയും നിങ്ങള്‍ ഭയപ്പെടുത്തുവാന്‍ വേണ്ടി. നിങ്ങള്‍ അല്ലാഹുവിന്‍റെ മാര്‍ഗത്തില്‍ ഏതൊരു വസ്തു ചെലവഴിച്ചാലും നിങ്ങള്‍ക്കതിന്‍റെ പൂര്‍ണ്ണമായ പ്രതിഫലം നല്‍കപ്പെടും. നിങ്ങളോട് അനീതി കാണിക്കപ്പെടുന്നതല്ല.
\end{malayalam}}
\flushright{\begin{Arabic}
\quranayah[8][61]
\end{Arabic}}
\flushleft{\begin{malayalam}
ഇനി, അവര്‍ സമാധാനത്തിലേക്ക് ചായ്‌വ് കാണിക്കുകയാണെങ്കില്‍ നീയും അതിലേക്ക് ചായ്‌വ് കാണിക്കുകയും, അല്ലാഹുവിന്‍റെ മേല്‍ ഭരമേല്‍പിക്കുകയും ചെയ്യുക. തീര്‍ച്ചയായും അവനാണ് എല്ലാം കേള്‍ക്കുകയും അറിയുകയും ചെയ്യുന്നവന്‍.
\end{malayalam}}
\flushright{\begin{Arabic}
\quranayah[8][62]
\end{Arabic}}
\flushleft{\begin{malayalam}
ഇനി അവര്‍ നിന്നെ വഞ്ചിക്കാന്‍ ഉദ്ദേശിക്കുന്ന പക്ഷം തീര്‍ച്ചയായും നിനക്ക് അല്ലാഹു മതി. അവനാണ് അവന്‍റെ സഹായം മുഖേനയും, വിശ്വാസികള്‍ മുഖേനയും നിനക്ക് പിന്‍ബലം നല്‍കിയവന്‍.
\end{malayalam}}
\flushright{\begin{Arabic}
\quranayah[8][63]
\end{Arabic}}
\flushleft{\begin{malayalam}
അവരുടെ (വിശ്വാസികളുടെ) ഹൃദയങ്ങള്‍ തമ്മില്‍ അവന്‍ ഇണക്കിചേര്‍ക്കുകയും ചെയ്തിരിക്കുന്നു. ഭൂമിയിലുള്ളത് മുഴുവന്‍ നീ ചെലവഴിച്ചാല്‍ പോലും അവരുടെ ഹൃദയങ്ങള്‍ തമ്മില്‍ ഇണക്കിചേര്‍ക്കാന്‍ നിനക്ക് സാധിക്കുമായിരുന്നില്ല. എന്നാല്‍ അല്ലാഹു അവരെ തമ്മില്‍ ഇണക്കിചേര്‍ത്തിരിക്കുന്നു. തീര്‍ച്ചയായും അവന്‍ പ്രതാപിയും യുക്തിമാനുമാകുന്നു.
\end{malayalam}}
\flushright{\begin{Arabic}
\quranayah[8][64]
\end{Arabic}}
\flushleft{\begin{malayalam}
നബിയേ, നിനക്കും നിന്നെ പിന്‍പറ്റിയ സത്യവിശ്വാസികള്‍ക്കും അല്ലാഹു തന്നെ മതി.
\end{malayalam}}
\flushright{\begin{Arabic}
\quranayah[8][65]
\end{Arabic}}
\flushleft{\begin{malayalam}
നബിയേ, നീ വിശ്വാസികളെ യുദ്ധത്തിന് പ്രോത്സാഹിപ്പിക്കുക. നിങ്ങളുടെ കൂട്ടത്തില്‍ ക്ഷമാശീലരായ ഇരുപത് പേരുണ്ടായിരുന്നാല്‍ ഇരുനൂറ് പേരെ അവര്‍ക്ക് ജയിച്ചടക്കാവുന്നതാണ്‌. നിങ്ങളുടെ കൂട്ടത്തില്‍ നൂറ് പേരുണ്ടായിരുന്നാല്‍ സത്യനിഷേധികളില്‍ നിന്ന് ആയിരം പേരെ അവര്‍ക്ക് ജയിച്ചടക്കാവുന്നതാണ്‌. അവര്‍ കാര്യം ഗ്രഹിക്കാത്ത ഒരു ജനവിഭാഗമാണ് എന്നതുകൊണ്ടത്രെ അത്‌.
\end{malayalam}}
\flushright{\begin{Arabic}
\quranayah[8][66]
\end{Arabic}}
\flushleft{\begin{malayalam}
ഇപ്പോള്‍ അല്ലാഹു നിങ്ങള്‍ക്ക് ഭാരം കുറച്ച് തന്നിരിക്കുന്നു. നിങ്ങളില്‍ ബലഹീനതയുണ്ടെന്ന് അവന്‍ അറിയുകയും ചെയ്തിരിക്കുന്നു. അതിനാല്‍ നിങ്ങളുടെ കൂട്ടത്തില്‍ ക്ഷമാശീലരായ നൂറുപേരുണ്ടായിരുന്നാല്‍ അവര്‍ക്ക് ഇരുനൂറ് പേരെ ജയിച്ചടക്കാവുന്നതാണ്‌. നിങ്ങളുടെ കൂട്ടത്തില്‍ ആയിരം പേരുണ്ടായിരുന്നാല്‍ അല്ലാഹുവിന്‍റെ അനുമതി പ്രകാരം രണ്ടായിരം പേരെ അവര്‍ക്കു ജയിച്ചടക്കാവുന്നതാണ്‌. അല്ലാഹു ക്ഷമാശീലരോടൊപ്പമാകുന്നു.
\end{malayalam}}
\flushright{\begin{Arabic}
\quranayah[8][67]
\end{Arabic}}
\flushleft{\begin{malayalam}
ഒരു പ്രവാചകന്നും (ശത്രുക്കളെ കീഴടക്കി) നാട്ടില്‍ ശക്തി പ്രാപിക്കുന്നത് വരെ യുദ്ധത്തടവുകാരുണ്ടായിരിക്കാന്‍ പാടുള്ളതല്ല. നിങ്ങള്‍ ഇഹലോകത്തെ ക്ഷണികമായ നേട്ടം ആഗ്രഹിക്കുന്നു. അല്ലാഹുവാകട്ടെ പരലോകത്തെയും ഉദ്ദേശിക്കുന്നു. അല്ലാഹു പ്രതാപിയും യുക്തിമാനുമാകുന്നു.
\end{malayalam}}
\flushright{\begin{Arabic}
\quranayah[8][68]
\end{Arabic}}
\flushleft{\begin{malayalam}
അല്ലാഹുവിങ്കല്‍ നിന്നുള്ള നിശ്ചയം മുന്‍കൂട്ടി ഉണ്ടായിരുന്നില്ലെങ്കില്‍ നിങ്ങള്‍ ആ വാങ്ങിയതിന്‍റെ പേരില്‍ നിങ്ങളെ വമ്പിച്ച ശിക്ഷ ബാധിക്കുക തന്നെ ചെയ്യുമായിരുന്നു.
\end{malayalam}}
\flushright{\begin{Arabic}
\quranayah[8][69]
\end{Arabic}}
\flushleft{\begin{malayalam}
എന്നാല്‍ (യുദ്ധത്തിനിടയില്‍) നിങ്ങള്‍ നേടിയെടുത്തതില്‍ നിന്ന് അനുവദനീയവും ഉത്തമവുമായത് നിങ്ങള്‍ ഭക്ഷിച്ചു കൊള്ളുക. അല്ലാഹുവെ നിങ്ങള്‍ സൂക്ഷിക്കുകയും ചെയ്യുക. തീര്‍ച്ചയായും അല്ലാഹു ഏറെ പൊറുക്കുന്നവനും കരുണാനിധിയുമാകുന്നു.
\end{malayalam}}
\flushright{\begin{Arabic}
\quranayah[8][70]
\end{Arabic}}
\flushleft{\begin{malayalam}
നബിയേ, നിങ്ങളുടെ കൈവശമുള്ള യുദ്ധത്തടവുകാരോട് നീ പറയുക: നിങ്ങളുടെ ഹൃദയങ്ങളില്‍ വല്ല നന്‍മയുമുള്ളതായി അല്ലാഹു അറിയുന്ന പക്ഷം നിങ്ങളുടെ പക്കല്‍ നിന്ന് മേടിക്കപ്പെട്ടതിനേക്കാള്‍ ഉത്തമമായത് അവന്‍ നിങ്ങള്‍ക്ക് തരികയും നിങ്ങള്‍ക്ക് അവന്‍ പൊറുത്തുതരികയും ചെയ്യുന്നതാണ്‌. അല്ലാഹു ഏറെ പൊറുക്കുന്നവനും കരുണാനിധിയുമാകുന്നു.
\end{malayalam}}
\flushright{\begin{Arabic}
\quranayah[8][71]
\end{Arabic}}
\flushleft{\begin{malayalam}
ഇനി നിന്നെ വഞ്ചിക്കാനാണ് അവര്‍ ഉദ്ദേശിക്കുന്നതെങ്കില്‍ മുമ്പ് അവര്‍ അല്ലാഹുവോടും വഞ്ചന കാണിച്ചിട്ടുണ്ട്‌. അത് കൊണ്ടാണ് അവന്‍ അവരെ (നിങ്ങള്‍ക്കു) കീഴ്പെടുത്തി തന്നത്‌. അല്ലാഹു എല്ലാം അറിയുന്നവനും യുക്തിമാനുമാകുന്നു.
\end{malayalam}}
\flushright{\begin{Arabic}
\quranayah[8][72]
\end{Arabic}}
\flushleft{\begin{malayalam}
തീര്‍ച്ചയായും വിശ്വസിക്കുകയും, സ്വദേശം വെടിഞ്ഞ് പോകുകയും തങ്ങളുടെ സ്വത്തുക്കളും ശരീരങ്ങളും കൊണ്ട് അല്ലാഹുവിന്‍റെ മാര്‍ഗത്തില്‍ സമരത്തില്‍ ഏര്‍പെടുകയും ചെയ്തവരും, അവര്‍ക്ക് അഭയം നല്‍കുകയും സഹായിക്കുകയും ചെയ്തവരും അന്യോന്യം ഉറ്റമിത്രങ്ങളാകുന്നു. വിശ്വസിക്കുകയും എന്നാല്‍ സ്വദേശം വെടിഞ്ഞ് പോകാതിരിക്കുകയും ചെയ്തവരോട് അവര്‍ സ്വദേശം വെടിഞ്ഞ് പോരുന്നത് വരെ നിങ്ങള്‍ക്ക് യാതൊരു സംരക്ഷണ ബാധ്യതയുമില്ല. ഇനി മതകാര്യത്തില്‍ അവര്‍ നിങ്ങളുടെ സഹായം തേടുകയാണെങ്കില്‍ സഹായിക്കാന്‍ നിങ്ങള്‍ക്ക് ബാധ്യതയുണ്ട്‌. എന്നാല്‍ നിങ്ങളുമായി കരാറില്‍ ഏര്‍പെട്ടുകഴിയുന്ന ജനതയ്ക്കെതിരെ (നിങ്ങളവരെ സഹായിക്കാന്‍) പാടില്ല. അല്ലാഹു നിങ്ങള്‍ പ്രവര്‍ത്തിക്കുന്നതെല്ലാം കണ്ടറിയുന്നവനാകുന്നു.
\end{malayalam}}
\flushright{\begin{Arabic}
\quranayah[8][73]
\end{Arabic}}
\flushleft{\begin{malayalam}
സത്യനിഷേധികളും അന്യോന്യം മിത്രങ്ങളാകുന്നു. ഇത് (ഈ നിര്‍ദേശങ്ങള്‍) നിങ്ങള്‍ പ്രാവര്‍ത്തികമാക്കിയിട്ടില്ലെങ്കില്‍ നാട്ടില്‍ കുഴപ്പവും വലിയ നാശവും ഉണ്ടായിത്തീരുന്നതാണ്‌.
\end{malayalam}}
\flushright{\begin{Arabic}
\quranayah[8][74]
\end{Arabic}}
\flushleft{\begin{malayalam}
വിശ്വസിക്കുകയും സ്വദേശം വെടിഞ്ഞ് പോകുകയും അല്ലാഹുവിന്‍റെ മാര്‍ഗത്തില്‍ സമരത്തില്‍ ഏര്‍പെടുകയും ചെയ്തവരും, അവര്‍ക്ക് അഭയം നല്‍കുകയും സഹായിക്കുകയും ചെയ്തവരും തന്നെയാണ് യഥാര്‍ത്ഥത്തില്‍ സത്യവിശ്വാസികള്‍. അവര്‍ക്ക് പാപമോചനവും മാന്യമായ ഉപജീവനവും ഉണ്ടായിരിക്കും.
\end{malayalam}}
\flushright{\begin{Arabic}
\quranayah[8][75]
\end{Arabic}}
\flushleft{\begin{malayalam}
അതിന് ശേഷം വിശ്വസിക്കുകയും, സ്വദേശം വെടിയുകയും, നിങ്ങളോടൊപ്പം സമരത്തില്‍ ഏര്‍പെടുകയും ചെയ്തവരും നിങ്ങളുടെ കൂട്ടത്തില്‍ തന്നെ. എന്നാല്‍ രക്തബന്ധമുള്ളവര്‍ അല്ലാഹുവിന്‍റെ രേഖയില്‍ (നിയമത്തില്‍) അന്യോന്യം കൂടുതല്‍ ബന്ധപ്പെട്ടവരാകുന്നു. തീര്‍ച്ചയായും അല്ലാഹു ഏത് കാര്യത്തെപ്പറ്റിയും അറിവുള്ളവനാകുന്നു.
\end{malayalam}}
\chapter{\textmalayalam{തൌബ ( പശ്ചാത്താപം )}}
\begin{Arabic}
\Huge{\centerline{\basmalah}}\end{Arabic}
\flushright{\begin{Arabic}
\quranayah[9][1]
\end{Arabic}}
\flushleft{\begin{malayalam}
ബഹുദൈവവിശ്വാസികളില്‍ നിന്ന് ആരുമായി നിങ്ങള്‍ കരാറില്‍ ഏര്‍പെട്ടിട്ടുണ്ടോ അവരോട് അല്ലാഹുവിന്‍റെയും അവന്‍റെ ദൂതന്‍റെയും ഭാഗത്ത് നിന്നുള്ള ബാധ്യത ഒഴിഞ്ഞതായി ഇതാ പ്രഖ്യാപിക്കുന്നു.
\end{malayalam}}
\flushright{\begin{Arabic}
\quranayah[9][2]
\end{Arabic}}
\flushleft{\begin{malayalam}
അതിനാല്‍ (ബഹുദൈവവിശ്വാസികളേ,) നിങ്ങള്‍ നാലുമാസക്കാലം ഭൂമിയില്‍ യഥേഷ്ടം സഞ്ചരിച്ച് കൊള്ളുക. നിങ്ങള്‍ക്ക് അല്ലാഹുവിനെ തോല്‍പിക്കാനാവില്ലെന്നും, സത്യനിഷേധികള്‍ക്കു അല്ലാഹു അപമാനം വരുത്തുന്നതാണെന്നും നിങ്ങള്‍ അറിഞ്ഞിരിക്കുകയും ചെയ്യുക.
\end{malayalam}}
\flushright{\begin{Arabic}
\quranayah[9][3]
\end{Arabic}}
\flushleft{\begin{malayalam}
മഹത്തായ ഹജ്ജിന്‍റെ ദിവസത്തില്‍ മനുഷ്യരോട് (പൊതുവായി) അല്ലാഹുവിന്‍റെയും റസൂലിന്‍റെയും ഭാഗത്തുനിന്ന് ഇതാ അറിയിക്കുകയും ചെയ്യുന്നു; അല്ലാഹുവിനും അവന്‍റെ ദൂതന്നും ബഹുദൈവവിശ്വാസികളോട് യാതൊരു ബാധ്യതയുമില്ലെന്ന്‌. എന്നാല്‍ (ബഹുദൈവവിശ്വാസികളേ,) നിങ്ങള്‍ പശ്ചാത്തപിക്കുകയാണെങ്കില്‍ അതാണ് നിങ്ങള്‍ക്ക് ഉത്തമം. നിങ്ങള്‍ പിന്തിരിഞ്ഞ് കളയുകയാണെങ്കില്‍ നിങ്ങള്‍ക്ക് അല്ലാഹുവെ തോല്‍പിക്കാനാവില്ലെന്ന് നിങ്ങള്‍ അറിഞ്ഞിരിക്കുക. (നബിയേ,) സത്യനിഷേധികള്‍ക്ക് വേദനയേറിയ ശിക്ഷയെപ്പറ്റി നീ സന്തോഷവാര്‍ത്ത അറിയിക്കുക.
\end{malayalam}}
\flushright{\begin{Arabic}
\quranayah[9][4]
\end{Arabic}}
\flushleft{\begin{malayalam}
എന്നാല്‍ ബഹുദൈവവിശ്വാസികളുടെ കൂട്ടത്തില്‍ നിന്ന് നിങ്ങള്‍ കരാറില്‍ ഏര്‍പെടുകയും, എന്നിട്ട് നിങ്ങളോട് (അത് പാലിക്കുന്നതില്‍) യാതൊരു ന്യൂനതയും വരുത്താതിരിക്കുകയും, നിങ്ങള്‍ക്കെതിരില്‍ ആര്‍ക്കും സഹായം നല്‍കാതിരിക്കുകയും ചെയ്തവര്‍ ഇതില്‍ നിന്ന് ഒഴിവാണ്‌. അപ്പോള്‍ അവരോടുള്ള കരാര്‍ അവരുടെ കാലാവധിവരെ നിങ്ങള്‍ നിറവേറ്റുക. തീര്‍ച്ചയായും അല്ലാഹു സൂക്ഷ്മത പാലിക്കുന്നവരെ ഇഷ്ടപ്പെടുന്നു.
\end{malayalam}}
\flushright{\begin{Arabic}
\quranayah[9][5]
\end{Arabic}}
\flushleft{\begin{malayalam}
അങ്ങനെ ആ വിലക്കപ്പെട്ടമാസങ്ങള്‍ കഴിഞ്ഞാല്‍ ആ ബഹുദൈവവിശ്വാസികളെ നിങ്ങള്‍ കണ്ടെത്തിയേടത്ത് വെച്ച് കൊന്നുകളയുക. അവരെ പിടികൂടുകയും വളയുകയും അവര്‍ക്കുവേണ്ടി പതിയിരിക്കാവുന്നിടത്തെല്ലാം പതിയിരിക്കുകയും ചെയ്യുക. ഇനി അവര്‍ പശ്ചാത്തപിക്കുകയും നമസ്കാരം മുറപോലെ നിര്‍വഹിക്കുകയും സകാത്ത് നല്‍കുകയും ചെയ്യുന്ന പക്ഷം നിങ്ങള്‍ അവരുടെ വഴി ഒഴിവാക്കികൊടുക്കുക. തീര്‍ച്ചയായും അല്ലാഹു ഏറെ പൊറുക്കുന്നവനും കരുണാനിധിയുമാണ്‌.
\end{malayalam}}
\flushright{\begin{Arabic}
\quranayah[9][6]
\end{Arabic}}
\flushleft{\begin{malayalam}
ബഹുദൈവവിശ്വാസികളില്‍ വല്ലവനും നിന്‍റെ അടുക്കല്‍ അഭയം തേടി വന്നാല്‍ അല്ലാഹുവിന്‍റെ വചനം അവന്‍ കേട്ടു ഗ്രഹിക്കാന്‍ വേണ്ടി അവന്ന് അഭയം നല്‍കുക. എന്നിട്ട് അവന്ന് സുരക്ഷിതത്വമുള്ള സ്ഥലത്ത് അവനെ എത്തിച്ചുകൊടുക്കുകയും ചെയ്യുക. അവര്‍ അറിവില്ലാത്ത ഒരു ജനവിഭാഗമാണ് എന്നതു കൊണ്ടാണത്‌.
\end{malayalam}}
\flushright{\begin{Arabic}
\quranayah[9][7]
\end{Arabic}}
\flushleft{\begin{malayalam}
എങ്ങനെയാണ് ആ ബഹുദൈവവിശ്വാസികള്‍ക്ക് അല്ലാഹുവിന്‍റെ അടുക്കലും അവന്‍റെ ദൂതന്‍റെ അടുക്കലും ഉടമ്പടി നിലനില്‍ക്കുക? നിങ്ങള്‍ ആരുമായി മസ്ജിദുല്‍ ഹറാമിന്‍റെ അടുത്ത് വെച്ച് കരാറില്‍ ഏര്‍പെട്ടുവോ അവര്‍ക്കല്ലാതെ. എന്നാല്‍ അവര്‍ നിങ്ങളോട് ശരിയായി വര്‍ത്തിക്കുന്നേടത്തോളം നിങ്ങള്‍ അവരോടും ശരിയായി വര്‍ത്തിക്കുക. തീര്‍ച്ചയായും അല്ലാഹു സൂക്ഷ്മത പാലിക്കുന്നവരെ ഇഷ്ടപ്പെടുന്നു.
\end{malayalam}}
\flushright{\begin{Arabic}
\quranayah[9][8]
\end{Arabic}}
\flushleft{\begin{malayalam}
അതെങ്ങനെ (നിലനില്‍ക്കും?) നിങ്ങളുടെ മേല്‍ അവര്‍ വിജയം നേടുന്ന പക്ഷം നിങ്ങളുടെ കാര്യത്തില്‍ കുടുംബബന്ധമോ ഉടമ്പടിയോ അവര്‍ പരിഗണിക്കുകയില്ല. അവരുടെ വായ്കൊണ്ട് അവര്‍ നിങ്ങളെ തൃപ്തിപ്പെടുത്തും. അവരുടെ മനസ്സുകള്‍ വെറുക്കുകയും ചെയ്യും. അവരില്‍ അധികപേരും ധിക്കാരികളാകുന്നു.
\end{malayalam}}
\flushright{\begin{Arabic}
\quranayah[9][9]
\end{Arabic}}
\flushleft{\begin{malayalam}
അവര്‍ അല്ലാഹുവിന്‍റെ ദൃഷ്ടാന്തങ്ങളെ തുച്ഛമായ വിലയ്ക്ക് വിറ്റുകളയുകയും, അങ്ങനെ അവന്‍റെ മാര്‍ഗത്തില്‍ നിന്ന് (ആളുകളെ) തടയുകയും ചെയ്തു. തീര്‍ച്ചയായും അവര്‍ പ്രവര്‍ത്തിച്ചു വരുന്നത് വളരെ ചീത്തയാകുന്നു.
\end{malayalam}}
\flushright{\begin{Arabic}
\quranayah[9][10]
\end{Arabic}}
\flushleft{\begin{malayalam}
ഒരു സത്യവിശ്വാസിയുടെ കാര്യത്തിലും കുടുംബബന്ധമോ ഉടമ്പടിയോ അവര്‍ പരിഗണിക്കാറില്ല. അവര്‍ തന്നെയാണ് അതിക്രമകാരികള്‍.
\end{malayalam}}
\flushright{\begin{Arabic}
\quranayah[9][11]
\end{Arabic}}
\flushleft{\begin{malayalam}
എന്നാല്‍ അവര്‍ പശ്ചാത്തപിക്കുകയും, നമസ്കാരം മുറപോലെ നിര്‍വഹിക്കുകയും, സകാത്ത് നല്‍കുകയും ചെയ്യുന്ന പക്ഷം അവര്‍ മതത്തില്‍ നിങ്ങളുടെ സഹോദരങ്ങളാകുന്നു. മനസ്സിലാക്കുന്ന ആളുകള്‍ക്ക് വേണ്ടി നാം ദൃഷ്ടാന്തങ്ങള്‍ വിശദീകരിക്കുന്നു.
\end{malayalam}}
\flushright{\begin{Arabic}
\quranayah[9][12]
\end{Arabic}}
\flushleft{\begin{malayalam}
ഇനി അവര്‍ കരാറില്‍ ഏര്‍പെട്ടതിന് ശേഷം തങ്ങളുടെ ശപഥങ്ങള്‍ ലംഘിക്കുകയും, നിങ്ങളുടെ മതത്തെ പരിഹസിക്കുകയും ചെയ്യുകയാണെങ്കില്‍ സത്യനിഷേധത്തിന്‍റെ നേതാക്കളോട് നിങ്ങള്‍ യുദ്ധം ചെയ്യുക. തീര്‍ച്ചയായും അവര്‍ക്ക് ശപഥങ്ങളേയില്ല. അവര്‍ വിരമിച്ചേക്കാം.
\end{malayalam}}
\flushright{\begin{Arabic}
\quranayah[9][13]
\end{Arabic}}
\flushleft{\begin{malayalam}
തങ്ങളുടെ ശപഥങ്ങള്‍ ലംഘിക്കുകയും, റസൂലിനെ പുറത്താക്കാന്‍ മുതിരുകയും ചെയ്ത ഒരു ജനവിഭാഗത്തോട് നിങ്ങള്‍ യുദ്ധം ചെയ്യുന്നില്ലേ? അവരാണല്ലോ നിങ്ങളോട് ആദ്യതവണ (യുദ്ധം) തുടങ്ങിയത്‌. അവരെ നിങ്ങള്‍ ഭയപ്പെടുകയാണോ? എന്നാല്‍ നിങ്ങള്‍ ഭയപ്പെടാന്‍ ഏറ്റവും അര്‍ഹതയുള്ളത് അല്ലാഹുവെയാണ്‌; നിങ്ങള്‍ വിശ്വാസികളാണെങ്കില്‍.
\end{malayalam}}
\flushright{\begin{Arabic}
\quranayah[9][14]
\end{Arabic}}
\flushleft{\begin{malayalam}
നിങ്ങള്‍ അവരോട് യുദ്ധം ചെയ്യുക. നിങ്ങളുടെ കൈകളാല്‍ അല്ലാഹു അവരെ ശിക്ഷിക്കുകയും അവരെ അവന്‍ അപമാനിക്കുകയും, അവര്‍ക്കെതിരില്‍ നിങ്ങളെ അവന്‍ സഹായിക്കുകയും, വിശ്വാസികളായ ആളുകളുടെ ഹൃദയങ്ങള്‍ക്ക് അവന്‍ ശമനം നല്‍കുകയും ചെയ്യുന്നതാണ്‌.
\end{malayalam}}
\flushright{\begin{Arabic}
\quranayah[9][15]
\end{Arabic}}
\flushleft{\begin{malayalam}
അവരുടെ മനസ്സുകളിലെ രോഷം അവന്‍ നീക്കികളയുകയും ചെയ്യുന്നതാണ്‌. അല്ലാഹു താന്‍ ഉദ്ദേശിക്കുന്നവരുടെ പശ്ചാത്താപം സ്വീകരിക്കുന്നു. അല്ലാഹു എല്ലാം അറിയുന്നവനും യുക്തിമാനുമാകുന്നു.
\end{malayalam}}
\flushright{\begin{Arabic}
\quranayah[9][16]
\end{Arabic}}
\flushleft{\begin{malayalam}
അതല്ല, നിങ്ങളില്‍ നിന്ന് സമരം ചെയ്യുകയും, അല്ലാഹുവിന്നും അവന്‍റെ ദൂതന്നും സത്യവിശ്വാസികള്‍ക്കും പുറമെ യാതൊരു രഹസ്യകൂട്ടുകെട്ടും സ്വീകരിക്കാതിരിക്കുകയും ചെയ്തവര്‍ ആരെന്ന് അല്ലാഹു അറിഞ്ഞിട്ടല്ലാതെ നിങ്ങളെ വിട്ടേക്കുമെന്ന് നിങ്ങള്‍ ധരിച്ചിരിക്കുകയാണോ? അല്ലാഹുവാകട്ടെ നിങ്ങള്‍ പ്രവര്‍ത്തിക്കുന്നതെല്ലാം സൂക്ഷ്മമായി അറിയുന്നവനാകുന്നു.
\end{malayalam}}
\flushright{\begin{Arabic}
\quranayah[9][17]
\end{Arabic}}
\flushleft{\begin{malayalam}
ബഹുദൈവവാദികള്‍ക്ക്‌, സത്യനിഷേധത്തിന് സ്വയം സാക്ഷ്യം വഹിക്കുന്നവരായിക്കൊണ്ട് അല്ലാഹുവിന്‍റെ പള്ളികള്‍ പരിപാലിക്കാനവകാശമില്ല. അത്തരക്കാരുടെ കര്‍മ്മങ്ങള്‍ നിഷ്ഫലമായിരിക്കുന്നു. നരകത്തില്‍ അവര്‍ നിത്യവാസികളായിരിക്കുകയും ചെയ്യും.
\end{malayalam}}
\flushright{\begin{Arabic}
\quranayah[9][18]
\end{Arabic}}
\flushleft{\begin{malayalam}
അല്ലാഹുവിന്‍റെ പള്ളികള്‍ പരിപാലിക്കേണ്ടത് അല്ലാഹുവിലും അന്ത്യദിനത്തിലും വിശ്വസിക്കുകയും, നമസ്കാരം മുറപോലെ നിര്‍വഹിക്കുകയും, സകാത്ത് നല്‍കുകയും അല്ലാഹുവെയല്ലാതെ ഭയപ്പെടാതിരിക്കുകയും ചെയ്തവര്‍ മാത്രമാണ്‌. എന്നാല്‍ അത്തരക്കാര്‍ സന്‍മാര്‍ഗം പ്രാപിക്കുന്നവരുടെ കൂട്ടത്തിലായേക്കാം.
\end{malayalam}}
\flushright{\begin{Arabic}
\quranayah[9][19]
\end{Arabic}}
\flushleft{\begin{malayalam}
തീര്‍ത്ഥാടകന്ന് കുടിക്കാന്‍ കൊടുക്കുന്നതും, മസ്ജിദുല്‍ ഹറാം പരിപാലിക്കുന്നതും അല്ലാഹുവിലും അന്ത്യദിനത്തിലും വിശ്വസിക്കുകയും, അല്ലാഹുവിന്‍റെ മാര്‍ഗത്തില്‍ സമരം നടത്തുകയും ചെയ്യുന്നവരുടെ പ്രവര്‍ത്തനത്തിന് തുല്യമായി നിങ്ങള്‍ കണക്കാക്കിയിരിക്കയാണോ ? അവര്‍ അല്ലാഹുവിങ്കല്‍ ഒരുപോലെയാവുകയില്ല. അല്ലാഹു അക്രമികളായ ആളുകളെ സന്‍മാര്‍ഗത്തിലാക്കുന്നതല്ല.
\end{malayalam}}
\flushright{\begin{Arabic}
\quranayah[9][20]
\end{Arabic}}
\flushleft{\begin{malayalam}
വിശ്വസിക്കുകയും സ്വദേശം വെടിയുകയും തങ്ങളുടെ സ്വത്തുക്കളും ശരീരങ്ങളും കൊണ്ട് അല്ലാഹുവിന്‍റെ മാര്‍ഗത്തില്‍ സമരം നടത്തുകയും ചെയ്തവര്‍ അല്ലാഹുവിങ്കല്‍ ഏറ്റവും മഹത്തായ പദവിയുള്ളവരാണ്‌. അവര്‍ തന്നെയാണ് വിജയം പ്രാപിച്ചവര്‍.
\end{malayalam}}
\flushright{\begin{Arabic}
\quranayah[9][21]
\end{Arabic}}
\flushleft{\begin{malayalam}
അവര്‍ക്ക് അവരുടെ രക്ഷിതാവ് അവന്‍റെ പക്കല്‍ നിന്നുള്ള കാരുണ്യത്തെയും പ്രീതിയെയും സ്വര്‍ഗത്തോപ്പുകളെയും പറ്റി സന്തോഷവാര്‍ത്ത അറിയിക്കുന്നു. അവര്‍ക്ക് അവിടെ ശാശ്വതമായ സുഖാനുഭവമാണുള്ളത്‌.
\end{malayalam}}
\flushright{\begin{Arabic}
\quranayah[9][22]
\end{Arabic}}
\flushleft{\begin{malayalam}
അവരതില്‍ നിത്യവാസികളായിരിക്കും. തീര്‍ച്ചയായും അല്ലാഹുവിന്‍റെ അടുക്കലാണ് മഹത്തായ പ്രതിഫലമുള്ളത്‌.
\end{malayalam}}
\flushright{\begin{Arabic}
\quranayah[9][23]
\end{Arabic}}
\flushleft{\begin{malayalam}
സത്യവിശ്വാസികളേ, നിങ്ങളുടെ പിതാക്കളും നിങ്ങളുടെ സഹോദരങ്ങളും സത്യവിശ്വാസത്തേക്കാള്‍ സത്യനിഷേധത്തെ പ്രിയങ്കരമായി കരുതുകയാണെങ്കില്‍ അവരെ നിങ്ങള്‍ രക്ഷാകര്‍ത്താക്കളായി സ്വീകരിക്കരുത്‌. നിങ്ങളില്‍ നിന്ന് ആരെങ്കിലും അവരെ രക്ഷാകര്‍ത്താക്കളായി സ്വീകരിക്കുന്ന പക്ഷം അവര്‍ തന്നെയാണ് അക്രമികള്‍.
\end{malayalam}}
\flushright{\begin{Arabic}
\quranayah[9][24]
\end{Arabic}}
\flushleft{\begin{malayalam}
(നബിയേ,) പറയുക: നിങ്ങളുടെ പിതാക്കളും, നിങ്ങളുടെ പുത്രന്‍മാരും, നിങ്ങളുടെ സഹോദരങ്ങളും, നിങ്ങളുടെ ഇണകളും, നിങ്ങളുടെ ബന്ധുക്കളും, നിങ്ങള്‍ സമ്പാദിച്ചുണ്ടാക്കിയ സ്വത്തുക്കളും, മാന്ദ്യം നേരിടുമെന്ന് നിങ്ങള്‍ ഭയപ്പെടുന്ന കച്ചവടവും, നിങ്ങള്‍ തൃപ്തിപ്പെടുന്ന പാര്‍പ്പിടങ്ങളും നിങ്ങള്‍ക്ക് അല്ലാഹുവെക്കാളും അവന്‍റെ ദൂതനെക്കാളും അവന്‍റെ മാര്‍ഗത്തിലുള്ള സമരത്തെക്കാളും പ്രിയപ്പെട്ടതായിരുന്നാല്‍ അല്ലാഹു അവന്‍റെ കല്‍പന കൊണ്ടുവരുന്നത് വരെ നിങ്ങള്‍ കാത്തിരിക്കുക. അല്ലാഹു ധിക്കാരികളായ ജനങ്ങളെ നേര്‍വഴിയിലാക്കുന്നതല്ല.
\end{malayalam}}
\flushright{\begin{Arabic}
\quranayah[9][25]
\end{Arabic}}
\flushleft{\begin{malayalam}
തീര്‍ച്ചയായും ധാരാളം (യുദ്ധ) രംഗങ്ങളില്‍ അല്ലാഹു നിങ്ങളെ സഹായിച്ചിട്ടുണ്ട്‌. ഹുനൈന്‍ (യുദ്ധ) ദിവസത്തിലും (സഹായിച്ചു.) അതായത് നിങ്ങളുടെ എണ്ണപ്പെരുപ്പം നിങ്ങളെ ആഹ്ലാദം കൊള്ളിക്കുകയും എന്നാല്‍ അത് നിങ്ങള്‍ക്ക് യാതൊരു പ്രയോജനവും ഉണ്ടാക്കാതിരിക്കുകയും, ഭൂമിവിശാലമായിട്ടും നിങ്ങള്‍ക്ക് ഇടുങ്ങിയതാവുകയും, അനന്തരം നിങ്ങള്‍ പിന്തിരിഞ്ഞോടുകയും ചെയ്ത സന്ദര്‍ഭം.
\end{malayalam}}
\flushright{\begin{Arabic}
\quranayah[9][26]
\end{Arabic}}
\flushleft{\begin{malayalam}
പിന്നീട് അല്ലാഹു അവന്‍റെ ദൂതന്നും സത്യവിശ്വാസികള്‍ക്കും അവന്‍റെ പക്കല്‍ നിന്നുള്ള മനസ്സമാധാനം ഇറക്കികൊടുക്കുകയും, നിങ്ങള്‍ കാണാത്ത ചില സൈന്യങ്ങളെ ഇറക്കുകയും, സത്യനിഷേധികളെ അവന്‍ ശിക്ഷിക്കുകയും ചെയ്തു. അതത്രെ സത്യനിഷേധികള്‍ക്കുള്ള പ്രതിഫലം.
\end{malayalam}}
\flushright{\begin{Arabic}
\quranayah[9][27]
\end{Arabic}}
\flushleft{\begin{malayalam}
പിന്നീട് അതിന് ശേഷം താന്‍ ഉദ്ദേശിക്കുന്നവരുടെ പശ്ചാത്താപം അല്ലാഹു സ്വീകരിക്കുന്നതാണ്‌. അല്ലാഹു ഏറെ പൊറുക്കുന്നവനും കരുണാനിധിയുമാകുന്നു.
\end{malayalam}}
\flushright{\begin{Arabic}
\quranayah[9][28]
\end{Arabic}}
\flushleft{\begin{malayalam}
സത്യവിശ്വാസികളേ, ബഹുദൈവവിശ്വാസികള്‍ അശുദ്ധര്‍ തന്നെയാകുന്നു. അതിനാല്‍ അവര്‍ ഈ കൊല്ലത്തിന് ശേഷം മസ്ജിദുല്‍ ഹറാമിനെ സമീപിക്കരുത്‌. (അവരുടെ അഭാവത്താല്‍) ദാരിദ്ര്യം നേരിടുമെന്ന് നിങ്ങള്‍ ഭയപ്പെടുകയാണെങ്കില്‍ അല്ലാഹു അവന്‍റെ അനുഗ്രഹത്താല്‍ അവന്‍ ഉദ്ദേശിക്കുന്ന പക്ഷം നിങ്ങള്‍ക്ക് ഐശ്വര്യം വരുത്തുന്നതാണ്‌. തീര്‍ച്ചയായും അല്ലാഹു എല്ലാം അറിയുന്നവനും യുക്തിമാനുമാണ്‌.
\end{malayalam}}
\flushright{\begin{Arabic}
\quranayah[9][29]
\end{Arabic}}
\flushleft{\begin{malayalam}
വേദം നല്‍കപ്പെട്ടവരുടെ കൂട്ടത്തില്‍ അല്ലാഹുവിലും അന്ത്യദിനത്തിലും വിശ്വസിക്കാതിരിക്കുകയും, അല്ലാഹുവും അവന്‍റെ ദൂതനും നിഷിദ്ധമാക്കിയത് നിഷിദ്ധമായി ഗണിക്കാതിരിക്കുകയും, സത്യമതത്തെ മതമായി സ്വീകരിക്കാതിരിക്കുകയും ചെയ്യുന്നവരോട് നിങ്ങള്‍ യുദ്ധം ചെയ്ത് കൊള്ളുക. അവര്‍ കീഴടങ്ങിക്കൊണ്ട് കയ്യോടെ കപ്പം കൊടുക്കുന്നത് വരെ.
\end{malayalam}}
\flushright{\begin{Arabic}
\quranayah[9][30]
\end{Arabic}}
\flushleft{\begin{malayalam}
ഉസൈര്‍ (എസ്രാ പ്രവാചകന്‍) ദൈവപുത്രനാണെന്ന് യഹൂദന്‍മാര്‍ പറഞ്ഞു. മസീഹ് (മിശിഹാ) ദൈവപുത്രനാണെന്ന് ക്രിസ്ത്യാനികളും പറഞ്ഞു. അതവരുടെ വായ കൊണ്ടുള്ള വാക്ക് മാത്രമാണ്‌. മുമ്പ് അവിശ്വസിച്ചവരുടെ വാക്കിനെ അവര്‍ അനുകരിക്കുകയാകുന്നു. അല്ലാഹു അവരെ ശപിച്ചിരിക്കുന്നു എങ്ങനെയാണവര്‍ തെറ്റിക്കപ്പെടുന്നത്‌?
\end{malayalam}}
\flushright{\begin{Arabic}
\quranayah[9][31]
\end{Arabic}}
\flushleft{\begin{malayalam}
അവരുടെ പണ്ഡിതന്‍മാരെയും പുരോഹിതന്‍മാരെയും മര്‍യമിന്‍റെ മകനായ മസീഹിനെയും അല്ലാഹുവിന് പുറമെ അവര്‍ രക്ഷിതാക്കളായി സ്വീകരിച്ചു. എന്നാല്‍ ഏകദൈവത്തെ ആരാധിക്കാന്‍ മാത്രമായിരുന്നു അവര്‍ കല്‍പിക്കപ്പെട്ടിരുന്നത്‌. അവനല്ലാതെ യാതൊരു ദൈവവുമില്ല. അവര്‍ പങ്കുചേര്‍ക്കുന്നതില്‍ നിന്ന് അവനെത്രയോ പരിശുദ്ധന്‍!
\end{malayalam}}
\flushright{\begin{Arabic}
\quranayah[9][32]
\end{Arabic}}
\flushleft{\begin{malayalam}
അവരുടെ വായ്കൊണ്ട് അല്ലാഹുവിന്‍റെ പ്രകാശം കെടുത്തിക്കളയാമെന്ന് അവര്‍ ആഗ്രഹിക്കുന്നു. അല്ലാഹുവാകട്ടെ തന്‍റെ പ്രകാശം പൂര്‍ണ്ണമാക്കാതെ സമ്മതിക്കുകയില്ല. സത്യനിഷേധികള്‍ക്ക് അത് അനിഷ്ടകരമായാലും.
\end{malayalam}}
\flushright{\begin{Arabic}
\quranayah[9][33]
\end{Arabic}}
\flushleft{\begin{malayalam}
അവനാണ് സന്‍മാര്‍ഗവും സത്യമതവുമായി തന്‍റെ ദൂതനെ അയച്ചവന്‍. എല്ലാ മതത്തെയും അത് അതിജയിക്കുന്നതാക്കാന്‍ വേണ്ടി. ബഹുദൈവവിശ്വാസികള്‍ക്ക് അത് അനിഷ്ടകരമായാലും.
\end{malayalam}}
\flushright{\begin{Arabic}
\quranayah[9][34]
\end{Arabic}}
\flushleft{\begin{malayalam}
സത്യവിശ്വാസികളേ, പണ്ഡിതന്‍മാരിലും പുരോഹിതന്‍മാരിലും പെട്ട ധാരാളം പേര്‍ ജനങ്ങളുടെ ധനം അന്യായമായി തിന്നുകയും, അല്ലാഹുവിന്‍റെ മാര്‍ഗത്തില്‍ നിന്ന് (അവരെ) തടയുകയും ചെയ്യുന്നു. സ്വര്‍ണവും വെള്ളിയും നിക്ഷേപമാക്കിവെക്കുകയും, അല്ലാഹുവിന്‍റെ മാര്‍ഗത്തില്‍ അത് ചെലവഴിക്കാതിരിക്കുകയും ചെയ്യുന്നവരാരോ അവര്‍ക്ക് വേദനയേറിയ ശിക്ഷയെപ്പറ്റി സന്തോഷവാര്‍ത്ത അറിയിക്കുക.
\end{malayalam}}
\flushright{\begin{Arabic}
\quranayah[9][35]
\end{Arabic}}
\flushleft{\begin{malayalam}
നരകാഗ്നിയില്‍ വെച്ച് അവ ചുട്ടുപഴുപ്പിക്കപ്പെടുകയും, എന്നിട്ടത് കൊണ്ട് അവരുടെ നെറ്റികളിലും പാര്‍ശ്വങ്ങളിലും മുതുകുകളിലും ചൂടുവെക്കപ്പെടുകയും ചെയ്യുന്ന ദിവസം (അവരോട് പറയപ്പെടും) : നിങ്ങള്‍ നിങ്ങള്‍ക്ക് വേണ്ടി തന്നെ നിക്ഷേപിച്ചുവെച്ചതാണിത്‌. അതിനാല്‍ നിങ്ങള്‍ നിക്ഷേപിച്ച് വെച്ചിരുന്നത് നിങ്ങള്‍ ആസ്വദിച്ച് കൊള്ളുക.
\end{malayalam}}
\flushright{\begin{Arabic}
\quranayah[9][36]
\end{Arabic}}
\flushleft{\begin{malayalam}
ആകാശങ്ങളും ഭൂമിയും സൃഷ്ടിച്ച ദിവസം അല്ലാഹു രേഖപ്പെടുത്തിയതനുസരിച്ച് അല്ലാഹുവിന്‍റെ അടുക്കല്‍ മാസങ്ങളുടെ എണ്ണം പന്ത്രണ്ടാകുന്നു. അവയില്‍ നാലെണ്ണം (യുദ്ധം) വിലക്കപ്പെട്ടമാസങ്ങളാകുന്നു. അതാണ് വക്രതയില്ലാത്ത മതം. അതിനാല്‍ ആ (നാല്‌) മാസങ്ങളില്‍ നിങ്ങള്‍ നിങ്ങളോട് തന്നെ അക്രമം പ്രവര്‍ത്തിക്കരുത്‌. ബഹുദൈവവിശ്വാസികള്‍ നിങ്ങളോട് ആകമാനം യുദ്ധം ചെയ്യുന്നത് പോലെ നിങ്ങള്‍ അവരോടും ആകമാനം യുദ്ധം ചെയ്യുക. അല്ലാഹു സൂക്ഷ്മത പാലിക്കുന്നവരുടെ കൂടെയാണെന്ന് നിങ്ങള്‍ മനസ്സിലാക്കുകയും ചെയ്യുക.
\end{malayalam}}
\flushright{\begin{Arabic}
\quranayah[9][37]
\end{Arabic}}
\flushleft{\begin{malayalam}
വിലക്കപ്പെട്ടമാസം പുറകോട്ട് മാറ്റുക എന്നത് സത്യനിഷേധത്തിന്‍റെ വര്‍ദ്ധനവ് തന്നെയാകുന്നു. സത്യനിഷേധികള്‍ അത് മൂലം തെറ്റിലേക്ക് നയിക്കപ്പെടുന്നു. ഒരു കൊല്ലം അവരത് അനുവദനീയമാക്കുകയും മറ്റൊരു കൊല്ലം നിഷിദ്ധമാക്കുകയും ചെയ്യുന്നു. അല്ലാഹു നിഷിദ്ധമാക്കിയതിന്‍റെ (മാസത്തിന്‍റെ) എണ്ണമൊപ്പിക്കുവാനും എന്നിട്ട്‌, അല്ലാഹു നിഷിദ്ധമാക്കിയത് ഏതോ അത് അനുവദനീയമാക്കുവാനും വേണ്ടിയാണ് അവരങ്ങനെ ചെയ്യുന്നത്‌. അവരുടെ ദുഷ്പ്രവൃത്തികള്‍ അവര്‍ക്ക് ഭംഗിയായി തോന്നിക്കപ്പെട്ടിരിക്കുന്നു. സത്യനിഷേധികളായ ജനങ്ങളെ അല്ലാഹു നേര്‍വഴിയിലാക്കുകയില്ല.
\end{malayalam}}
\flushright{\begin{Arabic}
\quranayah[9][38]
\end{Arabic}}
\flushleft{\begin{malayalam}
സത്യവിശ്വാസികളേ, നിങ്ങള്‍ക്കെന്തുപറ്റി ? അല്ലാഹുവിന്‍റെ മാര്‍ഗത്തില്‍ (ധര്‍മ്മസമരത്തിന്ന്‌) നിങ്ങള്‍ ഇറങ്ങിപ്പുറപ്പെട്ട് കൊള്ളുക. എന്ന് നിങ്ങളോട് പറയപ്പെട്ടാല്‍ നിങ്ങള്‍ ഭൂമിയിലേക്ക് തൂങ്ങിക്കളയുന്നു! പരലോകത്തിന് പകരം ഇഹലോകജീവിതം കൊണ്ട് നിങ്ങള്‍ തൃപ്തിപ്പെട്ടിരിക്കുകയാണോ ? എന്നാല്‍ പരലോകത്തിന്‍റെ മുമ്പില്‍ ഇഹലോകത്തിലെ സുഖാനുഭവം തുച്ഛം മാത്രമാകുന്നു.
\end{malayalam}}
\flushright{\begin{Arabic}
\quranayah[9][39]
\end{Arabic}}
\flushleft{\begin{malayalam}
നിങ്ങള്‍ (യുദ്ധത്തിന്നു) ഇറങ്ങിപ്പുറപ്പെടുന്നില്ലെങ്കില്‍ അല്ലാഹു നിങ്ങള്‍ക്ക് വേദനയേറിയ ശിക്ഷ നല്‍കുകയും, നിങ്ങളല്ലാത്ത വല്ലജനതയെയും അവന്‍ പകരം കൊണ്ടുവരികയും ചെയ്യും. അവന്ന് ഒരു ഉപദ്രവവും ചെയ്യാന്‍ നിങ്ങള്‍ക്കാവില്ല. അല്ലാഹു ഏത് കാര്യത്തിനും കഴിവുള്ളവനാകുന്നു.
\end{malayalam}}
\flushright{\begin{Arabic}
\quranayah[9][40]
\end{Arabic}}
\flushleft{\begin{malayalam}
നിങ്ങള്‍ അദ്ദേഹത്തെ സഹായിക്കുന്നില്ലെങ്കില്‍; സത്യനിഷേധികള്‍ അദ്ദേഹത്തെ പുറത്താക്കുകയും, അദ്ദേഹം രണ്ടുപേരില്‍ ഒരാള്‍ ആയിരിക്കുകയും ചെയ്ത സന്ദര്‍ഭത്തില്‍ അഥവാ അവര്‍ രണ്ടുപേരും (നബിയും അബൂബക്കറും) ആ ഗുഹയിലായിരുന്നപ്പോള്‍ അല്ലാഹു അദ്ദേഹത്തെ സഹായിച്ചിട്ടുണ്ട്‌. അദ്ദേഹം തന്‍റെ കൂട്ടുകാരനോട്‌, ദുഃഖിക്കേണ്ട. തീര്‍ച്ചയായും അല്ലാഹു നമ്മുടെ കൂടെയുണ്ട് എന്ന് പറയുന്ന സന്ദര്‍ഭം. അപ്പോള്‍ അല്ലാഹു തന്‍റെ വകയായുള്ള സമാധാനം അദ്ദേഹത്തിന് ഇറക്കികൊടുക്കുകയും, നിങ്ങള്‍ കാണാത്ത സൈന്യങ്ങളെക്കൊണ്ട് അദ്ദേഹത്തിന് പിന്‍ബലം നല്‍കുകയും, സത്യനിഷേധികളുടെ വാക്കിനെ അവന്‍ അങ്ങേയറ്റം താഴ്ത്തിക്കളയുകയും ചെയ്തു. അല്ലാഹുവിന്‍റെ വാക്കാണ് ഏറ്റവും ഉയര്‍ന്ന് നില്‍ക്കുന്നത്‌. അല്ലാഹു പ്രതാപിയും യുക്തിമാനുമാകുന്നു.
\end{malayalam}}
\flushright{\begin{Arabic}
\quranayah[9][41]
\end{Arabic}}
\flushleft{\begin{malayalam}
നിങ്ങള്‍ സൌകര്യമുള്ളവരാണെങ്കിലും ഞെരുക്കമുള്ളവരാണെങ്കിലും (ധര്‍മ്മസമരത്തിന്‌) ഇറങ്ങിപുറപ്പെട്ട് കൊള്ളുക. നിങ്ങളുടെ സ്വത്തുക്കള്‍ കൊണ്ടും ശരീരങ്ങള്‍ കൊണ്ടും അല്ലാഹുവിന്‍റെ മാര്‍ഗത്തില്‍ നിങ്ങള്‍ സമരം ചെയ്യുക. അതാണ് നിങ്ങള്‍ക്ക് ഉത്തമം. നിങ്ങള്‍ മനസ്സിലാക്കുന്നുണ്ടെങ്കില്‍.
\end{malayalam}}
\flushright{\begin{Arabic}
\quranayah[9][42]
\end{Arabic}}
\flushleft{\begin{malayalam}
അടുത്തു തന്നെയുള്ള ഒരു നേട്ടവും വിഷമകരമല്ലാത്ത യാത്രയുമായിരുന്നെങ്കില്‍ അവര്‍ നിന്നെ പിന്തുടരുമായിരുന്നു. പക്ഷെ, വിഷമകരമായ ഒരു യാത്രാലക്ഷ്യം അവര്‍ക്ക് വിദൂരമായി തോന്നിയിരിക്കുന്നു. ഞങ്ങള്‍ക്ക് സാധിച്ചിരുന്നെങ്കില്‍ ഞങ്ങള്‍ നിങ്ങളുടെ കൂടെ പുറപ്പെടുമായിരുന്നു. എന്ന് അവര്‍ അല്ലാഹുവിന്‍റെ പേരില്‍ സത്യം ചെയ്ത് പറഞ്ഞേക്കും. അവര്‍ അവര്‍ക്കുതന്നെ നാശമുണ്ടാക്കുകയാകുന്നു. തീര്‍ച്ചയായും അവര്‍ കള്ളം പറയുന്നവരാണെന്ന് അല്ലാഹുവിന്നറിയാം.
\end{malayalam}}
\flushright{\begin{Arabic}
\quranayah[9][43]
\end{Arabic}}
\flushleft{\begin{malayalam}
(നബിയേ,) നിനക്ക് അല്ലാഹു മാപ്പുനല്‍കിയിക്കുന്നു. സത്യം പറഞ്ഞവര്‍ ആരെന്ന് നിനക്ക് വ്യക്തമായി ബോധ്യപ്പെടുകയും കള്ളം പറയുന്നവരെ നിനക്ക് തിരിച്ചറിയുകയും ചെയ്യുന്നത് വരെ നീ എന്തിനാണ് അവര്‍ക്ക് അനുവാദം നല്‍കിയത്‌?
\end{malayalam}}
\flushright{\begin{Arabic}
\quranayah[9][44]
\end{Arabic}}
\flushleft{\begin{malayalam}
അല്ലാഹുവിലും അന്ത്യദിനത്തിലും വിശ്വസിക്കുന്നവരാരോ അവര്‍ തങ്ങളുടെ സ്വത്തുക്കള്‍കൊണ്ടും ശരീരങ്ങള്‍കൊണ്ടും സമരം ചെയ്യുന്നതില്‍ നിന്ന് ഒഴിഞ്ഞുനില്‍ക്കാന്‍ നിന്നോട് അനുവാദം ചോദിക്കുകയില്ല. സൂക്ഷ്മത പാലിക്കുന്നവരെപ്പറ്റി അല്ലാഹു നല്ലവണ്ണം അറിയുന്നവനാണ്‌.
\end{malayalam}}
\flushright{\begin{Arabic}
\quranayah[9][45]
\end{Arabic}}
\flushleft{\begin{malayalam}
അല്ലാഹുവിലും അന്ത്യദിനത്തിലും വിശ്വസിക്കാതിരിക്കുകയും, മനസ്സുകളില്‍ സംശയം കുടികൊള്ളുകയും ചെയ്യുന്നവര്‍ മാത്രമാണ് നിന്നോട് അനുവാദം ചോദിക്കുന്നത്‌. കാരണം അവര്‍ അവരുടെ സംശയത്തില്‍ ആടിക്കളിച്ച് കൊണ്ടിരിക്കുകയാണ്‌.
\end{malayalam}}
\flushright{\begin{Arabic}
\quranayah[9][46]
\end{Arabic}}
\flushleft{\begin{malayalam}
അവര്‍ പുറപ്പെടാന്‍ ഉദ്ദേശിച്ചിരുന്നെങ്കില്‍ അതിനുവേണ്ടി ഒരുക്കേണ്ടതെല്ലാം അവര്‍ ഒരുക്കുമായിരുന്നു. പക്ഷെ അവരുടെ പുറപ്പാട് അല്ലാഹു ഇഷ്ടപെടാതിരുന്നതുകൊണ്ട് അവരെ പിന്തിരിപ്പിച്ചു നിര്‍ത്തിയിരിക്കുകയാണ്‌. മുടങ്ങിയിരിക്കുന്നവരോടൊപ്പം നിങ്ങളും ഇരുന്നുകൊള്ളുക എന്ന് അവരോട് പറയപ്പെടുകയും ചെയ്തിരിക്കുന്നു.
\end{malayalam}}
\flushright{\begin{Arabic}
\quranayah[9][47]
\end{Arabic}}
\flushleft{\begin{malayalam}
നിങ്ങളുടെ കൂട്ടത്തില്‍ അവര്‍ പുറപ്പെട്ടിരുന്നെങ്കില്‍ നാശമല്ലാതെ മറ്റൊന്നും അവര്‍ നിങ്ങള്‍ക്ക് കൂടുതല്‍ നേടിത്തരുമായിരുന്നില്ല. നിങ്ങള്‍ക്ക് കുഴപ്പം വരുത്താന്‍ ആഗ്രഹിച്ചുകൊണ്ട് നിങ്ങളുടെ ഇടയിലൂടെ അവര്‍ പരക്കംപായുകയും ചെയ്യുമായിരുന്നു. നിങ്ങളുടെ കൂട്ടത്തില്‍ അവര്‍ പറയുന്നത് ചെവികൊടുത്ത് കേള്‍ക്കുന്ന ചിലരുണ്ട് താനും. അല്ലാഹു അക്രമികളെപ്പറ്റി നന്നായി അറിയുന്നവനാണ്‌.
\end{malayalam}}
\flushright{\begin{Arabic}
\quranayah[9][48]
\end{Arabic}}
\flushleft{\begin{malayalam}
മുമ്പും അവര്‍ കുഴപ്പമുണ്ടാക്കാന്‍ ആഗ്രഹിക്കുകയും നിനക്കെതിരില്‍ അവര്‍ കാര്യങ്ങള്‍ കുഴച്ചു മറിക്കുകയും ചെയ്തിട്ടുണ്ട്‌. അവസാനം അവര്‍ക്ക് ഇഷ്ടമില്ലാതിരുന്നിട്ടും സത്യം വന്നെത്തുകയും അല്ലാഹുവിന്‍റെ കാര്യം വിജയിക്കുകയും ചെയ്തു.
\end{malayalam}}
\flushright{\begin{Arabic}
\quranayah[9][49]
\end{Arabic}}
\flushleft{\begin{malayalam}
എനിക്ക് (യുദ്ധത്തിന് പോകാതിരിക്കാന്‍) സമ്മതം തരണേ, എന്നെ കുഴപ്പത്തിലാക്കരുതേ എന്ന് പറയുന്ന ചില ആളുകളും അവരുടെ കൂട്ടത്തിലുണ്ട്‌. അറിയുക: അവര്‍ കുഴപ്പത്തില്‍ തന്നെയാണ് വീണിരിക്കുന്നത്‌. തീര്‍ച്ചയായും നരകം സത്യനിഷേധികളെ വലയം ചെയ്യുന്നതാകുന്നു.
\end{malayalam}}
\flushright{\begin{Arabic}
\quranayah[9][50]
\end{Arabic}}
\flushleft{\begin{malayalam}
നിനക്ക് വല്ല നന്‍മയും വന്നെത്തുന്ന പക്ഷം അതവരെ ദുഃഖിതരാക്കുകയും നിനക്ക് വല്ല ആപത്തും വന്നെത്തുന്ന പക്ഷം ഞങ്ങള്‍ ഞങ്ങളുടെ കാര്യം മുമ്പുതന്നെ സൂക്ഷിച്ചിട്ടുണ്ട് എന്ന് അവര്‍ പറയുകയും ആഹ്ലാദിച്ചു കൊണ്ട് അവര്‍ പിന്തിരിഞ്ഞ് പോകുകയും ചെയ്യും.
\end{malayalam}}
\flushright{\begin{Arabic}
\quranayah[9][51]
\end{Arabic}}
\flushleft{\begin{malayalam}
പറയുക: അല്ലാഹു ഞങ്ങള്‍ക്ക് രേഖപ്പെടുത്തിയതല്ലാതെ ഞങ്ങള്‍ക്കൊരിക്കലും ബാധിക്കുകയില്ല. അവനാണ് ഞങ്ങളുടെ യജമാനന്‍. അല്ലാഹുവിന്‍റെ മേലാണ് സത്യവിശ്വാസികള്‍ ഭരമേല്‍പിക്കേണ്ടത്‌.
\end{malayalam}}
\flushright{\begin{Arabic}
\quranayah[9][52]
\end{Arabic}}
\flushleft{\begin{malayalam}
പറയുക: (രക്തസാക്ഷിത്വം, വിജയം എന്നീ) രണ്ടു നല്ലകാര്യങ്ങളില്‍ ഏതെങ്കിലും ഒന്നല്ലാതെ ഞങ്ങളുടെ കാര്യത്തില്‍ നിങ്ങള്‍ പ്രതീക്ഷിക്കുന്നുണ്ടോ? എന്നാല്‍ നിങ്ങളുടെ കാര്യത്തില്‍ ഞങ്ങള്‍ പ്രതീക്ഷിക്കുന്നത് നിങ്ങള്‍ക്ക് അല്ലാഹു തന്‍റെ പക്കല്‍ നിന്ന് നേരിട്ടോ, ഞങ്ങളുടെ കൈക്കോ ശിക്ഷ ഏല്‍പിക്കും എന്നാണ്‌. അതിനാല്‍ നിങ്ങള്‍ പ്രതീക്ഷിച്ചു കൊള്ളുക. ഞങ്ങളും നിങ്ങളോടൊപ്പം പ്രതീക്ഷിച്ചിരിക്കുന്നവരാണ്‌.
\end{malayalam}}
\flushright{\begin{Arabic}
\quranayah[9][53]
\end{Arabic}}
\flushleft{\begin{malayalam}
പറയുക: നിങ്ങള്‍ അനുസരണത്തോടെയോ വെറുപ്പോടെയോ ചെലവഴിച്ച് കൊള്ളുക. (എങ്ങനെയായാലും) നിങ്ങളുടെ പക്കല്‍ നിന്നത് സ്വീകരിക്കപ്പെടുന്നതേയല്ല. തീര്‍ച്ചയായും നിങ്ങള്‍ ധിക്കാരികളായ ഒരു ജനവിഭാഗമായിരിക്കുന്നു.
\end{malayalam}}
\flushright{\begin{Arabic}
\quranayah[9][54]
\end{Arabic}}
\flushleft{\begin{malayalam}
അവര്‍ അല്ലാഹുവിലും അവന്‍റെ ദൂതനിലും അവിശ്വസിച്ചിരിക്കുന്നു എന്നതും, മടിയന്‍മാരായിക്കൊണ്ടല്ലാതെ അവര്‍ നമസ്കാരത്തിന് ചെല്ലുകയില്ല എന്നതും, വെറുപ്പുള്ളവരായിക്കൊണ്ടല്ലാതെ അവര്‍ ചെലവഴിക്കുകയില്ല എന്നതും മാത്രമാണ് അവരുടെ പക്കല്‍ നിന്ന് അവരുടെ ദാനങ്ങള്‍ സ്വീകരിക്കപ്പെടുന്നതിന് തടസ്സമായിട്ടുള്ളത്‌.
\end{malayalam}}
\flushright{\begin{Arabic}
\quranayah[9][55]
\end{Arabic}}
\flushleft{\begin{malayalam}
അവരുടെ സ്വത്തുക്കളും സന്താനങ്ങളും നിന്നെ ആശ്ചര്യപ്പെടുത്താതിരിക്കട്ടെ! അവ മുഖേന ഇഹലോകജീവിതത്തില്‍ അവരെ ശിക്ഷിക്കണമെന്നും, സത്യനിഷേധികളായിരിക്കെതന്നെ അവര്‍ ജീവനാശമടയണമെന്നും മാത്രമാണ് അല്ലാഹു ഉദ്ദേശിക്കുന്നത്‌.
\end{malayalam}}
\flushright{\begin{Arabic}
\quranayah[9][56]
\end{Arabic}}
\flushleft{\begin{malayalam}
തീര്‍ച്ചയായും അവര്‍ നിങ്ങളുടെ കൂട്ടത്തില്‍ പെട്ടവരാണെന്ന് അവര്‍ അല്ലാഹുവിന്‍റെ പേരില്‍ സത്യം ചെയ്തു പറയും. എന്നാല്‍ അവര്‍ നിങ്ങളുടെ കൂട്ടത്തില്‍ പെട്ടവരല്ല. പക്ഷെ അവര്‍ പേടിച്ചു കഴിയുന്ന ഒരു ജനവിഭാഗമാകുന്നു.
\end{malayalam}}
\flushright{\begin{Arabic}
\quranayah[9][57]
\end{Arabic}}
\flushleft{\begin{malayalam}
ഏതെങ്കിലും അഭയസ്ഥാനമോ, ഗുഹകളോ, കടന്ന് കൂടാന്‍ പറ്റിയ ഏതെങ്കിലും സ്ഥലമോ അവര്‍ കണ്ടെത്തുകയാണെങ്കില്‍ കുതറിച്ചാടിക്കൊണ്ട് അവരങ്ങോട്ട് തിരിഞ്ഞുപോകുന്നതാണ്‌.
\end{malayalam}}
\flushright{\begin{Arabic}
\quranayah[9][58]
\end{Arabic}}
\flushleft{\begin{malayalam}
അവരുടെ കൂട്ടത്തില്‍ ദാനധര്‍മ്മങ്ങളുടെ കാര്യത്തില്‍ നിന്നെ ആക്ഷേപിക്കുന്ന ചിലരുണ്ട്‌. അതില്‍ നിന്ന് അവര്‍ക്ക് നല്‍കപ്പെടുന്ന പക്ഷം അവര്‍ തൃപ്തിപ്പെടും. അവര്‍ക്കതില്‍ നിന്ന് നല്‍കപ്പെട്ടില്ലെങ്കിലോ അവരതാ കോപിക്കുന്നു.
\end{malayalam}}
\flushright{\begin{Arabic}
\quranayah[9][59]
\end{Arabic}}
\flushleft{\begin{malayalam}
അല്ലാഹുവും അവന്‍റെ റസൂലും കൊടുത്തതില്‍ അവര്‍ തൃപ്തിയടയുകയും, ഞങ്ങള്‍ക്ക് അല്ലാഹു മതി, അല്ലാഹുവിന്‍റെ അനുഗ്രഹത്തില്‍ നിന്ന് അവനും അവന്‍റെ റസൂലും ഞങ്ങള്‍ക്ക് തന്നുകൊള്ളും. തീര്‍ച്ചയായും ഞങ്ങള്‍ അല്ലാഹുവിങ്കലേക്കാണ് ആഗ്രഹങ്ങള്‍ തിരിക്കുന്നത്‌. എന്ന് അവര്‍ പറയുകയും ചെയ്തിരുന്നെങ്കില്‍ (എത്ര നന്നായിരുന്നേനെ!)
\end{malayalam}}
\flushright{\begin{Arabic}
\quranayah[9][60]
\end{Arabic}}
\flushleft{\begin{malayalam}
ദാനധര്‍മ്മങ്ങള്‍ (നല്‍കേണ്ടത്‌) ദരിദ്രന്‍മാര്‍ക്കും, അഗതികള്‍ക്കും, അതിന്‍റെ കാര്യത്തില്‍ പ്രവര്‍ത്തിക്കുന്നവര്‍ക്കും (ഇസ്ലാമുമായി) മനസ്സുകള്‍ ഇണക്കപ്പെട്ടവര്‍ക്കും, അടിമകളുടെ (മോചനത്തിന്‍റെ) കാര്യത്തിലും, കടം കൊണ്ട് വിഷമിക്കുന്നവര്‍ക്കും, അല്ലാഹുവിന്‍റെ മാര്‍ഗത്തിലും, വഴിപോക്കന്നും മാത്രമാണ്‌. അല്ലാഹുവിങ്കല്‍ നിന്ന് നിശ്ചയിക്കപ്പെട്ടതത്രെ ഇത്‌. അല്ലാഹു എല്ലാം അറിയുന്നവനും യുക്തിമാനുമാണ്‌.
\end{malayalam}}
\flushright{\begin{Arabic}
\quranayah[9][61]
\end{Arabic}}
\flushleft{\begin{malayalam}
നബിയെ ദ്രോഹിക്കുകയും അദ്ദേഹം എല്ലാം ചെവിക്കൊള്ളുന്ന ആളാണ് എന്ന് പറയുകയും ചെയ്യുന്ന ചിലര്‍ അവരുടെ കൂട്ടത്തിലുണ്ട്‌. പറയുക: അദ്ദേഹം നിങ്ങള്‍ക്ക് ഗുണമുള്ളത് ചെവിക്കൊള്ളുന്ന ആളാകുന്നു. അദ്ദേഹം അല്ലാഹുവില്‍ വിശ്വസിക്കുന്നു. യഥാര്‍ത്ഥവിശ്വാസികളെയും അദ്ദേഹം വിശ്വസിക്കുന്നു. നിങ്ങളില്‍ നിന്ന് വിശ്വസിച്ചവര്‍ക്ക് ഒരു അനുഗ്രഹവുമാണദ്ദേഹം. അല്ലാഹുവിന്‍റെ ദൂതനെ ദ്രോഹിക്കുന്നവരാരോ അവര്‍ക്കാണ് വേദനയേറിയ ശിക്ഷയുള്ളത്‌.
\end{malayalam}}
\flushright{\begin{Arabic}
\quranayah[9][62]
\end{Arabic}}
\flushleft{\begin{malayalam}
നിങ്ങളെ തൃപ്തിപ്പെടുത്താന്‍ വേണ്ടി നിങ്ങളോടവര്‍ അല്ലാഹുവിന്‍റെ പേരില്‍ സത്യം ചെയ്ത് സംസാരിക്കുന്നു. എന്നാല്‍ അവര്‍ സത്യവിശ്വാസികളാണെങ്കില്‍ അവര്‍ തൃപ്തിപ്പെടുത്തുവാന്‍ ഏറ്റവും അവകാശപ്പെട്ടവര്‍ അല്ലാഹുവും അവന്‍റെ ദൂതനുമാണ്‌.
\end{malayalam}}
\flushright{\begin{Arabic}
\quranayah[9][63]
\end{Arabic}}
\flushleft{\begin{malayalam}
വല്ലവനും അല്ലാഹുവോടും അവന്‍റെ ദൂതനോടും എതിര്‍ത്ത് നില്‍ക്കുന്ന പക്ഷം അവന്ന് നരകാഗ്നിയാണുണ്ടായിരിക്കുക എന്നും, അവനതില്‍ നിത്യവാസിയായിരിക്കുമെന്നും അവര്‍ മനസ്സിലാക്കിയിട്ടില്ലേ? അതാണ് വമ്പിച്ച അപമാനം.
\end{malayalam}}
\flushright{\begin{Arabic}
\quranayah[9][64]
\end{Arabic}}
\flushleft{\begin{malayalam}
തങ്ങളുടെ മനസ്സുകളില്‍ ഉള്ളതിനെപ്പറ്റി അവരെ വിവരമറിയിക്കുന്ന (ഖുര്‍ആനില്‍ നിന്നുള്ള) ഏതെങ്കിലും ഒരു അദ്ധ്യായം അവരുടെ കാര്യത്തില്‍ അവതരിപ്പിക്കപ്പെടുമോ എന്ന് കപടവിശ്വാസികള്‍ ഭയപ്പെട്ടുകൊണ്ടിരിക്കുന്നു. പറയുക: നിങ്ങള്‍ പരിഹസിച്ചു കൊള്ളൂ. തീര്‍ച്ചയായും നിങ്ങള്‍ ഭയപ്പെട്ടുകൊണ്ടിരിക്കുന്നത് അല്ലാഹു വെളിയില്‍ കൊണ്ടു വരുന്നതാണ്‌.
\end{malayalam}}
\flushright{\begin{Arabic}
\quranayah[9][65]
\end{Arabic}}
\flushleft{\begin{malayalam}
നീ അവരോട് (അതിനെപ്പറ്റി) ചോദിച്ചാല്‍ അവര്‍ പറയും: ഞങ്ങള്‍ തമാശ പറഞ്ഞു കളിക്കുക മാത്രമായിരുന്നു. പറയുക: അല്ലാഹുവെയും അവന്‍റെ ദൃഷ്ടാന്തങ്ങളെയും അവന്‍റെ ദൂതനെയുമാണോ നിങ്ങള്‍ പരിഹസിച്ചു കൊണ്ടിരിക്കുന്നത്‌?
\end{malayalam}}
\flushright{\begin{Arabic}
\quranayah[9][66]
\end{Arabic}}
\flushleft{\begin{malayalam}
നിങ്ങള്‍ ഒഴികഴിവുകളൊന്നും പറയേണ്ട. വിശ്വസിച്ചതിന് ശേഷം നിങ്ങള്‍ അവിശ്വസിച്ചു കഴിഞ്ഞിരിക്കുന്നു. നിങ്ങളില്‍ ഒരു വിഭാഗത്തിന് നാം മാപ്പുനല്‍കുകയാണെങ്കില്‍ തന്നെ മറ്റൊരു വിഭാഗത്തിന് അവര്‍ കുറ്റവാളികളായിരുന്നതിനാല്‍ നാം ശിക്ഷ നല്‍കുന്നതാണ്‌.
\end{malayalam}}
\flushright{\begin{Arabic}
\quranayah[9][67]
\end{Arabic}}
\flushleft{\begin{malayalam}
കപടവിശ്വാസികളും കപടവിശ്വാസിനികളും എല്ലാം ഒരേ തരക്കാരാകുന്നു. അവര്‍ ദുരാചാരം കല്‍പിക്കുകയും, സദാചാരത്തില്‍ നിന്ന് വിലക്കുകയും, തങ്ങളുടെ കൈകള്‍ അവര്‍ പിന്‍വലിക്കുകയും ചെയ്യുന്നു. അവര്‍ അല്ലാഹുവെ മറന്നു. അപ്പോള്‍ അവന്‍ അവരെയും മറന്നു. തീര്‍ച്ചയായും കപടവിശ്വാസികള്‍ തന്നെയാണ് ധിക്കാരികള്‍.
\end{malayalam}}
\flushright{\begin{Arabic}
\quranayah[9][68]
\end{Arabic}}
\flushleft{\begin{malayalam}
കപടവിശ്വാസികള്‍ക്കും കപടവിശ്വാസിനികള്‍ക്കും, സത്യനിഷേധികള്‍ക്കും അല്ലാഹു നരകാഗ്നി വാഗ്ദാനം ചെയ്തിരിക്കുന്നു. അവരതില്‍ നിത്യവാസികളായിരിക്കും. അവര്‍ക്കതു മതി. അല്ലാഹു അവരെ ശപിക്കുകയും ചെയ്തിരിക്കുന്നു. അവര്‍ക്ക് സ്ഥിരമായ ശിക്ഷയുണ്ടായിരിക്കുന്നതാണ്‌.
\end{malayalam}}
\flushright{\begin{Arabic}
\quranayah[9][69]
\end{Arabic}}
\flushleft{\begin{malayalam}
നിങ്ങളുടെ മുമ്പുണ്ടായിരുന്നവരെപ്പോലെത്തന്നെ. നിങ്ങളെക്കാള്‍ കനത്ത ശക്തിയുള്ളവരും, കൂടുതല്‍ സ്വത്തുക്കളും സന്തതികളുമുള്ളവരുമായിരുന്നു അവര്‍. അങ്ങനെ തങ്ങളുടെ ഓഹരികൊണ്ട് അവര്‍ സുഖമനുഭവിച്ചു. എന്നാല്‍ നിങ്ങളുടെ ആ മുന്‍ഗാമികള്‍ അവരുടെ ഓഹരികൊണ്ട് സുഖമനുഭവിച്ചത് പോലെ ഇപ്പോള്‍ നിങ്ങളുടെ ഓഹരികൊണ്ട് നിങ്ങളും സുഖമനുഭവിച്ചു. അവര്‍ (അധര്‍മ്മത്തില്‍) മുഴുകിയത് പോലെ നിങ്ങളും മുഴുകി. അത്തരക്കാരുടെ കര്‍മ്മങ്ങള്‍ ഇഹത്തിലും പരത്തിലും നിഷ്ഫലമായിരിക്കുന്നു. അവര്‍ തന്നെയാണ് നഷ്ടം പറ്റിയവര്‍.
\end{malayalam}}
\flushright{\begin{Arabic}
\quranayah[9][70]
\end{Arabic}}
\flushleft{\begin{malayalam}
ഇവര്‍ക്ക് മുമ്പുള്ളവരുടെ വൃത്താന്തം ഇവര്‍ക്കു വന്നെത്തിയില്ലേ? അതായത് നൂഹിന്‍റെ ജനതയുടെയും, ആദ്‌, ഥമൂദ് ജനവിഭാഗങ്ങളുടെയും, ഇബ്രാഹീമിന്‍റെ ജനതയുടെയും മദ്‌യങ്കാരുടെയും കീഴ്മേല്‍ മറിഞ്ഞ രാജ്യങ്ങളുടെയും (വൃത്താന്തം.) അവരിലേക്കുള്ള ദൂതന്‍മാര്‍ വ്യക്തമായ തെളിവുകളും കൊണ്ട് അവരുടെ അടുത്ത് ചെല്ലുകയുണ്ടായി. അപ്പോള്‍ അല്ലാഹു അവരോട് അക്രമം കാണിക്കുകയുണ്ടായില്ല. പക്ഷെ, അവര്‍ അവരോടു തന്നെ അക്രമം കാണിക്കുകയായിരുന്നു.
\end{malayalam}}
\flushright{\begin{Arabic}
\quranayah[9][71]
\end{Arabic}}
\flushleft{\begin{malayalam}
സത്യവിശ്വാസികളും സത്യവിശ്വാസിനികളും അന്യോന്യം മിത്രങ്ങളാകുന്നു. അവര്‍ സദാചാരം കല്‍പിക്കുകയും, ദുരാചാരത്തില്‍ നിന്ന് വിലക്കുകയും, നമസ്കാരം മുറപോലെ നിര്‍വഹിക്കുകയും, സകാത്ത് നല്‍കുകയും, അല്ലാഹുവെയും അവന്‍റെ ദൂതനെയും അനുസരിക്കുകയും ചെയ്യുന്നു. അത്തരക്കാരോട് അല്ലാഹു കരുണ കാണിക്കുന്നതാണ്‌. തീര്‍ച്ചയായും അല്ലാഹു പ്രതാപിയും യുക്തിമാനുമാണ്‌.
\end{malayalam}}
\flushright{\begin{Arabic}
\quranayah[9][72]
\end{Arabic}}
\flushleft{\begin{malayalam}
സത്യവിശ്വാസികള്‍ക്കും സത്യവിശ്വാസിനികള്‍ക്കും താഴ്ഭാഗത്ത് കൂടി അരുവികള്‍ ഒഴുകുന്ന സ്വര്‍ഗത്തോപ്പുകള്‍ അല്ലാഹു വാഗ്ദാനം ചെയ്തിരിക്കുന്നു. അവര്‍ അതില്‍ നിത്യവാസികളായിരിക്കും. സ്ഥിരവാസത്തിനുള്ള തോട്ടങ്ങളില്‍ വിശിഷ്ടമായ പാര്‍പ്പിടങ്ങളും (വാഗ്ദാനം ചെയ്തിരിക്കുന്നു.) എന്നാല്‍ അല്ലാഹുവിങ്കല്‍ നിന്നുള്ള പ്രീതിയാണ് ഏറ്റവും വലുത്‌. അതത്രെ മഹത്തായ വിജയം.
\end{malayalam}}
\flushright{\begin{Arabic}
\quranayah[9][73]
\end{Arabic}}
\flushleft{\begin{malayalam}
നബിയേ, സത്യനിഷേധികളോടും, കപടവിശ്വാസികളോടും സമരം ചെയ്യുകയും, അവരോട് പരുഷമായി പെരുമാറുകയും ചെയ്യുക. അവര്‍ക്കുള്ള സങ്കേതം നരകമത്രെ. ചെന്നുചേരാനുള്ള ആ സ്ഥലം വളരെ ചീത്തതന്നെ.
\end{malayalam}}
\flushright{\begin{Arabic}
\quranayah[9][74]
\end{Arabic}}
\flushleft{\begin{malayalam}
തങ്ങള്‍ (അങ്ങനെ) പറഞ്ഞിട്ടില്ല എന്ന് അവര്‍ അല്ലാഹുവിന്‍റെ പേരില്‍ സത്യം ചെയ്ത് പറയും, തീര്‍ച്ചയായും അവിശ്വാസത്തിന്‍റെ വാക്ക് അവര്‍ ഉച്ചരിക്കുകയും, ഇസ്ലാം സ്വീകരിച്ചതിനു ശേഷം അവര്‍ അവിശ്വസിച്ച് കളയുകയും അവര്‍ക്ക് നേടാന്‍ കഴിയാത്ത കാര്യത്തിന് അവര്‍ ആലോചന നടത്തുകയും ചെയ്തിരിക്കുന്നു. അല്ലാഹുവിന്‍റെ അനുഗ്രഹത്താല്‍ അവനും അവന്‍റെ ദൂതനും അവര്‍ക്ക് ഐശ്വര്യമുണ്ടാക്കികൊടുത്തു എന്നതൊഴിച്ച് അവരുടെ എതിര്‍പ്പിന് ഒരു കാരണവുമില്ല. ആകയാല്‍ അവര്‍ പശ്ചാത്തപിക്കുകയാണെങ്കില്‍ അതവര്‍ക്ക് ഉത്തമമായിരിക്കും. അവര്‍ പിന്തിരിഞ്ഞ് കളയുന്ന പക്ഷം അല്ലാഹു അവര്‍ക്ക് ഇഹത്തിലും പരത്തിലും വേദനയേറിയ ശിക്ഷ നല്‍കുന്നതാണ്‌. ഭൂമിയില്‍ അവര്‍ക്ക് ഒരു മിത്രമോ സഹായിയോ ഉണ്ടായിരിക്കുകയുമില്ല.
\end{malayalam}}
\flushright{\begin{Arabic}
\quranayah[9][75]
\end{Arabic}}
\flushleft{\begin{malayalam}
അല്ലാഹു അവന്‍റെ അനുഗ്രഹത്തില്‍ നിന്ന് ഞങ്ങള്‍ക്ക് നല്‍കുകയാണെങ്കില്‍ തീര്‍ച്ചയായും ഞങ്ങള്‍ ദാനം ചെയ്യുകയും, ഞങ്ങള്‍ സജ്ജനങ്ങളുടെ കൂട്ടത്തിലായിരിക്കുകയും ചെയ്യുമെന്ന് അവനുമായി കരാര്‍ ചെയ്ത ചിലരും ആ കൂട്ടത്തിലുണ്ട്‌.
\end{malayalam}}
\flushright{\begin{Arabic}
\quranayah[9][76]
\end{Arabic}}
\flushleft{\begin{malayalam}
എന്നിട്ട് അവന്‍ അവര്‍ക്ക് തന്‍റെ അനുഗ്രഹത്തില്‍ നിന്ന് നല്‍കിയപ്പോള്‍ അവര്‍ അതില്‍ പിശുക്ക് കാണിക്കുകയും, അവഗണിച്ചുകൊണ്ട് തിരിഞ്ഞുകളയുകയും ചെയ്തു.
\end{malayalam}}
\flushright{\begin{Arabic}
\quranayah[9][77]
\end{Arabic}}
\flushleft{\begin{malayalam}
അവര്‍ അവനെ കണ്ടുമുട്ടുന്ന ദിവസം (ന്യായവിധിയുടെ ദിവസം) വരെ അവരുടെ ഹൃദയങ്ങളില്‍ കാപട്യമുണ്ടായിരിക്കുക എന്നതാണ് അതിന്‍റെ അനന്തരഫലമായി അവന്‍ അവര്‍ക്ക് നല്‍കിയത്‌. അല്ലാഹുവോട് അവര്‍ ചെയ്ത വാഗ്ദാനം അവര്‍ ലംഘിച്ചത് കൊണ്ടും, അവര്‍ കള്ളം പറഞ്ഞിരുന്നതുകൊണ്ടുമാണത്‌.
\end{malayalam}}
\flushright{\begin{Arabic}
\quranayah[9][78]
\end{Arabic}}
\flushleft{\begin{malayalam}
അവരുടെ രഹസ്യവും അവരുടെ ഗൂഢമന്ത്രവും അല്ലാഹു അറിയുന്നുണ്ടെന്നും, അല്ലാഹു അദൃശ്യകാര്യങ്ങള്‍ നന്നായി അറിയുന്നവനാണെന്നും അവര്‍ മനസ്സിലാക്കിയിട്ടില്ലേ?
\end{malayalam}}
\flushright{\begin{Arabic}
\quranayah[9][79]
\end{Arabic}}
\flushleft{\begin{malayalam}
സത്യവിശ്വാസികളില്‍ നിന്ന് ദാനധര്‍മ്മങ്ങള്‍ ചെയ്യാന്‍ സ്വയം സന്നദ്ധരായി വരുന്നവരെയും, സ്വന്തം അദ്ധ്വാനമല്ലാതെ മറ്റൊന്നും (ദാനം ചെയ്യാന്‍) കണ്ടെത്താത്തവരെയും അധിക്ഷേപിക്കുന്നവരത്രെ അവര്‍. അങ്ങനെ ആ വിശ്വാസികളെ അവര്‍ പരിഹസിക്കുന്നു. അല്ലാഹു അവരെയും പരിഹസിച്ചിരിക്കുകയാണ്‌. അവര്‍ക്ക് വേദനയേറിയ ശിക്ഷയാണുള്ളത്‌.
\end{malayalam}}
\flushright{\begin{Arabic}
\quranayah[9][80]
\end{Arabic}}
\flushleft{\begin{malayalam}
(നബിയേ,) നീ അവര്‍ക്ക് വേണ്ടി പാപമോചനം തേടിക്കൊള്ളുക. അല്ലെങ്കില്‍ അവര്‍ക്ക് വേണ്ടി പാപമോചനം തേടാതിരിക്കുക. നീ അവര്‍ക്ക് വേണ്ടി എഴുപത് പ്രാവശ്യം പാപമോചനം തേടിയാലും അല്ലാഹു അവര്‍ക്ക് പൊറുത്തുകൊടുക്കുകയില്ല. അവര്‍ അല്ലാഹുവിലും അവന്‍റെ ദൂതനിലും അവിശ്വസിച്ചത് കൊണ്ടത്രെ അത്‌. ധിക്കാരികളായ ജനങ്ങളെ അല്ലാഹു നേര്‍വഴിയിലാക്കുകയില്ല.
\end{malayalam}}
\flushright{\begin{Arabic}
\quranayah[9][81]
\end{Arabic}}
\flushleft{\begin{malayalam}
(യുദ്ധത്തിനു പോകാതെ) പിന്‍മാറി ഇരുന്നവര്‍ അല്ലാഹുവിന്‍റെ ദൂതന്‍റെ കല്‍പനക്കെതിരായുള്ള അവരുടെ ഇരുത്തത്തില്‍ സന്തോഷം പൂണ്ടു. തങ്ങളുടെ സ്വത്തുക്കള്‍കൊണ്ടും ശരീരങ്ങള്‍കൊണ്ടും അല്ലാഹുവിന്‍റെ മാര്‍ഗത്തില്‍ സമരം ചെയ്യുവാന്‍ അവര്‍ ഇഷ്ടപ്പെട്ടില്ല. അവര്‍ പറഞ്ഞു: ഈ ഉഷ്ണത്തില്‍ നിങ്ങള്‍ ഇറങ്ങിപുറപ്പെടേണ്ട. പറയുക. നരകാഗ്നി കൂടുതല്‍ കഠിനമായ ചൂടുള്ളതാണ്‌. അവര്‍ കാര്യം ഗ്രഹിക്കുന്നവരായിരുന്നെങ്കില്‍!
\end{malayalam}}
\flushright{\begin{Arabic}
\quranayah[9][82]
\end{Arabic}}
\flushleft{\begin{malayalam}
അതിനാല്‍ അവര്‍ അല്‍പം ചിരിക്കുകയും കൂടുതല്‍ കരയുകയും ചെയ്തുകൊള്ളട്ടെ; അവര്‍ ചെയ്തുവെച്ചിരുന്നതിന്‍റെ ഫലമായിട്ട്‌.
\end{malayalam}}
\flushright{\begin{Arabic}
\quranayah[9][83]
\end{Arabic}}
\flushleft{\begin{malayalam}
ഇനി (യുദ്ധം കഴിഞ്ഞിട്ട്‌) അവരില്‍ ഒരു വിഭാഗത്തിന്‍റെ അടുത്തേക്ക് നിന്നെ അല്ലാഹു (സുരക്ഷിതനായി) തിരിച്ചെത്തിക്കുകയും, അനന്തരം (മറ്റൊരു യുദ്ധത്തിന് നിന്‍റെ കൂടെ) പുറപ്പെടാന്‍ അവര്‍ സമ്മതം തേടുകയും ചെയ്യുന്ന പക്ഷം നീ പറയുക: നിങ്ങളൊരിക്കലും എന്‍റെ കൂടെ പുറപ്പെടുന്നതല്ല. നിങ്ങള്‍ എന്‍റെ കൂടെ ഒരു ശത്രുവോടും യുദ്ധം ചെയ്യുന്നതുമല്ല. തീര്‍ച്ചയായും നിങ്ങള്‍ ആദ്യത്തെപ്രാവശ്യം ഒഴിഞ്ഞിരിക്കുന്നതില്‍ തൃപ്തി അടയുകയാണല്ലോ ചെയ്തത്‌. അതിനാല്‍ ഒഴിഞ്ഞിരുന്നവരുടെ കൂടെ നിങ്ങളും ഇരുന്ന് കൊള്ളുക.
\end{malayalam}}
\flushright{\begin{Arabic}
\quranayah[9][84]
\end{Arabic}}
\flushleft{\begin{malayalam}
അവരുടെ കൂട്ടത്തില്‍ നിന്ന് മരണപ്പെട്ട യാതൊരാളുടെ പേരിലും നീ ഒരിക്കലും നമസ്കരിക്കരുത്‌. അവന്‍റെ ഖബ്‌റിന്നരികില്‍ നില്‍ക്കുകയും ചെയ്യരുത്‌. തീര്‍ച്ചയായും അവര്‍ അല്ലാഹുവിലും അവന്‍റെ ദൂതനിലും അവിശ്വസിക്കുകയും, ധിക്കാരികളായിക്കൊണ്ട് മരിക്കുകയും ചെയ്തിരിക്കുന്നു.
\end{malayalam}}
\flushright{\begin{Arabic}
\quranayah[9][85]
\end{Arabic}}
\flushleft{\begin{malayalam}
അവരുടെ സ്വത്തുക്കളും സന്താനങ്ങളും നിന്നെ അത്ഭുതപ്പെടുത്താതിരിക്കട്ടെ. ഇഹലോകത്തില്‍ അവ മൂലം അവരെ ശിക്ഷിക്കുവാനും സത്യനിഷേധികളായിക്കൊണ്ട് അവര്‍ ജീവനാശമടയുവാനും മാത്രമാണ് അല്ലാഹു ഉദ്ദേശിക്കുന്നത്‌.
\end{malayalam}}
\flushright{\begin{Arabic}
\quranayah[9][86]
\end{Arabic}}
\flushleft{\begin{malayalam}
നിങ്ങള്‍ അല്ലാഹുവില്‍ വിശ്വസിക്കുകയും, അവന്‍റെ ദൂതനോടൊപ്പം സമരത്തില്‍ ഏര്‍പെടുകയും ചെയ്യുക എന്ന് (നിര്‍ദേശിച്ചു കൊണ്ട്‌) വല്ല അദ്ധ്യായവും അവതരിപ്പിക്കപ്പെട്ടാല്‍ അവരുടെ കൂട്ടത്തില്‍ കഴിവുള്ളവര്‍ നിന്നോട് (യുദ്ധത്തിന് പോകാതിരിക്കാന്‍) സമ്മതം തേടുന്നതാണ്‌. അവര്‍ പറയും: ഞങ്ങളെ വിട്ടേക്കണം. ഞങ്ങള്‍ ഒഴിഞ്ഞിരിക്കുന്നവരുടെ കൂട്ടത്തില്‍ ആകാം.
\end{malayalam}}
\flushright{\begin{Arabic}
\quranayah[9][87]
\end{Arabic}}
\flushleft{\begin{malayalam}
(യുദ്ധത്തിനു പോകാതെ) ഒഴിഞ്ഞിരിക്കുന്ന സ്ത്രീകളുടെ കൂട്ടത്തിലായിരിക്കുന്നതില്‍ അവര്‍ തൃപ്തിയടഞ്ഞിരിക്കുന്നു. അവരുടെ ഹൃദയങ്ങളില്‍ മുദ്രവെക്കപ്പെടുകയും ചെയ്തിരിക്കുന്നു. അതിനാല്‍ അവര്‍ (കാര്യം) ഗ്രഹിക്കുകയില്ല.
\end{malayalam}}
\flushright{\begin{Arabic}
\quranayah[9][88]
\end{Arabic}}
\flushleft{\begin{malayalam}
പക്ഷെ, റസൂലും അദ്ദേഹത്തോടൊപ്പം വിശ്വസിച്ചവരും തങ്ങളുടെ സ്വത്തുക്കള്‍ കൊണ്ടും ശരീരങ്ങള്‍ കൊണ്ടും സമരം ചെയ്തു. അവര്‍ക്കാണ് നന്‍മകളുള്ളത്‌. അവര്‍ തന്നെയാണ് വിജയം പ്രാപിച്ചവര്‍.
\end{malayalam}}
\flushright{\begin{Arabic}
\quranayah[9][89]
\end{Arabic}}
\flushleft{\begin{malayalam}
അല്ലാഹു അവര്‍ക്ക് താഴ്ഭാഗത്ത് കൂടി അരുവികള്‍ ഒഴുകുന്ന സ്വര്‍ഗത്തോപ്പുകള്‍ ഒരുക്കിവെച്ചിരിക്കുന്നു. അവരതില്‍ നിത്യവാസികളായിരിക്കും. അതത്രെ മഹത്തായ വിജയം.
\end{malayalam}}
\flushright{\begin{Arabic}
\quranayah[9][90]
\end{Arabic}}
\flushleft{\begin{malayalam}
ഗ്രാമീണ അറബികളില്‍ നിന്ന് (യുദ്ധത്തിന് പോകാതിരിക്കാന്‍) ഒഴികഴിവ് ബോധിപ്പിക്കാനുള്ളവര്‍ തങ്ങള്‍ക്ക് സമ്മതം നല്‍കപ്പെടുവാന്‍ വേണ്ടി (റസൂലിന്‍റെ അടുത്തു) വന്നു. അല്ലാഹുവിനോടും അവന്‍റെ ദൂതനോടും കള്ളം പറഞ്ഞവര്‍ (വീട്ടില്‍) ഇരിക്കുകയും ചെയ്തു. അവരില്‍ നിന്ന് അവിശ്വസിച്ചിട്ടുള്ളവര്‍ക്ക് വേദനയേറിയ ശിക്ഷ ബാധിക്കുന്നതാണ്‌.
\end{malayalam}}
\flushright{\begin{Arabic}
\quranayah[9][91]
\end{Arabic}}
\flushleft{\begin{malayalam}
ബലഹീനരുടെ മേലും, രോഗികളുടെ മേലും, ചെലവഴിക്കാന്‍ യാതൊന്നും കിട്ടാത്തവരുടെ മേലും -അവര്‍ അല്ലാഹുവോടും റസൂലിനോടും ഗുണകാംക്ഷയുള്ളവരാണെങ്കില്‍ -(യുദ്ധത്തിന് പോകാത്തതിന്‍റെ പേരില്‍) യാതൊരു കുറ്റവുമില്ല. സദ്‌വൃത്തരായ ആളുകള്‍ക്കെതിരില്‍ (കുറ്റം ചുമത്താന്‍) യാതൊരു മാര്‍ഗവുമില്ല. അല്ലാഹു ഏറെ പൊറുക്കുന്നവനും കരുണാനിധിയുമാകുന്നു.
\end{malayalam}}
\flushright{\begin{Arabic}
\quranayah[9][92]
\end{Arabic}}
\flushleft{\begin{malayalam}
മറ്റൊരു വിഭാഗത്തിന്‍റെ മേലും കുറ്റമില്ല.(യുദ്ധത്തിനു പോകാന്‍) നീ അവര്‍ക്കു വാഹനം നല്‍കുന്നതിന് വേണ്ടി അവര്‍ നിന്‍റെ അടുത്ത് വന്നപ്പോള്‍ നീ പറഞ്ഞു: നിങ്ങള്‍ക്ക് നല്‍കാന്‍ യാതൊരു വാഹനവും ഞാന്‍ കണ്ടെത്തുന്നില്ല. അങ്ങനെ (യുദ്ധത്തിന് വേണ്ടി) ചെലവഴിക്കാന്‍ യാതൊന്നും കണ്ടെത്താത്തതിന്‍റെ പേരിലുള്ള ദുഃഖത്താല്‍ കണ്ണുകളില്‍ നിന്ന് കണ്ണുനീര്‍ ഒഴുകിക്കൊണ്ട് അവര്‍ തിരിച്ചുപോയി. (അങ്ങനെയുള്ള ഒരു വിഭാഗത്തിന്‍റെ മേല്‍.)
\end{malayalam}}
\flushright{\begin{Arabic}
\quranayah[9][93]
\end{Arabic}}
\flushleft{\begin{malayalam}
ഐശ്വര്യമുള്ളവരായിരിക്കെ (ഒഴിഞ്ഞു നില്‍ക്കാന്‍) നിന്നോട് സമ്മതം തേടുകയും, ഒഴിഞ്ഞിരിക്കുന്ന സ്ത്രീകളുടെ കൂട്ടത്തില്‍ ആയിരിക്കുന്നതില്‍ തൃപ്തി അടയുകയും ചെയ്ത വിഭാഗത്തിനെതിരില്‍ മാത്രമാണ് (കുറ്റം ആരോപിക്കാന്‍) മാര്‍ഗമുള്ളത്‌. അവരുടെ ഹൃദയങ്ങളില്‍ അല്ലാഹു മുദ്രവെക്കുകയും ചെയ്തിരിക്കുന്നു. അതിനാല്‍ അവര്‍ (കാര്യം) മനസ്സിലാക്കുന്നില്ല.
\end{malayalam}}
\flushright{\begin{Arabic}
\quranayah[9][94]
\end{Arabic}}
\flushleft{\begin{malayalam}
അവരുടെ അടുക്കലേക്ക് (യുദ്ധം കഴിഞ്ഞ്‌) നിങ്ങള്‍ മടങ്ങിയെത്തിയാല്‍ അവര്‍ നിങ്ങളോട് ഒഴികഴിവ് പറയുന്നതാണ്‌. പറയുക: നിങ്ങള്‍ ഒഴികഴിവൊന്നും പറയേണ്ട. നിങ്ങളെ ഞങ്ങള്‍ വിശ്വസിക്കുകയേ ഇല്ല. (കാരണം) നിങ്ങളുടെ ചില വര്‍ത്തമാനങ്ങള്‍ അല്ലാഹു ഞങ്ങള്‍ക്ക് അറിയിച്ച് തന്നിട്ടുണ്ട്‌. നിങ്ങളുടെ പ്രവര്‍ത്തനം അല്ലാഹുവും അവന്‍റെ ദൂതനും കാണുന്നതുമാണ്‌. പിന്നീട് അദൃശ്യവും ദൃശ്യവും അറിയുന്നവന്‍റെ അടുത്തേക്ക് നിങ്ങള്‍ മടക്കപ്പെടുന്നതാണ്‌. നിങ്ങള്‍ പ്രവര്‍ത്തിച്ച് കൊണ്ടിരുന്നതിനെപ്പറ്റി അപ്പോള്‍ അവന്‍ നിങ്ങള്‍ക്ക് വിവരം നല്‍കുന്നതാണ്‌.
\end{malayalam}}
\flushright{\begin{Arabic}
\quranayah[9][95]
\end{Arabic}}
\flushleft{\begin{malayalam}
നിങ്ങള്‍ അവരുടെ അടുത്തേക്ക് തിരിച്ചുചെന്നാല്‍ നിങ്ങളോട് അവര്‍ അല്ലാഹുവിന്‍റെ പേരില്‍ സത്യം ചെയ്യും. നിങ്ങള്‍ അവരെ വിട്ടു ഒഴിഞ്ഞുകളയുവാന്‍ വേണ്ടി യത്രെ അത്‌. അത് കൊണ്ട് നിങ്ങള്‍ അവരെ ഒഴിവാക്കി വിട്ടേക്കുക. തീര്‍ച്ചയായും അവര്‍ വൃത്തികെട്ടവരാകുന്നു. അവരുടെ സങ്കേതം നരകമത്രെ. അവര്‍ പ്രവര്‍ത്തിച്ച് കൊണ്ടിരുന്നതിനുള്ള പ്രതിഫലമാണത്‌.
\end{malayalam}}
\flushright{\begin{Arabic}
\quranayah[9][96]
\end{Arabic}}
\flushleft{\begin{malayalam}
നിങ്ങളോടവര്‍ സത്യം ചെയ്യുന്നത് നിങ്ങള്‍ക്ക് അവരെപ്പറ്റി തൃപ്തിയാകുവാന്‍ വേണ്ടിയാണ്‌. ഇനി നിങ്ങള്‍ക്ക് അവരെപ്പറ്റി തൃപ്തിയായാല്‍ തന്നെയും അല്ലാഹു അധര്‍മ്മകാരികളായ ജനങ്ങളെപ്പറ്റി തൃപ്തിപ്പെടുകയില്ല; തീര്‍ച്ച.
\end{malayalam}}
\flushright{\begin{Arabic}
\quranayah[9][97]
\end{Arabic}}
\flushleft{\begin{malayalam}
അഅ്‌റാബികള്‍ (മരുഭൂവാസികള്‍) കൂടുതല്‍ കടുത്ത അവിശ്വാസവും കാപട്യവുമുള്ളവരത്രെ. അല്ലാഹു അവന്‍റെ ദൂതന്ന് അവതരിപ്പിച്ചു കൊടുത്തതിലെ നിയമപരിധികളറിയാതിരിക്കാന്‍ കൂടുതല്‍ തരപ്പെട്ടവരുമാണവര്‍. അല്ലാഹു എല്ലാം അറിയുന്നവനും യുക്തിമാനുമാകുന്നു.
\end{malayalam}}
\flushright{\begin{Arabic}
\quranayah[9][98]
\end{Arabic}}
\flushleft{\begin{malayalam}
തങ്ങള്‍ (ദാനമായി) ചെലവഴിക്കുന്നത് ഒരു ധനനഷ്ടമായി ഗണിക്കുകയും, നിങ്ങള്‍ക്ക് കാലക്കേടുകള്‍ വരുന്നത് കാത്തിരിക്കുകയും ചെയ്യുന്ന ഒരു വിഭാഗം അഅ്‌റാബികളുടെ കൂട്ടത്തിലുണ്ട്‌. അവരുടെ മേല്‍ തന്നെയായിരിക്കട്ടെ ഹീനമായ കാലക്കേട്‌. അല്ലാഹു എല്ലാം കേള്‍ക്കുന്നവനും അറിയുന്നവനുമത്രെ.
\end{malayalam}}
\flushright{\begin{Arabic}
\quranayah[9][99]
\end{Arabic}}
\flushleft{\begin{malayalam}
അല്ലാഹുവിലും അന്ത്യദിനത്തിലും വിശ്വസിക്കുകയും, തങ്ങള്‍ ചെലവഴിക്കുന്നതിനെ അല്ലാഹുവിങ്കല്‍ സാമീപ്യത്തിനുതകുന്ന പുണ്യകര്‍മ്മങ്ങളും, റസൂലിന്‍റെ പ്രാര്‍ത്ഥനയ്ക്കുള്ള മാര്‍ഗവും ആക്കിത്തീര്‍ക്കുകയും ചെയ്യുന്ന ചിലരും അഅ്‌റാബികളുടെ കൂട്ടത്തിലുണ്ട്‌. ശ്രദ്ധിക്കുക: തീര്‍ച്ചയായും അതവര്‍ക്ക് ദൈവസാമീപ്യം നല്‍കുന്നതാണ്‌. അല്ലാഹു അവരെ തന്‍റെ കാരുണ്യത്തില്‍ പ്രവേശിപ്പിക്കുന്നതാണ്‌. തീര്‍ച്ചയായും അല്ലാഹു ഏറെ പൊറുക്കുന്നവനും കരുണാനിധിയുമാകുന്നു.
\end{malayalam}}
\flushright{\begin{Arabic}
\quranayah[9][100]
\end{Arabic}}
\flushleft{\begin{malayalam}
മുഹാജിറുകളില്‍ നിന്നും അന്‍സാറുകളില്‍ നിന്നും ആദ്യമായി മുന്നോട്ട് വന്നവരും, സുകൃതം ചെയ്തുകൊണ്ട് അവരെ പിന്തുടര്‍ന്നവരും ആരോ അവരെപ്പറ്റി അല്ലാഹു സംതൃപ്തനായിരിക്കുന്നു. അവനെപ്പറ്റി അവരും സംതൃപ്തരായിരിക്കുന്നു. താഴ്ഭാഗത്ത് അരുവികള്‍ ഒഴുകിക്കൊണ്ടിരിക്കുന്ന സ്വര്‍ഗത്തോപ്പുകള്‍ അവര്‍ക്ക് അവന്‍ ഒരുക്കിവെക്കുകയും ചെയ്തിരിക്കുന്നു. എന്നെന്നും അവരതില്‍ നിത്യവാസികളായിരിക്കും. അതത്രെ മഹത്തായ ഭാഗ്യം.
\end{malayalam}}
\flushright{\begin{Arabic}
\quranayah[9][101]
\end{Arabic}}
\flushleft{\begin{malayalam}
നിങ്ങളുടെ ചുറ്റുമുള്ള അഅ്‌റാബികളുടെ കൂട്ടത്തിലും കപടവിശ്വാസികളുണ്ട്‌. മദീനക്കാരുടെ കൂട്ടത്തിലുമുണ്ട്‌. കാപട്യത്തില്‍ അവര്‍ കടുത്തുപോയിരിക്കുന്നു. നിനക്ക് അവരെ അറിയില്ല. നമുക്ക് അവരെ അറിയാം. രണ്ട് പ്രാവശ്യം നാം അവരെ ശിക്ഷിക്കുന്നതാണ്‌.പിന്നീട് വമ്പിച്ച ശിക്ഷയിലേക്ക് അവര്‍ തള്ളപ്പെടുന്നതുമാണ്‌.
\end{malayalam}}
\flushright{\begin{Arabic}
\quranayah[9][102]
\end{Arabic}}
\flushleft{\begin{malayalam}
തങ്ങളുടെ കുറ്റങ്ങള്‍ ഏറ്റുപറഞ്ഞ വേറെ ചിലരുണ്ട്‌. (കുറെ) സല്‍കര്‍മ്മവും, വേറെ ദുഷ്കര്‍മ്മവുമായി അവര്‍ കൂട്ടികലര്‍ത്തിയിരിക്കുന്നു. അല്ലാഹു അവരുടെ പശ്ചാത്താപം സ്വീകരിച്ചെന്ന് വരാം. തീര്‍ച്ചയായും അല്ലാഹു ഏറെ പൊറുക്കുന്നവനും കരുണാനിധിയുമാകുന്നു.
\end{malayalam}}
\flushright{\begin{Arabic}
\quranayah[9][103]
\end{Arabic}}
\flushleft{\begin{malayalam}
അവരെ ശുദ്ധീകരിക്കുകയും , അവരെ സംസ്കരിക്കുകയും ചെയ്യാനുതകുന്ന ദാനം അവരുടെ സ്വത്തുകളില്‍ നിന്ന് നീ വാങ്ങുകയും, അവര്‍ക്കുവേണ്ടി (അനുഗ്രഹത്തിന്നായി) പ്രാര്‍ത്ഥിക്കുകയും ചെയ്യുക. തീര്‍ച്ചയായും നിന്‍റെ പ്രാര്‍ത്ഥന അവര്‍ക്ക് ശാന്തി നല്‍കുന്നതത്രെ. അല്ലാഹു എല്ലാം കേള്‍ക്കുന്നവനും അറിയുന്നവനുമാകുന്നു.
\end{malayalam}}
\flushright{\begin{Arabic}
\quranayah[9][104]
\end{Arabic}}
\flushleft{\begin{malayalam}
അല്ലാഹു തന്നെയാണ് തന്‍റെ ദാസന്‍മാരുടെ പശ്ചാത്താപം സ്വീകരിക്കുകയും, ദാനധര്‍മ്മങ്ങള്‍ ഏറ്റുവാങ്ങുകയും ചെയ്യുന്നതെന്നും അല്ലാഹു തന്നെയാണ് പശ്ചാത്താപം ഏറെ സ്വീകരിക്കുന്നവനും കരുണാനിധിയുമെന്നും അവര്‍ മനസ്സിലാക്കിയിട്ടില്ലേ?
\end{malayalam}}
\flushright{\begin{Arabic}
\quranayah[9][105]
\end{Arabic}}
\flushleft{\begin{malayalam}
(നബിയേ,) പറയുക: നിങ്ങള്‍ പ്രവര്‍ത്തിച്ച് കൊള്ളുക. അല്ലാഹുവും അവന്‍റെ ദൂതനും സത്യവിശ്വാസികളും നിങ്ങളുടെ പ്രവര്‍ത്തനം കണ്ടുകൊള്ളും. അദൃശ്യവും ദൃശ്യവും അറിയുന്നവന്‍റെ അടുക്കലേക്ക് നിങ്ങള്‍ മടക്കപ്പെടുന്നതും, നിങ്ങള്‍ പ്രവര്‍ത്തിച്ചിരുന്നതിനെപ്പറ്റി അപ്പോള്‍ അവന്‍ നിങ്ങളെ വിവരമറിയിക്കുന്നതുമാണ്‌.
\end{malayalam}}
\flushright{\begin{Arabic}
\quranayah[9][106]
\end{Arabic}}
\flushleft{\begin{malayalam}
അല്ലാഹുവിന്‍റെ കല്‍പന കിട്ടുന്നത് വരെ തീരുമാനം മേറ്റീവ്ക്കപ്പെട്ട മറ്റുചിലരുമുണ്ട്‌. ഒന്നുകില്‍ അവന്‍ അവരെ ശിക്ഷിക്കും. അല്ലെങ്കില്‍ അവരുടെ പശ്ചാത്താപം സ്വീകരിക്കും. അല്ലാഹു എല്ലാം അറിയുന്നവനും യുക്തിമാനുമാകുന്നു.
\end{malayalam}}
\flushright{\begin{Arabic}
\quranayah[9][107]
\end{Arabic}}
\flushleft{\begin{malayalam}
ദ്രോഹബുദ്ധിയാലും, സത്യനിഷേധത്താലും, വിശ്വാസികള്‍ക്കിടയില്‍ ഭിന്നതയുണ്ടാക്കാന്‍ വേണ്ടിയും മുമ്പുതന്നെ അല്ലാഹുവോടും അവന്‍റെ ദൂതനോടും യുദ്ധം ചെയ്തവര്‍ക്ക് താവളമുണ്ടാക്കികൊടുക്കുവാന്‍ വേണ്ടിയും ഒരു പള്ളിയുണ്ടാക്കിയവരും (ആ കപടന്‍മാരുടെ കൂട്ടത്തിലുണ്ട്‌). ഞങ്ങള്‍ നല്ലതല്ലാതെ ഒന്നും ഉദ്ദേശിച്ചിട്ടില്ല എന്ന് അവര്‍ ആണയിട്ട് പറയുകയും ചെയ്യും. തീര്‍ച്ചയായും അവര്‍ കള്ളം പറയുന്നവര്‍ തന്നെയാണ് എന്നതിന് അല്ലാഹു സാക്ഷ്യം വഹിക്കുന്നു.
\end{malayalam}}
\flushright{\begin{Arabic}
\quranayah[9][108]
\end{Arabic}}
\flushleft{\begin{malayalam}
(നബിയേ,) നീ ഒരിക്കലും അതില്‍ നമസ്കാരത്തിനു നില്‍ക്കരുത്‌. ആദ്യ ദിവസം തന്നെ ഭക്തിയിന്‍മേല്‍ സ്ഥാപിക്കപ്പെട്ടിട്ടുള്ള പള്ളിയാണ് നീ നിന്നു നമസ്കരിക്കുവാന്‍ ഏറ്റവും അര്‍ഹതയുള്ളത്‌. ശുദ്ധികൈവരിക്കുവാന്‍ ഇഷ്ടപ്പെടുന്ന ചില ആളുകളുണ്ട് ആ പള്ളിയില്‍. ശുദ്ധികൈവരിക്കുന്നവരെ അല്ലാഹു ഇഷ്ടപ്പെടുന്നു.
\end{malayalam}}
\flushright{\begin{Arabic}
\quranayah[9][109]
\end{Arabic}}
\flushleft{\begin{malayalam}
അല്ലാഹുവെ സംബന്ധിച്ച ഭക്തിയിന്‍മേലും അവന്‍റെ പ്രീതിയിന്‍മേലും തന്‍റെ കെട്ടിടം സ്ഥാപിച്ചവനോ അതല്ല, പൊളിഞ്ഞുവീഴാന്‍ പോകുന്ന ഒരു മണല്‍തിട്ടയുടെ വക്കത്ത് കെട്ടിടം സ്ഥാപിക്കുകയും എന്നിട്ടത് തന്നെയും കൊണ്ട് നരകാഗ്നിയില്‍ പൊളിഞ്ഞുവീഴുകയും ചെയ്തവനോ കൂടുതല്‍ ഉത്തമന്‍? അക്രമികളായ ജനങ്ങളെ അല്ലാഹു നേര്‍വഴിയിലാക്കുകയില്ല.
\end{malayalam}}
\flushright{\begin{Arabic}
\quranayah[9][110]
\end{Arabic}}
\flushleft{\begin{malayalam}
അവര്‍ സ്ഥാപിച്ച അവരുടെ കെട്ടിടം അവരുടെ ഹൃദയങ്ങളില്‍ ആശങ്കയായി തുടരുന്നതാണ്‌. അവരുടെ ഹൃദയങ്ങള്‍ കഷ്ണം കഷ്ണമായി തീര്‍ന്നെങ്കിലല്ലാതെ. അല്ലാഹു എല്ലാം അറിയുന്നവനും യുക്തിമാനുമാകുന്നു.
\end{malayalam}}
\flushright{\begin{Arabic}
\quranayah[9][111]
\end{Arabic}}
\flushleft{\begin{malayalam}
തീര്‍ച്ചയായും സത്യവിശ്വാസികളുടെ പക്കല്‍ നിന്ന്‌, അവര്‍ക്ക് സ്വര്‍ഗമുണ്ടായിരിക്കുക എന്നതിനുപകരമായി അവരുടെ ദേഹങ്ങളും അവരുടെ ധനവും അല്ലാഹു വാങ്ങിയിരിക്കുന്നു. അവര്‍ അല്ലാഹുവിന്‍റെ മാര്‍ഗത്തില്‍ യുദ്ധം ചെയ്യുന്നു. അങ്ങനെ അവര്‍ കൊല്ലുകയും കൊല്ലപ്പെടുകയും ചെയ്യുന്നു. (അങ്ങനെ അവര്‍ സ്വര്‍ഗാവകാശികളാകുന്നു.) തൌറാത്തിലും ഇന്‍ജീലിലും ഖുര്‍ആനിലും തന്‍റെ മേല്‍ ബാധ്യതയായി അല്ലാഹു പ്രഖ്യാപിച്ച സത്യവാഗ്ദാനമത്രെ അത്‌. അല്ലാഹുവെക്കാളധികം തന്‍റെ കരാര്‍ നിറവേറ്റുന്നവനായി ആരുണ്ട്‌? അതിനാല്‍ നിങ്ങള്‍ (അല്ലാഹുവുമായി) നടത്തിയിട്ടുള്ള ആ ഇടപാടില്‍ സന്തോഷം കൊള്ളുവിന്‍. അതു തന്നെയാണ് മഹത്തായ ഭാഗ്യം.
\end{malayalam}}
\flushright{\begin{Arabic}
\quranayah[9][112]
\end{Arabic}}
\flushleft{\begin{malayalam}
പശ്ചാത്തപിക്കുന്നവര്‍, ആരാധനയില്‍ ഏര്‍പെടുന്നവര്‍, സ്തുതികീര്‍ത്തനം ചെയ്യുന്നവര്‍, (അല്ലാഹുവിന്‍റെ മാര്‍ഗത്തില്‍) സഞ്ചരിക്കുന്നവര്‍, കുമ്പിടുകയും സാഷ്ടാംഗം നടത്തുകയും ചെയ്യുന്നവര്‍, സദാചാരം കല്‍പിക്കുകയും ദുരാചാരത്തില്‍നിന്ന് വിലക്കുകയും ചെയ്യുന്നവര്‍, അല്ലാഹുവിന്‍റെ അതിര്‍വരമ്പുകളെ കാത്തുസൂക്ഷിക്കുന്നവര്‍. (ഇങ്ങനെയുള്ള) സത്യവിശ്വാസികള്‍ക്ക് സന്തോഷവാര്‍ത്ത അറിയിക്കുക.
\end{malayalam}}
\flushright{\begin{Arabic}
\quranayah[9][113]
\end{Arabic}}
\flushleft{\begin{malayalam}
ബഹുദൈവവിശ്വാസികള്‍ ജ്വലിക്കുന്ന നരകാഗ്നിയുടെ അവകാശികളാണെന്ന് തങ്ങള്‍ക്കു വ്യക്തമായിക്കഴിഞ്ഞതിന് ശേഷം അവര്‍ക്കുവേണ്ടി പാപമോചനം തേടുവാന്‍ - അവര്‍ അടുത്ത ബന്ധമുള്ളവരായാല്‍ പോലും - പ്രവാചകന്നും സത്യവിശ്വാസികള്‍ക്കും പാടുള്ളതല്ല.
\end{malayalam}}
\flushright{\begin{Arabic}
\quranayah[9][114]
\end{Arabic}}
\flushleft{\begin{malayalam}
ഇബ്രാഹീം അദ്ദേഹത്തിന്‍റെ പിതാവിന് വേണ്ടി പാപമോചനം തേടിയത് അദ്ദേഹം പിതാവിനോട് അങ്ങനെ വാഗ്ദാനം ചെയ്തത് കൊണ്ട് മാത്രമായിരുന്നു. എന്നാല്‍ അയാള്‍ (പിതാവ്‌) അല്ലാഹുവിന്‍റെ ശത്രുവാണെന്ന് അദ്ദേഹത്തിന് വ്യക്തമായപ്പോള്‍ അദ്ദേഹം അയാളെ (പിതാവിനെ) വിട്ടൊഴിഞ്ഞു. തീര്‍ച്ചയായും ഇബ്രാഹീം ഏറെ താഴ്മയുള്ളവനും സഹനശീലനുമാകുന്നു.
\end{malayalam}}
\flushright{\begin{Arabic}
\quranayah[9][115]
\end{Arabic}}
\flushleft{\begin{malayalam}
ഒരു ജനതയ്ക്ക് മാര്‍ഗദര്‍ശനം നല്‍കിയതിന് ശേഷം, അവര്‍ കാത്തുസൂക്ഷിക്കേണ്ടതെന്തെന്ന് അവര്‍ക്ക് വ്യക്തമാക്കികൊടുക്കുന്നതു വരെ അല്ലാഹു അവരെ പിഴച്ചവരായി ഗണിക്കുന്നതല്ല. തീര്‍ച്ചയായും അല്ലാഹു ഏത് കാര്യത്തെപ്പറ്റിയും അറിവുള്ളവനാകുന്നു.
\end{malayalam}}
\flushright{\begin{Arabic}
\quranayah[9][116]
\end{Arabic}}
\flushleft{\begin{malayalam}
തീര്‍ച്ചയായും അല്ലാഹുവിന്നുള്ളതാകുന്നു ആകാശങ്ങളുടെയും ഭൂമിയുടെയും ആധിപത്യം. അവന്‍ ജീവിപ്പിക്കുകയും മരിപ്പിക്കുകയും ചെയ്യുന്നു. അല്ലാഹുവിന് പുറമെ നിങ്ങള്‍ക്ക് യാതൊരു രക്ഷാധികാരിയും സഹായിയുമില്ല.
\end{malayalam}}
\flushright{\begin{Arabic}
\quranayah[9][117]
\end{Arabic}}
\flushleft{\begin{malayalam}
തീര്‍ച്ചയായും പ്രവാചകന്‍റെയും, ഞെരുക്കത്തിന്‍റെ ഘട്ടത്തില്‍ അദ്ദേഹത്തെ പിന്തുടര്‍ന്നവരായ മുഹാജിറുകളുടെയും അന്‍സാറുകളുടെയും നേരെ അല്ലാഹു കനിഞ്ഞ് മടങ്ങിയിരിക്കുന്നു-അവരില്‍ നിന്ന് ഒരു വിഭാഗത്തിന്‍റെ ഹൃദയങ്ങള്‍ തെറ്റിപ്പോകുമാറായതിനു ശേഷം. എന്നിട്ട് അല്ലാഹു അവരുടെ നേരെ കനിഞ്ഞു മടങ്ങി. തീര്‍ച്ചയായും അവന്‍ അവരോട് ഏറെ കൃപയുള്ളവനും കരുണാനിധിയുമാകുന്നു.
\end{malayalam}}
\flushright{\begin{Arabic}
\quranayah[9][118]
\end{Arabic}}
\flushleft{\begin{malayalam}
പിന്നേക്ക് മേറ്റീവ്ക്കപ്പെട്ട ആ മൂന്ന് പേരുടെ നേരെയും (അല്ലാഹു കനിഞ്ഞ് മടങ്ങിയിരിക്കുന്നു.) അങ്ങനെ ഭൂമി വിശാലമായിട്ടുകൂടി അവര്‍ക്ക് ഇടുങ്ങിയതായിത്തീരുകയും, തങ്ങളുടെ മനസ്സുകള്‍ തന്നെ അവര്‍ക്ക് ഞെരുങ്ങിപ്പോകുകയും, അല്ലാഹുവിങ്കല്‍ നിന്ന് രക്ഷതേടുവാന്‍ അവങ്കലല്ലാതെ അഭയസ്ഥാനമില്ലെന്ന് അവര്‍ മനസ്സിലാക്കുകയും ചെയ്തപ്പോള്‍. അവന്‍ വീണ്ടും അവരുടെ നേരെ കനിഞ്ഞു മടങ്ങി. അവര്‍ ഖേദിച്ചുമടങ്ങുന്നവരായിരിക്കാന്‍ വേണ്ടിയത്രെ അത്‌. തീര്‍ച്ചയായും അല്ലാഹു ഏറെ പശ്ചാത്താപം സ്വീകരിക്കുന്നവനും കരുണാനിധിയുമാകുന്നു.
\end{malayalam}}
\flushright{\begin{Arabic}
\quranayah[9][119]
\end{Arabic}}
\flushleft{\begin{malayalam}
സത്യവിശ്വാസികളേ, നിങ്ങള്‍ അല്ലാഹുവെ സൂക്ഷിക്കുകയും, സത്യവാന്‍മാരുടെ കൂട്ടത്തില്‍ ആയിരിക്കുകയും ചെയ്യുക.
\end{malayalam}}
\flushright{\begin{Arabic}
\quranayah[9][120]
\end{Arabic}}
\flushleft{\begin{malayalam}
മദീനക്കാര്‍ക്കും അവരുടെ ചുറ്റുമുള്ള അഅ്‌റാബികള്‍ക്കും അല്ലാഹുവിന്‍റെ ദൂതനെ വിട്ട് പിന്‍മാറി നില്‍ക്കാനോ, അദ്ദേഹത്തിന്‍റെ കാര്യം അവഗണിച്ചുകൊണ്ട് അവരവരുടെ (സ്വന്തം) കാര്യങ്ങളില്‍ താല്‍പര്യം കാണിക്കാനോ പാടുള്ളതല്ല. അതെന്തുകൊണ്ടെന്നാല്‍ അല്ലാഹുവിന്‍റെ മാര്‍ഗത്തില്‍ അവര്‍ക്ക് ദാഹവും ക്ഷീണവും വിശപ്പും നേരിടുകയോ, അവിശ്വാസികളെ പ്രകോപിപ്പിക്കുന്ന വല്ല സ്ഥാനത്തും അവര്‍ കാല്‍ വെക്കുകയോ, ശത്രുവിന് വല്ല നാശവും ഏല്‍പിക്കുകയോ ചെയ്യുന്ന പക്ഷം അതു കാരണം അവര്‍ക്ക് ഒരു സല്‍കര്‍മ്മം രേഖപ്പെടുത്തപ്പെടാതിരിക്കുകയില്ല. തീര്‍ച്ചയായും സുകൃതം ചെയ്യുന്നവര്‍ക്കുള്ള പ്രതിഫലം അല്ലാഹു നഷ്ടപ്പെടുത്തിക്കളയുന്നതല്ല.
\end{malayalam}}
\flushright{\begin{Arabic}
\quranayah[9][121]
\end{Arabic}}
\flushleft{\begin{malayalam}
ചെറുതാകട്ടെ വലുതാകട്ടെ എന്തൊന്ന് അവര്‍ ചെലവഴിക്കുന്നതും, വല്ല താഴ്‌വരയും അവര്‍ മുറിച്ചുകടന്ന് പോകുന്നതും അവര്‍ക്ക് (പുണ്യകര്‍മ്മമായി) രേഖപ്പെടുത്തപ്പെടാതിരിക്കുകയില്ല. അങ്ങനെ അവര്‍ പ്രവര്‍ത്തിച്ചുകൊണ്ടിരിക്കുന്ന അത്യുത്തമമായ കാര്യത്തിന് അല്ലാഹു അവര്‍ക്ക് പ്രതിഫലം നല്‍കുന്നതാണ്‌.
\end{malayalam}}
\flushright{\begin{Arabic}
\quranayah[9][122]
\end{Arabic}}
\flushleft{\begin{malayalam}
സത്യവിശ്വാസികള്‍ ആകമാനം (യുദ്ധത്തിന്ന്‌) പുറപ്പെടാവതല്ല. എന്നാല്‍ അവരിലെ ഓരോ വിഭാഗത്തില്‍ നിന്നും ഓരോ സംഘം പുറപ്പെട്ട് പോയിക്കൂടേ ? എങ്കില്‍ (ബാക്കിയുള്ളവര്‍ക്ക് നബിയോടൊപ്പം നിന്ന്‌) മതകാര്യങ്ങളില്‍ ജ്ഞാനം നേടുവാനും തങ്ങളുടെ ആളുകള്‍ (യുദ്ധരംഗത്ത് നിന്ന്‌) അവരുടെ അടുത്തേക്ക് തിരിച്ചുവന്നാല്‍ അവര്‍ക്ക് താക്കീത് നല്‍കുവാനും കഴിയുമല്ലോ? അവര്‍ സൂക്ഷ്മത പാലിച്ചേക്കാം.
\end{malayalam}}
\flushright{\begin{Arabic}
\quranayah[9][123]
\end{Arabic}}
\flushleft{\begin{malayalam}
സത്യവിശ്വാസികളേ, നിങ്ങളുടെ അടുത്ത് താമസിക്കുന്ന സത്യനിഷേധികളോട് നിങ്ങള്‍ യുദ്ധം ചെയ്യുക. അവര്‍ നിങ്ങളില്‍ രൂക്ഷത കണ്ടെത്തണം. അല്ലാഹു സൂക്ഷ്മത പാലിക്കുന്നവരോടൊപ്പമാണെന്ന് നിങ്ങള്‍ മനസ്സിലാക്കുകയും ചെയ്യുക.
\end{malayalam}}
\flushright{\begin{Arabic}
\quranayah[9][124]
\end{Arabic}}
\flushleft{\begin{malayalam}
(ഖുര്‍ആനിലെ) ഏതെങ്കിലും ഒരു അദ്ധ്യായം അവതരിപ്പിക്കപ്പെട്ടാല്‍ അവരില്‍ ചിലര്‍ പറയും: നിങ്ങളില്‍ ആര്‍ക്കാണ് ഇത് വിശ്വാസം വര്‍ദ്ധിപ്പിച്ചു തന്നത്‌? എന്നാല്‍ സത്യവിശ്വാസികള്‍ക്കാകട്ടെ, അതവരുടെ വിശ്വാസം വര്‍ദ്ധിപ്പിക്കുക തന്നെയാണ് ചെയ്തത്‌. അവര്‍ (അതില്‍) സന്തോഷം കൊള്ളുകയും ചെയ്യുന്നു.
\end{malayalam}}
\flushright{\begin{Arabic}
\quranayah[9][125]
\end{Arabic}}
\flushleft{\begin{malayalam}
എന്നാല്‍ മനസ്സുകളില്‍ രോഗമുള്ളവര്‍ക്കാകട്ടെ അവര്‍ക്ക് അവരുടെ ദുഷ്ടതയിലേക്ക് കൂടുതല്‍ ദുഷ്ടത കൂട്ടിചേര്‍ക്കുകയാണ് അത് ചെയ്തത്‌. അവര്‍ സത്യനിഷേധികളായിരിക്കെത്തന്നെ മരിക്കുകയും ചെയ്തു.
\end{malayalam}}
\flushright{\begin{Arabic}
\quranayah[9][126]
\end{Arabic}}
\flushleft{\begin{malayalam}
അവര്‍ ഓരോ കൊല്ലവും ഒന്നോ, രണ്ടോ തവണ പരീക്ഷിക്കപ്പെട്ടുകൊണ്ടിരിക്കുന്നു എന്ന് അവര്‍ കാണുന്നില്ലേ? എന്നിട്ടും അവര്‍ ഖേദിച്ചുമടങ്ങുന്നില്ല. ചിന്തിച്ചു മനസ്സിലാക്കുന്നുമില്ല.
\end{malayalam}}
\flushright{\begin{Arabic}
\quranayah[9][127]
\end{Arabic}}
\flushleft{\begin{malayalam}
ഏതെങ്കിലും ഒരു അദ്ധ്യായം അവതരിപ്പിക്കപ്പെട്ടാല്‍ അവരില്‍ ചിലര്‍ മറ്റു ചിലരെ, നിങ്ങളെ ആരെങ്കിലും കാണുന്നുണ്ടോ എന്ന ചോദ്യഭാവത്തില്‍ നോക്കും. എന്നിട്ട് അവര്‍ തിരിഞ്ഞുകളയുകയും ചെയ്യും അവര്‍ (കാര്യം) ഗ്രഹിക്കാത്ത ഒരു ജനവിഭാഗമായതിനാല്‍ അല്ലാഹു അവരുടെ മനസ്സുകളെ തിരിച്ചുകളഞ്ഞിരിക്കുകയാണ്‌.
\end{malayalam}}
\flushright{\begin{Arabic}
\quranayah[9][128]
\end{Arabic}}
\flushleft{\begin{malayalam}
തീര്‍ച്ചയായും നിങ്ങള്‍ക്കിതാ നിങ്ങളില്‍ നിന്നുതന്നെയുള്ള ഒരു ദൂതന്‍ വന്നിരിക്കുന്നു. നിങ്ങള്‍ കഷ്ടപ്പെടുന്നത് സഹിക്കാന്‍ കഴിയാത്തവനും, നിങ്ങളുടെ കാര്യത്തില്‍ അതീവതാല്‍പര്യമുള്ളവനും, സത്യവിശ്വാസികളോട് അത്യന്തം ദയാലുവും കാരുണ്യവാനുമാണ് അദ്ദേഹം.
\end{malayalam}}
\flushright{\begin{Arabic}
\quranayah[9][129]
\end{Arabic}}
\flushleft{\begin{malayalam}
എന്നാല്‍ അവര്‍ തിരിഞ്ഞുകളയുന്ന പക്ഷം (നബിയേ,) നീ പറയുക: എനിക്ക് അല്ലാഹു മതി. അവനല്ലാതെ ഒരു ദൈവവുമില്ല. അവന്‍റെ മേലാണ് ഞാന്‍ ഭരമേല്‍പിച്ചിരിക്കുന്നത്‌. അവനാണ് മഹത്തായ സിംഹാസനത്തിന്‍റെ നാഥന്‍.
\end{malayalam}}
\chapter{\textmalayalam{യൂനുസ്}}
\begin{Arabic}
\Huge{\centerline{\basmalah}}\end{Arabic}
\flushright{\begin{Arabic}
\quranayah[10][1]
\end{Arabic}}
\flushleft{\begin{malayalam}
അലിഫ് ലാം റാ. വിജ്ഞാനപ്രദമായ വേദഗ്രന്ഥത്തിലെ വചനങ്ങളാണവ.
\end{malayalam}}
\flushright{\begin{Arabic}
\quranayah[10][2]
\end{Arabic}}
\flushleft{\begin{malayalam}
ജനങ്ങള്‍ക്ക് താക്കീത് നല്‍കുകയും, സത്യവിശ്വാസികളെ, അവര്‍ക്ക് അവരുടെ രക്ഷിതാവിങ്കല്‍ സത്യത്തിന്‍റെതായ പദവിയുണ്ട് എന്ന സന്തോഷവാര്‍ത്ത അറിയിക്കുകയും ചെയ്യുക എന്ന് അവരുടെ കൂട്ടത്തില്‍ നിന്നുതന്നെയുള്ള ഒരാള്‍ക്ക് നാം ദിവ്യസന്ദേശം നല്‍കിയത് ജനങ്ങള്‍ക്ക് ഒരു അത്ഭുതമായിപ്പോയോ? സത്യനിഷേധികള്‍ പറഞ്ഞു: ഇയാള്‍ സ്പഷ്ടമായും ഒരു മാരണക്കാരന്‍ തന്നെയാകുന്നു.
\end{malayalam}}
\flushright{\begin{Arabic}
\quranayah[10][3]
\end{Arabic}}
\flushleft{\begin{malayalam}
തീര്‍ച്ചയായും നിങ്ങളുടെ രക്ഷിതാവ് ആകാശങ്ങളും ഭൂമിയും ആറുദിവസങ്ങളിലായി സൃഷ്ടിക്കുകയും, പിന്നീട് കാര്യങ്ങള്‍ നിയന്ത്രിച്ചു കൊണ്ട് സിംഹാസനസ്ഥനാവുകയും ചെയ്ത അല്ലാഹുവാകുന്നു. അവന്‍റെ അനുവാദത്തിന് ശേഷമല്ലാതെ യാതൊരു ശുപാര്‍ശക്കാരനും ശുപാര്‍ശ നടത്തുന്നതല്ല. അവനത്രെ നിങ്ങളുടെ രക്ഷിതാവായ അല്ലാഹു. അതിനാല്‍ അവനെ നിങ്ങള്‍ ആരാധിക്കുക. നിങ്ങള്‍ ചിന്തിച്ചു മനസ്സിലാക്കുന്നില്ലേ?
\end{malayalam}}
\flushright{\begin{Arabic}
\quranayah[10][4]
\end{Arabic}}
\flushleft{\begin{malayalam}
അവങ്കലേക്കാണ് നിങ്ങളുടെയെല്ലാം മടക്കം. അല്ലാഹുവിന്‍റെ സത്യവാഗ്ദാനമത്രെ അത്‌. തീര്‍ച്ചയായും അവന്‍ സൃഷ്ടി ആരംഭിക്കുന്നു. വിശ്വസിക്കുകയും സല്‍കര്‍മ്മങ്ങള്‍ പ്രവര്‍ത്തിക്കുകയും ചെയ്തവര്‍ക്ക് നീതിപൂര്‍വ്വം പ്രതിഫലം നല്‍കുവാന്‍ വേണ്ടി അവന്‍ സൃഷ്ടികര്‍മ്മം ആവര്‍ത്തിക്കുകയും ചെയ്യുന്നു. എന്നാല്‍ നിഷേധിച്ചതാരോ അവര്‍ക്ക് ചുട്ടുതിളയ്ക്കുന്ന പാനീയവും വേദനയേറിയ ശിക്ഷയും ഉണ്ടായിരിക്കും. അവര്‍ നിഷേധിച്ചിരുന്നതിന്‍റെ ഫലമത്രെ അത്‌.
\end{malayalam}}
\flushright{\begin{Arabic}
\quranayah[10][5]
\end{Arabic}}
\flushleft{\begin{malayalam}
സൂര്യനെ ഒരു പ്രകാശമാക്കിയത് അവനാകുന്നു. ചന്ദ്രനെ അവനൊരു ശോഭയാക്കുകയും, അതിന് ഘട്ടങ്ങള്‍ നിര്‍ണയിക്കുകയും ചെയ്തിരിക്കുന്നു. നിങ്ങള്‍ കൊല്ലങ്ങളുടെ എണ്ണവും കണക്കും അറിയുന്നതിന് വേണ്ടി. യഥാര്‍ത്ഥ മുറപ്രകാരമല്ലാതെ അല്ലാഹു അതൊന്നും സൃഷ്ടിച്ചിട്ടില്ല. മനസ്സിലാക്കുന്ന ആളുകള്‍ക്കു വേണ്ടി അല്ലാഹു തെളിവുകള്‍ വിശദീകരിക്കുന്നു.
\end{malayalam}}
\flushright{\begin{Arabic}
\quranayah[10][6]
\end{Arabic}}
\flushleft{\begin{malayalam}
തീര്‍ച്ചയായും രാപകലുകള്‍ വ്യത്യാസപ്പെടുന്നതിലും, ആകാശങ്ങളിലും ഭൂമിയിലും അല്ലാഹു സൃഷ്ടിച്ചിട്ടുള്ളവയിലും സൂക്ഷ്മത പാലിക്കുന്ന ആളുകള്‍ക്ക് പല തെളിവുകളുമുണ്ട്‌.
\end{malayalam}}
\flushright{\begin{Arabic}
\quranayah[10][7]
\end{Arabic}}
\flushleft{\begin{malayalam}
നമ്മെ കണ്ടുമുട്ടും എന്ന് പ്രതീക്ഷിക്കാത്തവരും, ഇഹലോകജീവിതം കൊണ്ട് തൃപ്തിപ്പെടുകയും, അതില്‍ സമാധാനമടയുകയും ചെയ്തവരും, നമ്മുടെ തെളിവുകളെപ്പറ്റി അശ്രദ്ധരായി കഴിയുന്നവരും ആരോ.
\end{malayalam}}
\flushright{\begin{Arabic}
\quranayah[10][8]
\end{Arabic}}
\flushleft{\begin{malayalam}
അവരുടെ സങ്കേതം നരകം തന്നെയാകുന്നു. അവര്‍ പ്രവര്‍ത്തിച്ചു കൊണ്ടിരുന്നതിന്‍റെ ഫലമായിട്ടത്രെ അത്‌.
\end{malayalam}}
\flushright{\begin{Arabic}
\quranayah[10][9]
\end{Arabic}}
\flushleft{\begin{malayalam}
തീര്‍ച്ചയായും വിശ്വസിക്കുകയും സല്‍കര്‍മ്മങ്ങള്‍ പ്രവര്‍ത്തിക്കുകയും ചെയ്തവരാരോ, അവരുടെ വിശ്വാസത്തിന്‍റെ ഫലമായി അവരുടെ രക്ഷിതാവ് അവരെ നേര്‍വഴിയിലാക്കുന്നതാണ്‌. അനുഗ്രഹങ്ങള്‍ നിറഞ്ഞ സ്വര്‍ഗത്തോപ്പുകളില്‍ അവരുടെ താഴ്ഭാഗത്തു കൂടി അരുവികള്‍ ഒഴുകിക്കൊണ്ടിരിക്കും.
\end{malayalam}}
\flushright{\begin{Arabic}
\quranayah[10][10]
\end{Arabic}}
\flushleft{\begin{malayalam}
അതിനകത്ത് അവരുടെ പ്രാര്‍ത്ഥന അല്ലാഹുവേ, നിനക്ക് സ്തോത്രം എന്നായിരിക്കും. അതിനകത്ത് അവര്‍ക്കുള്ള അഭിവാദ്യം സമാധാനം! എന്നായിരിക്കും.അവരുടെ പ്രാര്‍ത്ഥനയുടെ അവസാനം ലോകരക്ഷിതാവായ അല്ലാഹുവിന് സ്തുതി എന്നായിരിക്കും.
\end{malayalam}}
\flushright{\begin{Arabic}
\quranayah[10][11]
\end{Arabic}}
\flushleft{\begin{malayalam}
ജനങ്ങള്‍ നേട്ടത്തിന് ധൃതികൂട്ടുന്നതു പോലെ അവര്‍ക്ക് ദോഷം വരുത്തുന്ന കാര്യത്തില്‍ അല്ലാഹു ധൃതികൂട്ടുകയായിരുന്നുവെങ്കില്‍ അവരുടെ ജീവിതാവധി അവസാനിപ്പിക്കപ്പെടുക തന്നെ ചെയ്യുമായിരുന്നു. എന്നാല്‍ നമ്മെ കണ്ടുമുട്ടുമെന്ന് പ്രതീക്ഷിക്കാത്തവരെ അവരുടെ ധിക്കാരത്തില്‍ വിഹരിച്ചു കൊള്ളാന്‍ നാം വിടുകയാകുന്നു.
\end{malayalam}}
\flushright{\begin{Arabic}
\quranayah[10][12]
\end{Arabic}}
\flushleft{\begin{malayalam}
മനുഷ്യന് കഷ്ടത ബാധിച്ചാല്‍ കിടന്നിട്ടോ ഇരുന്നിട്ടോ നിന്നിട്ടോ അവന്‍ നമ്മോട് പ്രാര്‍ത്ഥിക്കുന്നു. അങ്ങനെ അവനില്‍ നിന്ന് നാം കഷ്ടത നീക്കികൊടുത്താല്‍, അവനെ ബാധിച്ച കഷ്ടതയുടെ കാര്യത്തില്‍ നമ്മോടവന്‍ പ്രാര്‍ത്ഥിച്ചിട്ടേയില്ല എന്ന ഭാവത്തില്‍ അവന്‍ നടന്നു പോകുന്നു. അതിരുകവിയുന്നവര്‍ക്ക് അപ്രകാരം, അവര്‍ ചെയ്തുകൊണ്ടിരിക്കുന്നത് അലങ്കാരമായി തോന്നിക്കപ്പെട്ടിരിക്കുന്നു.
\end{malayalam}}
\flushright{\begin{Arabic}
\quranayah[10][13]
\end{Arabic}}
\flushleft{\begin{malayalam}
തീര്‍ച്ചയായും നിങ്ങള്‍ക്ക് മുമ്പുള്ള പല തലമുറകളെയും അവര്‍ അക്രമം പ്രവര്‍ത്തിച്ചപ്പോള്‍ നാം നശിപ്പിച്ചിട്ടുണ്ട്‌. വ്യക്തമായ തെളിവുകളുമായി നമ്മുടെ ദൂതന്‍മാര്‍ അവരുടെ അടുത്ത് ചെല്ലുകയുണ്ടായി. അവര്‍ വിശ്വസിക്കുകയുണ്ടായില്ല. അപ്രകാരമാണ് കുറ്റവാളികളായ ജനങ്ങള്‍ക്കു നാം പ്രതിഫലം നല്‍കുന്നത്‌.
\end{malayalam}}
\flushright{\begin{Arabic}
\quranayah[10][14]
\end{Arabic}}
\flushleft{\begin{malayalam}
പിന്നെ, അവര്‍ക്ക് ശേഷം നിങ്ങളെ നാം ഭൂമിയില്‍ പിന്‍ഗാമികളാക്കി. നിങ്ങള്‍ എങ്ങനെ പ്രവര്‍ത്തിക്കുന്നു എന്ന് നാം നോക്കുവാന്‍ വേണ്ടി.
\end{malayalam}}
\flushright{\begin{Arabic}
\quranayah[10][15]
\end{Arabic}}
\flushleft{\begin{malayalam}
നമ്മുടെ സ്പഷ്ടമായ തെളിവുകള്‍ അവര്‍ക്ക് വായിച്ചുകേള്‍പിക്കപ്പെടുമ്പോള്‍, നമ്മെ കണ്ടുമുട്ടുമെന്ന് പ്രതീക്ഷിക്കാത്തവര്‍ പറയും: നീ ഇതല്ലാത്ത ഒരു ഖുര്‍ആന്‍ കൊണ്ടു വരികയോ, ഇതില്‍ ഭേദഗതി വരുത്തുകയോ ചെയ്യുക. (നബിയേ,) പറയുക: എന്‍റെ സ്വന്തം വകയായി അത് ഭേദഗതി ചെയ്യുവാന്‍ എനിക്ക് പാടുള്ളതല്ല. എനിക്ക് ബോധനം നല്‍കപ്പെടുന്നതിനെ പിന്‍പറ്റുക മാത്രമാണ് ഞാന്‍ ചെയ്യുന്നത്‌. തീര്‍ച്ചയായും എന്‍റെ രക്ഷിതാവിനെ ഞാന്‍ ധിക്കരിക്കുന്ന പക്ഷം ഭയങ്കരമായ ഒരു ദിവസത്തെ ശിക്ഷ ഞാന്‍ പേടിക്കുന്നു.
\end{malayalam}}
\flushright{\begin{Arabic}
\quranayah[10][16]
\end{Arabic}}
\flushleft{\begin{malayalam}
പറയുക: അല്ലാഹു ഉദ്ദേശിച്ചിരുന്നെങ്കില്‍ നിങ്ങള്‍ക്ക് ഞാനിത് ഓതികേള്‍പിക്കുകയോ, നിങ്ങളെ അവന്‍ ഇത് അറിയിക്കുകയോ ചെയ്യില്ലായിരുന്നു. ഇതിനു മുമ്പ് കുറെ കാലം ഞാന്‍ നിങ്ങള്‍ക്കിടയില്‍ ജീവിച്ചിട്ടുണ്ടല്ലോ. നിങ്ങള്‍ ചിന്തിക്കുന്നില്ലേ?
\end{malayalam}}
\flushright{\begin{Arabic}
\quranayah[10][17]
\end{Arabic}}
\flushleft{\begin{malayalam}
അപ്പോള്‍ അല്ലാഹുവിന്‍റെ പേരില്‍ കള്ളം കെട്ടിച്ചമയ്ക്കുകയോ, അവന്‍റെ ദൃഷ്ടാന്തങ്ങളെ നിഷേധിച്ചു തള്ളുകയോ ചെയ്തവനെക്കാള്‍ കടുത്ത അക്രമി ആരുണ്ട്‌? തീര്‍ച്ചയായും കുറ്റവാളികള്‍ വിജയം പ്രാപിക്കുകയില്ല.
\end{malayalam}}
\flushright{\begin{Arabic}
\quranayah[10][18]
\end{Arabic}}
\flushleft{\begin{malayalam}
അല്ലാഹുവിന് പുറമെ, അവര്‍ക്ക് ഉപദ്രവമോ ഉപകാരമോ ചെയ്യാത്തതിനെ അവര്‍ ആരാധിച്ചു കൊണ്ടിരിക്കുന്നു. ഇവര്‍ (ആരാധ്യര്‍) അല്ലാഹുവിന്‍റെ അടുക്കല്‍ ഞങ്ങള്‍ക്കുള്ള ശുപാര്‍ശക്കാരാണ് എന്ന് പറയുകയും ചെയ്യുന്നു. (നബിയേ,) പറയുക: ആകാശങ്ങളിലോ ഭൂമിയിലോ ഉള്ളതായി അല്ലാഹുവിനറിയാത്ത വല്ലകാര്യവും നിങ്ങളവന്ന് അറിയിച്ചു കൊടുക്കുകയാണോ? അല്ലാഹു അവര്‍ പങ്കുചേര്‍ക്കുന്നതില്‍ നിന്നെല്ലാം എത്രയോ പരിശുദ്ധനും ഉന്നതനുമായിരിക്കുന്നു.
\end{malayalam}}
\flushright{\begin{Arabic}
\quranayah[10][19]
\end{Arabic}}
\flushleft{\begin{malayalam}
മനുഷ്യര്‍ ഒരൊറ്റ സമൂഹം മാത്രമായിരുന്നു. എന്നിട്ടവര്‍ ഭിന്നിച്ചിരിക്കുകയാണ്‌. നിന്‍റെ രക്ഷിതാവിങ്കല്‍ നിന്ന് ഒരു വചനം മുന്‍കൂട്ടി ഉണ്ടായിരുന്നില്ലെങ്കില്‍ അവര്‍ ഭിന്നിച്ചു കൊണ്ടിരിക്കുന്ന വിഷയത്തില്‍ അവര്‍ക്കിടയില്‍ (ഇതിനകം) തീര്‍പ്പുകല്‍പിക്കപ്പെട്ടിരുന്നേനെ.
\end{malayalam}}
\flushright{\begin{Arabic}
\quranayah[10][20]
\end{Arabic}}
\flushleft{\begin{malayalam}
അവര്‍ പറയുന്നു: അദ്ദേഹത്തിന് (നബിക്ക്‌) തന്‍റെ രക്ഷിതാവിങ്കല്‍ നിന്ന് ഒരു തെളിവ് (നേരിട്ട്‌) ഇറക്കികൊടുക്കപ്പെടാത്തതെന്തുകൊണ്ട്‌? (നബിയേ,) പറയുക: അദൃശ്യജ്ഞാനം അല്ലാഹുവിന് മാത്രമാകുന്നു. അതിനാല്‍ നിങ്ങള്‍ കാത്തിരിക്കൂ. തീര്‍ച്ചയായും ഞാനും നിങ്ങളോടൊപ്പം കാത്തിരിക്കുന്നവരുടെ കൂട്ടത്തിലാകുന്നു.
\end{malayalam}}
\flushright{\begin{Arabic}
\quranayah[10][21]
\end{Arabic}}
\flushleft{\begin{malayalam}
ജനങ്ങള്‍ക്കു കഷ്ടത ബാധിച്ചതിനു ശേഷം നാമവര്‍ക്ക് ഒരു കാരുണ്യം അനുഭവിപ്പിച്ചാല്‍ അപ്പോഴതാ നമ്മുടെ ദൃഷ്ടാന്തങ്ങളുടെ കാര്യത്തില്‍ അവരുടെ ഒരു കുതന്ത്രം.! പറയുക: അല്ലാഹു അതിവേഗം തന്ത്രം പ്രയോഗിക്കുന്നവനാകുന്നു. നിങ്ങള്‍ തന്ത്രം പ്രയോഗിച്ചു കൊണ്ടിരിക്കുന്നത് നമ്മുടെ ദൂതന്‍മാര്‍ രേഖപ്പെടുത്തുന്നതാണ്‌; തീര്‍ച്ച.
\end{malayalam}}
\flushright{\begin{Arabic}
\quranayah[10][22]
\end{Arabic}}
\flushleft{\begin{malayalam}
അവനാകുന്നു കരയിലും കടലിലും നിങ്ങള്‍ക്ക് സഞ്ചാരസൌകര്യം നല്‍കുന്നത്‌. അങ്ങനെ നിങ്ങള്‍ കപ്പലുകളിലായിരിക്കുകയും, നല്ല ഒരു കാറ്റ് നിമിത്തം യാത്രക്കാരെയും കൊണ്ട് അവ സഞ്ചരിക്കുകയും, അവരതില്‍ സന്തുഷ്ടരായിരിക്കുകയും ചെയ്തപ്പോഴതാ ഒരു കൊടുങ്കാറ്റ് അവര്‍ക്ക് വന്നെത്തി. എല്ലായിടത്തുനിന്നും തിരമാലകള്‍ അവരുടെ നേര്‍ക്ക് വന്നു. തങ്ങള്‍ വലയം ചെയ്യപ്പെട്ടിരിക്കുന്നു എന്ന് അവര്‍ വിചാരിച്ചു. അപ്പോള്‍ കീഴ്‌വണക്കം അല്ലാഹുവിന്ന് നിഷ്കളങ്കമാക്കിക്കൊണ്ട് അവനോടവര്‍ പ്രാര്‍ത്ഥിച്ചു: ഞങ്ങളെ നീ ഇതില്‍ നിന്ന് രക്ഷപ്പെടുത്തുന്ന പക്ഷം തീര്‍ച്ചയായും ഞങ്ങള്‍ നന്ദിയുള്ളവരുടെ കൂട്ടത്തിലായിരിക്കും.
\end{malayalam}}
\flushright{\begin{Arabic}
\quranayah[10][23]
\end{Arabic}}
\flushleft{\begin{malayalam}
അങ്ങനെ അല്ലാഹു അവരെ രക്ഷപ്പെടുത്തിയപ്പോള്‍ അവരതാ ന്യായമില്ലാതെ ഭൂമിയില്‍ അതിക്രമം പ്രവര്‍ത്തിക്കുന്നുഃഏ; മനുഷ്യരേ, നിങ്ങള്‍ ചെയ്യുന്ന അതിക്രമം നിങ്ങള്‍ക്കെതിരില്‍ തന്നെയായിരിക്കും (ഭവിക്കുക.) ഇഹലോകജീവിതത്തിലെ സുഖാനുഭവം മാത്രമാണ് (അത് വഴി നിങ്ങള്‍ക്ക് കിട്ടുന്നത്‌) . പിന്നെ നമ്മുടെ അടുത്തേക്കാണ് നിങ്ങളുടെ മടക്കം. അപ്പോള്‍ നിങ്ങള്‍ പ്രവര്‍ത്തിച്ചു കൊണ്ടിരുന്നതിനെപ്പറ്റി നിങ്ങളെ നാം വിവരമറിയിക്കുന്നതാണ്‌.
\end{malayalam}}
\flushright{\begin{Arabic}
\quranayah[10][24]
\end{Arabic}}
\flushleft{\begin{malayalam}
നാം ആകാശത്ത് നിന്ന് വെള്ളം ഇറക്കിയിട്ട് അതുമൂലം മനുഷ്യര്‍ക്കും കാലികള്‍ക്കും ഭക്ഷിക്കാനുള്ള ഭൂമിയിലെ സസ്യങ്ങള്‍ ഇടകലര്‍ന്നു വളര്‍ന്നു. അങ്ങനെ ഭൂമി അതിന്‍റെ അലങ്കാരമണിയുകയും, അത് അഴകാര്‍ന്നതാകുകയും, അവയൊക്കെ കരസ്ഥമാക്കാന്‍ തങ്ങള്‍ക്ക് കഴിയുമാറായെന്ന് അതിന്‍റെ ഉടമസ്ഥര്‍ വിചാരിക്കുകയും ചെയ്തപ്പോഴതാ ഒരു രാത്രിയോ പകലോ നമ്മുടെ കല്‍പന അതിന് വന്നെത്തുകയും, തലേദിവസം അവയൊന്നും അവിടെ നിലനിന്നിട്ടേയില്ലാത്ത മട്ടില്‍ നാമവയെ ഉന്‍മൂലനം ചെയ്യപ്പെട്ട അവസ്ഥയിലാക്കുകയും ചെയ്യുന്നു. ഇതുപോലെ മാത്രമാകുന്നു ഐഹികജീവിതത്തിന്‍റെ ഉപമ. ചിന്തിക്കുന്ന ആളുകള്‍ക്കു വേണ്ടി അപ്രകാരം നാം തെളിവുകള്‍ വിശദീകരിക്കുന്നു.
\end{malayalam}}
\flushright{\begin{Arabic}
\quranayah[10][25]
\end{Arabic}}
\flushleft{\begin{malayalam}
അല്ലാഹു ശാന്തിയുടെ ഭവനത്തിലേക്ക് ക്ഷണിക്കുന്നു. അവന്‍ ഉദ്ദേശിക്കുന്നവരെ അവന്‍ നേരായ പാതയിലേക്ക് നയിക്കുകയും ചെയ്യുന്നു.
\end{malayalam}}
\flushright{\begin{Arabic}
\quranayah[10][26]
\end{Arabic}}
\flushleft{\begin{malayalam}
സുകൃതം ചെയ്തവര്‍ക്ക് ഏറ്റവും ഉത്തമമായ പ്രതിഫലവും കൂടുതല്‍ നേട്ടവുമുണ്ട്‌. ഇരുളോ അപമാനമോ അവരുടെ മുഖത്തെ തീണ്ടുകയില്ല. അവരാകുന്നു സ്വര്‍ഗാവകാശികള്‍. അവരതില്‍ നിത്യവാസികളായിരിക്കും.
\end{malayalam}}
\flushright{\begin{Arabic}
\quranayah[10][27]
\end{Arabic}}
\flushleft{\begin{malayalam}
തിന്‍മകള്‍ പ്രവര്‍ത്തിച്ചവര്‍ക്കാകട്ടെ തിന്‍മയ്ക്കുള്ള പ്രതിഫലം അതിന് തുല്യമായതു തന്നെയായിരിക്കും. അപമാനം അവരെ ബാധിക്കുകയും ചെയ്യും. അല്ലാഹുവില്‍ നിന്ന് അവരെ രക്ഷിക്കുന്ന ഒരാളുമില്ല. ഇരുണ്ട രാവിന്‍റെ കഷ്ണങ്ങള്‍കൊണ്ട് അവരുടെ മുഖങ്ങള്‍ പൊതിഞ്ഞതു പോലെയിരിക്കും. അവരാകുന്നു നരകാവകാശികള്‍. അവരതില്‍ നിത്യവാസികളായിരിക്കും.
\end{malayalam}}
\flushright{\begin{Arabic}
\quranayah[10][28]
\end{Arabic}}
\flushleft{\begin{malayalam}
അവരെയെല്ലാം നാം ഒരുമിച്ചുകൂട്ടുകയും, എന്നിട്ട് ബഹുദൈവവിശ്വാസികളോട് നിങ്ങളും നിങ്ങള്‍ പങ്കാളികളായി ചേര്‍ത്തവരും അവിടെത്തന്നെ നില്‍ക്കൂ. എന്ന് പറയുകയും ചെയ്യുന്ന ദിവസം (ശ്രദ്ധേയമത്രെ.) അനന്തരം നാം അവരെ തമ്മില്‍ വേര്‍പെടുത്തും. അവര്‍ പങ്കാളികളായി ചേര്‍ത്തവര്‍ പറയും: നിങ്ങള്‍ ഞങ്ങളെയല്ല ആരാധിച്ചിരുന്നത്‌.
\end{malayalam}}
\flushright{\begin{Arabic}
\quranayah[10][29]
\end{Arabic}}
\flushleft{\begin{malayalam}
അതിനാല്‍ ഞങ്ങള്‍ക്കും നിങ്ങള്‍ക്കുമിടയില്‍ സാക്ഷിയായി അല്ലാഹു മതി. നിങ്ങളുടെ ആരാധനയെപ്പറ്റി ഞങ്ങള്‍ തീര്‍ത്തും അറിവില്ലാത്തവരായിരുന്നു.
\end{malayalam}}
\flushright{\begin{Arabic}
\quranayah[10][30]
\end{Arabic}}
\flushleft{\begin{malayalam}
അവിടെവെച്ച് ഓരോ ആത്മാവും അത് മുന്‍കൂട്ടി ചെയ്തത് പരീക്ഷിച്ചറിയും. അവരുടെ യഥാര്‍ത്ഥ രക്ഷാധികാരിയായ അല്ലാഹുവിങ്കലേക്ക് അവര്‍ മടക്കപ്പെടുകയും, അവര്‍ പറഞ്ഞുണ്ടാക്കിയിരുന്നതെല്ലാം അവരില്‍ നിന്ന് തെറ്റിപ്പോകുകയും ചെയ്യുന്നതാണ്‌.
\end{malayalam}}
\flushright{\begin{Arabic}
\quranayah[10][31]
\end{Arabic}}
\flushleft{\begin{malayalam}
പറയുക: ആകാശത്തുനിന്നും ഭൂമിയില്‍ നിന്നും നിങ്ങള്‍ക്ക് ആഹാരം നല്‍കുന്നത് ആരാണ്‌? അതല്ലെങ്കില്‍ കേള്‍വിയും കാഴ്ചകളും അധീനപ്പെടുത്തുന്നത് ആരാണ്‌? ജീവനില്ലാത്തതില്‍ നിന്ന് ജീവനുള്ളതും, ജീവനുള്ളതില്‍ നിന്ന് ജീവനില്ലാത്തതും പുറപ്പെടുവിക്കുന്നതും ആരാണ്‌? കാര്യങ്ങള്‍ നിയന്ത്രിക്കുന്നതും ആരാണ്‌? അവര്‍ പറയും: അല്ലാഹു എന്ന്‌. അപ്പോള്‍ പറയുക: എന്നിട്ടും നിങ്ങള്‍ സൂക്ഷ്മത പാലിക്കുന്നില്ലേ?
\end{malayalam}}
\flushright{\begin{Arabic}
\quranayah[10][32]
\end{Arabic}}
\flushleft{\begin{malayalam}
അവനാണ് നിങ്ങളുടെ യഥാര്‍ത്ഥ രക്ഷിതാവായ അല്ലാഹു. എന്നിരിക്കെ യഥാര്‍ത്ഥമായുള്ളതിന് പുറമെ വഴികേടല്ലാതെ എന്താണുള്ളത്‌? അപ്പോള്‍ എങ്ങനെയാണ് നിങ്ങള്‍ തെറ്റിക്കപ്പെടുന്നത്‌?
\end{malayalam}}
\flushright{\begin{Arabic}
\quranayah[10][33]
\end{Arabic}}
\flushleft{\begin{malayalam}
അപ്രകാരം ധിക്കാരം കൈക്കൊണ്ടവരുടെ കാര്യത്തില്‍, അവര്‍ വിശ്വസിക്കുകയില്ല എന്നുള്ള നിന്‍റെ രക്ഷിതാവിന്‍റെ വചനം സത്യമായിരിക്കുന്നു.
\end{malayalam}}
\flushright{\begin{Arabic}
\quranayah[10][34]
\end{Arabic}}
\flushleft{\begin{malayalam}
(നബിയേ,) പറയുക: സൃഷ്ടി ആദ്യമായി തുടങ്ങുകയും പിന്നെ അത് ആവര്‍ത്തിക്കുകയും ചെയ്യുന്ന വല്ലവരും നിങ്ങള്‍ പങ്കാളികളായി ചേര്‍ത്തവരുടെ കൂട്ടത്തിലുണ്ടോ? പറയുക: അല്ലാഹുവാണ് സൃഷ്ടി ആദ്യമായി തുടങ്ങുകയും പിന്നെ അത് ആവര്‍ത്തിക്കുകയും ചെയ്യുന്നത്‌. എന്നിരിക്കെ നിങ്ങള്‍ എങ്ങനെയാണ് തെറ്റിക്കപ്പെടുന്നത്‌?
\end{malayalam}}
\flushright{\begin{Arabic}
\quranayah[10][35]
\end{Arabic}}
\flushleft{\begin{malayalam}
(നബിയേ,) പറയുക: സത്യത്തിലേക്ക് വഴി കാട്ടുന്ന വല്ലവരും നിങ്ങള്‍ പങ്കാളികളായി ചേര്‍ത്തവരുടെ കൂട്ടത്തിലുണ്ടോ? പറയുക: അല്ലാഹുവത്രെ സത്യത്തിലേക്ക് വഴി കാട്ടുന്നത്‌. ആകയാല്‍ സത്യത്തിലേക്ക് വഴി കാണിക്കുന്നവനാണോ, അതല്ല, ആരെങ്കിലും വഴി കാണിച്ചെങ്കിലല്ലാതെ നേര്‍മാര്‍ഗം പ്രാപിക്കാത്തവനാണോ പിന്തുടരാന്‍ കൂടുതല്‍ അര്‍ഹതയുള്ളവന്‍? അപ്പോള്‍ നിങ്ങള്‍ക്കെന്തുപറ്റി? എങ്ങനെയാണ് നിങ്ങള്‍ വിധി കല്‍പിക്കുന്നത്‌?
\end{malayalam}}
\flushright{\begin{Arabic}
\quranayah[10][36]
\end{Arabic}}
\flushleft{\begin{malayalam}
അവരില്‍ അധികപേരും ഊഹത്തെ മാത്രമാണ് പിന്തുടരുന്നത്‌. തീര്‍ച്ചയായും സത്യത്തിന്‍റെ സ്ഥാനത്ത് ഊഹം ഒട്ടും പര്യാപ്തമാകുകയില്ല. തീര്‍ച്ചയായും അല്ലാഹു അവര്‍ ചെയ്തു കൊണ്ടിരിക്കുന്നതെല്ലാം അറിയുന്നവനാകുന്നു.
\end{malayalam}}
\flushright{\begin{Arabic}
\quranayah[10][37]
\end{Arabic}}
\flushleft{\begin{malayalam}
അല്ലാഹുവിന് പുറമെ (മറ്റാരാലും) ഈ ഖുര്‍ആന്‍ കെട്ടിച്ചമയ്ക്കപ്പെടാവുന്നതല്ല. പ്രത്യുത അതിന്‍റെ മുമ്പുള്ള ദിവ്യസന്ദേശത്തെ സത്യപ്പെടുത്തുന്നതും, ദൈവികപ്രമാണത്തിന്‍റെ വിശദീകരണവുമത്രെ അത്‌. അതില്‍ യാതൊരു സംശയവുമില്ല. ലോകരക്ഷിതാവിങ്കല്‍ നിന്നുള്ളതാണത്‌.
\end{malayalam}}
\flushright{\begin{Arabic}
\quranayah[10][38]
\end{Arabic}}
\flushleft{\begin{malayalam}
അതല്ല, അദ്ദേഹം (നബി) അത് കെട്ടിച്ചമച്ചതാണ് എന്നാണോ അവര്‍ പറയുന്നത്‌? (നബിയേ,) പറയുക: എന്നാല്‍ അതിന്ന് തുല്യമായ ഒരു അദ്ധ്യായം നിങ്ങള്‍ കൊണ്ടു വരൂ. അല്ലാഹുവിന് പുറമെ നിങ്ങള്‍ക്ക് സാധിക്കുന്നവരെയെല്ലാം വിളിച്ചുകൊള്ളുകയും ചെയ്യുക. നിങ്ങള്‍ സത്യവാന്‍മാരാണെങ്കില്‍.
\end{malayalam}}
\flushright{\begin{Arabic}
\quranayah[10][39]
\end{Arabic}}
\flushleft{\begin{malayalam}
അല്ല, മുഴുവന്‍ വശവും അവര്‍ സൂക്ഷ്മമായി അറിഞ്ഞിട്ടില്ലാത്ത, അനുഭവസാക്ഷ്യം അവര്‍ക്കു വന്നു കഴിഞ്ഞിട്ടില്ലാത്ത ഒരു കാര്യത്തെ അവര്‍ നിഷേധിച്ചു തള്ളിയിരിക്കയാണ്‌. അപ്രകാരം തന്നെയാണ് അവരുടെ മുമ്പുള്ളവരും നിഷേധിച്ചു തള്ളിയത്‌. എന്നിട്ട് ആ അക്രമികളുടെ പര്യവസാനം എങ്ങനെയായിരുന്നു എന്ന് നോക്കൂ.
\end{malayalam}}
\flushright{\begin{Arabic}
\quranayah[10][40]
\end{Arabic}}
\flushleft{\begin{malayalam}
അതില്‍ (ഖുര്‍ആനില്‍) വിശ്വസിക്കുന്ന ചിലര്‍ അവരുടെ കൂട്ടത്തിലുണ്ട്‌. അതില്‍ വിശ്വസിക്കാത്ത ചിലരും അവരുടെ കൂട്ടത്തിലുണ്ട്‌. നിന്‍റെ രക്ഷിതാവ് കുഴപ്പമുണ്ടാക്കുന്നവരെപ്പറ്റി നല്ലവണ്ണം അറിയുന്നവനാകുന്നു.
\end{malayalam}}
\flushright{\begin{Arabic}
\quranayah[10][41]
\end{Arabic}}
\flushleft{\begin{malayalam}
അവര്‍ നിന്നെ നിഷേധിച്ചു തള്ളുകയാണെങ്കില്‍ നീ പറഞ്ഞേക്കുക. എനിക്കുള്ളത് എന്‍റെ കര്‍മ്മമാകുന്നു. നിങ്ങള്‍ക്കുള്ളത് നിങ്ങളുടെ കര്‍മ്മവും. ഞാന്‍ പ്രവര്‍ത്തിക്കുന്നതില്‍ നിന്ന് നിങ്ങള്‍ വിമുക്തരാണ്‌. നിങ്ങള്‍ പ്രവര്‍ത്തിക്കുന്നതില്‍ നിന്ന് ഞാനും വിമുക്തനാണ്‌.
\end{malayalam}}
\flushright{\begin{Arabic}
\quranayah[10][42]
\end{Arabic}}
\flushleft{\begin{malayalam}
അവരുടെ കൂട്ടത്തില്‍ നീ പറയുന്നത് ശ്രദ്ധിച്ച് കേള്‍ക്കുന്ന ചിലരുണ്ട്‌. എന്നാല്‍ ബധിരന്‍മാരെ - അവര്‍ ചിന്തിക്കാന്‍ ഭാവമില്ലെങ്കിലും -നിനക്ക് കേള്‍പിക്കാന്‍ കഴിയുമോ?
\end{malayalam}}
\flushright{\begin{Arabic}
\quranayah[10][43]
\end{Arabic}}
\flushleft{\begin{malayalam}
അവരുടെ കൂട്ടത്തില്‍ നിന്നെ ഉറ്റുനോക്കുന്ന ചിലരുമുണ്ട്‌. എന്നാല്‍ അന്ധന്‍മാര്‍ക്ക്‌- അവര്‍ കണ്ടറിയാന്‍ ഭാവമില്ലെങ്കിലും- നേര്‍വഴി കാണിക്കുവാന്‍ നിനക്ക് സാധിക്കുമോ?
\end{malayalam}}
\flushright{\begin{Arabic}
\quranayah[10][44]
\end{Arabic}}
\flushleft{\begin{malayalam}
തീര്‍ച്ചയായും അല്ലാഹു മനുഷ്യരോട് ഒട്ടും അനീതി കാണിക്കുന്നില്ല. എങ്കിലും മനുഷ്യര്‍ അവരവരോട് തന്നെ അനീതി കാണിക്കുന്നു.
\end{malayalam}}
\flushright{\begin{Arabic}
\quranayah[10][45]
\end{Arabic}}
\flushleft{\begin{malayalam}
അവന്‍ അവരെ ഒരുമിച്ചുകൂട്ടുന്ന ദിവസം പകലില്‍ നിന്ന് അല്‍പസമയം മാത്രമേ അവര്‍ (ഇഹലോകത്ത്‌) കഴിച്ചുകൂട്ടിയിട്ടുള്ളൂ എന്ന പോലെ തോന്നും. അവര്‍ അന്യോന്യം തിരിച്ചറിയുന്നതുമാണ്‌. അല്ലാഹുവുമായി കണ്ടുമുട്ടുന്നതിനെ നിഷേധിച്ചുതള്ളിയവര്‍ നഷ്ടത്തിലായിരിക്കുന്നു. അവര്‍ സന്‍മാര്‍ഗം പ്രാപിക്കുന്നവരായതുമില്ല.
\end{malayalam}}
\flushright{\begin{Arabic}
\quranayah[10][46]
\end{Arabic}}
\flushleft{\begin{malayalam}
(നബിയേ,) അവര്‍ക്കു നാം വാഗ്ദാനം ചെയ്തുകൊണ്ടിരിക്കുന്ന ശിക്ഷകളില്‍ ചിലത് നാം നിനക്ക് കാണിച്ചുതരികയോ, അല്ലെങ്കില്‍ (അതിനു മുമ്പ്‌) നിന്നെ നാം മരിപ്പിക്കുകയോ ചെയ്യുന്ന പക്ഷം നമ്മുടെ അടുത്തേക്ക് തന്നെയാണ് അവരുടെ മടക്കം. പിന്നെ അവര്‍ ചെയ്തു കൊണ്ടിരിക്കുന്നതിനെല്ലാം അല്ലാഹു സാക്ഷിയായിരിക്കും.
\end{malayalam}}
\flushright{\begin{Arabic}
\quranayah[10][47]
\end{Arabic}}
\flushleft{\begin{malayalam}
ഓരോ സമൂഹത്തിനും ഓരോ ദൂതനുണ്ട്‌. അങ്ങനെ അവരിലേക്കുള്ള ദൂതന്‍ വന്നാല്‍ അവര്‍ക്കിടയല്‍ നീതിപൂര്‍വ്വം തീരുമാനമെടുക്കപ്പെടുന്നതാണ്‌. അവരോട് അനീതി കാണിക്കപ്പെടുന്നതല്ല.
\end{malayalam}}
\flushright{\begin{Arabic}
\quranayah[10][48]
\end{Arabic}}
\flushleft{\begin{malayalam}
അവര്‍ (സത്യനിഷേധികള്‍) പറയും: എപ്പോഴാണ് ഈ വാഗ്ദാനം (നിറവേറുന്നത്‌?) (പറയൂ,) നിങ്ങള്‍ സത്യവാന്‍മാരാണെങ്കില്‍.
\end{malayalam}}
\flushright{\begin{Arabic}
\quranayah[10][49]
\end{Arabic}}
\flushleft{\begin{malayalam}
(നബിയേ,) പറയുക: എനിക്ക് തന്നെ ഉപകാരമോ ഉപദ്രവമോ ഉണ്ടാക്കുക എന്നത് എന്‍റെ അധീനത്തിലല്ല- അല്ലാഹു ഉദ്ദേശിച്ചതല്ലാതെ. ഓരോ സമൂഹത്തിനും ഒരു അവധിയുണ്ട്‌. അവരുടെ അവധി വന്നെത്തിയാല്‍ ഒരു നാഴിക നേരം പോലും അവര്‍ക്ക് വൈകിക്കാനാവില്ല. അവര്‍ക്കത് നേരത്തെയാക്കാനും കഴിയില്ല.
\end{malayalam}}
\flushright{\begin{Arabic}
\quranayah[10][50]
\end{Arabic}}
\flushleft{\begin{malayalam}
(നബിയേ,) പറയുക: അല്ലാഹുവിന്‍റെ ശിക്ഷ രാത്രിയോ പകലോ നിങ്ങള്‍ക്ക് വന്നാല്‍ (നിങ്ങളുടെ അവസ്ഥ എങ്ങനെയായിരിക്കുമെന്ന്‌) നിങ്ങള്‍ ചിന്തിച്ചിട്ടുണ്ടോ? അതില്‍ നിന്ന് ഏതു ശിക്ഷയ്ക്കായിരിക്കും കുറ്റവാളികള്‍ ധൃതി കാണിക്കുന്നത്‌?
\end{malayalam}}
\flushright{\begin{Arabic}
\quranayah[10][51]
\end{Arabic}}
\flushleft{\begin{malayalam}
എന്നിട്ട് അത് (ശിക്ഷ) അനുഭവിക്കുമ്പോഴാണോ നിങ്ങളതില്‍ വിശ്വസിക്കുന്നത്‌? (അപ്പോള്‍ നിങ്ങളോട് പറയപ്പെടും:) നിങ്ങള്‍ ഈ ശിക്ഷയ്ക്ക് തിടുക്കം കാണിക്കുന്നവരായിരുന്നല്ലോ. എന്നിട്ട് ഇപ്പോഴാണോ (നിങ്ങളുടെ വിശ്വാസം?)
\end{malayalam}}
\flushright{\begin{Arabic}
\quranayah[10][52]
\end{Arabic}}
\flushleft{\begin{malayalam}
പിന്നീട് അക്രമകാരികളോട് പറയപ്പെടും: നിങ്ങള്‍ ശാശ്വത ശിക്ഷ ആസ്വദിച്ച് കൊള്ളുക. നിങ്ങള്‍ പ്രവര്‍ത്തിച്ചിരുന്നതിനനുസരിച്ചല്ലാതെ നിങ്ങള്‍ക്ക് പ്രതിഫലം നല്‍കപ്പെടുമോ?
\end{malayalam}}
\flushright{\begin{Arabic}
\quranayah[10][53]
\end{Arabic}}
\flushleft{\begin{malayalam}
ഇത് സത്യമാണോ എന്ന് നിന്നോട് അവര്‍ അന്വേഷിക്കുന്നു. പറയുക: അതെ; എന്‍റെ രക്ഷിതാവിനെതന്നെയാണ! തീര്‍ച്ചയായും അത് സത്യം തന്നെയാണ്‌. നിങ്ങള്‍ക്ക് തോല്‍പിച്ചു കളയാനാവില്ല.
\end{malayalam}}
\flushright{\begin{Arabic}
\quranayah[10][54]
\end{Arabic}}
\flushleft{\begin{malayalam}
അക്രമം പ്രവര്‍ത്തിച്ച ഓരോ വ്യക്തിക്കും ഭൂമിയിലുള്ളത് മുഴുവന്‍ കൈവശമുണ്ടായിരുന്നാല്‍ പോലും അതയാള്‍ പ്രായശ്ചിത്തമായി നല്‍കുമായിരുന്നു. ശിക്ഷ കാണുമ്പോള്‍ അവര്‍ ഖേദം മനസ്സില്‍ ഒളിപ്പിക്കുകയും ചെയ്യും. അവര്‍ക്കിടയില്‍ നീതിയനുസരിച്ച് തീര്‍പ്പുകല്‍പിക്കപ്പെടുകയും ചെയ്യും. അവരോട് അനീതി കാണിക്കപ്പെടുകയില്ല.
\end{malayalam}}
\flushright{\begin{Arabic}
\quranayah[10][55]
\end{Arabic}}
\flushleft{\begin{malayalam}
ശ്രദ്ധിക്കുക; തീര്‍ച്ചയായും ആകാശങ്ങളിലും ഭൂമിയിലും ഉള്ളതൊക്കെ അല്ലാഹുവിന്‍റെതാകുന്നു. ശ്രദ്ധിക്കുക; തീര്‍ച്ചയായും അല്ലാഹുവിന്‍റെ വാഗ്ദാനം സത്യമാകുന്നു. പക്ഷെ അവരില്‍ അധികപേരും (കാര്യം) മനസ്സിലാക്കുന്നില്ല.
\end{malayalam}}
\flushright{\begin{Arabic}
\quranayah[10][56]
\end{Arabic}}
\flushleft{\begin{malayalam}
അവന്‍ ജീവിപ്പിക്കുകയും മരിപ്പിക്കുകയും ചെയ്യുന്നു. അവങ്കലേക്ക് തന്നെ നിങ്ങള്‍ മടക്കപ്പെടുകയും ചെയ്യുന്നു.
\end{malayalam}}
\flushright{\begin{Arabic}
\quranayah[10][57]
\end{Arabic}}
\flushleft{\begin{malayalam}
മനുഷ്യരേ, നിങ്ങളുടെ രക്ഷിതാവിങ്കല്‍ നിന്നുള്ള സദുപദേശവും, മനസ്സുകളിലുള്ള രോഗത്തിന് ശമനവും നിങ്ങള്‍ക്കു വന്നുകിട്ടിയിരിക്കുന്നു. സത്യവിശ്വാസികള്‍ക്ക് മാര്‍ഗദര്‍ശനവും കാരുണ്യവും (വന്നുകിട്ടിയിരിക്കുന്നു.)
\end{malayalam}}
\flushright{\begin{Arabic}
\quranayah[10][58]
\end{Arabic}}
\flushleft{\begin{malayalam}
പറയുക: അല്ലാഹുവിന്‍റെ അനുഗ്രഹം കൊണ്ടും കാരുണ്യം കൊണ്ടുമാണത്‌. അതുകൊണ്ട് അവര്‍ സന്തോഷിച്ചു കൊള്ളട്ടെ. അതാണ് അവര്‍ സമ്പാദിച്ചു കൂട്ടികൊണ്ടിരിക്കുന്നതിനെക്കാള്‍ ഉത്തമമായിട്ടുള്ളത്‌.
\end{malayalam}}
\flushright{\begin{Arabic}
\quranayah[10][59]
\end{Arabic}}
\flushleft{\begin{malayalam}
പറയുക: അല്ലാഹു നിങ്ങള്‍ക്കിറക്കിത്തന്ന ആഹാരത്തെപ്പറ്റി നിങ്ങള്‍ ചിന്തിച്ചിട്ടുണ്ടോ? എന്നിട്ട് അതില്‍ (ചിലത്‌) നിങ്ങള്‍ നിഷിദ്ധവും (വേറെ ചിലത്‌) അനുവദനീയവുമാക്കിയിരിക്കുന്നു. പറയുക: അല്ലാഹുവാണോ നിങ്ങള്‍ക്ക് (അതിന്‌) അനുവാദം തന്നത്‌? അതല്ല, നിങ്ങള്‍ അല്ലാഹുവിന്‍റെ പേരില്‍ കെട്ടിച്ചമയ്ക്കുകയാണോ?
\end{malayalam}}
\flushright{\begin{Arabic}
\quranayah[10][60]
\end{Arabic}}
\flushleft{\begin{malayalam}
അല്ലാഹുവിന്‍റെ പേരില്‍ കള്ളം കെട്ടിച്ചമയ്ക്കുന്നവരുടെ വിചാരം ഉയിര്‍ത്തെഴുന്നേല്‍പിന്‍റെ നാളില്‍ എന്തായിരിക്കും? തീര്‍ച്ചയായും അല്ലാഹു ജനങ്ങളോട് ഔദാര്യം കാണിക്കുന്നവനാകുന്നു. പക്ഷെ, അവരില്‍ അധികപേരും നന്ദികാണിക്കുന്നില്ല.
\end{malayalam}}
\flushright{\begin{Arabic}
\quranayah[10][61]
\end{Arabic}}
\flushleft{\begin{malayalam}
(നബിയേ,) നീ വല്ലകാര്യത്തിലും ഏര്‍പെടുകയോ, അതിനെപ്പറ്റി ഖുര്‍ആനില്‍ നിന്ന് വല്ലതും ഓതികേള്‍പിക്കുകയോ, നിങ്ങള്‍ ഏതെങ്കിലും പ്രവര്‍ത്തനത്തില്‍ ഏര്‍പെടുകയോ ചെയ്യുന്നുവെങ്കില്‍ നിങ്ങളതില്‍ മുഴുകുന്ന സമയത്ത് നിങ്ങളുടെ മേല്‍ സാക്ഷിയായി നാം ഉണ്ടാകാതിരിക്കുകയില്ല. ഭൂമിയിലോ ആകാശത്തോ ഉള്ള ഒരു അണുവോളമുള്ള യാതൊന്നും നിന്‍റെ രക്ഷിതാവി (ന്‍റെ ശ്രദ്ധയി) ല്‍ നിന്ന് വിട്ടുപോകുകയില്ല. അതിനെക്കാള്‍ ചെറുതോ വലുതോ ആയിട്ടുള്ള യാതൊന്നും സ്പഷ്ടമായ ഒരു രേഖയില്‍ ഉള്‍പെടാത്തതായി ഇല്ല.
\end{malayalam}}
\flushright{\begin{Arabic}
\quranayah[10][62]
\end{Arabic}}
\flushleft{\begin{malayalam}
ശ്രദ്ധിക്കുക: തീര്‍ച്ചയായും അല്ലാഹുവിന്‍റെ മിത്രങ്ങളാരോ അവര്‍ക്ക് യാതൊരു ഭയവുമില്ല. അവര്‍ ദുഃഖിക്കേണ്ടി വരികയുമില്ല.
\end{malayalam}}
\flushright{\begin{Arabic}
\quranayah[10][63]
\end{Arabic}}
\flushleft{\begin{malayalam}
വിശ്വസിക്കുകയും സൂക്ഷ്മത പാലിച്ചു കൊണ്ടിരിക്കുകയും ചെയ്യുന്നവരത്രെ അവര്‍
\end{malayalam}}
\flushright{\begin{Arabic}
\quranayah[10][64]
\end{Arabic}}
\flushleft{\begin{malayalam}
അവര്‍ക്കാണ് ഐഹികജീവിതത്തിലും പരലോകത്തും സന്തോഷവാര്‍ത്തയുള്ളത്‌. അല്ലാഹുവിന്‍റെ വചനങ്ങള്‍ക്ക് യാതൊരു മാറ്റവുമില്ല. അതു (സന്തോഷവാര്‍ത്ത) തന്നെയാണ് മഹത്തായ ഭാഗ്യം.
\end{malayalam}}
\flushright{\begin{Arabic}
\quranayah[10][65]
\end{Arabic}}
\flushleft{\begin{malayalam}
(നബിയേ,) അവരുടെ വാക്ക് നിനക്ക് വ്യസനമുണ്ടാക്കാതിരിക്കട്ടെ. തീര്‍ച്ചയായും പ്രതാപം മുഴുവന്‍ അല്ലാഹുവിനാകുന്നു. അവന്‍ എല്ലാം കേള്‍ക്കുന്നവനും അറിയുന്നവനുമത്രെ.
\end{malayalam}}
\flushright{\begin{Arabic}
\quranayah[10][66]
\end{Arabic}}
\flushleft{\begin{malayalam}
ശ്രദ്ധിക്കുക: തീര്‍ച്ചയായും അല്ലാഹുവിനുള്ളതാകുന്നു ആകാശങ്ങളിലുള്ളവരും ഭൂമിയിലുള്ളവരുമെല്ലാം. അല്ലാഹുവിന് പുറമെ പങ്കാളികളെ വിളിച്ചു പ്രാര്‍ത്ഥിക്കുന്നവര്‍ എന്തൊന്നിനെയാണ് പിന്‍പറ്റുന്നത്‌? അവര്‍ ഊഹത്തെ മാത്രമാണ് പിന്തുടരുന്നത്‌. അവര്‍ അനുമാനിച്ച് (കള്ളം) പറയുക മാത്രമാണ് ചെയ്യുന്നത്‌.
\end{malayalam}}
\flushright{\begin{Arabic}
\quranayah[10][67]
\end{Arabic}}
\flushleft{\begin{malayalam}
അവനത്രെ നിങ്ങള്‍ക്ക് വേണ്ടി രാത്രിയെ ശാന്തമായി കഴിയത്തക്കവിധവും പകലിനെ വെളിച്ചമുള്ളതും ആക്കിത്തന്നത്‌. തീര്‍ച്ചയായും കേട്ട് മനസ്സിലാക്കുന്ന ആളുകള്‍ക്ക് അതില്‍ ദൃഷ്ടാന്തങ്ങളുണ്ട്‌.
\end{malayalam}}
\flushright{\begin{Arabic}
\quranayah[10][68]
\end{Arabic}}
\flushleft{\begin{malayalam}
അല്ലാഹു ഒരു സന്താനത്തെ സ്വീകരിച്ചിരിക്കുന്നു എന്നവര്‍ പറഞ്ഞു. അവന്‍ എത്ര പരിശുദ്ധന്‍! അവന്‍ പരാശ്രയമുക്തനത്രെ. ആകാശങ്ങളിലുള്ളതും ഭൂമിയിലുള്ളതുമെല്ലാം അവന്‍റെതാകുന്നു. നിങ്ങളുടെ പക്കല്‍ ഇതിന് (ദൈവത്തിന് സന്താനം ഉണ്ടെന്നതിന്‌) യാതൊരു പ്രമാണവുമില്ല. അല്ലാഹുവിന്‍റെ പേരില്‍ നിങ്ങള്‍ക്ക് അറിവില്ലാത്തത് നിങ്ങള്‍ പറഞ്ഞുണ്ടാക്കുകയാണോ?
\end{malayalam}}
\flushright{\begin{Arabic}
\quranayah[10][69]
\end{Arabic}}
\flushleft{\begin{malayalam}
പറയുക: അല്ലാഹുവിന്‍റെ പേരില്‍ കള്ളം കെട്ടിച്ചമയ്ക്കുന്നവര്‍ വിജയിക്കുകയില്ല; തീര്‍ച്ച.
\end{malayalam}}
\flushright{\begin{Arabic}
\quranayah[10][70]
\end{Arabic}}
\flushleft{\begin{malayalam}
(അവര്‍ക്കുള്ളത്‌) ഇഹലോകത്തെ സുഖാനുഭവമത്രെ. പിന്നെ നമ്മുടെ അടുക്കലേക്കാണ് അവരുടെ മടക്കം. എന്നിട്ട് അവര്‍ അവിശ്വസിച്ചിരുന്നതിന്‍റെ ഫലമായി കഠിനമായ ശിക്ഷ നാം അവര്‍ക്ക് ആസ്വദിപ്പിക്കുന്നതാണ്‌.
\end{malayalam}}
\flushright{\begin{Arabic}
\quranayah[10][71]
\end{Arabic}}
\flushleft{\begin{malayalam}
(നബിയേ,) നീ അവര്‍ക്ക് നൂഹിനെപ്പറ്റിയുള്ള വിവരം ഓതികേള്‍പിക്കുക. അദ്ദേഹം തന്‍റെ ജനതയോട് പറഞ്ഞ സന്ദര്‍ഭം: എന്‍റെ ജനങ്ങളേ, എന്‍റെ സാന്നിദ്ധ്യവും അല്ലാഹുവിന്‍റെ ദൃഷ്ടാന്തങ്ങളെപ്പറ്റിയുള്ള എന്‍റെ ഉല്‍ബോധനവും നിങ്ങള്‍ക്ക് ഒരു വലിയ ഭാരമായിത്തീര്‍ന്നിട്ടുണ്ടെങ്കില്‍ അല്ലാഹുവിന്‍റെ മേല്‍ ഞാനിതാ ഭരമേല്‍പിച്ചിരിക്കുന്നു. എന്നാല്‍ നിങ്ങളുടെ കാര്യം നിങ്ങളും നിങ്ങള്‍ പങ്കാളികളാക്കിയവരും കൂടി തീരുമാനിച്ചുറപ്പിച്ചു കൊള്ളൂ. പിന്നെ നിങ്ങളുടെ കാര്യത്തില്‍ (തീരുമാനത്തില്‍) നിങ്ങള്‍ക്ക് ഒരു അവ്യക്തതയും ഉണ്ടായിരിക്കരുത്‌. എന്നിട്ട് എന്‍റെ നേരെ നിങ്ങള്‍ (ആ തീരുമാനം) നടപ്പില്‍ വരുത്തൂ. എനിക്ക് നിങ്ങള്‍ ഇടതരികയേ വേണ്ട.
\end{malayalam}}
\flushright{\begin{Arabic}
\quranayah[10][72]
\end{Arabic}}
\flushleft{\begin{malayalam}
ഇനി നിങ്ങള്‍ പിന്തിരിഞ്ഞു കളയുന്ന പക്ഷം ഞാന്‍ നിങ്ങളോട് യാതൊരു പ്രതിഫലവും ചോദിച്ചിട്ടില്ല. എനിക്ക് പ്രതിഫലം തരേണ്ടത് അല്ലാഹു മാത്രമാകുന്നു. (അല്ലാഹുവിന്‌) കീഴ്പെടുന്നവരുടെ കൂട്ടത്തില്‍ ആയിരിക്കുവാനാണ് ഞാന്‍ കല്‍പിക്കപ്പെട്ടിട്ടുള്ളത് .
\end{malayalam}}
\flushright{\begin{Arabic}
\quranayah[10][73]
\end{Arabic}}
\flushleft{\begin{malayalam}
എന്നിട്ട് അവര്‍ അദ്ദേഹത്തെ നിഷേധിച്ചു തള്ളിക്കളഞ്ഞു. അപ്പോള്‍ അദ്ദേഹത്തെയും അദ്ദേഹത്തോടൊപ്പമുള്ളവരെയും നാം കപ്പലില്‍ രക്ഷപ്പെടുത്തുകയും, അവരെ നാം (ഭൂമിയില്‍) പിന്‍ഗാമികളാക്കുകയും ചെയ്തു. നമ്മുടെ ദൃഷ്ടാന്തങ്ങള്‍ നിഷേധിച്ചുതള്ളിയവരെ നാം മുക്കിക്കളഞ്ഞു. അപ്പോള്‍ നോക്കൂ; താക്കീത് നല്‍കപ്പെട്ട ആ വിഭാഗത്തിന്‍റെ പര്യവസാനം എങ്ങനെയായിരുന്നുവെന്ന്‌.
\end{malayalam}}
\flushright{\begin{Arabic}
\quranayah[10][74]
\end{Arabic}}
\flushleft{\begin{malayalam}
പിന്നെ അദ്ദേഹത്തിനു ശേഷം പല ദൂതന്‍മാരെയും അവരുടെ ജനതയിലേക്ക് നാം നിയോഗിച്ചു. അങ്ങനെ അവരുടെ അടുത്ത് തെളിവുകളും കൊണ്ട് അവര്‍ ചെന്നു. എന്നാല്‍ മുമ്പ് ഏതൊന്ന് അവര്‍ നിഷേധിച്ചു തള്ളിയോ അതില്‍ അവര്‍ വിശ്വസിക്കുവാന്‍ തയ്യാറുണ്ടായിരുന്നില്ല. അതിക്രമകാരികളുടെ ഹൃദയങ്ങളിന്‍മേല്‍ അപ്രകാരം നാം മുദ്രവെക്കുന്നു.
\end{malayalam}}
\flushright{\begin{Arabic}
\quranayah[10][75]
\end{Arabic}}
\flushleft{\begin{malayalam}
പിന്നീട് അവര്‍ക്ക് ശേഷം, നമ്മുടെ ദൃഷ്ടാന്തങ്ങളുമായി ഫിര്‍ഔന്‍റെയും അവന്‍റെ പ്രമാണിമാരുടെയും അടുത്തേക്ക് മൂസായെയും ഹാറൂനെയും നാം നിയോഗിച്ചു. എന്നാല്‍ അവര്‍ അഹങ്കരിക്കുകയാണ് ചെയ്തത്‌. അവര്‍ കുറ്റവാളികളായ ഒരു ജനവിഭാഗമായിരുന്നു.
\end{malayalam}}
\flushright{\begin{Arabic}
\quranayah[10][76]
\end{Arabic}}
\flushleft{\begin{malayalam}
അങ്ങനെ നമ്മുടെ പക്കല്‍ നിന്നുള്ള സത്യം അവര്‍ക്ക് വന്നെത്തിയപ്പോള്‍ അവര്‍ പറഞ്ഞു: തീര്‍ച്ചയായും ഇത് സ്പഷ്ടമായ ഒരു ജാലവിദ്യതന്നെയാകുന്നു.
\end{malayalam}}
\flushright{\begin{Arabic}
\quranayah[10][77]
\end{Arabic}}
\flushleft{\begin{malayalam}
മൂസാപറഞ്ഞു: സത്യം നിങ്ങള്‍ക്ക് വന്നെത്തിയപ്പോള്‍ അതിനെപ്പറ്റി (ജാലവിദ്യയെന്ന്‌) നിങ്ങള്‍ പറയുകയോ? ജാലവിദ്യയാണോ ഇത്‌?(യഥാര്‍ത്ഥത്തില്‍) ജാലവിദ്യക്കാര്‍ വിജയം പ്രാപിക്കുകയില്ല.
\end{malayalam}}
\flushright{\begin{Arabic}
\quranayah[10][78]
\end{Arabic}}
\flushleft{\begin{malayalam}
അവര്‍ പറഞ്ഞു: ഞങ്ങളുടെ പിതാക്കന്‍മാര്‍ എന്തൊന്നില്‍ നിലകൊള്ളുന്നവരായി ഞങ്ങള്‍ കണ്ടുവോ അതില്‍ നിന്ന് ഞങ്ങളെ തിരിച്ചുകളയാന്‍ വേണ്ടിയും, ഭൂമിയില്‍ മേധാവിത്വം നിങ്ങള്‍ക്ക് രണ്ടു പേര്‍ക്കുമാകാന്‍ വേണ്ടിയുമാണോ നീ ഞങ്ങളുടെ അടുത്ത് വന്നിരിക്കുന്നത്‌? നിങ്ങള്‍ ഇരുവരെയും ഞങ്ങള്‍ വിശ്വസിക്കുന്നതേ അല്ല.
\end{malayalam}}
\flushright{\begin{Arabic}
\quranayah[10][79]
\end{Arabic}}
\flushleft{\begin{malayalam}
ഫിര്‍ഔന്‍ പറഞ്ഞു: എല്ലാ വിവരമുള്ള ജാലവിദ്യക്കാരെയും നിങ്ങള്‍ എന്‍റെ അടുക്കല്‍ കൊണ്ട് വരൂ
\end{malayalam}}
\flushright{\begin{Arabic}
\quranayah[10][80]
\end{Arabic}}
\flushleft{\begin{malayalam}
അങ്ങനെ ജാലവിദ്യക്കാര്‍ വന്നപ്പോള്‍ മൂസാ അവരോട് പറഞ്ഞു: നിങ്ങള്‍ക്ക് ഇടാനുള്ളതല്ലാം ഇട്ടേക്കൂ.
\end{malayalam}}
\flushright{\begin{Arabic}
\quranayah[10][81]
\end{Arabic}}
\flushleft{\begin{malayalam}
അങ്ങനെ അവര്‍ ഇട്ടപ്പോള്‍ മൂസാ പറഞ്ഞു: നിങ്ങള്‍ ഈ അവതരിപ്പിച്ചത് ജാലവിദ്യയാകുന്നു. തീര്‍ച്ചയായും അല്ലാഹു അതിനെ പൊളിച്ചു കളയുന്നതാണ്‌. കുഴപ്പമുണ്ടാക്കുന്നവരുടെ പ്രവര്‍ത്തനം അല്ലാഹു ഫലവത്താക്കിത്തീര്‍ക്കുകയില്ല; തീര്‍ച്ച.
\end{malayalam}}
\flushright{\begin{Arabic}
\quranayah[10][82]
\end{Arabic}}
\flushleft{\begin{malayalam}
സത്യത്തെ അവന്‍റെ വചനങ്ങളിലൂടെ അവന്‍ യാഥാര്‍ത്ഥ്യമാക്കിത്തീര്‍ക്കുന്നതാണ്‌. കുറ്റവാളികള്‍ക്ക് അത് അനിഷ്ടകരമായാലും ശരി.
\end{malayalam}}
\flushright{\begin{Arabic}
\quranayah[10][83]
\end{Arabic}}
\flushleft{\begin{malayalam}
എന്നാല്‍ മൂസായെ തന്‍റെ ജനതയില്‍ നിന്നുള്ള ചില ചെറുപ്പക്കാരല്ലാതെ മറ്റാരും വിശ്വസിച്ചില്ല. (അത് തന്നെ) ഫിര്‍ഔനും അവരിലുള്ള പ്രധാനികളും അവരെ മര്‍ദ്ദിച്ചേക്കുമോ എന്ന ഭയപ്പാടോടുകൂടിയായിരുന്നു. തീര്‍ച്ചയായും ഫിര്‍ഔന്‍ ഭൂമിയില്‍ ഔന്നത്യം നടിക്കുന്നവന്‍ തന്നെയാകുന്നു. തീര്‍ച്ചയായും അവന്‍ അതിരുകവിഞ്ഞവരുടെ കൂട്ടത്തില്‍ത്തന്നെയാകുന്നു.
\end{malayalam}}
\flushright{\begin{Arabic}
\quranayah[10][84]
\end{Arabic}}
\flushleft{\begin{malayalam}
മൂസാ പറഞ്ഞു: എന്‍റെ ജനങ്ങളേ,നിങ്ങള്‍ അല്ലാഹുവില്‍ വിശ്വസിച്ചിട്ടുണ്ടെങ്കില്‍ അവന്‍റെ മേല്‍ നിങ്ങള്‍ ഭരമേല്‍പിക്കുക- നിങ്ങള്‍ അവന്ന് കീഴ്പെട്ടവരാണെങ്കില്‍.
\end{malayalam}}
\flushright{\begin{Arabic}
\quranayah[10][85]
\end{Arabic}}
\flushleft{\begin{malayalam}
അപ്പോള്‍ അവര്‍ പറഞ്ഞു: അല്ലാഹുവിന്‍റെ മേല്‍ ഞങ്ങള്‍ ഭരമേല്‍പിച്ചിരിക്കുന്നു. ഞങ്ങളുടെ രക്ഷിതാവേ, ഞങ്ങളെ നീ അക്രമികളായ ഈ ജനവിഭാഗത്തിന്‍റെ മര്‍ദ്ദനത്തിന് ഇരയാക്കരുതേ.
\end{malayalam}}
\flushright{\begin{Arabic}
\quranayah[10][86]
\end{Arabic}}
\flushleft{\begin{malayalam}
നിന്‍റെ കാരുണ്യം കൊണ്ട് സത്യനിഷേധികളായ ഈ ജനതയില്‍ നിന്ന് ഞങ്ങളെ നീ രക്ഷപ്പെടുത്തേണമേ.
\end{malayalam}}
\flushright{\begin{Arabic}
\quranayah[10][87]
\end{Arabic}}
\flushleft{\begin{malayalam}
മൂസായ്ക്കും അദ്ദേഹത്തിന്‍റെ സഹോദരന്നും നാം ഇപ്രകാരം സന്ദേശം നല്‍കി: നിങ്ങള്‍ രണ്ടുപേരും നിങ്ങളുടെ ആളുകള്‍ക്ക് വേണ്ടി ഈജിപ്തില്‍ (പ്രത്യേകം) വീടുകള്‍ സൌകര്യപ്പെടുത്തുകയും, നിങ്ങളുടെ വീടുകള്‍ ഖിബ്ലയാക്കുകയും, നിങ്ങള്‍ നമസ്കാരം മുറപോലെ നിര്‍വഹിക്കുകയും ചെയ്യുക. സത്യവിശ്വാസികള്‍ക്ക് നീ സന്തോഷവാര്‍ത്ത അറിയിക്കുക.
\end{malayalam}}
\flushright{\begin{Arabic}
\quranayah[10][88]
\end{Arabic}}
\flushleft{\begin{malayalam}
മൂസാ പറഞ്ഞു: ഞങ്ങളുടെ രക്ഷിതാവേ, ഫിര്‍ഔന്നും അവന്‍റെ പ്രമാണിമാര്‍ക്കും നീ ഐഹികജീവിതത്തില്‍ അലങ്കാരവും സമ്പത്തുകളും നല്‍കിയിരിക്കുന്നു. ഞങ്ങളുടെ രക്ഷിതാവേ, നിന്‍റെ മാര്‍ഗത്തില്‍ നിന്ന് ആളുകളെ തെറ്റിക്കുവാന്‍ വേണ്ടിയാണ് (അവരത് ഉപയോഗിക്കുന്നത്‌.) ഞങ്ങളുടെ രക്ഷിതാവേ, നീ അവരുടെ സ്വത്തുക്കള്‍ തുടച്ചുനീക്കേണമേ. വേദനയേറിയ ശിക്ഷ കാണുന്നതുവരെയും അവര്‍ വിശ്വസിക്കാതിരിക്കത്തക്കവണ്ണം അവരുടെ ഹൃദയങ്ങള്‍ക്ക് നീ കാഠിന്യം നല്‍കുകയും ചെയ്യേണമേ.
\end{malayalam}}
\flushright{\begin{Arabic}
\quranayah[10][89]
\end{Arabic}}
\flushleft{\begin{malayalam}
അവന്‍ (അല്ലാഹു) പറഞ്ഞു: നിങ്ങളുടെ ഇരുവരുടെയും പ്രാര്‍ത്ഥന ഇതാ സ്വീകരിക്കപ്പെട്ടിരിക്കുന്നു. അതിനാല്‍ നിങ്ങള്‍ ഇരുവരും നേരെ നിലകൊള്ളുക. വിവരമില്ലാത്തവരുടെ വഴി നിങ്ങള്‍ ഇരുവരും പിന്തുടര്‍ന്ന് പോകരുത്‌.
\end{malayalam}}
\flushright{\begin{Arabic}
\quranayah[10][90]
\end{Arabic}}
\flushleft{\begin{malayalam}
ഇസ്രായീല്‍ സന്തതികളെ നാം കടല്‍ കടത്തികൊണ്ടു പോയി. അപ്പോള്‍ ഫിര്‍ഔനും അവന്‍റെ സൈന്യങ്ങളും ധിക്കാരവും അതിക്രമവുമായി അവരെ പിന്തുടര്‍ന്നു. ഒടുവില്‍ മുങ്ങിമരിക്കാറായപ്പോള്‍ അവന്‍ പറഞ്ഞു: ഇസ്രായീല്‍ സന്തതികള്‍ ഏതൊരു ദൈവത്തില്‍ വിശ്വസിച്ചിരിക്കുന്നുവോ അവനല്ലാതെ യാതൊരു ദൈവവുമില്ല എന്ന് ഞാന്‍ വിശ്വസിച്ചിരിക്കുന്നു. ഞാന്‍ (അവന്ന്‌) കീഴ്പെട്ടവരുടെ കൂട്ടത്തിലാകുന്നു.
\end{malayalam}}
\flushright{\begin{Arabic}
\quranayah[10][91]
\end{Arabic}}
\flushleft{\begin{malayalam}
(അല്ലാഹു അവനോട് പറഞ്ഞു:) മുമ്പൊക്കെ ധിക്കരിക്കുകയും കുഴപ്പക്കാരുടെ കൂട്ടത്തിലായിരിക്കുകയും ചെയ്തിട്ട് ഇപ്പോഴാണോ (നീ വിശ്വസിക്കുന്നത് ?)
\end{malayalam}}
\flushright{\begin{Arabic}
\quranayah[10][92]
\end{Arabic}}
\flushleft{\begin{malayalam}
എന്നാല്‍ നിന്‍റെ പുറകെ വരുന്നവര്‍ക്ക് നീ ഒരു ദൃഷ്ടാന്തമായിരിക്കേണ്ടതിനുവേണ്ടി ഇന്നു നിന്‍റെ ശരീരത്തെ നാം രക്ഷപ്പെടുത്തി എടുക്കുന്നതാണ്‌. തീര്‍ച്ചയായും മനുഷ്യരില്‍ ധാരാളം പേര്‍ നമ്മുടെ ദൃഷ്ടാന്തങ്ങളെപ്പറ്റി അശ്രദ്ധരാകുന്നു.
\end{malayalam}}
\flushright{\begin{Arabic}
\quranayah[10][93]
\end{Arabic}}
\flushleft{\begin{malayalam}
തീര്‍ച്ചയായും ഇസ്രായീല്‍ സന്തതികളെ ഉചിതമായ ഒരു താവളത്തില്‍ നാം കുടിയിരുത്തുകയും, വിശിഷ്ടമായ വസ്തുക്കളില്‍ നിന്ന് അവര്‍ക്ക് നാം ആഹാരം നല്‍കുകയും ചെയ്തു. എന്നാല്‍ അവര്‍ക്ക് അറിവ് വന്നുകിട്ടിയതിന് ശേഷം തന്നെയാണ് അവര്‍ ഭിന്നിച്ചത്‌. അവര്‍ ഭിന്നിച്ചു കൊണ്ടിരുന്ന കാര്യത്തില്‍ ഉയിര്‍ത്തെഴുന്നേല്‍പിന്‍റെ നാളില്‍ നിന്‍റെ രക്ഷിതാവ് അവര്‍ക്കിടയില്‍ വിധികല്‍പിക്കുക തന്നെ ചെയ്യും.
\end{malayalam}}
\flushright{\begin{Arabic}
\quranayah[10][94]
\end{Arabic}}
\flushleft{\begin{malayalam}
ഇനി നിനക്കു നാം അവതരിപ്പിച്ചു തന്നതിനെപ്പറ്റി നിനക്ക് വല്ല സംശയവുമുണ്ടെങ്കില്‍ നിനക്ക് മുമ്പുതന്നെ വേദഗ്രന്ഥം വായിച്ച് വരുന്നവരോട് ചോദിച്ചു നോക്കുക. തീര്‍ച്ചയായും നിനക്ക് നിന്‍റെ രക്ഷിതാവിങ്കല്‍ നിന്നുള്ള സത്യം വന്നുകിട്ടിയിരിക്കുന്നു. അതിനാല്‍ നീ സംശയാലുക്കളുടെ കൂട്ടത്തിലായിപ്പോകരുത്‌.
\end{malayalam}}
\flushright{\begin{Arabic}
\quranayah[10][95]
\end{Arabic}}
\flushleft{\begin{malayalam}
അല്ലാഹുവിന്‍റെ ദൃഷ്ടാന്തങ്ങളെ നിഷേധിച്ചു തള്ളിയവരുടെ കൂട്ടത്തിലും നീ ആയിപ്പോകരുത്‌. എങ്കില്‍ നീ നഷ്ടം പറ്റിയവരുടെ കൂട്ടത്തിലായിരിക്കും.
\end{malayalam}}
\flushright{\begin{Arabic}
\quranayah[10][96]
\end{Arabic}}
\flushleft{\begin{malayalam}
തീര്‍ച്ചയായും ഏതൊരു വിഭാഗത്തിന്‍റെ മേല്‍ നിന്‍റെ രക്ഷിതാവിന്‍റെ വചനം സ്ഥിരപ്പെട്ടിരിക്കുന്നുവോ അവര്‍ വിശ്വസിക്കുകയില്ല.
\end{malayalam}}
\flushright{\begin{Arabic}
\quranayah[10][97]
\end{Arabic}}
\flushleft{\begin{malayalam}
ഏതൊരു തെളിവ് അവര്‍ക്ക് വന്നുകിട്ടിയാലും. വേദനയേറിയ ശിക്ഷ നേരില്‍ കാണുന്നതുവരെ.
\end{malayalam}}
\flushright{\begin{Arabic}
\quranayah[10][98]
\end{Arabic}}
\flushleft{\begin{malayalam}
ഏതെങ്കിലുമൊരു രാജ്യം വിശ്വാസം സ്വീകരിക്കുകയും, വിശ്വാസം അതിന് പ്രയോജനപ്പെടുകയും ചെയ്യാത്തതെന്ത്‌? യൂനുസിന്‍റെ ജനത ഒഴികെ. അവര്‍ വിശ്വസിച്ചപ്പോള്‍ ഇഹലോകജീവിതത്തിലെ അപമാനകരമായ ശിക്ഷ അവരില്‍ നിന്ന് നാം നീക്കം ചെയ്യുകയും, ഒരു നിശ്ചിത കാലം വരെ നാം അവര്‍ക്ക് സൌഖ്യം നല്‍കുകയും ചെയ്തു.
\end{malayalam}}
\flushright{\begin{Arabic}
\quranayah[10][99]
\end{Arabic}}
\flushleft{\begin{malayalam}
നിന്‍റെ രക്ഷിതാവ് ഉദ്ദേശിച്ചിരുന്നുവെങ്കില്‍ ഭൂമിയിലുള്ളവരെല്ലാം ഒന്നിച്ച് വിശ്വസിക്കുമായിരുന്നു. എന്നിരിക്കെ ജനങ്ങള്‍ സത്യവിശ്വാസികളാകുവാന്‍ നീ അവരെ നിര്‍ബന്ധിക്കുകയോ?
\end{malayalam}}
\flushright{\begin{Arabic}
\quranayah[10][100]
\end{Arabic}}
\flushleft{\begin{malayalam}
യാതൊരാള്‍ക്കും അല്ലാഹുവിന്‍റെ അനുമതിപ്രകാരമല്ലാതെ വിശ്വസിക്കാന്‍ കഴിയുന്നതല്ല. ചിന്തിച്ചു മനസ്സിലാക്കാത്തവര്‍ക്ക് അല്ലാഹു നികൃഷ്ടത വരുത്തിവെക്കുന്നതാണ്‌.
\end{malayalam}}
\flushright{\begin{Arabic}
\quranayah[10][101]
\end{Arabic}}
\flushleft{\begin{malayalam}
(നബിയേ,) പറയുക: ആകാശങ്ങളിലും ഭൂമിയിലും എന്തൊക്കെയാണുള്ളതെന്ന് നിങ്ങള്‍ നോക്കുവിന്‍. വിശ്വസിക്കാത്ത ജനങ്ങള്‍ക്ക് ദൃഷ്ടാന്തങ്ങളും താക്കീതുകളും എന്തുഫലം ചെയ്യാനാണ്‌?
\end{malayalam}}
\flushright{\begin{Arabic}
\quranayah[10][102]
\end{Arabic}}
\flushleft{\begin{malayalam}
അപ്പോള്‍ അവരുടെ മുമ്പ് കഴിഞ്ഞുപോയവരുടെ അനുഭവങ്ങള്‍ പോലുള്ളതല്ലാതെ മറ്റുവല്ലതും അവര്‍ കാത്തിരിക്കുകയാണോ? പറയുക: എന്നാല്‍ നിങ്ങള്‍ കാത്തിരിക്കുക. തീര്‍ച്ചയായും ഞാനും നിങ്ങളോടൊപ്പം കാത്തിരിക്കുന്നവരുടെ കൂട്ടത്തിലാണ്‌.
\end{malayalam}}
\flushright{\begin{Arabic}
\quranayah[10][103]
\end{Arabic}}
\flushleft{\begin{malayalam}
പിന്നീട് നമ്മുടെ ദൂതന്‍മാരെയും വിശ്വസിച്ചവരെയും നാം രക്ഷപ്പെടുത്തുന്നു. അപ്രകാരം നമ്മുടെ മേലുള്ള ഒരു ബാധ്യത എന്ന നിലയില്‍ നാം വിശ്വാസികളെ രക്ഷപ്പെടുത്തുന്നു.
\end{malayalam}}
\flushright{\begin{Arabic}
\quranayah[10][104]
\end{Arabic}}
\flushleft{\begin{malayalam}
പറയുക: ജനങ്ങളേ, എന്‍റെ മതത്തെ സംബന്ധിച്ച് നിങ്ങള്‍ സംശയത്തിലാണെങ്കില്‍ (നിങ്ങള്‍ മനസ്സിലാക്കുക;) അല്ലാഹുവിന് പുറമെ നിങ്ങള്‍ ആരാധിക്കുന്നവരെ ഞാന്‍ ആരാധിക്കുകയില്ല. പക്ഷെ നിങ്ങളെ മരിപ്പിക്കുന്നവനായ അല്ലാഹുവെ ഞാന്‍ ആരാധിക്കുന്നു. സത്യവിശ്വാസികളുടെ കൂട്ടത്തിലായിരിക്കുവാനാണ് ഞാന്‍ കല്‍പിക്കപ്പെട്ടിരിക്കുന്നത്‌.
\end{malayalam}}
\flushright{\begin{Arabic}
\quranayah[10][105]
\end{Arabic}}
\flushleft{\begin{malayalam}
വക്രതയില്ലാത്തവനായിക്കൊണ്ട് നിന്‍റെ മുഖം മതത്തിന് നേരെയാക്കി നിര്‍ത്തണമെന്നും നീ ബഹുദൈവവിശ്വാസികളില്‍ പെട്ടവനായിരിക്കരുതെന്നും (ഞാന്‍ കല്‍പിക്കപ്പെട്ടിരിക്കുന്നു.)
\end{malayalam}}
\flushright{\begin{Arabic}
\quranayah[10][106]
\end{Arabic}}
\flushleft{\begin{malayalam}
അല്ലാഹുവിന് പുറമെ നിനക്ക് ഉപകാരം ചെയ്യാത്തതും, നിനക്ക് ഉപദ്രവം ചെയ്യാത്തതുമായ യാതൊന്നിനോടും നീ പ്രാര്‍ത്ഥിക്കരുത്‌. നീ അപ്രകാരം ചെയ്യുന്ന പക്ഷം തീര്‍ച്ചയായും നീ അക്രമികളുടെ കൂട്ടത്തിലായിരിക്കും.
\end{malayalam}}
\flushright{\begin{Arabic}
\quranayah[10][107]
\end{Arabic}}
\flushleft{\begin{malayalam}
നിനക്ക് അല്ലാഹു വല്ല ദോഷവും ഏല്‍പിക്കുന്ന പക്ഷം അവനൊഴികെ അത് നീക്കം ചെയ്യാന്‍ ഒരാളുമില്ല. അവന്‍ നിനക്ക് വല്ല ഗുണവും ഉദ്ദേശിക്കുന്ന പക്ഷം അവന്‍റെ അനുഗ്രഹം തട്ടിമാറ്റാന്‍ ഒരാളുമില്ല. തന്‍റെ ദാസന്‍മാരില്‍ നിന്ന് താന്‍ ഇച്ഛിക്കുന്നവര്‍ക്ക് അത് (അനുഗ്രഹം) അവന്‍ അനുഭവിപ്പിക്കുന്നു. അവന്‍ ഏറെ പൊറുക്കുന്നവനും കരുണാനിധിയുമത്രെ.
\end{malayalam}}
\flushright{\begin{Arabic}
\quranayah[10][108]
\end{Arabic}}
\flushleft{\begin{malayalam}
പറയുക: ഹേ; ജനങ്ങളേ, നിങ്ങളുടെ രക്ഷിതാവിങ്കല്‍ നിന്നുള്ള സത്യം നിങ്ങള്‍ക്ക് വന്നെത്തിയിരിക്കുന്നു. അതിനാല്‍ ആര്‍ നേര്‍വഴി സ്വീകരിക്കുന്നുവോ അവന്‍ തന്‍റെ ഗുണത്തിന് തന്നെയാണ് നേര്‍വഴി സ്വീകരിക്കുന്നത്‌. വല്ലവനും വഴിപിഴച്ച് പോയാല്‍ അതിന്‍റെ ദോഷവും അവന് തന്നെയാണ്‌. ഞാന്‍ നിങ്ങളുടെ മേല്‍ ഉത്തരവാദിത്തം ഏല്‍പിക്കപ്പെട്ടവനല്ല.
\end{malayalam}}
\flushright{\begin{Arabic}
\quranayah[10][109]
\end{Arabic}}
\flushleft{\begin{malayalam}
നിനക്ക് സന്ദേശം നല്‍കപ്പെടുന്നതിനെ നീ പിന്തുടരുക. അല്ലാഹു തീര്‍പ്പുകല്‍പിക്കുന്നത് വരെ ക്ഷമിക്കുകയും ചെയ്യുക. തീര്‍പ്പുകല്‍പിക്കുന്നവരില്‍ ഉത്തമനത്രെ അവന്‍.
\end{malayalam}}
\chapter{\textmalayalam{ഹൂദ്}}
\begin{Arabic}
\Huge{\centerline{\basmalah}}\end{Arabic}
\flushright{\begin{Arabic}
\quranayah[11][1]
\end{Arabic}}
\flushleft{\begin{malayalam}
അലിഫ്‌-ലാം-റാ. ഒരു പ്രമാണഗ്രന്ഥമത്രെ ഇത്‌. അതിലെ വചനങ്ങള്‍ ആശയഭദ്രതയുള്ളതാക്കപ്പെട്ടിരിക്കുന്നു. പിന്നീടത് വിശദീകരിക്കപ്പെട്ടിരിക്കുന്നു. യുക്തിമാനും സൂക്ഷ്മജ്ഞാനിയുമായ അല്ലാഹുവിന്‍റെ അടുക്കല്‍ നിന്നുള്ളതത്രെ അത്‌.
\end{malayalam}}
\flushright{\begin{Arabic}
\quranayah[11][2]
\end{Arabic}}
\flushleft{\begin{malayalam}
എന്തെന്നാല്‍ അല്ലാഹുവിനെയല്ലാതെ നിങ്ങള്‍ ആരാധിക്കരുത്‌. തീര്‍ച്ചയായും അവങ്കല്‍ നിന്ന് നിങ്ങളിലേക്ക് നിയോഗിക്കപ്പെട്ട താക്കീതുകാരനും സന്തോഷവാര്‍ത്തക്കാരനുമത്രെ ഞാന്‍.
\end{malayalam}}
\flushright{\begin{Arabic}
\quranayah[11][3]
\end{Arabic}}
\flushleft{\begin{malayalam}
നിങ്ങള്‍ നിങ്ങളുടെ രക്ഷിതാവിനോട് പാപമോചനം തേടുകയും എന്നിട്ട് അവനിലേക്ക് ഖേദിച്ചുമടങ്ങുകയും ചെയ്യുക. എങ്കില്‍ നിര്‍ണിതമായ ഒരു അവധിവരെ അവന്‍ നിങ്ങള്‍ക്ക് നല്ല സൌഖ്യമനുഭവിപ്പിക്കുകയും, ഉദാരമനസ്ഥിതിയുള്ള എല്ലാവര്‍ക്കും തങ്ങളുടെ ഉദാരതയ്ക്കുള്ള പ്രതിഫലം നല്‍കുകയും ചെയ്യുന്നതാണ്‌. നിങ്ങള്‍ തിരിഞ്ഞുകളയുന്ന പക്ഷം ഭയങ്കരമായ ഒരു ദിവസത്തിലെ ശിക്ഷ നിങ്ങളുടെ മേല്‍ ഞാന്‍ നിശ്ചയമായും ഭയപ്പെടുന്നു.
\end{malayalam}}
\flushright{\begin{Arabic}
\quranayah[11][4]
\end{Arabic}}
\flushleft{\begin{malayalam}
അല്ലാഹുവിങ്കലേക്കണ് നിങ്ങളുടെ മടക്കം. അവന്‍ എല്ലാകാര്യത്തിനും കഴിവുള്ളവനത്രെ.
\end{malayalam}}
\flushright{\begin{Arabic}
\quranayah[11][5]
\end{Arabic}}
\flushleft{\begin{malayalam}
ശ്രദ്ധിക്കുക: അവനില്‍ നിന്ന് (അല്ലാഹുവില്‍ നിന്ന്‌) ഒളിക്കാന്‍ വേണ്ടി അവര്‍ തങ്ങളുടെ നെഞ്ചുകള്‍ മടക്കിക്കളയുന്നു. ശ്രദ്ധിക്കുക: അവര്‍ തങ്ങളുടെ വസ്ത്രങ്ങള്‍കൊണ്ട് പുതച്ച് മൂടുമ്പോള്‍ പോലും അവര്‍ രഹസ്യമാക്കുന്നതും പരസ്യമാക്കുന്നതും അവന്‍ അറിയുന്നു. തീര്‍ച്ചയായും അവന്‍ നെഞ്ചുകളിലുള്ളത് അറിയുന്നവനാകുന്നു.
\end{malayalam}}
\flushright{\begin{Arabic}
\quranayah[11][6]
\end{Arabic}}
\flushleft{\begin{malayalam}
ഭൂമിയില്‍ യാതൊരു ജന്തുവും അതിന്‍റെ ഉപജീവനം അല്ലാഹു ബാധ്യത ഏറ്റതായിട്ടല്ലാതെ ഇല്ല. അവയുടെ താമസസ്ഥലവും സൂക്ഷിപ്പുസ്ഥലവും അവന്‍ അറിയുന്നു. എല്ലാം സ്പഷ്ടമായ ഒരു രേഖയിലുണ്ട്‌.
\end{malayalam}}
\flushright{\begin{Arabic}
\quranayah[11][7]
\end{Arabic}}
\flushleft{\begin{malayalam}
ആറുദിവസങ്ങളിലായി (അഥവാ ഘട്ടങ്ങളിലായി) ആകാശങ്ങളും ഭൂമിയും സൃഷ്ടിച്ചത് അവനത്രെ. അവന്‍റെ അര്‍ശ് (സിംഹാസനം) വെള്ളത്തിന്‍മേലായിരുന്നു. നിങ്ങളില്‍ ആരാണ് കര്‍മ്മം കൊണ്ട് ഏറ്റവും നല്ലവന്‍ എന്നറിയുന്നതിന് നിങ്ങളെ പരീക്ഷിക്കുവാന്‍ വേണ്ടിയത്രെ അത്‌. തീര്‍ച്ചയായും നിങ്ങള്‍ മരണത്തിന് ശേഷം ഉയിര്‍ത്തെഴുന്നേല്‍പിക്കപ്പെടുന്നവരാണ്‌. എന്ന് നീ പറഞ്ഞാല്‍ അവിശ്വസിച്ചവര്‍ പറയും; ഇത് സ്പഷ്ടമായ ജാലവിദ്യയല്ലാതെ മറ്റൊന്നുമല്ല.
\end{malayalam}}
\flushright{\begin{Arabic}
\quranayah[11][8]
\end{Arabic}}
\flushleft{\begin{malayalam}
ഒരു നിര്‍ണിത കാലപരിധി വരെ അവരില്‍ നിന്നും നാം ശിക്ഷ മേറ്റീവ്ച്ചാല്‍ അവര്‍ പറയുക തന്നെ ചെയ്യും; അതിനെ തടഞ്ഞു നിര്‍ത്തുന്ന കാര്യമെന്താണ് എന്ന്‌. ശ്രദ്ധിക്കുക. അതവര്‍ക്ക് വന്നെത്തുന്ന ദിവസം അതവരില്‍ നിന്ന് തിരിച്ചുകളയപ്പെടുന്നതല്ല. എന്തൊന്നിനെപ്പറ്റി അവര്‍ പരിഹസിച്ചിരുന്നുവോ അതവരില്‍ വന്നെത്തുകയും ചെയ്യും.
\end{malayalam}}
\flushright{\begin{Arabic}
\quranayah[11][9]
\end{Arabic}}
\flushleft{\begin{malayalam}
മനുഷ്യന്ന് നാം നമ്മുടെ പക്കല്‍ നിന്നുള്ള വല്ല കാരുണ്യവും ആസ്വദിപ്പിക്കുകയും, എന്നിട്ട് നാം അതവനില്‍ നിന്ന് എടുത്തുനീക്കുകയും ചെയ്താല്‍ തീര്‍ച്ചയായും അവന്‍ നിരാശനും ഏറ്റവും നന്ദികെട്ടവനുമായിരിക്കും.
\end{malayalam}}
\flushright{\begin{Arabic}
\quranayah[11][10]
\end{Arabic}}
\flushleft{\begin{malayalam}
അവന്ന് ഒരു കഷ്ടത ബാധിച്ചതിന് ശേഷം നാമവന്ന് ഒരു അനുഗ്രഹം ആസ്വദിപ്പിച്ചുവെങ്കിലോ നിശ്ചയമായും അവന്‍ പറയും; തിന്‍മകള്‍ എന്നില്‍ നിന്ന് ഒഴിഞ്ഞ് പോയിരിക്കുന്നു എന്ന്‌. തീര്‍ച്ചയായും അവന്‍ ആഹ്ലാദഭരിതനും അഹങ്കാരിയുമാകുന്നു.
\end{malayalam}}
\flushright{\begin{Arabic}
\quranayah[11][11]
\end{Arabic}}
\flushleft{\begin{malayalam}
ക്ഷമിക്കുകയും സല്‍കര്‍മ്മങ്ങള്‍ പ്രവര്‍ത്തിക്കുകയും ചെയ്തവരൊഴികെ. അവര്‍ക്കാകുന്നു പാപമോചനവും വലിയ പ്രതിഫലവുമുള്ളത്‌.
\end{malayalam}}
\flushright{\begin{Arabic}
\quranayah[11][12]
\end{Arabic}}
\flushleft{\begin{malayalam}
ഇയാള്‍ക്ക് ഒരു നിധി ഇറക്കപ്പെടുകയോ, ഇയാളോടൊപ്പം ഒരു മലക്ക് വരികയോ ചെയ്യാത്തതെന്ത് എന്ന് (നിന്നെപറ്റി) അവര്‍ പറയുന്ന കാരണത്താല്‍ നിനക്ക് നല്‍കപ്പെടുന്ന സന്ദേശങ്ങളില്‍ ചിലത് നീ വിട്ടുകളയുകയും, അതിന്‍റെ പേരില്‍ നിനക്ക് മനഃപ്രയാസമുണ്ടാകുകയും ചെയ്തേക്കാം. എന്നാല്‍ നീ ഒരു താക്കീതുകാരന്‍ മാത്രമാകുന്നു. അല്ലാഹു എല്ലാകാര്യത്തിന്‍റെയും സംരക്ഷണമേറ്റവനാകുന്നു.
\end{malayalam}}
\flushright{\begin{Arabic}
\quranayah[11][13]
\end{Arabic}}
\flushleft{\begin{malayalam}
അതല്ല, അദ്ദേഹം അത് കെട്ടിച്ചമച്ചു എന്നാണോ അവര്‍ പറയുന്നത്‌? പറയുക: എന്നാല്‍ ഇതുപേലെയുള്ള പത്ത് അദ്ധ്യായങ്ങള്‍ ചമച്ചുണ്ടാക്കിയത് നിങ്ങള്‍ കൊണ്ട് വരൂ. അല്ലാഹുവിന് പുറമെ നിങ്ങള്‍ക്ക് സാധിക്കുന്നവരെയെല്ലാം നിങ്ങള്‍ വിളിച്ചുകൊള്ളുകയും ചെയ്യുക. നിങ്ങള്‍ സത്യവാന്‍മാരാണെങ്കില്‍.
\end{malayalam}}
\flushright{\begin{Arabic}
\quranayah[11][14]
\end{Arabic}}
\flushleft{\begin{malayalam}
അവരാരും നിങ്ങളുടെ വിളിക്ക് ഉത്തരം നല്‍കിയില്ലെങ്കില്‍, അല്ലാഹുവിന്‍റെ അറിവോട് കൂടി മാത്രമാണ് ഇത് അവതരിപ്പിക്കപ്പെട്ടിട്ടുള്ളതെന്നും, അവനല്ലാതെ യാതൊരു ദൈവവുമില്ലെന്നും നിങ്ങള്‍ മനസ്സിലാക്കുക. ഇനിയെങ്കിലും നിങ്ങള്‍ കീഴ്പെടാന്‍ സന്നദ്ധരാണോ?
\end{malayalam}}
\flushright{\begin{Arabic}
\quranayah[11][15]
\end{Arabic}}
\flushleft{\begin{malayalam}
ഐഹികജീവിതത്തെയും അതിന്‍റെ അലങ്കാരത്തെയുമാണ് ആരെങ്കിലും ഉദ്ദേശിക്കുന്നതെങ്കില്‍ അവരുടെ പ്രവര്‍ത്തനങ്ങള്‍ അവിടെ (ഇഹലോകത്ത്‌) വെച്ച് അവര്‍ക്ക് നാം നിറവേറ്റികൊടുക്കുന്നതാണ്‌. അവര്‍ക്കവിടെ യാതൊരു കുറവും വരുത്തപ്പെടുകയില്ല.
\end{malayalam}}
\flushright{\begin{Arabic}
\quranayah[11][16]
\end{Arabic}}
\flushleft{\begin{malayalam}
പരലോകത്ത് നരകമല്ലാതെ മറ്റൊന്നും കിട്ടാനില്ലാത്തവരാകുന്നു അക്കൂട്ടര്‍. അവര്‍ ഇവിടെ പ്രവര്‍ത്തിച്ചതെല്ലാം പൊളിഞ്ഞുപോയിരിക്കുന്നു. അവര്‍ ചെയ്തുകൊണ്ടിരിക്കുന്നതെല്ലാം ഫലശൂന്യമത്രെ.
\end{malayalam}}
\flushright{\begin{Arabic}
\quranayah[11][17]
\end{Arabic}}
\flushleft{\begin{malayalam}
എന്നാല്‍ ഒരാള്‍ തന്‍റെ രക്ഷിതാവിങ്കല്‍ നിന്ന് ലഭിച്ച വ്യക്തമായ തെളിവിനെ അവലംബിക്കുന്നു. അവങ്കല്‍ നിന്നുള്ള ഒരു സാക്ഷി (ഖുര്‍ആന്‍) അതിനെ തുടര്‍ന്ന് വരുകയും ചെയ്യുന്നു. അതിന് മുമ്പ് മാതൃകയും കാരുണ്യവുമായിക്കൊണ്ട് മൂസായുടെ ഗ്രന്ഥം വന്നിട്ടുമുണ്ട്‌. (അങ്ങനെയുള്ള ഒരാള്‍ ആ ദുന്‍യാ പ്രേമികളെ പോലെ ഖുര്‍ആന്‍ നിഷേധിക്കുമോ? ഇല്ല.) അത്തരക്കാര്‍ അതില്‍ വിശ്വസിക്കും. വിവിധ സംഘങ്ങളില്‍ നിന്ന് അതില്‍ അവിശ്വസിക്കുന്നവരാരോ അവരുടെ വാഗ്ദത്തസ്ഥാനം നരകമാകുന്നു. ആകയാല്‍ നീ അതിനെപ്പറ്റി സംശയത്തിലാവരുത്‌. തീര്‍ച്ചയായും അത് നിന്‍റെ രക്ഷിതാവിങ്കല്‍ നിന്നുള്ള സത്യമാകുന്നു. പക്ഷെ ജനങ്ങളില്‍ അധികപേരും വിശ്വസിക്കുന്നില്ല.
\end{malayalam}}
\flushright{\begin{Arabic}
\quranayah[11][18]
\end{Arabic}}
\flushleft{\begin{malayalam}
അല്ലാഹുവിന്‍റെ പേരില്‍ കള്ളം കെട്ടിച്ചമച്ചവനേക്കാള്‍ അക്രമിയായി ആരുണ്ട്‌? അവര്‍ അവരുടെ രക്ഷിതാവിന്‍റെ മുമ്പില്‍ ഹാജരാക്കപ്പെടുന്നതാണ്‌. സാക്ഷികള്‍ പറയും: ഇവരാകുന്നു തങ്ങളുടെ രക്ഷിതാവിന്‍റെ പേരില്‍ കള്ളം പറഞ്ഞവര്‍, ശ്രദ്ധിക്കുക: അല്ലാഹുവിന്‍റെ ശാപം ആ അക്രമികളുടെ മേലുണ്ടായിരിക്കും.
\end{malayalam}}
\flushright{\begin{Arabic}
\quranayah[11][19]
\end{Arabic}}
\flushleft{\begin{malayalam}
അല്ലാഹുവിന്‍റെ മാര്‍ഗത്തില്‍ നിന്ന് തടയുകയും, അതിന് വക്രത വരുത്താന്‍ ആഗ്രഹിക്കുകയും ചെയ്യുന്നവരത്രെ അവര്‍. അവരാകട്ടെ പരലോകത്തില്‍ വിശ്വാസമില്ലാത്തവരുമാണ്‌.
\end{malayalam}}
\flushright{\begin{Arabic}
\quranayah[11][20]
\end{Arabic}}
\flushleft{\begin{malayalam}
അക്കൂട്ടര്‍ ഭൂമിയില്‍ (അല്ലാഹുവെ) തോല്‍പിക്കാന്‍ കഴിയുന്നവരായിട്ടില്ല. അല്ലാഹുവിന് പുറമെ അവര്‍ക്ക് രക്ഷാധികാരികളാരും ഉണ്ടായിട്ടുമില്ല. അവര്‍ക്ക് ശിക്ഷ ഇരട്ടിപ്പിക്കപ്പെടുന്നതാണ്‌. അവര്‍ കേള്‍ക്കാന്‍ കഴിയുന്നവരായില്ല. അവര്‍ കണ്ടറിയുന്നവരുമായില്ല.
\end{malayalam}}
\flushright{\begin{Arabic}
\quranayah[11][21]
\end{Arabic}}
\flushleft{\begin{malayalam}
അത്തരക്കാരാകുന്നു ആത്മനഷ്ടം പറ്റിയവര്‍. അവര്‍ കെട്ടിച്ചമച്ചിരുന്നതെല്ലാം അവരെ വിട്ടുമാറിക്കളയുകയും ചെയ്തു.
\end{malayalam}}
\flushright{\begin{Arabic}
\quranayah[11][22]
\end{Arabic}}
\flushleft{\begin{malayalam}
നിസ്സംശയം, അവര്‍ തന്നെയാണ് പരലോകത്തില്‍ ഏറ്റവും നഷ്ടം പറ്റിയവര്‍.
\end{malayalam}}
\flushright{\begin{Arabic}
\quranayah[11][23]
\end{Arabic}}
\flushleft{\begin{malayalam}
തീര്‍ച്ചയായും വിശ്വസിക്കുകയും സല്‍കര്‍മ്മങ്ങള്‍ പ്രവര്‍ത്തിക്കുകയും, തങ്ങളുടെ രക്ഷിതാവിങ്കലേക്ക് വിനയപൂര്‍വ്വം മടങ്ങുകയും ചെയ്തവരാരോ അവരാകുന്നു സ്വര്‍ഗാവകാശികള്‍. അവരതില്‍ നിത്യവാസികളായിരിക്കും.
\end{malayalam}}
\flushright{\begin{Arabic}
\quranayah[11][24]
\end{Arabic}}
\flushleft{\begin{malayalam}
ഈ രണ്ട് വിഭാഗങ്ങളുടെയും ഉപമ അന്ധനും ബധിരനുമായ ഒരാളെപ്പോലെയും, കാഴ്ചയും കേള്‍വിയുമുള്ള മറ്റൊരാളെപ്പോലെയുമാകുന്നു. ഇവര്‍ ഇരുവരും ഉപമയില്‍ തുല്യരാകുമോ? അപ്പോള്‍ നിങ്ങള്‍ ചിന്തിച്ചുനോക്കുന്നില്ലേ?
\end{malayalam}}
\flushright{\begin{Arabic}
\quranayah[11][25]
\end{Arabic}}
\flushleft{\begin{malayalam}
നൂഹിനെ അദ്ദേഹത്തിന്‍റെ ജനതയിലേക്ക് നാം നിയോഗിക്കുകയുണ്ടായി. (അദ്ദേഹം പറഞ്ഞു:) തീര്‍ച്ചയായും ഞാന്‍ നിങ്ങള്‍ക്ക് സ്പഷ്ടമായ താക്കീത് നല്‍കുന്നവനാകുന്നു.
\end{malayalam}}
\flushright{\begin{Arabic}
\quranayah[11][26]
\end{Arabic}}
\flushleft{\begin{malayalam}
എന്തെന്നാല്‍ അല്ലാഹുവെയല്ലാതെ നിങ്ങള്‍ ആരാധിക്കരുത്‌. വേദനയേറിയ ഒരു ദിവസത്തെ ശിക്ഷ നിങ്ങളുടെ മേല്‍ തീര്‍ച്ചയായും ഞാന്‍ ഭയപ്പെടുന്നു.
\end{malayalam}}
\flushright{\begin{Arabic}
\quranayah[11][27]
\end{Arabic}}
\flushleft{\begin{malayalam}
അപ്പോള്‍ അദ്ദേഹത്തിന്‍റെ ജനതയില്‍ നിന്ന് അവിശ്വസിച്ചവരായ പ്രമാണിമാര്‍ പറഞ്ഞു: ഞങ്ങളെപോലെയുള്ള മനുഷ്യനായിട്ട് മാത്രമേ നിന്നെ ഞങ്ങള്‍ കാണുന്നുള്ളൂ. ഞങ്ങളുടെ കൂട്ടത്തില്‍ ഏറ്റവും നിസ്സാരന്‍മാരായിട്ടുള്ളവര്‍ പ്രഥമവീക്ഷണത്തില്‍ (ശരിയായി ചിന്തിക്കാതെ) നിന്നെ പിന്തുടര്‍ന്നതായിട്ട് മാത്രമാണ് ഞങ്ങള്‍ കാണുന്നത്‌. നിങ്ങള്‍ക്ക് ഞങ്ങളെക്കാള്‍ യാതൊരു ശ്രേഷ്ഠതയും ഞങ്ങള്‍ കാണുന്നുമില്ല. പ്രത്യുത, നിങ്ങള്‍ വ്യാജവാദികളാണെന്ന് ഞങ്ങള്‍ കരുതുന്നു.
\end{malayalam}}
\flushright{\begin{Arabic}
\quranayah[11][28]
\end{Arabic}}
\flushleft{\begin{malayalam}
അദ്ദേഹം പറഞ്ഞു: എന്‍റെ ജനങ്ങളേ, നിങ്ങള്‍ ചിന്തിച്ച് നോക്കിയിട്ടുണ്ടോ? ഞാന്‍ എന്‍റെ രക്ഷിതാവിങ്കല്‍ നിന്നുള്ള വ്യക്തമായ തെളിവിനെ അവലംബിക്കുന്നവനായിരിക്കുകയും അവന്‍റെ അടുക്കല്‍ നിന്നുള്ള കാരുണ്യം അവന്‍ എനിക്ക് തന്നിരിക്കുകയും, എന്നിട്ട് നിങ്ങള്‍ക്ക് (അത് കണ്ടറിയാനാവാത്ത വിധം) അന്ധത വരുത്തപ്പെടുകയുമാണ് ഉണ്ടായിട്ടുള്ളതെങ്കില്‍ (ഞാന്‍ എന്ത് ചെയ്യും?) നിങ്ങള്‍ അത് ഇഷ്ടപ്പെടാത്തവരായിരിക്കെ നിങ്ങളുടെ മേല്‍ നാം അതിന് നിര്‍ബന്ധം ചെലുത്തുകയോ ?
\end{malayalam}}
\flushright{\begin{Arabic}
\quranayah[11][29]
\end{Arabic}}
\flushleft{\begin{malayalam}
എന്‍റെ ജനങ്ങളേ, ഇതിന്‍റെ പേരില്‍ നിങ്ങളോട് ഞാന്‍ ധനം ചോദിക്കുന്നില്ല. എനിക്കുള്ള പ്രതിഫലം അല്ലാഹു തരേണ്ടത് മാത്രമാകുന്നു. വിശ്വസിച്ചവരെ ഞാന്‍ ആട്ടിയോടിക്കുന്നതല്ല. തീര്‍ച്ചയായും അവര്‍ അവരുടെ രക്ഷിതാവിനെ കണ്ടുമുട്ടാന്‍ പോകുന്നവരാണ്‌. എന്നാല്‍ ഞാന്‍ നിങ്ങളെ കാണുന്നത് വിവരമില്ലാത്ത ഒരു ജനവിഭാഗമായിട്ടാണ്‌.
\end{malayalam}}
\flushright{\begin{Arabic}
\quranayah[11][30]
\end{Arabic}}
\flushleft{\begin{malayalam}
എന്‍റെ ജനങ്ങളേ, ഞാനവരെ ആട്ടിയോടിക്കുന്ന പക്ഷം അല്ലാഹുവിന്‍റെ ശിക്ഷയില്‍ നിന്ന് എന്നെ രക്ഷിക്കുവാനാരാണുള്ളത്‌. നിങ്ങള്‍ ആലോചിച്ച് നോക്കുന്നില്ലേ?
\end{malayalam}}
\flushright{\begin{Arabic}
\quranayah[11][31]
\end{Arabic}}
\flushleft{\begin{malayalam}
അല്ലാഹുവിന്‍റെ ഖജനാവുകള്‍ എന്‍റെ പക്കലുണ്ടെന്ന് ഞാന്‍ നിങ്ങളോട് പറയുന്നുമില്ല. ഞാന്‍ അദൃശ്യകാര്യം അറിയുകയുമില്ല. നിങ്ങളുടെ കണ്ണുകള്‍ നിസ്സാരമായി കാണുന്നവരെപറ്റി, അവര്‍ക്ക് അല്ലാഹു യാതൊരു ഗുണവും നല്‍കുന്നതേയല്ല എന്നും ഞാന്‍ പറയുകയില്ല. അല്ലാഹുവാണ് അവരുടെ മനസ്സുകളിലുള്ളതിനെപ്പറ്റി നല്ലവണ്ണം അറിയുന്നവന്‍. അപ്പോള്‍ (അവരെ ദുഷിച്ച് പറയുന്ന പക്ഷം) തീര്‍ച്ചയായും ഞാന്‍ അക്രമികളില്‍ പെട്ടവനായിരിക്കും.
\end{malayalam}}
\flushright{\begin{Arabic}
\quranayah[11][32]
\end{Arabic}}
\flushleft{\begin{malayalam}
അവര്‍ പറഞ്ഞു: നൂഹേ, നീ ഞങ്ങളോട് തര്‍ക്കിച്ചു. വളരെയേറെ തര്‍ക്കിച്ചു. എന്നാല്‍ നീ സത്യവാന്‍മാരുടെ കൂട്ടത്തിലാണെങ്കില്‍ നീ ഞങ്ങള്‍ക്ക് താക്കീത് നല്‍കിക്കൊണ്ടിരിക്കുന്നത് (ശിക്ഷ) ഞങ്ങള്‍ക്ക് നീ ഇങ്ങു കൊണ്ട് വരൂ.
\end{malayalam}}
\flushright{\begin{Arabic}
\quranayah[11][33]
\end{Arabic}}
\flushleft{\begin{malayalam}
അദ്ദേഹം പറഞ്ഞു: അല്ലാഹു മാത്രമാണ് നിങ്ങള്‍ക്കത് കൊണ്ട് വരുക; അവന്‍ ഉദ്ദേശിച്ചെങ്കില്‍, നിങ്ങള്‍ക്ക് (അവനെ) തോല്‍പിച്ച് കളയാനാവില്ല.
\end{malayalam}}
\flushright{\begin{Arabic}
\quranayah[11][34]
\end{Arabic}}
\flushleft{\begin{malayalam}
അല്ലാഹു നിങ്ങളെ വഴിതെറ്റിച്ചുവിടാന്‍ ഉദ്ദേശിക്കുന്ന പക്ഷം ഞാന്‍ നിങ്ങള്‍ക്ക് ഉപദേശം നല്‍കാന്‍ ഉദ്ദേശിച്ചാലും എന്‍റെ ഉപദേശം നിങ്ങള്‍ക്ക് പ്രയോജനപ്പെടുകയില്ല. അവനാണ് നിങ്ങളുടെ രക്ഷിതാവ്‌. അവങ്കലേക്കാണ് നിങ്ങള്‍ മടക്കപ്പെടുന്നത്‌.
\end{malayalam}}
\flushright{\begin{Arabic}
\quranayah[11][35]
\end{Arabic}}
\flushleft{\begin{malayalam}
അതല്ല, അദ്ദേഹമത് കെട്ടിച്ചമച്ചു എന്നാണോ അവര്‍ പറയുന്നത്‌? പറയുക: ഞാനത് കെട്ടിച്ചമച്ചുവെങ്കില്‍ ഞാന്‍ കുറ്റം ചെയ്യുന്നതിന്‍റെ ദോഷം എനിക്കു തന്നെയായിരിക്കും. നിങ്ങള്‍ ചെയ്യുന്ന കുറ്റത്തിന്‍റെ കാര്യത്തില്‍ ഞാന്‍ നിരപരാധിയുമാണ്‌.
\end{malayalam}}
\flushright{\begin{Arabic}
\quranayah[11][36]
\end{Arabic}}
\flushleft{\begin{malayalam}
നിന്‍റെ ജനതയില്‍ നിന്ന് വിശ്വസിച്ചുകഴിഞ്ഞിട്ടുള്ളവരല്ലാതെ ഇനിയാരും വിശ്വസിക്കുകയേയില്ല. അതിനാല്‍ അവര്‍ ചെയ്തുകൊണ്ടിരിക്കുന്നതിനെപ്പറ്റി നീ സങ്കടപ്പെടരുത് എന്ന് നൂഹിന് സന്ദേശം നല്‍കപ്പെട്ടു.
\end{malayalam}}
\flushright{\begin{Arabic}
\quranayah[11][37]
\end{Arabic}}
\flushleft{\begin{malayalam}
നമ്മുടെ മേല്‍നോട്ടത്തിലും, നമ്മുടെ നിര്‍ദേശപ്രകാരവും നീ കപ്പല്‍ നിര്‍മിക്കുക. അക്രമം ചെയ്തവരുടെ കാര്യത്തില്‍ നീ എന്നോട് സംസാരിക്കരുത്‌. തീര്‍ച്ചയായും അവര്‍ മുക്കി നശിപ്പിക്കപ്പെടാന്‍ പോകുകയാണ്‌.
\end{malayalam}}
\flushright{\begin{Arabic}
\quranayah[11][38]
\end{Arabic}}
\flushleft{\begin{malayalam}
അദ്ദേഹം കപ്പല്‍ നിര്‍മിച്ചുകൊണ്ടിരിക്കുന്നു. അദ്ദേഹത്തിന്‍റെ ജനതയിലെ ഓരോ പ്രമാണിക്കൂട്ടം അദ്ദേഹത്തിന്‍റെ അടുത്ത് കൂടി കടന്ന് പോയപ്പോഴല്ലാം അദ്ദേഹത്തെ പരിഹസിച്ചു. അദ്ദേഹം പറഞ്ഞു: നിങ്ങള്‍ ഞങ്ങളെ പരിഹസിക്കുന്ന പക്ഷം തീര്‍ച്ചയായും നിങ്ങള്‍ പരിഹസിക്കുന്നത് പോലെത്തന്നെ ഞങ്ങള്‍ നിങ്ങളെയും പരിഹസിക്കുന്നതാണ്‌.
\end{malayalam}}
\flushright{\begin{Arabic}
\quranayah[11][39]
\end{Arabic}}
\flushleft{\begin{malayalam}
അപമാനകരമായ ശിക്ഷ ആര്‍ക്കാണ് വന്നെത്തുന്നതെന്നും, സ്ഥിരമായ ശിക്ഷ ആരുടെ മേലാണ് വന്നുഭവിക്കുന്നതെന്നും നിങ്ങള്‍ക്ക് വഴിയെ അറിയാം.
\end{malayalam}}
\flushright{\begin{Arabic}
\quranayah[11][40]
\end{Arabic}}
\flushleft{\begin{malayalam}
അങ്ങനെ നമ്മുടെ കല്‍പന വരികയും അടുപ്പ് ഉറവപൊട്ടി ഒഴുകുകയും ചെയ്തപ്പോള്‍ നാം പറഞ്ഞു: എല്ലാ വര്‍ഗത്തില്‍ നിന്നും രണ്ട് ഇണകളെ വീതവും, നിന്‍റെ കുടുംബാംഗങ്ങളെയും അതില്‍ കയറ്റികൊള്ളുക. (അവരുടെ കൂട്ടത്തില്‍ നിന്ന്‌) ആര്‍ക്കെതിരില്‍ (ശിക്ഷയുടെ) വചനം മുന്‍കൂട്ടി ഉണ്ടായിട്ടുണ്ടോ അവരൊഴികെ. വിശ്വസിച്ചവരെയും (കയറ്റികൊള്ളുക.) അദ്ദേഹത്തോടൊപ്പം കുറച്ച് പേരല്ലാതെ വിശ്വസിച്ചിട്ടുണ്ടായിരുന്നില്ല.
\end{malayalam}}
\flushright{\begin{Arabic}
\quranayah[11][41]
\end{Arabic}}
\flushleft{\begin{malayalam}
അദ്ദേഹം (അവരോട്‌) പറഞ്ഞു: നിങ്ങളതില്‍ കയറിക്കൊള്ളുക. അതിന്‍റെ ഓട്ടവും നിര്‍ത്തവും അല്ലാഹുവിന്‍റെ പേരിലാകുന്നു. തീര്‍ച്ചയായും എന്‍റെ രക്ഷിതാവ് ഏറെ പൊറുക്കുന്നവനും കരുണാനിധിയുമാണ്‌.
\end{malayalam}}
\flushright{\begin{Arabic}
\quranayah[11][42]
\end{Arabic}}
\flushleft{\begin{malayalam}
പര്‍വ്വതതുല്യമായ തിരമാലകള്‍ക്കിടയിലൂടെ അത് (കപ്പല്‍) അവരെയും കൊണ്ട് സഞ്ചരിച്ചുകൊണ്ടിരിക്കുകയാണ്‌. നൂഹ് തന്‍റെ മകനെ വിളിച്ചു. അവന്‍ അകലെ ഒരു സ്ഥലത്തായിരുന്നു. എന്‍റെ കുഞ്ഞുമകനേ, നീ ഞങ്ങളോടൊപ്പം കയറിക്കൊള്ളുക. നീ സത്യനിഷേധികളുടെ കൂടെ ആയിപ്പോകരുത്‌
\end{malayalam}}
\flushright{\begin{Arabic}
\quranayah[11][43]
\end{Arabic}}
\flushleft{\begin{malayalam}
അവന്‍ പറഞ്ഞു: വെള്ളത്തില്‍ നിന്ന് എനിക്ക് രക്ഷനല്‍കുന്ന വല്ല മലയിലും ഞാന്‍ അഭയം പ്രാപിച്ചുകൊള്ളാം. അദ്ദേഹം പറഞ്ഞു: അല്ലാഹുവിന്‍റെ കല്‍പനയില്‍ നിന്ന് ഇന്ന് രക്ഷനല്‍കാന്‍ ആരുമില്ല; അവന്‍ കരുണ ചെയ്തവര്‍ക്കൊഴികെ. (അപ്പോഴേക്കും) അവര്‍ രണ്ട് പേര്‍ക്കുമിടയില്‍ തിരമാല മറയിട്ടു. അങ്ങനെ അവന്‍ മുക്കി നശിപ്പിക്കപ്പെട്ടവരുടെ കൂട്ടത്തിലായി.
\end{malayalam}}
\flushright{\begin{Arabic}
\quranayah[11][44]
\end{Arabic}}
\flushleft{\begin{malayalam}
ഭൂമീ! നിന്‍റെ വെള്ളം നീ വിഴുങ്ങൂ. ആകാശമേ! മഴ നിര്‍ത്തൂ! എന്ന് കല്‍പന നല്‍കപ്പെട്ടു. വെള്ളം വറ്റുകയും ഉത്തരവ് നിറവേറ്റപ്പെടുകയും ചെയ്തു. അത് (കപ്പല്‍) ജൂദി പര്‍വ്വതത്തിന് മേല്‍ ഉറച്ചുനില്‍ക്കുകയും ചെയ്തു. അക്രമികളായ ജനതയ്ക്ക് നാശം എന്ന് പറയപ്പെടുകയും ചെയ്തു.
\end{malayalam}}
\flushright{\begin{Arabic}
\quranayah[11][45]
\end{Arabic}}
\flushleft{\begin{malayalam}
നൂഹ് തന്‍റെ രക്ഷിതാവിനെ വിളിച്ചുകൊണ്ട് പറഞ്ഞു: എന്‍റെ രക്ഷിതാവേ, എന്‍റെ മകന്‍ എന്‍റെ കുടുംബാംഗങ്ങളില്‍ പെട്ടവന്‍ തന്നെയാണല്ലോ. തീര്‍ച്ചയായും നിന്‍റെ വാഗ്ദാനം സത്യമാണുതാനും. നീ വിധികര്‍ത്താക്കളില്‍ വെച്ച് ഏറ്റവും നല്ല വിധികര്‍ത്താവുമാണ്‌
\end{malayalam}}
\flushright{\begin{Arabic}
\quranayah[11][46]
\end{Arabic}}
\flushleft{\begin{malayalam}
അവന്‍ (അല്ലാഹു) പറഞ്ഞു: നൂഹേ, തീര്‍ച്ചയായും അവന്‍ നിന്‍റെ കുടുംബത്തില്‍ പെട്ടവനല്ല. തീര്‍ച്ചയായും അവന്‍ ശരിയല്ലാത്തത് ചെയ്തവനാണ്‌. അതിനാല്‍ നിനക്ക് അറിവില്ലാത്ത കാര്യം എന്നോട് ആവശ്യപ്പെടരുത്‌. നീ വിവരമില്ലാത്തവരുടെ കൂട്ടത്തിലായിപോകരുതെന്ന് ഞാന്‍ നിന്നോട് ഉപദേശിക്കുകയാണ്‌.
\end{malayalam}}
\flushright{\begin{Arabic}
\quranayah[11][47]
\end{Arabic}}
\flushleft{\begin{malayalam}
അദ്ദേഹം (നൂഹ്‌) പറഞ്ഞു: എന്‍റെ രക്ഷിതാവേ, എനിക്ക് അറിവില്ലാത്ത കാര്യം നിന്നോട് ആവശ്യപ്പെടുന്നതില്‍ നിന്ന് ഞാന്‍ നിന്നോട് ശരണം തേടുന്നു. നീ എനിക്ക് പൊറുത്തുതരികയും, നീ എന്നോട് കരുണ കാണിക്കുകയും ചെയ്യാത്ത പക്ഷം ഞാന്‍ നഷ്ടക്കാരുടെ കൂട്ടത്തിലായിരിക്കും.
\end{malayalam}}
\flushright{\begin{Arabic}
\quranayah[11][48]
\end{Arabic}}
\flushleft{\begin{malayalam}
(അദ്ദേഹത്തോട്‌) പറയപ്പെട്ടു: നൂഹേ, നമ്മുടെ പക്കല്‍ നിന്നുള്ള ശാന്തിയോടുകൂടിയും, നിനക്കും നിന്‍റെ കൂടെയുള്ളവരില്‍ നിന്നുള്ള സമൂഹങ്ങള്‍ക്കും അനുഗ്രഹങ്ങളോടുകൂടിയും നീ ഇറങ്ങിക്കൊള്ളുക. എന്നാല്‍ (വേറെ) ചില സമൂഹങ്ങളുണ്ട്‌. അവര്‍ക്ക് നാം സൌഖ്യം നല്‍കുന്നതാണ്‌. പിന്നീട് നമ്മുടെ പക്കല്‍ നിന്നുള്ള വേദനയേറിയ ശിക്ഷയും അവര്‍ക്ക് ബാധിക്കുന്നതാണ്‌.
\end{malayalam}}
\flushright{\begin{Arabic}
\quranayah[11][49]
\end{Arabic}}
\flushleft{\begin{malayalam}
(നബിയേ,) അവയൊക്കെ അദൃശ്യവാര്‍ത്തകളില്‍ പെട്ടതാകുന്നു. നിനക്ക് നാം അത് സന്ദേശമായി നല്‍കുന്നു. നീയോ, നിന്‍റെ ജനതയോ ഇതിനു മുമ്പ് അതറിയുമായിരുന്നില്ല. അതുകൊണ്ട് ക്ഷമിക്കുക. തീര്‍ച്ചയായും അനന്തരഫലം സൂക്ഷ്മത പാലിക്കുന്നവര്‍ക്ക് അനുകൂലമായിരിക്കും.
\end{malayalam}}
\flushright{\begin{Arabic}
\quranayah[11][50]
\end{Arabic}}
\flushleft{\begin{malayalam}
ആദ് ജനതയിലേക്ക് അവരുടെ സഹോദരനായ ഹൂദിനെയും (നാം നിയോഗിക്കുകയുണ്ടായി.) അദ്ദേഹം പറഞ്ഞു: എന്‍റെ ജനങ്ങളേ, നിങ്ങള്‍ അല്ലാഹുവിനെ ആരാധിക്കുക. നിങ്ങള്‍ക്ക് അവനല്ലാതെ യാതൊരു ദൈവവുമില്ല. നിങ്ങള്‍ കെട്ടിച്ചമച്ച് പറയുന്നവര്‍ മാത്രമാകുന്നു.
\end{malayalam}}
\flushright{\begin{Arabic}
\quranayah[11][51]
\end{Arabic}}
\flushleft{\begin{malayalam}
എന്‍റെ ജനങ്ങളേ, ഞാന്‍ നിങ്ങളോട് ഇതിന്‍റെ പേരില്‍ യാതൊരു പ്രതിഫലവും ആവശ്യപ്പെടുന്നില്ല. എനിക്കുള്ള പ്രതിഫലം എന്നെ സൃഷ്ടിച്ചവന്‍ തരേണ്ടത് മാത്രമാണ്‌. നിങ്ങള്‍ ചിന്തിച്ച് ഗ്രഹിക്കുന്നില്ലേ?
\end{malayalam}}
\flushright{\begin{Arabic}
\quranayah[11][52]
\end{Arabic}}
\flushleft{\begin{malayalam}
എന്‍റെ ജനങ്ങളേ, നിങ്ങള്‍ നിങ്ങളുടെ രക്ഷിതാവിനോട് പാപമോചനം തേടുക. എന്നിട്ട് അവങ്കലേക്ക് ഖേദിച്ചുമടങ്ങുകയും ചെയ്യുക. എങ്കില്‍ അവന്‍ നിങ്ങള്‍ക്ക് സമൃദ്ധമായി മഴ അയച്ചുതരികയും, നിങ്ങളുടെ ശക്തിയിലേക്ക് അവന്‍ കൂടുതല്‍ ശക്തി ചേര്‍ത്തുതരികയും ചെയ്യുന്നതാണ്‌. നിങ്ങള്‍ കുറ്റവാളികളായിക്കൊണ്ട് പിന്തിരിഞ്ഞ് പോകരുത്‌.
\end{malayalam}}
\flushright{\begin{Arabic}
\quranayah[11][53]
\end{Arabic}}
\flushleft{\begin{malayalam}
അവര്‍ പറഞ്ഞു: ഹൂടേ, നീ ഞങ്ങള്‍ക്ക് വ്യക്തമായ ഒരു തെളിവും കൊണ്ടു വന്നിട്ടില്ല. നീ പറഞ്ഞതിനാല്‍ മാത്രം ഞങ്ങള്‍ ഞങ്ങളുടെ ദൈവങ്ങളെ വിട്ടുകളയുന്നതല്ല. ഞങ്ങള്‍ നിന്നെ വിശ്വസിക്കുന്നതുമല്ല.
\end{malayalam}}
\flushright{\begin{Arabic}
\quranayah[11][54]
\end{Arabic}}
\flushleft{\begin{malayalam}
ഞങ്ങളുടെ ദൈവങ്ങളില്‍ ഒരാള്‍ നിനക്ക് എന്തോ ദോഷബാധ ഉളവാക്കിയിരിക്കുന്നു എന്ന് മാത്രമാണ് ഞങ്ങള്‍ക്ക് പറയാനുള്ളത്‌. ഹൂദ് പറഞ്ഞു: നിങ്ങള്‍ പങ്കാളികളായി ചേര്‍ക്കുന്ന യാതൊന്നുമായും എനിക്ക് ബന്ധമില്ല എന്നതിന് ഞാന്‍ അല്ലാഹുവെ സാക്ഷി നിര്‍ത്തുന്നു. (നിങ്ങളും) അതിന്ന് സാക്ഷികളായിരിക്കുക.
\end{malayalam}}
\flushright{\begin{Arabic}
\quranayah[11][55]
\end{Arabic}}
\flushleft{\begin{malayalam}
അല്ലാഹുവിന് പുറമെ. അതുകൊണ്ട് നിങ്ങളെല്ലാവരും കൂടി എനിക്കെതിരില്‍ തന്ത്രം പ്രയോഗിച്ച് കൊള്ളുക. എന്നിട്ട് നിങ്ങള്‍ എനിക്ക് ഇടതരികയും വേണ്ട.
\end{malayalam}}
\flushright{\begin{Arabic}
\quranayah[11][56]
\end{Arabic}}
\flushleft{\begin{malayalam}
എന്‍റെയും നിങ്ങളുടെയും രക്ഷിതാവായ അല്ലാഹുവിന്‍റെ മേല്‍ ഞാനിതാ ഭരമേല്‍പിച്ചിരിക്കുന്നു. യാതൊരു ജന്തുവും അവന്‍ അതിന്‍റെ നെറുകയില്‍ പിടിക്കുന്ന (നിയന്ത്രിക്കുന്ന) തായിട്ടില്ലാതെയില്ല. തീര്‍ച്ചയായും എന്‍റെ രക്ഷിതാവ് നേരായ പാതയിലാകുന്നു.
\end{malayalam}}
\flushright{\begin{Arabic}
\quranayah[11][57]
\end{Arabic}}
\flushleft{\begin{malayalam}
ഇനി നിങ്ങള്‍ പിന്തിരിഞ്ഞ് കളയുകയാണെങ്കില്‍, ഞാന്‍ നിങ്ങളുടെ അടുത്തേക്ക് അയക്കപ്പെട്ടത് ഏതൊരു കാര്യവുമായിട്ടാണോ അത് ഞാന്‍ നിങ്ങള്‍ക്ക് എത്തിച്ചുതന്നിട്ടുണ്ട്‌. നിങ്ങളല്ലാത്ത ഒരു ജനതയെ എന്‍റെ രക്ഷിതാവ് പകരം കൊണ്ടുവരുന്നതുമാണ്‌. അവന്ന് യാതൊരു ഉപദ്രവവും വരുത്താന്‍ നിങ്ങള്‍ക്കാവില്ല. തീര്‍ച്ചയായും എന്‍റെ രക്ഷിതാവ് എല്ലാ കാര്യവും സംരക്ഷിച്ച് പോരുന്നവനാകുന്നു.
\end{malayalam}}
\flushright{\begin{Arabic}
\quranayah[11][58]
\end{Arabic}}
\flushleft{\begin{malayalam}
നമ്മുടെ കല്‍പന വന്നപ്പോള്‍ ഹൂദിനെയും അദ്ദേഹത്തോടൊപ്പം വിശ്വസിച്ചവരെയും നമ്മുടെ കാരുണ്യം കൊണ്ട് നാം രക്ഷിച്ചു. കഠിനമായ ശിക്ഷയില്‍ നിന്ന് നാം അവരെ രക്ഷപ്പെടുത്തി.
\end{malayalam}}
\flushright{\begin{Arabic}
\quranayah[11][59]
\end{Arabic}}
\flushleft{\begin{malayalam}
അതാണ് ആദ് ജനത. തങ്ങളുടെ രക്ഷിതാവിന്‍റെ ദൃഷ്ടാന്തങ്ങളെ അവര്‍ നിഷേധിക്കുകയും, അവന്‍റെ ദൂതന്‍മാരെ അവര്‍ ധിക്കരിക്കുകയും, മര്‍ക്കടമുഷ്ടിക്കാരായ എല്ലാ സ്വേച്ഛാധിപതികളുടെയും കല്‍പന അവന്‍ പിന്‍പറ്റുകയും ചെയ്തു.
\end{malayalam}}
\flushright{\begin{Arabic}
\quranayah[11][60]
\end{Arabic}}
\flushleft{\begin{malayalam}
ഈ ഐഹികജീവിതത്തിലും ഉയിര്‍ത്തെഴുന്നേല്‍പിന്‍റെ നാളിലും ശാപം അവരുടെ പിന്നാലെ അയക്കപ്പെട്ടു. ശ്രദ്ധിക്കുക: തീര്‍ച്ചയായും ആദ് ജനത തങ്ങളുടെ രക്ഷിതാവിനോട് നന്ദികേട് കാണിച്ചിരിക്കുന്നു. ശ്രദ്ധിക്കുക: ഹൂദിന്‍റെ ജനതയായ ആദിന് നാശം!
\end{malayalam}}
\flushright{\begin{Arabic}
\quranayah[11][61]
\end{Arabic}}
\flushleft{\begin{malayalam}
ഥമൂദ് ജനതയിലേക്ക് അവരുടെ സഹോദരനായ സ്വാലിഹിനെയും (നാം നിയോഗിക്കുകയുണ്ടായി.) അദ്ദേഹം പറഞ്ഞു: എന്‍റെ ജനങ്ങളേ, നിങ്ങള്‍ അല്ലാഹുവെ ആരാധിക്കുക. നിങ്ങള്‍ക്ക് അവനല്ലാതെ യാതൊരു ദൈവവുമില്ല. അവന്‍ നിങ്ങളെ ഭൂമിയില്‍ നിന്ന് സൃഷ്ടിച്ച് വളര്‍ത്തുകയും നിങ്ങളെ അവിടെ അധിവസിപ്പിക്കുകയും ചെയ്തിരിക്കുന്നു. ആകയാല്‍ നിങ്ങള്‍ അവനോട് പാപമോചനം തേടുകയും, എന്നിട്ട് അവനിലേക്ക് ഖേദിച്ചുമടങ്ങുകയും ചെയ്യുക. തീര്‍ച്ചയായും എന്‍റെ രക്ഷിതാവ് അടുത്തു തന്നെയുള്ളവനും (പ്രാര്‍ത്ഥനക്ക്‌) ഉത്തരം നല്‍കുന്നവനുമാകുന്നു.
\end{malayalam}}
\flushright{\begin{Arabic}
\quranayah[11][62]
\end{Arabic}}
\flushleft{\begin{malayalam}
അവര്‍ പറഞ്ഞു: സ്വാലിഹേ, ഇതിനു മുമ്പ് നീ ഞങ്ങള്‍ക്കിടയില്‍ അഭിലഷണീയനായിരുന്നു. ഞങ്ങളുടെ പിതാക്കന്‍മാര്‍ ആരാധിച്ചു വരുന്നതിനെ ഞങ്ങള്‍ ആരാധിക്കുന്നതില്‍ നിന്ന് നീ ഞങ്ങളെ വിലക്കുകയാണോ? നീ ഞങ്ങളെ ക്ഷണിച്ച് കൊണ്ടിരിക്കുന്ന കാര്യത്തെപ്പറ്റി ഞങ്ങള്‍ അവിശ്വാസജനകമായ സംശയത്തിലാണ്‌.
\end{malayalam}}
\flushright{\begin{Arabic}
\quranayah[11][63]
\end{Arabic}}
\flushleft{\begin{malayalam}
അദ്ദേഹം പറഞ്ഞു: എന്‍റെ ജനങ്ങളേ, നിങ്ങള്‍ ചിന്തിച്ചിട്ടുണ്ടോ? ഞാന്‍ എന്‍റെ രക്ഷിതാവിങ്കല്‍ നിന്നുള്ള വ്യക്തമായ തെളിവിനെ അവലംബിക്കുന്നവനായിരിക്കുകയും, അവന്‍റെ പക്കല്‍ നിന്നുള്ള കാരുണ്യം അവനെനിക്ക് നല്‍കിയിരിക്കുകയുമാണെങ്കില്‍ -അല്ലാഹുവോട് ഞാന്‍ അനുസരണക്കേട് കാണിക്കുന്ന പക്ഷം- അവന്‍റെ ശിക്ഷയില്‍ നിന്ന് (രക്ഷിച്ചുകൊണ്ട്‌) എന്നെ സഹായിക്കാനാരുണ്ട്‌? അപ്പോള്‍ (കാര്യം ഇങ്ങനെയാണെങ്കില്‍) നിങ്ങള്‍ എനിക്ക് കൂടുതല്‍ നഷ്ടം വരുത്തിവെക്കുക മാത്രമേ ചെയ്യൂ.
\end{malayalam}}
\flushright{\begin{Arabic}
\quranayah[11][64]
\end{Arabic}}
\flushleft{\begin{malayalam}
എന്‍റെ ജനങ്ങളേ, ഇതാ നിങ്ങള്‍ക്കു ഒരു ദൃഷ്ടാന്തമായിക്കൊണ്ട് അല്ലാഹുവിന്‍റെ ഒട്ടകം. അല്ലാഹുവിന്‍റെ ഭൂമിയില്‍ നടന്ന് തിന്നുവാന്‍ നിങ്ങളതിനെ വിട്ടേക്കുക. നിങ്ങളതിന് ഒരു ദോഷവും വരുത്തിവെക്കരുത്‌. അങ്ങനെ ചെയ്യുന്ന പക്ഷം അടുത്തു തന്നെ ശിക്ഷ നിങ്ങളെ പിടികൂടുന്നതാണ്‌.
\end{malayalam}}
\flushright{\begin{Arabic}
\quranayah[11][65]
\end{Arabic}}
\flushleft{\begin{malayalam}
എന്നിട്ട് അവരതിനെ വെട്ടിക്കൊന്നു. അപ്പോള്‍ അദ്ദേഹം പറഞ്ഞു: നിങ്ങള്‍ മൂന്ന് ദിവസം നിങ്ങളുടെ വീടുകളില്‍ സൌഖ്യമനുഭവിച്ചു കൊള്ളുക. (അതോടെ ശിക്ഷ വന്നെത്തും.) തെറ്റാകാനിടയില്ലാത്ത ഒരു വാഗ്ദാനമാണിത്‌.
\end{malayalam}}
\flushright{\begin{Arabic}
\quranayah[11][66]
\end{Arabic}}
\flushleft{\begin{malayalam}
അങ്ങനെ നമ്മുടെ കല്‍പന വന്നപ്പോള്‍ സ്വാലിഹിനെയും അദ്ദേഹത്തോടൊപ്പം വിശ്വസിച്ചവരെയും നമ്മുടെ കാരുണ്യം കൊണ്ട് നാം രക്ഷപ്പെടുത്തി. ആ ദിവസത്തെ അപമാനത്തില്‍ നിന്നും (അവരെ നാം മോചിപ്പിച്ചു.) തീര്‍ച്ചയായും നിന്‍റെ രക്ഷിതാവ് തന്നെയാണ് ശക്തനും പ്രതാപവാനും.
\end{malayalam}}
\flushright{\begin{Arabic}
\quranayah[11][67]
\end{Arabic}}
\flushleft{\begin{malayalam}
അക്രമം പ്രവര്‍ത്തിച്ചവരെ ഘോരശബ്ദം പിടികൂടി. അങ്ങനെ പ്രഭാതമായപ്പോള്‍ അവര്‍ അവരുടെ വീടുകളില്‍ കമിഴ്ന്ന് വീണ അവസ്ഥയിലായിരുന്നു.
\end{malayalam}}
\flushright{\begin{Arabic}
\quranayah[11][68]
\end{Arabic}}
\flushleft{\begin{malayalam}
അവര്‍ അവിടെ താമസിച്ചിട്ടില്ലാത്ത പോലെ (അവര്‍ ഉന്‍മൂലനം ചെയ്യപ്പെട്ടു.) ശ്രദ്ധിക്കുക: തീര്‍ച്ചയായും ഥമൂദ് ജനത തങ്ങളുടെ രക്ഷിതാവിനോട് നന്ദികേട് കാണിച്ചു.ശ്രദ്ധിക്കുക: ഥമൂദ് ജനതയ്ക്ക് നാശം!
\end{malayalam}}
\flushright{\begin{Arabic}
\quranayah[11][69]
\end{Arabic}}
\flushleft{\begin{malayalam}
നമ്മുടെ ദൂതന്‍മാര്‍ ഇബ്രാഹീമിന്‍റെ അടുത്ത് സന്തോഷവാര്‍ത്തയും കൊണ്ട് വരികയുണ്ടായി. അവര്‍ പറഞ്ഞു: സലാം. അദ്ദേഹം പ്രതിവചിച്ചു. സലാം വൈകിയില്ല. അദ്ദേഹം ഒരു പൊരിച്ച മൂരിക്കുട്ടിയെ കൊണ്ട് വന്നു.
\end{malayalam}}
\flushright{\begin{Arabic}
\quranayah[11][70]
\end{Arabic}}
\flushleft{\begin{malayalam}
എന്നിട്ട് അവരുടെ കൈകള്‍ അതിലേക്ക് നീളുന്നില്ലെന്ന് കണ്ടപ്പോള്‍ അദ്ദേഹത്തിന് അവരുടെ കാര്യത്തില്‍ പന്തികേട് തോന്നുകയും അവരെ പറ്റി ഭയം അനുഭവപ്പെടുകയും ചെയ്തു. അവര്‍ പറഞ്ഞു: ഭയപ്പെടേണ്ട. ഞങ്ങള്‍ ലൂത്വിന്‍റെ ജനതയിലേക്ക് നിയോഗിക്കപ്പെട്ടിരിക്കുകയാണ്‌.
\end{malayalam}}
\flushright{\begin{Arabic}
\quranayah[11][71]
\end{Arabic}}
\flushleft{\begin{malayalam}
അദ്ദേഹത്തിന്‍റെ (ഇബ്രാഹീം നബി (അ) യുടെ) ഭാര്യ അവിടെ നില്‍ക്കുന്നുണ്ടായിരുന്നു. അവര്‍ ചിരിച്ചു. അപ്പോള്‍ അവര്‍ക്ക് ഇഷാഖിനെപ്പറ്റിയും, ഇഷാഖിന്‍റെ പിന്നാലെ യഅ്ഖൂബിനെപ്പറ്റിയും സന്തോഷവാര്‍ത്ത അറിയിച്ചു.
\end{malayalam}}
\flushright{\begin{Arabic}
\quranayah[11][72]
\end{Arabic}}
\flushleft{\begin{malayalam}
അവര്‍ പറഞ്ഞു: കഷ്ടം! ഞാനൊരു കിഴവിയായിട്ടും പ്രസവിക്കുകയോ? എന്‍റെ ഭര്‍ത്താവ് ഇതാ ഒരു വൃദ്ധന്‍! തീര്‍ച്ചയായും ഇതൊരു അത്ഭുതകരമായ കാര്യം തന്നെ.
\end{malayalam}}
\flushright{\begin{Arabic}
\quranayah[11][73]
\end{Arabic}}
\flushleft{\begin{malayalam}
അവര്‍ (ദൂതന്‍മാര്‍) പറഞ്ഞു: അല്ലാഹുവിന്‍റെ കല്‍പനയെപ്പറ്റി നീ അത്ഭുതപ്പെടുകയോ? ഹേ, വീട്ടുകാരേ, നിങ്ങളില്‍ അല്ലാഹുവിന്‍റെ കാരുണ്യവും അനുഗ്രഹങ്ങളുമുണ്ടായിരിക്കട്ടെ. തീര്‍ച്ചയായും അവന്‍ സ്തുത്യര്‍ഹനും മഹത്വമേറിയവനും ആകുന്നു.
\end{malayalam}}
\flushright{\begin{Arabic}
\quranayah[11][74]
\end{Arabic}}
\flushleft{\begin{malayalam}
അങ്ങനെ ഇബ്രാഹീമില്‍ നിന്ന് ഭയം വിട്ടുമാറുകയും, അദ്ദേഹത്തിന് സന്തോഷവാര്‍ത്ത വന്നുകിട്ടുകയും ചെയ്തപ്പോള്‍ അദ്ദേഹമതാ ലൂത്വിന്‍റെ ജനതയുടെ കാര്യത്തില്‍ നമ്മോട് തര്‍ക്കിക്കുന്നു.
\end{malayalam}}
\flushright{\begin{Arabic}
\quranayah[11][75]
\end{Arabic}}
\flushleft{\begin{malayalam}
തീര്‍ച്ചയായും ഇബ്രാഹീം സഹനശീലനും, ഏറെ അനുകമ്പയുള്ളവനും പശ്ചാത്താപമുള്ളവനും തന്നെയാണ്‌.
\end{malayalam}}
\flushright{\begin{Arabic}
\quranayah[11][76]
\end{Arabic}}
\flushleft{\begin{malayalam}
ഇബ്രാഹീമേ, ഇതില്‍ നിന്ന് പിന്തിരിഞ്ഞേക്കുക. തീര്‍ച്ചയായും നിന്‍റെ രക്ഷിതാവിന്‍റെ കല്‍പന വന്നു കഴിഞ്ഞു. തീര്‍ച്ചയായും അവര്‍ക്ക് റദ്ദാക്കപ്പെടാത്ത ശിക്ഷ വരുകയാകുന്നു.
\end{malayalam}}
\flushright{\begin{Arabic}
\quranayah[11][77]
\end{Arabic}}
\flushleft{\begin{malayalam}
നമ്മുടെ ദൂതന്‍മാര്‍ (മലക്കുകള്‍) ലൂത്വിന്‍റെ അടുക്കല്‍ ചെന്നപ്പോള്‍ അവരുടെ കാര്യത്തില്‍ അദ്ദേഹത്തിന് ദുഃഖം തോന്നുകയും അവരെ പറ്റി ചിന്തിച്ചിട്ട് അദ്ദേഹത്തിന് മനഃപ്രയാസമുണ്ടാവുകയും ചെയ്തു. ഇതൊരു വിഷമകരമായ ദിവസം തന്നെ എന്ന് അദ്ദേഹം പറയുകയും ചെയ്തു.
\end{malayalam}}
\flushright{\begin{Arabic}
\quranayah[11][78]
\end{Arabic}}
\flushleft{\begin{malayalam}
ലൂത്വിന്‍റെ ജനങ്ങള്‍ അദ്ദേഹത്തിന്‍റെ അടുത്തേക്ക് ഓടിവന്നു. മുമ്പു തന്നെ അവര്‍ ദുര്‍നടപ്പുകാരായിരുന്നു. അദ്ദേഹം പറഞ്ഞു: എന്‍റെ ജനങ്ങളേ, ഇതാ എന്‍റെ പെണ്‍മക്കള്‍. അവരാണ് നിങ്ങള്‍ക്ക് കൂടുതല്‍ പരിശുദ്ധിയുള്ളവര്‍. (അവരെ നിങ്ങള്‍ക്ക് വിവാഹം കഴിക്കാമല്ലോ?) അതിനാല്‍ നിങ്ങള്‍ അല്ലാഹുവെ സൂക്ഷിക്കുകയും എന്‍റെ അതിഥികളുടെ കാര്യത്തില്‍ എന്നെ അപമാനിക്കാതിരിക്കുകയും ചെയ്യുക. നിങ്ങളുടെ കൂട്ടത്തില്‍ വിവേകമുള്ള ഒരു പുരുഷനുമില്ലേ?
\end{malayalam}}
\flushright{\begin{Arabic}
\quranayah[11][79]
\end{Arabic}}
\flushleft{\begin{malayalam}
അവര്‍ പറഞ്ഞു: നിന്‍റെ പെണ്‍മക്കളെ ഞങ്ങള്‍ക്ക് ആവശ്യമില്ലെന്ന് നിനക്ക് അറിവുണ്ടല്ലോ? തീര്‍ച്ചയായും നിനക്കറിയാം; ഞങ്ങള്‍ എന്താണ് ഉദ്ദേശിക്കുന്നതെന്ന്‌.
\end{malayalam}}
\flushright{\begin{Arabic}
\quranayah[11][80]
\end{Arabic}}
\flushleft{\begin{malayalam}
അദ്ദേഹം പറഞ്ഞു: എനിക്ക് നിങ്ങളെ തടയുവാന്‍ ശക്തിയുണ്ടായിരുന്നുവെങ്കില്‍! അല്ലെങ്കില്‍ ശക്തനായ ഒരു സഹായിയെ എനിക്ക് ആശ്രയിക്കാനുണ്ടായിരുന്നുവെങ്കില്‍.
\end{malayalam}}
\flushright{\begin{Arabic}
\quranayah[11][81]
\end{Arabic}}
\flushleft{\begin{malayalam}
അവര്‍ പറഞ്ഞു: ലൂത്വേ, തീര്‍ച്ചയായും ഞങ്ങള്‍ നിന്‍റെ രക്ഷിതാവിന്‍റെ ദൂതന്‍മാരാണ്‌. അവര്‍ക്ക് (ജനങ്ങള്‍ക്ക്‌) നിന്‍റെ അടുത്തേക്കെത്താനാവില്ല. ആകയാല്‍ നീ രാത്രിയില്‍ നിന്നുള്ള ഒരു യാമത്തില്‍ നിന്‍റെ കുടുംബത്തേയും കൊണ്ട് യാത്ര പുറപ്പെട്ട് കൊള്ളുക. നിങ്ങളുടെ കൂട്ടത്തില്‍ നിന്ന് ഒരാളും തിരിഞ്ഞ് നോക്കരുത്‌. നിന്‍റെ ഭാര്യയൊഴികെ. തീര്‍ച്ചയായും അവര്‍ക്ക് (ജനങ്ങള്‍ക്ക്‌) വന്നുഭവിച്ച ശിക്ഷ അവള്‍ക്കും വന്നുഭവിക്കുന്നതാണ്‌. തീര്‍ച്ചയായും അവര്‍ക്ക് നിശ്ചയിച്ച അവധി പ്രഭാതമാകുന്നു. പ്രഭാതം അടുത്ത് തന്നെയല്ലേ?
\end{malayalam}}
\flushright{\begin{Arabic}
\quranayah[11][82]
\end{Arabic}}
\flushleft{\begin{malayalam}
അങ്ങനെ നമ്മുടെ കല്‍പന വന്നപ്പോള്‍ ആ രാജ്യത്തെ നാം കീഴ്മേല്‍ മറിക്കുകയും, അട്ടിയട്ടിയായി ചൂളവെച്ച ഇഷ്ടികക്കല്ലുകള്‍ നാം അവരുടെ മേല്‍ വര്‍ഷിക്കുകയും ചെയ്തു.
\end{malayalam}}
\flushright{\begin{Arabic}
\quranayah[11][83]
\end{Arabic}}
\flushleft{\begin{malayalam}
നിന്‍റെ രക്ഷിതാവിന്‍റെ അടുക്കല്‍ അടയാളം വെക്കപ്പെട്ടവയത്രെ (ആ കല്ലുകള്‍) അത് ഈ അക്രമികളില്‍ നിന്ന് അകലെയല്ല.
\end{malayalam}}
\flushright{\begin{Arabic}
\quranayah[11][84]
\end{Arabic}}
\flushleft{\begin{malayalam}
മദ്‌യങ്കാരിലേക്ക് അവരുടെ സഹോദരനായ ശുഐബിനേയും (നാം നിയോഗിക്കുകയുണ്ടായി.) അദ്ദേഹം പറഞ്ഞു: എന്‍റെ ജനങ്ങളേ, നിങ്ങള്‍ അല്ലാഹുവിനെ ആരാധിക്കുക. നിങ്ങള്‍ക്ക് അവനല്ലാതെ യാതൊരു ദൈവവുമില്ല. അളവിലും തൂക്കത്തിലും നിങ്ങള്‍ കുറവ് വരുത്തരുത്‌. തീര്‍ച്ചയായും നിങ്ങളെ ഞാന്‍ കാണുന്നത് ക്ഷേമത്തിലായിട്ടാണ്‌. നിങ്ങളെ ആകെ വലയം ചെയ്യുന്ന ഒരു ദിവസത്തെ ശിക്ഷ നിങ്ങളുടെ മേല്‍ തീര്‍ച്ചയായും ഞാന്‍ ഭയപ്പെടുന്നു.
\end{malayalam}}
\flushright{\begin{Arabic}
\quranayah[11][85]
\end{Arabic}}
\flushleft{\begin{malayalam}
എന്‍റെ ജനങ്ങളേ, നിങ്ങള്‍ അളവും തൂക്കവും നീതിപൂര്‍വ്വം പൂര്‍ണ്ണമാക്കികൊടുക്കുക. ജനങ്ങള്‍ക്ക് അവരുടെ സാധനങ്ങളില്‍ നിങ്ങള്‍ കമ്മിവരുത്താതിരിക്കുകയും ചെയ്യുക. നാശകാരികളായിക്കൊണ്ട് ഭൂമിയില്‍ നിങ്ങള്‍ കുഴപ്പമുണ്ടാക്കരുത്‌.
\end{malayalam}}
\flushright{\begin{Arabic}
\quranayah[11][86]
\end{Arabic}}
\flushleft{\begin{malayalam}
അല്ലാഹു ബാക്കിയാക്കിത്തരുന്നതാണ് നിങ്ങള്‍ക്ക് ഗുണകരമായിട്ടുള്ളത്‌; നിങ്ങള്‍ വിശ്വാസികളാണെങ്കില്‍. ഞാന്‍ നിങ്ങളുടെ മേല്‍ കാവല്‍ക്കാരനൊന്നുമല്ല.
\end{malayalam}}
\flushright{\begin{Arabic}
\quranayah[11][87]
\end{Arabic}}
\flushleft{\begin{malayalam}
അവര്‍ പറഞ്ഞു: ശുഐബേ, ഞങ്ങളുടെ പിതാക്കന്‍മാര്‍ ആരാധിച്ച് വരുന്നതിനെ ഞങ്ങള്‍ ഉപേക്ഷിക്കണമെന്നോ, ഞങ്ങളുടെ സ്വത്തുക്കളില്‍ ഞങ്ങള്‍ക്ക് ഇഷ്ടമുള്ള പ്രകാരം പ്രവര്‍ത്തിക്കാന്‍ പാടില്ലെന്നോ നിനക്ക് കല്‍പന നല്‍കുന്നത് നിന്‍റെ ഈ നമസ്കാരമാണോ? തീര്‍ച്ചയായും നീ സഹനശീലനും വിവേകശാലിയുമാണല്ലോ ?
\end{malayalam}}
\flushright{\begin{Arabic}
\quranayah[11][88]
\end{Arabic}}
\flushleft{\begin{malayalam}
അദ്ദേഹം പറഞ്ഞു: എന്‍റെ ജനങ്ങളേ, നിങ്ങള്‍ ചിന്തിച്ച് നോക്കിയിട്ടുണ്ടോ? ഞാന്‍ എന്‍റെ രക്ഷിതാവിങ്കല്‍ നിന്നുള്ള വ്യക്തമായ തെളിവിനെ അവലംബിക്കുന്നവനായിരിക്കുകയും, അവന്‍ എനിക്ക് അവന്‍റെ വകയായി ഉത്തമമായ ഉപജീവനം നല്‍കിയിരിക്കുകയുമാണെങ്കില്‍ (എനിക്കെങ്ങനെ സത്യം മറച്ചു വെക്കാന്‍ കഴിയും.) നിങ്ങളെ ഞാന്‍ ഒരു കാര്യത്തില്‍ നിന്ന് വിലക്കുകയും, എന്നിട്ട് നിങ്ങളില്‍ നിന്ന് വ്യത്യസ്തനായിക്കൊണ്ട് ഞാന്‍ തന്നെ അത് പ്രവര്‍ത്തിക്കുകയും ചെയ്യണമെന്ന് ഉദ്ദേശിക്കുന്നുമില്ല. എനിക്ക് സാധ്യമായത്ര നന്‍മവരുത്താനല്ലാതെ മറ്റൊന്നും ഞാന്‍ ഉദ്ദേശിക്കുന്നില്ല. അല്ലാഹു മുഖേന മാത്രമാണ് എനിക്ക് (അതിന്‌) അനുഗ്രഹം ലഭിക്കുന്നത്‌. അവന്‍റെ മേലാണ് ഞാന്‍ ഭരമേല്‍പിച്ചിരിക്കുന്നത്‌. അവനിലേക്ക് ഞാന്‍ താഴ്മയോടെ മടങ്ങുകയും ചെയ്യുന്നു.
\end{malayalam}}
\flushright{\begin{Arabic}
\quranayah[11][89]
\end{Arabic}}
\flushleft{\begin{malayalam}
എന്‍റെ ജനങ്ങളേ, നൂഹിന്‍റെ ജനതയ്ക്കോ, ഹൂദിന്‍റെ ജനതയ്ക്കോ, സ്വാലിഹിന്‍റെ ജനതയ്ക്കോ ബാധിച്ചത് പോലെയുള്ള ശിക്ഷ നിങ്ങള്‍ക്കും ബാധിക്കുവാന്‍ എന്നോടുള്ള മാത്സര്യം നിങ്ങള്‍ക്ക് ഇടവരുത്താതിരിക്കട്ടെ. ലൂത്വിന്‍റെ ജനത നിങ്ങളില്‍ നിന്ന് അകലെയല്ല താനും.
\end{malayalam}}
\flushright{\begin{Arabic}
\quranayah[11][90]
\end{Arabic}}
\flushleft{\begin{malayalam}
നിങ്ങള്‍ നിങ്ങളുടെ രക്ഷിതാവിനോട് പാപമോചനം തേടുകയും എന്നിട്ട് അവനിലേക്ക് ഖേദിച്ചുമടങ്ങുകയും ചെയ്യുക. തീര്‍ച്ചയായും എന്‍റെ രക്ഷിതാവ് ഏറെ കരുണയുള്ളവനും ഏറെ സ്നേഹമുള്ളവനുമത്രെ.
\end{malayalam}}
\flushright{\begin{Arabic}
\quranayah[11][91]
\end{Arabic}}
\flushleft{\begin{malayalam}
അവര്‍ പറഞ്ഞു: ശുഐബേ, നീ പറയുന്നതില്‍ നിന്ന് അധികഭാഗവും ഞങ്ങള്‍ക്ക് മനസ്സിലാകുന്നില്ല. തീര്‍ച്ചയായും ഞങ്ങളില്‍ ബലഹീനനായിട്ടാണ് നിന്നെ ഞങ്ങള്‍ കാണുന്നത്‌. നിന്‍റെ കുടുംബങ്ങള്‍ ഇല്ലായിരുന്നെങ്കില്‍ നിന്നെ ഞങ്ങള്‍ എറിഞ്ഞ് കൊല്ലുക തന്നെ ചെയ്യുമായിരുന്നു. ഞങ്ങളെ സംബന്ധിച്ചേടത്തോളം നീയൊരു പ്രതാപവാനൊന്നുമല്ല.
\end{malayalam}}
\flushright{\begin{Arabic}
\quranayah[11][92]
\end{Arabic}}
\flushleft{\begin{malayalam}
അദ്ദേഹം പറഞ്ഞു: എന്‍റെ ജനങ്ങളേ, എന്‍റെ കുടുംബങ്ങളാണോ നിങ്ങളെ സംബന്ധിച്ചിടത്തോളം അല്ലാഹുവെക്കാള്‍ കൂടുതല്‍ പ്രതാപമുള്ളവര്‍? എന്നിട്ട് അവനെ നിങ്ങള്‍ നിങ്ങളുടെ പിന്നിലേക്ക് പുറം തള്ളിക്കളഞ്ഞിരിക്കുകയാണോ? തീര്‍ച്ചയായും എന്‍റെ രക്ഷിതാവ് നിങ്ങള്‍ പ്രവര്‍ത്തിക്കുന്നതെല്ലാം സൂക്ഷ്മമായി അറിയുന്നവനാകുന്നു.
\end{malayalam}}
\flushright{\begin{Arabic}
\quranayah[11][93]
\end{Arabic}}
\flushleft{\begin{malayalam}
എന്‍റെ ജനങ്ങളേ, നിങ്ങളുടെ നിലപാടനുസരിച്ച് നിങ്ങള്‍ പ്രവര്‍ത്തിച്ച് കൊള്ളുക. തീര്‍ച്ചയായും ഞാനും പ്രവര്‍ത്തിച്ച് കൊണ്ടിരിക്കുകയാണ്‌. ആര്‍ക്കാണ് അപമാനകരമായ ശിക്ഷ വരുന്നതെന്നും ആരാണ് കള്ളം പറയുന്നവരെന്നും പുറകെ നിങ്ങള്‍ക്കറിയാം. നിങ്ങള്‍ കാത്തിരിക്കുക. തീര്‍ച്ചയായും ഞാനും നിങ്ങളോടൊപ്പം കാത്തിരിക്കുകയാണ്‌.
\end{malayalam}}
\flushright{\begin{Arabic}
\quranayah[11][94]
\end{Arabic}}
\flushleft{\begin{malayalam}
നമ്മുടെ കല്‍പന വന്നപ്പോള്‍ ശുഐബിനെയും അദ്ദേഹത്തോടൊപ്പം വിശ്വസിച്ചവരെയും നമ്മുടെ കാരുണ്യം കൊണ്ട് നാം രക്ഷപ്പെടുത്തി. അക്രമം ചെയ്തവരെ ഘോരശബ്ദം പിടികൂടുകയും ചെയ്തു. അങ്ങനെ നേരം പുലര്‍ന്നപ്പോള്‍ അവര്‍ തങ്ങളുടെ പാര്‍പ്പിടങ്ങളില്‍ കമിഴ്ന്നു വീണുകിടക്കുകയായിരുന്നു.
\end{malayalam}}
\flushright{\begin{Arabic}
\quranayah[11][95]
\end{Arabic}}
\flushleft{\begin{malayalam}
അവര്‍ അവിടെ താമസിച്ചിട്ടില്ലാത്ത പോലെ (സ്ഥലം ശൂന്യമായി) ശ്രദ്ധിക്കുക: ഥമൂദ് നശിച്ചത് പോലെതന്നെ മദ്‌യനിന്നും നാശം.
\end{malayalam}}
\flushright{\begin{Arabic}
\quranayah[11][96]
\end{Arabic}}
\flushleft{\begin{malayalam}
നമ്മുടെ ദൃഷ്ടാന്തങ്ങളും വ്യക്തമായ പ്രമാണവുമായി മൂസായെ നാം നിയോഗിക്കുകയുണ്ടായി.
\end{malayalam}}
\flushright{\begin{Arabic}
\quranayah[11][97]
\end{Arabic}}
\flushleft{\begin{malayalam}
ഫിര്‍ഔന്‍റെയും അവന്‍റെ പ്രമാണികളുടെയും അടുത്തേക്ക്‌. എന്നിട്ട് അവര്‍ (പ്രമാണിമാര്‍) ഫിര്‍ഔന്‍റെ കല്‍പന പിന്‍പറ്റുകയാണ് ചെയ്തത്‌. ഫിര്‍ഔന്‍റെ കല്‍പനയാകട്ടെ വിവേകപൂര്‍ണ്ണമല്ലതാനും.
\end{malayalam}}
\flushright{\begin{Arabic}
\quranayah[11][98]
\end{Arabic}}
\flushleft{\begin{malayalam}
ഉയിര്‍ത്തെഴുന്നേല്‍പിന്‍റെ നാളില്‍ അവന്‍ (ഫിര്‍ഔന്‍) തന്‍റെ ജനതയുടെ മുമ്പിലുണ്ടായിരിക്കും. എന്നിട്ട് അവരെ അവന്‍ നരകത്തിലേക്കാനയിക്കും. (അവര്‍) ആനയിക്കപ്പെടുന്ന ആ സ്ഥാനം എത്ര ചീത്ത!
\end{malayalam}}
\flushright{\begin{Arabic}
\quranayah[11][99]
\end{Arabic}}
\flushleft{\begin{malayalam}
ഈ ലോകത്തും ഉയിര്‍ത്തെഴുന്നേല്‍പിന്‍റെ നാളിലും ശാപം അവരുടെ പിന്നാലെ അയക്കപ്പെട്ടിരിക്കുന്നു. (അവര്‍ക്ക്‌) നല്‍കപ്പെട്ട ആ സമ്മാനം എത്ര ചീത്ത!
\end{malayalam}}
\flushright{\begin{Arabic}
\quranayah[11][100]
\end{Arabic}}
\flushleft{\begin{malayalam}
വിവിധ രാജ്യങ്ങളുടെ വൃത്താന്തങ്ങളില്‍ ചിലതത്രെ അത്‌. നാമത് നിനക്ക് വിവരിച്ചുതരുന്നു. അവയില്‍ (ആ രാജ്യങ്ങളില്‍) ചിലതു നിലനില്‍ക്കുന്നുണ്ട്‌. ചിലത് ഉന്‍മൂലനം ചെയ്യപ്പെട്ടിട്ടുമുണ്ട്‌.
\end{malayalam}}
\flushright{\begin{Arabic}
\quranayah[11][101]
\end{Arabic}}
\flushleft{\begin{malayalam}
നാം അവരോട് അക്രമം ചെയ്തിട്ടില്ല. പക്ഷെ അവര്‍ അവരോട് തന്നെ അക്രമം പ്രവര്‍ത്തിക്കുകയാണുണ്ടായത്‌. എന്നാല്‍ നിന്‍റെ രക്ഷിതാവിന്‍റെ കല്‍പന വന്ന സമയത്ത് അല്ലാഹുവിന് പുറമെ അവര്‍ വിളിച്ച് പ്രാര്‍ത്ഥിച്ച് കൊണ്ടിരിക്കുന്ന അവരുടെ ദൈവങ്ങള്‍ അവര്‍ക്ക് യാതൊരു ഉപകാരവും ചെയ്തില്ല. അവര്‍ (ദൈവങ്ങള്‍) അവര്‍ക്ക് നാശം വര്‍ദ്ധിപ്പിക്കുക മാത്രമാണ് ചെയ്തത്‌.
\end{malayalam}}
\flushright{\begin{Arabic}
\quranayah[11][102]
\end{Arabic}}
\flushleft{\begin{malayalam}
വിവിധ രാജ്യക്കാര്‍ അക്രമികളായിരിക്കെ അവരെ പിടികൂടി ശിക്ഷിക്കുമ്പോള്‍ നിന്‍റെ രക്ഷിതാവിന്‍റെ പിടുത്തം അപ്രകാരമാകുന്നു. തീര്‍ച്ചയായും അവന്‍റെ പിടുത്തം വേദനയേറിയതും കഠിനമായതുമാണ്‌.
\end{malayalam}}
\flushright{\begin{Arabic}
\quranayah[11][103]
\end{Arabic}}
\flushleft{\begin{malayalam}
പരലോകശിക്ഷയെ ഭയപ്പെടുന്നവര്‍ക്ക് തീര്‍ച്ചയായും അതില്‍ ദൃഷ്ടാന്തമുണ്ട്‌. സര്‍വ്വ മനുഷ്യരും സമ്മേളിപ്പിക്കപ്പെടുന്ന ഒരു ദിവസമാണത്‌. (സര്‍വ്വരുടെയും) സാന്നിദ്ധ്യമുണ്ടാകുന്ന ഒരു ദിവസമാകുന്നു അത്‌.
\end{malayalam}}
\flushright{\begin{Arabic}
\quranayah[11][104]
\end{Arabic}}
\flushleft{\begin{malayalam}
നിര്‍ണിതമായ ഒരു അവധിവരെ മാത്രമാണ് നാമത് നീട്ടിവെക്കുന്നത്‌.
\end{malayalam}}
\flushright{\begin{Arabic}
\quranayah[11][105]
\end{Arabic}}
\flushleft{\begin{malayalam}
ആ അവധി വന്നെത്തുന്ന ദിവസം യാതൊരാളും അവന്‍റെ (അല്ലാഹുവിന്‍റെ) അനുമതിയോടെയല്ലാതെ സംസാരിക്കുകയില്ല. അപ്പോള്‍ അവരുടെ കൂട്ടത്തില്‍ നിര്‍ഭാഗ്യവാനും സൌഭാഗ്യവാനുമുണ്ടാകും.
\end{malayalam}}
\flushright{\begin{Arabic}
\quranayah[11][106]
\end{Arabic}}
\flushleft{\begin{malayalam}
എന്നാല്‍ നിര്‍ഭാഗ്യമടഞ്ഞവരാകട്ടെ അവര്‍ നരകത്തിലായിരിക്കും. അവര്‍ക്കവിടെ നെടുവീര്‍പ്പും തേങ്ങിക്കരച്ചിലുമാണുണ്ടായിരിക്കുക.
\end{malayalam}}
\flushright{\begin{Arabic}
\quranayah[11][107]
\end{Arabic}}
\flushleft{\begin{malayalam}
ആകാശങ്ങളും ഭൂമിയും നിലനില്‍ക്കുന്നേടത്തോളം () അവരതില്‍ നിത്യവാസികളായിരിക്കും. നിന്‍റെ രക്ഷിതാവ് ഉദ്ദേശിച്ചതൊഴികെ. തീര്‍ച്ചയായും നിന്‍റെ രക്ഷിതാവ് താന്‍ ഉദ്ദേശിക്കുന്നത് തികച്ചും നടപ്പിലാക്കുന്നവനാകുന്നു.
\end{malayalam}}
\flushright{\begin{Arabic}
\quranayah[11][108]
\end{Arabic}}
\flushleft{\begin{malayalam}
എന്നാല്‍ സൌഭാഗ്യം സിദ്ധിച്ചവരാകട്ടെ, അവര്‍ സ്വര്‍ഗത്തിലായിരിക്കും. ആകാശങ്ങളും ഭൂമിയും നിലനില്‍ക്കുന്നിടത്തോളം അവരതില്‍ നിത്യവാസികളായിരിക്കും. നിന്‍റെ രക്ഷിതാവ് ഉദ്ദേശിച്ചതൊഴികെ. നിലച്ചുപോകാത്ത ഒരു ദാനമായിരിക്കും അത്‌.
\end{malayalam}}
\flushright{\begin{Arabic}
\quranayah[11][109]
\end{Arabic}}
\flushleft{\begin{malayalam}
അപ്പോള്‍ ഇക്കൂട്ടര്‍ ആരാധിച്ച് വരുന്നതിനെ സംബന്ധിച്ച് നീ യാതൊരു സംശയത്തിലും അകപ്പെടരുത്‌. മുമ്പ് ഇവരുടെ പിതാക്കന്‍മാര്‍ ആരാധിച്ചിരുന്ന അതേ രീതിയില്‍ ആരാധന നടത്തുക മാത്രമാണിവര്‍ ചെയ്യുന്നത്‌. തീര്‍ച്ചയായും അവര്‍ക്കുള്ള ഓഹരി കുറവൊന്നും വരുത്താതെ നാമവര്‍ക്ക് നിറവേറ്റികൊടുക്കുന്നതാണ്‌.
\end{malayalam}}
\flushright{\begin{Arabic}
\quranayah[11][110]
\end{Arabic}}
\flushleft{\begin{malayalam}
മൂസായ്ക്ക് നാം വേദഗ്രന്ഥം നല്‍കുകയുണ്ടായി. എന്നിട്ട് അതിന്‍റെ കാര്യത്തില്‍ അഭിപ്രായഭിന്നതകള്‍ ഉണ്ടായി. നിന്‍റെ രക്ഷിതാവിങ്കല്‍ നിന്ന് ഒരു വചനം മുന്‍കൂട്ടി ഉണ്ടായിരുന്നില്ലെങ്കില്‍ അവര്‍ക്കിടയില്‍ വിധി നടത്തപ്പെടുമായിരുന്നു. തീര്‍ച്ചയായും അവര്‍ ഇതിനെപ്പറ്റി അവിശ്വാസജനകമായ സംശയത്തിലാകുന്നു.
\end{malayalam}}
\flushright{\begin{Arabic}
\quranayah[11][111]
\end{Arabic}}
\flushleft{\begin{malayalam}
തീര്‍ച്ചയായും അവരില്‍ ഓരോ വിഭാഗത്തിനും നിന്‍റെ രക്ഷിതാവ് അവരവരുടെ കര്‍മ്മങ്ങള്‍ക്കുള്ള പ്രതിഫലം പൂര്‍ണ്ണമായി നല്‍കുകതന്നെ ചെയ്യും. തീര്‍ച്ചയായും അവന്‍ അവര്‍ പ്രവര്‍ത്തിക്കുന്നതിനെപ്പറ്റി സൂക്ഷ്മമായി അറിയുന്നവനാകുന്നു.
\end{malayalam}}
\flushright{\begin{Arabic}
\quranayah[11][112]
\end{Arabic}}
\flushleft{\begin{malayalam}
ആകയാല്‍ നീ കല്‍പിക്കപ്പെട്ടതു പോലെ നീയും നിന്നോടൊപ്പം (അല്ലാഹുവിങ്കലേക്ക്‌) മടങ്ങിയവരും നേരായ മാര്‍ഗത്തില്‍ നിലകൊള്ളുക. നിങ്ങള്‍ അതിരുവിട്ട് പ്രവര്‍ത്തിക്കരുത്‌. തീര്‍ച്ചയായും അവന്‍ നിങ്ങള്‍ പ്രവര്‍ത്തിക്കുന്നതെല്ലാം കണ്ടറിയുന്നവനാണ്‌.
\end{malayalam}}
\flushright{\begin{Arabic}
\quranayah[11][113]
\end{Arabic}}
\flushleft{\begin{malayalam}
അക്രമം പ്രവര്‍ത്തിച്ചവരുടെ പക്ഷത്തേക്ക് നിങ്ങള്‍ ചായരുത്‌. എങ്കില്‍ നരകം നിങ്ങളെ സ്പര്‍ശിക്കുന്നതാണ്‌. അല്ലാഹുവിന് പുറമെ നിങ്ങള്‍ക്ക് രക്ഷാധികാരികളേയില്ല. പിന്നീട് നിങ്ങള്‍ സഹായിക്കപ്പെടുന്നതല്ല.
\end{malayalam}}
\flushright{\begin{Arabic}
\quranayah[11][114]
\end{Arabic}}
\flushleft{\begin{malayalam}
പകലിന്‍റെ രണ്ടറ്റങ്ങളിലും, രാത്രിയിലെ ആദ്യയാമങ്ങളിലും നീ നമസ്കാരം മുറപോലെ നിര്‍വഹിക്കുക. തീര്‍ച്ചയായും സല്‍കര്‍മ്മങ്ങള്‍ ദുഷ്കര്‍മ്മങ്ങളെ നീക്കികളയുന്നതാണ്‌. ചിന്തിച്ചു ഗ്രഹിക്കുന്നവര്‍ക്ക് ഒരു ഉല്‍ബോധനമാണത്‌.
\end{malayalam}}
\flushright{\begin{Arabic}
\quranayah[11][115]
\end{Arabic}}
\flushleft{\begin{malayalam}
നീ ക്ഷമിക്കുക. സുകൃതവാന്‍മാരുടെ പ്രതിഫലം അല്ലാഹു നഷ്ടപ്പെടുത്തിക്കളയുകയില്ല;തീര്‍ച്ച.
\end{malayalam}}
\flushright{\begin{Arabic}
\quranayah[11][116]
\end{Arabic}}
\flushleft{\begin{malayalam}
ഭൂമിയില്‍ നാശമുണ്ടാക്കുന്നതില്‍ നിന്ന് (ജനങ്ങളെ) തടയുന്ന, (നന്‍മയുടെ) പാരമ്പര്യമുള്ള ഒരു വിഭാഗം നിങ്ങള്‍ക്കുമുമ്പുള്ള തലമുറകളില്‍ നിന്ന് എന്തുകൊണ്ട് ഉണ്ടായില്ല? അവരില്‍ നിന്ന് നാം രക്ഷപ്പെടുത്തി എടുത്ത കൂട്ടത്തില്‍ പെട്ട ചുരുക്കം ചിലരൊഴികെ. എന്നാല്‍ അക്രമകാരികള്‍ തങ്ങള്‍ക്ക് നല്‍കപ്പെട്ട സുഖാഡംബരങ്ങളുടെ പിന്നാലെ പോകുകയാണ് ചെയ്തത്‌. അവര്‍ കുറ്റവാളികളായിരിക്കുന്നു.
\end{malayalam}}
\flushright{\begin{Arabic}
\quranayah[11][117]
\end{Arabic}}
\flushleft{\begin{malayalam}
നാട്ടുകാര്‍ സല്‍പ്രവൃത്തികള്‍ ചെയ്യുന്നവരായിരിക്കെ നിന്‍റെ രക്ഷിതാവ് അന്യായമായി രാജ്യങ്ങള്‍ നശിപ്പിക്കുന്നതല്ല.
\end{malayalam}}
\flushright{\begin{Arabic}
\quranayah[11][118]
\end{Arabic}}
\flushleft{\begin{malayalam}
നിന്‍റെ രക്ഷിതാവ് ഉദ്ദേശിച്ചിരുന്നെങ്കില്‍ മനുഷ്യരെ അവന്‍ ഒരൊറ്റ സമുദായമാക്കുമായിരുന്നു. (എന്നാല്‍) അവര്‍ ഭിന്നിച്ചുകൊണേ്ടയിരിക്കുന്നതാണ്‌.
\end{malayalam}}
\flushright{\begin{Arabic}
\quranayah[11][119]
\end{Arabic}}
\flushleft{\begin{malayalam}
നിന്‍റെ രക്ഷിതാവ് കരുണ ചെയ്തവരൊഴികെ. അതിന്നുവേണ്ടിയാണ് അവന്‍ അവരെ സൃഷ്ടിച്ചത്‌. ജിന്നുകള്‍, മനുഷ്യര്‍ എന്നീ രണ്ട് വിഭാഗത്തെയും കൊണ്ട് ഞാന്‍ നരകം നിറക്കുക തന്നെ ചെയ്യുന്നതാണ് എന്ന നിന്‍റെ രക്ഷിതാവിന്‍റെ വചനം നിറവേറിയിരിക്കുന്നു.
\end{malayalam}}
\flushright{\begin{Arabic}
\quranayah[11][120]
\end{Arabic}}
\flushleft{\begin{malayalam}
ദൈവദൂതന്‍മാരുടെ വൃത്താന്തങ്ങളില്‍ നിന്ന് നിന്‍റെ മനസ്സിന് സൈ്ഥര്യം നല്‍കുന്നതെല്ലാം നിനക്ക് നാം വിവരിച്ചുതരുന്നു. ഇതിലൂടെ യഥാര്‍ത്ഥ വിവരവും, സത്യവിശ്വാസികള്‍ക്ക് വേണ്ട സദുപദേശവും, ഉല്‍ബോധനവും നിനക്ക് വന്നുകിട്ടിയിരിക്കുകയാണ്‌.
\end{malayalam}}
\flushright{\begin{Arabic}
\quranayah[11][121]
\end{Arabic}}
\flushleft{\begin{malayalam}
വിശ്വസിക്കാത്തവരോട് നീ പറയുക: നിങ്ങള്‍ നിങ്ങളുടെ നിലപാടനുസരിച്ച് പ്രവര്‍ത്തിച്ച് കൊള്ളുക. തീര്‍ച്ചയായും ഞങ്ങളും പ്രവര്‍ത്തിക്കുകയാണ്‌.
\end{malayalam}}
\flushright{\begin{Arabic}
\quranayah[11][122]
\end{Arabic}}
\flushleft{\begin{malayalam}
നിങ്ങള്‍ കാത്തിരിക്കുക. തീര്‍ച്ചയായും ഞങ്ങളും കാത്തിരിക്കുകയാണ്‌.
\end{malayalam}}
\flushright{\begin{Arabic}
\quranayah[11][123]
\end{Arabic}}
\flushleft{\begin{malayalam}
ആകാശഭൂമികളിലെ അദൃശ്യയാഥാര്‍ത്ഥ്യങ്ങളെ പറ്റിയുള്ള അറിവ് അല്ലാഹുവിന്നുള്ളതാണ്‌. അവങ്കലേക്ക് തന്നെ കാര്യമെല്ലാം മടക്കപ്പെടുകയും ചെയ്യും. ആകയാല്‍ നീ അവനെ ആരാധിക്കുകയും, അവന്‍റെ മേല്‍ ഭരമേല്‍പിക്കുകയും ചെയ്യുക. നിങ്ങള്‍ പ്രവര്‍ത്തിക്കുന്നതിനെപ്പറ്റിയൊന്നും നിന്‍റെ രക്ഷിതാവ് അശ്രദ്ധനല്ല.
\end{malayalam}}
\chapter{\textmalayalam{യൂസുഫ്}}
\begin{Arabic}
\Huge{\centerline{\basmalah}}\end{Arabic}
\flushright{\begin{Arabic}
\quranayah[12][1]
\end{Arabic}}
\flushleft{\begin{malayalam}
അലിഫ്‌-ലാം-റാ. സ്പഷ്ടമായ വേദഗ്രന്ഥത്തിലെ വചനങ്ങളാകുന്നു അവ.
\end{malayalam}}
\flushright{\begin{Arabic}
\quranayah[12][2]
\end{Arabic}}
\flushleft{\begin{malayalam}
നിങ്ങള്‍ ഗ്രഹിക്കുന്നതിന് വേണ്ടി അത് അറബിഭാഷയില്‍ വായിക്കപ്പെടുന്ന ഒരു പ്രമാണമായി അവതരിപ്പിച്ചിരിക്കുന്നു.
\end{malayalam}}
\flushright{\begin{Arabic}
\quranayah[12][3]
\end{Arabic}}
\flushleft{\begin{malayalam}
നിനക്ക് ഈ ഖുര്‍ആന്‍ ബോധനം നല്‍കിയത് വഴി ഏറ്റവും നല്ല ചരിത്രവിവരണമാണ് നാം നിനക്ക് നല്‍കിക്കൊണ്ടിരിക്കുന്നത്‌. തീര്‍ച്ചയായും ഇതിനുമുമ്പ് നീ അതിനെപ്പറ്റി ബോധമില്ലാത്തവനായിരുന്നു.
\end{malayalam}}
\flushright{\begin{Arabic}
\quranayah[12][4]
\end{Arabic}}
\flushleft{\begin{malayalam}
യൂസുഫ് തന്‍റെ പിതാവിനോട് പറഞ്ഞ സന്ദര്‍ഭം: എന്‍റെ പിതാവേ, പതിനൊന്നു നക്ഷത്രങ്ങളും സൂര്യനും ചന്ദ്രനും എനിക്ക് സാഷ്ടാംഗം ചെയ്യുന്നതായി ഞാന്‍ സ്വപ്നം കണ്ടിരിക്കുന്നു.
\end{malayalam}}
\flushright{\begin{Arabic}
\quranayah[12][5]
\end{Arabic}}
\flushleft{\begin{malayalam}
അദ്ദേഹം (പിതാവ്‌) പറഞ്ഞു: എന്‍റെ കുഞ്ഞുമകനേ, നിന്‍റെ സ്വപ്നം നീ നിന്‍റെ സഹോദരന്‍മാര്‍ക്ക് വിവരിച്ചുകൊടുക്കരുത്‌. അവര്‍ നിനക്കെതിരെ വല്ല തന്ത്രവും പ്രയോഗിച്ചേക്കും. തീര്‍ച്ചയായും പിശാച് മനുഷ്യന്‍റെ പ്രത്യക്ഷ ശത്രുവാകുന്നു.
\end{malayalam}}
\flushright{\begin{Arabic}
\quranayah[12][6]
\end{Arabic}}
\flushleft{\begin{malayalam}
അപ്രകാരം നിന്‍റെ രക്ഷിതാവ് നിന്നെ തെരഞ്ഞെടുക്കുകയും, സ്വപ്നവാര്‍ത്തകളുടെ വ്യാഖ്യാനത്തില്‍ നിന്ന് നിനക്കവന്‍ പഠിപ്പിച്ചുതരികയും, നിന്‍റെ മേലും യഅ്ഖൂബ് കുടുംബത്തിന്‍റെ മേലും അവന്‍റെ അനുഗ്രഹങ്ങള്‍ അവന്‍ നിറവേറ്റുകയും ചെയ്യുന്നതാണ്‌. മുമ്പ് നിന്‍റെ രണ്ട് പിതാക്കളായ ഇബ്രാഹീമിന്‍റെയും ഇഷാഖിന്‍റെയും കാര്യത്തില്‍ അതവന്‍ നിറവേറ്റിയത് പോലെത്തന്നെ. തീര്‍ച്ചയായും നിന്‍റെ രക്ഷിതാവ് സര്‍വ്വജ്ഞനും യുക്തിമാനുമാകുന്നു.
\end{malayalam}}
\flushright{\begin{Arabic}
\quranayah[12][7]
\end{Arabic}}
\flushleft{\begin{malayalam}
തീര്‍ച്ചയായും യൂസുഫിലും അദ്ദേഹത്തിന്‍റെ സഹോദരന്‍മാരിലും ചോദിച്ച് മനസ്സിലാക്കുന്നവര്‍ക്ക് പല ദൃഷ്ടാന്തങ്ങളുമുണ്ട്‌.
\end{malayalam}}
\flushright{\begin{Arabic}
\quranayah[12][8]
\end{Arabic}}
\flushleft{\begin{malayalam}
യൂസുഫും അവന്‍റെ സഹോദരനുമാണ് നമ്മുടെ പിതാവിന് നമ്മളെക്കാള്‍ ഇഷ്ടപ്പെട്ടവര്‍. നമ്മളാകട്ടെ ഒരു (പ്രബലമായ) സംഘമാണ് താനും. തീര്‍ച്ചയായും നമ്മുടെ പിതാവ് വ്യക്തമായ വഴിപിഴവില്‍ തന്നെയാണ്‌.
\end{malayalam}}
\flushright{\begin{Arabic}
\quranayah[12][9]
\end{Arabic}}
\flushleft{\begin{malayalam}
നിങ്ങള്‍ യൂസുഫിനെ കൊന്നുകളയുക. അല്ലെങ്കില്‍ വല്ല ഭൂപ്രദേശത്തും അവനെ (കൊണ്ടുപോയി) ഇട്ടേക്കുക. എങ്കില്‍ നിങ്ങളുടെ പിതാവിന്‍റെ മുഖം നിങ്ങള്‍ക്ക് ഒഴിഞ്ഞ് കിട്ടും. അതിന് ശേഷം നിങ്ങള്‍ക്ക് നല്ല ആളുകളായികഴിയുകയും ചെയ്യാം. എന്ന് അവര്‍ പറഞ്ഞ സന്ദര്‍ഭം (ശ്രദ്ധേയമത്രെ.)
\end{malayalam}}
\flushright{\begin{Arabic}
\quranayah[12][10]
\end{Arabic}}
\flushleft{\begin{malayalam}
അവരില്‍ നിന്ന് ഒരു വക്താവ് പറഞ്ഞു: യൂസുഫിനെ നിങ്ങള്‍ കൊല്ലരുത്‌. നിങ്ങള്‍ക്ക് വല്ലതും ചെയ്യണമെന്നുണ്ടെങ്കില്‍ അവനെ നിങ്ങള്‍ (ഒരു) കിണറ്റിന്‍റെ അടിയിലേക്ക് ഇട്ടേക്കുക. ഏതെങ്കിലും യാത്രാസംഘം അവനെ കണ്ടെടുത്ത് കൊള്ളും.
\end{malayalam}}
\flushright{\begin{Arabic}
\quranayah[12][11]
\end{Arabic}}
\flushleft{\begin{malayalam}
(തുടര്‍ന്ന് പിതാവിന്‍റെ അടുത്ത് ചെന്ന്‌) അവര്‍ പറഞ്ഞു: ഞങ്ങളുടെ പിതാവേ: താങ്കള്‍ക്കെന്തുപറ്റി? യൂസുഫിന്‍റെ കാര്യത്തില്‍ താങ്കള്‍ ഞങ്ങളെ വിശ്വസിക്കുന്നില്ല! ഞങ്ങളാകട്ടെ തീര്‍ച്ചയായും അവന്‍റെ ഗുണകാംക്ഷികളാണ് താനും.
\end{malayalam}}
\flushright{\begin{Arabic}
\quranayah[12][12]
\end{Arabic}}
\flushleft{\begin{malayalam}
നാളെ അവനെ ഞങ്ങളോടൊപ്പം അയച്ചുതരിക. അവന്‍ ഉല്ലസിച്ച് നടന്നുകളിക്കട്ടെ. തീര്‍ച്ചയായും ഞങ്ങള്‍ അവനെ കാത്തുരക്ഷിച്ച് കൊള്ളാം.
\end{malayalam}}
\flushright{\begin{Arabic}
\quranayah[12][13]
\end{Arabic}}
\flushleft{\begin{malayalam}
അദ്ദേഹം പറഞ്ഞു: നിങ്ങള്‍ അവനെ കൊണ്ടുപോകുക എന്നത് തീര്‍ച്ചയായും എനിക്ക് സങ്കടമുണ്ടാക്കുന്നതാണ്‌. നിങ്ങള്‍ അവനെപ്പറ്റി അശ്രദ്ധരായിരിക്കെ അവനെ ചെന്നായ തിന്നേക്കുമെന്ന് ഞാന്‍ ഭയപ്പെടുന്നു.
\end{malayalam}}
\flushright{\begin{Arabic}
\quranayah[12][14]
\end{Arabic}}
\flushleft{\begin{malayalam}
അവര്‍ പറഞ്ഞു: ഞങ്ങള്‍ ഒരു (പ്രബലമായ) സംഘമുണ്ടായിട്ടും അവനെ ചെന്നായ തിന്നുകയാണെങ്കില്‍ തീര്‍ച്ചയായും ഞങ്ങള്‍ മഹാനഷ്ടക്കാര്‍ തന്നെയായിരിക്കും.
\end{malayalam}}
\flushright{\begin{Arabic}
\quranayah[12][15]
\end{Arabic}}
\flushleft{\begin{malayalam}
അങ്ങനെ അവര്‍ അവനെ (യൂസുഫിനെ) യും കൊണ്ടുപോകുകയും, അവനെ കിണറ്റിന്‍റെ അടിയിലേക്ക് ഇടുവാന്‍ അവര്‍ ഒന്നിച്ച് തീരുമാനിക്കുകയും ചെയ്തപ്പോള്‍ (അവര്‍ ആ കടും കൈ പ്രവര്‍ത്തിക്കുക തന്നെ ചെയ്തു.) തീര്‍ച്ചയായും നീ അവര്‍ക്ക് അവരുടെ ഈ ചെയ്തിയെപ്പറ്റി (ഒരിക്കല്‍) വിവരിച്ചുകൊടുക്കുമെന്ന് അവന്ന് (യൂസുഫിന്‌) നാം ബോധനം നല്‍കുകയും ചെയ്തു. (അന്ന്‌) അവര്‍ അതിനെപറ്റി ബോധവാന്‍മാരായിരിക്കുകയില്ല.
\end{malayalam}}
\flushright{\begin{Arabic}
\quranayah[12][16]
\end{Arabic}}
\flushleft{\begin{malayalam}
അവര്‍ സന്ധ്യാസമയത്ത് അവരുടെ പിതാവിന്‍റെ അടുക്കല്‍ കരഞ്ഞുകൊണ്ട് ചെന്നു.
\end{malayalam}}
\flushright{\begin{Arabic}
\quranayah[12][17]
\end{Arabic}}
\flushleft{\begin{malayalam}
അവര്‍ പറഞ്ഞു: ഞങ്ങളുടെ പിതാവേ, ഞങ്ങള്‍ മത്സരിച്ച് ഓടിപ്പോകുകയും, യൂസുഫിനെ ഞങ്ങളുടെ സാധനങ്ങളുടെ അടുത്ത് വിട്ടുപോകുകയും ചെയ്തു. അപ്പോള്‍ അവനെ ചെന്നായ തിന്നുകളഞ്ഞു. ഞങ്ങള്‍ സത്യം പറയുന്നവരാണെങ്കില്‍പോലും താങ്കള്‍ വിശ്വസിക്കുകയില്ലല്ലോ.
\end{malayalam}}
\flushright{\begin{Arabic}
\quranayah[12][18]
\end{Arabic}}
\flushleft{\begin{malayalam}
യൂസുഫിന്‍റെ കുപ്പായത്തില്‍ കള്ളച്ചോരയുമായാണ് അവര്‍ വന്നത്‌. പിതാവ് പറഞ്ഞു: അങ്ങനെയല്ല, നിങ്ങളുടെ മനസ്സ് നിങ്ങള്‍ക്ക് ഒരു കാര്യം ഭംഗിയായി തോന്നിച്ചിരിക്കുകയാണ്‌. അതിനാല്‍ നല്ല ക്ഷമ കൈക്കൊള്ളുക തന്നെ. നിങ്ങളീ പറഞ്ഞുണ്ടാക്കുന്ന കാര്യത്തില്‍ (എനിക്ക്‌) സഹായം തേടാനുള്ളത് അല്ലാഹുവോടത്രെ.
\end{malayalam}}
\flushright{\begin{Arabic}
\quranayah[12][19]
\end{Arabic}}
\flushleft{\begin{malayalam}
ഒരു യാത്രാസംഘം വന്നു. അവര്‍ അവര്‍ക്ക് വെള്ളം കൊണ്ട് വരുന്ന ജോലിക്കാരനെ അയച്ചു. അവന്‍ തന്നെ തൊട്ടിയിറക്കി. അവന്‍ പറഞ്ഞു: ഹാ, സന്തോഷം! ഇതാ ഒരു ബാലന്‍! അവര്‍ ബാലനെ ഒരു കച്ചവടച്ചരക്കായി ഒളിച്ചുവെച്ചു. അവര്‍ പ്രവര്‍ത്തിച്ചിരുന്നതിനെ പറ്റി അല്ലാഹു നല്ലവണ്ണം അറിയുന്നവനാകുന്നു.
\end{malayalam}}
\flushright{\begin{Arabic}
\quranayah[12][20]
\end{Arabic}}
\flushleft{\begin{malayalam}
അവര്‍ അവനെ തുച്ഛമായ ഒരു വിലയ്ക്ക്‌- ഏതാനും വെള്ളിക്കാശിന് - വില്‍ക്കുകയും ചെയ്തു. അവര്‍ അവന്‍റെ കാര്യത്തില്‍ താല്‍പര്യമില്ലാത്തവരുടെ കൂട്ടത്തിലായിരുന്നു.
\end{malayalam}}
\flushright{\begin{Arabic}
\quranayah[12][21]
\end{Arabic}}
\flushleft{\begin{malayalam}
ഈജിപ്തില്‍ നിന്ന് അവനെ (യൂസുഫിനെ) വിലക്കെടുത്ത ആള്‍ തന്‍റെ ഭാര്യയോട് പറഞ്ഞു: ഇവന്ന് മാന്യമായ താമസസൌകര്യം നല്‍കുക. അവന്‍ നമുക്ക് പ്രയോജനപ്പെട്ടേക്കാം. അല്ലെങ്കില്‍ നമുക്കവനെ മകനായി സ്വീകരിക്കാം. അപ്രകാരം യൂസുഫിന് നാം ആ ഭൂപ്രദേശത്ത് സൌകര്യമുണ്ടാക്കികൊടുത്തു. സ്വപ്നവാര്‍ത്തകളുടെ വ്യാഖ്യാനത്തില്‍ നിന്ന് അദ്ദേഹത്തിന് നാം അറിയിച്ച് കൊടുക്കാന്‍ വേണ്ടിയും കൂടിയാണത്‌. അല്ലാഹു തന്‍റെ കാര്യം ജയിച്ചടക്കുന്നവനത്രെ. പക്ഷെ മനുഷ്യരില്‍ അധികപേരും അത് മനസ്സിലാക്കുന്നില്ല.
\end{malayalam}}
\flushright{\begin{Arabic}
\quranayah[12][22]
\end{Arabic}}
\flushleft{\begin{malayalam}
അങ്ങനെ അദ്ദേഹം പൂര്‍ണ്ണവളര്‍ച്ചയെത്തിയപ്പോള്‍ അദ്ദേഹത്തിന് നാം യുക്തിബോധവും അറിവും നല്‍കി. സുകൃതം ചെയ്യുന്നവര്‍ക്ക് അപ്രകാരം നാം പ്രതിഫലം നല്‍കുന്നു.
\end{malayalam}}
\flushright{\begin{Arabic}
\quranayah[12][23]
\end{Arabic}}
\flushleft{\begin{malayalam}
അവന്‍ (യൂസുഫ്‌) ഏതൊരുവളുടെ വീട്ടിലാണോ അവള്‍ അവനെ വശീകരിക്കുവാന്‍ ശ്രമം നടത്തി. വാതിലുകള്‍ അടച്ച് പൂട്ടിയിട്ട് അവള്‍ പറഞ്ഞു: ഇങ്ങോട്ട് വാ. അവന്‍ പറഞ്ഞു. അല്ലാഹുവില്‍ ശരണം! നിശ്ചയമായും അവനാണ് എന്‍റെ രക്ഷിതാവ്‌. അവന്‍ എന്‍റെ താമസം ക്ഷേമകരമാക്കിയിരിക്കുന്നു. തീര്‍ച്ചയായും അക്രമം പ്രവര്‍ത്തിക്കുന്നവര്‍ വിജയിക്കുകയില്ല.
\end{malayalam}}
\flushright{\begin{Arabic}
\quranayah[12][24]
\end{Arabic}}
\flushleft{\begin{malayalam}
അവള്‍ക്ക് അവനില്‍ ആഗ്രഹം ജനിച്ചു. തന്‍റെ രക്ഷിതാവിന്‍റെ പ്രമാണം കണ്ടറിഞ്ഞില്ലായിരുന്നെങ്കില്‍ അവന്ന് അവളിലും ആഗ്രഹം ജനിച്ചേനെ. അപ്രകാരം (സംഭവിച്ചത്‌) തിന്‍മയും നീചവൃത്തിയും അവനില്‍ നിന്ന് നാം തിരിച്ചുവിടുന്നതിന് വേണ്ടിയത്രെ. തീര്‍ച്ചയായും അവന്‍ നമ്മുടെ നിഷ്കളങ്കരായ ദാസന്‍മാരില്‍ പെട്ടവനാകുന്നു.
\end{malayalam}}
\flushright{\begin{Arabic}
\quranayah[12][25]
\end{Arabic}}
\flushleft{\begin{malayalam}
അവര്‍ രണ്ടുപേരും വാതില്‍ക്കലേക്ക് മത്സരിച്ചോടി. അവള്‍ പിന്നില്‍ നിന്ന് അവന്‍റെ കുപ്പായം (പിടിച്ചു. അത്‌) കീറി. അവര്‍ ഇരുവരും വാതില്‍ക്കല്‍ വെച്ച് അവളുടെ നാഥനെ (ഭര്‍ത്താവിനെ) കണ്ടുമുട്ടി. അവള്‍ പറഞ്ഞു: താങ്കളുടെ ഭാര്യയുടെ കാര്യത്തില്‍ ദുരുദ്ദേശം പുലര്‍ത്തിയവനുള്ള പ്രതിഫലം അവന്‍ തടവിലാക്കപ്പെടുക എന്നതോ, വേദനയേറിയ മറ്റെന്തെങ്കിലും ശിക്ഷയോ തന്നെ ആയിരിക്കണം.
\end{malayalam}}
\flushright{\begin{Arabic}
\quranayah[12][26]
\end{Arabic}}
\flushleft{\begin{malayalam}
യൂസുഫ് പറഞ്ഞു: അവളാണ് എന്നെ വശീകരിക്കുവാന്‍ ശ്രമം നടത്തിയത്‌. അവളുടെ കുടുംബത്തില്‍ പെട്ട ഒരു സാക്ഷി ഇപ്രകാരം സാക്ഷ്യപ്പെടുത്തി: അവന്‍റെ കുപ്പായം മുന്നില്‍ നിന്നാണ് കീറിയിട്ടുള്ളതെങ്കില്‍ അവള്‍ സത്യമാണ് പറഞ്ഞത്‌. അവനാകട്ടെ കളവ് പറയുന്നവരുടെ കൂട്ടത്തിലാണ്‌.
\end{malayalam}}
\flushright{\begin{Arabic}
\quranayah[12][27]
\end{Arabic}}
\flushleft{\begin{malayalam}
എന്നാല്‍ അവന്‍റെ കുപ്പായം പിന്നില്‍ നിന്നാണ് കീറിയിട്ടുള്ളതെങ്കില്‍ അവള്‍ കളവാണ് പറഞ്ഞത്‌. അവനാകട്ടെ സത്യം പറഞ്ഞവരുടെ കൂട്ടത്തിലാണ്‌.
\end{malayalam}}
\flushright{\begin{Arabic}
\quranayah[12][28]
\end{Arabic}}
\flushleft{\begin{malayalam}
അങ്ങനെ അവന്‍റെ (യൂസുഫിന്‍റെ) കുപ്പായം പിന്നില്‍ നിന്നാണ് കീറിയിട്ടുള്ളത് എന്ന് കണ്ടപ്പോള്‍ അയാള്‍ (ഗൃഹനാഥന്‍-തന്‍റെ ഭാര്യയോട്‌) പറഞ്ഞു: തീര്‍ച്ചയായും ഇത് നിങ്ങളുടെ (സ്ത്രീകളുടെ) തന്ത്രത്തില്‍ പെട്ടതാണ്‌. നിങ്ങളുടെ തന്ത്രം ഭയങ്കരം തന്നെ.
\end{malayalam}}
\flushright{\begin{Arabic}
\quranayah[12][29]
\end{Arabic}}
\flushleft{\begin{malayalam}
യൂസുഫേ നീ ഇത് അവഗണിച്ചേക്കുക. (പെണ്ണേ,) നീ നിന്‍റെ പാപത്തിന് മാപ്പുതേടുക. തീര്‍ച്ചയായും നീ പിഴച്ചവരുടെ കൂട്ടത്തിലാകുന്നു.
\end{malayalam}}
\flushright{\begin{Arabic}
\quranayah[12][30]
\end{Arabic}}
\flushleft{\begin{malayalam}
നഗരത്തിലെ ചില സ്ത്രീകള്‍ പറഞ്ഞു: പ്രഭുവിന്‍റെ ഭാര്യ തന്‍റെ വേലക്കാരനെ വശീകരിക്കാന്‍ ശ്രമിക്കുന്നു. അവള്‍ അവനോട് പ്രേമബദ്ധയായിക്കഴിഞ്ഞിരിക്കുന്നു. തീര്‍ച്ചയായും അവള്‍ വ്യക്തമായ പിഴവില്‍ അകപ്പെട്ടതായി ഞങ്ങള്‍ കാണുന്നു.
\end{malayalam}}
\flushright{\begin{Arabic}
\quranayah[12][31]
\end{Arabic}}
\flushleft{\begin{malayalam}
അങ്ങനെ ആ സ്ത്രീകളുടെ കുസൃതിയെപ്പറ്റി അവള്‍ കേട്ടറിഞ്ഞപ്പോള്‍ അവരുടെ അടുത്തേക്ക് അവള്‍ ആളെ അയക്കുകയും അവര്‍ക്ക് ചാരിയിരിക്കാവുന്ന ഇരിപ്പിടങ്ങളൊരുക്കുകയും ചെയ്തു. അവരില്‍ ഓരോരുത്തര്‍ക്കും (പഴങ്ങള്‍ മുറിക്കാന്‍) അവള്‍ ഓരോ കത്തി കൊടുത്തു. (യൂസുഫിനോട്‌) അവള്‍ പറഞ്ഞു: നീ അവരുടെ മുമ്പിലേക്ക് പുറപ്പെടുക. അങ്ങനെ അവനെ അവര്‍ കണ്ടപ്പോള്‍ അവര്‍ക്ക് അവനെപ്പറ്റി വിസ്മയം തോന്നുകയും, അവരുടെ സ്വന്തം കൈകള്‍ അവര്‍ തന്നെ അറുത്ത് പോകുകയും ചെയ്തു. അവര്‍ പറഞ്ഞു: അല്ലാഹു എത്ര പരിശുദ്ധന്‍! ഇതൊരു മനുഷ്യനല്ല. ആദരണീയനായ ഒരു മലക്ക് തന്നെയാണ്‌.
\end{malayalam}}
\flushright{\begin{Arabic}
\quranayah[12][32]
\end{Arabic}}
\flushleft{\begin{malayalam}
അവള്‍ പറഞ്ഞു: എന്നാല്‍ ഏതൊരുവന്‍റെ കാര്യത്തില്‍ നിങ്ങളെന്നെ ആക്ഷേപിച്ചുവോ അവനാണിത്‌. തീര്‍ച്ചയായും ഞാന്‍ അവനെ വശീകരിക്കാന്‍ ശ്രമിച്ചിട്ടുണ്ട്‌. അപ്പോള്‍ അവന്‍ (സ്വയം കളങ്കപ്പെടുത്താതെ) കാത്തുസൂക്ഷിക്കുകയാണ് ചെയ്തത്‌. ഞാനവനോട് കല്‍പിക്കുന്ന പ്രകാരം അവന്‍ ചെയ്തില്ലെങ്കില്‍ തീര്‍ച്ചയായും അവന്‍ തടവിലാക്കപ്പെടുകയും, നിന്ദ്യരുടെ കൂട്ടത്തിലായിരിക്കുകയും ചെയ്യും.
\end{malayalam}}
\flushright{\begin{Arabic}
\quranayah[12][33]
\end{Arabic}}
\flushleft{\begin{malayalam}
അവന്‍ (യൂസുഫ്‌) പറഞ്ഞു: എന്‍റെ രക്ഷിതാവേ, ഇവര്‍ എന്നെ ഏതൊന്നിലേക്ക് ക്ഷണിക്കുന്നുവോ അതിനെക്കാളും എനിക്ക് കൂടുതല്‍ പ്രിയപ്പെട്ടത് ജയിലാകുന്നു. ഇവരുടെ കുതന്ത്രം എന്നെ വിട്ട് നീ തിരിച്ചുകളയാത്ത പക്ഷം ഞാന്‍ അവരിലേക്ക് ചാഞ്ഞുപോയേക്കും. അങ്ങനെ ഞാന്‍ അവിവേകികളുടെ കൂട്ടത്തില്‍ ആയിപോകുകയും ചെയ്യും.
\end{malayalam}}
\flushright{\begin{Arabic}
\quranayah[12][34]
\end{Arabic}}
\flushleft{\begin{malayalam}
അപ്പോള്‍ അവന്‍റെ പ്രാര്‍ത്ഥന തന്‍റെ രക്ഷിതാവ് സ്വീകരിക്കുകയും അവരുടെ കുതന്ത്രം അവനില്‍ നിന്ന് അവന്‍ തട്ടിത്തിരിച്ചുകളയുകയും ചെയ്തു. തീര്‍ച്ചയായും അവന്‍ എല്ലാം കേള്‍ക്കുന്നവനും കാണുന്നവനുമത്രെ.
\end{malayalam}}
\flushright{\begin{Arabic}
\quranayah[12][35]
\end{Arabic}}
\flushleft{\begin{malayalam}
പിന്നീട് തെളിവുകള്‍ കണ്ടറിഞ്ഞതിന് ശേഷവും അവര്‍ക്ക് തോന്നി; അവനെ ഒരു അവധിവരെ തടവിലാക്കുക തന്നെ വേണമെന്ന്‌.
\end{malayalam}}
\flushright{\begin{Arabic}
\quranayah[12][36]
\end{Arabic}}
\flushleft{\begin{malayalam}
അവനോടൊപ്പം രണ്ട് യുവാക്കളും ജയിലില്‍ പ്രവേശിച്ചു. അവരില്‍ ഒരാള്‍ പറഞ്ഞു: ഞാന്‍ വീഞ്ഞ് പിഴിഞ്ഞെടുക്കുന്നതായി സ്വപ്നം കാണുന്നു. മറ്റൊരാള്‍ പറഞ്ഞു: ഞാന്‍ എന്‍റെ തലയില്‍ റൊട്ടി ചുമക്കുകയും, എന്നിട്ട് അതില്‍ നിന്ന് പറവകള്‍ തിന്നുകയും ചെയ്യുന്നതായി ഞാന്‍ സ്വപ്നം കാണുന്നു. ഞങ്ങള്‍ക്ക് താങ്കള്‍ അതിന്‍റെ വ്യാഖ്യാനം വിവരിച്ചുതരൂ. തീര്‍ച്ചയായും ഞങ്ങള്‍ താങ്കളെ കാണുന്നത് സദ്‌വൃത്തരില്‍ ഒരാളായിട്ടാണ്‌.
\end{malayalam}}
\flushright{\begin{Arabic}
\quranayah[12][37]
\end{Arabic}}
\flushleft{\begin{malayalam}
അവന്‍ (യൂസുഫ്‌) പറഞ്ഞു: നിങ്ങള്‍ക്ക് (കൊണ്ടുവന്ന്‌) നല്‍കപ്പെടാറുള്ള ഭക്ഷണം നിങ്ങള്‍ക്ക് വന്നെത്തുന്നതിന്‍റെ മുമ്പായി അതിന്‍റെ വ്യാഖ്യാനം ഞാന്‍ നിങ്ങള്‍ക്ക് വിവരിച്ചുതരാതിരിക്കുകയില്ല. എന്‍റെ രക്ഷിതാവ് എനിക്ക് പഠിപ്പിച്ചുതന്നതില്‍ പെട്ടതത്രെ അത്‌. അല്ലാഹുവില്‍ വിശ്വസിക്കാത്തവരും പരലോകത്തെ നിഷേധിക്കുന്നവരുമായിട്ടുള്ളവരുടെ മാര്‍ഗം തീര്‍ച്ചയായും ഞാന്‍ ഉപേക്ഷിച്ചിരിക്കുന്നു.
\end{malayalam}}
\flushright{\begin{Arabic}
\quranayah[12][38]
\end{Arabic}}
\flushleft{\begin{malayalam}
എന്‍റെ പിതാക്കളായ ഇബ്രാഹീം, ഇഷാഖ്‌, യഅ്ഖൂബ് എന്നിവരുടെ മാര്‍ഗം ഞാന്‍ പിന്തുടര്‍ന്നിരിക്കുന്നു. അല്ലാഹുവിനോട് യാതൊന്നിനെയും പങ്കുചേര്‍ക്കുവാന്‍ ഞങ്ങള്‍ക്ക് പാടുള്ളതല്ല. ഞങ്ങള്‍ക്കും മനുഷ്യര്‍ക്കും അല്ലാഹു നല്‍കിയ അനുഗ്രഹത്തില്‍ പെട്ടതത്രെ അത് (സന്‍മാര്‍ഗദര്‍ശനം.) പക്ഷെ മനുഷ്യരില്‍ അധികപേരും നന്ദികാണിക്കുന്നില്ല.
\end{malayalam}}
\flushright{\begin{Arabic}
\quranayah[12][39]
\end{Arabic}}
\flushleft{\begin{malayalam}
ജയിലിലെ രണ്ട് സുഹൃത്തുക്കളേ, വ്യത്യസ്ത രക്ഷാധികാരികളാണോ ഉത്തമം; അതല്ല, ഏകനും സര്‍വ്വാധികാരിയുമായ അല്ലാഹുവാണോ?
\end{malayalam}}
\flushright{\begin{Arabic}
\quranayah[12][40]
\end{Arabic}}
\flushleft{\begin{malayalam}
അവന്നുപുറമെ നിങ്ങള്‍ ആരാധിക്കുന്നവ നിങ്ങളും നിങ്ങളുടെ പിതാക്കളും നാമകരണം ചെയ്തിട്ടുള്ള ചില നാമങ്ങളല്ലാതെ മറ്റൊന്നുമല്ല. അവയെപ്പറ്റി അല്ലാഹു യാതൊരു പ്രമാണവും അവതരിപ്പിച്ചിട്ടില്ല. വിധികര്‍ത്തൃത്വം അല്ലാഹുവിന് മാത്രമാകുന്നു. അവനെയല്ലാതെ നിങ്ങള്‍ ആരാധിക്കരുതെന്ന് അവന്‍ കല്‍പിച്ചിരിക്കുന്നു. വക്രതയില്ലാത്ത മതം അതത്രെ. പക്ഷെ മനുഷ്യരില്‍ അധികപേരും മനസ്സിലാക്കുന്നില്ല.
\end{malayalam}}
\flushright{\begin{Arabic}
\quranayah[12][41]
\end{Arabic}}
\flushleft{\begin{malayalam}
ജയിലിലെ രണ്ട് സുഹൃത്തുക്കളേ, എന്നാല്‍ നിങ്ങളിലൊരുവന്‍ തന്‍റെ യജമാനന്ന് വീഞ്ഞ് കുടിപ്പിച്ച് കൊണ്ടിരിക്കും. എന്നാല്‍ മറ്റേ ആള്‍ ക്രൂശിക്കപ്പെടും. എന്നിട്ട് അയാളുടെ തലയില്‍ നിന്ന് പറവകള്‍ കൊത്തിത്തിന്നും. ഏതൊരു കാര്യത്തെപ്പറ്റി നിങ്ങള്‍ ഇരുവരും വിധി ആരായുന്നുവോ ആ കാര്യം തീരുമാനിക്കപ്പെട്ട് കഴിഞ്ഞിരിക്കുന്നു.
\end{malayalam}}
\flushright{\begin{Arabic}
\quranayah[12][42]
\end{Arabic}}
\flushleft{\begin{malayalam}
അവര്‍ രണ്ട് പേരില്‍ നിന്ന് രക്ഷപ്പെടുന്നവനാണ് എന്ന് വിചാരിച്ച ആളോട് അദ്ദേഹം (യൂസുഫ്‌) പറഞ്ഞു: നിന്‍റെ യജമാനന്‍റെ അടുക്കല്‍ നീ എന്നെ പറ്റി പ്രസ്താവിക്കുക. എന്നാല്‍ തന്‍റെ യജമാനനോട് അത് പ്രസ്താവിക്കുന്ന കാര്യം പിശാച് അവനെ മറപ്പിച്ച് കളഞ്ഞു. അങ്ങനെ ഏതാനും കൊല്ലങ്ങള്‍ അദ്ദേഹം (യൂസുഫ്‌) ജയിലില്‍ താമസിച്ചു.
\end{malayalam}}
\flushright{\begin{Arabic}
\quranayah[12][43]
\end{Arabic}}
\flushleft{\begin{malayalam}
(ഒരിക്കല്‍) രാജാവ് പറഞ്ഞു: തടിച്ചുകൊഴുത്ത ഏഴ് പശുക്കളെ ഏഴ് മെലിഞ്ഞ പശുക്കള്‍ തിന്നുന്നതായി ഞാന്‍ സ്വപ്നം കാണുന്നു. ഏഴ് പച്ചക്കതിരുകളും, ഏഴ് ഉണങ്ങിയ കതിരുകളും ഞാന്‍ കാണുന്നു. ഹേ, പ്രധാനികളേ, നിങ്ങള്‍ സ്വപ്നത്തിന് വ്യാഖ്യാനം നല്‍കുന്നവരാണെങ്കില്‍ എന്‍റെ ഈ സ്വപ്നത്തിന്‍റെ കാര്യത്തില്‍ നിങ്ങളെനിക്ക് വിധി പറഞ്ഞുതരൂ.
\end{malayalam}}
\flushright{\begin{Arabic}
\quranayah[12][44]
\end{Arabic}}
\flushleft{\begin{malayalam}
അവര്‍ പറഞ്ഞു: പലതരം പേക്കിനാവുകള്‍! ഞങ്ങള്‍ അത്തരം പേക്കിനാവുകളുടെ വ്യാഖ്യാനത്തെപ്പറ്റി അറിവുള്ളവരല്ല.
\end{malayalam}}
\flushright{\begin{Arabic}
\quranayah[12][45]
\end{Arabic}}
\flushleft{\begin{malayalam}
ആ രണ്ട് പേരില്‍ (യൂസുഫിന്‍റെ രണ്ട് ജയില്‍ സുഹൃത്തുക്കളില്‍) നിന്ന് രക്ഷപ്പെട്ടവന്‍ ഒരു നീണ്ടകാലയളവിന് ശേഷം (യൂസുഫിന്‍റെ കാര്യം) ഓര്‍മിച്ച് കൊണ്ട് പറഞ്ഞു: അതിന്‍റെ വ്യാഖ്യാനത്തെപ്പറ്റി ഞാന്‍ നിങ്ങള്‍ക്ക് വിവരമറിയിച്ചു തരാം. നിങ്ങള്‍ (അതിന്‌) എന്നെ നിയോഗിച്ചേക്കൂ.
\end{malayalam}}
\flushright{\begin{Arabic}
\quranayah[12][46]
\end{Arabic}}
\flushleft{\begin{malayalam}
(അവന്‍ യൂസുഫിന്‍റെ അടുത്ത് ചെന്ന് പറഞ്ഞു:) ഹേ, സത്യസന്ധനായ യൂസുഫ്‌, തടിച്ച് കൊഴുത്ത ഏഴ് പശുക്കളെ ഏഴ് മെലിഞ്ഞ പശുക്കള്‍ തിന്നുന്ന കാര്യത്തിലും ഏഴ് പച്ചക്കതിരുകളുടെയും വേറെ ഏഴ് ഉണങ്ങിയ കതിരുകളുടെയും കാര്യത്തിലും താങ്കള്‍ ഞങ്ങള്‍ക്കു വിധി പറഞ്ഞുതരണം. ജനങ്ങള്‍ അറിയുവാനായി ആ വിവരവും കൊണ്ട് എനിക്ക് അവരുടെ അടുത്തേക്ക് മടങ്ങാമല്ലോ.
\end{malayalam}}
\flushright{\begin{Arabic}
\quranayah[12][47]
\end{Arabic}}
\flushleft{\begin{malayalam}
അദ്ദേഹം (യൂസുഫ്‌) പറഞ്ഞു: നിങ്ങള്‍ ഏഴുകൊല്ലം തുടര്‍ച്ചയായി കൃഷി ചെയ്യുന്നതാണ്‌. എന്നിട്ട് നിങ്ങള്‍ കൊയ്തെടുത്തതില്‍ നിന്ന് നിങ്ങള്‍ക്ക് ഭക്ഷിക്കുവാന്‍ അല്‍പം ഒഴിച്ച് ബാക്കി അതിന്‍റെ കതിരില്‍ തന്നെ വിട്ടേക്കുക.
\end{malayalam}}
\flushright{\begin{Arabic}
\quranayah[12][48]
\end{Arabic}}
\flushleft{\begin{malayalam}
പിന്നീടതിന് ശേഷം പ്രയാസകരമായ ഏഴ് വര്‍ഷം വരും. ആ വര്‍ഷങ്ങള്‍, അന്നേക്കായി നിങ്ങള്‍ മുന്‍കൂട്ടി സൂക്ഷിച്ച് വെച്ചിട്ടുള്ളതിനെയെല്ലാം തിന്നുതീര്‍ക്കുന്നതാണ്‌. നിങ്ങള്‍ കാത്തുവെക്കുന്നതില്‍ നിന്ന് അല്‍പം ഒഴികെ.
\end{malayalam}}
\flushright{\begin{Arabic}
\quranayah[12][49]
\end{Arabic}}
\flushleft{\begin{malayalam}
പിന്നീടതിന് ശേഷം ഒരു വര്‍ഷം വരും. അന്ന് ജനങ്ങള്‍ക്ക് സമൃദ്ധി നല്‍കപ്പെടുകയും, അന്ന് അവര്‍ (വീഞ്ഞും മറ്റും) പിഴിഞ്ഞെടുക്കുകയും ചെയ്യും.
\end{malayalam}}
\flushright{\begin{Arabic}
\quranayah[12][50]
\end{Arabic}}
\flushleft{\begin{malayalam}
രാജാവ് പറഞ്ഞു: നിങ്ങള്‍ യൂസുഫിനെ എന്‍റെ അടുത്ത് കൊണ്ട് വരൂ. അങ്ങനെ തന്‍റെ അടുത്ത് ദൂതന്‍ വന്നപ്പോള്‍ അദ്ദേഹം (യൂസുഫ്‌) പറഞ്ഞു: നീ നിന്‍റെ യജമാനന്‍റെ അടുത്തേക്ക് തിരിച്ചുപോയിട്ട് സ്വന്തം കൈകള്‍ മുറിപ്പെടുത്തിയ ആ സ്ത്രീകളുടെ നിലപാടെന്താണെന്ന് അദ്ദേഹത്തോട് ചോദിച്ച് നോക്കുക. തീര്‍ച്ചയായും എന്‍റെ രക്ഷിതാവ് അവരുടെ തന്ത്രത്തെപ്പറ്റി നന്നായി അറിയുന്നവനാകുന്നു.
\end{malayalam}}
\flushright{\begin{Arabic}
\quranayah[12][51]
\end{Arabic}}
\flushleft{\begin{malayalam}
(ആ സ്ത്രീകളെ വിളിച്ചുവരുത്തിയിട്ട്‌) അദ്ദേഹം (രാജാവ്‌) ചോദിച്ചു: യൂസുഫിനെ വശീകരിക്കുവാന്‍ നിങ്ങള്‍ ശ്രമം നടത്തിയപ്പോള്‍ നിങ്ങളുടെ സ്ഥിതി എന്തായിരുന്നു? അവര്‍ പറഞ്ഞു: അല്ലാഹു എത്ര പരിശുദ്ധന്‍! ഞങ്ങള്‍ യൂസുഫിനെപ്പറ്റി ദോഷകരമായ ഒന്നും മനസ്സിലാക്കിയിട്ടില്ല. പ്രഭുവിന്‍റെ ഭാര്യ പറഞ്ഞു: ഇപ്പോള്‍ സത്യം വെളിപ്പെട്ടിരിക്കുന്നു. ഞാന്‍ അദ്ദേഹത്തെ വശീകരിക്കാന്‍ ശ്രമിക്കുകയാണുണ്ടായത്‌. തീര്‍ച്ചയായും അദ്ദേഹം സത്യവാന്‍മാരുടെ കൂട്ടത്തില്‍ തന്നെയാകുന്നു.
\end{malayalam}}
\flushright{\begin{Arabic}
\quranayah[12][52]
\end{Arabic}}
\flushleft{\begin{malayalam}
അത് (ഞാനങ്ങനെ പറയുന്നത്‌, അദ്ദേഹത്തിന്‍റെ) അസാന്നിദ്ധ്യത്തില്‍ ഞാന്‍ അദ്ദേഹത്തെ വഞ്ചിച്ചിട്ടില്ലെന്ന് അദ്ദേഹം അറിയുന്നതിന് വേണ്ടിയാകുന്നു. വഞ്ചകന്‍മാരുടെ തന്ത്രത്തെ അല്ലാഹു ലക്ഷ്യത്തിലെത്തിക്കുകയില്ല എന്നതിനാലുമാകുന്നു.
\end{malayalam}}
\flushright{\begin{Arabic}
\quranayah[12][53]
\end{Arabic}}
\flushleft{\begin{malayalam}
ഞാന്‍ എന്‍റെ മനസ്സിനെ കുറ്റത്തില്‍ നിന്ന് ഒഴിവാക്കുന്നില്ല. തീര്‍ച്ചയായും മനസ്സ് ദുഷ്പ്രവൃത്തിക്ക് പ്രേരിപ്പിക്കുന്നത് തന്നെയാകുന്നു. എന്‍റെ രക്ഷിതാവിന്‍റെ കരുണ ലഭിച്ച മനസ്സൊഴികെ. തീര്‍ച്ചയായും എന്‍റെ രക്ഷിതാവ് ഏറെ പൊറുക്കുന്നവനും കരുണാനിധിയുമാകുന്നു.
\end{malayalam}}
\flushright{\begin{Arabic}
\quranayah[12][54]
\end{Arabic}}
\flushleft{\begin{malayalam}
രാജാവ് പറഞ്ഞു: നിങ്ങള്‍ അദ്ദേഹത്തെ എന്‍റെ അടുത്ത് കൊണ്ട് വരൂ. ഞാന്‍ അദ്ദേഹത്തെ എന്‍റെ ഒരു പ്രത്യേകക്കാരനായി സ്വീകരിക്കുന്നതാണ്‌. അങ്ങനെ അദ്ദേഹത്തോട് സംസാരിച്ചപ്പോള്‍ രാജാവ് പറഞ്ഞു: തീര്‍ച്ചയായും താങ്കള്‍ ഇന്ന് നമ്മുടെ അടുക്കല്‍ സ്ഥാനമുള്ളവനും വിശ്വസ്തനുമാകുന്നു.
\end{malayalam}}
\flushright{\begin{Arabic}
\quranayah[12][55]
\end{Arabic}}
\flushleft{\begin{malayalam}
അദ്ദേഹം (യൂസുഫ്‌) പറഞ്ഞു: താങ്കള്‍ എന്നെ ഭൂമിയിലെ ഖജനാവുകളുടെ അധികാരമേല്‍പിക്കൂ. തീര്‍ച്ചയായും ഞാന്‍ വിവരമുള്ള ഒരു സൂക്ഷിപ്പുകാരനായിരിക്കും.
\end{malayalam}}
\flushright{\begin{Arabic}
\quranayah[12][56]
\end{Arabic}}
\flushleft{\begin{malayalam}
അപ്രകാരം യൂസുഫിന് ആ ഭൂപ്രദേശത്ത്‌, അദ്ദേഹം ഉദ്ദേശിക്കുന്നിടത്ത് താമസമുറപ്പിക്കാവുന്ന വിധം നാം സ്വാധീനം നല്‍കി. നമ്മുടെ കാരുണ്യം നാം ഉദ്ദേശിക്കുന്നവര്‍ക്ക് നാം അനുഭവിപ്പിക്കുന്നു. സദ്‌വൃത്തര്‍ക്കുള്ള പ്രതിഫലം നാം നഷ്ടപ്പെടുത്തിക്കളയുകയില്ല.
\end{malayalam}}
\flushright{\begin{Arabic}
\quranayah[12][57]
\end{Arabic}}
\flushleft{\begin{malayalam}
വിശ്വസിക്കുകയും സൂക്ഷ്മത പാലിക്കുന്നവരായിരിക്കുകയും ചെയ്തവര്‍ക്ക് പരലോകത്തെ പ്രതിഫലമാകുന്നു കൂടുതല്‍ ഉത്തമം.
\end{malayalam}}
\flushright{\begin{Arabic}
\quranayah[12][58]
\end{Arabic}}
\flushleft{\begin{malayalam}
യൂസുഫിന്‍റെ സഹോദരന്‍മാര്‍ വന്നു അദ്ദേഹത്തിന്‍റെ അടുത്ത് പ്രവേശിച്ചു. അപ്പോള്‍ അദ്ദേഹം അവരെ തിരിച്ചറിഞ്ഞു. അവര്‍ അദ്ദേഹത്തെ തിരിച്ചറിഞ്ഞിരുന്നില്ല.
\end{malayalam}}
\flushright{\begin{Arabic}
\quranayah[12][59]
\end{Arabic}}
\flushleft{\begin{malayalam}
അങ്ങനെ അവര്‍ക്ക് വേണ്ട സാധനങ്ങള്‍ അവര്‍ക്ക് ഒരുക്കികൊടുത്തപ്പോള്‍ അദ്ദേഹം പറഞ്ഞു: നിങ്ങളുടെ ബാപ്പയൊത്ത ഒരു സഹോദരന്‍ നിങ്ങള്‍ക്കുണ്ടല്ലോ. അവനെ നിങ്ങള്‍ എന്‍റെ അടുത്ത് കൊണ്ട് വരണം. ഞാന്‍ അളവ് തികച്ചുതരുന്നുവെന്നും, ഏറ്റവും നല്ല ആതിഥ്യമാണ് ഞാന്‍ നല്‍കുന്നത് എന്നും നിങ്ങള്‍ കാണുന്നില്ലേ?
\end{malayalam}}
\flushright{\begin{Arabic}
\quranayah[12][60]
\end{Arabic}}
\flushleft{\begin{malayalam}
എന്നാല്‍ അവനെ നിങ്ങള്‍ എന്‍റെ അടുത്ത് കൊണ്ട് വരുന്നില്ലെങ്കില്‍ നിങ്ങള്‍ക്കിനി എന്‍റെ അടുക്കല്‍ നിന്ന് അളന്നുതരുന്നതല്ല. നിങ്ങള്‍ എന്നെ സമീപിക്കേണ്ടതുമില്ല.
\end{malayalam}}
\flushright{\begin{Arabic}
\quranayah[12][61]
\end{Arabic}}
\flushleft{\begin{malayalam}
അവര്‍ പറഞ്ഞു: ഞങ്ങള്‍ അവന്‍റെ കാര്യത്തില്‍ അവന്‍റെ പിതാവിനോട് ഒരു ശ്രമം നടത്തിനോക്കാം. തീര്‍ച്ചയായും ഞങ്ങളത് ചെയ്യും.
\end{malayalam}}
\flushright{\begin{Arabic}
\quranayah[12][62]
\end{Arabic}}
\flushleft{\begin{malayalam}
അദ്ദേഹം (യൂസുഫ്‌) തന്‍റെ ഭൃത്യന്‍മാരോട് പറഞ്ഞു: അവര്‍ കൊണ്ട് വന്ന ചരക്കുകള്‍ അവരുടെ ഭാണ്ഡങ്ങളില്‍ തന്നെ നിങ്ങള്‍ വെച്ചേക്കുക. അവര്‍ അവരുടെ കുടുംബത്തില്‍ തിരിച്ചെത്തുമ്പോള്‍ അവരത് മനസ്സിലാക്കിക്കൊള്ളും. അവര്‍ ഒരുവേള മടങ്ങി വന്നേക്കാം.
\end{malayalam}}
\flushright{\begin{Arabic}
\quranayah[12][63]
\end{Arabic}}
\flushleft{\begin{malayalam}
അങ്ങനെ അവര്‍ തങ്ങളുടെ പിതാവിന്‍റെ അടുത്ത് തിരിച്ചെത്തിയപ്പോള്‍ അവര്‍ പറഞ്ഞു: ഞങ്ങളുടെ പിതാവേ, ഞങ്ങള്‍ക്ക് അളന്നുതരുന്നത് മുടക്കപ്പെട്ടിരിക്കുന്നു. അത് കൊണ്ട് ഞങ്ങളോടൊപ്പം ഞങ്ങളുടെ സഹോദരനെയും കൂടി താങ്കള്‍ അയച്ചുതരണം. എങ്കില്‍ ഞങ്ങള്‍ക്ക് അളന്നുകിട്ടുന്നതാണ്‌. തീര്‍ച്ചയായും ഞങ്ങള്‍ അവനെ കാത്തുസൂക്ഷിക്കുക തന്നെ ചെയ്യും.
\end{malayalam}}
\flushright{\begin{Arabic}
\quranayah[12][64]
\end{Arabic}}
\flushleft{\begin{malayalam}
അദ്ദേഹം (പിതാവ്‌) പറഞ്ഞു: അവന്‍റെ സഹോദരന്‍റെ കാര്യത്തില്‍ മുമ്പ് ഞാന്‍ നിങ്ങളെ വിശ്വസിച്ചത് പോലെയല്ലാതെ അവന്‍റെ കാര്യത്തില്‍ നിങ്ങളെ എനിക്ക് വിശ്വസിക്കാനാകുമോ? എന്നാല്‍ അല്ലാഹുവാണ് നല്ലവണ്ണം കാത്തുസൂക്ഷിക്കുന്നവന്‍. അവന്‍ കരുണയുള്ളവരില്‍ ഏറ്റവും കാരുണികനാകുന്നു.
\end{malayalam}}
\flushright{\begin{Arabic}
\quranayah[12][65]
\end{Arabic}}
\flushleft{\begin{malayalam}
അവര്‍ അവരുടെ സാധനങ്ങള്‍ തുറന്നുനോക്കിയപ്പോള്‍ തങ്ങളുടെ ചരക്കുകള്‍ തങ്ങള്‍ക്ക് തിരിച്ചുനല്‍കപ്പെട്ടതായി അവര്‍ കണ്ടെത്തി. അവര്‍ പറഞ്ഞു: ഞങ്ങളുടെ പിതാവേ, നമുക്കിനി എന്തുവേണം? നമ്മുടെ ചരക്കുകള്‍ ഇതാ നമുക്ക് തന്നെ തിരിച്ചുനല്‍കപ്പെട്ടിരിക്കുന്നു. (മേലിലും) ഞങ്ങള്‍ ഞങ്ങളുടെ കുടുംബത്തിന് ആഹാരം കൊണ്ട് വരാം. ഞങ്ങളുടെ സഹോദരനെ ഞങ്ങള്‍ കാത്തുകൊള്ളുകയും ചെയ്യാം. ഒരു ഒട്ടകത്തിന് വഹിക്കാവുന്ന അളവ് ഞങ്ങള്‍ക്ക് കൂടുതല്‍ കിട്ടുകയും ചെയ്യും. കുറഞ്ഞ ഒരു അളവാകുന്നു അത്‌.
\end{malayalam}}
\flushright{\begin{Arabic}
\quranayah[12][66]
\end{Arabic}}
\flushleft{\begin{malayalam}
അദ്ദേഹം പറഞ്ഞു: തീര്‍ച്ചയായും നിങ്ങള്‍ അവനെ എന്‍റെ അടുക്കല്‍ കൊണ്ട് വന്നുതരുമെന്ന് അല്ലാഹുവിന്‍റെ പേരില്‍ എനിക്ക് ഉറപ്പ് നല്‍കുന്നത് വരെ ഞാനവനെ നിങ്ങളുടെ കുടെ അയക്കുകയില്ല തന്നെ. നിങ്ങള്‍ (ആപത്തുകളാല്‍) വലയം ചെയ്യപ്പെടുന്നുവെങ്കില്‍ ഒഴികെ. അങ്ങനെ അവരുടെ ഉറപ്പ് അദ്ദേഹത്തിന് അവര്‍ നല്‍കിയപ്പോള്‍ അദ്ദേഹം പറഞ്ഞു: അല്ലാഹു നാം പറയുന്നതിന് മേല്‍നോട്ടം വഹിക്കുന്നവനാകുന്നു.
\end{malayalam}}
\flushright{\begin{Arabic}
\quranayah[12][67]
\end{Arabic}}
\flushleft{\begin{malayalam}
അദ്ദേഹം പറഞ്ഞു: എന്‍റെ മക്കളേ, നിങ്ങള്‍ ഒരേ വാതിലിലൂടെ പ്രവേശിക്കാതെ വ്യത്യസ്ത വാതിലുകളിലൂടെ പ്രവേശിക്കുക. അല്ലാഹുവിങ്കല്‍ നിന്നുണ്ടാകുന്ന യാതൊന്നും നിങ്ങളില്‍ നിന്ന് തടുക്കുവാന്‍ എനിക്കാവില്ല. വിധികര്‍ത്തൃത്വം അല്ലാഹുവിന് മാത്രമാകുന്നു. അവന്‍റെ മേല്‍ ഞാന്‍ ഭരമേല്‍പിക്കുന്നു. അവന്‍റെ മേല്‍ തന്നെയാണ് ഭരമേല്‍പിക്കുന്നവര്‍ ഭരമേല്‍പിക്കേണ്ടത്‌.
\end{malayalam}}
\flushright{\begin{Arabic}
\quranayah[12][68]
\end{Arabic}}
\flushleft{\begin{malayalam}
അവരുടെ പിതാവ് അവരോട് കല്‍പിച്ച വിധത്തില്‍ അവര്‍ പ്രവേശിച്ചപ്പോള്‍ അല്ലാഹുവിങ്കല്‍ നിന്നുണ്ടാകുന്ന യാതൊന്നും അവരില്‍ നിന്ന് തടുക്കുവാന്‍ അദ്ദേഹത്തിന് കഴിഞ്ഞിരുന്നില്ല. യഅ്ഖൂബിന്‍റെ മനസ്സിലുണ്ടായിരുന്ന ഒരു ആവശ്യം അദ്ദേഹം നിറവേറ്റി എന്ന് മാത്രം. നാം അദ്ദേഹത്തിന് പഠിപ്പിച്ചുകൊടുത്തിട്ടുള്ളതിനാല്‍ തീര്‍ച്ചയായും അദ്ദേഹം അറിവുള്ളവന്‍ തന്നെയാണ്‌. പക്ഷെ മനുഷ്യരില്‍ അധികപേരും അറിയുന്നില്ല.
\end{malayalam}}
\flushright{\begin{Arabic}
\quranayah[12][69]
\end{Arabic}}
\flushleft{\begin{malayalam}
അവര്‍ യൂസുഫിന്‍റെ അടുത്ത് കടന്ന് ചെന്നപ്പോള്‍ അദ്ദേഹം തന്‍റെ സഹോദരനെ തന്നിലേക്ക് അടുപ്പിച്ചു. എന്നിട്ട് അദ്ദേഹം പറഞ്ഞു: തീര്‍ച്ചയായും ഞാന്‍ തന്നെയാണ് നിന്‍റെ സഹോദരന്‍. ആകയാല്‍ അവര്‍ (മൂത്ത സഹോദരന്‍മാര്‍) ചെയ്ത് വരുന്നതിനെപ്പറ്റി നീ ദുഃഖിക്കേണ്ടതില്ല.
\end{malayalam}}
\flushright{\begin{Arabic}
\quranayah[12][70]
\end{Arabic}}
\flushleft{\begin{malayalam}
അങ്ങനെ അവര്‍ക്കുള്ള സാധനങ്ങള്‍ അവര്‍ക്ക് ഒരുക്കികൊടുത്തപ്പോള്‍ അദ്ദേഹം (യൂസുഫ്‌) പാനപാത്രം തന്‍റെ സഹോദരന്‍റെ ഭാണ്ഡത്തില്‍ വെച്ചു. പിന്നെ ഒരാള്‍ വിളിച്ചുപറഞ്ഞു: ഹേ; യാത്രാസംഘമേ, തീര്‍ച്ചയായും നിങ്ങള്‍ മോഷ്ടാക്കള്‍ തന്നെയാണ്‌.
\end{malayalam}}
\flushright{\begin{Arabic}
\quranayah[12][71]
\end{Arabic}}
\flushleft{\begin{malayalam}
അവരുടെ നേരെ തിരിഞ്ഞ് കൊണ്ട് (യാത്രാസംഘം) പറഞ്ഞു: എന്താണ് നിങ്ങള്‍ക്ക് നഷ്ടപ്പെട്ടിട്ടുള്ളത്‌?
\end{malayalam}}
\flushright{\begin{Arabic}
\quranayah[12][72]
\end{Arabic}}
\flushleft{\begin{malayalam}
അവര്‍ പറഞ്ഞു: ഞങ്ങള്‍ക്ക് രാജാവിന്‍റെ അളവുപാത്രം നഷ്ടപ്പെട്ടിരിക്കുന്നു. അത് കൊണ്ട് വന്ന് തരുന്നവന് ഒരു ഒട്ടകത്തിന് വഹിക്കാവുന്നത് (ധാന്യം) നല്‍കുന്നതാണ്‌. ഞാനത് ഏറ്റിരിക്കുന്നു.
\end{malayalam}}
\flushright{\begin{Arabic}
\quranayah[12][73]
\end{Arabic}}
\flushleft{\begin{malayalam}
അവര്‍ പറഞ്ഞു: അല്ലാഹുവെ തന്നെയാണ,ഞങ്ങള്‍ നാട്ടില്‍ കുഴപ്പമുണ്ടാക്കാന്‍ വേണ്ടി വന്നതല്ലെന്ന് നിങ്ങള്‍ക്കറിയാമല്ലോ. ഞങ്ങള്‍ മോഷ്ടാക്കളായിരുന്നിട്ടുമില്ല.
\end{malayalam}}
\flushright{\begin{Arabic}
\quranayah[12][74]
\end{Arabic}}
\flushleft{\begin{malayalam}
അവര്‍ ചോദിച്ചു: എന്നാല്‍ നിങ്ങള്‍ കള്ളം പറയുന്നവരാണെങ്കില്‍ അതിനു എന്ത് ശിക്ഷയാണ് നല്‍കേണ്ടത് ?
\end{malayalam}}
\flushright{\begin{Arabic}
\quranayah[12][75]
\end{Arabic}}
\flushleft{\begin{malayalam}
അവര്‍ പറഞ്ഞു: അതിനുള്ള ശിക്ഷ ഇപ്രകാരമത്രെ. ഏതൊരുവന്‍റെ യാത്രാ ഭാണ്ഡത്തിലാണോ അതു കാണപ്പെടുന്നത് അവനെ പിടിച്ച് വെക്കുകയാണ് അതിനുള്ള ശിക്ഷ. അപ്രകാരമാണ് ഞങ്ങള്‍ അക്രമികള്‍ക്ക് പ്രതിഫലം നല്‍കുന്നത്‌.
\end{malayalam}}
\flushright{\begin{Arabic}
\quranayah[12][76]
\end{Arabic}}
\flushleft{\begin{malayalam}
എന്നിട്ട് അദ്ദേഹം (യൂസുഫ്‌) തന്‍റെ സഹോദരന്‍റെ ഭാണ്ഡത്തേക്കാള്‍ മുമ്പായി അവരുടെ ഭാണ്ഡങ്ങള്‍ പരിശോധിക്കുവാന്‍ തുടങ്ങി. പിന്നീട് തന്‍റെ സഹോദരന്‍റെ ഭാണ്ഡത്തില്‍ നിന്ന് അദ്ദേഹമത് പുറത്തെടുത്തു. അപ്രകാരം യൂസുഫിന് വേണ്ടി നാം തന്ത്രം പ്രയോഗിച്ചു. അല്ലാഹു ഉദ്ദേശിക്കുന്നുവെങ്കിലല്ലാതെ രാജാവിന്‍റെ നിയമമനുസരിച്ച് അദ്ദേഹത്തിന് തന്‍റെ സഹോദരനെ പിടിച്ചുവെക്കാന്‍ പറ്റുമായിരുന്നില്ല. നാം ഉദ്ദേശിക്കുന്നവരെ നാം പല പദവികള്‍ ഉയര്‍ത്തുന്നു. അറിവുള്ളവരുടെയെല്ലാം മീതെ എല്ലാം അറിയുന്നവനുണ്ട്‌.
\end{malayalam}}
\flushright{\begin{Arabic}
\quranayah[12][77]
\end{Arabic}}
\flushleft{\begin{malayalam}
അവര്‍ (സഹോദരന്‍മാര്‍) പറഞ്ഞു: അവന്‍ മോഷ്ടിക്കുന്നുവെങ്കില്‍ (അതില്‍ അത്ഭുതമില്ല.) മുമ്പ് അവന്‍റെ സഹോദരനും മോഷ്ടിക്കുകയുണ്ടായിട്ടുണ്ട്‌. എന്നാല്‍ യൂസുഫ് അത് തന്‍റെ മനസ്സില്‍ ഗോപ്യമാക്കിവെച്ചു. അവരോട് അദ്ദേഹം അത് (പ്രതികരണം) പ്രകടിപ്പിച്ചില്ല. അദ്ദേഹം (മനസ്സില്‍) പറഞ്ഞു: നിങ്ങളാണ് മോശമായ നിലപാടുകാര്‍. നിങ്ങള്‍ പറഞ്ഞുണ്ടാക്കുന്നതിനെപ്പറ്റി അല്ലാഹു നല്ലവണ്ണം അറിയുന്നവനാണ്‌.
\end{malayalam}}
\flushright{\begin{Arabic}
\quranayah[12][78]
\end{Arabic}}
\flushleft{\begin{malayalam}
അവര്‍ പറഞ്ഞു: പ്രഭോ! ഇവന് വലിയ വൃദ്ധനായ പിതാവുണ്ട്‌. അതിനാല്‍ ഇവന്‍റെ സ്ഥാനത്ത് ഞങ്ങളില്‍ ഒരാളെ പിടിച്ച് വെക്കുക. തീര്‍ച്ചയായും താങ്കളെ ഞങ്ങള്‍ കാണുന്നത് സദ്‌വൃത്തരില്‍പെട്ട ഒരാളായിട്ടാണ്‌.
\end{malayalam}}
\flushright{\begin{Arabic}
\quranayah[12][79]
\end{Arabic}}
\flushleft{\begin{malayalam}
അദ്ദേഹം പറഞ്ഞു: അല്ലാഹുവില്‍ ശരണം. നമ്മുടെ സാധനം ആരുടെ കയ്യില്‍ കണ്ടെത്തിയോ അവനെയല്ലാതെ നാം പിടിച്ച് വെക്കുകയോ? എങ്കില്‍ തീര്‍ച്ചയായും നാം അക്രമകാരികള്‍ തന്നെയായിരിക്കും.
\end{malayalam}}
\flushright{\begin{Arabic}
\quranayah[12][80]
\end{Arabic}}
\flushleft{\begin{malayalam}
അങ്ങനെ അവനെ (സഹോദരനെ) പ്പറ്റി അവര്‍ നിരാശരായി കഴിഞ്ഞപ്പോള്‍ അവര്‍ തനിച്ച് മാറിയിരുന്ന് കൂടിയാലോചന നടത്തി. അവരില്‍ വലിയ ആള്‍ പറഞ്ഞു: നിങ്ങളുടെ പിതാവ് അല്ലാഹുവിന്‍റെ പേരില്‍ നിങ്ങളോട് ഉറപ്പ് വാങ്ങിയിട്ടുണ്ടെന്നും, യൂസുഫിന്‍റെ കാര്യത്തില്‍ മുമ്പ് നിങ്ങള്‍ വീഴ്ചവരുത്തിയിട്ടുണ്ടെന്നും നിങ്ങള്‍ക്കറിഞ്ഞ് കൂടെ? അതിനാല്‍ എന്‍റെ പിതാവ് എനിക്ക് അനുവാദം തരികയോ, അല്ലാഹു എനിക്ക് വിധി തരികയോ ചെയ്യുന്നത് വരെ ഞാന്‍ ഈ ഭൂപ്രദേശം വിട്ടുപോരുകയേ ഇല്ല. വിധികര്‍ത്താക്കളില്‍ ഏറ്റവും ഉത്തമനത്രെ അവന്‍.
\end{malayalam}}
\flushright{\begin{Arabic}
\quranayah[12][81]
\end{Arabic}}
\flushleft{\begin{malayalam}
നിങ്ങള്‍ നിങ്ങളുടെ പിതാവിന്‍റെ അടുത്തേക്ക് മടങ്ങിച്ചെന്നിട്ട് പറയൂ. ഞങ്ങളുടെ പിതാവേ, താങ്കളുടെ മകന്‍ മോഷണം നടത്തിയിരിക്കുന്നു. ഞങ്ങള്‍ അറിഞ്ഞതിന്‍റെ അടിസ്ഥാനത്തില്‍ മാത്രമാണ് ഞങ്ങള്‍ സാക്ഷ്യം വഹിച്ചിട്ടുള്ളത്‌. അദൃശ്യകാര്യം ഞങ്ങള്‍ക്ക് അറിയുമായിരുന്നില്ലല്ലോ.
\end{malayalam}}
\flushright{\begin{Arabic}
\quranayah[12][82]
\end{Arabic}}
\flushleft{\begin{malayalam}
ഞങ്ങള്‍ പോയിരുന്ന രാജ്യക്കാരോടും, ഞങ്ങള്‍ (ഇങ്ങോട്ട്‌) ഒന്നിച്ച് യാത്ര ചെയ്ത യാത്രാസംഘത്തോടും താങ്കള്‍ ചോദിച്ച് നോക്കുക. തീര്‍ച്ചയായും ഞങ്ങള്‍ സത്യം പറയുന്നവരാകുന്നു.
\end{malayalam}}
\flushright{\begin{Arabic}
\quranayah[12][83]
\end{Arabic}}
\flushleft{\begin{malayalam}
അദ്ദേഹം (പിതാവ്‌) പറഞ്ഞു: അല്ല, നിങ്ങളുടെ മനസ്സുകള്‍ നിങ്ങള്‍ക്ക് എന്തോകാര്യം ഭംഗിയായി തോന്നിച്ചിരിക്കുന്നു. അതിനാല്‍ നന്നായി ക്ഷമിക്കുക തന്നെ. അവരെല്ലാവരെയും അല്ലാഹു എന്‍റെ അടുത്ത് കൊണ്ടുവന്നു തന്നേക്കാവുന്നതാണ്‌. തീര്‍ച്ചയായും അവന്‍ എല്ലാം അറിയുന്നവനും യുക്തിമാനുമാകുന്നു.
\end{malayalam}}
\flushright{\begin{Arabic}
\quranayah[12][84]
\end{Arabic}}
\flushleft{\begin{malayalam}
അവരില്‍ നിന്നു തിരിഞ്ഞുകളഞ്ഞിട്ട് അദ്ദേഹം പറഞ്ഞു: യൂസുഫിന്‍റെ കാര്യം എത്ര സങ്കടകരം! ദുഃഖം നിമിത്തം അദ്ദേഹത്തിന്‍റെ ഇരുകണ്ണുകളും വെളുത്ത് പോയി. അങ്ങനെ അദ്ദേഹം (ദുഃഖം) ഉള്ളിലൊതുക്കി കഴിയുകയാണ്‌.
\end{malayalam}}
\flushright{\begin{Arabic}
\quranayah[12][85]
\end{Arabic}}
\flushleft{\begin{malayalam}
അവര്‍ പറഞ്ഞു: അല്ലാഹുവിനെ തന്നെയാണ, താങ്കള്‍ തീര്‍ത്തും അവശനാകുകയോ, അല്ലെങ്കില്‍ മരണമടയുകയോ ചെയ്യുന്നതു വരെ താങ്കള്‍ യൂസുഫിനെ ഓര്‍ത്തു കൊണേ്ടയിരിക്കും.
\end{malayalam}}
\flushright{\begin{Arabic}
\quranayah[12][86]
\end{Arabic}}
\flushleft{\begin{malayalam}
അദ്ദേഹം പറഞ്ഞു: എന്‍റെ വേവലാതിയും വ്യസനവും ഞാന്‍ അല്ലാഹുവോട് മാത്രമാണ് ബോധിപ്പിക്കുന്നത്‌. അല്ലാഹുവിങ്കല്‍ നിന്നും നിങ്ങള്‍ അറിയാത്ത ചിലത് ഞാനറിയുന്നുമുണ്ട്‌.
\end{malayalam}}
\flushright{\begin{Arabic}
\quranayah[12][87]
\end{Arabic}}
\flushleft{\begin{malayalam}
എന്‍റെ മക്കളേ, നിങ്ങള്‍ പോയി യൂസുഫിനെയും അവന്‍റെ സഹോദരനെയും സംബന്ധിച്ച് അന്വേഷിച്ച് നോക്കുക. അല്ലാഹുവിങ്കല്‍ നിന്നുള്ള ആശ്വാസത്തെപ്പറ്റി നിങ്ങള്‍ നിരാശപ്പെടരുത്‌. അവിശ്വാസികളായ ജനങ്ങളല്ലാതെ അല്ലാഹുവിങ്കല്‍ നിന്നുള്ള ആശ്വാസത്തെപ്പറ്റി നിരാശപ്പെടുകയില്ല, തീര്‍ച്ച.
\end{malayalam}}
\flushright{\begin{Arabic}
\quranayah[12][88]
\end{Arabic}}
\flushleft{\begin{malayalam}
അങ്ങനെ യൂസുഫിന്‍റെ അടുക്കല്‍ കടന്ന് ചെന്നിട്ട് അവര്‍ പറഞ്ഞു: പ്രഭോ, ഞങ്ങളെയും ഞങ്ങളുടെ കുടുംബത്തേയും ദുരിതം ബാധിച്ചിരിക്കുന്നു. മോശമായ ചരക്കുകളേ ഞങ്ങള്‍ കൊണ്ടുവന്നിട്ടുള്ളൂ. അതിനാല്‍ താങ്കള്‍ ഞങ്ങള്‍ക്ക് അളവ് തികച്ചുതരികയും, ഞങ്ങളോട് ഔദാര്യം കാണിക്കുകയും ചെയ്യണം. തീര്‍ച്ചയായും അല്ലാഹു ഉദാരമതികള്‍ക്ക് പ്രതിഫലം നല്‍കുന്നതാണ്‌.
\end{malayalam}}
\flushright{\begin{Arabic}
\quranayah[12][89]
\end{Arabic}}
\flushleft{\begin{malayalam}
അദ്ദേഹം പറഞ്ഞു: നിങ്ങള്‍ അറിവില്ലാത്തവരായിരുന്നപ്പോള്‍ യൂസുഫിന്‍റെയും അവന്‍റെ സഹോദരന്‍റെയും കാര്യത്തില്‍ നിങ്ങള്‍ ചെയ്തതെന്താണെന്ന് നിങ്ങള്‍ മനസ്സിലാക്കിയിട്ടുണ്ടോ?
\end{malayalam}}
\flushright{\begin{Arabic}
\quranayah[12][90]
\end{Arabic}}
\flushleft{\begin{malayalam}
അവര്‍ ചോദിച്ചു: നീ തന്നെയാണോ യൂസുഫ്‌? അദ്ദേഹം പറഞ്ഞു: ഞാന്‍ തന്നെയാണ് യൂസുഫ്‌. ഇതെന്‍റെ സഹോദരനും! അല്ലാഹു ഞങ്ങളോട് ഔദാര്യം കാണിച്ചിരിക്കുന്നു. തീര്‍ച്ചയായും ആര്‍ സൂക്ഷ്മത പാലിക്കുകയും ക്ഷമിക്കുകയും ചെയ്യുന്നുവോ ആ സദ്‌വൃത്തര്‍ക്കുള്ള പ്രതിഫലം അല്ലാഹു നഷ്ടപ്പെടുത്തിക്കളയുകയില്ല; തീര്‍ച്ച.
\end{malayalam}}
\flushright{\begin{Arabic}
\quranayah[12][91]
\end{Arabic}}
\flushleft{\begin{malayalam}
അവര്‍ പറഞ്ഞു: അല്ലാഹുവെതന്നെയാണ, തീര്‍ച്ചയായും അല്ലാഹു നിനക്ക് ഞങ്ങളെക്കാള്‍ മുന്‍ഗണന നല്‍കിയിരിക്കുന്നു. തീര്‍ച്ചയായും ഞങ്ങള്‍ തെറ്റുകാരായിരിക്കുന്നു.
\end{malayalam}}
\flushright{\begin{Arabic}
\quranayah[12][92]
\end{Arabic}}
\flushleft{\begin{malayalam}
അദ്ദേഹം പറഞ്ഞു: ഇന്ന് നിങ്ങളുടെ മേല്‍ ഒരു ആക്ഷേപവുമില്ല. അല്ലാഹു നിങ്ങള്‍ക്ക് പൊറുത്തുതരട്ടെ. അവന്‍ കരുണയുള്ളവരില്‍ വെച്ച് ഏറ്റവും കാരുണികനാകുന്നു.
\end{malayalam}}
\flushright{\begin{Arabic}
\quranayah[12][93]
\end{Arabic}}
\flushleft{\begin{malayalam}
നിങ്ങള്‍ എന്‍റെ ഈ കുപ്പായം കൊണ്ട് പോയിട്ട് എന്‍റെ പിതാവിന്‍റെ മുഖത്ത് ഇട്ടുകൊടുക്കുക. എങ്കില്‍ അദ്ദേഹം കാഴ്ചയുള്ളവനായിത്തീരും. നിങ്ങളുടെ മുഴുവന്‍ കുടുംബാംഗങ്ങളെയും കൊണ്ട് നിങ്ങള്‍ എന്‍റെ അടുത്ത് വരുകയും ചെയ്യുക.
\end{malayalam}}
\flushright{\begin{Arabic}
\quranayah[12][94]
\end{Arabic}}
\flushleft{\begin{malayalam}
യാത്രാസംഘം (ഈജിപ്തില്‍ നിന്ന്‌) പുറപ്പെട്ടപ്പോള്‍ അവരുടെ പിതാവ് (അടുത്തുള്ളവരോട്‌) പറഞ്ഞു: തീര്‍ച്ചയായും എനിക്ക് യൂസുഫിന്‍റെ വാസന അനുഭവപ്പെടുന്നുണ്ട്‌. നിങ്ങളെന്നെ ബുദ്ധിഭ്രമം പറ്റിയവനായി കരുതുന്നില്ലെങ്കില്‍ (നിങ്ങള്‍ക്കിത് വിശ്വസിക്കാവുന്നതാണ്‌.)
\end{malayalam}}
\flushright{\begin{Arabic}
\quranayah[12][95]
\end{Arabic}}
\flushleft{\begin{malayalam}
അവര്‍ പറഞ്ഞു: അല്ലാഹുവെ തന്നെയാണ, തീര്‍ച്ചയായും താങ്കള്‍ താങ്കളുടെ പഴയ വഴികേടില്‍ തന്നെയാണ്‌.
\end{malayalam}}
\flushright{\begin{Arabic}
\quranayah[12][96]
\end{Arabic}}
\flushleft{\begin{malayalam}
അനന്തരം സന്തോഷവാര്‍ത്ത അറിയിക്കുന്ന ആള്‍ വന്നപ്പോള്‍ അയാള്‍ ആ കുപ്പായം അദ്ദേഹത്തിന്‍റെ മുഖത്ത് വെച്ച് കൊടുത്തു. അപ്പോള്‍ അദ്ദേഹം കാഴ്ചയുള്ളവനായി മാറി. അദ്ദേഹം പറഞ്ഞു: നിങ്ങള്‍ക്കറിഞ്ഞുകൂടാത്ത ചിലത് അല്ലാഹുവിങ്കല്‍ നിന്ന് ഞാന്‍ അറിയുന്നുണ്ട് എന്ന് ഞാന്‍ നിങ്ങളോട് പറഞ്ഞിട്ടില്ലേ.
\end{malayalam}}
\flushright{\begin{Arabic}
\quranayah[12][97]
\end{Arabic}}
\flushleft{\begin{malayalam}
അവര്‍ പറഞ്ഞു: ഞങ്ങളുടെ പിതാവേ, ഞങ്ങള്‍ക്കു വേണ്ടി ഞങ്ങളുടെ പാപങ്ങള്‍ പൊറുത്തുകിട്ടാന്‍ താങ്കള്‍ പ്രാര്‍ത്ഥിക്കണേ-തീര്‍ച്ചയായും ഞങ്ങള്‍ തെറ്റുകാരായിരിക്കുന്നു.
\end{malayalam}}
\flushright{\begin{Arabic}
\quranayah[12][98]
\end{Arabic}}
\flushleft{\begin{malayalam}
അദ്ദേഹം പറഞ്ഞു: നിങ്ങള്‍ക്ക് വേണ്ടി എന്‍റെ രക്ഷിതാവിനോട് ഞാന്‍ പാപമോചനം തേടാം. തീര്‍ച്ചയായും അവന്‍ ഏറെ പൊറുക്കുന്നവനും കരുണാനിധിയുമാകുന്നു.
\end{malayalam}}
\flushright{\begin{Arabic}
\quranayah[12][99]
\end{Arabic}}
\flushleft{\begin{malayalam}
അനന്തരം അവര്‍ യൂസുഫിന്‍റെ മുമ്പാകെ പ്രവേശിച്ചപ്പോള്‍ അദ്ദേഹം (യൂസുഫ്‌) തന്‍റെ മാതാപിതാക്കളെ തന്നിലേക്ക് അണച്ചു കൂട്ടി. അദ്ദേഹം പറഞ്ഞു: അല്ലാഹു ഉദ്ദേശിക്കുന്ന പക്ഷം നിങ്ങള്‍ നിര്‍ഭയരായിക്കൊണ്ട് ഈജിപ്തില്‍ പ്രവേശിച്ചു കൊള്ളുക.
\end{malayalam}}
\flushright{\begin{Arabic}
\quranayah[12][100]
\end{Arabic}}
\flushleft{\begin{malayalam}
അദ്ദേഹം തന്‍റെ മാതാപിതാക്കളെ രാജപീഠത്തിന്‍മേല്‍ കയറ്റിയിരുത്തി. അവര്‍ അദ്ദേഹത്തിന്‍റെ മുമ്പില്‍ പ്രണാമം ചെയ്യുന്നവരായിക്കൊണ്ട് വീണു. അദ്ദേഹം പറഞ്ഞു: എന്‍റെ പിതാവേ, മുമ്പ് ഞാന്‍ കണ്ട സ്വപ്നം പുലര്‍ന്നതാണിത്‌. എന്‍റെ രക്ഷിതാവ് അതൊരു യാഥാര്‍ത്ഥ്യമാക്കിത്തീര്‍ത്തിരിക്കുന്നു. എന്നെ അവന്‍ ജയിലില്‍ നിന്ന് പുറത്തുകൊണ്ട് വന്ന സന്ദര്‍ഭത്തിലും എന്‍റെയും എന്‍റെ സഹോദരങ്ങളുടെയും ഇടയില്‍ പിശാച് കുഴപ്പം ഇളക്കിവിട്ടതിന് ശേഷം മരുഭൂമിയില്‍ നിന്ന് അവന്‍ നിങ്ങളെയെല്ലാവരെയും (എന്‍റെ അടുത്തേക്ക്‌) കൊണ്ടുവന്ന സന്ദര്‍ഭത്തിലും അവന്‍ എനിക്ക് ഉപകാരം ചെയ്തിരിക്കുന്നു. തീര്‍ച്ചയായും എന്‍റെ രക്ഷിതാവ് താന്‍ ഉദ്ദേശിക്കുന്ന കാര്യങ്ങള്‍ സൂക്ഷ്മമായി നിയന്ത്രിക്കുന്നവനത്രെ. തീര്‍ച്ചയായും അവന്‍ എല്ലാം അറിയുന്നവനും യുക്തിമാനുമാകുന്നു.
\end{malayalam}}
\flushright{\begin{Arabic}
\quranayah[12][101]
\end{Arabic}}
\flushleft{\begin{malayalam}
(യൂസുഫ് പ്രാര്‍ത്ഥിച്ചു:) എന്‍റെ രക്ഷിതാവേ, നീ എനിക്ക് ഭരണാധികാരത്തില്‍ നിന്ന് (ഒരംശം) നല്‍കുകയും, സ്വപ്നവാര്‍ത്തകളുടെ വ്യാഖ്യാനത്തില്‍ നിന്നും (ചിലത്‌) നീ എനിക്ക് പഠിപ്പിച്ചുതരികയും ചെയ്തിരിക്കുന്നു. ആകാശങ്ങളുടെയും ഭൂമിയുടെയും സ്രഷ്ടാവേ, നീ ഇഹത്തിലും പരത്തിലും എന്‍റെ രക്ഷാധികാരിയാകുന്നു. നീ എന്നെ മുസ്ലിമായി മരിപ്പിക്കുകയും സജ്ജനങ്ങളുടെ കൂട്ടത്തില്‍ ചേര്‍ക്കുകയും ചെയ്യേണമേ.
\end{malayalam}}
\flushright{\begin{Arabic}
\quranayah[12][102]
\end{Arabic}}
\flushleft{\begin{malayalam}
(നബിയേ,) നിനക്ക് നാം സന്ദേശമായി നല്‍കുന്ന അദൃശ്യവാര്‍ത്തകളില്‍ പെട്ടതത്രെ അത്‌. (യൂസുഫിനെതിരില്‍) തന്ത്രം പ്രയോഗിച്ചുകൊണ്ട് അവര്‍ തങ്ങളുടെ പദ്ധതി കൂടിത്തീരുമാനിച്ചപ്പോള്‍ നീ അവരുടെ അടുക്കല്‍ ഉണ്ടായിരുന്നില്ലല്ലോ.
\end{malayalam}}
\flushright{\begin{Arabic}
\quranayah[12][103]
\end{Arabic}}
\flushleft{\begin{malayalam}
എന്നാല്‍ നീ അതിയായി ആഗ്രഹിച്ചാലും മനുഷ്യരില്‍ അധികപേരും വിശ്വസിക്കുന്നവരല്ല.
\end{malayalam}}
\flushright{\begin{Arabic}
\quranayah[12][104]
\end{Arabic}}
\flushleft{\begin{malayalam}
നീ അവരോട് ഇതിന്‍റെ പേരില്‍ യാതൊരു പ്രതിഫലവും ചോദിക്കുന്നുമില്ല. ഇത് ലോകര്‍ക്ക് വേണ്ടിയുള്ള ഒരു ഉല്‍ബോധനം മാത്രമാകുന്നു.
\end{malayalam}}
\flushright{\begin{Arabic}
\quranayah[12][105]
\end{Arabic}}
\flushleft{\begin{malayalam}
ആകാശങ്ങളിലും ഭൂമിയിലും എത്രയെത്ര ദൃഷ്ടാന്തങ്ങള്‍! അവയെ അവഗണിച്ചുകൊണ്ട് അവര്‍ അവയുടെ അടുത്ത് കൂടി കടന്ന് പോകുന്നു.
\end{malayalam}}
\flushright{\begin{Arabic}
\quranayah[12][106]
\end{Arabic}}
\flushleft{\begin{malayalam}
അവരില്‍ അധികപേരും അല്ലാഹുവില്‍ വിശ്വസിക്കുന്നത് അവനോട് (മറ്റുള്ളവരെ) പങ്കുചേര്‍ക്കുന്നവരായിക്കൊണ്ട് മാത്രമാണ്‌.
\end{malayalam}}
\flushright{\begin{Arabic}
\quranayah[12][107]
\end{Arabic}}
\flushleft{\begin{malayalam}
അവരെ വലയം ചെയ്യുന്ന തരത്തിലുള്ള അല്ലാഹുവിന്‍റെ ശിക്ഷ അവര്‍ക്ക് വന്നെത്തുന്നതിനെപ്പറ്റി, അല്ലെങ്കില്‍ അവര്‍ ഓര്‍ക്കാതിരിക്കെ പെട്ടെന്ന് അന്ത്യദിനം അവര്‍ക്ക് വന്നെത്തുന്നതിനെപ്പറ്റി അവര്‍ നിര്‍ഭയരായിരിക്കുകയാണോ?
\end{malayalam}}
\flushright{\begin{Arabic}
\quranayah[12][108]
\end{Arabic}}
\flushleft{\begin{malayalam}
(നബിയേ,) പറയുക: ഇതാണ് എന്‍റെ മാര്‍ഗം. ദൃഢബോധ്യത്തോട് കൂടി അല്ലാഹുവിലേക്ക് ഞാന്‍ ക്ഷണിക്കുന്നു. ഞാനും എന്നെ പിന്‍പറ്റിയവരും. അല്ലാഹു എത്ര പരിശുദ്ധന്‍! ഞാന്‍ (അവനോട്‌) പങ്കുചേര്‍ക്കുന്ന കൂട്ടത്തിലല്ല തന്നെ.
\end{malayalam}}
\flushright{\begin{Arabic}
\quranayah[12][109]
\end{Arabic}}
\flushleft{\begin{malayalam}
വിവിധ രാജ്യക്കാരില്‍ നിന്ന് നാം സന്ദേശം നല്‍കിക്കൊണ്ടിരുന്ന ചില പുരുഷന്‍മാരെത്തന്നെയാണ് നിനക്ക് മുമ്പും നാം ദൂതന്‍മാരായി നിയോഗിച്ചിട്ടുള്ളത് അവര്‍ (അവിശ്വാസികള്‍) ഭൂമിയിലൂടെ സഞ്ചരിച്ചിട്ട് തങ്ങളുടെ മുന്‍ഗാമികളുടെ പര്യവസാനം എങ്ങനെയായിരുന്നുവെന്ന് നോക്കിയിട്ടില്ലേ? എന്നാല്‍ പരലോകമാണ് സൂക്ഷ്മത പാലിച്ചവര്‍ക്ക് കൂടുതല്‍ ഉത്തമമായിട്ടുള്ളത്‌. അപ്പോള്‍ നിങ്ങള്‍ ഗ്രഹിക്കുന്നില്ലേ?
\end{malayalam}}
\flushright{\begin{Arabic}
\quranayah[12][110]
\end{Arabic}}
\flushleft{\begin{malayalam}
അങ്ങനെ ദൈവദൂതന്‍മാര്‍ നിരാശപ്പെടുകയും (അവര്‍) തങ്ങളോട് പറഞ്ഞത് കളവാണെന്ന് ജനങ്ങള്‍ വിചാരിക്കുകയും ചെയ്തപ്പോള്‍ നമ്മുടെ സഹായം അവര്‍ക്ക് (ദൂതന്‍മാര്‍ക്ക്‌) വന്നെത്തി. അങ്ങനെ നാം ഉദ്ദേശിച്ചിരുന്നവര്‍ക്ക് രക്ഷനല്‍കപ്പെട്ടു. കുറ്റവാളികളായ ജനങ്ങളില്‍ നിന്നും നമ്മുടെ ശിക്ഷ തടുക്കപ്പെടുന്നതല്ല.
\end{malayalam}}
\flushright{\begin{Arabic}
\quranayah[12][111]
\end{Arabic}}
\flushleft{\begin{malayalam}
തീര്‍ച്ചയായും അവരുടെ ചരിത്രത്തില്‍ ബുദ്ധിമാന്‍മാര്‍ക്ക് പാഠമുണ്ട്‌. അത് കെട്ടിയുണ്ടാക്കാവുന്ന ഒരു വര്‍ത്തമാനമല്ല. പ്രത്യുത; അതിന്‍റെ മുമ്പുള്ളതിനെ (വേദങ്ങളെ) ശരിവെക്കുന്നതും, എല്ലാകാര്യത്തെയും സംബന്ധിച്ചുള്ള ഒരു വിശദീകരണവും വിശ്വസിക്കുന്ന ജനങ്ങള്‍ക്ക് മാര്‍ഗദര്‍ശനവും കാരുണ്യവുമാകുന്നു അത്‌.
\end{malayalam}}
\chapter{\textmalayalam{‍റഅദ് ( ഇടിനാദം )}}
\begin{Arabic}
\Huge{\centerline{\basmalah}}\end{Arabic}
\flushright{\begin{Arabic}
\quranayah[13][1]
\end{Arabic}}
\flushleft{\begin{malayalam}
അലിഫ് ലാം മീം റാ. വേദഗ്രന്ഥത്തിലെ വചനങ്ങളത്രെ അവ. നിന്‍റെ രക്ഷിതാവിങ്കല്‍ നിന്ന് നിനക്ക് അവതരിപ്പിക്കപ്പെട്ടിട്ടുള്ളത് സത്യമാകുന്നു. പക്ഷെ, ജനങ്ങളിലധികപേരും വിശ്വസിക്കുന്നില്ല.
\end{malayalam}}
\flushright{\begin{Arabic}
\quranayah[13][2]
\end{Arabic}}
\flushleft{\begin{malayalam}
അല്ലാഹുവാകുന്നു നിങ്ങള്‍ക്ക് കാണാവുന്ന അവലംബങ്ങള്‍ കൂടാതെ ആകാശങ്ങള്‍ ഉയര്‍ത്തി നിര്‍ത്തിയവന്‍.പിന്നെ അവന്‍ സിംഹാസനസ്ഥനാകുകയും, സൂര്യനെയും ചന്ദ്രനെയും കീഴ്പെടുത്തുകയും ചെയ്തിരിക്കുന്നു. എല്ലാം ഒരു നിശ്ചിത അവധി വരെ സഞ്ചരിക്കുന്നു. അവന്‍ കാര്യം നിയന്ത്രിച്ചു കൊണ്ടിരിക്കുന്നു. നിങ്ങളുടെ രക്ഷിതാവുമായി കണ്ടുമുട്ടുന്നതിനെപ്പറ്റി നിങ്ങള്‍ ദൃഢബോധ്യമുള്ളവരായിരിക്കുന്നതിന് വേണ്ടി അവന്‍ ദൃഷ്ടാന്തങ്ങള്‍ വിവരിച്ചുതരുന്നു.
\end{malayalam}}
\flushright{\begin{Arabic}
\quranayah[13][3]
\end{Arabic}}
\flushleft{\begin{malayalam}
അവനാണ് ഭൂമിയെ വിശാലമാക്കുകയും, അതില്‍ ഉറച്ചുനില്‍ക്കുന്ന പര്‍വ്വതങ്ങളും നദികളും ഉണ്ടാക്കുകയും ചെയ്തവന്‍. എല്ലാ ഫലവര്‍ഗങ്ങളില്‍ നിന്നും അവനതില്‍ ഈ രണ്ട് ഇണകളെ ഉണ്ടാക്കിയിരിക്കുന്നു. അവന്‍ രാത്രിയെക്കൊണ്ട് പകലിനെ മൂടുന്നു. തീര്‍ച്ചയായും അതില്‍ ചിന്തിക്കുന്ന ജനങ്ങള്‍ക്ക് ദൃഷ്ടാന്തങ്ങളുണ്ട്‌.
\end{malayalam}}
\flushright{\begin{Arabic}
\quranayah[13][4]
\end{Arabic}}
\flushleft{\begin{malayalam}
ഭൂമിയില്‍ തൊട്ടുതൊട്ടു കിടക്കുന്ന ഖണ്ഡങ്ങളുണ്ട്‌. മുന്തിരിത്തോട്ടങ്ങളും കൃഷികളും, ഒരു മുരട്ടില്‍ നിന്ന് പല ശാഖങ്ങളായി വളരുന്നതും, വേറെ വേറെ മുരടുകളില്‍ നിന്ന് വളരുന്നതുമായ ഈന്തപ്പനകളും ഉണ്ട്‌. ഒരേ വെള്ളം കൊണ്ടാണ് അത് നനയ്ക്കപ്പെടുന്നത്‌. ഫലങ്ങളുടെ കാര്യത്തില്‍ അവയില്‍ ചിലതിനെ മറ്റു ചിലതിനെക്കാള്‍ നാം മെച്ചപ്പെടുത്തുന്നു. തീര്‍ച്ചയായും അതില്‍ ചിന്തിക്കുന്ന ജനങ്ങള്‍ക്ക് ദൃഷ്ടാന്തങ്ങളുണ്ട്‌.
\end{malayalam}}
\flushright{\begin{Arabic}
\quranayah[13][5]
\end{Arabic}}
\flushleft{\begin{malayalam}
നീ അത്ഭുതപ്പെടുന്നുവെങ്കില്‍ അവരുടെ ഈ വാക്കത്രെ അത്ഭുതകരമായിട്ടുള്ളത്‌. ഞങ്ങള്‍ മണ്ണായിക്കഴിഞ്ഞിട്ടോ? ഞങ്ങള്‍ പുതുതായി സൃഷ്ടിക്കപ്പെടുക തന്നെ ചെയ്യുമോ? അക്കൂട്ടരാണ് തങ്ങളുടെ രക്ഷിതാവില്‍ അവിശ്വസിച്ചവര്‍. അക്കൂട്ടരാണ് കഴുത്തുകളില്‍ വിലങ്ങുകളുള്ളവര്‍. അക്കുട്ടരാണ് നരകാവകാശികള്‍. അവരതില്‍ നിത്യവാസികളായിരിക്കും.
\end{malayalam}}
\flushright{\begin{Arabic}
\quranayah[13][6]
\end{Arabic}}
\flushleft{\begin{malayalam}
(നബിയേ,) നിന്നോട് അവര്‍ നന്‍മയേക്കാള്‍ മുമ്പായി തിന്‍മയ്ക്ക് (ശിക്ഷയ്ക്ക്‌) വേണ്ടി തിടുക്കം കൂട്ടിക്കൊണ്ടിരിക്കുന്നു. അവരുടെ മുമ്പ് മാതൃകാപരമായ ശിക്ഷകള്‍ കഴിഞ്ഞുപോയിട്ടുണ്ട് താനും. തീര്‍ച്ചയായും, നിന്‍റെ രക്ഷിതാവ് മനുഷ്യര്‍ അക്രമം പ്രവര്‍ത്തിച്ചിട്ടുകൂടി അവര്‍ക്ക് പാപമോചനം നല്‍കുന്നവനത്രെ, തീര്‍ച്ചയായും നിന്‍റെ രക്ഷിതാവ് കഠിനമായി ശിക്ഷിക്കുന്നവനുമാണ്‌.
\end{malayalam}}
\flushright{\begin{Arabic}
\quranayah[13][7]
\end{Arabic}}
\flushleft{\begin{malayalam}
(നബിയെ പരിഹസിച്ചുകൊണ്ട്‌) സത്യനിഷേധികള്‍ പറയുന്നു: ഇവന്‍റെ രക്ഷിതാവിങ്കല്‍ നിന്ന് ഇവന്‍റെ മേല്‍ എന്താണ് ഒരു ദൃഷ്ടാന്തം ഇറക്കപ്പെടാത്തത്‌? (നബിയേ,) നീ ഒരു മുന്നറിയിപ്പുകാരന്‍ മാത്രമാകുന്നു. എല്ലാ ജനവിഭാഗത്തിനുമുണ്ട് ഒരു മാര്‍ഗദര്‍ശി.
\end{malayalam}}
\flushright{\begin{Arabic}
\quranayah[13][8]
\end{Arabic}}
\flushleft{\begin{malayalam}
ഓരോ സ്ത്രീയും ഗര്‍ഭം ധരിക്കുന്നതെന്തെന്ന് അല്ലാഹു അറിയുന്നു. ഗര്‍ഭാശയങ്ങള്‍ കമ്മിവരുത്തുന്നതും വര്‍ദ്ധനവുണ്ടാക്കുന്നതും അവനറിയുന്നു. ഏതൊരുകാര്യവും അവന്‍റെ അടുക്കല്‍ ഒരു നിശ്ചിത തോതനുസരിച്ചാകുന്നു.
\end{malayalam}}
\flushright{\begin{Arabic}
\quranayah[13][9]
\end{Arabic}}
\flushleft{\begin{malayalam}
അദൃശ്യത്തേയും ദൃശ്യത്തേയും അറിയുന്നവനും മഹാനും ഉന്നതനുമാകുന്നു അവന്‍.
\end{malayalam}}
\flushright{\begin{Arabic}
\quranayah[13][10]
\end{Arabic}}
\flushleft{\begin{malayalam}
നിങ്ങളുടെ കൂട്ടത്തില്‍ നിന്ന് രഹസ്യമായി സംസാരിച്ചവനും പരസ്യമായി സംസാരിച്ചവനും രാത്രിയില്‍ ഒളിഞ്ഞിരിക്കുന്നവനും പകലില്‍ പുറത്തിറങ്ങി നടക്കുന്നവനുമെല്ലാം (അവനെ സംബന്ധിച്ചിടത്തോളം) സമമാകുന്നു.
\end{malayalam}}
\flushright{\begin{Arabic}
\quranayah[13][11]
\end{Arabic}}
\flushleft{\begin{malayalam}
മനുഷ്യന്ന് അവന്‍റെ മുമ്പിലൂടെയും പിന്നിലൂടെയും തുടരെത്തുടരെ വന്ന് കൊണ്ട് അല്ലാഹുവിന്‍റെ കല്‍പനപ്രകാരം അവനെ കാത്തുസൂക്ഷിച്ച് കൊണ്ടിരിക്കുന്നവര്‍ (മലക്കുകള്‍) ഉണ്ട്‌. ഏതൊരു ജനതയും തങ്ങളുടെ സ്വന്തം നിലപാടുകളില്‍ മാറ്റം വരുത്തുന്നത് വരെ അല്ലാഹു അവരുടെ സ്ഥിതിക്ക് വ്യത്യാസം വരുത്തുകയില്ല; തീര്‍ച്ച. ഒരു ജനതയ്ക്ക് വല്ല ദോഷവും വരുത്താന്‍ അല്ലാഹു ഉദ്ദേശിച്ചാല്‍ അത് തട്ടിമാറ്റാനാവില്ല. അവന്നു പുറമെ അവര്‍ക്ക് യാതൊരു രക്ഷാധികാരിയുമില്ല.
\end{malayalam}}
\flushright{\begin{Arabic}
\quranayah[13][12]
\end{Arabic}}
\flushleft{\begin{malayalam}
ഭയവും ആശയും ജനിപ്പിച്ച് കൊണ്ട് നിങ്ങള്‍ക്ക് മിന്നല്‍പിണര്‍ കാണിച്ചുതരുന്നത് അവനത്രെ. (ജല) ഭാരമുള്ള മേഘങ്ങളെ അവന്‍ ഉണ്ടാക്കുകയും ചെയ്യുന്നു.
\end{malayalam}}
\flushright{\begin{Arabic}
\quranayah[13][13]
\end{Arabic}}
\flushleft{\begin{malayalam}
ഇടിനാദം അവനെ സ്തുതിക്കുന്നതോടൊപ്പം (അവനെ) പ്രകീര്‍ത്തിക്കുന്നു. അവനെപ്പറ്റിയുള്ള ഭയത്താല്‍ മലക്കുകളും (അവനെ പ്രകീര്‍ത്തിക്കുന്നു.) അവന്‍ ഇടിവാളുകള്‍ അയക്കുകയും, താന്‍ ഉദ്ദേശിക്കുന്നവര്‍ക്ക് അവ ഏല്‍പിക്കുകയും ചെയ്യുന്നു. അവര്‍(അവിശ്വാസികള്‍) അല്ലാഹുവിന്‍റെ കാര്യത്തില്‍ തര്‍ക്കിച്ച് കൊണ്ടിരിക്കുന്നു. അതിശക്തമായി തന്ത്രം പ്രയോഗിക്കുന്നവനത്രെ അവന്‍.
\end{malayalam}}
\flushright{\begin{Arabic}
\quranayah[13][14]
\end{Arabic}}
\flushleft{\begin{malayalam}
അവനോടുള്ളതുമാത്രമാണ് ന്യായമായ പ്രാര്‍ത്ഥന. അവന്നു പുറമെ ആരോടെല്ലാം അവര്‍ പ്രാര്‍ത്ഥിച്ച് കൊണ്ടിരിക്കുന്നുവോ അവരാരും അവര്‍ക്ക് യാതൊരു ഉത്തരവും നല്‍കുന്നതല്ല. വെള്ളം തന്‍റെ വായില്‍ (തനിയെ) വന്നെത്താന്‍ വേണ്ടി തന്‍റെ ഇരുകൈകളും അതിന്‍റെ നേരെ നീട്ടിക്കാണിക്കുന്നവനെപ്പോലെ മാത്രമാകുന്നു അവര്‍. അത് (വെള്ളം) വായില്‍ വന്നെത്തുകയില്ലല്ലോ. സത്യനിഷേധികളുടെ പ്രാര്‍ത്ഥന നഷ്ടത്തില്‍ തന്നെയാകുന്നു.
\end{malayalam}}
\flushright{\begin{Arabic}
\quranayah[13][15]
\end{Arabic}}
\flushleft{\begin{malayalam}
അല്ലാഹുവിന്നാണ് ആകാശങ്ങളിലും ഭൂമിയിലും ഉള്ളവരെല്ലാം പ്രണാമം ചെയ്തുകൊണ്ടിരിക്കുന്നത്‌. സ്വമനസ്സോടെയും നിര്‍ബന്ധിതരായിട്ടും. പ്രഭാതങ്ങളിലും സായാഹ്നങ്ങളിലും അവരുടെ നിഴലുകളും (അവന്ന് പ്രണാമം ചെയ്യുന്നു.)
\end{malayalam}}
\flushright{\begin{Arabic}
\quranayah[13][16]
\end{Arabic}}
\flushleft{\begin{malayalam}
(നബിയേ,) ചോദിക്കുക: ആരാണ് ആകാശങ്ങളുടെയും ഭൂമിയുടെയും രക്ഷിതാവ്‌? പറയുക: അല്ലാഹുവാണ്‌. പറയുക: എന്നിട്ടും അവന്നു പുറമെ അവരവര്‍ക്കു തന്നെ ഉപകാരമോ ഉപദ്രവമോ ചെയ്യാന്‍ സ്വാധീനമില്ലാത്ത ചില രക്ഷാധികാരികളെ നിങ്ങള്‍ സ്വീകരിച്ചിരിക്കുകയാണോ? പറയുക: അന്ധനും കാഴ്ചയുള്ളവനും തുല്യരാകുമോ? അഥവാ ഇരുട്ടുകളും വെളിച്ചവും തുല്യമാകുമോ? അതല്ല, അല്ലാഹുവിന് പുറമെ അവര്‍ പങ്കാളികളാക്കി വെച്ചവര്‍, അവന്‍ സൃഷ്ടിക്കുന്നത് പോലെത്തന്നെ സൃഷ്ടി നടത്തിയിട്ട് (ഇരു വിഭാഗത്തിന്‍റെയും) സൃഷ്ടികള്‍ അവര്‍ക്ക് തിരിച്ചറിയാതാവുകയാണോ ഉണ്ടായത്‌? പറയുക: അല്ലാഹുവത്രെ എല്ലാ വസ്തുക്കളുടെയും സ്രഷ്ടാവ്‌. അവന്‍ ഏകനും സര്‍വ്വാധിപതിയുമാകുന്നു.
\end{malayalam}}
\flushright{\begin{Arabic}
\quranayah[13][17]
\end{Arabic}}
\flushleft{\begin{malayalam}
അവന്‍ (അല്ലാഹു) ആകാശത്ത് നിന്ന് വെള്ളം ചൊരിഞ്ഞു. എന്നിട്ട് താഴ്‌വരകളിലൂടെ അവയുടെ (വലുപ്പത്തിന്‍റെ) തോത് അനുസരിച്ച് വെള്ളമൊഴുകി. അപ്പോള്‍ ആ ഒഴുക്ക് പൊങ്ങി നില്‍ക്കുന്ന നുരയെ വഹിച്ചുകൊണ്ടാണ് വന്നത്‌. വല്ല ആഭരണമോ ഉപകരണമോ ഉണ്ടാക്കാന്‍ ആഗ്രഹിച്ച് കൊണ്ട് അവര്‍ തീയിലിട്ടു കത്തിക്കുന്ന ലോഹത്തില്‍ നിന്നും അത് പോലുള്ള നുരയുണ്ടാകുന്നു. അതു പോലെയാകുന്നു അല്ലാഹു സത്യത്തെയും അസത്യത്തെയും ഉപമിക്കുന്നത്‌. എന്നാല്‍ ആ നുര ചവറായി പോകുന്നു. മനുഷ്യര്‍ക്ക് ഉപകാരമുള്ളതാകട്ടെ ഭൂമിയില്‍ തങ്ങിനില്‍ക്കുന്നു. അപ്രകാരം അല്ലാഹു ഉപമകള്‍ വിവരിക്കുന്നു.
\end{malayalam}}
\flushright{\begin{Arabic}
\quranayah[13][18]
\end{Arabic}}
\flushleft{\begin{malayalam}
തങ്ങളുടെ രക്ഷിതാവിന്‍റെ ആഹ്വാനം സ്വീകരിച്ചവര്‍ക്കാണ് ഏറ്റവും ഉത്തമമായ പ്രതിഫലമുള്ളത്‌. അവന്‍റെ ആഹ്വാനം സ്വീകരിക്കാത്തവരാകട്ടെ ഭൂമിയിലുള്ളത് മുഴുവനും, അതോടൊപ്പം അത്രയും കൂടിയും അവര്‍ക്ക് ഉണ്ടായിരുന്നാല്‍ പോലും (തങ്ങളുടെ രക്ഷയ്ക്കു വേണ്ടി) അതൊക്കെയും അവര്‍ പ്രായശ്ചിത്തമായി നല്‍കുമായിരുന്നു. അവര്‍ക്കാണ് കടുത്ത വിചാരണയുള്ളത്‌. അവരുടെ സങ്കേതം നരകമത്രെ. ആ വാസസ്ഥലം എത്ര മോശം!
\end{malayalam}}
\flushright{\begin{Arabic}
\quranayah[13][19]
\end{Arabic}}
\flushleft{\begin{malayalam}
അപ്പോള്‍ നിനക്ക് നിന്‍റെ രക്ഷിതാവിങ്കല്‍ നിന്ന് അവതരിപ്പിക്കപ്പെട്ടിട്ടുള്ളത് സത്യമാണെന്ന് മനസ്സിലാക്കുന്ന ഒരാള്‍ അന്ധനായിക്കഴിയുന്ന ഒരാളെപ്പോലെയാണോ? ബുദ്ധിമാന്‍മാര്‍ മാത്രമേ ചിന്തിച്ച് മനസ്സിലാക്കുകയുള്ളൂ.
\end{malayalam}}
\flushright{\begin{Arabic}
\quranayah[13][20]
\end{Arabic}}
\flushleft{\begin{malayalam}
അല്ലാഹുവോടുള്ള ബാധ്യത നിറവേറ്റുകയും കരാര്‍ ലംഘിക്കാതിരിക്കുകയും ചെയ്യുന്നവരത്രെ അവര്‍.
\end{malayalam}}
\flushright{\begin{Arabic}
\quranayah[13][21]
\end{Arabic}}
\flushleft{\begin{malayalam}
കൂട്ടിയിണക്കപ്പെടാന്‍ അല്ലാഹു കല്‍പിച്ചത് (ബന്ധങ്ങള്‍) കൂട്ടിയിണക്കുകയും, തങ്ങളുടെ രക്ഷിതാവിനെ പേടിക്കുകയും കടുത്ത വിചാരണയെ ഭയപ്പെടുകയും ചെയ്യുന്നവര്‍.
\end{malayalam}}
\flushright{\begin{Arabic}
\quranayah[13][22]
\end{Arabic}}
\flushleft{\begin{malayalam}
തങ്ങളുടെ രക്ഷിതാവിന്‍റെ പ്രീതി ആഗ്രഹിച്ച് കൊണ്ട് ക്ഷമ കൈക്കൊള്ളുകയും, നമസ്കാരം മുറപോലെ നിര്‍വഹിക്കുകയും, നാം നല്‍കിയിട്ടുള്ളതില്‍ നിന്ന് രഹസ്യമായും പരസ്യമായും ചെലവഴിക്കുകയും, തിന്‍മയെ നന്‍മ കൊണ്ട് തടുക്കുകയും ചെയ്യുന്നവര്‍. അത്തരക്കാര്‍ക്ക് അനുകൂലമത്രെ ലോകത്തിന്‍റെ പര്യവസാനം.
\end{malayalam}}
\flushright{\begin{Arabic}
\quranayah[13][23]
\end{Arabic}}
\flushleft{\begin{malayalam}
അതായത്‌, സ്ഥിരവാസത്തിനുള്ള സ്വര്‍ഗത്തോപ്പുകള്‍. അവരും, അവരുടെ പിതാക്കളില്‍ നിന്നും, ഇണകളില്‍ നിന്നും സന്തതികളില്‍ നിന്നും സദ്‌വൃത്തരായിട്ടുള്ളവരും അതില്‍ പ്രവേശിക്കുന്നതാണ്‌. മലക്കുകള്‍ എല്ലാ വാതിലിലൂടെയും അവരുടെ അടുക്കല്‍ കടന്നുവന്നിട്ട് പറയും:
\end{malayalam}}
\flushright{\begin{Arabic}
\quranayah[13][24]
\end{Arabic}}
\flushleft{\begin{malayalam}
നിങ്ങള്‍ ക്ഷമ കൈക്കൊണ്ടതിനാല്‍ നിങ്ങള്‍ക്ക് സമാധാനം! അപ്പോള്‍ ലോകത്തിന്‍റെ പര്യവസാനം എത്ര നല്ലത്‌!
\end{malayalam}}
\flushright{\begin{Arabic}
\quranayah[13][25]
\end{Arabic}}
\flushleft{\begin{malayalam}
അല്ലാഹുവോടുള്ള ബാധ്യത ഉറപ്പിച്ചതിന് ശേഷം ലംഘിക്കുകയും, കൂട്ടിയിണക്കപ്പെടാന്‍ അല്ലാഹു കല്‍പിച്ചതിനെ (ബന്ധങ്ങളെ) അറുത്ത് കളയുകയും, ഭൂമിയില്‍ കുഴപ്പമുണ്ടാക്കുകയും ചെയ്യുന്നവരാരോ അവര്‍ക്കാണ് ശാപം. അവര്‍ക്കാണ് ചീത്ത ഭവനം.
\end{malayalam}}
\flushright{\begin{Arabic}
\quranayah[13][26]
\end{Arabic}}
\flushleft{\begin{malayalam}
അല്ലാഹു അവന്‍ ഉദ്ദേശിക്കുന്ന ചിലര്‍ക്ക് ഉപജീവനം വിശാലമാക്കുകയും (മറ്റു ചിലര്‍ക്ക് അത്‌) പരിമിതപ്പെടുത്തുകയും ചെയ്യുന്നു. അവര്‍ ഇഹലോകജീവിതത്തില്‍ സന്തോഷമടഞ്ഞിരിക്കുന്നു. പരലോകത്തെ അപേക്ഷിച്ച് ഇഹലോകജീവിതം (നിസ്സാരമായ) ഒരു സുഖാനുഭവം മാത്രമാകുന്നു.
\end{malayalam}}
\flushright{\begin{Arabic}
\quranayah[13][27]
\end{Arabic}}
\flushleft{\begin{malayalam}
അവിശ്വസിച്ചവര്‍ (നബിയെപറ്റി) പറയുന്നു: ഇവന്‍റെ മേല്‍ എന്തുകൊണ്ടാണ് ഇവന്‍റെ രക്ഷിതാവിങ്കല്‍ നിന്ന് വല്ല ദൃഷ്ടാന്തവും ഇറക്കപ്പെടാത്തത്‌? (നബിയേ,) പറയുക: തീര്‍ച്ചയായും അല്ലാഹു താന്‍ ഉദ്ദേശിക്കുന്നവരെ വഴികേടിലാക്കുന്നു. പശ്ചാത്തപിച്ച് മടങ്ങിയവരെ തന്‍റെ മാര്‍ഗത്തിലേക്ക് അവന്‍ നയിക്കുകയും ചെയ്യുന്നു.
\end{malayalam}}
\flushright{\begin{Arabic}
\quranayah[13][28]
\end{Arabic}}
\flushleft{\begin{malayalam}
അതായത് വിശ്വസിക്കുകയും അല്ലാഹുവെ പറ്റിയുള്ള ഓര്‍മ കൊണ്ട് മനസ്സുകള്‍ ശാന്തമായിത്തീരുകയും ചെയ്യുന്നവരെ. ശ്രദ്ധിക്കുക; അല്ലാഹുവെപ്പറ്റിയുള്ള ഓര്‍മ കൊണ്ടത്രെ മനസ്സുകള്‍ ശാന്തമായിത്തീരുന്നത്‌.
\end{malayalam}}
\flushright{\begin{Arabic}
\quranayah[13][29]
\end{Arabic}}
\flushleft{\begin{malayalam}
വിശ്വസിക്കുകയും, സല്‍കര്‍മ്മങ്ങള്‍ പ്രവര്‍ത്തിക്കുകയും ചെയ്തവരാരോ അവര്‍ക്കാണ് മംഗളം! മടങ്ങിച്ചെല്ലാനുള്ള നല്ല സങ്കേതവും (അവര്‍ക്കു തന്നെ.)
\end{malayalam}}
\flushright{\begin{Arabic}
\quranayah[13][30]
\end{Arabic}}
\flushleft{\begin{malayalam}
അപ്രകാരം നിന്നെ നാം ഒരു സമുദായത്തില്‍ ദൂതനായി നിയോഗിച്ചിരിക്കുന്നു. അതിന്നു മുമ്പ് പല സമുദായങ്ങളും കഴിഞ്ഞുപോയിട്ടുണ്ട്‌. നിനക്ക് നാം ദിവ്യസന്ദേശമായി നല്‍കിയിട്ടുള്ളത് അവര്‍ക്ക് ഓതികേള്‍പിക്കുവാന്‍ വേണ്ടിയാണ് (നിന്നെ നിയോഗിച്ചത്‌.) അവരാകട്ടെ, പരമകാരുണികനായ ദൈവത്തില്‍ അവിശ്വസിക്കുന്നു. പറയുക: അവനാണ് എന്‍റെ രക്ഷിതാവ്‌. അവനല്ലാതെ യാതൊരു ദൈവവുമില്ല. അവന്‍റെ മേലാണ് ഞാന്‍ ഭരമേല്‍പിച്ചിരിക്കുന്നത്‌. അവനിലേക്കാണ് എന്‍റെ മടക്കം.
\end{malayalam}}
\flushright{\begin{Arabic}
\quranayah[13][31]
\end{Arabic}}
\flushleft{\begin{malayalam}
പാരായണം ചെയ്യപ്പെടുന്ന ഒരു ഗ്രന്ഥം മൂലം പര്‍വ്വതങ്ങള്‍ നടത്തപ്പെടുകയോ, അല്ലെങ്കില്‍ അതു കാരണമായി ഭൂമി തുണ്ടംതുണ്ടമായി മുറിക്കപ്പെടുകയോ, അല്ലെങ്കില്‍ അതുമുഖേന മരിച്ചവരോട് സംസാരിക്കപ്പെടുകയോ ചെയ്തിരുന്നെങ്കില്‍ പോലും (അവര്‍ വിശ്വസിക്കുമായിരുന്നില്ല.) എന്നാല്‍ കാര്യം മുഴുവന്‍ അല്ലാഹുവിന്‍റെ നിയന്ത്രണത്തിലത്രെ. അപ്പോള്‍ അല്ലാഹു ഉദ്ദേശിച്ചിരുന്നുവെങ്കില്‍ മനുഷ്യരെ മുഴുവന്‍ അവന്‍ നേര്‍വഴിയിലാക്കുമായിരുന്നുവെന്ന് സത്യവിശ്വാസികള്‍ മനസ്സിലാക്കിയിട്ടില്ലേ? സത്യനിഷേധികള്‍ക്ക് തങ്ങള്‍ പ്രവര്‍ത്തിച്ചതിന്‍റെ ഫലമായി ഏതെങ്കിലും അത്യാപത്ത് ബാധിച്ച് കൊണേ്ടയിരിക്കുന്നതാണ്‌. അല്ലെങ്കില്‍ അവരുടെ താമസസ്ഥലത്തിനടുത്തു തന്നെ അത് (ശിക്ഷ) വന്നിറങ്ങിക്കൊണ്ടിരിക്കും; അല്ലാഹുവിന്‍റെ വാഗ്ദത്തം വന്നെത്തുന്നത് വരെ. അല്ലാഹു വാഗ്ദാനം ലംഘിക്കുകയില്ല; തീര്‍ച്ച.
\end{malayalam}}
\flushright{\begin{Arabic}
\quranayah[13][32]
\end{Arabic}}
\flushleft{\begin{malayalam}
തീര്‍ച്ചയായും നിനക്കു മുമ്പും ദൂതന്‍മാര്‍ പരിഹസിക്കപ്പെട്ടിട്ടുണ്ട്‌. അപ്പോള്‍ അവിശ്വസിച്ചവര്‍ക്ക് ഞാന്‍ സമയം നീട്ടികൊടുക്കുകയും, പിന്നീട് അവരെ ഞാന്‍ പിടികൂടുകയും ചെയ്തു. അപ്പോള്‍ എന്‍റെ ശിക്ഷ എങ്ങനെയായിരുന്നു!
\end{malayalam}}
\flushright{\begin{Arabic}
\quranayah[13][33]
\end{Arabic}}
\flushleft{\begin{malayalam}
പ്പോള്‍ ഓരോ വ്യക്തിയും പ്രവര്‍ത്തിച്ചുകൊണ്ടിരുന്ന കാര്യത്തിനു മേല്‍നോട്ടം വഹിച്ചുകൊണ്ടിരിക്കുന്നവന്‍ (അല്ലാഹു) (യാതൊന്നും അറിയാത്തവരെപ്പോലെയാണോ?) അവര്‍ അല്ലാഹുവിന് പങ്കാളികളെ ആക്കിയിരിക്കുന്നു. (നബിയേ,) പറയുക: നിങ്ങള്‍ അവരുടെ പേരൊന്നു പറഞ്ഞുതരൂ, അതല്ല, ഭൂമിയില്‍ അവന്‍ (അല്ലാഹു) അറിയാത്ത ഒരു കാര്യത്തെപ്പറ്റി നിങ്ങള്‍ അവന്ന് പറഞ്ഞറിയിച്ച് കൊടുക്കുകയാണോ? അതല്ല, (നിങ്ങള്‍ പറയുന്നത്‌) ഉപരിപ്ലവമായ ഒരു സംസാരമാണോ ? അല്ല, സത്യനിഷേധികള്‍ക്ക് അവരുടെ കുതന്ത്രം അലംകൃതമായി തോന്നിക്കപ്പെട്ടിരിക്കുന്നു. (ശരിയായ) മാര്‍ഗത്തില്‍ നിന്ന് അവര്‍ തട്ടിത്തിരിക്കപ്പെടുകയും ചെയ്തിരിക്കുന്നു. അല്ലാഹു വല്ലവനെയും ദുര്‍മാര്‍ഗത്തിലാക്കുന്ന പക്ഷം അവനെ നേര്‍വഴിയിലാക്കാന്‍ ആരുമില്ല.
\end{malayalam}}
\flushright{\begin{Arabic}
\quranayah[13][34]
\end{Arabic}}
\flushleft{\begin{malayalam}
അവര്‍ക്ക് ഇഹലോകജീവിതത്തില്‍ ശിക്ഷയുണ്ടായിരിക്കും. പരലോകശിക്ഷയാകട്ടെ ഏറ്റവും വിഷമമേറിയതു തന്നെയായിരിക്കും. അവര്‍ക്ക് അല്ലാഹുവിന്‍റെ ശിക്ഷയില്‍ നിന്ന് (തങ്ങളെ) കാത്തുരക്ഷിക്കാന്‍ ആരുമില്ല.
\end{malayalam}}
\flushright{\begin{Arabic}
\quranayah[13][35]
\end{Arabic}}
\flushleft{\begin{malayalam}
സൂക്ഷ്മത പാലിക്കുന്നവര്‍ക്ക് വാഗ്ദാനം ചെയ്യപ്പെട്ടിട്ടുള്ള സ്വര്‍ഗത്തിന്‍റെ അവസ്ഥ (ഇതത്രെ:) അതിന്‍റെ താഴ്ഭാഗത്ത് കൂടി അരുവികള്‍ ഒഴുകിക്കൊണ്ടിരിക്കും. അതിലെ കനികളും അതിലെ തണലും ശാശ്വതമായിരിക്കും. അതത്രെ സൂക്ഷ്മത പാലിച്ചവരുടെ പര്യവസാനം. സത്യനിഷേധികളുടെ പര്യവസാനം നരകമാകുന്നു.
\end{malayalam}}
\flushright{\begin{Arabic}
\quranayah[13][36]
\end{Arabic}}
\flushleft{\begin{malayalam}
നാം (മുമ്പ്‌) വേദഗ്രന്ഥം നല്‍കിയിട്ടുള്ളതാര്‍ക്കാണോ അവര്‍ നിനക്ക് അവതരിപ്പിക്കപ്പെട്ടതില്‍ (ഖുര്‍ആനില്‍) സന്തോഷം കൊള്ളുന്നു. ആ കക്ഷികളുടെ കൂട്ടത്തില്‍ തന്നെ അതിന്‍റെ ചില ഭാഗം നിഷേധിക്കുന്നവരുമുണ്ട്‌. പറയുക: അല്ലാഹുവെ ഞാന്‍ ആരാധിക്കണമെന്നും, അവനോട് ഞാന്‍ പങ്കുചേര്‍ക്കരുത് എന്നും മാത്രമാണ് ഞാന്‍ കല്‍പിക്കപ്പെട്ടിട്ടുള്ളത്‌. അവനിലേക്കാണ് ഞാന്‍ ക്ഷണിക്കുന്നത്‌. അവനിലേക്ക് തന്നെയാണ് എന്‍റെ മടക്കവും.
\end{malayalam}}
\flushright{\begin{Arabic}
\quranayah[13][37]
\end{Arabic}}
\flushleft{\begin{malayalam}
അപ്രകാരം ഇതിനെ (ഖുര്‍ആനിനെ) അറബിഭാഷയിലുള്ള ഒരു ന്യായപ്രമാണമായി നാം അവതരിപ്പിച്ചിരിക്കുന്നു. നിനക്ക് അറിവ് വന്നുകിട്ടിയതിന് ശേഷം അവരുടെ തന്നിഷ്ടങ്ങളെ നീ പിന്‍പറ്റിയാല്‍ അല്ലാഹുവിന്‍റെ ശിക്ഷയില്‍ നിന്ന് രക്ഷിക്കുന്ന യാതൊരു രക്ഷാധികാരിയും, യാതൊരു കാവല്‍ക്കാരനും നിനക്ക് ഉണ്ടായിരിക്കുകയില്ല.
\end{malayalam}}
\flushright{\begin{Arabic}
\quranayah[13][38]
\end{Arabic}}
\flushleft{\begin{malayalam}
നിനക്ക് മുമ്പും നാം ദൂതന്‍മാരെ നിയോഗിച്ചിട്ടുണ്ട്‌. അവര്‍ക്ക് നാം ഭാര്യമാരെയും സന്താനങ്ങളെയും നല്‍കിയിട്ടുണ്ട്‌. ഒരു ദൂതന്നും അല്ലാഹുവിന്‍റെ അനുമതിയോട് കൂടിയല്ലാതെ യാതൊരു ദൃഷ്ടാന്തവും കൊണ്ടുവരാനാവില്ല. ഓരോ കാലാവധിക്കും ഓരോ (പ്രമാണ) ഗ്രന്ഥമുണ്ട്‌.
\end{malayalam}}
\flushright{\begin{Arabic}
\quranayah[13][39]
\end{Arabic}}
\flushleft{\begin{malayalam}
ല്ലാഹു താന്‍ ഉദ്ദേശിക്കുന്നത് മായ്ച്ചുകളയുകയും (താന്‍ ഉദ്ദേശിക്കുന്നത്‌) സ്ഥിരപ്പെടുത്തുകയും ചെയ്യുന്നു. മൂലഗ്രന്ഥം അവന്‍റെ പക്കലുള്ളതാണ്‌.
\end{malayalam}}
\flushright{\begin{Arabic}
\quranayah[13][40]
\end{Arabic}}
\flushleft{\begin{malayalam}
നാം അവര്‍ക്ക് മുന്നറിയിപ്പ് നല്‍കുന്നതില്‍ (ശിക്ഷാനടപടികളില്‍) ചിലത് നിനക്ക് നാം കാണിച്ചുതരികയോ, അല്ലെങ്കില്‍ (അതിനു മുമ്പ്‌) നിന്‍റെ ജീവിതം നാം അവസാനിപ്പിക്കുകയോ ചെയ്യുന്ന പക്ഷം (ഇത് രണ്ടില്‍ ഏതാണ് സംഭവിക്കുന്നതെങ്കിലും) നിന്‍റെ മേല്‍ പ്രബോധന ബാധ്യത മാത്രമേയുള്ളൂ. (അവരുടെ കണക്കു) നോക്കുന്ന ബാധ്യത നമുക്കാകുന്നു.
\end{malayalam}}
\flushright{\begin{Arabic}
\quranayah[13][41]
\end{Arabic}}
\flushleft{\begin{malayalam}
നാം (അവരുടെ) ഭൂമിയില്‍ ചെന്ന് അതിന്‍റെ നാനാവശങ്ങളില്‍ നിന്ന് അതിനെ ചുരുക്കിക്കൊണ്ടിരിക്കുന്നത് അവര്‍ കണ്ടില്ലേ ? അല്ലാഹു വിധിക്കുന്നു. അവന്‍റെ വിധി ഭേദഗതി ചെയ്യാന്‍ ആരും തന്നെയില്ല. അവന്‍ അതിവേഗത്തില്‍ കണക്ക് നോക്കുന്നവനത്രെ.
\end{malayalam}}
\flushright{\begin{Arabic}
\quranayah[13][42]
\end{Arabic}}
\flushleft{\begin{malayalam}
ഇവരുടെ മുമ്പുള്ളവരും തന്ത്രം പ്രയോഗിച്ചിട്ടുണ്ട്‌. എന്നാല്‍ മുഴുവന്‍ തന്ത്രവും അല്ലാഹുവിന്നാണുള്ളത്‌. ഓരോ വ്യക്തിയും പ്രവര്‍ത്തിച്ച് കൊണ്ടിരിക്കുന്നത് അവര്‍ അറിയുന്നു. ലോകത്തിന്‍റെ പര്യവസാനം ആര്‍ക്ക് അനുകൂലമാണെന്ന് സത്യനിഷേധികള്‍ അറിഞ്ഞ് കൊള്ളും.
\end{malayalam}}
\flushright{\begin{Arabic}
\quranayah[13][43]
\end{Arabic}}
\flushleft{\begin{malayalam}
നീ (ദൈവത്താല്‍) നിയോഗിക്കപ്പെട്ടവനല്ലെന്ന് സത്യനിഷേധികള്‍ പറയുന്നു. പറയുക: എനിക്കും നിങ്ങള്‍ക്കുമിടയില്‍ സാക്ഷിയായി അല്ലാഹു മതി. ആരുടെ പക്കലാണോ വേദവിജ്ഞാനമുള്ളത് അവരും മതി.
\end{malayalam}}
\chapter{\textmalayalam{ഇബ്രാഹീം}}
\begin{Arabic}
\Huge{\centerline{\basmalah}}\end{Arabic}
\flushright{\begin{Arabic}
\quranayah[14][1]
\end{Arabic}}
\flushleft{\begin{malayalam}
അലിഫ് ലാം റാ മനുഷ്യരെ അവന്‍റെ രക്ഷിതാവിന്‍റെ അനുമതി പ്രകാരം ഇരുട്ടുകളില്‍ നിന്ന് വെളിച്ചത്തിലേക്ക് കൊണ്ടുവരുവാന്‍ വേണ്ടി നിനക്ക് അവതരിപ്പിച്ചു തന്നിട്ടുള്ള ഗ്രന്ഥമാണിത്‌. അതായത്‌, പ്രതാപിയും സ്തുത്യര്‍ഹനും ആയിട്ടുള്ളവന്‍റെ മാര്‍ഗത്തിലേക്ക്‌.
\end{malayalam}}
\flushright{\begin{Arabic}
\quranayah[14][2]
\end{Arabic}}
\flushleft{\begin{malayalam}
ആകാശങ്ങളിലുള്ളതിന്‍റെയും ഭൂമിയിലുള്ളതിന്‍റെയും ഉടമയായ അല്ലാഹുവിന്‍റെ (മാര്‍ഗത്തിലേക്ക് അവരെ കൊണ്ട് വരുവാന്‍ വേണ്ടി) . സത്യനിഷേധികള്‍ക്ക് കഠിനമായ ശിക്ഷയാല്‍ മഹാനാശം തന്നെ.
\end{malayalam}}
\flushright{\begin{Arabic}
\quranayah[14][3]
\end{Arabic}}
\flushleft{\begin{malayalam}
അതായത്‌, പരലോകത്തെക്കാള്‍ ഇഹലോകജീവിതത്തെ കൂടുതല്‍ സ്നേഹിക്കുകയും, അല്ലാഹുവിന്‍റെ മാര്‍ഗത്തില്‍ നിന്ന് (ജനങ്ങളെ) പിന്തിരിപ്പിക്കുകയും അതിന് (ആ മാര്‍ഗത്തിന്‌) വക്രത വരുത്തുവാന്‍ ആഗ്രഹിക്കുകയും ചെയ്യുന്നവരാരോ അവര്‍ക്ക്‌. അക്കൂട്ടര്‍ വിദൂരമായ വഴികേടിലാകുന്നു.
\end{malayalam}}
\flushright{\begin{Arabic}
\quranayah[14][4]
\end{Arabic}}
\flushleft{\begin{malayalam}
യാതൊരു ദൈവദൂതനെയും തന്‍റെ ജനതയ്ക്ക് (കാര്യങ്ങള്‍) വിശദീകരിച്ച് കൊടുക്കുന്നതിന് വേണ്ടി, അവരുടെ ഭാഷയില്‍ (സന്ദേശം നല്‍കിക്കൊണ്ട്‌) അല്ലാതെ നാം നിയോഗിച്ചിട്ടില്ല. അങ്ങനെ താന്‍ ഉദ്ദേശിക്കുന്നവരെ അല്ലാഹു ദുര്‍മാര്‍ഗത്തിലാക്കുകയും, താന്‍ ഉദ്ദേശിക്കുന്നവരെ നേര്‍വഴിയിലാക്കുകയും ചെയ്യുന്നു. അവനത്രെ പ്രതാപിയും യുക്തിമാനുമായിട്ടുള്ളവന്‍.
\end{malayalam}}
\flushright{\begin{Arabic}
\quranayah[14][5]
\end{Arabic}}
\flushleft{\begin{malayalam}
നിന്‍റെ ജനതയെ ഇരുട്ടുകളില്‍ നിന്ന് വെളിച്ചത്തിലേക്ക് കൊണ്ട് വരികയും, അല്ലാഹുവിന്‍റെ (അനുഗ്രഹത്തിന്‍റെ) നാളുകളെപ്പറ്റി അവരെ ഓര്‍മിപ്പിക്കുകയും ചെയ്യുക എന്ന് നിര്‍ദേശിച്ചുകൊണ്ട് മൂസായെ നമ്മുടെ ദൃഷ്ടാന്തങ്ങളുമായി നാം അയക്കുകയുണ്ടായി. തികഞ്ഞ ക്ഷമാശീലമുള്ളവരും ഏറെ നന്ദിയുള്ളവരുമായ എല്ലാവര്‍ക്കും അതില്‍ ദൃഷ്ടാന്തങ്ങളുണ്ട്‌. തീര്‍ച്ച.
\end{malayalam}}
\flushright{\begin{Arabic}
\quranayah[14][6]
\end{Arabic}}
\flushleft{\begin{malayalam}
മൂസാ തന്‍റെ ജനതയോട് പറഞ്ഞ സന്ദര്‍ഭം (ശ്രദ്ധേയമാണ്‌.) നിങ്ങള്‍ക്ക് കടുത്ത ശിക്ഷ ആസ്വദിപ്പിക്കുകയും, നിങ്ങളുടെ ആണ്‍മക്കളെ അറുകൊലനടത്തുകയും, നിങ്ങളുടെ പെണ്ണുങ്ങളെ ജീവിക്കാന്‍ വിടുകയും ചെയ്തുകൊണ്ടിരുന്ന ഫിര്‍ഔനിന്‍റെ കൂട്ടരില്‍ നിന്ന് നിങ്ങളെ രക്ഷപ്പെടുത്തിയ സന്ദര്‍ഭത്തില്‍ അല്ലാഹു നിങ്ങള്‍ക്കു ചെയ്ത അനുഗ്രഹം നിങ്ങള്‍ ഓര്‍മ്മിക്കുക. അതില്‍ നിങ്ങളുടെ രക്ഷിതാവിങ്കല്‍ നിന്നുള്ള വമ്പിച്ച പരീക്ഷണമുണ്ട്‌.
\end{malayalam}}
\flushright{\begin{Arabic}
\quranayah[14][7]
\end{Arabic}}
\flushleft{\begin{malayalam}
നിങ്ങള്‍ നന്ദികാണിച്ചാല്‍ തീര്‍ച്ചയായും ഞാന്‍ നിങ്ങള്‍ക്ക് (അനുഗ്രഹം) വര്‍ദ്ധിപ്പിച്ചു തരുന്നതാണ്‌. എന്നാല്‍, നിങ്ങള്‍ നന്ദികേട് കാണിക്കുകയാണെങ്കില്‍ തീര്‍ച്ചയായും എന്‍റെ ശിക്ഷ കഠിനമായിരിക്കും. എന്ന് നിങ്ങളുടെ രക്ഷിതാവ് പ്രഖ്യാപിച്ച സന്ദര്‍ഭം (ശ്രദ്ധേയമത്രെ.)
\end{malayalam}}
\flushright{\begin{Arabic}
\quranayah[14][8]
\end{Arabic}}
\flushleft{\begin{malayalam}
മൂസാ പറഞ്ഞു: നിങ്ങളും, ഭൂമിയിലുള്ള മുഴുവന്‍ പേരും കൂടി നന്ദികേട് കാണിക്കുന്ന പക്ഷം, തീര്‍ച്ചയായും അല്ലാഹു പരാശ്രയമുക്തനും, സ്തുത്യര്‍ഹനുമാണ് (എന്ന് നിങ്ങള്‍ അറിഞ്ഞ് കൊള്ളുക.)
\end{malayalam}}
\flushright{\begin{Arabic}
\quranayah[14][9]
\end{Arabic}}
\flushleft{\begin{malayalam}
നൂഹിന്‍റെ ജനത, ആദ്‌, ഥമൂദ് സമുദായങ്ങള്‍, അവര്‍ക്ക് ശേഷമുള്ള അല്ലാഹുവിന്ന് മാത്രം (കൃത്യമായി) അറിയാവുന്ന ജനവിഭാഗങ്ങള്‍ എന്നിവരെല്ലാം അടങ്ങുന്ന നിങ്ങളുടെ മുന്‍ഗാമികളെപ്പറ്റിയുള്ള വര്‍ത്തമാനം നിങ്ങള്‍ക്ക് വന്നുകിട്ടിയില്ലേ? നമ്മുടെ ദൂതന്‍മാര്‍ വ്യക്തമായ തെളിവുകളും കൊണ്ട് അവരുടെ അടുക്കല്‍ ചെന്നു. അപ്പോള്‍ അവര്‍ തങ്ങളുടെ കൈകള്‍ വായിലേക്ക് മടക്കിക്കൊണ്ട്‌, നിങ്ങള്‍ ഏതൊന്നുമായി അയക്കപ്പെട്ടിരിക്കുന്നുവോ, അതില്‍ ഞങ്ങള്‍ അവിശ്വസിച്ചിരിക്കുന്നു. തീര്‍ച്ചയായും നിങ്ങള്‍ ഞങ്ങളെ ഏതൊന്നിലേക്ക് ക്ഷണിക്കുന്നുവോ അതിനെപ്പറ്റി ഞങ്ങള്‍ അവിശ്വാസജനകമായ സംശയത്തിലാണ് എന്ന് പറയുകയാണ് ചെയ്തത്‌.
\end{malayalam}}
\flushright{\begin{Arabic}
\quranayah[14][10]
\end{Arabic}}
\flushleft{\begin{malayalam}
അവരിലേക്ക് നിയോഗിക്കപ്പെട്ട ദൂതന്‍മാര്‍ പറഞ്ഞു: ആകാശങ്ങളുടെയും ഭൂമിയുടെയും സൃഷ്ടികര്‍ത്താവായ അല്ലാഹുവിന്‍റെ കാര്യത്തിലാണോ സംശയമുള്ളത്‌? നിങ്ങളുടെ പാപങ്ങള്‍ നിങ്ങള്‍ക്ക് പൊറുത്തുതരാനും, നിര്‍ണിതമായ ഒരു അവധി വരെ നിങ്ങള്‍ക്ക് സമയം നീട്ടിത്തരുവാനുമായി അവന്‍ നിങ്ങളെ ക്ഷണിക്കുന്നു. അവര്‍ (ജനങ്ങള്‍) പറഞ്ഞു: നിങ്ങള്‍ ഞങ്ങളെപ്പോലെയുള്ള മനുഷ്യര്‍ മാത്രമാകുന്നു. ഞങ്ങളുടെ പിതാക്കള്‍ ആരാധിച്ച് വരുന്നതില്‍ നിന്നു ഞങ്ങളെ പിന്തിരിപ്പിക്കാനാണ് നിങ്ങള്‍ ഉദ്ദേശിക്കുന്നത്‌. അതിനാല്‍ വ്യക്തമായ വല്ല രേഖയും നിങ്ങള്‍ ഞങ്ങള്‍ക്ക് കൊണ്ട് വന്നുതരൂ.
\end{malayalam}}
\flushright{\begin{Arabic}
\quranayah[14][11]
\end{Arabic}}
\flushleft{\begin{malayalam}
അവരോട് അവരിലേക്കുള്ള ദൈവദൂതന്‍മാര്‍ പറഞ്ഞു: ഞങ്ങള്‍ നിങ്ങളെപ്പോലെയുള്ള മനുഷ്യന്‍മാര്‍ തന്നെയാണ്‌. എങ്കിലും, അല്ലാഹു തന്‍റെ ദാസന്‍മാരില്‍ നിന്ന് താന്‍ ഉദ്ദേശിക്കുന്നവരോട് ഔദാര്യം കാണിക്കുന്നു. അല്ലാഹുവിന്‍റെ അനുമതി പ്രകാരമല്ലാതെ നിങ്ങള്‍ക്ക് യാതൊരു തെളിവും കൊണ്ട് വന്ന് തരാന്‍ ഞങ്ങള്‍ക്കാവില്ല. അല്ലാഹുവിന്‍റെ മേലാണ് വിശ്വാസികള്‍ ഭരമേല്‍പിക്കേണ്ടത്‌.
\end{malayalam}}
\flushright{\begin{Arabic}
\quranayah[14][12]
\end{Arabic}}
\flushleft{\begin{malayalam}
അല്ലാഹു ഞങ്ങളെ ഞങ്ങളുടെ വഴികളില്‍ ചേര്‍ത്ത് തന്നിരിക്കെ അവന്‍റെ മേല്‍ ഭരമേല്‍പിക്കാതിരിക്കാന്‍ ഞങ്ങള്‍ക്കെന്തു ന്യായമാണുള്ളത്‌? നിങ്ങള്‍ ഞങ്ങളെ ദ്രോഹിച്ചതിനെപ്പറ്റി ഞങ്ങള്‍ ക്ഷമിക്കുക തന്നെ ചെയ്യും. അല്ലാഹുവിന്‍റെ മേലാണ് ഭരമേല്‍പിക്കുന്നവരെല്ലാം ഭരമേല്‍പിക്കേണ്ടത്‌.
\end{malayalam}}
\flushright{\begin{Arabic}
\quranayah[14][13]
\end{Arabic}}
\flushleft{\begin{malayalam}
അവിശ്വാസികള്‍ തങ്ങളിലേക്കുള്ള ദൈവദൂതന്‍മാരോട് പറഞ്ഞു: ഞങ്ങളുടെ നാട്ടില്‍ നിന്ന് നിങ്ങളെ ഞങ്ങള്‍ പുറത്താക്കുക തന്നെ ചെയ്യും. അല്ലാത്ത പക്ഷം നിങ്ങള്‍ ഞങ്ങളുടെ മതത്തിലേക്ക് തിരിച്ചുവന്നേ തീരു. അപ്പോള്‍ അവര്‍ക്ക് (ആ ദൂതന്‍മാര്‍ക്ക്‌) അവരുടെ രക്ഷിതാവ് സന്ദേശം നല്‍കി. തീര്‍ച്ചയായും നാം ആ അക്രമികളെ നശിപ്പിക്കുക തന്നെ ചെയ്യും.
\end{malayalam}}
\flushright{\begin{Arabic}
\quranayah[14][14]
\end{Arabic}}
\flushleft{\begin{malayalam}
അവര്‍ക്കു ശേഷം നിങ്ങളെ നാം നാട്ടില്‍ അധിവസിപ്പിക്കുകയും ചെയ്യുന്നതാണ്‌. എന്‍റെ സ്ഥാനത്തെ ഭയപ്പെടുകയും, എന്‍റെ താക്കീതിനെ ഭയപ്പെടുകയും ചെയ്തവര്‍ക്കുള്ളതാണ് ആ അനുഗ്രഹം.
\end{malayalam}}
\flushright{\begin{Arabic}
\quranayah[14][15]
\end{Arabic}}
\flushleft{\begin{malayalam}
അവര്‍ (ആ ദൂതന്‍മാര്‍) വിജയത്തിനായി (അല്ലാഹുവോട്‌) അപേക്ഷിച്ചു. ഏത് ദുര്‍വാശിക്കാരനായ സര്‍വ്വാധിപതിയും പരാജയപ്പെടുകയും ചെയ്തു.
\end{malayalam}}
\flushright{\begin{Arabic}
\quranayah[14][16]
\end{Arabic}}
\flushleft{\begin{malayalam}
അവന്‍റെ പിന്നാലെ തന്നെയുണ്ട് നരകം. ചോരയും ചലവും കലര്‍ന്ന നീരില്‍ നിന്നായിരിക്കും അവന്ന് കുടിക്കാന്‍ നല്‍കപ്പെടുന്നത്‌.
\end{malayalam}}
\flushright{\begin{Arabic}
\quranayah[14][17]
\end{Arabic}}
\flushleft{\begin{malayalam}
അതവന്‍ കീഴ്പോട്ടിറക്കാന്‍ ശ്രമിക്കും. അത് തൊണ്ടയില്‍ നിന്ന് ഇറക്കാന്‍ അവന്ന് കഴിഞ്ഞേക്കുകയില്ല. എല്ലായിടത്ത് നിന്നും മരണം അവന്‍റെ നേര്‍ക്ക് വരും. എന്നാല്‍ അവന്‍ മരണപ്പെടുകയില്ല താനും. അതിന്‍റെ പിന്നാലെ തന്നെയുണ്ട് കഠോരമായ വേറെയും ശിക്ഷ.
\end{malayalam}}
\flushright{\begin{Arabic}
\quranayah[14][18]
\end{Arabic}}
\flushleft{\begin{malayalam}
തങ്ങളുടെ രക്ഷിതാവിനെ നിഷേധിച്ചവരെ, അവരുടെ കര്‍മ്മങ്ങളെ ഉപമിക്കാവുന്നത് കൊടുങ്കാറ്റുള്ള ഒരു ദിവസം കനത്ത കാറ്റടിച്ചു പാറിപ്പോയ വെണ്ണീറിനോടാകുന്നു. അവര്‍ പ്രവര്‍ത്തിച്ചുണ്ടാക്കിയതില്‍ നിന്ന് യാതൊന്നും അനുഭവിക്കാന്‍ അവര്‍ക്ക് സാധിക്കുന്നതല്ല. അത് തന്നെയാണ് വിദൂരമായ മാര്‍ഗഭ്രംശം.
\end{malayalam}}
\flushright{\begin{Arabic}
\quranayah[14][19]
\end{Arabic}}
\flushleft{\begin{malayalam}
ആകാശങ്ങളും ഭൂമിയും അല്ലാഹു ശരിയായ ക്രമത്തിലാണ് സൃഷ്ടിച്ചിട്ടുള്ളതെന്ന് നീ കണ്ടില്ലേ? അവന്‍ ഉദ്ദേശിക്കുന്ന പക്ഷം നിങ്ങളെ അവന്‍ നീക്കം ചെയ്യുകയും, ഒരു പുതിയ സൃഷ്ടിയെ അവന്‍ കൊണ്ട് വരികയും ചെയ്യുന്നതാണ്‌.
\end{malayalam}}
\flushright{\begin{Arabic}
\quranayah[14][20]
\end{Arabic}}
\flushleft{\begin{malayalam}
അല്ലാഹുവെ സംബന്ധിച്ചിടത്തോളം അത് ഒരു വിഷമകരമായ കാര്യമല്ല.
\end{malayalam}}
\flushright{\begin{Arabic}
\quranayah[14][21]
\end{Arabic}}
\flushleft{\begin{malayalam}
അവരെല്ലാവരും അല്ലാഹുവിങ്കലേക്ക് പുറപ്പെട്ട് വന്നിരിക്കുകയാണ്‌. അപ്പോഴതാ ദുര്‍ബലര്‍ അഹങ്കരിച്ചിരുന്നവരോട് പറയുന്നു: തീര്‍ച്ചയായും ഞങ്ങള്‍ നിങ്ങളുടെ അനുയായികളായിരുന്നല്ലോ. ആകയാല്‍ അല്ലാഹുവിന്‍റെ ശിക്ഷയില്‍ നിന്ന് അല്‍പമെങ്കിലും നിങ്ങള്‍ ഞങ്ങളില്‍ നിന്ന് ഒഴിവാക്കിത്തരുമോ? അവര്‍ (അഹങ്കരിച്ചിരുന്നവര്‍) പറയും: അല്ലാഹു ഞങ്ങളെ നേര്‍വഴിയിലാക്കിയിരുന്നെങ്കില്‍ ഞങ്ങള്‍ നിങ്ങളെയും നേര്‍വഴിയിലാക്കുമായിരുന്നു. നമ്മെ സംബന്ധിച്ചേടത്തോളം നാം ക്ഷമകേട് കാണിച്ചാലും ക്ഷമിച്ചാലും ഒരു പോലെയാകുന്നു. നമുക്ക് യാതൊരു രക്ഷാമാര്‍ഗവുമില്ല.
\end{malayalam}}
\flushright{\begin{Arabic}
\quranayah[14][22]
\end{Arabic}}
\flushleft{\begin{malayalam}
കാര്യം തീരുമാനിക്കപ്പെട്ട് കഴിഞ്ഞാല്‍ പിശാച് പറയുന്നതാണ്‌. തീര്‍ച്ചയായും അല്ലാഹു നിങ്ങളോട് ഒരു വാഗ്ദാനം ചെയ്തു. സത്യവാഗ്ദാനം. ഞാനും നിങ്ങളോട് വാഗ്ദാനം ചെയ്തു. എന്നാല്‍ നിങ്ങളോട് (ഞാന്‍ ചെയ്ത വാഗ്ദാനം) ഞാന്‍ ലംഘിച്ചു. എനിക്ക് നിങ്ങളുടെ മേല്‍ യാതൊരു അധികാരവും ഉണ്ടായിരുന്നില്ല. ഞാന്‍ നിങ്ങളെ ക്ഷണിച്ചു. അപ്പോള്‍ നിങ്ങളെനിക്ക് ഉത്തരം നല്‍കി എന്ന് മാത്രം. ആകയാല്‍, നിങ്ങള്‍ എന്നെ കുറ്റപ്പെടുത്തേണ്ട, നിങ്ങള്‍ നിങ്ങളെത്തന്നെ കുറ്റപ്പെടുത്തുക. എനിക്ക് നിങ്ങളെ സഹായിക്കാനാവില്ല. നിങ്ങള്‍ക്ക് എന്നെയും സഹായിക്കാനാവില്ല. മുമ്പ് നിങ്ങള്‍ എന്നെ പങ്കാളിയാക്കിയിരുന്നതിനെ ഞാനിതാ നിഷേധിച്ചിരിക്കുന്നു. തീര്‍ച്ചയായും അക്രമകാരികളാരോ അവര്‍ക്കാണ് വേദനയേറിയ ശിക്ഷയുള്ളത്‌.
\end{malayalam}}
\flushright{\begin{Arabic}
\quranayah[14][23]
\end{Arabic}}
\flushleft{\begin{malayalam}
വിശ്വസിക്കുകയും സല്‍കര്‍മ്മങ്ങള്‍ പ്രവര്‍ത്തിക്കുകയും ചെയ്തവര്‍ താഴ്ഭാഗത്ത് കൂടി അരുവികള്‍ ഒഴുകുന്ന സ്വര്‍ഗത്തോപ്പുകളില്‍ പ്രവേശിപ്പിക്കപ്പെടുന്നതാണ്‌. അവരുടെ രക്ഷിതാവിന്‍റെ അനുമതിപ്രകാരം അവരതില്‍ നിത്യവാസികളായിരിക്കും. അവര്‍ക്ക് അവിടെയുള്ള അഭിവാദ്യം സലാം ആയിരിക്കും.
\end{malayalam}}
\flushright{\begin{Arabic}
\quranayah[14][24]
\end{Arabic}}
\flushleft{\begin{malayalam}
അല്ലാഹു നല്ല വചനത്തിന് എങ്ങനെയാണ് ഉപമ നല്‍കിയിരിക്കുന്നത് എന്ന് നീ കണ്ടില്ലേ? (അത്‌) ഒരു നല്ല മരം പോലെയാകുന്നു. അതിന്‍റെ മുരട് ഉറച്ചുനില്‍ക്കുന്നതും അതിന്‍റെ ശാഖകള്‍ ആകാശത്തേക്ക് ഉയര്‍ന്ന് നില്‍ക്കുന്നതുമാകുന്നു.
\end{malayalam}}
\flushright{\begin{Arabic}
\quranayah[14][25]
\end{Arabic}}
\flushleft{\begin{malayalam}
അതിന്‍റെ രക്ഷിതാവിന്‍റെ ഉത്തരവനുസരിച്ച് അത് എല്ലാ കാലത്തും അതിന്‍റെ ഫലം നല്‍കിക്കൊണ്ടിരിക്കും. മനുഷ്യര്‍ക്ക് അവര്‍ ആലോചിച്ച് മനസ്സിലാക്കുന്നതിനായി അല്ലാഹു ഉപമകള്‍ വിവരിച്ചുകൊടുക്കുന്നു.
\end{malayalam}}
\flushright{\begin{Arabic}
\quranayah[14][26]
\end{Arabic}}
\flushleft{\begin{malayalam}
ദുഷിച്ച വചനത്തെ ഉപമിക്കാവുന്നതാകട്ടെ, ഒരു ദുഷിച്ച വൃക്ഷത്തോടാകുന്നു. ഭൂതലത്തില്‍ നിന്ന് അത് പിഴുതെടുക്കപ്പെട്ടിരിക്കുന്നു. അതിന്ന് യാതൊരു നിലനില്‍പുമില്ല.
\end{malayalam}}
\flushright{\begin{Arabic}
\quranayah[14][27]
\end{Arabic}}
\flushleft{\begin{malayalam}
ഐഹികജീവിതത്തിലും, പരലോകത്തും സുസ്ഥിരമായ വാക്കുകൊണ്ട് സത്യവിശ്വാസികളെ അല്ലാഹു ഉറപ്പിച്ച് നിര്‍ത്തുന്നതാണ്‌. അക്രമകാരികളെ അല്ലാഹു ദുര്‍മാര്‍ഗത്തിലാക്കുകയും ചെയ്യും. അല്ലാഹു താന്‍ ഉദ്ദേശിക്കുന്നതെന്തോ അത് പ്രവര്‍ത്തിക്കുന്നു.
\end{malayalam}}
\flushright{\begin{Arabic}
\quranayah[14][28]
\end{Arabic}}
\flushleft{\begin{malayalam}
അല്ലാഹുവിന്‍റെ അനുഗ്രഹത്തിന് (നന്ദികാണിക്കേണ്ടതിനു) പകരം നന്ദികേട് കാണിക്കുകയും, തങ്ങളുടെ ജനതയെ നാശത്തിന്‍റെ ഭവനത്തില്‍ ഇറക്കിക്കളയുകയും ചെയ്ത ഒരു വിഭാഗത്തെ നീ കണ്ടില്ലേ?
\end{malayalam}}
\flushright{\begin{Arabic}
\quranayah[14][29]
\end{Arabic}}
\flushleft{\begin{malayalam}
അഥവാ നരകത്തില്‍. അതില്‍ അവര്‍ എരിയുന്നതാണ്‌. അത് എത്ര മോശമായ താമസസ്ഥലം!
\end{malayalam}}
\flushright{\begin{Arabic}
\quranayah[14][30]
\end{Arabic}}
\flushleft{\begin{malayalam}
അല്ലാഹുവിന്‍റെ മാര്‍ഗത്തില്‍ നിന്ന് (ജനങ്ങളെ) തെറ്റിച്ചുകളയാന്‍ വേണ്ടി അവര്‍ അവന്ന് ചില സമന്‍മാരെ ഉണ്ടാക്കി വെച്ചിരിക്കുന്നു. പറയുക: നിങ്ങള്‍ സുഖിച്ച് കൊള്ളൂ. നിങ്ങളുടെ മടക്കം നരകത്തിലേക്ക് തന്നെയാണ്‌.
\end{malayalam}}
\flushright{\begin{Arabic}
\quranayah[14][31]
\end{Arabic}}
\flushleft{\begin{malayalam}
വിശ്വാസികളായ എന്‍റെ ദാസന്‍മാരോട് നീ പറയുക: അവര്‍ നമസ്കാരം മുറപോലെ നിര്‍വഹിക്കുകയും, നാം അവര്‍ക്കു നല്‍കിയ ധനത്തില്‍ നിന്ന്‌, യാതൊരു ക്രയവിക്രയവും ചങ്ങാത്തവും നടക്കാത്ത ഒരു ദിവസം വരുന്നതിന് മുമ്പായി രഹസ്യമായും പരസ്യമായും അവര്‍ (നല്ല വഴിയില്‍) ചെലവഴിക്കുകയും ചെയ്ത് കൊള്ളട്ടെ.
\end{malayalam}}
\flushright{\begin{Arabic}
\quranayah[14][32]
\end{Arabic}}
\flushleft{\begin{malayalam}
അല്ലാഹുവത്രെ ആകാശങ്ങളും ഭൂമിയും സൃഷ്ടിക്കുകയും, എന്നിട്ട് അതുമൂലം നിങ്ങളുടെ ഉപജീവനത്തിനായി കായ്കനികള്‍ ഉല്‍പാദിപ്പിക്കുകയും ചെയ്തത്‌. അവന്‍റെ കല്‍പന(നിയമ) പ്രകാരം കടലിലൂടെ, സഞ്ചരിക്കുന്നതിനായി അവന്‍ നിങ്ങള്‍ക്കു കപ്പലുകള്‍ വിധേയമാക്കിത്തരികയും ചെയ്തിരിക്കുന്നു. നദികളെയും അവന്‍ നിങ്ങള്‍ക്ക് വിധേയമാക്കിത്തന്നിരിക്കുന്നു.
\end{malayalam}}
\flushright{\begin{Arabic}
\quranayah[14][33]
\end{Arabic}}
\flushleft{\begin{malayalam}
സൂര്യനെയും ചന്ദ്രനെയും പതിവായി സഞ്ചരിച്ച് കൊണ്ടിരിക്കുന്ന നിലയില്‍ അവന്‍ നിങ്ങള്‍ക്കു വിധേയമാക്കി തന്നിരിക്കുന്നു. രാവിനെയും പകലിനെയും അവന്‍ നിങ്ങള്‍ക്കു വിധേയമാക്കിത്തന്നിരിക്കുന്നു.
\end{malayalam}}
\flushright{\begin{Arabic}
\quranayah[14][34]
\end{Arabic}}
\flushleft{\begin{malayalam}
നിങ്ങളവനോട് ആവശ്യപ്പെട്ടതില്‍ നിന്നെല്ലാം നിങ്ങള്‍ക്ക് അവന്‍ നല്‍കിയിരിക്കുന്നു. അല്ലാഹുവിന്‍റെ അനുഗ്രഹം നിങ്ങള്‍ എണ്ണുകയാണെങ്കില്‍ നിങ്ങള്‍ക്കതിന്‍റെ കണക്കെടുക്കാനാവില്ല. തീര്‍ച്ചയായും മനുഷ്യന്‍ മഹാ അക്രമകാരിയും വളരെ നന്ദികെട്ടവനും തന്നെ.
\end{malayalam}}
\flushright{\begin{Arabic}
\quranayah[14][35]
\end{Arabic}}
\flushleft{\begin{malayalam}
ഇബ്രാഹീം ഇപ്രകാരം പറഞ്ഞ സന്ദര്‍ഭം (ശ്രദ്ധേയമാകുന്നു.) എന്‍റെ രക്ഷിതാവേ, നീ ഈ നാടിനെ (മക്കയെ) നിര്‍ഭയത്വമുള്ളതാക്കുകയും, എന്നെയും എന്‍റെ മക്കളെയും ഞങ്ങള്‍ വിഗ്രഹങ്ങള്‍ക്ക് ആരാധന നടത്തുന്നതില്‍ നിന്ന് അകറ്റി നിര്‍ത്തുകയും ചെയ്യേണമേ.
\end{malayalam}}
\flushright{\begin{Arabic}
\quranayah[14][36]
\end{Arabic}}
\flushleft{\begin{malayalam}
എന്‍റെ രക്ഷിതാവേ! തീര്‍ച്ചയായും അവ (വിഗ്രഹങ്ങള്‍) മനുഷ്യരില്‍ നിന്ന് വളരെപ്പേരെ പിഴപ്പിച്ച് കളഞ്ഞിരിക്കുന്നു. അതിനാല്‍ എന്നെ ആര്‍ പിന്തുടര്‍ന്നുവോ അവന്‍ എന്‍റെ കൂട്ടത്തില്‍ പെട്ടവനാകുന്നു. ആരെങ്കിലും എന്നോട് അനുസരണക്കേട് കാണിക്കുന്ന പക്ഷം തീര്‍ച്ചയായും നീ ഏറെ പൊറുക്കുന്നവനും കരുണാനിധിയുമാണല്ലോ.
\end{malayalam}}
\flushright{\begin{Arabic}
\quranayah[14][37]
\end{Arabic}}
\flushleft{\begin{malayalam}
ഞങ്ങളുടെ രക്ഷിതാവേ, എന്‍റെ സന്തതികളില്‍ നിന്ന് (ചിലരെ) കൃഷിയൊന്നും ഇല്ലാത്ത ഒരു താഴ്‌വരയില്‍, നിന്‍റെ പവിത്രമായ ഭവനത്തിന്‍റെ (കഅ്ബയുടെ) അടുത്ത് ഞാനിതാ താമസിപ്പിച്ചിരിക്കുന്നു. ഞങ്ങളുടെ രക്ഷിതാവേ, അവര്‍ നമസ്കാരം മുറപ്രകാരം നിര്‍വഹിക്കുവാന്‍ വേണ്ടിയാണ് (അങ്ങനെ ചെയ്തത്‌.) അതിനാല്‍ മനുഷ്യരില്‍ ചിലരുടെ മനസ്സുകളെ നീ അവരോട് ചായ്‌വുള്ളതാക്കുകയും, അവര്‍ക്ക് കായ്കനികളില്‍ നിന്ന് നീ ഉപജീവനം നല്‍കുകയും ചെയ്യേണമേ. അവര്‍ നന്ദികാണിച്ചെന്ന് വരാം.
\end{malayalam}}
\flushright{\begin{Arabic}
\quranayah[14][38]
\end{Arabic}}
\flushleft{\begin{malayalam}
ഞങ്ങളുടെ രക്ഷിതാവേ, തീര്‍ച്ചയായും ഞങ്ങള്‍ മറച്ചുവെക്കുന്നതും പരസ്യമാക്കുന്നതും എല്ലാം നീ അറിയും. ഭൂമിയിലുള്ളതോ ആകാശത്തുള്ളതോ ആയ യാതൊരു വസ്തുവും അല്ലാഹുവിന് അവ്യക്തമാകുകയില്ല.
\end{malayalam}}
\flushright{\begin{Arabic}
\quranayah[14][39]
\end{Arabic}}
\flushleft{\begin{malayalam}
വാര്‍ദ്ധക്യകാലത്ത് എനിക്ക് ഇസ്മാഈലിനെയും ഇഷാഖിനെയും പ്രദാനം ചെയ്ത അല്ലാഹുവിന് സ്തുതി. തീര്‍ച്ചയായും എന്‍റെ രക്ഷിതാവ് പ്രാര്‍ത്ഥന കേള്‍ക്കുന്നവനാണ്‌.
\end{malayalam}}
\flushright{\begin{Arabic}
\quranayah[14][40]
\end{Arabic}}
\flushleft{\begin{malayalam}
എന്‍റെ രക്ഷിതാവേ, എന്നെ നീ നമസ്കാരം മുറപ്രകാരം നിര്‍വഹിക്കുന്നവനാക്കേണമേ. എന്‍റെ സന്തതികളില്‍ പെട്ടവരെയും (അപ്രകാരം ആക്കേണമേ) ഞങ്ങളുടെ രക്ഷിതാവേ, എന്‍റെ പ്രാര്‍ത്ഥന നീ സ്വീകരിക്കുകയും ചെയ്യേണമേ.
\end{malayalam}}
\flushright{\begin{Arabic}
\quranayah[14][41]
\end{Arabic}}
\flushleft{\begin{malayalam}
ഞങ്ങളുടെ രക്ഷിതാവേ, വിചാരണ നിലവില്‍ വരുന്ന ദിവസം എനിക്കും എന്‍റെ മാതാപിതാക്കള്‍ക്കും സത്യവിശ്വാസികള്‍ക്കും നീ പൊറുത്തുതരേണമേ.
\end{malayalam}}
\flushright{\begin{Arabic}
\quranayah[14][42]
\end{Arabic}}
\flushleft{\begin{malayalam}
അക്രമികള്‍ പ്രവര്‍ത്തിച്ച് കൊണ്ടിരിക്കുന്നതിനെപ്പറ്റി അല്ലാഹു അശ്രദ്ധനാണെന്ന് നീ വിചാരിച്ച് പോകരുത്‌. കണ്ണുകള്‍ തള്ളിപ്പോകുന്ന ഒരു (ഭയാനകമായ) ദിവസം വരെ അവര്‍ക്കു സമയം നീട്ടികൊടുക്കുക മാത്രമാണ് ചെയ്യുന്നത്‌.
\end{malayalam}}
\flushright{\begin{Arabic}
\quranayah[14][43]
\end{Arabic}}
\flushleft{\begin{malayalam}
(അന്ന്‌) ബദ്ധപ്പെട്ട് ഓടിക്കൊണ്ടും, തലകള്‍ ഉയര്‍ത്തിപ്പിടിച്ച് കൊണ്ടും (അവര്‍ വരും) അവരുടെ ദൃഷ്ടികള്‍ അവരിലേക്ക് തിരിച്ചുവരികയില്ല. അവരുടെ മനസ്സുകള്‍ ശൂന്യവുമായിരിക്കും.
\end{malayalam}}
\flushright{\begin{Arabic}
\quranayah[14][44]
\end{Arabic}}
\flushleft{\begin{malayalam}
മനുഷ്യര്‍ക്ക് ശിക്ഷ വന്നെത്തുന്ന ഒരു ദിവസത്തെപ്പറ്റി നീ അവര്‍ക്ക് താക്കീത് നല്‍കുക. അക്രമം ചെയ്തവര്‍ അപ്പോള്‍ പറയും: ഞങ്ങളുടെ രക്ഷിതാവേ, അടുത്ത ഒരു അവധി വരെ ഞങ്ങള്‍ക്ക് നീ സമയം നീട്ടിത്തരേണമേ. എങ്കില്‍ നിന്‍റെ വിളിക്ക് ഞങ്ങള്‍ ഉത്തരം നല്‍കുകയും, ദൂതന്‍മാരെ ഞങ്ങള്‍ പിന്തുടരുകയും ചെയ്തുകൊള്ളാം. നിങ്ങള്‍ക്കു (മറ്റൊരു ലോകത്തേക്കു) മാറേണ്ടിവരില്ലെന്ന് നിങ്ങള്‍ സത്യം ചെയ്തു പറഞ്ഞിട്ടുണ്ടായിരുന്നില്ലേ? (എന്നായിരിക്കും അവര്‍ക്ക് നല്‍കപ്പെടുന്ന മറുപടി.)
\end{malayalam}}
\flushright{\begin{Arabic}
\quranayah[14][45]
\end{Arabic}}
\flushleft{\begin{malayalam}
അവരവര്‍ക്കു തന്നെ ദ്രോഹം വരുത്തിവെച്ച ഒരു ജനവിഭാഗത്തിന്‍റെ വാസസ്ഥലങ്ങളിലാണ് നിങ്ങള്‍ താമസിച്ചിരുന്നത്‌. അവരെക്കൊണ്ട് നാം എങ്ങനെയാണ് പ്രവര്‍ത്തിച്ചതെന്ന് നിങ്ങള്‍ക്ക് വ്യക്തമായി മനസ്സിലായിട്ടുമുണ്ട്‌. നിങ്ങള്‍ക്ക് നാം ഉപമകള്‍ വിവരിച്ചുതന്നിട്ടുമുണ്ട്‌.
\end{malayalam}}
\flushright{\begin{Arabic}
\quranayah[14][46]
\end{Arabic}}
\flushleft{\begin{malayalam}
അവരാല്‍ കഴിയുന്ന തന്ത്രം അവര്‍ പ്രയോഗിച്ചിട്ടുണ്ട്‌. അല്ലാഹുവിങ്കലുണ്ട് അവര്‍ക്കായുള്ള തന്ത്രം അവരുടെ തന്ത്രം നിമിത്തം പര്‍വ്വതങ്ങള്‍ നീങ്ങിപ്പോകാന്‍ മാത്രമൊന്നുമായിട്ടില്ല.
\end{malayalam}}
\flushright{\begin{Arabic}
\quranayah[14][47]
\end{Arabic}}
\flushleft{\begin{malayalam}
ആകയാല്‍ അല്ലാഹു തന്‍റെ ദൂതന്‍മാരോട് ചെയ്ത വാഗ്ദാനം ലംഘിക്കുന്നവനാണെന്ന് നീ വിചാരിച്ച് പോകരുത്‌. തീര്‍ച്ചയായും അല്ലാഹു പ്രതാപിയും ശിക്ഷാനടപടി കൈക്കൊള്ളുന്നവനുമാണ്‌;
\end{malayalam}}
\flushright{\begin{Arabic}
\quranayah[14][48]
\end{Arabic}}
\flushleft{\begin{malayalam}
ഭൂമി ഈ ഭൂമിയല്ലാത്ത മറ്റൊന്നായും, അത് പോലെ ആകാശങ്ങളും മാറ്റപ്പെടുകയും ഏകനും സര്‍വ്വാധികാരിയുമായ അല്ലാഹുവിങ്കലേക്ക് അവരെല്ലാം പുറപ്പെട്ട് വരുകയും ചെയ്യുന്ന ദിവസം.
\end{malayalam}}
\flushright{\begin{Arabic}
\quranayah[14][49]
\end{Arabic}}
\flushleft{\begin{malayalam}
ആ ദിവസം കുറ്റവാളികളെ ചങ്ങലകളില്‍ അന്യോന്യം ചേര്‍ത്ത് ബന്ധിക്കപ്പെട്ടതായിട്ട് നിനക്ക് കാണാം.
\end{malayalam}}
\flushright{\begin{Arabic}
\quranayah[14][50]
\end{Arabic}}
\flushleft{\begin{malayalam}
അവരുടെ കുപ്പായങ്ങള്‍ കറുത്ത കീല് (ടാര്‍) കൊണ്ടുള്ളതായിരിക്കും. അവരുടെ മുഖങ്ങളെ തീ പൊതിയുന്നതുമാണ്‌.
\end{malayalam}}
\flushright{\begin{Arabic}
\quranayah[14][51]
\end{Arabic}}
\flushleft{\begin{malayalam}
ഓരോ വ്യക്തിക്കും താന്‍ സമ്പാദിച്ചുണ്ടാക്കിയതിനുള്ള പ്രതിഫലം അല്ലാഹു നല്‍കുവാന്‍ വേണ്ടിയത്രെ അത്‌. തീര്‍ച്ചയായും അല്ലാഹു അതിവേഗത്തില്‍ കണക്ക് നോക്കുന്നവനത്രെ.
\end{malayalam}}
\flushright{\begin{Arabic}
\quranayah[14][52]
\end{Arabic}}
\flushleft{\begin{malayalam}
ഇത് മനുഷ്യര്‍ക്ക് വേണ്ടി വ്യക്തമായ ഒരു ഉല്‍ബോധനമാകുന്നു. ഇതു മുഖേന അവര്‍ക്കു മുന്നറിയിപ്പ് നല്‍കപ്പെടേണ്ടതിനും, അവന്‍ ഒരേയൊരു ആരാധ്യന്‍ മാത്രമാണെന്ന് അവര്‍ മനസ്സിലാക്കുന്നതിനും ബുദ്ധിമാന്‍മാര്‍ ആലോചിച്ച് മനസ്സിലാക്കുന്നതിനും വേണ്ടിയുള്ള (ഉല്‍ബോധനം) .
\end{malayalam}}
\chapter{\textmalayalam{ഹിജ്റ്}}
\begin{Arabic}
\Huge{\centerline{\basmalah}}\end{Arabic}
\flushright{\begin{Arabic}
\quranayah[15][1]
\end{Arabic}}
\flushleft{\begin{malayalam}
അലിഫ് ലാംറാ-വേദഗ്രന്ഥത്തിലെ അഥവാ (കാര്യങ്ങള്‍) സ്പഷ്ടമാക്കുന്ന ഖുര്‍ആനിലെ വചനങ്ങളാകുന്നു അവ.
\end{malayalam}}
\flushright{\begin{Arabic}
\quranayah[15][2]
\end{Arabic}}
\flushleft{\begin{malayalam}
തങ്ങള്‍ മുസ്ലിംകളായിരുന്നെങ്കില്‍ എത്ര നന്നായിരുന്നേനെ എന്ന് ചിലപ്പോള്‍ സത്യനിഷേധികള്‍ കൊതിച്ച് പോകും.
\end{malayalam}}
\flushright{\begin{Arabic}
\quranayah[15][3]
\end{Arabic}}
\flushleft{\begin{malayalam}
നീ അവരെ വിട്ടേക്കുക. അവര്‍ തിന്നുകയും സുഖിക്കുകയും വ്യാമോഹത്തില്‍ വ്യാപൃതരാകുകയും ചെയ്തു കൊള്ളട്ടെ. (പിന്നീട്‌) അവര്‍ മനസ്സിലാക്കിക്കൊള്ളും.
\end{malayalam}}
\flushright{\begin{Arabic}
\quranayah[15][4]
\end{Arabic}}
\flushleft{\begin{malayalam}
ഒരു രാജ്യത്തെയും നാം നശിപ്പിച്ചിട്ടില്ല; നിര്‍ണിതമായ ഒരു അവധി അതിന്ന് നല്‍കപ്പെട്ടിട്ടല്ലാതെ.
\end{malayalam}}
\flushright{\begin{Arabic}
\quranayah[15][5]
\end{Arabic}}
\flushleft{\begin{malayalam}
യാതൊരു സമുദായവും അതിന്‍റെ അവധിയേക്കാള്‍ മുമ്പിലാവുകയില്ല. (അവധി വിട്ട്‌) അവര്‍ പിന്നോട്ട് പോകുകയുമില്ല.
\end{malayalam}}
\flushright{\begin{Arabic}
\quranayah[15][6]
\end{Arabic}}
\flushleft{\begin{malayalam}
അവര്‍ (അവിശ്വാസികള്‍) പറഞ്ഞു: ഹേ; ഉല്‍ബോധനം അവതരിപ്പിക്കപ്പെട്ടിട്ടുള്ള മനുഷ്യാ! തീര്‍ച്ചയായും നീ ഒരു ഭ്രാന്തന്‍ തന്നെ.
\end{malayalam}}
\flushright{\begin{Arabic}
\quranayah[15][7]
\end{Arabic}}
\flushleft{\begin{malayalam}
നീ സത്യവാന്‍മാരില്‍ പെട്ടവനാണെങ്കില്‍ നീ ഞങ്ങളുടെ അടുക്കല്‍ മലക്കുകളെ കൊണ്ട് വരാത്തതെന്ത്‌?
\end{malayalam}}
\flushright{\begin{Arabic}
\quranayah[15][8]
\end{Arabic}}
\flushleft{\begin{malayalam}
എന്നാല്‍ ന്യായമായ കാരണത്താലല്ലാതെ നാം മലക്കുകളെ ഇറക്കുന്നതല്ല. അന്നേരം അവര്‍ക്ക് (സത്യനിഷേധികള്‍ക്ക്‌) സാവകാശം നല്‍കപ്പെടുന്നതുമല്ല.
\end{malayalam}}
\flushright{\begin{Arabic}
\quranayah[15][9]
\end{Arabic}}
\flushleft{\begin{malayalam}
തീര്‍ച്ചയായും നാമാണ് ആ ഉല്‍ബോധനം അവതരിപ്പിച്ചത്‌. തീര്‍ച്ചയായും നാം അതിനെ കാത്തുസൂക്ഷിക്കുന്നതുമാണ്‌.
\end{malayalam}}
\flushright{\begin{Arabic}
\quranayah[15][10]
\end{Arabic}}
\flushleft{\begin{malayalam}
തീര്‍ച്ചയായും നിനക്ക് മുമ്പ് പൂര്‍വ്വികന്‍മാരിലെ പല കക്ഷികളിലേക്ക് നാം ദൂതന്‍മാരെ അയച്ചിട്ടുണ്ട്‌.
\end{malayalam}}
\flushright{\begin{Arabic}
\quranayah[15][11]
\end{Arabic}}
\flushleft{\begin{malayalam}
ഏതൊരു ദൂതന്‍ അവരുടെ അടുത്ത് ചെല്ലുമ്പോഴും അവര്‍ അദ്ദേഹത്തെ പരിഹസിക്കാതിരുന്നിട്ടില്ല.
\end{malayalam}}
\flushright{\begin{Arabic}
\quranayah[15][12]
\end{Arabic}}
\flushleft{\begin{malayalam}
അപ്രകാരം കുറ്റവാളികളുടെ ഹൃദയങ്ങളില്‍ അത് (പരിഹാസം) നാം ചെലുത്തി വിടുന്നതാണ്‌.
\end{malayalam}}
\flushright{\begin{Arabic}
\quranayah[15][13]
\end{Arabic}}
\flushleft{\begin{malayalam}
പൂര്‍വ്വികന്‍മാരില്‍ (ദൈവത്തിന്‍റെ) നടപടി നടന്ന് കഴിഞ്ഞിട്ടും അവര്‍ ഇതില്‍ വിശ്വസിക്കുന്നില്ല.
\end{malayalam}}
\flushright{\begin{Arabic}
\quranayah[15][14]
\end{Arabic}}
\flushleft{\begin{malayalam}
അവരുടെ മേല്‍ ആകാശത്ത് നിന്ന് നാം ഒരു കവാടം തുറന്നുകൊടുക്കുകയും, എന്നിട്ട് അതിലൂടെ അവര്‍ കയറിപ്പോയിക്കൊണ്ടിരിക്കുകയും ചെയ്താല്‍ പോലും.
\end{malayalam}}
\flushright{\begin{Arabic}
\quranayah[15][15]
\end{Arabic}}
\flushleft{\begin{malayalam}
അവര്‍ പറയും: ഞങ്ങളുടെ കണ്ണുകള്‍ക്ക് മത്ത് ബാധിച്ചത് മാത്രമാണ്‌. അല്ല, ഞങ്ങള്‍ മാരണം ചെയ്യപ്പെട്ട ഒരു കൂട്ടം ആളുകളാണ്‌.
\end{malayalam}}
\flushright{\begin{Arabic}
\quranayah[15][16]
\end{Arabic}}
\flushleft{\begin{malayalam}
ആകാശത്ത് നാം നക്ഷത്രമണ്ഡലങ്ങള്‍ നിശ്ചയിക്കുകയും, നോക്കുന്നവര്‍ക്ക് അവയെ നാം അലംകൃതമാക്കുകയും ചെയ്തിരിക്കുന്നു.
\end{malayalam}}
\flushright{\begin{Arabic}
\quranayah[15][17]
\end{Arabic}}
\flushleft{\begin{malayalam}
ആട്ടിയകറ്റപ്പെട്ട എല്ലാ പിശാചുക്കളില്‍ നിന്നും അതിനെ നാം കാത്തുസൂക്ഷിക്കുകയും ചെയ്തിരിക്കുന്നു.
\end{malayalam}}
\flushright{\begin{Arabic}
\quranayah[15][18]
\end{Arabic}}
\flushleft{\begin{malayalam}
എന്നാല്‍ കട്ടുകേള്‍ക്കാന്‍ ശ്രമിച്ചവനാകട്ടെ, പ്രത്യക്ഷമായ ഒരു തീജ്വാല അവനെ പിന്തുടരുന്നതാണ്‌.
\end{malayalam}}
\flushright{\begin{Arabic}
\quranayah[15][19]
\end{Arabic}}
\flushleft{\begin{malayalam}
ഭൂമിയെ നാം വിശാലമാക്കുകയും അതില്‍ ഉറച്ചുനില്‍ക്കുന്ന പര്‍വ്വതങ്ങള്‍ സ്ഥാപിക്കുകയും, അളവ് നിര്‍ണയിക്കപ്പെട്ട എല്ലാ വസ്തുക്കളും അതില്‍ നാം മുളപ്പിക്കുകയും ചെയ്തിരിക്കുന്നു.
\end{malayalam}}
\flushright{\begin{Arabic}
\quranayah[15][20]
\end{Arabic}}
\flushleft{\begin{malayalam}
നിങ്ങള്‍ക്കും, നിങ്ങള്‍ ആഹാരം നല്‍കിക്കൊണ്ടിരിക്കുന്നവരല്ലാത്തവര്‍ക്കും അതില്‍ നാം ഉപജീവനമാര്‍ഗങ്ങള്‍ ഏര്‍പെടുത്തുകയും ചെയ്തിരിക്കുന്നു.
\end{malayalam}}
\flushright{\begin{Arabic}
\quranayah[15][21]
\end{Arabic}}
\flushleft{\begin{malayalam}
യാതൊരു വസ്തുവും നമ്മുടെ പക്കല്‍ അതിന്‍റെ ഖജനാവുകള്‍ ഉള്ളതായിട്ടല്ലാതെയില്ല. (എന്നാല്‍) ഒരു നിര്‍ണിതമായ തോതനുസരിച്ചല്ലാതെ നാമത് ഇറക്കുന്നതല്ല.
\end{malayalam}}
\flushright{\begin{Arabic}
\quranayah[15][22]
\end{Arabic}}
\flushleft{\begin{malayalam}
മേഘങ്ങളുല്‍പാദിപ്പിക്കുന്ന കാറ്റുകളെ നാം അയക്കുകയും, എന്നിട്ട് ആകാശത്ത് നിന്ന് വെള്ളം ചൊരിഞ്ഞുതരികയും, എന്നിട്ട് നിങ്ങള്‍ക്ക് അത് കുടിക്കുമാറാക്കുകയും ചെയ്തു. നിങ്ങള്‍ക്കത് സംഭരിച്ച് വെക്കാന്‍ കഴിയുമായിരുന്നില്ല.
\end{malayalam}}
\flushright{\begin{Arabic}
\quranayah[15][23]
\end{Arabic}}
\flushleft{\begin{malayalam}
തീര്‍ച്ചയായും, നാം തന്നെയാണ് ജീവിപ്പിക്കുകയും മരിപ്പിക്കുകയും ചെയ്യുന്നത്‌. (എല്ലാറ്റിന്‍റെയും) അനന്തരാവകാശിയും നാം തന്നെയാണ്‌.
\end{malayalam}}
\flushright{\begin{Arabic}
\quranayah[15][24]
\end{Arabic}}
\flushleft{\begin{malayalam}
തീര്‍ച്ചയായും നിങ്ങളില്‍ നിന്ന് മുമ്പിലായവര്‍ ആരെന്ന് നാം അറിഞ്ഞിട്ടുണ്ട്‌. പിന്നിലായവര്‍ ആരെന്നും നാം അറിഞ്ഞിട്ടുണ്ട്‌.
\end{malayalam}}
\flushright{\begin{Arabic}
\quranayah[15][25]
\end{Arabic}}
\flushleft{\begin{malayalam}
തീര്‍ച്ചയായും നിന്‍റെ രക്ഷിതാവ് തന്നെ അവരെ ഒരുമിച്ചുകൂട്ടുന്നതുമാണ്‌. തീര്‍ച്ചയായും അവന്‍ യുക്തിമാനും സര്‍വ്വജ്ഞനുമത്രെ.
\end{malayalam}}
\flushright{\begin{Arabic}
\quranayah[15][26]
\end{Arabic}}
\flushleft{\begin{malayalam}
കറുത്ത ചെളി പാകപ്പെടുത്തിയുണ്ടാക്കിയ (മുട്ടിയാല്‍) മുഴക്കമുണ്ടാക്കുന്ന കളിമണ്‍ രൂപത്തില്‍ നിന്ന് നാം മനുഷ്യനെ സൃഷ്ടിച്ചിരിക്കുന്നു.
\end{malayalam}}
\flushright{\begin{Arabic}
\quranayah[15][27]
\end{Arabic}}
\flushleft{\begin{malayalam}
അതിന്നു മുമ്പ് ജിന്നിനെ അത്യുഷ്ണമുള്ള അഗ്നിജ്വാലയില്‍ നിന്നു നാം സൃഷ്ടിച്ചു.
\end{malayalam}}
\flushright{\begin{Arabic}
\quranayah[15][28]
\end{Arabic}}
\flushleft{\begin{malayalam}
നിന്‍റെ രക്ഷിതാവ് മലക്കുകളോട് ഇപ്രകാരം പറഞ്ഞ സന്ദര്‍ഭം ശ്രദ്ധേയമാകുന്നു: കറുത്ത ചെളി പാകപ്പെടുത്തിയുണ്ടാക്കിയ മുഴക്കമുണ്ടാക്കുന്ന കളിമണ്‍ രൂപത്തില്‍ നിന്ന് ഞാന്‍ ഒരു മനുഷ്യനെ സൃഷ്ടിക്കാന്‍ പോകുകയാണ്‌.
\end{malayalam}}
\flushright{\begin{Arabic}
\quranayah[15][29]
\end{Arabic}}
\flushleft{\begin{malayalam}
അങ്ങനെ ഞാന്‍ അവനെ ശരിയായ രൂപത്തിലാക്കുകയും, എന്‍റെ ആത്മാവില്‍ നിന്ന് അവനില്‍ ഞാന്‍ ഊതുകയും ചെയ്താല്‍, അപ്പോള്‍ അവന്ന് പ്രണമിക്കുന്നവരായിക്കൊണ്ട് നിങ്ങള്‍ വീഴുവിന്‍.
\end{malayalam}}
\flushright{\begin{Arabic}
\quranayah[15][30]
\end{Arabic}}
\flushleft{\begin{malayalam}
അപ്പോള്‍ മലക്കുകള്‍ എല്ലാവരും പ്രണമിച്ചു.
\end{malayalam}}
\flushright{\begin{Arabic}
\quranayah[15][31]
\end{Arabic}}
\flushleft{\begin{malayalam}
ഇബ്ലീസ് ഒഴികെ. പ്രണമിക്കുന്നവരുടെ കൂട്ടത്തിലായിരിക്കാന്‍ അവന്‍ വിസമ്മതിച്ചു.
\end{malayalam}}
\flushright{\begin{Arabic}
\quranayah[15][32]
\end{Arabic}}
\flushleft{\begin{malayalam}
അല്ലാഹു പറഞ്ഞു: ഇബ്ലീസേ, പ്രണമിക്കുന്നവരുടെ കൂട്ടത്തില്‍ ചേരാതിരിക്കുവാന്‍ നിനക്കെന്താണ് ന്യായം?
\end{malayalam}}
\flushright{\begin{Arabic}
\quranayah[15][33]
\end{Arabic}}
\flushleft{\begin{malayalam}
അവന്‍ പറഞ്ഞു : കറുത്ത ചെളി പാകപ്പെടുത്തിയുണ്ടാക്കിയ (മുട്ടിയാല്‍) മുഴക്കമുണ്ടാക്കുന്ന കളിമണ്‍ രൂപത്തില്‍ നിന്ന് നീ സൃഷ്ടിച്ച മനുഷ്യന് ഞാന്‍ പ്രണമിക്കേണ്ടവനല്ല.
\end{malayalam}}
\flushright{\begin{Arabic}
\quranayah[15][34]
\end{Arabic}}
\flushleft{\begin{malayalam}
അവന്‍ പറഞ്ഞു: നീ ഇവിടെ നിന്ന് പുറത്ത് പോ. തീര്‍ച്ചയായും നീ ആട്ടിയകറ്റപ്പെട്ടവനാകുന്നു.
\end{malayalam}}
\flushright{\begin{Arabic}
\quranayah[15][35]
\end{Arabic}}
\flushleft{\begin{malayalam}
തീര്‍ച്ചയായും ന്യായവിധിയുടെ നാള്‍ വരെയും നിന്‍റെ മേല്‍ ശാപമുണ്ടായിരിക്കുന്നതാണ്‌.
\end{malayalam}}
\flushright{\begin{Arabic}
\quranayah[15][36]
\end{Arabic}}
\flushleft{\begin{malayalam}
അവന്‍ പറഞ്ഞു: എന്‍റെ രക്ഷിതാവേ, അവര്‍ ഉയിര്‍ത്തെഴുന്നേല്‍പിക്കപ്പെടുന്ന ദിവസം വരെ എനിക്ക് നീ അവധി നീട്ടിത്തരേണമേ.
\end{malayalam}}
\flushright{\begin{Arabic}
\quranayah[15][37]
\end{Arabic}}
\flushleft{\begin{malayalam}
അല്ലാഹു പറഞ്ഞു: എന്നാല്‍ തീര്‍ച്ചയായും നീ അവധി നല്‍കപ്പെടുന്നവരുടെ കൂട്ടത്തില്‍ തന്നെയായിരിക്കും.
\end{malayalam}}
\flushright{\begin{Arabic}
\quranayah[15][38]
\end{Arabic}}
\flushleft{\begin{malayalam}
ആ നിശ്ചിത സന്ദര്‍ഭം വന്നെത്തുന്ന ദിവസം വരെ.
\end{malayalam}}
\flushright{\begin{Arabic}
\quranayah[15][39]
\end{Arabic}}
\flushleft{\begin{malayalam}
അവന്‍ പറഞ്ഞു: എന്‍റെ രക്ഷിതാവേ, നീ എന്നെ വഴികേടിലാക്കിയതിനാല്‍, ഭൂലോകത്ത് അവര്‍ക്കു ഞാന്‍ (ദുഷ്പ്രവൃത്തികള്‍) അലംകൃതമായി തോന്നിക്കുകയും, അവരെ മുഴുവന്‍ ഞാന്‍ വഴികേടിലാക്കുകയും ചെയ്യും; തീര്‍ച്ച.
\end{malayalam}}
\flushright{\begin{Arabic}
\quranayah[15][40]
\end{Arabic}}
\flushleft{\begin{malayalam}
അവരുടെ കൂട്ടത്തില്‍ നിന്ന് നിന്‍റെ നിഷ്കളങ്കരായ ദാസന്‍മാരൊഴികെ.
\end{malayalam}}
\flushright{\begin{Arabic}
\quranayah[15][41]
\end{Arabic}}
\flushleft{\begin{malayalam}
അവന്‍ (അല്ലാഹു) പറഞ്ഞു: എന്നിലേക്ക് നേര്‍ക്കുനേരെയുള്ള മാര്‍ഗമാകുന്നു ഇത്‌.
\end{malayalam}}
\flushright{\begin{Arabic}
\quranayah[15][42]
\end{Arabic}}
\flushleft{\begin{malayalam}
തീര്‍ച്ചയായും എന്‍റെ ദാസന്‍മാരുടെ മേല്‍ നിനക്ക് യാതൊരു ആധിപത്യവുമില്ല. നിന്നെ പിന്‍പറ്റിയ ദുര്‍മാര്‍ഗികളുടെ മേലല്ലാതെ.
\end{malayalam}}
\flushright{\begin{Arabic}
\quranayah[15][43]
\end{Arabic}}
\flushleft{\begin{malayalam}
തീര്‍ച്ചയായും നരകം അവര്‍ക്കെല്ലാം നിശ്ചയിക്കപ്പെട്ട സ്ഥാനം തന്നെയാകുന്നു.
\end{malayalam}}
\flushright{\begin{Arabic}
\quranayah[15][44]
\end{Arabic}}
\flushleft{\begin{malayalam}
അതിന് ഏഴ് കവാടങ്ങളുണ്ട്‌. ഓരോ വാതിലിലൂടെയും കടക്കുവാനായി വീതിക്കപ്പെട്ട ഓരോ വിഭാഗം അവരിലുണ്ട്‌.
\end{malayalam}}
\flushright{\begin{Arabic}
\quranayah[15][45]
\end{Arabic}}
\flushleft{\begin{malayalam}
തീര്‍ച്ചയായും സൂക്ഷ്മത പാലിച്ചവര്‍ തോട്ടങ്ങളിലും അരുവികളിലുമായിരിക്കും.
\end{malayalam}}
\flushright{\begin{Arabic}
\quranayah[15][46]
\end{Arabic}}
\flushleft{\begin{malayalam}
നിര്‍ഭയരായി ശാന്തിയോടെ അതില്‍ പ്രവേശിച്ച് കൊള്ളുക. (എന്ന് അവര്‍ക്ക് സ്വാഗതം ആശംസിക്കപ്പെടും.)
\end{malayalam}}
\flushright{\begin{Arabic}
\quranayah[15][47]
\end{Arabic}}
\flushleft{\begin{malayalam}
അവരുടെ ഹൃദയങ്ങളില്‍ വല്ല വിദ്വേഷവുമുണ്ടെങ്കില്‍ നാമത് നീക്കം ചെയ്യുന്നതാണ്‌. സഹോദരങ്ങളെന്ന നിലയില്‍ അവര്‍ കട്ടിലുകളില്‍ പരസ്പരം അഭിമുഖമായി ഇരിക്കുന്നവരായിരിക്കും.
\end{malayalam}}
\flushright{\begin{Arabic}
\quranayah[15][48]
\end{Arabic}}
\flushleft{\begin{malayalam}
അവിടെവെച്ച് യാതൊരു ക്ഷീണവും അവരെ ബാധിക്കുന്നതല്ല. അവിടെ നിന്ന് അവര്‍ പുറത്താക്കപ്പെടുന്നതുമല്ല.
\end{malayalam}}
\flushright{\begin{Arabic}
\quranayah[15][49]
\end{Arabic}}
\flushleft{\begin{malayalam}
(നബിയേ,) ഞാന്‍ ഏറെ പൊറുക്കുന്നവനും കരുണാനിധിയുമാണ് എന്ന് എന്‍റെ ദാസന്‍മാരെ വിവരമറിയിക്കുക.
\end{malayalam}}
\flushright{\begin{Arabic}
\quranayah[15][50]
\end{Arabic}}
\flushleft{\begin{malayalam}
എന്‍റെ ശിക്ഷ തന്നെയാണ് വേദനയേറിയ ശിക്ഷ എന്നും (വിവരമറിയിക്കുക.)
\end{malayalam}}
\flushright{\begin{Arabic}
\quranayah[15][51]
\end{Arabic}}
\flushleft{\begin{malayalam}
ഇബ്രാഹീമിന്‍റെ (അടുത്ത് വന്ന) അതിഥികളെപ്പറ്റി നീ അവരെ വിവരമറിയിക്കുക.
\end{malayalam}}
\flushright{\begin{Arabic}
\quranayah[15][52]
\end{Arabic}}
\flushleft{\begin{malayalam}
അദ്ദേഹത്തിന്‍റെ അടുത്ത് കടന്ന് വന്ന് അവര്‍ സലാം എന്ന് പറഞ്ഞ സന്ദര്‍ഭം. അദ്ദേഹം പറഞ്ഞു: തീര്‍ച്ചയായും ഞങ്ങള്‍ നിങ്ങളെപ്പറ്റി ഭയമുള്ളവരാകുന്നു.
\end{malayalam}}
\flushright{\begin{Arabic}
\quranayah[15][53]
\end{Arabic}}
\flushleft{\begin{malayalam}
അവര്‍ പറഞ്ഞു: താങ്കള്‍ ഭയപ്പെടേണ്ട. ജ്ഞാനിയായ ഒരു ആണ്‍കുട്ടിയെപ്പറ്റി ഞങ്ങളിതാ താങ്കള്‍ക്കു സന്തോഷവാര്‍ത്ത അറിയിക്കുന്നു.
\end{malayalam}}
\flushright{\begin{Arabic}
\quranayah[15][54]
\end{Arabic}}
\flushleft{\begin{malayalam}
അദ്ദേഹം പറഞ്ഞു: എനിക്ക് വാര്‍ദ്ധക്യം ബാധിച്ചു കഴിഞ്ഞിട്ടാണോ എനിക്ക് നിങ്ങള്‍ (സന്താനത്തെപറ്റി) സന്തോഷവാര്‍ത്ത അറിയിക്കുന്നത്‌? അപ്പോള്‍ എന്തൊന്നിനെപ്പറ്റിയാണ് നിങ്ങളീ സന്തോഷവാര്‍ത്ത അറിയിക്കുന്നത്‌?
\end{malayalam}}
\flushright{\begin{Arabic}
\quranayah[15][55]
\end{Arabic}}
\flushleft{\begin{malayalam}
അവര്‍ പറഞ്ഞു: ഞങ്ങള്‍ താങ്കള്‍ക്ക് സന്തോഷവാര്‍ത്ത നല്‍കിയിട്ടുള്ളത് ഒരു യാഥാര്‍ത്ഥ്യത്തെപറ്റിതന്നെയാണ്‌. അതിനാല്‍ താങ്കള്‍ നിരാശരുടെ കൂട്ടത്തിലായിരിക്കരുത്‌.
\end{malayalam}}
\flushright{\begin{Arabic}
\quranayah[15][56]
\end{Arabic}}
\flushleft{\begin{malayalam}
അദ്ദേഹം (ഇബ്രാഹീം) പറഞ്ഞു: തന്‍റെ രക്ഷിതാവിന്‍റെ കാരുണ്യത്തെപ്പറ്റി ആരാണ് നിരാശപ്പെടുക? വഴിപിഴച്ചവരല്ലാതെ.
\end{malayalam}}
\flushright{\begin{Arabic}
\quranayah[15][57]
\end{Arabic}}
\flushleft{\begin{malayalam}
അദ്ദേഹം (ഇബ്രാഹീം) പറഞ്ഞു: ഹേ; ദൂതന്‍മാരേ, എന്നാല്‍ നിങ്ങളുടെ (മുഖ്യ) വിഷയമെന്താണ്‌?
\end{malayalam}}
\flushright{\begin{Arabic}
\quranayah[15][58]
\end{Arabic}}
\flushleft{\begin{malayalam}
അവര്‍ പറഞ്ഞു: ഞങ്ങള്‍ കുറ്റവാളികളായ ഒരു ജനതയിലേക്ക് അയക്കപ്പെട്ടിരിക്കുകയാണ്‌.
\end{malayalam}}
\flushright{\begin{Arabic}
\quranayah[15][59]
\end{Arabic}}
\flushleft{\begin{malayalam}
(എന്നാല്‍) ലൂത്വിന്‍റെ കുടുംബം അതില്‍ നിന്നൊഴിവാണ്‌. തീര്‍ച്ചയായും അവരെ മുഴുവന്‍ ഞങ്ങള്‍ രക്ഷപ്പെടുത്തുന്നതാണ്‌.
\end{malayalam}}
\flushright{\begin{Arabic}
\quranayah[15][60]
\end{Arabic}}
\flushleft{\begin{malayalam}
അദ്ദേഹത്തിന്‍റെ ഭാര്യ ഒഴികെ. തീര്‍ച്ചയായും അവള്‍ ശിക്ഷയില്‍ അകപ്പെടുന്നവരുടെ കൂട്ടത്തിലാണെന്ന് ഞങ്ങള്‍ കണക്കാക്കിയിരിക്കുന്നു.
\end{malayalam}}
\flushright{\begin{Arabic}
\quranayah[15][61]
\end{Arabic}}
\flushleft{\begin{malayalam}
അങ്ങനെ ലൂത്വിന്‍റെ കുടുംബത്തില്‍ ആ ദൂതന്‍മാര്‍ വന്നെത്തിയപ്പോള്‍.
\end{malayalam}}
\flushright{\begin{Arabic}
\quranayah[15][62]
\end{Arabic}}
\flushleft{\begin{malayalam}
അദ്ദേഹം പറഞ്ഞു: തീര്‍ച്ചയായും നിങ്ങള്‍ അപരിചിതരായ ആളുകളാണല്ലോ.
\end{malayalam}}
\flushright{\begin{Arabic}
\quranayah[15][63]
\end{Arabic}}
\flushleft{\begin{malayalam}
അവര്‍ (ആ ദൂതന്‍മാരായ മലക്കുകള്‍) പറഞ്ഞു: അല്ല, ഏതൊരു (ശിക്ഷയുടെ) കാര്യത്തില്‍ അവര്‍ (ജനങ്ങള്‍) സംശയിച്ചിരുന്നുവോ അതും കൊണ്ടാണ് ഞങ്ങള്‍ താങ്കളുടെ അടുത്ത് വന്നിരിക്കുന്നത്‌.
\end{malayalam}}
\flushright{\begin{Arabic}
\quranayah[15][64]
\end{Arabic}}
\flushleft{\begin{malayalam}
യാഥാര്‍ത്ഥ്യവും കൊണ്ടാണ് ഞങ്ങള്‍ താങ്കളുടെ അടുത്ത് വന്നിരിക്കുന്നത്‌. തീര്‍ച്ചയായും ഞങ്ങള്‍ സത്യം പറയുന്നവരാകുന്നു.
\end{malayalam}}
\flushright{\begin{Arabic}
\quranayah[15][65]
\end{Arabic}}
\flushleft{\begin{malayalam}
അതിനാല്‍ താങ്കളുടെ കുടുംബത്തെയും കൊണ്ട് രാത്രിയില്‍ അല്‍പസമയം ബാക്കിയുള്ളപ്പോള്‍ യാത്രചെയ്ത് കൊള്ളുക. താങ്കള്‍ അവരുടെ പിന്നാലെ അനുഗമിക്കുകയും ചെയ്യുക. നിങ്ങളില്‍ ഒരാളും തിരിഞ്ഞ് നോക്കരുത്‌. നിങ്ങള്‍ കല്‍പിക്കപ്പെടുന്ന ഭാഗത്തേക്ക് നടന്ന് പോയിക്കൊള്ളുക.
\end{malayalam}}
\flushright{\begin{Arabic}
\quranayah[15][66]
\end{Arabic}}
\flushleft{\begin{malayalam}
ആ കാര്യം, അതായത് പ്രഭാതമാകുന്നതോടെ ഇക്കൂട്ടരുടെ മുരടുതന്നെ മുറിച്ചുനീക്കപ്പെടുന്നതാണ് എന്ന കാര്യം നാം അദ്ദേഹത്തിന് (ലൂത്വ് നബിക്ക്‌) ഖണ്ഡിതമായി അറിയിച്ച് കൊടുത്തു.
\end{malayalam}}
\flushright{\begin{Arabic}
\quranayah[15][67]
\end{Arabic}}
\flushleft{\begin{malayalam}
രാജ്യക്കാര്‍ സന്തോഷം പ്രകടിപ്പിച്ചു കൊണ്ട് വന്നു.
\end{malayalam}}
\flushright{\begin{Arabic}
\quranayah[15][68]
\end{Arabic}}
\flushleft{\begin{malayalam}
അദ്ദേഹം (ലൂത്വ്‌) പറഞ്ഞു: തീര്‍ച്ചയായും ഇവര്‍ എന്‍റെ അതിഥികളാണ്‌. അതിനാല്‍ നിങ്ങളെന്നെ വഷളാക്കരുത്‌.
\end{malayalam}}
\flushright{\begin{Arabic}
\quranayah[15][69]
\end{Arabic}}
\flushleft{\begin{malayalam}
നിങ്ങള്‍ അല്ലാഹുവെ സൂക്ഷിക്കുകയും, എന്നെ അപമാനിക്കാതിരിക്കുകയും ചെയ്യുക.
\end{malayalam}}
\flushright{\begin{Arabic}
\quranayah[15][70]
\end{Arabic}}
\flushleft{\begin{malayalam}
അവര്‍ പറഞ്ഞു: ലോകരുടെ കാര്യത്തില്‍ (ഇടപെടുന്നതില്‍) നിന്നു നിന്നെ ഞങ്ങള്‍ വിലക്കിയിട്ടില്ലേ?
\end{malayalam}}
\flushright{\begin{Arabic}
\quranayah[15][71]
\end{Arabic}}
\flushleft{\begin{malayalam}
അദ്ദേഹം പറഞ്ഞു: ഇതാ എന്‍റെ പെണ്‍മക്കള്‍. (അവരെ നിങ്ങള്‍ക്ക് വിവാഹം കഴിക്കാം.) നിങ്ങള്‍ക്ക് ചെയ്യാം എന്നുണ്ടെങ്കില്‍
\end{malayalam}}
\flushright{\begin{Arabic}
\quranayah[15][72]
\end{Arabic}}
\flushleft{\begin{malayalam}
നിന്‍റെ ജീവിതം തന്നെയാണ സത്യം തീര്‍ച്ചയായും അവര്‍ അവരുടെ ലഹരിയില്‍ വിഹരിക്കുകയായിരുന്നു.
\end{malayalam}}
\flushright{\begin{Arabic}
\quranayah[15][73]
\end{Arabic}}
\flushleft{\begin{malayalam}
അങ്ങനെ സൂര്യോദയത്തോടെ ആ ഘോരശബ്ദം അവരെ പിടികൂടി.
\end{malayalam}}
\flushright{\begin{Arabic}
\quranayah[15][74]
\end{Arabic}}
\flushleft{\begin{malayalam}
അങ്ങനെ ആ രാജ്യത്തെ നാം തലകീഴായി മറിക്കുകയും, ചുട്ടുപഴുത്ത ഇഷ്ടികക്കല്ലുകള്‍ അവരുടെ മേല്‍ നാം വര്‍ഷിക്കുകയും ചെയ്തു.
\end{malayalam}}
\flushright{\begin{Arabic}
\quranayah[15][75]
\end{Arabic}}
\flushleft{\begin{malayalam}
നിരീക്ഷിച്ച് മനസ്സിലാക്കുന്നവര്‍ക്ക് തീര്‍ച്ചയായും അതില്‍ പല ദൃഷ്ടാന്തങ്ങളുമുണ്ട്‌.
\end{malayalam}}
\flushright{\begin{Arabic}
\quranayah[15][76]
\end{Arabic}}
\flushleft{\begin{malayalam}
തീര്‍ച്ചയായും അത് (ആ രാജ്യം) (ഇന്നും) നിലനിന്ന് വരുന്ന ഒരു പാതയിലാണ് സ്ഥിതി ചെയ്യുന്നത്‌.
\end{malayalam}}
\flushright{\begin{Arabic}
\quranayah[15][77]
\end{Arabic}}
\flushleft{\begin{malayalam}
തീര്‍ച്ചയായും അതില്‍ വിശ്വാസികള്‍ക്ക് ഒരു ദൃഷ്ടാന്തമുണ്ട്‌.
\end{malayalam}}
\flushright{\begin{Arabic}
\quranayah[15][78]
\end{Arabic}}
\flushleft{\begin{malayalam}
തീര്‍ച്ചയായും മരക്കൂട്ടത്തില്‍ താമസിച്ചിരുന്ന ജനവിഭാഗം അക്രമികളായിരുന്നു.
\end{malayalam}}
\flushright{\begin{Arabic}
\quranayah[15][79]
\end{Arabic}}
\flushleft{\begin{malayalam}
അതിനാല്‍ നാം അവരുടെ നേരെ ശിക്ഷാനടപടി സ്വീകരിച്ചു. തീര്‍ച്ചയായും ഈ രണ്ട് പ്രദേശവും തുറന്ന പാതയില്‍ തന്നെയാണ് സ്ഥിതിചെയ്യുന്നത്‌.
\end{malayalam}}
\flushright{\begin{Arabic}
\quranayah[15][80]
\end{Arabic}}
\flushleft{\begin{malayalam}
തീര്‍ച്ചയായും ഹിജ്‌റിലെ നിവാസികള്‍ ദൈവദൂതന്‍മാരെ നിഷേധിച്ച് കളയുകയുണ്ടായി.
\end{malayalam}}
\flushright{\begin{Arabic}
\quranayah[15][81]
\end{Arabic}}
\flushleft{\begin{malayalam}
അവര്‍ക്ക് നാം നമ്മുടെ ദൃഷ്ടാന്തങ്ങള്‍ നല്‍കുകയും ചെയ്തു. എന്നിട്ട് അവര്‍ അവയെ അവഗണിച്ച് കളയുകയായിരുന്നു.
\end{malayalam}}
\flushright{\begin{Arabic}
\quranayah[15][82]
\end{Arabic}}
\flushleft{\begin{malayalam}
അവര്‍ പര്‍വ്വതങ്ങളില്‍ നിന്ന് (പാറകള്‍) വെട്ടിത്തുരന്ന് വീടുകളുണ്ടാക്കി നിര്‍ഭയരായി കഴിയുകയായിരുന്നു.
\end{malayalam}}
\flushright{\begin{Arabic}
\quranayah[15][83]
\end{Arabic}}
\flushleft{\begin{malayalam}
അങ്ങനെയിരിക്കെ പ്രഭാതവേളയില്‍ ഒരു ഘോരശബ്ദം അവരെ പിടികൂടി.
\end{malayalam}}
\flushright{\begin{Arabic}
\quranayah[15][84]
\end{Arabic}}
\flushleft{\begin{malayalam}
അപ്പോള്‍ അവര്‍ പ്രവര്‍ത്തിച്ചുണ്ടാക്കിയിരുന്നതൊന്നും അവര്‍ക്ക് ഉപകരിച്ചില്ല.
\end{malayalam}}
\flushright{\begin{Arabic}
\quranayah[15][85]
\end{Arabic}}
\flushleft{\begin{malayalam}
ആകാശങ്ങളും ഭൂമിയും അവ രണ്ടിനും ഇടയിലുള്ളതും യുക്തിപൂര്‍വ്വകമായല്ലാതെ നാം സൃഷ്ടിച്ചിട്ടില്ല. തീര്‍ച്ചയായും അന്ത്യസമയം വരുക തന്നെ ചെയ്യും. അതിനാല്‍ നീ ഭംഗിയായി മാപ്പ് ചെയ്ത് കൊടുക്കുക.
\end{malayalam}}
\flushright{\begin{Arabic}
\quranayah[15][86]
\end{Arabic}}
\flushleft{\begin{malayalam}
തീര്‍ച്ചയായും നിന്‍റെ രക്ഷിതാവ് എല്ലാം സൃഷ്ടിക്കുന്നവനും എല്ലാം അറിയുന്നവനുമാകുന്നു.
\end{malayalam}}
\flushright{\begin{Arabic}
\quranayah[15][87]
\end{Arabic}}
\flushleft{\begin{malayalam}
ആവര്‍ത്തിച്ചു പാരായണം ചെയ്യപ്പെടുന്ന ഏഴ് വചനങ്ങളും മഹത്തായ ഖുര്‍ആനും തീര്‍ച്ചയായും നിനക്ക് നാം നല്‍കിയിട്ടുണ്ട്‌.
\end{malayalam}}
\flushright{\begin{Arabic}
\quranayah[15][88]
\end{Arabic}}
\flushleft{\begin{malayalam}
അവരില്‍ (അവിശ്വാസികളില്‍) പെട്ട പല വിഭാഗക്കാര്‍ക്കും നാം സുഖഭോഗങ്ങള്‍ നല്‍കിയിട്ടുള്ളതിന്‍റെ നേര്‍ക്ക് നീ നിന്‍റെ ദൃഷ്ടികള്‍ നീട്ടിപ്പോകരുത്‌. അവരെപ്പറ്റി നീ വ്യസനിക്കേണ്ട. സത്യവിശ്വാസികള്‍ക്ക് നീ നിന്‍റെ ചിറക് താഴ്ത്തികൊടുക്കുക.
\end{malayalam}}
\flushright{\begin{Arabic}
\quranayah[15][89]
\end{Arabic}}
\flushleft{\begin{malayalam}
തീര്‍ച്ചയായും ഞാന്‍ വ്യക്തമായ ഒരു താക്കീതുകാരന്‍ തന്നെയാണ് എന്ന് പറയുകയും ചെയ്യുക.
\end{malayalam}}
\flushright{\begin{Arabic}
\quranayah[15][90]
\end{Arabic}}
\flushleft{\begin{malayalam}
വിഭജനം നടത്തിക്കളഞ്ഞവരുടെ മേല്‍ നാം ഇറക്കിയത് പോലെത്തന്നെ.
\end{malayalam}}
\flushright{\begin{Arabic}
\quranayah[15][91]
\end{Arabic}}
\flushleft{\begin{malayalam}
അതായത് ഖുര്‍ആനിനെ വ്യത്യസ്ത ഖണ്ഡങ്ങളാക്കി മാറ്റിയവരുടെ മേല്‍.
\end{malayalam}}
\flushright{\begin{Arabic}
\quranayah[15][92]
\end{Arabic}}
\flushleft{\begin{malayalam}
എന്നാല്‍ നിന്‍റെ രക്ഷിതാവിനെത്തന്നെയാണ, അവരെ മുഴുവന്‍ നാം ചോദ്യം ചെയ്യുക തന്നെ ചെയ്യും.
\end{malayalam}}
\flushright{\begin{Arabic}
\quranayah[15][93]
\end{Arabic}}
\flushleft{\begin{malayalam}
അവര്‍ പ്രവര്‍ത്തിച്ച് കൊണ്ടിരുന്നതിനെ സംബന്ധിച്ച്‌.
\end{malayalam}}
\flushright{\begin{Arabic}
\quranayah[15][94]
\end{Arabic}}
\flushleft{\begin{malayalam}
അതിനാല്‍ നീ കല്‍പിക്കപ്പെടുന്നതെന്തോ അത് ഉറക്കെ പ്രഖ്യാപിച്ച് കൊള്ളുക. ബഹുദൈവവാദികളില്‍ നിന്ന് തിരിഞ്ഞുകളയുകയും ചെയ്യുക.
\end{malayalam}}
\flushright{\begin{Arabic}
\quranayah[15][95]
\end{Arabic}}
\flushleft{\begin{malayalam}
പരിഹാസക്കാരില്‍ നിന്ന് നിന്നെ സംരക്ഷിക്കാന്‍ തീര്‍ച്ചയായും നാം മതിയായിരിക്കുന്നു.
\end{malayalam}}
\flushright{\begin{Arabic}
\quranayah[15][96]
\end{Arabic}}
\flushleft{\begin{malayalam}
അതായത് അല്ലാഹുവോടൊപ്പം മറ്റുദൈവത്തെ സ്ഥാപിക്കുന്നവര്‍ (പിന്നീട്‌) അവര്‍ അറിഞ്ഞ് കൊള്ളും.
\end{malayalam}}
\flushright{\begin{Arabic}
\quranayah[15][97]
\end{Arabic}}
\flushleft{\begin{malayalam}
അവര്‍ പറഞ്ഞുകൊണ്ടിരിക്കുന്നത് നിമിത്തം നിനക്ക് മനഃപ്രയാസം അനുഭവപ്പെടുന്നുണ്ട് എന്ന് തീര്‍ച്ചയായും നാം അറിയുന്നുണ്ട്‌.
\end{malayalam}}
\flushright{\begin{Arabic}
\quranayah[15][98]
\end{Arabic}}
\flushleft{\begin{malayalam}
ആകയാല്‍ നിന്‍റെ രക്ഷിതാവിനെ സ്തുതിച്ച് കൊണ്ട് നീ സ്തോത്രകീര്‍ത്തനം നടത്തുകയും, നീ സുജൂദ് ചെയ്യുന്നവരുടെ കൂട്ടത്തിലായിരിക്കുകയും ചെയ്യുക.
\end{malayalam}}
\flushright{\begin{Arabic}
\quranayah[15][99]
\end{Arabic}}
\flushleft{\begin{malayalam}
ഉറപ്പായ കാര്യം (മരണം) നിനക്ക് വന്നെത്തുന്നത് വരെ നീ നിന്‍റെ രക്ഷിതാവിനെ ആരാധിക്കുകയും ചെയ്യുക.
\end{malayalam}}
\chapter{\textmalayalam{നഹ്ല്‍ ( തേനീച്ച )}}
\begin{Arabic}
\Huge{\centerline{\basmalah}}\end{Arabic}
\flushright{\begin{Arabic}
\quranayah[16][1]
\end{Arabic}}
\flushleft{\begin{malayalam}
അല്ലാഹുവിന്‍റെ കല്‍പന വരാനായിരിക്കുന്നു, എന്നാല്‍ നിങ്ങളതിന് ധൃതികൂട്ടേണ്ട. അവര്‍ പങ്കുചേര്‍ക്കുന്നതില്‍ നിന്നെല്ലാം അല്ലാഹു എത്രയോ പരിശുദ്ധനും ഉന്നതനുമായിരിക്കുന്നു.
\end{malayalam}}
\flushright{\begin{Arabic}
\quranayah[16][2]
\end{Arabic}}
\flushleft{\begin{malayalam}
തന്‍റെ ദാസന്‍മാരില്‍ നിന്ന് താന്‍ ഉദ്ദേശിക്കുന്നവരുടെ മേല്‍ തന്‍റെ കല്‍പനപ്രകാരം (സത്യസന്ദേശമാകുന്ന) ചൈതന്യവും കൊണ്ട് മലക്കുകളെ അവന്‍ ഇറക്കുന്നു. ഞാനല്ലാതെ യാതൊരു ദൈവവുമില്ല. അതിനാല്‍ നിങ്ങളെന്നെ സൂക്ഷിച്ച് കൊള്ളുവിന്‍ എന്ന് നിങ്ങള്‍ താക്കീത് നല്‍കുക. (എന്നത്രെ ആ സന്ദേശം)
\end{malayalam}}
\flushright{\begin{Arabic}
\quranayah[16][3]
\end{Arabic}}
\flushleft{\begin{malayalam}
ആകാശങ്ങളും ഭൂമിയും അവന്‍ യുക്തിപൂര്‍വ്വം സൃഷ്ടിച്ചിരിക്കുന്നു. അവര്‍ പങ്കുചേര്‍ക്കുന്നതിനെല്ലാം അവന്‍ അതീതനായിരിക്കുന്നു.
\end{malayalam}}
\flushright{\begin{Arabic}
\quranayah[16][4]
\end{Arabic}}
\flushleft{\begin{malayalam}
മനുഷ്യനെ അവന്‍ ഒരു ബീജകണത്തില്‍ നിന്ന് സൃഷ്ടിച്ചു. എന്നിട്ട് അവനതാ വ്യക്തമായ എതിര്‍പ്പുകാരനായിരിക്കുന്നു.
\end{malayalam}}
\flushright{\begin{Arabic}
\quranayah[16][5]
\end{Arabic}}
\flushleft{\begin{malayalam}
കാലികളെയും അവന്‍ സൃഷ്ടിച്ചിരിക്കുന്നു; നിങ്ങള്‍ക്ക് അവയില്‍ തണുപ്പകറ്റാനുള്ളതും (കമ്പിളി) മറ്റു പ്രയോജനങ്ങളുമുണ്ട്‌. അവയില്‍ നിന്നു തന്നെ നിങ്ങള്‍ ഭക്ഷിക്കുകയും ചെയ്യുന്നു.
\end{malayalam}}
\flushright{\begin{Arabic}
\quranayah[16][6]
\end{Arabic}}
\flushleft{\begin{malayalam}
നിങ്ങള്‍ (വൈകുന്നേരം ആലയിലേക്ക്‌) തിരിച്ച് കൊണ്ട് വരുന്ന സമയത്തും, നിങ്ങള്‍ മേയാന്‍ വിടുന്ന സമയത്തും അവയില്‍ നിങ്ങള്‍ക്ക് കൌതുകമുണ്ട്‌.
\end{malayalam}}
\flushright{\begin{Arabic}
\quranayah[16][7]
\end{Arabic}}
\flushleft{\begin{malayalam}
ശാരീരിക ക്ലേശത്തോട് കൂടിയല്ലാതെ നിങ്ങള്‍ക്ക് ചെന്നെത്താനാകാത്ത നാട്ടിലേക്ക് അവ നിങ്ങളുടെ ഭാരങ്ങള്‍ വഹിച്ച് കൊണ്ട് പോകുകയും ചെയ്യുന്നു. തീര്‍ച്ചയായും നിങ്ങളുടെ രക്ഷിതാവ് ഏറെ ദയയുള്ളവനും കരുണാനിധിയുമാകുന്നു.
\end{malayalam}}
\flushright{\begin{Arabic}
\quranayah[16][8]
\end{Arabic}}
\flushleft{\begin{malayalam}
കുതിരകളെയും കോവര്‍കഴുതകളെയും, കഴുതകളെയും (അവന്‍ സൃഷ്ടിച്ചിരിക്കുന്നു.) അവയെ നിങ്ങള്‍ക്ക് വാഹനമായി ഉപയോഗിക്കുവാനും, അലങ്കാരത്തിന് വേണ്ടിയും. നിങ്ങള്‍ക്ക് അറിവില്ലാത്തതും അവന്‍ സൃഷ്ടിക്കുന്നു.
\end{malayalam}}
\flushright{\begin{Arabic}
\quranayah[16][9]
\end{Arabic}}
\flushleft{\begin{malayalam}
അല്ലാഹുവിന്‍റെ ബാധ്യതയാകുന്നു നേരായ മാര്‍ഗം (കാണിച്ചുതരിക) എന്നത്‌. അവയുടെ (മാര്‍ഗങ്ങളുടെ) കൂട്ടത്തില്‍ പിഴച്ചവയുമുണ്ട്‌. അവന്‍ ഉദ്ദേശിച്ചിരുന്നുവെങ്കില്‍ നിങ്ങളെയെല്ലാം അവന്‍ നേര്‍വഴിയിലാക്കുമായിരുന്നു.
\end{malayalam}}
\flushright{\begin{Arabic}
\quranayah[16][10]
\end{Arabic}}
\flushleft{\begin{malayalam}
അവനാണ് ആകാശത്ത് നിന്ന് വെള്ളം ചൊരിഞ്ഞുതന്നത്‌. അതില്‍ നിന്നാണ് നിങ്ങളുടെ കുടിനീര്‌. അതില്‍ നിന്നുതന്നെയാണ് നിങ്ങള്‍ (കാലികളെ) മേക്കുവാനുള്ള ചെടികളുമുണ്ടാകുന്നത്‌.
\end{malayalam}}
\flushright{\begin{Arabic}
\quranayah[16][11]
\end{Arabic}}
\flushleft{\begin{malayalam}
അത് (വെള്ളം) മൂലം ധാന്യവിളകളും, ഒലീവും, ഈന്തപ്പനയും, മുന്തിരികളും നിങ്ങള്‍ക്ക് മുളപ്പിച്ച് തരുന്നു. എല്ലാതരം ഫലവര്‍ഗങ്ങളും (അവന്‍ ഉല്‍പാദിപ്പിച്ച് തരുന്നു.) ചിന്തിക്കുന്ന ആളുകള്‍ക്ക് തീര്‍ച്ചയായും അതില്‍ ദൃഷ്ടാന്തമുണ്ട്‌.
\end{malayalam}}
\flushright{\begin{Arabic}
\quranayah[16][12]
\end{Arabic}}
\flushleft{\begin{malayalam}
രാവിനെയും പകലിനെയും സൂര്യനെയും ചന്ദ്രനെയും അവന്‍ നിങ്ങള്‍ക്ക് വിധേയമാക്കിത്തന്നിരിക്കുന്നു. നക്ഷത്രങ്ങളും അവന്‍റെ കല്‍പനയാല്‍ വിധേയമാക്കപ്പെട്ടത് തന്നെ. ചിന്തിക്കുന്ന ആളുകള്‍ക്ക് തീര്‍ച്ചയായും അതില്‍ ദൃഷ്ടാന്തങ്ങളുണ്ട്‌.
\end{malayalam}}
\flushright{\begin{Arabic}
\quranayah[16][13]
\end{Arabic}}
\flushleft{\begin{malayalam}
നിങ്ങള്‍ക്ക് വേണ്ടി ഭൂമിയില്‍ വ്യത്യസ്ത വര്‍ണങ്ങളില്‍ അവന്‍ സൃഷ്ടിച്ചുണ്ടാക്കിതന്നിട്ടുള്ളവയും (അവന്‍റെ കല്‍പനയ്ക്ക് വിധേയം തന്നെ.) ആലോചിച്ച് മനസ്സിലാക്കുന്ന ആളുകള്‍ക്ക് തീര്‍ച്ചയായും അതില്‍ ദൃഷ്ടാന്തങ്ങളുണ്ട്‌.
\end{malayalam}}
\flushright{\begin{Arabic}
\quranayah[16][14]
\end{Arabic}}
\flushleft{\begin{malayalam}
നിങ്ങള്‍ക്ക് പുതുമാംസം എടുത്ത് തിന്നുവാനും നിങ്ങള്‍ക്ക് അണിയാനുള്ള ആഭരണങ്ങള്‍ പുറത്തെടുക്കുവാനും പാകത്തില്‍ കടലിനെ വിധേയമാക്കിയവനും അവന്‍ തന്നെ. കപ്പലുകള്‍ അതിലൂടെ വെള്ളം പിളര്‍ന്ന് മാറ്റിക്കൊണ്ട് ഓടുന്നതും നിനക്ക് കാണാം. അവന്‍റെ അനുഗ്രഹത്തില്‍ നിന്ന് നിങ്ങള്‍ തേടുവാനും നിങ്ങള്‍ നന്ദികാണിക്കുവാനും വേണ്ടിയാണ്‌. (അവനത് നിങ്ങള്‍ക്ക് വിധേയമാക്കിത്തന്നത്‌.)
\end{malayalam}}
\flushright{\begin{Arabic}
\quranayah[16][15]
\end{Arabic}}
\flushleft{\begin{malayalam}
ഭൂമി നിങ്ങളെയും കൊണ്ട് ഇളകാതിരിക്കുവാനായി അതില്‍ ഉറച്ചുനില്‍ക്കുന്ന പര്‍വ്വതങ്ങള്‍ അവന്‍ സ്ഥാപിച്ചിരിക്കുന്നു. നിങ്ങള്‍ക്ക് വഴി കണ്ടെത്തുവാന്‍ വേണ്ടി നദികളും പാതകളും (അവന്‍ ഏര്‍പെടുത്തുകയും ചെയ്തിരിക്കുന്നു.)
\end{malayalam}}
\flushright{\begin{Arabic}
\quranayah[16][16]
\end{Arabic}}
\flushleft{\begin{malayalam}
(പുറമെ) പല വഴിയടയാളങ്ങളും ഉണ്ട്‌. നക്ഷത്രം മുഖേനയും അവര്‍ വഴി കണ്ടെത്തുന്നു.
\end{malayalam}}
\flushright{\begin{Arabic}
\quranayah[16][17]
\end{Arabic}}
\flushleft{\begin{malayalam}
അപ്പോള്‍, സൃഷ്ടിക്കുന്നവന്‍ സൃഷ്ടിക്കാത്തവരെപ്പോലെയാണോ? നിങ്ങളെന്താണ് ആലോചിച്ച് മനസ്സിലാക്കാത്തത്‌?
\end{malayalam}}
\flushright{\begin{Arabic}
\quranayah[16][18]
\end{Arabic}}
\flushleft{\begin{malayalam}
അല്ലാഹുവിന്‍റെ അനുഗ്രഹം നിങ്ങള്‍ എണ്ണുകയാണെങ്കില്‍ നിങ്ങള്‍ക്കതിന്‍റെ കണക്കെടുക്കാനാവില്ല. തീര്‍ച്ചയായും അല്ലാഹു ഏറെ പൊറുക്കുന്നവനും കരുണാനിധിയും തന്നെ.
\end{malayalam}}
\flushright{\begin{Arabic}
\quranayah[16][19]
\end{Arabic}}
\flushleft{\begin{malayalam}
നിങ്ങള്‍ രഹസ്യമാക്കുന്നതും, പരസ്യമാക്കുന്നതും അല്ലാഹു അറിയുന്നു.
\end{malayalam}}
\flushright{\begin{Arabic}
\quranayah[16][20]
\end{Arabic}}
\flushleft{\begin{malayalam}
അല്ലാഹുവിന് പുറമെ നിങ്ങള്‍ ആരെയൊക്കെ വിളിച്ച് പ്രാര്‍ത്ഥിച്ച് കൊണ്ടിരിക്കുന്നുവോ അവര്‍ യാതൊന്നും സൃഷ്ടിക്കുന്നില്ല. അവരാകട്ടെ സൃഷ്ടിക്കപ്പെടുന്നവരുമാണ്‌.
\end{malayalam}}
\flushright{\begin{Arabic}
\quranayah[16][21]
\end{Arabic}}
\flushleft{\begin{malayalam}
അവര്‍ (പ്രാര്‍ത്ഥിക്കപ്പെടുന്നവര്‍) മരിച്ചവരാണ്‌. ജീവനുള്ളവരല്ല. ഏത് സമയത്താണ് അവര്‍ ഉയിര്‍ത്തെഴുന്നേല്‍പിക്കപ്പെടുക എന്ന് അവര്‍ അറിയുന്നുമില്ല.
\end{malayalam}}
\flushright{\begin{Arabic}
\quranayah[16][22]
\end{Arabic}}
\flushleft{\begin{malayalam}
നിങ്ങളുടെ ദൈവം ഏകദൈവമത്രെ. എന്നാല്‍ പരലോകത്തില്‍ വിശ്വസിക്കാത്തവരാകട്ടെ, അവരുടെ ഹൃദയങ്ങള്‍ നിഷേധസ്വഭാവമുള്ളവയത്രെ. അവര്‍ അഹങ്കാരികളുമാകുന്നു.
\end{malayalam}}
\flushright{\begin{Arabic}
\quranayah[16][23]
\end{Arabic}}
\flushleft{\begin{malayalam}
അവര്‍ രഹസ്യമാക്കുന്നതും പരസ്യമാക്കുന്നതും അല്ലാഹു അറിയുന്നു എന്നതില്‍ യാതൊരു സംശയവുമില്ല. അവന്‍ അഹങ്കാരികളെ ഇഷ്ടപ്പെടുകയില്ല; തീര്‍ച്ച.
\end{malayalam}}
\flushright{\begin{Arabic}
\quranayah[16][24]
\end{Arabic}}
\flushleft{\begin{malayalam}
നിങ്ങളുടെ രക്ഷിതാവ് എന്താണ് അവതരിപ്പിച്ചിരിക്കുന്നത് എന്ന് അവരോട് ചോദിക്കപ്പെട്ടാല്‍ അവര്‍ പറയും. പൂര്‍വ്വികന്‍മാരുടെ പുരാണ കഥകള്‍ തന്നെ.
\end{malayalam}}
\flushright{\begin{Arabic}
\quranayah[16][25]
\end{Arabic}}
\flushleft{\begin{malayalam}
തങ്ങളുടെ പാപഭാരങ്ങള്‍ മുഴുവനായിട്ടും, യാതൊരു വിവരവുമില്ലാതെ തങ്ങള്‍ ആരെയെല്ലാം വഴിപിഴപ്പിച്ച് കൊണ്ടിരിക്കുന്നുവോ അവരുടെ പാപഭാരങ്ങളില്‍ ഒരു ഭാഗവും ഉയിര്‍ത്തെഴുന്നേല്‍പിന്‍റെ നാളില്‍ അവര്‍ വഹിക്കുവാനത്രെ (അത് ഇടയാക്കുക.) ശ്രദ്ധിക്കുക: അവര്‍ പേറുന്ന ആ ഭാരം എത്ര മോശം!
\end{malayalam}}
\flushright{\begin{Arabic}
\quranayah[16][26]
\end{Arabic}}
\flushleft{\begin{malayalam}
അവരുടെ മുമ്പുള്ളവരും തന്ത്രം പ്രയോഗിച്ചിട്ടുണ്ട്‌. അപ്പോള്‍ അവര്‍ കെട്ടിപൊക്കിയതിന്‍റെ അടിത്തറകള്‍ക്ക് തന്നെ അല്ലാഹു നാശം വരുത്തി. അങ്ങനെ അവരുടെ മുകള്‍ ഭാഗത്ത് നിന്ന് മേല്‍ക്കൂര അവരുടെ മേല്‍ പൊളിഞ്ഞുവീണു. അവര്‍ ഓര്‍ക്കാത്ത ഭാഗത്ത് നിന്ന് ശിക്ഷ അവര്‍ക്ക് വരികയും ചെയ്തു.
\end{malayalam}}
\flushright{\begin{Arabic}
\quranayah[16][27]
\end{Arabic}}
\flushleft{\begin{malayalam}
പിന്നെ ഉയിര്‍ത്തെഴുന്നേല്‍പിന്‍റെ നാളില്‍ അവന്‍ അവര്‍ക്ക് അപമാനം വരുത്തുന്നതാണ്‌. എനിക്ക് പങ്കുകാരുണ്ടെന്ന് വാദിച്ച് കൊണ്ടായിരുന്നല്ലോ നിങ്ങള്‍ ചേരി പിരിഞ്ഞ് നിന്നിരുന്നത് അവര്‍ എവിടെ? എന്ന് അവന്‍ ചോദിക്കുകയും ചെയ്യും. അറിവ് നല്‍കപ്പെട്ടവര്‍ പറയും: ഇന്ന് അപമാനവും ശിക്ഷയും സത്യനിഷേധികള്‍ക്കാകുന്നു; തീര്‍ച്ച.
\end{malayalam}}
\flushright{\begin{Arabic}
\quranayah[16][28]
\end{Arabic}}
\flushleft{\begin{malayalam}
അതായത് അവരവര്‍ക്കു തന്നെ ദ്രോഹം ചെയ്തുകൊണ്ടിരിക്കെ മലക്കുകള്‍ ഏതൊരു കൂട്ടരുടെ ജീവിതം അവസാനിപ്പിക്കുന്നുവോ അവര്‍ക്ക്‌. ഞങ്ങള്‍ യാതൊരു തിന്‍മയും ചെയ്തിരുന്നില്ല എന്ന് പറഞ്ഞ് കൊണ്ട് അന്നേരം അവര്‍ കീഴ്‌വണക്കത്തിന് സന്നദ്ധത പ്രകടിപ്പിക്കും അങ്ങനെയല്ല, തീര്‍ച്ചയായും അല്ലാഹു നിങ്ങള്‍ പ്രവര്‍ത്തിച്ച് കൊണ്ടിരിക്കുന്നതിനെപ്പറ്റി അറിയുന്നവനാകുന്നു.
\end{malayalam}}
\flushright{\begin{Arabic}
\quranayah[16][29]
\end{Arabic}}
\flushleft{\begin{malayalam}
അതിനാല്‍ നരകത്തിന്‍റെ കവാടങ്ങളിലൂടെ നിങ്ങള്‍ കടന്ന് കൊള്ളുക. (നിങ്ങള്‍) അതില്‍ നിത്യവാസികളായിരിക്കും. അപ്പോള്‍ അഹങ്കാരികളുടെ വാസസ്ഥലം മോശം തന്നെ!
\end{malayalam}}
\flushright{\begin{Arabic}
\quranayah[16][30]
\end{Arabic}}
\flushleft{\begin{malayalam}
നിങ്ങളുടെ രക്ഷിതാവ് എന്താണ് അവതരിപ്പിച്ചിട്ടുള്ളത് എന്ന് സൂക്ഷ്മത പാലിച്ചവരോട് ചോദിക്കപ്പെട്ടു. അവര്‍ പറഞ്ഞു: ഉത്തമമായത് തന്നെ. നല്ലത് ചെയ്തവര്‍ക്ക് ഈ ദുന്‍യാവില്‍തന്നെ നല്ല ഫലമുണ്ട്‌. പരലോകഭവനമാകട്ടെ കൂടുതല്‍ ഉത്തമമാകുന്നു. സൂക്ഷ്മത പാലിക്കുന്നവര്‍ക്കുള്ള ഭവനം എത്രയോ നല്ലത്‌!
\end{malayalam}}
\flushright{\begin{Arabic}
\quranayah[16][31]
\end{Arabic}}
\flushleft{\begin{malayalam}
അതെ, അവര്‍ പ്രവേശിക്കുന്ന സ്ഥിരവാസത്തിനുള്ള സ്വര്‍ഗത്തോപ്പുകള്‍. അവയുടെ താഴ്ഭാഗത്ത് കൂടി അരുവികള്‍ ഒഴുകിക്കൊണ്ടിരിക്കും. അവര്‍ക്ക് അവര്‍ ഉദ്ദേശിക്കുന്നതെന്തും അതില്‍ ഉണ്ടായിരിക്കും. അപ്രകാരമാണ് സൂക്ഷ്മത പാലിക്കുന്നവര്‍ക്ക് അല്ലാഹു പ്രതിഫലം നല്‍കുന്നത്‌.
\end{malayalam}}
\flushright{\begin{Arabic}
\quranayah[16][32]
\end{Arabic}}
\flushleft{\begin{malayalam}
അതായത്‌, നല്ലവരായിരിക്കെ മലക്കുകള്‍ ഏതൊരു കൂട്ടരുടെ ജീവിതം അവസാനിപ്പിക്കുന്നുവോ അവര്‍ക്ക്‌. അവര്‍ (മലക്കുകള്‍) പറയും: നിങ്ങള്‍ക്ക് സമാധാനം. നിങ്ങള്‍ പ്രവര്‍ത്തിച്ച് കൊണ്ടിരുന്നതിന്‍റെ ഫലമായി നിങ്ങള്‍ സ്വര്‍ഗത്തില്‍ പ്രവേശിച്ച് കൊള്ളുക.
\end{malayalam}}
\flushright{\begin{Arabic}
\quranayah[16][33]
\end{Arabic}}
\flushleft{\begin{malayalam}
തങ്ങളുടെ അടുക്കല്‍ മലക്കുകള്‍ വരുന്നതോ, നിന്‍റെ രക്ഷിതാവിന്‍റെ കല്‍പന വരുന്നതോ അല്ലാതെ (മറ്റുവല്ലതും) അവര്‍ കാത്തിരിക്കുന്നുവോ ? അപ്രകാരം തന്നെയാണ് അവര്‍ക്ക് മുമ്പുള്ളവരും ചെയ്തത്‌. അല്ലാഹു അവരോട് അക്രമം ചെയ്തിട്ടില്ല. പക്ഷെ, അവര്‍ അവരോട് തന്നെ അക്രമം ചെയ്യുകയായിരുന്നു.
\end{malayalam}}
\flushright{\begin{Arabic}
\quranayah[16][34]
\end{Arabic}}
\flushleft{\begin{malayalam}
അങ്ങനെ അവര്‍ പ്രവര്‍ത്തിച്ചതിന്‍റെ ദുഷ്ഫലങ്ങള്‍ അവരെ ബാധിക്കുകയും, അവര്‍ ഏതൊന്നിനെപ്പറ്റി പരിഹസിച്ചിരുന്നുവോ അത് അവരെ വലയം ചെയ്യുകയും ചെയ്തു.
\end{malayalam}}
\flushright{\begin{Arabic}
\quranayah[16][35]
\end{Arabic}}
\flushleft{\begin{malayalam}
(അല്ലാഹുവോട്‌) പങ്കാളികളെ ചേര്‍ത്തവര്‍ പറഞ്ഞു: അല്ലാഹു ഉദ്ദേശിച്ചിരുന്നുവെങ്കില്‍ ഞങ്ങളോ ഞങ്ങളുടെ പിതാക്കന്‍മാരോ അവന്നു പുറമെ യാതൊന്നിനെയും ആരാധിക്കുമായിരുന്നില്ല. അവന്‍റെ കല്‍പന കൂടാതെ ഞങ്ങള്‍ യാതൊന്നും നിഷിദ്ധമാക്കുകയും ഇല്ലായിരുന്നു. അത് പോലെത്തന്നെ അവര്‍ക്കു മുമ്പുള്ളവരും ചെയ്തിട്ടുണ്ട്‌. എന്നാല്‍ ദൈവദൂതന്‍മാരുടെ മേല്‍ സ്പഷ്ടമായ പ്രബോധനമല്ലാതെ വല്ല ബാധ്യതയുമുണ്ടോ ?
\end{malayalam}}
\flushright{\begin{Arabic}
\quranayah[16][36]
\end{Arabic}}
\flushleft{\begin{malayalam}
തീര്‍ച്ചയായും ഓരോ സമുദായത്തിലും നാം ദൂതനെ നിയോഗിച്ചിട്ടുണ്ട്‌. നിങ്ങള്‍ അല്ലാഹുവെ ആരാധിക്കുകയും, ദുര്‍മൂര്‍ത്തികളെ വെടിയുകയും ചെയ്യണം എന്ന് (പ്രബോധനം ചെയ്യുന്നതിന് വേണ്ടി.) എന്നിട്ട് അവരില്‍ ചിലരെ അല്ലാഹു നേര്‍വഴിയിലാക്കി. അവരില്‍ ചിലരുടെ മേല്‍ വഴികേട് സ്ഥിരപ്പെടുകയും ചെയ്തു. ആകയാല്‍ നിങ്ങള്‍ ഭൂമിയിലൂടെ നടന്നിട്ട് നിഷേധിച്ചുതള്ളിക്കളഞ്ഞവരുടെ പര്യവസാനം എപ്രകാരമായിരുന്നു എന്ന് നോക്കുക.
\end{malayalam}}
\flushright{\begin{Arabic}
\quranayah[16][37]
\end{Arabic}}
\flushleft{\begin{malayalam}
(നബിയേ,) അവര്‍ സന്‍മാര്‍ഗത്തിലായിത്തീരുവാന്‍ നീ കൊതിക്കുന്നുവെങ്കില്‍ (അത് വെറുതെയാകുന്നു. കാരണം) താന്‍ വഴികേടിലാക്കുന്നവരെ അല്ലാഹു നേര്‍വഴിയിലാക്കുന്നതല്ല; തീര്‍ച്ച. അവര്‍ക്ക് സഹായികളായി ആരും ഇല്ല താനും.
\end{malayalam}}
\flushright{\begin{Arabic}
\quranayah[16][38]
\end{Arabic}}
\flushleft{\begin{malayalam}
അവര്‍ പരമാവധി ഉറപ്പിച്ച് സത്യം ചെയ്യാറുള്ള രീതിയില്‍ അല്ലാഹുവിന്‍റെ പേരില്‍ ആണയിട്ടു പറഞ്ഞു; മരണപ്പെടുന്നവരെ അല്ലാഹു ഉയിര്‍ത്തെഴുന്നേല്‍പിക്കുകയില്ല എന്ന്‌. അങ്ങനെയല്ല. അത് അവന്‍ ബാധ്യതയേറ്റ സത്യവാഗ്ദാനമാകുന്നു. പക്ഷെ, മനുഷ്യരില്‍ അധികപേരും മനസ്സിലാക്കുന്നില്ല.
\end{malayalam}}
\flushright{\begin{Arabic}
\quranayah[16][39]
\end{Arabic}}
\flushleft{\begin{malayalam}
ഏതൊരു വിഷയത്തില്‍ അവര്‍ ഭിന്നത പുലര്‍ത്തുന്നുവോ അതവര്‍ക്ക് വ്യക്തമാക്കികൊടുക്കുവാനും തങ്ങള്‍ കള്ളം പറയുന്നവരായിരുന്നു എന്ന് സത്യനിഷേധികള്‍ മനസ്സിലാക്കുവാനും വേണ്ടിയത്രെ അത്‌. (അവരെ ഉയിര്‍ത്തെഴുന്നേല്‍പിക്കുന്നത്‌.)
\end{malayalam}}
\flushright{\begin{Arabic}
\quranayah[16][40]
\end{Arabic}}
\flushleft{\begin{malayalam}
നാം ഒരു കാര്യം ഉദ്ദേശിച്ചാല്‍ അത് സംബന്ധിച്ച നമ്മുടെ വചനം ഉണ്ടാകൂ എന്ന് അതിനോട് നാം പറയുക മാത്രമാകുന്നു. അപ്പോഴതാ അതുണ്ടാകുന്നു.
\end{malayalam}}
\flushright{\begin{Arabic}
\quranayah[16][41]
\end{Arabic}}
\flushleft{\begin{malayalam}
അക്രമത്തിന് വിധേയരായതിന് ശേഷം അല്ലാഹുവിന്‍റെ മാര്‍ഗത്തില്‍ സ്വദേശം വെടിഞ്ഞ് പോയവരാരോ അവര്‍ക്ക് ഇഹലോകത്ത് നാം നല്ല താമസസൌകര്യം ഏര്‍പെടുത്തികൊടുക്കുകതന്നെ ചെയ്യും. എന്നാല്‍, പരലോകത്തെ പ്രതിഫലം തന്നെയാകുന്നു ഏറ്റവും മഹത്തായത്‌. അവര്‍ (അത്‌) അറിഞ്ഞിരുന്നുവെങ്കില്‍!
\end{malayalam}}
\flushright{\begin{Arabic}
\quranayah[16][42]
\end{Arabic}}
\flushleft{\begin{malayalam}
ക്ഷമിക്കുകയും തങ്ങളുടെ രക്ഷിതാവിന്‍റെ മേല്‍ ഭരമേല്‍പിക്കുകയും ചെയ്തവരത്രെ അവര്‍. (മുഹാജിറുകള്‍)
\end{malayalam}}
\flushright{\begin{Arabic}
\quranayah[16][43]
\end{Arabic}}
\flushleft{\begin{malayalam}
നിനക്ക് മുമ്പ് മനുഷ്യന്‍മാരെയല്ലാതെ നാം ദൂതന്‍മാരായി നിയോഗിച്ചിട്ടില്ല. അവര്‍ക്ക് നാം സന്ദേശം നല്‍കുന്നു. നിങ്ങള്‍ക്കറിഞ്ഞ് കൂടെങ്കില്‍ (വേദം മുഖേന) ഉല്‍ബോധനം ലഭിച്ചവരോട് നിങ്ങള്‍ ചോദിച്ച് നോക്കുക.
\end{malayalam}}
\flushright{\begin{Arabic}
\quranayah[16][44]
\end{Arabic}}
\flushleft{\begin{malayalam}
വ്യക്തമായ തെളിവുകളും വേദഗ്രന്ഥങ്ങളുമായി (അവരെ നാം നിയോഗിച്ചു.) നിനക്ക് നാം ഉല്‍ബോധനം അവതരിപ്പിച്ച് തന്നിരിക്കുന്നു. ജനങ്ങള്‍ക്കായി അവതരിപ്പിക്കപ്പെട്ടത് നീ അവര്‍ക്ക് വിവരിച്ചുകൊടുക്കാന്‍ വേണ്ടിയും, അവര്‍ ചിന്തിക്കാന്‍ വേണ്ടിയും.
\end{malayalam}}
\flushright{\begin{Arabic}
\quranayah[16][45]
\end{Arabic}}
\flushleft{\begin{malayalam}
എന്നാല്‍ ദുഷിച്ച കുതന്ത്രങ്ങള്‍ പ്രയോഗിച്ചവര്‍, അല്ലാഹു അവരെ ഭൂമിയില്‍ ആഴ്ത്തിക്കളയുകയില്ലെന്നോ, അവര്‍ ഓര്‍ക്കാത്ത ഭാഗത്ത് കൂടി ശിക്ഷ വരികയില്ലെന്നോ സമാധാനിച്ചിരിക്കുകയാണോ?
\end{malayalam}}
\flushright{\begin{Arabic}
\quranayah[16][46]
\end{Arabic}}
\flushleft{\begin{malayalam}
അല്ലെങ്കില്‍ അവരുടെ പോക്കുവരവുകള്‍ക്കിടയില്‍ അവര്‍ക്ക് തോല്‍പിച്ചുകളയാന്‍ പറ്റാത്തവിധത്തില്‍ അവന്‍ അവരെ പിടികൂടുകയില്ലെന്ന്‌.
\end{malayalam}}
\flushright{\begin{Arabic}
\quranayah[16][47]
\end{Arabic}}
\flushleft{\begin{malayalam}
അല്ലെങ്കില്‍ അവര്‍ ഭയപ്പെട്ടുകൊണ്ടിരിക്കെ അവരെ പിടികൂടുകയില്ലെന്ന്‌. എന്നാല്‍ തീര്‍ച്ചയായും നിങ്ങളുടെ രക്ഷിതാവ് ഏറെ ദയയുള്ളവനും കരുണാനിധിയും തന്നെയാകുന്നു.
\end{malayalam}}
\flushright{\begin{Arabic}
\quranayah[16][48]
\end{Arabic}}
\flushleft{\begin{malayalam}
അല്ലാഹു സൃഷ്ടിച്ചിട്ടുള്ള ഏതൊരു വസ്തുവിന്‍റെയും നേര്‍ക്ക് അവര്‍ നോക്കിയിട്ടില്ലേ? എളിയവരായിട്ടും അല്ലാഹുവിന് സുജൂദ് ചെയ്ത്കൊണ്ടും അതിന്‍റെ നിഴലുകള്‍ വലത്തോട്ടും ഇടത്തോട്ടും തിരിഞ്ഞ് കൊണ്ടിരിക്കുന്നു.
\end{malayalam}}
\flushright{\begin{Arabic}
\quranayah[16][49]
\end{Arabic}}
\flushleft{\begin{malayalam}
ആകാശങ്ങളിലുള്ളതും ഭൂമിയിലുള്ളതുമായ ഏതൊരു ജീവിയും അല്ലാഹുവിന് സുജൂദ് ചെയ്യുന്നു. മലക്കുകളും (സുജൂദ് ചെയ്യുന്നു.) അവര്‍ അഹങ്കാരം നടിക്കുന്നില്ല.
\end{malayalam}}
\flushright{\begin{Arabic}
\quranayah[16][50]
\end{Arabic}}
\flushleft{\begin{malayalam}
അവര്‍ക്കു മീതെയുള്ള അവരുടെ രക്ഷിതാവിനെ അവര്‍ ഭയപ്പെടുകയും, അവര്‍ കല്‍പിക്കപ്പെടുന്നതെന്തും അവര്‍ പ്രവര്‍ത്തിക്കുകയും ചെയ്യുന്നു.
\end{malayalam}}
\flushright{\begin{Arabic}
\quranayah[16][51]
\end{Arabic}}
\flushleft{\begin{malayalam}
അല്ലാഹു അരുളിയിരിക്കുന്നു: രണ്ട് ദൈവങ്ങളെ നിങ്ങള്‍ സ്വീകരിക്കരുത്‌. അവന്‍ ഒരേ ഒരു ദൈവം മാത്രമേയുള്ളൂ. അതിനാല്‍ (ഏകദൈവമായ) എന്നെ മാത്രം നിങ്ങള്‍ ഭയപ്പെടുവിന്‍.
\end{malayalam}}
\flushright{\begin{Arabic}
\quranayah[16][52]
\end{Arabic}}
\flushleft{\begin{malayalam}
അവന്‍റെതാകുന്നു ആകാശങ്ങളിലുള്ളതും ഭൂമിയിലുള്ളതും. നിരന്തരമായിട്ടുള്ള കീഴ്‌വണക്കം അവന്ന് മാത്രമാകുന്നു. എന്നിരിക്കെ അല്ലാഹു അല്ലാത്തവരോടാണോ നിങ്ങള്‍ ഭക്തികാണിക്കുന്നത്‌?
\end{malayalam}}
\flushright{\begin{Arabic}
\quranayah[16][53]
\end{Arabic}}
\flushleft{\begin{malayalam}
നിങ്ങളില്‍ അനുഗ്രഹമായി എന്തുണ്ടെങ്കിലും അത് അല്ലാഹുവിങ്കല്‍ നിന്നുള്ളതാകുന്നു. എന്നിട്ട് നിങ്ങള്‍ക്കൊരു കഷ്ടത ബാധിച്ചാല്‍ അവങ്കലേക്ക് തന്നെയാണ് നിങ്ങള്‍ മുറവിളികൂട്ടിച്ചെല്ലുന്നത്‌.
\end{malayalam}}
\flushright{\begin{Arabic}
\quranayah[16][54]
\end{Arabic}}
\flushleft{\begin{malayalam}
പിന്നെ നിങ്ങളില്‍ നിന്ന് അവന്‍ കഷ്ടത നീക്കിത്തന്നാല്‍ നിങ്ങളില്‍ ഒരു വിഭാഗമതാ തങ്ങളുടെ രക്ഷിതാവിനോട് പങ്കാളികളെ ചേര്‍ക്കുന്നു.
\end{malayalam}}
\flushright{\begin{Arabic}
\quranayah[16][55]
\end{Arabic}}
\flushleft{\begin{malayalam}
നാം അവര്‍ക്ക് നല്‍കിയിട്ടുള്ളതില്‍ അങ്ങനെ അവര്‍ നന്ദികേട് കാണിക്കുന്നു. നിങ്ങള്‍ സുഖിച്ച് കൊള്ളുക. എന്നാല്‍ വഴിയെ നിങ്ങള്‍ക്കറിയാം.
\end{malayalam}}
\flushright{\begin{Arabic}
\quranayah[16][56]
\end{Arabic}}
\flushleft{\begin{malayalam}
നാം അവര്‍ക്ക് നല്‍കിയിട്ടുള്ളതില്‍ നിന്ന് ഒരു ഓഹരി, അവര്‍ക്ക് തന്നെ ശരിയായ അറിവില്ലാത്ത ചിലതിന് (വ്യാജദൈവങ്ങള്‍ക്ക്‌) അവര്‍ നിശ്ചയിച്ച് വെക്കുന്നു. അല്ലാഹുവെതന്നെയാണ, നിങ്ങള്‍ കെട്ടിച്ചമയ്ക്കുന്നതിനെപ്പറ്റി തീര്‍ച്ചയായും നിങ്ങള്‍ ചോദ്യം ചെയ്യപ്പെടുന്നതാണ്‌.
\end{malayalam}}
\flushright{\begin{Arabic}
\quranayah[16][57]
\end{Arabic}}
\flushleft{\begin{malayalam}
അല്ലാഹുവിന് അവര്‍ പെണ്‍മക്കളെ സ്ഥാപിക്കുന്നു. അവന്‍ എത്രയോ പരിശുദ്ധന്‍. അവര്‍ക്കാകട്ടെ അവര്‍ ഇഷ്ടപ്പെടുന്നതും (ആണ്‍മക്കള്‍)
\end{malayalam}}
\flushright{\begin{Arabic}
\quranayah[16][58]
\end{Arabic}}
\flushleft{\begin{malayalam}
അവരില്‍ ഒരാള്‍ക്ക് ഒരു പെണ്‍കുഞ്ഞുണ്ടായ സന്തോഷവാര്‍ത്ത നല്‍കപ്പെട്ടാല്‍ കോപാകുലനായിട്ട് അവന്‍റെ മുഖം കറുത്തിരുണ്ട് പോകുന്നു.
\end{malayalam}}
\flushright{\begin{Arabic}
\quranayah[16][59]
\end{Arabic}}
\flushleft{\begin{malayalam}
അവന്ന് സന്തോഷവാര്‍ത്ത നല്‍കപ്പെട്ട ആ കാര്യത്തിലുള്ള അപമാനത്താല്‍ ആളുകളില്‍ നിന്ന് അവന്‍ ഒളിച്ച് കളയുന്നു. അപമാനത്തോടെ അതിനെ വെച്ചുകൊണ്ടിരിക്കണമോ, അതല്ല, അതിനെ മണ്ണില്‍ കുഴിച്ച് മൂടണമോ (എന്നതായിരിക്കും അവന്‍റെ ചിന്ത) ശ്രദ്ധിക്കുക: അവര്‍ എടുക്കുന്ന തീരുമാനം എത്ര മോശം!
\end{malayalam}}
\flushright{\begin{Arabic}
\quranayah[16][60]
\end{Arabic}}
\flushleft{\begin{malayalam}
പരലോകത്തില്‍ വിശ്വസിക്കാത്തവര്‍ക്കാകുന്നു ഹീനമായ അവസ്ഥ. അല്ലാഹുവിന്നാകുന്നു അത്യുന്നതമായ അവസ്ഥ. അവന്‍ പ്രതാപിയും യുക്തിമാനുമാകുന്നു.
\end{malayalam}}
\flushright{\begin{Arabic}
\quranayah[16][61]
\end{Arabic}}
\flushleft{\begin{malayalam}
അല്ലാഹു മനുഷ്യരെ അവരുടെ അക്രമം മൂലം (ഉടനടി) പിടികൂടിയിരുന്നെങ്കില്‍ ഭൂമുഖത്ത് യാതൊരു ജന്തുവെയും അവന്‍ വിട്ടേക്കുമായിരുന്നില്ല. എന്നാല്‍ നിര്‍ണിതമായ ഒരു അവധി വരെ അവന്‍ അവര്‍ക്ക് സമയം നീട്ടികൊടുക്കുകയാണ് ചെയ്യുന്നത്‌. അങ്ങനെ അവരുടെ അവധി വന്നാല്‍ ഒരു നാഴിക നേരം പോലും അവര്‍ക്ക് വൈകിക്കാന്‍ ആവുകയില്ല. അവര്‍ക്കത് നേരെത്തെയാക്കാനും കഴിയില്ല.
\end{malayalam}}
\flushright{\begin{Arabic}
\quranayah[16][62]
\end{Arabic}}
\flushleft{\begin{malayalam}
അവര്‍ക്ക് ഇഷ്ടമില്ലാത്തതിനെ അവര്‍ അല്ലാഹുവിന് നിശ്ചയിക്കുന്നു. ഏറ്റവും ഉത്തമായിട്ടുള്ളതെന്തോ അത് തങ്ങള്‍ക്കുള്ളതാണെന്ന് അവരുടെ നാവുകള്‍ വ്യാജവര്‍ണന നടത്തുകയും ചെയ്യുന്നു. ഒട്ടും സംശയമില്ല. അവര്‍ക്കുള്ളത് നരകം തന്നെയാണ്‌. അവര്‍ (അങ്ങോട്ട്‌) മുമ്പില്‍ നയിക്കപ്പെടുന്നതാണ്‌.
\end{malayalam}}
\flushright{\begin{Arabic}
\quranayah[16][63]
\end{Arabic}}
\flushleft{\begin{malayalam}
അല്ലാഹുവെ തന്നെയാണ, താങ്കള്‍ക്ക് മുമ്പ് പല സമുദായങ്ങളിലേക്കും നാം ദൂതന്‍മാരെ അയച്ചിട്ടുണ്ട്‌. എന്നാല്‍ പിശാച് അവര്‍ക്ക് അവരുടെ (ദുഷ്‌) പ്രവര്‍ത്തനങ്ങള്‍ അലങ്കാരമായി തോന്നിക്കുകയാണ് ചെയ്തത്‌. അങ്ങനെ അവനാണ് ഇന്ന് അവരുടെ മിത്രം. അവര്‍ക്കുള്ളതാകട്ടെ വേദനാജനകമായ ശിക്ഷയാണ് താനും.
\end{malayalam}}
\flushright{\begin{Arabic}
\quranayah[16][64]
\end{Arabic}}
\flushleft{\begin{malayalam}
അവര്‍ ഏതൊരു കാര്യത്തില്‍ ഭിന്നിച്ച് പോയിരിക്കുന്നുവോ, അതവര്‍ക്ക് വ്യക്തമാക്കികൊടുക്കുവാന്‍ വേണ്ടിയും, വിശ്വസിക്കുന്ന ജനങ്ങള്‍ക്ക് മാര്‍ഗദര്‍ശനവും കാരുണ്യവും ആയിക്കൊണ്ടും മാത്രമാണ് നിനക്ക് നാം വേദഗ്രന്ഥം അവതരിപ്പിച്ച് തന്നത്‌.
\end{malayalam}}
\flushright{\begin{Arabic}
\quranayah[16][65]
\end{Arabic}}
\flushleft{\begin{malayalam}
അല്ലാഹു ആകാശത്ത് നിന്ന് വെള്ളം ചൊരിഞ്ഞുതരികയും, അത് മൂലം ഭൂമിയെ- അത് നിര്‍ജീവമായികിടന്നതിന് ശേഷം- അവന്‍ സജീവമാക്കുകയും ചെയ്തു. കേട്ട് മനസ്സിലാക്കുന്ന ആളുകള്‍ക്ക് തീര്‍ച്ചയായും അതില്‍ ദൃഷ്ടാന്തമുണ്ട്‌.
\end{malayalam}}
\flushright{\begin{Arabic}
\quranayah[16][66]
\end{Arabic}}
\flushleft{\begin{malayalam}
കാലികളുടെ കാര്യത്തില്‍ തീര്‍ച്ചയായും നിങ്ങള്‍ക്ക് ഒരു പാഠമുണ്ട്‌. അവയുടെ ഉദരങ്ങളില്‍ നിന്ന്‌- കാഷ്ഠത്തിനും രക്തത്തിനും ഇടയില്‍ നിന്ന് കുടിക്കുന്നവര്‍ക്ക് സുഖദമായ ശുദ്ധമായ പാല്‍ നിങ്ങള്‍ക്കു കുടിക്കുവാനായി നാം നല്‍കുന്നു.
\end{malayalam}}
\flushright{\begin{Arabic}
\quranayah[16][67]
\end{Arabic}}
\flushleft{\begin{malayalam}
ഈന്തപ്പനകളുടെയും മുന്തിരിവള്ളികളുടെയും ഫലങ്ങളില്‍ നിന്നും (നിങ്ങള്‍ക്കു നാം പാനീയം നല്‍കുന്നു.) അതില്‍ നിന്ന് ലഹരി പദാര്‍ത്ഥവും, ഉത്തമമായ ആഹാരവും നിങ്ങളുണ്ടാക്കുന്നു. ചിന്തിക്കുന്ന ജനങ്ങള്‍ക്ക് തീര്‍ച്ചയായും അതില്‍ ദൃഷ്ടാന്തമുണ്ട്‌.
\end{malayalam}}
\flushright{\begin{Arabic}
\quranayah[16][68]
\end{Arabic}}
\flushleft{\begin{malayalam}
നിന്‍റെ നാഥന്‍ തേനീച്ചയ്ക്ക് ഇപ്രകാരം ബോധനം നല്‍കുകയും ചെയ്തിരിക്കുന്നു: മലകളിലും മരങ്ങളിലും മനുഷ്യര്‍ കെട്ടിയുയര്‍ത്തുന്നവയിലും നീ പാര്‍പ്പിടങ്ങളുണ്ടാക്കിക്കൊള്ളുക.
\end{malayalam}}
\flushright{\begin{Arabic}
\quranayah[16][69]
\end{Arabic}}
\flushleft{\begin{malayalam}
പിന്നെ എല്ലാതരം ഫലങ്ങളില്‍ നിന്നും നീ ഭക്ഷിച്ച് കൊള്ളുക. എന്നിട്ട് നിന്‍റെ രക്ഷിതാവ് സൌകര്യപ്രദമായി ഒരുക്കിത്തന്നിട്ടുള്ള മാര്‍ഗങ്ങളില്‍ നീ പ്രവേശിച്ച് കൊള്ളുക. അവയുടെ ഉദരങ്ങളില്‍ നിന്ന് വ്യത്യസ്ത വര്‍ണങ്ങളുള്ള പാനീയം പുറത്ത് വരുന്നു. അതില്‍ മനുഷ്യര്‍ക്ക് രോഗശമനം ഉണ്ട്‌. ചിന്തിക്കുന്ന ആളുകള്‍ക്ക് തീര്‍ച്ചയായും അതില്‍ ദൃഷ്ടാന്തമുണ്ട്‌.
\end{malayalam}}
\flushright{\begin{Arabic}
\quranayah[16][70]
\end{Arabic}}
\flushleft{\begin{malayalam}
അല്ലാഹുവാണ് നിങ്ങളെ സൃഷ്ടിച്ചത്‌. പിന്നീട് അവന്‍ നിങ്ങളെ മരിപ്പിക്കുന്നു. നിങ്ങളില്‍ ചിലര്‍ ഏറ്റവും അവശമായ പ്രായത്തിലേക്ക് തള്ളപ്പെടുന്നു; (പലതും) അറിഞ്ഞതിന് ശേഷം യാതൊന്നും അറിയാത്ത അവസ്ഥയില്‍ എത്തത്തക്കവണ്ണം. തീര്‍ച്ചയായും അല്ലാഹു എല്ലാം അറിയുന്നവനും എല്ലാ കഴിവുമുള്ളവനുമാകുന്നു.
\end{malayalam}}
\flushright{\begin{Arabic}
\quranayah[16][71]
\end{Arabic}}
\flushleft{\begin{malayalam}
അല്ലാഹു നിങ്ങളില്‍ ചിലരെ മറ്റു ചിലരെക്കാള്‍ ഉപജീവനത്തിന്‍റെ കാര്യത്തില്‍ മെച്ചപ്പെട്ടവരാക്കിയിരിക്കുന്നു. എന്നാല്‍ (ജീവിതത്തില്‍) മെച്ചം ലഭിച്ചവര്‍ തങ്ങളുടെ ഉപജീവനം തങ്ങളുടെ വലതുകൈകള്‍ അധീനപ്പെടുത്തിവെച്ചിട്ടുള്ളവര്‍ (അടിമകള്‍) ക്ക് വിട്ടുകൊടുക്കുകയും, അങ്ങനെ ഉപജീവനത്തില്‍ അവര്‍ (അടിമയും ഉടമയും) തുല്യരാകുകയും ചെയ്യുന്നില്ല. അപ്പോള്‍ അല്ലാഹുവിന്‍റെ അനുഗ്രഹത്തെയാണോ അവര്‍ നിഷേധിക്കുന്നത് ?
\end{malayalam}}
\flushright{\begin{Arabic}
\quranayah[16][72]
\end{Arabic}}
\flushleft{\begin{malayalam}
അല്ലാഹു നിങ്ങള്‍ക്ക് നിങ്ങളുടെ കൂട്ടത്തില്‍ നിന്ന് തന്നെ ഇണകളെ ഉണ്ടാക്കുകയും, നിങ്ങളുടെ ഇണകളിലൂടെ അവന്‍ നിങ്ങള്‍ക്ക് പുത്രന്‍മാരെയും പൌത്രന്‍മാരെയും ഉണ്ടാക്കിത്തരികയും, വിശിഷ്ട വസ്തുക്കളില്‍ നിന്നും അവന്‍ നിങ്ങള്‍ക്ക് ഉപജീവനം നല്‍കുകയും ചെയ്തിരിക്കുന്നു. എന്നിട്ടും അവര്‍ അസത്യത്തില്‍ വിശ്വസിക്കുകയും, അല്ലാഹുവിന്‍റെ അനുഗ്രഹത്തെ നിഷേധിക്കുകയുമാണോ ചെയ്യുന്നത്‌?
\end{malayalam}}
\flushright{\begin{Arabic}
\quranayah[16][73]
\end{Arabic}}
\flushleft{\begin{malayalam}
ആകാശങ്ങളില്‍ നിന്നോ ഭൂമിയില്‍ നിന്നോ അവര്‍ക്ക് വേണ്ടി യാതൊരു ഭക്ഷണവും അധീനപ്പെടുത്തികൊടുക്കാത്തവരും, (യാതൊന്നിനും) കഴിയാത്തവരുമായിട്ടുള്ളവരെയാണ് അല്ലാഹുവിന് പുറമെ അവര്‍ ആരാധിക്കുന്നത്‌.
\end{malayalam}}
\flushright{\begin{Arabic}
\quranayah[16][74]
\end{Arabic}}
\flushleft{\begin{malayalam}
ആകയാല്‍ അല്ലാഹുവിനു നിങ്ങള്‍ ഉപമകള്‍ പറയരുത്‌. തീര്‍ച്ചയായും അല്ലാഹു അറിയുന്നു. നിങ്ങള്‍ അറിയുന്നില്ല.
\end{malayalam}}
\flushright{\begin{Arabic}
\quranayah[16][75]
\end{Arabic}}
\flushleft{\begin{malayalam}
മറ്റൊരാളുടെ ഉടമസ്ഥതയിലുള്ള, യാതൊന്നിനും കഴിവില്ലാത്ത ഒരു അടിമയെയും, നമ്മുടെ വകയായി നാം നല്ല ഉപജീവനം നല്‍കിയിട്ട് അതില്‍ നിന്ന് രഹസ്യമായും പരസ്യമായും ചെലവഴിച്ച് കൊണ്ടിരിക്കുന്ന ഒരാളെയും അല്ലാഹു ഉപമയായി എടുത്തുകാണിക്കുന്നു. ഇവര്‍ തുല്യരാകുമോ? അല്ലാഹുവിന് സ്തുതി. പക്ഷെ, അവരില്‍ അധികപേരും മനസ്സിലാക്കുന്നില്ല.
\end{malayalam}}
\flushright{\begin{Arabic}
\quranayah[16][76]
\end{Arabic}}
\flushleft{\begin{malayalam}
(ഇനിയും) രണ്ട് പുരുഷന്‍മാരെ അല്ലാഹു ഉപമയായി എടുത്തുകാണിക്കുന്നു. അവരില്‍ ഒരാള്‍ യാതൊന്നിനും കഴിവില്ലാത്ത ഊമയാകുന്നു. അവന്‍ തന്‍റെ യജമാനന് ഒരു ഭാരവുമാണ്‌. അവനെ എവിടേക്ക് തിരിച്ചുവിട്ടാലും അവന്‍ യാതൊരു നന്‍മയും കൊണ്ട് വരില്ല. അവനും, നേരായ പാതയില്‍ നിലയുറപ്പിച്ചുകൊണ്ട് നീതി കാണിക്കാന്‍ കല്‍പിക്കുന്നവനും തുല്യരാകുമോ?
\end{malayalam}}
\flushright{\begin{Arabic}
\quranayah[16][77]
\end{Arabic}}
\flushleft{\begin{malayalam}
അല്ലാഹുവിന്നാണ് ആകാശങ്ങളിലെയും ഭൂമിയിലെയും അദൃശ്യജ്ഞാനമുള്ളത്‌. അന്ത്യസമയത്തിന്‍റെ കാര്യം കണ്ണ് ഇമവെട്ടും പോലെ മാത്രമാകുന്നു. അഥവാ അതിനെക്കാള്‍ വേഗത കൂടിയതാകുന്നു. തീര്‍ച്ചയായും അല്ലാഹു ഏത് കാര്യത്തിനും കഴിവുള്ളവനാകുന്നു.
\end{malayalam}}
\flushright{\begin{Arabic}
\quranayah[16][78]
\end{Arabic}}
\flushleft{\begin{malayalam}
നിങ്ങളുടെ മാതാക്കളുടെ ഉദരങ്ങളില്‍ നിന്ന് നിങ്ങള്‍ക്ക് യാതൊന്നും അറിഞ്ഞ് കൂടാത്ത അവസ്ഥയില്‍ അല്ലാഹു നിങ്ങളെ പുറത്ത് കൊണ്ട് വന്നു. നിങ്ങള്‍ക്കു അവന്‍ കേള്‍വിയും കാഴ്ചകളും ഹൃദയങ്ങളും നല്‍കുകയും ചെയ്തു. നിങ്ങള്‍ നന്ദിയുള്ളവരായിരിക്കാന്‍ വേണ്ടി.
\end{malayalam}}
\flushright{\begin{Arabic}
\quranayah[16][79]
\end{Arabic}}
\flushleft{\begin{malayalam}
അന്തരീക്ഷത്തില്‍ (ദൈവിക കല്‍പനയ്ക്ക്‌) വിധേയമായികൊണ്ടു പറക്കുന്ന പക്ഷികളുടെ നേര്‍ക്ക് അവര്‍ നോക്കിയില്ലേ? അല്ലാഹു അല്ലാതെ ആരും അവയെ താങ്ങി നിര്‍ത്തുന്നില്ല. വിശ്വസിക്കുന്ന ജനങ്ങള്‍ക്ക് തീര്‍ച്ചയായും അതില്‍ ദൃഷ്ടാന്തങ്ങളുണ്ട്‌.
\end{malayalam}}
\flushright{\begin{Arabic}
\quranayah[16][80]
\end{Arabic}}
\flushleft{\begin{malayalam}
അല്ലാഹു നിങ്ങള്‍ക്കു നിങ്ങളുടെ വീടുകളെ വിശ്രമസ്ഥാനമാക്കിയിരിക്കുന്നു. കാലികളുടെ തോലുകളില്‍ നിന്നും അവന്‍ നിങ്ങള്‍ക്ക് പാര്‍പ്പിടങ്ങള്‍ നല്‍കിയിരിക്കുന്നു. നിങ്ങള്‍ യാത്ര ചെയ്യുന്ന ദിവസവും നിങ്ങള്‍ താവളമടിക്കുന്ന ദിവസവും നിങ്ങള്‍ അവ അനായാസം ഉപയോഗപ്പെടുത്തുന്നു. ചെമ്മരിയാടുകളുടെയും ഒട്ടകങ്ങളുടെയും കോലാടുകളുടെയും രോമങ്ങളില്‍ നിന്ന് ഒരു അവധി വരെ ഉപയോഗിക്കാവുന്ന വീട്ടുപകരണങ്ങളും ഉപഭോഗസാധനങ്ങളും (അവന്‍ നല്‍കിയിരിക്കുന്നു.)
\end{malayalam}}
\flushright{\begin{Arabic}
\quranayah[16][81]
\end{Arabic}}
\flushleft{\begin{malayalam}
അല്ലാഹു താന്‍ സൃഷ്ടിച്ച വസ്തുക്കളില്‍ നിന്നു നിങ്ങള്‍ക്കു തണലുകളുണ്ടാക്കിത്തരികയും, നിങ്ങള്‍ക്ക് പര്‍വ്വതങ്ങളില്‍ അവന്‍ അഭയ കേന്ദ്രങ്ങളുണ്ടാക്കുകയും ചെയ്തിരിക്കുന്നു. നിങ്ങളെ ചൂടില്‍ നിന്നു കാത്തുരക്ഷിക്കുന്ന ഉടുപ്പുകളും, നിങ്ങള്‍ അന്യോന്യം നടത്തുന്ന ആക്രമണത്തില്‍ നിന്ന് നിങ്ങളെ കാത്തുരക്ഷിക്കുന്ന കവചങ്ങളും അവന്‍ നിങ്ങള്‍ക്കു നല്‍കിയിരിക്കുന്നു. അപ്രകാരം അവന്‍റെ അനുഗ്രഹം അവന്‍ നിങ്ങള്‍ക്ക് നിറവേറ്റിത്തരുന്നു; നിങ്ങള്‍ (അവന്ന്‌) കീഴ്പെടുന്നതിന് വേണ്ടി.
\end{malayalam}}
\flushright{\begin{Arabic}
\quranayah[16][82]
\end{Arabic}}
\flushleft{\begin{malayalam}
ഇനി അവര്‍ തിരിഞ്ഞുകളയുന്ന പക്ഷം നിനക്കുള്ള ബാധ്യത (കാര്യങ്ങള്‍) വ്യക്തമാക്കുന്ന പ്രബോധനം മാത്രമാകുന്നു.
\end{malayalam}}
\flushright{\begin{Arabic}
\quranayah[16][83]
\end{Arabic}}
\flushleft{\begin{malayalam}
അല്ലാഹുവിന്‍റെ അനുഗ്രഹം അവര്‍ മനസ്സിലാക്കുകയും, എന്നിട്ട് അതിനെ നിഷേധിക്കുകയുമാണ് ചെയ്യുന്നത്‌. അവ രില്‍ അധികപേരും നന്ദികെട്ടവരാകുന്നു.
\end{malayalam}}
\flushright{\begin{Arabic}
\quranayah[16][84]
\end{Arabic}}
\flushleft{\begin{malayalam}
ഓരോ സമുദായത്തില്‍ നിന്നും ഓരോ സാക്ഷിയെ നാം എഴുന്നേല്‍പിക്കുന്ന ദിവസം (ശ്രദ്ധേയമാകുന്നു.) പിന്നീട് സത്യനിഷേധികള്‍ക്കു (ഉരിയാടാന്‍) അനുവാദം നല്‍കപ്പെടുകയില്ല. പരിഹാരം ചെയ്യാന്‍ അവരോട് ആവശ്യപ്പെടുകയുമില്ല.
\end{malayalam}}
\flushright{\begin{Arabic}
\quranayah[16][85]
\end{Arabic}}
\flushleft{\begin{malayalam}
അക്രമം പ്രവര്‍ത്തിച്ചവര്‍ ശിക്ഷ നേരിട്ട് കാണുമ്പോഴാകട്ടെ അത് അവര്‍ക്ക് ലഘൂകരിച്ച് കൊടുക്കപ്പെടുകയില്ല. അവര്‍ക്ക് ഇടനല്‍കപ്പെടുകയുമില്ല.
\end{malayalam}}
\flushright{\begin{Arabic}
\quranayah[16][86]
\end{Arabic}}
\flushleft{\begin{malayalam}
(അല്ലാഹുവോട്‌) പങ്കുചേര്‍ത്തവര്‍ തങ്ങള്‍ പങ്കാളികളാക്കിയിരുന്നവരെ (പരലോകത്ത് വെച്ച്‌) കണ്ടാല്‍ ഇപ്രകാരം പറയും: ഞങ്ങളുടെ രക്ഷിതാവേ, നിനക്കു പുറമെ ഞങ്ങള്‍ വിളിച്ച് പ്രാര്‍ത്ഥിക്കാറുണ്ടായിരുന്ന ഞങ്ങളുടെ പങ്കാളികളാണിവര്‍. അപ്പോള്‍ അവര്‍ (പങ്കാളികള്‍) അവര്‍ക്ക് നല്‍കുന്ന മറുപടി തീര്‍ച്ചയായും നിങ്ങള്‍ കള്ളം പറയുന്നവരാകുന്നു എന്ന വാക്കായിരിക്കും.
\end{malayalam}}
\flushright{\begin{Arabic}
\quranayah[16][87]
\end{Arabic}}
\flushleft{\begin{malayalam}
ആ ദിവസം അവര്‍ അര്‍പ്പണം അല്ലാഹുവിന് നല്‍കുന്നതും അവര്‍ കെട്ടിച്ചമച്ചുകൊണ്ടിരുന്നതെല്ലാം അവരെ വിട്ടുമാറിക്കളയുന്നതുമാണ്‌.
\end{malayalam}}
\flushright{\begin{Arabic}
\quranayah[16][88]
\end{Arabic}}
\flushleft{\begin{malayalam}
അവിശ്വസിക്കുകയും അല്ലാഹുവിന്‍റെ മാര്‍ഗത്തില്‍ നിന്ന് (ആളുകളെ) തടയുകയും ചെയ്തവരാരോ അവര്‍ക്ക് നാം ശിക്ഷയ്ക്കുമേല്‍ ശിക്ഷ കൂട്ടികൊടുക്കുന്നതാണ്‌. അവര്‍ കുഴപ്പം സൃഷ്ടിച്ച് കൊണ്ടിരുന്നതിന്‍റെ ഫലമത്രെ അത്‌.
\end{malayalam}}
\flushright{\begin{Arabic}
\quranayah[16][89]
\end{Arabic}}
\flushleft{\begin{malayalam}
ഓരോ സമുദായത്തിലും അവരുടെ കാര്യത്തിന്ന് സാക്ഷിയായിക്കൊണ്ട് അവരില്‍ നിന്ന് തന്നെയുള്ള ഒരാളെ നാം നിയോഗിക്കുകയും, ഇക്കൂട്ടരുടെ കാര്യത്തിന് സാക്ഷിയായിക്കൊണ്ട് നിന്നെ നാം കൊണ്ട് വരികയും ചെയ്യുന്ന ദിവസം (ശ്രദ്ധേയമത്രെ.) എല്ലാകാര്യത്തിനും വിശദീകരണമായിക്കൊണ്ടും, മാര്‍ഗദര്‍ശനവും കാരുണ്യവും കീഴ്പെട്ടു ജീവിക്കുന്നവര്‍ക്ക് സന്തോഷവാര്‍ത്തയുമായിക്കൊണ്ടുമാണ് നിനക്ക് നാം വേദഗ്രന്ഥം അവതരിപ്പിച്ചിരിക്കുന്നത്‌.
\end{malayalam}}
\flushright{\begin{Arabic}
\quranayah[16][90]
\end{Arabic}}
\flushleft{\begin{malayalam}
തീര്‍ച്ചയായും അല്ലാഹു കല്‍പിക്കുന്നത് നീതി പാലിക്കുവാനും നന്‍മചെയ്യുവാനും കുടുംബബന്ധമുള്ളവര്‍ക്ക് (സഹായം) നല്‍കുവാനുമാണ് . അവന്‍ വിലക്കുന്നത് നീചവൃത്തിയില്‍ നിന്നും ദുരാചാരത്തില്‍ നിന്നും അതിക്രമത്തില്‍ നിന്നുമാണ്‌. നിങ്ങള്‍ ചിന്തിച്ചു ഗ്രഹിക്കുവാന്‍ വേണ്ടി അവന്‍ നിങ്ങള്‍ക്കു ഉപദേശം നല്‍കുന്നു.
\end{malayalam}}
\flushright{\begin{Arabic}
\quranayah[16][91]
\end{Arabic}}
\flushleft{\begin{malayalam}
നിങ്ങള്‍ കരാര്‍ ചെയ്യുന്ന പക്ഷം അല്ലാഹുവിന്‍റെ കരാര്‍ നിങ്ങള്‍ നിറവേറ്റുക. അല്ലാഹുവെ നിങ്ങളുടെ ജാമ്യക്കാരനാക്കിക്കൊണ്ട് നിങ്ങള്‍ ഉറപ്പിച്ചു സത്യം ചെയ്തശേഷം അത് ലംഘിക്കരുത്‌. തീര്‍ച്ചയായും അല്ലാഹു നിങ്ങള്‍ പ്രവര്‍ത്തിച്ച് കൊണ്ടിരിക്കുന്നത് അറിയുന്നു.
\end{malayalam}}
\flushright{\begin{Arabic}
\quranayah[16][92]
\end{Arabic}}
\flushleft{\begin{malayalam}
ഉറപ്പോടെ നൂല്‍ നൂറ്റ ശേഷം തന്‍റെ നൂല്‍ പലയിഴകളാക്കി പിരിയുടച്ച് കളഞ്ഞ ഒരു സ്ത്രീയെ പേലെ നിങ്ങള്‍ ആകരുത്‌. ഒരു ജനസമൂഹം മറ്റൊരു ജനസമൂഹത്തേക്കാള്‍ എണ്ണപ്പെരുപ്പമുള്ളതാകുന്നതിന്‍റെ പേരില്‍ നിങ്ങള്‍ നിങ്ങളുടെ ശപഥങ്ങളെ അന്യോന്യം ചതിപ്രയോഗത്തിനുള്ള മാര്‍ഗമാക്കിക്കളയുന്നു. അതു മുഖേന അല്ലാഹു നിങ്ങളെ പരീക്ഷിക്കുക മാത്രമാണ് ചെയ്യുന്നത്‌. നിങ്ങള്‍ ഏതൊരു കാര്യത്തില്‍ ഭിന്നത പുലര്‍ത്തുന്നവരായിരിക്കുന്നുവോ ആ കാര്യം ഉയിര്‍ത്തെഴുന്നേല്‍പിന്‍റെ നാളില്‍ അവന്‍ നിങ്ങള്‍ക്കു വ്യക്തമാക്കിത്തരിക തന്നെ ചെയ്യും.
\end{malayalam}}
\flushright{\begin{Arabic}
\quranayah[16][93]
\end{Arabic}}
\flushleft{\begin{malayalam}
അല്ലാഹു ഉദ്ദേശിച്ചിരുന്നുവെങ്കില്‍ നിങ്ങളെ അവന്‍ ഏകസമുദായമാക്കുമായിരുന്നു. എന്നാല്‍ താന്‍ ഉദ്ദേശിക്കുന്നവരെ അവന്‍ ദുര്‍മാര്‍ഗത്തിലാക്കുകയും, താന്‍ ഉദ്ദേശിക്കുന്നവരെ അവന്‍ നേര്‍വഴിയിലാക്കുകയും ചെയ്യും. നിങ്ങള്‍ പ്രവര്‍ത്തിച്ച് കൊണ്ടിരിക്കുന്നതിനെപ്പറ്റി നിങ്ങള്‍ ചോദ്യം ചെയ്യപ്പെടുക തന്നെ ചെയ്യും.
\end{malayalam}}
\flushright{\begin{Arabic}
\quranayah[16][94]
\end{Arabic}}
\flushleft{\begin{malayalam}
നിങ്ങള്‍ നിങ്ങളുടെ ശപഥങ്ങളെ അന്യോന്യം ചതിപ്രയോഗത്തിനുള്ള മാര്‍ഗമാക്കിക്കളയരുത്‌. (ഇസ്ലാമില്‍) നില്‍പുറച്ചതിന് ശേഷം പാദം ഇടറിപോകാനും, അല്ലാഹുവിന്‍റെ മാര്‍ഗത്തില്‍ നിന്ന് ആളുകളെ തടഞ്ഞതു നിമിത്തം നിങ്ങള്‍ കെടുതി അനുഭവിക്കാനും അത് കാരണമായിത്തീരും. നിങ്ങള്‍ക്ക് ഭയങ്കരമായ ശിക്ഷയുണ്ടായിരിക്കുകയും ചെയ്യും.
\end{malayalam}}
\flushright{\begin{Arabic}
\quranayah[16][95]
\end{Arabic}}
\flushleft{\begin{malayalam}
അല്ലാഹുവിന്‍റെ കരാറിനു പകരം നിങ്ങള്‍ തുച്ഛമായ വില വാങ്ങരുത്‌. തീര്‍ച്ചയായും അല്ലാഹുവിങ്കലുള്ളതു തന്നെയാണ് നിങ്ങള്‍ക്ക് ഉത്തമം; നിങ്ങള്‍ (കാര്യം) ഗ്രഹിക്കുന്നവരാണെങ്കില്‍.
\end{malayalam}}
\flushright{\begin{Arabic}
\quranayah[16][96]
\end{Arabic}}
\flushleft{\begin{malayalam}
നിങ്ങളുടെ അടുക്കലുള്ളത് തീര്‍ന്ന് പോകും. അല്ലാഹുവിങ്കലുള്ളത് അവശേഷിക്കുന്നതത്രെ. ക്ഷമ കൈക്കൊണ്ടവര്‍ക്ക് അവര്‍ പ്രവര്‍ത്തിച്ച് കൊണ്ടിരുന്നതില്‍ ഏറ്റവും ഉത്തമമായതിന് അനുസൃതമായി അവര്‍ക്കുള്ള പ്രതിഫലം നാം നല്‍കുക തന്നെ ചെയ്യും.
\end{malayalam}}
\flushright{\begin{Arabic}
\quranayah[16][97]
\end{Arabic}}
\flushleft{\begin{malayalam}
ഏതൊരു ആണോ പെണ്ണോ സത്യവിശ്വാസിയായിക്കൊണ്ട് സല്‍കര്‍മ്മം പ്രവര്‍ത്തിക്കുന്ന പക്ഷം നല്ലൊരു ജീവിതം തീര്‍ച്ചയായും ആ വ്യക്തിക്ക് നാം നല്‍കുന്നതാണ്‌. അവര്‍ പ്രവര്‍ത്തിച്ച് കൊണ്ടിരുന്നതില്‍ ഏറ്റവും ഉത്തമമായതിന് അനുസൃതമായി അവര്‍ക്കുള്ള പ്രതിഫലം തീര്‍ച്ചയായും നാം അവര്‍ക്ക് നല്‍കുകയും ചെയ്യും.
\end{malayalam}}
\flushright{\begin{Arabic}
\quranayah[16][98]
\end{Arabic}}
\flushleft{\begin{malayalam}
നീ ഖുര്‍ആന്‍ പാരായണം ചെയ്യുകയാണെങ്കില്‍ ശപിക്കപ്പെട്ട പിശാചില്‍ നിന്ന് അല്ലാഹുവോട് ശരണം തേടിക്കൊള്ളുക.
\end{malayalam}}
\flushright{\begin{Arabic}
\quranayah[16][99]
\end{Arabic}}
\flushleft{\begin{malayalam}
വിശ്വസിക്കുകയും, തങ്ങളുടെ രക്ഷിതാവിന്‍റെ മേല്‍ ഭരമേല്‍പിക്കുകയും ചെയ്യുന്നവരാരോ അവരുടെ മേല്‍ അവന്ന് (പിശാചിന്‌) യാതൊരു അധികാരവുമില്ല; തീര്‍ച്ച.
\end{malayalam}}
\flushright{\begin{Arabic}
\quranayah[16][100]
\end{Arabic}}
\flushleft{\begin{malayalam}
അവന്‍റെ അധികാരം അവനെ രക്ഷാധികാരിയാക്കുന്നവരുടെയും അല്ലാഹുവോട് പങ്കുചേര്‍ക്കുന്നവരുടെയും മേല്‍ മാത്രമാകുന്നു.
\end{malayalam}}
\flushright{\begin{Arabic}
\quranayah[16][101]
\end{Arabic}}
\flushleft{\begin{malayalam}
ഒരു വേദവാക്യത്തിന്‍റെ സ്ഥാനത്ത് മറ്റൊരു വേദവാക്യം നാം പകരം വെച്ചാല്‍ - അല്ലാഹുവാകട്ടെ താന്‍ അവതരിപ്പിക്കുന്നതിനെപ്പറ്റി നല്ലവണ്ണം അറിയുന്നവനാണ് താനും - അവര്‍ പറയും: നീ കെട്ടിച്ചമച്ചു പറയുന്നവന്‍ മാത്രമാകുന്നു എന്ന്‌. അല്ല, അവരില്‍ അധികപേരും (കാര്യം) മനസ്സിലാക്കുന്നില്ല.
\end{malayalam}}
\flushright{\begin{Arabic}
\quranayah[16][102]
\end{Arabic}}
\flushleft{\begin{malayalam}
പറയുക: വിശ്വസിച്ചവരെ ഉറപ്പിച്ച് നിര്‍ത്താന്‍ വേണ്ടിയും, കീഴ്പെട്ടുജീവിക്കുന്നവര്‍ക്ക് മാര്‍ഗദര്‍ശനവും സന്തോഷവാര്‍ത്തയും ആയിക്കൊണ്ടും സത്യപ്രകാരം പരിശുദ്ധാത്മാവ് നിന്‍റെ രക്ഷിതാവിങ്കല്‍ നിന്ന് അത് ഇറക്കിയിരിക്കുകയാണ്‌.
\end{malayalam}}
\flushright{\begin{Arabic}
\quranayah[16][103]
\end{Arabic}}
\flushleft{\begin{malayalam}
ഒരു മനുഷ്യന്‍ തന്നെയാണ് അദ്ദേഹത്തിന് (നബിക്ക്‌) പഠിപ്പിച്ചുകൊടുക്കുന്നത് എന്ന് അവര്‍ പറയുന്നുണ്ടെന്ന് തീര്‍ച്ചയായും നമുക്കറിയാം. അവര്‍ ദുസ്സൂചന നടത്തിക്കൊണ്ടിരിക്കുന്നത് ഏതൊരാളെപ്പറ്റിയാണോ ആ ആളുടെ ഭാഷ അനറബിയാകുന്നു. ഇതാകട്ടെ സ്പഷ്ടമായ അറബി ഭാഷയാകുന്നു.
\end{malayalam}}
\flushright{\begin{Arabic}
\quranayah[16][104]
\end{Arabic}}
\flushleft{\begin{malayalam}
അല്ലാഹുവിന്‍റെ ദൃഷ്ടാന്തങ്ങളില്‍ വിശ്വസിക്കാത്തവരാരോ അവരെ അല്ലാഹു നേര്‍വഴിയിലാക്കുകയില്ല; തീര്‍ച്ച. അവര്‍ക്ക് വേദനാജനകമായ ശിക്ഷയുണ്ടായിരിക്കുന്നതുമാണ്‌.
\end{malayalam}}
\flushright{\begin{Arabic}
\quranayah[16][105]
\end{Arabic}}
\flushleft{\begin{malayalam}
അല്ലാഹുവിന്‍റെ ദൃഷ്ടാന്തങ്ങളില്‍ വിശ്വസിക്കാത്തവര്‍ തന്നെയാണ് കള്ളം കെട്ടിച്ചമയ്ക്കുന്നത്‌. അവര്‍ തന്നെയാണ് വ്യാജവാദികള്‍.
\end{malayalam}}
\flushright{\begin{Arabic}
\quranayah[16][106]
\end{Arabic}}
\flushleft{\begin{malayalam}
വിശ്വസിച്ചതിന് ശേഷം അല്ലാഹുവില്‍ അവിശ്വസിച്ചവരാരോ അവരുടെ -തങ്ങളുടെ ഹൃദയം വിശ്വാസത്തില്‍ സമാധാനം പൂണ്ടതായിരിക്കെ നിര്‍ബന്ധിക്കപ്പെട്ടവരല്ല; പ്രത്യുത, തുറന്ന മനസ്സോടെ അവിശ്വാസം സ്വീകരിച്ചവരാരോ അവരുടെ- മേല്‍ അല്ലാഹുവിങ്കല്‍ നിന്നുള്ള കോപമുണ്ടായിരിക്കും. അവര്‍ക്ക് ഭയങ്കരമായ ശിക്ഷയുമുണ്ടായിരിക്കും.
\end{malayalam}}
\flushright{\begin{Arabic}
\quranayah[16][107]
\end{Arabic}}
\flushleft{\begin{malayalam}
അതെന്തുകൊണ്ടെന്നാല്‍ ഇഹലോകജീവിതത്തെ പരലോകത്തേക്കാള്‍ കൂടുതല്‍ അവര്‍ ഇഷ്ടപ്പെട്ടിരിക്കുന്നു. അല്ലാഹുവാകട്ടെ സത്യനിഷേധികളായ ആളുകളെ നേര്‍വഴിയിലാക്കുന്നതുമല്ല.
\end{malayalam}}
\flushright{\begin{Arabic}
\quranayah[16][108]
\end{Arabic}}
\flushleft{\begin{malayalam}
ഹൃദയങ്ങള്‍ക്കും കേള്‍വിക്കും കാഴ്ചകള്‍ക്കും അല്ലാഹു മുദ്രവെച്ചിട്ടുള്ള ഒരു വിഭാഗമാകുന്നു അക്കൂട്ടര്‍. അക്കൂട്ടര്‍ തന്നെയാകുന്നു അശ്രദ്ധര്‍.
\end{malayalam}}
\flushright{\begin{Arabic}
\quranayah[16][109]
\end{Arabic}}
\flushleft{\begin{malayalam}
ഒട്ടും സംശയമില്ല. അവര്‍ തന്നെയാണ് പരലോകത്ത് നഷ്ടക്കാര്‍.
\end{malayalam}}
\flushright{\begin{Arabic}
\quranayah[16][110]
\end{Arabic}}
\flushleft{\begin{malayalam}
പിന്നെ, തീര്‍ച്ചയായും നിന്‍റെ രക്ഷിതാവിന്‍റെ സഹായം മര്‍ദ്ദനത്തിന് ഇരയായ ശേഷം സ്വദേശം വെടിഞ്ഞ് പോകുകയും, അനന്തരം സമരത്തില്‍ ഏര്‍പെടുകയും, ക്ഷമിക്കുകയും ചെയ്തവര്‍ക്കായിരിക്കും. തീര്‍ച്ചയായും നിന്‍റെ രക്ഷിതാവ് അതിനു ശേഷം ഏറെ പൊറുക്കുന്നവനും കരുണാനിധിയുമാകുന്നു.
\end{malayalam}}
\flushright{\begin{Arabic}
\quranayah[16][111]
\end{Arabic}}
\flushleft{\begin{malayalam}
ഓരോ വ്യക്തിയും തന്‍റെ സ്വന്തം കാര്യത്തിനായി വാദിച്ച് കൊണ്ടുവരുന്ന, ഓരോ വ്യക്തിക്കും താന്‍ പ്രവര്‍ത്തിച്ചതെന്തോ അത് നിറവേറ്റികൊടുക്കപ്പെടുന്ന, അവര്‍ അനീതിക്ക് വിധേയരാകാത്ത ഒരു ദിവസത്തില്‍.
\end{malayalam}}
\flushright{\begin{Arabic}
\quranayah[16][112]
\end{Arabic}}
\flushleft{\begin{malayalam}
അല്ലാഹു ഒരു രാജ്യത്തെ ഉപമയായി എടുത്തുകാണിക്കുകയാകുന്നു. അത് സുരക്ഷിതവും ശാന്തവുമായിരുന്നു. അതിലെ ആവശ്യത്തിനുള്ള ഭക്ഷണം എല്ലായിടത്തുനിന്നും സമൃദ്ധമായി അവിടെ എത്തിക്കൊണ്ടിരിക്കും. എന്നിട്ട് ആ രാജ്യം അല്ലാഹുവിന്‍റെ അനുഗ്രഹങ്ങളെ നിഷേധിച്ചു. അപ്പോള്‍ അവര്‍ പ്രവര്‍ത്തിച്ച് കൊണ്ടിരുന്നത് നിമിത്തം വിശപ്പിന്‍റെയും ഭയത്തിന്‍റെയും ഉടുപ്പ് അല്ലാഹു ആ രാജ്യത്തിന് അനുഭവിക്കുമാറാക്കി.
\end{malayalam}}
\flushright{\begin{Arabic}
\quranayah[16][113]
\end{Arabic}}
\flushleft{\begin{malayalam}
അവരുടെ കൂട്ടത്തില്‍ പെട്ട ഒരു ദൂതന്‍ അവരുടെ അടുത്ത് ചെല്ലുകയുണ്ടായിട്ടുണ്ട്‌. അപ്പോള്‍ അവര്‍ അദ്ദേഹത്തെ നിഷേധിച്ചുതള്ളിക്കളഞ്ഞു. അങ്ങനെ അവര്‍ അക്രമകാരികളായിരിക്കെ ശിക്ഷ അവരെ പിടികൂടി.
\end{malayalam}}
\flushright{\begin{Arabic}
\quranayah[16][114]
\end{Arabic}}
\flushleft{\begin{malayalam}
ആകയാല്‍ അല്ലാഹു നിങ്ങള്‍ക്ക് നല്‍കിയിട്ടുള്ളതില്‍ നിന്ന് അനുവദനീയവും വിശിഷ്ടവുമായിട്ടുള്ളത് നിങ്ങള്‍ തിന്നുകൊള്ളുക. അല്ലാഹുവിന്‍റെ അനുഗ്രഹത്തിന് നിങ്ങള്‍ നന്ദികാണിക്കുകയും ചെയ്യുക; നിങ്ങള്‍ അവനെയാണ് ആരാധിക്കുന്നതെങ്കില്‍.
\end{malayalam}}
\flushright{\begin{Arabic}
\quranayah[16][115]
\end{Arabic}}
\flushleft{\begin{malayalam}
ശവം, രക്തം, പന്നിമാംസം, അല്ലാഹു അല്ലാത്തവരുടെ പേരില്‍ പ്രഖ്യാപിക്കപ്പെട്ടത് എന്നിവ മാത്രമേ അവന്‍ (അല്ലാഹു) നിങ്ങളുടെ മേല്‍ നിഷിദ്ധമാക്കിയിട്ടുള്ളൂ. വല്ലവനും (ഇവ ഭക്ഷിക്കുവാന്‍) നിര്‍ബന്ധിതനാകുന്ന പക്ഷം, അവന്‍ അതിന് ആഗ്രഹം കാണിക്കുന്നവനോ അതിരുവിട്ട് തിന്നുന്നവനോ അല്ലെങ്കില്‍ തീര്‍ച്ചയായും അല്ലാഹു ഏറെ പൊറുക്കുന്നവനും കരുണാനിധിയുമാകുന്നു.
\end{malayalam}}
\flushright{\begin{Arabic}
\quranayah[16][116]
\end{Arabic}}
\flushleft{\begin{malayalam}
നിങ്ങളുടെ നാവുകള്‍ വിശേഷിപ്പിക്കുന്നതിന്‍റെ അടിസ്ഥാനത്തില്‍ ഇത് അനുവദനീയമാണ്‌, ഇത് നിഷിദ്ധമാണ്‌. എന്നിങ്ങനെ കള്ളം പറയരുത്‌. നിങ്ങള്‍ അല്ലാഹുവിന്‍റെ പേരില്‍ കള്ളം കെട്ടിച്ചമയ്ക്കുകയത്രെ (അതിന്‍റെ ഫലം) അല്ലാഹുവിന്‍റെ പേരില്‍ കള്ളം കെട്ടിച്ചമയ്ക്കുന്നവര്‍ വിജയിക്കുകയില്ല; തീര്‍ച്ച.
\end{malayalam}}
\flushright{\begin{Arabic}
\quranayah[16][117]
\end{Arabic}}
\flushleft{\begin{malayalam}
തുച്ഛമായ സുഖാനുഭവമാണ് (ഇപ്പോള്‍ അവര്‍ക്കുള്ളത്‌.) അവര്‍ക്ക് (വരാനുള്ളതാകട്ടെ) വേദനയേറിയ ശിക്ഷയും.
\end{malayalam}}
\flushright{\begin{Arabic}
\quranayah[16][118]
\end{Arabic}}
\flushleft{\begin{malayalam}
മുമ്പ് നാം നിനക്ക് വിവരിച്ചുതന്നവ ജൂതന്‍മാരുടെ മേല്‍ നാം നിഷിദ്ധമാക്കുകയുണ്ടായി. നാം അവരോട് അനീതി ചെയ്തിട്ടില്ല. പക്ഷെ, അവര്‍ അവരോട് തന്നെ അനീതി ചെയ്യുകയായിരുന്നു.
\end{malayalam}}
\flushright{\begin{Arabic}
\quranayah[16][119]
\end{Arabic}}
\flushleft{\begin{malayalam}
പിന്നെ തീര്‍ച്ചയായും നിന്‍റെ രക്ഷിതാവ്‌, അവിവേകം മൂലം തിന്‍മ പ്രവര്‍ത്തിക്കുകയും പിന്നീട് അതിന് ശേഷം ഖേദിച്ചുമടങ്ങുകയും (ജീവിതം) നന്നാക്കിത്തീര്‍ക്കുകയും ചെയ്തവര്‍ക്ക് (വിട്ടുവീഴ്ച ചെയ്യുന്നവനാകുന്നു.) തീര്‍ച്ചയായും നിന്‍റെ രക്ഷിതാവ് അതിന് ശേഷം ഏറെ പൊറുക്കുന്നവനും കരുണാനിധിയുമാകുന്നു.
\end{malayalam}}
\flushright{\begin{Arabic}
\quranayah[16][120]
\end{Arabic}}
\flushleft{\begin{malayalam}
തീര്‍ച്ചയായും ഇബ്രാഹീം അല്ലാഹുവിന്ന് കീഴ്പെട്ട് ജീവിക്കുന്ന, നേര്‍വഴിയില്‍ (വ്യതിചലിക്കാതെ) നിലകൊള്ളുന്ന ഒരു സമുദായം തന്നെയായിരുന്നു. അദ്ദേഹം ബഹുദൈവവാദികളില്‍ പെട്ടവനായിരുന്നില്ല.
\end{malayalam}}
\flushright{\begin{Arabic}
\quranayah[16][121]
\end{Arabic}}
\flushleft{\begin{malayalam}
അവന്‍റെ (അല്ലാഹുവിന്‍റെ) അനുഗ്രഹങ്ങള്‍ക്ക് നന്ദികാണിക്കുന്നവനായിരുന്നുഅദ്ദേഹം. അദ്ദേഹത്തെ അവന്‍ തെരഞ്ഞെടുക്കുകയും നേരായ പാതയിലേക്ക് നയിക്കുകയും ചെയ്തു.
\end{malayalam}}
\flushright{\begin{Arabic}
\quranayah[16][122]
\end{Arabic}}
\flushleft{\begin{malayalam}
ഇഹലോകത്ത് അദ്ദേഹത്തിന് നാം നന്‍മ നല്‍കുകയും ചെയ്തിരിക്കുന്നു. പരലോകത്താകട്ടെ തീര്‍ച്ചയായും അദ്ദേഹം സദ്‌വൃത്തരുടെ കൂട്ടത്തിലായിരിക്കും.
\end{malayalam}}
\flushright{\begin{Arabic}
\quranayah[16][123]
\end{Arabic}}
\flushleft{\begin{malayalam}
പിന്നീട്‌, നേര്‍വഴിയില്‍ (വ്യതിചലിക്കാതെ) നിലകൊള്ളുന്നവനായിരുന്ന ഇബ്രാഹീമിന്‍റെ മാര്‍ഗത്തെ പിന്തുടരണം എന്ന് നിനക്ക് ഇതാ ബോധനം നല്‍കിയിരിക്കുന്നു. അദ്ദേഹം ബഹുദൈവവാദികളില്‍ പെട്ടവനായിരുന്നില്ല.
\end{malayalam}}
\flushright{\begin{Arabic}
\quranayah[16][124]
\end{Arabic}}
\flushleft{\begin{malayalam}
ശബ്ബത്ത് ദിനാചരണം നിശ്ചയിക്കപ്പെട്ടിട്ടുള്ളത് അതിന്‍റെ കാര്യത്തില്‍ ഭിന്നിച്ചു കഴിഞ്ഞിട്ടുള്ളവരാരോ അവരുടെ മേല്‍ തന്നെയാണ്‌. അവര്‍ ഭിന്നിച്ചിരുന്ന വിഷയത്തില്‍ ഉയിര്‍ത്തെഴുന്നേല്‍പിന്‍റെ നാളില്‍ തീര്‍ച്ചയായും നിന്‍റെ രക്ഷിതാവ് അവര്‍ക്കിടയില്‍ തീര്‍പ്പുകല്‍പിക്കുക തന്നെ ചെയ്യും.
\end{malayalam}}
\flushright{\begin{Arabic}
\quranayah[16][125]
\end{Arabic}}
\flushleft{\begin{malayalam}
യുക്തിദീക്ഷയോടു കൂടിയും, സദുപദേശം മുഖേനയും നിന്‍റെ രക്ഷിതാവിന്‍റെ മാര്‍ഗത്തിലേക്ക് നീ ക്ഷണിച്ച് കൊള്ളുക. ഏറ്റവും നല്ല രീതിയില്‍ അവരുമായി സംവാദം നടത്തുകയും ചെയ്യുക. തീര്‍ച്ചയായും നിന്‍റെ രക്ഷിതാവ് തന്‍റെ മാര്‍ഗം വിട്ട് പിഴച്ച് പോയവരെപ്പറ്റി നല്ലവണ്ണം അറിയുന്നവനത്രെ. സന്‍മാര്‍ഗം പ്രാപിച്ചവരെപ്പറ്റിയും നല്ലവണ്ണം അറിയുന്നവനത്രെ.
\end{malayalam}}
\flushright{\begin{Arabic}
\quranayah[16][126]
\end{Arabic}}
\flushleft{\begin{malayalam}
നിങ്ങള്‍ ശിക്ഷാനടപടി സ്വീകരിക്കുകയാണെങ്കില്‍ (എതിരാളികളില്‍ നിന്ന്‌) നിങ്ങളുടെ നേരെയുണ്ടായ ശിക്ഷാനടപടിക്ക് തുല്യമായ നടപടി നിങ്ങള്‍ സ്വീകരിച്ച് കൊള്ളുക. നിങ്ങള്‍ ക്ഷമിക്കുകയാണെങ്കിലോ അതു തന്നെയാണ് ക്ഷമാശീലര്‍ക്ക് കൂടുതല്‍ ഉത്തമം.
\end{malayalam}}
\flushright{\begin{Arabic}
\quranayah[16][127]
\end{Arabic}}
\flushleft{\begin{malayalam}
നീ ക്ഷമിക്കുക. അല്ലാഹുവിന്‍റെ അനുഗ്രഹത്താല്‍ മാത്രമാണ് നിനക്ക് ക്ഷമിക്കാന്‍ കഴിയുന്നത്‌. അവരുടെ (സത്യനിഷേധികളുടെ) പേരില്‍ നീ വ്യസനിക്കരുത്‌. അവര്‍ കുതന്ത്രം പ്രയോഗിക്കുന്നതിനെപ്പറ്റി നീ മനഃക്ലേശത്തിലാവുകയും അരുത്‌.
\end{malayalam}}
\flushright{\begin{Arabic}
\quranayah[16][128]
\end{Arabic}}
\flushleft{\begin{malayalam}
തീര്‍ച്ചയായും അല്ലാഹു സൂക്ഷ്മത പാലിച്ചവരോടൊപ്പമാകുന്നു. സദ്‌വൃത്തരായിട്ടുള്ളവരോടൊപ്പവും.
\end{malayalam}}
\chapter{\textmalayalam{ഇസ്റാഅ് ( നിശായാത്ര )}}
\begin{Arabic}
\Huge{\centerline{\basmalah}}\end{Arabic}
\flushright{\begin{Arabic}
\quranayah[17][1]
\end{Arabic}}
\flushleft{\begin{malayalam}
തന്‍റെ ദാസനെ (നബിയെ) ഒരു രാത്രിയില്‍ മസ്ജിദുല്‍ ഹറാമില്‍ നിന്ന് മസ്ജിദുല്‍ അഖ്സായിലേക്ക് - അതിന്‍റെ പരിസരം നാം അനുഗൃഹീതമാക്കിയിരിക്കുന്നു- നിശായാത്ര ചെയ്യിച്ചവന്‍ എത്രയോ പരിശുദ്ധന്‍! നമ്മുടെ ദൃഷ്ടാന്തങ്ങളില്‍ ചിലത് അദ്ദേഹത്തിന് നാം കാണിച്ചുകൊടുക്കാന്‍ വേണ്ടിയത്രെ അത്‌. തീര്‍ച്ചയായും അവന്‍ (അല്ലാഹു) എല്ലാം കേള്‍ക്കുന്നവനും കാണുന്നവനുമത്രെ.
\end{malayalam}}
\flushright{\begin{Arabic}
\quranayah[17][2]
\end{Arabic}}
\flushleft{\begin{malayalam}
മൂസായ്ക്ക് നാം വേദഗ്രന്ഥം നല്‍കുകയും, അതിനെ നാം ഇസ്രായീല്‍ സന്തതികള്‍ക്ക് മാര്‍ഗദര്‍ശകമാക്കുകയും ചെയ്തു. എനിക്കു പുറമെ യാതൊരു കൈകാര്യകര്‍ത്താവിനെയും നിങ്ങള്‍ സ്വീകരിക്കരുത് എന്ന് (അനുശാസിക്കുന്ന വേദഗ്രന്ഥം).
\end{malayalam}}
\flushright{\begin{Arabic}
\quranayah[17][3]
\end{Arabic}}
\flushleft{\begin{malayalam}
നൂഹിനോടൊപ്പം നാം കപ്പലില്‍ കയറ്റിയവരുടെ സന്തതികളേ, തീര്‍ച്ചയായും അദ്ദേഹം (നൂഹ്‌) വളരെ നന്ദിയുള്ള ഒരു ദാസനായിരുന്നു.
\end{malayalam}}
\flushright{\begin{Arabic}
\quranayah[17][4]
\end{Arabic}}
\flushleft{\begin{malayalam}
ഇസ്രായീല്‍ സന്തതികള്‍ക്ക് ഇപ്രകാരം നാം വേദഗ്രന്ഥത്തില്‍ വിധി നല്‍കിയിരിക്കുന്നു: തീര്‍ച്ചയായും നിങ്ങള്‍ ഭൂമിയില്‍ രണ്ട് പ്രാവശ്യം കുഴപ്പമുണ്ടാക്കുകയും വലിയ ഔന്നത്യം നടിക്കുകയും ചെയ്യുന്നതാണ്‌.
\end{malayalam}}
\flushright{\begin{Arabic}
\quranayah[17][5]
\end{Arabic}}
\flushleft{\begin{malayalam}
അങ്ങനെ ആ രണ്ട് സന്ദര്‍ഭങ്ങളില്‍ ഒന്നാമത്തേതിന്ന് നിശ്ചയിച്ച (ശിക്ഷയുടെ) സമയമായാല്‍ ഉഗ്രപരാക്രമശാലികളായ നമ്മുടെ ചില ദാസന്‍മാരെ നിങ്ങളുടെ നേരെ നാം അയക്കുന്നതാണ്‌. അങ്ങനെ അവര്‍ വീടുകള്‍ക്കിടയില്‍ (നിങ്ങളെ) തെരഞ്ഞു നടക്കും. അത് പ്രാവര്‍ത്തികമാക്കപ്പെട്ട ഒരു വാഗ്ദാനം തന്നെയാകുന്നു.
\end{malayalam}}
\flushright{\begin{Arabic}
\quranayah[17][6]
\end{Arabic}}
\flushleft{\begin{malayalam}
പിന്നെ നാം അവര്‍ക്കെതിരില്‍ നിങ്ങള്‍ക്ക് വിജയം തിരിച്ചുതന്നു. സ്വത്തുക്കളും സന്താനങ്ങളും കൊണ്ട് നിങ്ങളെ നാം പോഷിപ്പിക്കുകയും നിങ്ങളെ നാം കൂടുതല്‍ സംഘബലമുള്ളവരാക്കിത്തീര്‍ക്കുകയും ചെയ്തു.
\end{malayalam}}
\flushright{\begin{Arabic}
\quranayah[17][7]
\end{Arabic}}
\flushleft{\begin{malayalam}
നിങ്ങള്‍ നന്‍മ പ്രവര്‍ത്തിക്കുന്ന പക്ഷം നിങ്ങളുടെ ഗുണത്തിനായി തന്നെയാണ് നിങ്ങള്‍ നന്‍മ പ്രവര്‍ത്തിക്കുന്നത്‌. നിങ്ങള്‍ തിന്‍മ പ്രവര്‍ത്തിക്കുകയാണെങ്കില്‍ (അതിന്‍റെ ദോഷവും) നിങ്ങള്‍ക്കു തന്നെ. എന്നാല്‍ (ആ രണ്ട് സന്ദര്‍ഭങ്ങളില്‍) അവസാനത്തേതിന് നിശ്ചയിച്ച (ശിക്ഷയുടെ) സമയം വന്നാല്‍ നിങ്ങളുടെ മുഖങ്ങളെ അപമാനത്തിലാഴ്ത്തുവാനും, ആദ്യതവണ ആരാധനാലയത്തില്‍ പ്രവേശിച്ചത് പോലെ വീണ്ടും പ്രവേശിക്കുവാനും കീഴടക്കിയതെല്ലാം തകര്‍ത്ത് കളയുവാനും (നാം ശത്രുക്കളെ നിയോഗിക്കുന്നതാണ്‌.)
\end{malayalam}}
\flushright{\begin{Arabic}
\quranayah[17][8]
\end{Arabic}}
\flushleft{\begin{malayalam}
നിങ്ങളുടെ രക്ഷിതാവ് നിങ്ങളോട് കരുണ കാണിക്കുന്നവനായേക്കാം. നിങ്ങള്‍ ആവര്‍ത്തിക്കുന്ന പക്ഷം നമ്മളും ആവര്‍ത്തിക്കുന്നതാണ്‌. നരകത്തെ നാം സത്യനിഷേധികള്‍ക്ക് ഒരു തടവറ ആക്കിയിരിക്കുന്നു.
\end{malayalam}}
\flushright{\begin{Arabic}
\quranayah[17][9]
\end{Arabic}}
\flushleft{\begin{malayalam}
തീര്‍ച്ചയായും ഈ ഖുര്‍ആന്‍ ഏറ്റവും ശരിയായതിലേക്ക് വഴി കാണിക്കുകയും, സല്‍കര്‍മ്മങ്ങള്‍ പ്രവര്‍ത്തിക്കുന്ന സത്യവിശ്വാസികള്‍ക്ക് വലിയ പ്രതിഫലമുണ്ട് എന്ന സന്തോഷവാര്‍ത്ത അറിയിക്കുകയും ചെയ്യുന്നു.
\end{malayalam}}
\flushright{\begin{Arabic}
\quranayah[17][10]
\end{Arabic}}
\flushleft{\begin{malayalam}
പരലോകത്തില്‍ വിശ്വസിക്കാത്തവരാരോ അവര്‍ക്ക് നാം വേദനയേറിയ ശിക്ഷ ഒരുക്കിവെച്ചിട്ടുണ്ട് എന്നും (സന്തോഷവാര്‍ത്ത അറിയിക്കുന്നു.)
\end{malayalam}}
\flushright{\begin{Arabic}
\quranayah[17][11]
\end{Arabic}}
\flushleft{\begin{malayalam}
മനുഷ്യന്‍ ഗുണത്തിന് വേണ്ടി പ്രാര്‍ത്ഥിക്കുന്നത് പോലെ തന്നെ ദോഷത്തിന് വേണ്ടിയും പ്രാര്‍ത്ഥിക്കുന്നു. മനുഷ്യന്‍ ഏറെ തിടുക്കം കൂട്ടുന്നവനായിരിക്കുന്നു.
\end{malayalam}}
\flushright{\begin{Arabic}
\quranayah[17][12]
\end{Arabic}}
\flushleft{\begin{malayalam}
രാവിനെയും പകലിനെയും നാം രണ്ട് ദൃഷ്ടാന്തങ്ങളാക്കിയിരിക്കുന്നു. രാവാകുന്ന ദൃഷ്ടാന്തത്തെ നാം മങ്ങിയതാക്കുകയും, പകലാകുന്ന ദൃഷ്ടാന്തത്തെ നാം പ്രകാശം നല്‍കുന്നതാക്കുകയും ചെയ്തിരിക്കുന്നു. നിങ്ങളുടെ രക്ഷിതാവിങ്കല്‍ നിന്നുള്ള അനുഗ്രഹം നിങ്ങള്‍ തേടുന്നതിന് വേണ്ടിയും, കൊല്ലങ്ങളുടെ എണ്ണവും കണക്കും നിങ്ങള്‍ മനസ്സിലാക്കുവാന്‍ വേണ്ടിയും. ഓരോ കാര്യവും നാം നല്ലവണ്ണം വിശദീകരിച്ചിരിക്കുന്നു.
\end{malayalam}}
\flushright{\begin{Arabic}
\quranayah[17][13]
\end{Arabic}}
\flushleft{\begin{malayalam}
ഓരോ മനുഷ്യന്നും അവന്‍റെ ശകുനം അവന്‍റെ കഴുത്തില്‍ തന്നെ നാം ബന്ധിച്ചിരിക്കുന്നു ഉയിര്‍ത്തെഴുന്നേല്‍പിന്‍റെ നാളില്‍ ഒരു ഗ്രന്ഥം നാമവന്ന് വേണ്ടി പുറത്തെടുക്കുന്നതാണ്‌. അത് നിവര്‍ത്തിവെക്കപ്പെട്ടതായി അവന്‍ കണ്ടെത്തും.
\end{malayalam}}
\flushright{\begin{Arabic}
\quranayah[17][14]
\end{Arabic}}
\flushleft{\begin{malayalam}
നീ നിന്‍റെ ഗ്രന്ഥം വായിച്ചുനോക്കുക. നിന്നെ സ്സംബന്ധിച്ചിടത്തോളം കണക്ക് നോക്കാന്‍ ഇന്ന് നീ തന്നെ മതി. (എന്ന് അവനോട് അന്ന് പറയപ്പെടും)
\end{malayalam}}
\flushright{\begin{Arabic}
\quranayah[17][15]
\end{Arabic}}
\flushleft{\begin{malayalam}
വല്ലവനും നേര്‍മാര്‍ഗം സ്വീകരിക്കുന്ന പക്ഷം തന്‍റെ സ്വന്തം ഗുണത്തിനായി തന്നെയാണ് അവന്‍ നേര്‍മാര്‍ഗം സ്വീകരിക്കുന്നത്‌. വല്ലവനും വഴിപിഴച്ച് പോകുന്ന പക്ഷം തനിക്ക് ദോഷത്തിനായി തന്നെയാണ് അവന്‍ വഴിപിഴച്ചു പോകുന്നത്‌. പാപഭാരം ചുമക്കുന്ന യാതൊരാളും മറ്റൊരാളുടെ പാപഭാരം ചുമക്കുകയില്ല. ഒരു ദൂതനെ അയക്കുന്നത് വരെ നാം (ആരെയും) ശിക്ഷിക്കുന്നതുമല്ല.
\end{malayalam}}
\flushright{\begin{Arabic}
\quranayah[17][16]
\end{Arabic}}
\flushleft{\begin{malayalam}
ഏതെങ്കിലും ഒരു രാജ്യം നാം നശിപ്പിക്കാന്‍ ഉദ്ദേശിച്ചാല്‍ അവിടത്തെ സുഖലോലുപന്‍മാര്‍ക്ക് നാം ആജ്ഞകള്‍ നല്‍കും. എന്നാല്‍ (അത് വകവെക്കാതെ) അവര്‍ അവിടെ താന്തോന്നിത്തം നടത്തും. (ശിക്ഷയെപ്പറ്റിയുള്ള) വാക്ക് അങ്ങനെ അതിന്‍റെ (രാജ്യത്തിന്‍റെ) കാര്യത്തില്‍ സ്ഥിരപ്പെടുകയും, നാം അതിനെ നിശ്ശേഷം തകര്‍ക്കുകയും ചെയ്യുന്നതാണ്‌.
\end{malayalam}}
\flushright{\begin{Arabic}
\quranayah[17][17]
\end{Arabic}}
\flushleft{\begin{malayalam}
നൂഹിന്‍റെ ശേഷം എത്രയെത്ര തലമുറകളെ നാം നശിപ്പിച്ചിട്ടുണ്ട്‌! തന്‍റെ ദാസന്‍മാരുടെ പാപങ്ങളെ സംബന്ധിച്ച് സൂക്ഷ്മമായി അറിയുന്നവനും കാണുന്നവനുമായി നിന്‍റെ രക്ഷിതാവ് തന്നെ മതി.
\end{malayalam}}
\flushright{\begin{Arabic}
\quranayah[17][18]
\end{Arabic}}
\flushleft{\begin{malayalam}
ക്ഷണികമായതിനെ (ഇഹലോകത്തെ) യാണ് വല്ലവരും ഉദ്ദേശിക്കുന്നതെങ്കില്‍ അവര്‍ക്ക് അഥവാ (അവരില്‍ നിന്ന്‌) നാം ഉദ്ദേശിക്കുന്നവര്‍ക്ക് നാം ഉദ്ദേശിക്കുന്നത് ഇവിടെ വെച്ച് തന്നെ വേഗത്തില്‍ നല്‍കുന്നതാണ്‌. പിന്നെ നാം അങ്ങനെയുള്ളവന്ന് നല്‍കുന്നത് നരകമായിരിക്കും. അപമാനിതനും പുറന്തള്ളപ്പെട്ടവനുമായിക്കൊണ്ട് അവന്‍ അതില്‍ കടന്നെരിയുന്നതാണ്‌.
\end{malayalam}}
\flushright{\begin{Arabic}
\quranayah[17][19]
\end{Arabic}}
\flushleft{\begin{malayalam}
ആരെങ്കിലും പരലോകം ഉദ്ദേശിക്കുകയും, സത്യവിശ്വാസിയായിക്കൊണ്ട് അതിന്നു വേണ്ടി അതിന്‍റെതായ പരിശ്രമം നടത്തുകയും ചെയ്യുന്ന പക്ഷം അത്തരക്കാരുടെ പരിശ്രമം പ്രതിഫലാര്‍ഹമായിരിക്കും.
\end{malayalam}}
\flushright{\begin{Arabic}
\quranayah[17][20]
\end{Arabic}}
\flushleft{\begin{malayalam}
ഇക്കൂട്ടരെയും അക്കൂട്ടരെയും എല്ലാം തന്നെ (ഇവിടെ വെച്ച്‌) നാം സഹായിക്കുന്നതാണ്‌. നിന്‍റെ രക്ഷിതാവിന്‍റെ ദാനത്തില്‍ പെട്ടതത്രെ അത്‌. നിന്‍റെ രക്ഷിതാവിന്‍റെ ദാനം തടഞ്ഞ് വെക്കപ്പെടുന്നതല്ല.
\end{malayalam}}
\flushright{\begin{Arabic}
\quranayah[17][21]
\end{Arabic}}
\flushleft{\begin{malayalam}
നാം അവരില്‍ ചിലരെ മറ്റുചിലരെക്കാള്‍ മെച്ചപ്പെട്ടവരാക്കിയിരിക്കുന്നത് എങ്ങനെയെന്ന് നോക്കൂ. പരലോകജീവിതം ഏറ്റവും വലിയ പദവിയുള്ളതും, ഏറ്റവും വലിയ ഉല്‍കൃഷ്ടതയുള്ളതും തന്നെയാകുന്നു.
\end{malayalam}}
\flushright{\begin{Arabic}
\quranayah[17][22]
\end{Arabic}}
\flushleft{\begin{malayalam}
അല്ലാഹുവോടൊപ്പം മറ്റൊരു ദൈവത്തെയും നീ സ്ഥാപിക്കരുത്‌. എങ്കില്‍ അപമാനിതനും കയ്യൊഴിക്കപ്പെട്ടവനുമായി നീ ഇരിക്കേണ്ടി വരും.
\end{malayalam}}
\flushright{\begin{Arabic}
\quranayah[17][23]
\end{Arabic}}
\flushleft{\begin{malayalam}
തന്നെയല്ലാതെ നിങ്ങള്‍ ആരാധിക്കരുതെന്നും, മാതാപിതാക്കള്‍ക്ക് നന്‍മചെയ്യണമെന്നും നിന്‍റെ രക്ഷിതാവ് വിധിച്ചിരിക്കുന്നു. അവരില്‍ (മാതാപിതാക്കളില്‍) ഒരാളോ അവര്‍ രണ്ട് പേരും തന്നെയോ നിന്‍റെ അടുക്കല്‍ വെച്ച് വാര്‍ദ്ധക്യം പ്രാപിക്കുകയാണെങ്കില്‍ അവരോട് നീ ഛെ എന്ന് പറയുകയോ, അവരോട് കയര്‍ക്കുകയോ ചെയ്യരുത്‌. അവരോട് നീ മാന്യമായ വാക്ക് പറയുക.
\end{malayalam}}
\flushright{\begin{Arabic}
\quranayah[17][24]
\end{Arabic}}
\flushleft{\begin{malayalam}
കാരുണ്യത്തോട് കൂടി എളിമയുടെ ചിറക് നീ അവര്‍ ഇരുവര്‍ക്കും താഴ്ത്തികൊടുക്കുകയും ചെയ്യുക. എന്‍റെ രക്ഷിതാവേ, ചെറുപ്പത്തില്‍ ഇവര്‍ ഇരുവരും എന്നെ പോറ്റിവളര്‍ത്തിയത് പോലെ ഇവരോട് നീ കരുണ കാണിക്കണമേ എന്ന് നീ പറയുകയും ചെയ്യുക.
\end{malayalam}}
\flushright{\begin{Arabic}
\quranayah[17][25]
\end{Arabic}}
\flushleft{\begin{malayalam}
നിങ്ങളുടെ രക്ഷിതാവ് നിങ്ങളുടെ മനസ്സുകളിലുള്ളത് നല്ലവണ്ണം അറിയുന്നവനാണ്‌. നിങ്ങള്‍ നല്ലവരായിരിക്കുന്ന പക്ഷം തീര്‍ച്ചയായും അവന്‍ ഖേദിച്ചുമടങ്ങുന്നവര്‍ക്ക് ഏറെ പൊറുത്തുകൊടുക്കുന്നവനാകുന്നു.
\end{malayalam}}
\flushright{\begin{Arabic}
\quranayah[17][26]
\end{Arabic}}
\flushleft{\begin{malayalam}
കുടുംബബന്ധമുള്ളവന്ന് അവന്‍റെ അവകാശം നീ നല്‍കുക. അഗതിക്കും വഴിപോക്കന്നും (അവരുടെ അവകാശവും) . നീ (ധനം) ദുര്‍വ്യയം ചെയ്ത് കളയരുത്‌.
\end{malayalam}}
\flushright{\begin{Arabic}
\quranayah[17][27]
\end{Arabic}}
\flushleft{\begin{malayalam}
തീര്‍ച്ചയായും ദുര്‍വ്യയം ചെയ്യുന്നവര്‍ പിശാചുക്കളുടെ സഹോദരങ്ങളാകുന്നു. പിശാച് തന്‍റെ രക്ഷിതാവിനോട് ഏറെ നന്ദികെട്ടവനാകുന്നു.
\end{malayalam}}
\flushright{\begin{Arabic}
\quranayah[17][28]
\end{Arabic}}
\flushleft{\begin{malayalam}
നിന്‍റെ രക്ഷിതാവിങ്കല്‍ നിന്ന് നീ ആഗ്രഹിക്കുന്ന അനുഗ്രഹം തേടിക്കൊണ്ട് നിനക്കവരില്‍ നിന്ന് തിരിഞ്ഞുകളയേണ്ടി വരുന്ന പക്ഷം, നീ അവരോട് സൌമ്യമായ വാക്ക് പറഞ്ഞ് കൊള്ളുക
\end{malayalam}}
\flushright{\begin{Arabic}
\quranayah[17][29]
\end{Arabic}}
\flushleft{\begin{malayalam}
നിന്‍റെ കൈ നീ പിരടിയിലേക്ക് ബന്ധിക്കപ്പെട്ടതാക്കരുത്‌. അത് (കൈ) മുഴുവനായങ്ങ് നീട്ടിയിടുകയും ചെയ്യരുത്‌. അങ്ങനെ ചെയ്യുന്ന പക്ഷം നീ നിന്ദിതനും കഷ്ടപ്പെട്ടവനുമായിരിക്കേണ്ടിവരും.
\end{malayalam}}
\flushright{\begin{Arabic}
\quranayah[17][30]
\end{Arabic}}
\flushleft{\begin{malayalam}
തീര്‍ച്ചയായും നിന്‍റെ രക്ഷിതാവ് താന്‍ ഉദ്ദേശിക്കുന്നവര്‍ക്ക് ഉപജീവനമാര്‍ഗം വിശാലമാക്കികൊടുക്കുന്നു. (ചിലര്‍ക്കത്‌) ഇടുങ്ങിയതാക്കുകയും ചെയ്യുന്നു. തീര്‍ച്ചയായും അവന്‍ തന്‍റെ ദാസന്‍മാരെപ്പറ്റി സൂക്ഷ്മമായി അറിയുന്നവനും കാണുന്നവനുമാകുന്നു.
\end{malayalam}}
\flushright{\begin{Arabic}
\quranayah[17][31]
\end{Arabic}}
\flushleft{\begin{malayalam}
ദാരിദ്യ്‌രഭയത്താല്‍ നിങ്ങള്‍ നിങ്ങളുടെ കുഞ്ഞുങ്ങളെ കൊന്നുകളയരുത്‌. നാമാണ് അവര്‍ക്കും നിങ്ങള്‍ക്കും ഉപജീവനം നല്‍കുന്നത്‌. അവരെ കൊല്ലുന്നത് തീര്‍ച്ചയായും ഭീമമായ അപരാധമാകുന്നു.
\end{malayalam}}
\flushright{\begin{Arabic}
\quranayah[17][32]
\end{Arabic}}
\flushleft{\begin{malayalam}
നിങ്ങള്‍ വ്യഭിചാരത്തെ സമീപിച്ച് പോകരുത്‌. തീര്‍ച്ചയായും അത് ഒരു നീചവൃത്തിയും ദുഷിച്ച മാര്‍ഗവുമാകുന്നു.
\end{malayalam}}
\flushright{\begin{Arabic}
\quranayah[17][33]
\end{Arabic}}
\flushleft{\begin{malayalam}
അല്ലാഹു പവിത്രത നല്‍കിയിട്ടുള്ള ജീവനെ ന്യായപ്രകാരമല്ലാതെ നിങ്ങള്‍ ഹനിക്കരുത്‌. അക്രമത്തിനു വിധേയനായി വല്ലവനും കൊല്ലപ്പെടുന്ന പക്ഷം അവന്‍റെ അവകാശിക്ക് നാം (പ്രതികാരം ചെയ്യാന്‍) അധികാരം വെച്ച് കൊടുത്തിട്ടുണ്ട്‌. എന്നാല്‍ അവന്‍ കൊലയില്‍ അതിരുകവിയരുത്‌. തീര്‍ച്ചയായും അവന്‍ സഹായിക്കപ്പെടുന്നവനാകുന്നു.
\end{malayalam}}
\flushright{\begin{Arabic}
\quranayah[17][34]
\end{Arabic}}
\flushleft{\begin{malayalam}
അനാഥയ്ക്ക് പ്രാപ്തി എത്തുന്നത് വരെ ഏറ്റവും നല്ല രീതിയിലല്ലാതെ അവന്‍റെ സ്വത്തിനെ നിങ്ങള്‍ സമീപിക്കരുത്‌. നിങ്ങള്‍ കരാര്‍ നിറവേറ്റുക. തീര്‍ച്ചയായും കരാറിനെപ്പറ്റി ചോദ്യം ചെയ്യപ്പെടുന്നതാണ്‌.
\end{malayalam}}
\flushright{\begin{Arabic}
\quranayah[17][35]
\end{Arabic}}
\flushleft{\begin{malayalam}
നിങ്ങള്‍ അളന്നുകൊടുക്കുകയാണെങ്കില്‍ അളവ് നിങ്ങള്‍ തികച്ചുകൊടുക്കുക. ശരിയായ തുലാസ് കൊണ്ട് നിങ്ങള്‍ തൂക്കികൊടുക്കുകയും ചെയ്യുക. അതാണ് ഉത്തമവും അന്ത്യഫലത്തില്‍ ഏറ്റവും മെച്ചമായിട്ടുള്ളതും.
\end{malayalam}}
\flushright{\begin{Arabic}
\quranayah[17][36]
\end{Arabic}}
\flushleft{\begin{malayalam}
നിനക്ക് അറിവില്ലാത്ത യാതൊരു കാര്യത്തിന്‍റെയും പിന്നാലെ നീ പോകരുത്‌. തീര്‍ച്ചയായും കേള്‍വി, കാഴ്ച, ഹൃദയം എന്നിവയെപ്പറ്റിയെല്ലാം ചോദ്യം ചെയ്യപ്പെടുന്നതാണ്‌.
\end{malayalam}}
\flushright{\begin{Arabic}
\quranayah[17][37]
\end{Arabic}}
\flushleft{\begin{malayalam}
നീ ഭൂമിയില്‍ അഹന്തയോടെ നടക്കരുത്‌. തീര്‍ച്ചയായും നിനക്ക് ഭൂമിയെ പിളര്‍ക്കാനൊന്നുമാവില്ല. ഉയരത്തില്‍ നിനക്ക് പര്‍വ്വതങ്ങള്‍ക്കൊപ്പമെത്താനും ആവില്ല, തീര്‍ച്ച.
\end{malayalam}}
\flushright{\begin{Arabic}
\quranayah[17][38]
\end{Arabic}}
\flushleft{\begin{malayalam}
അവയില്‍ (മേല്‍പറഞ്ഞ കാര്യങ്ങളില്‍) നിന്നെല്ലാം ദുഷിച്ചത് നിന്‍റെ രക്ഷിതാവിങ്കല്‍ വെറുക്കപ്പെട്ടതാകുന്നു.
\end{malayalam}}
\flushright{\begin{Arabic}
\quranayah[17][39]
\end{Arabic}}
\flushleft{\begin{malayalam}
നിന്‍റെ രക്ഷിതാവ് നിനക്ക് ബോധനം നല്‍കിയ ജ്ഞാനത്തില്‍ പെട്ടതത്രെ അത്‌. അല്ലാഹുവോടൊപ്പം മറ്റൊരു ദൈവത്തെയും നീ സ്ഥാപിക്കരുത്‌. എങ്കില്‍ ആക്ഷേപിക്കപ്പെട്ടവനും പുറം തള്ളപ്പെട്ടവനുമായി നീ നരകത്തില്‍ എറിയപ്പെടുന്നതാണ്‌.
\end{malayalam}}
\flushright{\begin{Arabic}
\quranayah[17][40]
\end{Arabic}}
\flushleft{\begin{malayalam}
എന്നാല്‍ നിങ്ങളുടെ രക്ഷിതാവ് ആണ്‍മക്കളെ നിങ്ങള്‍ക്കു പ്രത്യേകമായി നല്‍കുകയും, അവന്‍ മലക്കുകളില്‍ നിന്ന് പെണ്‍മക്കളെ സ്വീകരിക്കുകയും ചെയ്തിരിക്കുകയാണോ? തീര്‍ച്ചയായും ഗുരുതരമായ ഒരു വാക്ക് തന്നെയാകുന്നു നിങ്ങള്‍ പറയുന്നത്‌.
\end{malayalam}}
\flushright{\begin{Arabic}
\quranayah[17][41]
\end{Arabic}}
\flushleft{\begin{malayalam}
അവര്‍ ആലോചിച്ച് മനസ്സിലാക്കുവാന്‍ വേണ്ടി ഈ ഖുര്‍ആനില്‍ നാം (കാര്യങ്ങള്‍) വിവിധ രൂപത്തില്‍ വിവരിച്ചിട്ടുണ്ട്‌. എന്നാല്‍ അവര്‍ക്ക് അത് അകല്‍ച്ച വര്‍ദ്ധിപ്പിക്കുക മാത്രമാണ് ചെയ്യുന്നത്‌.
\end{malayalam}}
\flushright{\begin{Arabic}
\quranayah[17][42]
\end{Arabic}}
\flushleft{\begin{malayalam}
(നബിയേ,) പറയുക: അവര്‍ പറയും പോലെ അവനോടൊപ്പം മറ്റുദൈവങ്ങളുണ്ടായിരുന്നെങ്കില്‍ സിംഹാസനാധിപന്‍റെ അടുക്കലേക്ക് അവര്‍ (ആ ദൈവങ്ങള്‍) വല്ല മാര്‍ഗവും തേടുക തന്നെ ചെയ്യുമായിരുന്നു.
\end{malayalam}}
\flushright{\begin{Arabic}
\quranayah[17][43]
\end{Arabic}}
\flushleft{\begin{malayalam}
അവന്‍ എത്ര പരിശുദ്ധന്‍! അവര്‍ പറഞ്ഞുണ്ടാക്കിയതിനെല്ലാം ഉപരിയായി അവന്‍ വലിയ ഔന്നത്യം പ്രാപിച്ചിരിക്കുന്നു.
\end{malayalam}}
\flushright{\begin{Arabic}
\quranayah[17][44]
\end{Arabic}}
\flushleft{\begin{malayalam}
ഏഴ് ആകാശങ്ങളും ഭൂമിയും അവയിലുള്ളവരും അവന്‍റെ പരിശുദ്ധിയെ പ്രകീര്‍ത്തിക്കുന്നു യാതൊരു വസ്തുവും അവനെ സ്തുതിച്ച് കൊണ്ട് (അവന്‍റെ) പരിശുദ്ധിയെ പ്രകീര്‍ത്തിക്കാത്തതായി ഇല്ല. പക്ഷെ അവരുടെ കീര്‍ത്തനം നിങ്ങള്‍ ഗ്രഹിക്കുകയില്ല. തീര്‍ച്ചയായും അവന്‍ സഹനശീലനും ഏറെ പൊറുക്കുന്നവനുമാകുന്നു.
\end{malayalam}}
\flushright{\begin{Arabic}
\quranayah[17][45]
\end{Arabic}}
\flushleft{\begin{malayalam}
നീ ഖുര്‍ആന്‍ പാരായണം ചെയ്താല്‍ നിന്‍റെയും പരലോകത്തില്‍ വിശ്വസിക്കാത്തവരുടെയും ഇടയില്‍ ദൃശ്യമല്ലാത്ത ഒരു മറ നാം വെക്കുന്നതാണ്‌.
\end{malayalam}}
\flushright{\begin{Arabic}
\quranayah[17][46]
\end{Arabic}}
\flushleft{\begin{malayalam}
അവരത് ഗ്രഹിക്കുന്നതിന് (തടസ്സമായി) അവരുടെ ഹൃദയങ്ങളിന്‍മേല്‍ നാം മൂടികള്‍ വെക്കുന്നതും, അവരുടെ കാതുകളില്‍ നാം ഒരു തരം ഭാരം വെക്കുന്നതുമാണ്‌. ഖുര്‍ആന്‍ പാരായണത്തില്‍ നിന്‍റെ രക്ഷിതാവിനെപ്പറ്റി മാത്രം പ്രസ്താവിച്ചാല്‍ അവര്‍ വിറളിയെടുത്ത് പുറം തിരിഞ്ഞ് പോകുന്നതാണ്‌.
\end{malayalam}}
\flushright{\begin{Arabic}
\quranayah[17][47]
\end{Arabic}}
\flushleft{\begin{malayalam}
നീ പറയുന്നത് അവര്‍ ശ്രദ്ധിച്ച് കേള്‍ക്കുന്ന സമയത്ത് എന്തൊരു കാര്യമാണ് അവര്‍ ശ്രദ്ധിച്ച് കേട്ട് കൊണ്ടിരിക്കുന്നത് എന്ന് നമുക്ക് നല്ലവണ്ണം അറിയാം. അവര്‍ സ്വകാര്യം പറയുന്ന സന്ദര്‍ഭം അഥവാ മാരണം ബാധിച്ച ഒരാളെ മാത്രമാണ് നിങ്ങള്‍ പിന്തുടരുന്നത് എന്ന് (നിന്നെ പരിഹസിച്ചുകൊണ്ട്‌) അക്രമികള്‍ പറയുന്ന സന്ദര്‍ഭവും (നമുക്ക് നല്ലവണ്ണം അറിയാം.)
\end{malayalam}}
\flushright{\begin{Arabic}
\quranayah[17][48]
\end{Arabic}}
\flushleft{\begin{malayalam}
(നബിയേ,) നോക്കൂ; എങ്ങനെയാണ് അവര്‍ നിനക്ക് ഉപമകള്‍ പറഞ്ഞുണ്ടാക്കിയതെന്ന്‌. അങ്ങനെ അവര്‍ പിഴച്ചു പോയിരിക്കുന്നു. അതിനാല്‍ അവര്‍ക്ക് ഒരു മാര്‍ഗവും പ്രാപിക്കാന്‍ സാധിക്കുകയില്ല.
\end{malayalam}}
\flushright{\begin{Arabic}
\quranayah[17][49]
\end{Arabic}}
\flushleft{\begin{malayalam}
അവര്‍ പറഞ്ഞു: നാം എല്ലുകളും ജീര്‍ണാവശിഷ്ടങ്ങളുമായിക്കഴിഞ്ഞാല്‍ തീര്‍ച്ചയായും നാം പുതിയൊരു സൃഷ്ടിയായി ഉയിര്‍ത്തെഴുന്നേല്‍പിക്കപ്പെടുന്നതാണോ ?
\end{malayalam}}
\flushright{\begin{Arabic}
\quranayah[17][50]
\end{Arabic}}
\flushleft{\begin{malayalam}
(നബിയേ,) നീ പറയുക: നിങ്ങള്‍ കല്ലോ ഇരുമ്പോ ആയിക്കൊള്ളുക.
\end{malayalam}}
\flushright{\begin{Arabic}
\quranayah[17][51]
\end{Arabic}}
\flushleft{\begin{malayalam}
അല്ലെങ്കില്‍ നിങ്ങളുടെ മനസ്സുകളില്‍ വലുതായി തോന്നുന്ന ഏതെങ്കിലുമൊരു സൃഷ്ടിയായിക്കൊള്ളുക (എന്നാലും നിങ്ങള്‍ പുനരുജ്ജീവിപ്പിക്കപ്പെടും) അപ്പോള്‍, ആരാണ് ഞങ്ങളെ (ജീവിതത്തിലേക്ക്‌) തിരിച്ച് കൊണ്ട് വരിക? എന്ന് അവര്‍ പറഞ്ഞേക്കും. നിങ്ങളെ ആദ്യതവണ സൃഷ്ടിച്ചവന്‍ തന്നെ എന്ന് നീ പറയുക. അപ്പോള്‍ നിന്‍റെ നേരെ (നോക്കിയിട്ട്‌) അവര്‍ തലയാട്ടിക്കൊണ്ട് പറയും: എപ്പോഴായിരിക്കും അത് ? നീ പറയുക അത് അടുത്ത് തന്നെ ആയേക്കാം.
\end{malayalam}}
\flushright{\begin{Arabic}
\quranayah[17][52]
\end{Arabic}}
\flushleft{\begin{malayalam}
അതെ, അവന്‍ നിങ്ങളെ വിളിക്കുകയും, അവനെ സ്തുതിച്ച് കൊണ്ട് നിങ്ങള്‍ ഉത്തരം നല്‍കുകയും ചെയ്യുന്ന ദിവസം. (അതിന്നിടക്ക്‌) വളരെ കുറച്ച് മാത്രമേ നിങ്ങള്‍ കഴിച്ചുകൂട്ടിയിട്ടുള്ളൂ എന്ന് നിങ്ങള്‍ വിചാരിക്കുകയും ചെയ്യും.
\end{malayalam}}
\flushright{\begin{Arabic}
\quranayah[17][53]
\end{Arabic}}
\flushleft{\begin{malayalam}
നീ എന്‍റെ ദാസന്‍മാരോട് പറയുക; അവര്‍ പറയുന്നത് ഏറ്റവും നല്ല വാക്കായിരിക്കണമെന്ന്‌. തീര്‍ച്ചയായും പിശാച് അവര്‍ക്കിടയില്‍ (കുഴപ്പം) ഇളക്കിവിടുന്നു. തീര്‍ച്ചയായും പിശാച് മനുഷ്യന്ന് പ്രത്യക്ഷ ശത്രുവാകുന്നു.
\end{malayalam}}
\flushright{\begin{Arabic}
\quranayah[17][54]
\end{Arabic}}
\flushleft{\begin{malayalam}
നിങ്ങളുടെ രക്ഷിതാവ് നിങ്ങളെപ്പറ്റി നല്ലവണ്ണം അറിയുന്നവനാകുന്നു. അവന്‍ ഉദ്ദേശിക്കുന്ന പക്ഷം അവന്‍ നിങ്ങളോട് കരുണ കാണിക്കും.അല്ലെങ്കില്‍ അവന്‍ ഉദ്ദേശിക്കുന്ന പക്ഷം അവന്‍ നിങ്ങളെ ശിക്ഷിക്കും. അവരുടെ മേല്‍ മേല്‍നോട്ടക്കാരനായി നിന്നെ നാം നിയോഗിച്ചിട്ടില്ല.
\end{malayalam}}
\flushright{\begin{Arabic}
\quranayah[17][55]
\end{Arabic}}
\flushleft{\begin{malayalam}
നിന്‍റെ രക്ഷിതാവ് ആകാശങ്ങളിലും ഭൂമിയിലും ഉള്ളവരെപ്പറ്റി നല്ലവണ്ണം അറിയുന്നവനത്രെ. തീര്‍ച്ചയായും പ്രവാചകന്‍മാരില്‍ ചിലര്‍ക്ക് ചിലരേക്കാള്‍ നാം ശ്രേഷ്ഠത നല്‍കിയിട്ടുണ്ട്‌. ദാവൂദിന് നാം സബൂര്‍ എന്ന വേദം നല്‍കുകയും ചെയ്തിരിക്കുന്നു.
\end{malayalam}}
\flushright{\begin{Arabic}
\quranayah[17][56]
\end{Arabic}}
\flushleft{\begin{malayalam}
(നബിയേ,) പറയുക: അല്ലാഹുവിന് പുറമെ നിങ്ങള്‍ (ദൈവങ്ങളെന്ന്‌) വാദിച്ച് പോന്നവരെ നിങ്ങള്‍ വിളിച്ച് നോക്കൂ. നിങ്ങളില്‍ നിന്ന് ഉപദ്രവം നീക്കുവാനോ (നിങ്ങളുടെ സ്ഥിതിക്ക്‌) മാറ്റം വരുത്തുവാനോ ഉള്ള കഴിവ് അവരുടെ അധീനത്തിലില്ല.
\end{malayalam}}
\flushright{\begin{Arabic}
\quranayah[17][57]
\end{Arabic}}
\flushleft{\begin{malayalam}
അവര്‍ വിളിച്ച് പ്രാര്‍ത്ഥിച്ചുകൊണ്ടിരിക്കുന്നത് ആരെയാണോ അവര്‍ തന്നെ തങ്ങളുടെ രക്ഷിതാവിങ്കലേക്ക് സമീപനമാര്‍ഗം തേടിക്കൊണ്ടിരിക്കുകയാണ്‌. അതെ, അവരുടെ കൂട്ടത്തില്‍ അല്ലാഹുവോട് ഏറ്റവും അടുത്തവര്‍ തന്നെ (അപ്രകാരം തേടുന്നു.) അവര്‍ അവന്‍റെ കാരുണ്യം ആഗ്രഹിക്കുകയും അവന്‍റെ ശിക്ഷ ഭയപ്പെടുകയും ചെയ്യുന്നു, നിന്‍റെ രക്ഷിതാവിന്‍റെ ശിക്ഷ തീര്‍ച്ചയായും ഭയപ്പെടേണ്ടതാകുന്നു.
\end{malayalam}}
\flushright{\begin{Arabic}
\quranayah[17][58]
\end{Arabic}}
\flushleft{\begin{malayalam}
ഉയിര്‍ത്തെഴുന്നേല്‍പിന്‍റെ ദിവസത്തിന് മുമ്പായി നാം നശിപ്പിച്ച് കളയുന്നതോ അല്ലെങ്കില്‍ നാം കഠിനമായി ശിക്ഷിക്കുന്നതോ ആയിട്ടല്ലാതെ ഒരു രാജ്യവുമില്ല. അത് ഗ്രന്ഥത്തില്‍ രേഖപ്പെടുത്തപ്പെട്ട കാര്യമാകുന്നു.
\end{malayalam}}
\flushright{\begin{Arabic}
\quranayah[17][59]
\end{Arabic}}
\flushleft{\begin{malayalam}
നാം ദൃഷ്ടാന്തങ്ങള്‍ അയക്കുന്നതിന് നമുക്ക് തടസ്സമായത് പൂര്‍വ്വികന്‍മാര്‍ അത്തരം ദൃഷ്ടാന്തങ്ങളെ നിഷേധിച്ച് തള്ളിക്കളഞ്ഞു എന്നത് മാത്രമാണ്‌. നാം ഥമൂദ് സമുദായത്തിന് പ്രത്യക്ഷ ദൃഷ്ടാന്തമായിക്കൊണ്ട് ഒട്ടകത്തെ നല്‍കുകയുണ്ടായി. എന്നിട്ട് അവര്‍ അതിന്‍റെ കാര്യത്തില്‍ അക്രമം പ്രവര്‍ത്തിച്ചു. ഭയപ്പെടുത്താന്‍ മാത്രമാകുന്നു നാം ദൃഷ്ടാന്തങ്ങള്‍ അയക്കുന്നത്‌.
\end{malayalam}}
\flushright{\begin{Arabic}
\quranayah[17][60]
\end{Arabic}}
\flushleft{\begin{malayalam}
തീര്‍ച്ചയായും നിന്‍റെ രക്ഷിതാവ് മനുഷ്യരെ വലയം ചെയ്തിരിക്കുന്നു. എന്ന് നാം നിന്നോട് പറഞ്ഞ സന്ദര്‍ഭവും ശ്രദ്ധേയമാണ്‌. നിനക്ക് നാം കാണിച്ചുതന്ന ആ ദര്‍ശനത്തെ നാം ജനങ്ങള്‍ക്ക് ഒരു പരീക്ഷണം മാത്രമാക്കിയിരിക്കുകയാണ്‌. ഖുര്‍ആനിലെ ശപിക്കപ്പെട്ട വൃക്ഷത്തേയും (ഒരു പരീക്ഷണമാക്കിയിരിക്കുന്നു.) നാം അവരെ ഭയപ്പെടുത്തുന്നു. എന്നാല്‍ വലിയ ധിക്കാരം മാത്രമാണ് അത് അവര്‍ക്ക് വര്‍ദ്ധിപ്പിച്ച് കൊണ്ടിരിക്കുന്നത്‌.
\end{malayalam}}
\flushright{\begin{Arabic}
\quranayah[17][61]
\end{Arabic}}
\flushleft{\begin{malayalam}
നിങ്ങള്‍ ആദമിന് പ്രണാമം ചെയ്യുക എന്ന് നാം മലക്കുകളോട് പറഞ്ഞ സന്ദര്‍ഭം (ശ്രദ്ധേയമാകുന്നു.) അപ്പോള്‍ അവര്‍ പ്രണമിച്ചു. ഇബ്ലീസൊഴികെ.അവന്‍ പറഞ്ഞു: നീ കളിമണ്ണിനാല്‍ സൃഷ്ടിച്ചവന്ന് ഞാന്‍ പ്രണാമം ചെയ്യുകയോ?
\end{malayalam}}
\flushright{\begin{Arabic}
\quranayah[17][62]
\end{Arabic}}
\flushleft{\begin{malayalam}
അവന്‍ പറഞ്ഞു: എന്നെക്കാള്‍ നീ ആദരിച്ചിട്ടുള്ള ഇവനാരെന്ന് നീ എനിക്ക് പറഞ്ഞുതരൂ. തീര്‍ച്ചയായും ഉയിര്‍ത്തെഴുന്നേല്‍പിന്‍റെ നാളുവരെ നീ എനിക്ക് അവധി നീട്ടിത്തരുന്ന പക്ഷം, ഇവന്‍റെ സന്തതികളില്‍ ചുരുക്കം പേരൊഴിച്ച് എല്ലാവരെയും ഞാന്‍ കീഴ്പെടുത്തുക തന്നെ ചെയ്യും.
\end{malayalam}}
\flushright{\begin{Arabic}
\quranayah[17][63]
\end{Arabic}}
\flushleft{\begin{malayalam}
അവന്‍ (അല്ലാഹു) പറഞ്ഞു: നീ പോയിക്കൊള്ളൂ. അവരില്‍ നിന്ന് വല്ലവരും നിന്നെ പിന്തുടരുന്ന പക്ഷം നിങ്ങള്‍ക്കെല്ലാമുള്ള പ്രതിഫലം നരകം തന്നെയായിരിക്കും. അതെ; തികഞ്ഞ പ്രതിഫലം തന്നെ.
\end{malayalam}}
\flushright{\begin{Arabic}
\quranayah[17][64]
\end{Arabic}}
\flushleft{\begin{malayalam}
അവരില്‍ നിന്ന് നിനക്ക് സാധ്യമായവരെ നിന്‍റെ ശബ്ദം മുഖേന നീ ഇളക്കിവിട്ട് കൊള്ളുക. അവര്‍ക്കെതിരില്‍ നിന്‍റെ കുതിരപ്പടയെയും കാലാള്‍പ്പടയെയും നീ വിളിച്ചുകൂട്ടുകയും ചെയ്ത് കൊള്ളുക. സ്വത്തുക്കളിലും സന്താനങ്ങളിലും നീ അവരോടൊപ്പം പങ്ക് ചേരുകയും അവര്‍ക്കു നീ വാഗ്ദാനങ്ങള്‍ നല്‍കുകയും ചെയ്തുകൊള്ളുക. പിശാച് അവരോട് ചെയ്യുന്ന വാഗ്ദാനം വഞ്ചന മാത്രമാകുന്നു.
\end{malayalam}}
\flushright{\begin{Arabic}
\quranayah[17][65]
\end{Arabic}}
\flushleft{\begin{malayalam}
തീര്‍ച്ചയായും എന്‍റെ ദാസന്‍മാരാരോ അവരുടെ മേല്‍ നിനക്ക് യാതൊരു അധികാരവുമില്ല. കൈകാര്യകര്‍ത്താവായി നിന്‍റെ രക്ഷിതാവ് തന്നെ മതി.
\end{malayalam}}
\flushright{\begin{Arabic}
\quranayah[17][66]
\end{Arabic}}
\flushleft{\begin{malayalam}
നിങ്ങളുടെ രക്ഷിതാവ് നിങ്ങള്‍ക്ക് വേണ്ടി കടലിലൂടെ കപ്പല്‍ ഓടിക്കുന്നവനാകുന്നു.അവന്‍റെ ഔദാര്യത്തില്‍ നിന്ന് നിങ്ങള്‍ തേടിക്കൊണ്ട് വരുന്നതിന് വേണ്ടിയത്രെ അത്‌. തീര്‍ച്ചയായും അവന്‍ നിങ്ങളോട് ഏറെ കരുണയുള്ളവനാകുന്നു.
\end{malayalam}}
\flushright{\begin{Arabic}
\quranayah[17][67]
\end{Arabic}}
\flushleft{\begin{malayalam}
കടലില്‍ വെച്ച് നിങ്ങള്‍ക്ക് കഷ്ടത (അപായം) നേരിട്ടാല്‍ അവന്‍ ഒഴികെ, നിങ്ങള്‍ ആരെയെല്ലാം വിളിച്ച് പ്രാര്‍ത്ഥിച്ചിരുന്നുവോ അവര്‍ അപ്രത്യക്ഷരാകും. എന്നാല്‍ നിങ്ങളെ അവന്‍ രക്ഷപ്പെടുത്തി കരയിലെത്തിക്കുമ്പോള്‍ നിങ്ങള്‍ തിരിഞ്ഞുകളയുകയായി. മനുഷ്യന്‍ ഏറെ നന്ദികെട്ടവനായിരിക്കുന്നു.
\end{malayalam}}
\flushright{\begin{Arabic}
\quranayah[17][68]
\end{Arabic}}
\flushleft{\begin{malayalam}
കരയുടെ ഭാഗത്ത് തന്നെ അവന്‍ നിങ്ങളെ ആഴ്ത്തിക്കളയുകയോ, അല്ലെങ്കില്‍ അവന്‍ നിങ്ങളുടെ നേരെ ഒരു ചരല്‍ മഴ അയക്കുകയോ ചെയ്യുകയും, നിങ്ങളുടെ സംരക്ഷണം ഏല്‍ക്കാന്‍ യാതൊരാളെയും നിങ്ങള്‍ കണ്ടെത്താതിരിക്കുകയും ചെയ്യുന്നതിനെപ്പറ്റി നിങ്ങള്‍ നിര്‍ഭയരായിരിക്കുകയാണോ?
\end{malayalam}}
\flushright{\begin{Arabic}
\quranayah[17][69]
\end{Arabic}}
\flushleft{\begin{malayalam}
അതല്ലെങ്കില്‍ മറ്റൊരു പ്രാവശ്യം അവന്‍ നിങ്ങളെ അവിടേക്ക് (കടലിലേക്ക്‌) തിരിച്ച് കൊണ്ട് പോകുകയും, എന്നിട്ട് നിങ്ങളുടെ നേര്‍ക്ക് അവന്‍ ഒരു തകര്‍പ്പന്‍ കാറ്റയച്ചിട്ട് നിങ്ങള്‍ നന്ദികേട് കാണിച്ചതിന് നിങ്ങളെ അവന്‍ മുക്കിക്കളയുകയും, അനന്തരം ആ കാര്യത്തില്‍ നിങ്ങള്‍ക്ക് വേണ്ടി നമുക്കെതിരില്‍ നടപടി എടുക്കാന്‍ യാതൊരാളെയും നിങ്ങള്‍ കണ്ടെത്താതിരിക്കുകയും ചെയ്യുന്നതിനെപറ്റി നിങ്ങള്‍ നിര്‍ഭയരായിരിക്കുകയാണോ?
\end{malayalam}}
\flushright{\begin{Arabic}
\quranayah[17][70]
\end{Arabic}}
\flushleft{\begin{malayalam}
തീര്‍ച്ചയായും ആദം സന്തതികളെ നാം ആദരിക്കുകയും, കടലിലും കരയിലും അവരെ നാം വാഹനത്തില്‍ കയറ്റുകയും, വിശിഷ്ടമായ വസ്തുക്കളില്‍ നിന്ന് നാം അവര്‍ക്ക് ഉപജീവനം നല്‍കുകയും, നാം സൃഷ്ടിച്ചിട്ടുള്ളവരില്‍ മിക്കവരെക്കാളും അവര്‍ക്ക് നാം സവിശേഷമായ ശ്രേഷ്ഠത നല്‍കുകയും ചെയ്തിരിക്കുന്നു.
\end{malayalam}}
\flushright{\begin{Arabic}
\quranayah[17][71]
\end{Arabic}}
\flushleft{\begin{malayalam}
എല്ലാ മനുഷ്യരെയും അവരുടെ നേതാവിനോടൊപ്പം നാം വിളിച്ചുകൂട്ടുന്ന ദിവസം (ശ്രദ്ധേയമാകുന്നു.) അപ്പോള്‍ ആര്‍ക്ക് തന്‍റെ (കര്‍മ്മങ്ങളുടെ) രേഖ തന്‍റെ വലതുകൈയ്യില്‍ നല്‍കപ്പെട്ടുവോ അത്തരക്കാര്‍ അവരുടെ ഗ്രന്ഥം വായിച്ചുനോക്കുന്നതാണ്‌. അവരോട് ഒരു തരിമ്പും അനീതി ചെയ്യപ്പെടുന്നതുമല്ല.
\end{malayalam}}
\flushright{\begin{Arabic}
\quranayah[17][72]
\end{Arabic}}
\flushleft{\begin{malayalam}
വല്ലവനും ഈ ലോകത്ത് അന്ധനായിരുന്നാല്‍ പരലോകത്തും അവന്‍ അന്ധനായിരിക്കും. ഏറ്റവും വഴിപിഴച്ചവനുമായിരിക്കും.
\end{malayalam}}
\flushright{\begin{Arabic}
\quranayah[17][73]
\end{Arabic}}
\flushleft{\begin{malayalam}
തീര്‍ച്ചയായും നാം നിനക്ക് ബോധനം നല്‍കിയിട്ടുള്ളതില്‍ നിന്ന് അവര്‍ നിന്നെ തെറ്റിച്ചുകളയാന്‍ ഒരുങ്ങിയിരിക്കുന്നു. നീ നമ്മുടെ മേല്‍ അതല്ലാത്ത വല്ലതും കെട്ടിച്ചമയ്ക്കുവാനാണ് (അവര്‍ ആഗ്രഹിക്കുന്നത്‌). അപ്പോള്‍ അവര്‍ നിന്നെ മിത്രമായി സ്വീകരിക്കുക തന്നെ ചെയ്യും.
\end{malayalam}}
\flushright{\begin{Arabic}
\quranayah[17][74]
\end{Arabic}}
\flushleft{\begin{malayalam}
നിന്നെ നാം ഉറപ്പിച്ചു നിര്‍ത്തിയിട്ടില്ലായിരുന്നുവെങ്കില്‍ തീര്‍ച്ചയായും നീ അവരിലേക്ക് അല്‍പമൊക്കെ ചാഞ്ഞുപോയേക്കുമായിരുന്നു.
\end{malayalam}}
\flushright{\begin{Arabic}
\quranayah[17][75]
\end{Arabic}}
\flushleft{\begin{malayalam}
എങ്കില്‍ ജീവിതത്തിലും ഇരട്ടി ശിക്ഷ, മരണത്തിലും ഇരട്ടി ശിക്ഷ അതായിരിക്കും നാം നിനക്ക് ആസ്വദിപ്പിക്കുന്നത്‌. പിന്നീട് നമുക്കെതിരില്‍ നിനക്ക് സഹായം നല്‍കാന്‍ യാതൊരാളെയും നീ കണ്ടെത്തുകയില്ല.
\end{malayalam}}
\flushright{\begin{Arabic}
\quranayah[17][76]
\end{Arabic}}
\flushleft{\begin{malayalam}
തീര്‍ച്ചയായും അവര്‍ നിന്നെ നാട്ടില്‍ നിന്ന് വിരട്ടി വിടുവാന്‍ ഒരുങ്ങിയിരിക്കുന്നു. നിന്നെ അവിടെ നിന്ന് പുറത്താക്കുകയത്രെ അവരുടെ ലക്ഷ്യം. എങ്കില്‍ നിന്‍റെ (പുറത്താക്കലിന്‌) ശേഷം കുറച്ച് കാലമല്ലാതെ അവര്‍ (അവിടെ) താമസിക്കുകയില്ല.
\end{malayalam}}
\flushright{\begin{Arabic}
\quranayah[17][77]
\end{Arabic}}
\flushleft{\begin{malayalam}
നിനക്ക് മുമ്പ് നാം അയച്ച നമ്മുടെ ദൂതന്‍മാരുടെ കാര്യത്തിലുണ്ടായ നടപടിക്രമം തന്നെ. നമ്മുടെ നടപടിക്രമത്തിന് യാതൊരു ഭേദഗതിയും നീ കണ്ടെത്തുകയില്ല.
\end{malayalam}}
\flushright{\begin{Arabic}
\quranayah[17][78]
\end{Arabic}}
\flushleft{\begin{malayalam}
സൂര്യന്‍ (ആകാശമദ്ധ്യത്തില്‍ നിന്ന്‌) തെറ്റിയത് മുതല്‍ രാത്രി ഇരുട്ടുന്നത് വരെ (നിശ്ചിത സമയങ്ങളില്‍) നീ നമസ്കാരം മുറപ്രകാരം നിര്‍വഹിക്കുക ഖുര്‍ആന്‍ പാരായണം ചെയ്തുകൊണ്ടുള്ള പ്രഭാത നമസ്കാരവും (നിലനിര്‍ത്തുക) തീര്‍ച്ചയായും പ്രഭാതനമസ്കാരത്തിലെ ഖുര്‍ആന്‍ പാരായണം സാക്ഷ്യം വഹിക്കപ്പെടുന്നതാകുന്നു.
\end{malayalam}}
\flushright{\begin{Arabic}
\quranayah[17][79]
\end{Arabic}}
\flushleft{\begin{malayalam}
രാത്രിയില്‍ നിന്ന് അല്‍പസമയം നീ ഉറക്കമുണര്‍ന്ന് അതോടെ (ഖുര്‍ആന്‍ പാരായണത്തോടെ) നമസ്കരിക്കുകയും ചെയ്യുക. അത് നിനക്ക് കൂടുതലായുള്ള ഒരു പുണ്യകര്‍മ്മമാകുന്നു. നിന്‍റെ രക്ഷിതാവ് നിന്നെ സ്തുത്യര്‍ഹമായ ഒരു സ്ഥാനത്ത് നിയോഗിച്ചേക്കാം.
\end{malayalam}}
\flushright{\begin{Arabic}
\quranayah[17][80]
\end{Arabic}}
\flushleft{\begin{malayalam}
എന്‍റെ രക്ഷിതാവേ, സത്യത്തിന്‍റെ പ്രവേശനമാര്‍ഗത്തിലൂടെ നീ എന്നെ പ്രവേശിപ്പിക്കുകയും, സത്യത്തിന്‍റെ ബഹിര്‍ഗ്ഗമനമാര്‍ഗത്തിലൂടെ നീ എന്നെ പുറപ്പെടുവിക്കുകയും ചെയ്യേണമേ. നിന്‍റെ പക്കല്‍ നിന്ന് എനിക്ക് സഹായകമായ ഒരു ആധികാരിക ശക്തി നീ ഏര്‍പെടുത്തിത്തരികയും ചെയ്യേണമേ എന്ന് നീ പറയുകയും ചെയ്യുക.
\end{malayalam}}
\flushright{\begin{Arabic}
\quranayah[17][81]
\end{Arabic}}
\flushleft{\begin{malayalam}
സത്യം വന്നിരിക്കുന്നു. അസത്യം മാഞ്ഞുപോയിരിക്കുന്നു. തീര്‍ച്ചയായും അസത്യം മാഞ്ഞുപോകുന്നതാകുന്നു. എന്നും നീ പറയുക.
\end{malayalam}}
\flushright{\begin{Arabic}
\quranayah[17][82]
\end{Arabic}}
\flushleft{\begin{malayalam}
സത്യവിശ്വാസികള്‍ക്ക് ശമനവും കാരുണ്യവുമായിട്ടുള്ളത് ഖുര്‍ആനിലൂടെ നാം അവതരിപ്പിച്ചുകൊണ്ടിരിക്കുന്നു. അക്രമികള്‍ക്ക് അത് നഷ്ടമല്ലാതെ (മറ്റൊന്നും) വര്‍ദ്ധിപ്പിക്കുന്നില്ല.
\end{malayalam}}
\flushright{\begin{Arabic}
\quranayah[17][83]
\end{Arabic}}
\flushleft{\begin{malayalam}
നാം മനുഷ്യന്ന് അനുഗ്രഹം ചെയ്ത് കൊടുത്താല്‍ അവന്‍ തിരിഞ്ഞുകളയുകയും, അവന്‍റെ പാട്ടിന് മാറിപ്പോകുകയും ചെയ്യുന്നു. അവന്ന് ദോഷം ബാധിച്ചാലാകട്ടെ അവന്‍ വളരെ നിരാശനായിരിക്കുകയും ചെയ്യും.
\end{malayalam}}
\flushright{\begin{Arabic}
\quranayah[17][84]
\end{Arabic}}
\flushleft{\begin{malayalam}
പറയുക: എല്ലാവരും അവരവരുടെ സമ്പ്രദായമനുസരിച്ച് പ്രവര്‍ത്തിച്ചുകൊണ്ടിരിക്കുന്നു. എന്നാല്‍ കൂടുതല്‍ ശരിയായ മാര്‍ഗം സ്വീകരിച്ചവന്‍ ആരാണെന്നതിനെപ്പറ്റി നിങ്ങളുടെ രക്ഷിതാവ് നല്ലവണ്ണം അറിയുന്നവനാകുന്നു.
\end{malayalam}}
\flushright{\begin{Arabic}
\quranayah[17][85]
\end{Arabic}}
\flushleft{\begin{malayalam}
നിന്നോടവര്‍ ആത്മാവിനെപ്പറ്റി ചോദിക്കുന്നു. പറയുക: ആത്മാവ് എന്‍റെ രക്ഷിതാവിന്‍റെ കാര്യത്തില്‍ പെട്ടതാകുന്നു. അറിവില്‍ നിന്ന് അല്‍പമല്ലാതെ നിങ്ങള്‍ക്ക് നല്‍കപ്പെട്ടിട്ടില്ല.
\end{malayalam}}
\flushright{\begin{Arabic}
\quranayah[17][86]
\end{Arabic}}
\flushleft{\begin{malayalam}
തീര്‍ച്ചയായും നാം ഉദ്ദേശിച്ചിരുന്നുവെങ്കില്‍ നിനക്ക് നാം നല്‍കിയ സന്ദേശം നാം പിന്‍വലിക്കുമായിരുന്നു. പിന്നീട് അതിന്‍റെ കാര്യത്തില്‍ നമുക്കെതിരായി നിനക്ക് ഭരമേല്‍പിക്കാവുന്ന യാതൊരാളെയും നീ കണ്ടെത്തുകയുമില്ല.
\end{malayalam}}
\flushright{\begin{Arabic}
\quranayah[17][87]
\end{Arabic}}
\flushleft{\begin{malayalam}
നിന്‍റെ രക്ഷിതാവിങ്കല്‍ നിന്നുള്ള കാരുണ്യം മാത്രമാകുന്നു അത്‌. നിന്‍റെ മേല്‍ അവന്‍റെ അനുഗ്രഹം തീര്‍ച്ചയായും മഹത്തരമായിരിക്കുന്നു.
\end{malayalam}}
\flushright{\begin{Arabic}
\quranayah[17][88]
\end{Arabic}}
\flushleft{\begin{malayalam}
(നബിയേ,) പറയുക: ഈ ഖുര്‍ആന്‍ പോലൊന്ന് കൊണ്ട് വരുന്നതിന്നായി മനുഷ്യരും ജിന്നുകളും ഒന്നിച്ചുചേര്‍ന്നാലും തീര്‍ച്ചയായും അതുപോലൊന്ന് അവര്‍ കൊണ്ട് വരികയില്ല. അവരില്‍ ചിലര്‍ ചിലര്‍ക്ക് പിന്തുണ നല്‍കുന്നതായാല്‍ പോലും.
\end{malayalam}}
\flushright{\begin{Arabic}
\quranayah[17][89]
\end{Arabic}}
\flushleft{\begin{malayalam}
തീര്‍ച്ചയായും ഈ ഖുര്‍ആനില്‍ എല്ലാവിധ ഉപമകളും ജനങ്ങള്‍ക്ക് വേണ്ടി വിവിധ രൂപത്തില്‍ നാം വിവരിച്ചിട്ടുണ്ട്‌. എന്നാല്‍ മനുഷ്യരില്‍ അധികപേര്‍ക്കും നിഷേധിക്കാനല്ലാതെ മനസ്സുവന്നില്ല.
\end{malayalam}}
\flushright{\begin{Arabic}
\quranayah[17][90]
\end{Arabic}}
\flushleft{\begin{malayalam}
അവര്‍ പറഞ്ഞു: ഈ ഭൂമിയില്‍ നിന്ന് നീ ഞങ്ങള്‍ക്ക് ഒരു ഉറവ് ഒഴുക്കിത്തരുന്നത് വരെ ഞങ്ങള്‍ നിന്നെ വിശ്വസിക്കുകയേ ഇല്ല.
\end{malayalam}}
\flushright{\begin{Arabic}
\quranayah[17][91]
\end{Arabic}}
\flushleft{\begin{malayalam}
അല്ലെങ്കില്‍ നിനക്ക് ഈന്തപ്പനയുടെയും മുന്തിരിയുടെയും ഒരു തോട്ടമുണ്ടായിരിക്കുകയും, അതിന്നിടയിലൂടെ നീ സമൃദ്ധമായി അരുവികള്‍ ഒഴുക്കുകയും ചെയ്യുന്നത് വരെ.
\end{malayalam}}
\flushright{\begin{Arabic}
\quranayah[17][92]
\end{Arabic}}
\flushleft{\begin{malayalam}
അല്ലെങ്കില്‍ നീ ജല്‍പിച്ചത് പോലെ ആകാശത്തെ ഞങ്ങളുടെ മേല്‍ കഷ്ണം കഷ്ണമായി നീ വീഴ്ത്തുന്നത് വരെ. അല്ലെങ്കില്‍ അല്ലാഹുവെയും മലക്കുകളെയും കൂട്ടം കൂട്ടമായി നീ കൊണ്ട് വരുന്നത് വരെ.
\end{malayalam}}
\flushright{\begin{Arabic}
\quranayah[17][93]
\end{Arabic}}
\flushleft{\begin{malayalam}
അല്ലെങ്കില്‍ നിനക്ക് സ്വര്‍ണം കൊണ്ടുള്ള ഒരു വീടുണ്ടാകുന്നത് വരെ, അല്ലെങ്കില്‍ ആകാശത്ത് കൂടി നീ കയറിപ്പോകുന്നത് വരെ. ഞങ്ങള്‍ക്ക് വായിക്കാവുന്ന ഒരു ഗ്രന്ഥം ഞങ്ങളുടെ അടുത്തേക്ക് നീ ഇറക്കികൊണ്ട് വരുന്നത് വരെ നീ കയറിപ്പോയതായി ഞങ്ങള്‍ വിശ്വസിക്കുകയേ ഇല്ല. (നബിയേ,) പറയുക: എന്‍റെ രക്ഷിതാവ് എത്ര പരിശുദ്ധന്‍! ഞാനൊരു മനുഷ്യന്‍ മാത്രമായ ദൂതനല്ലേ ?
\end{malayalam}}
\flushright{\begin{Arabic}
\quranayah[17][94]
\end{Arabic}}
\flushleft{\begin{malayalam}
ജനങ്ങള്‍ക്ക് സന്‍മാര്‍ഗം വന്നപ്പോള്‍ അവര്‍ അത് വിശ്വസിക്കുന്നതിന് തടസ്സമായത്‌, അല്ലാഹു ഒരു മനുഷ്യനെ ദൂതനായി നിയോഗിച്ചിരിക്കുകയാണോ എന്ന അവരുടെ വാക്ക് മാത്രമായിരുന്നു.
\end{malayalam}}
\flushright{\begin{Arabic}
\quranayah[17][95]
\end{Arabic}}
\flushleft{\begin{malayalam}
(നബിയേ,) പറയുക: ഭൂമിയിലുള്ളത് ശാന്തരായി നടന്ന് പോകുന്ന മലക്കുകളായിരുന്നെങ്കില്‍ അവരിലേക്ക് ആകാശത്ത് നിന്ന് ഒരു മലക്കിനെ നാം ദൂതനായി ഇറക്കുമായിരുന്നു.
\end{malayalam}}
\flushright{\begin{Arabic}
\quranayah[17][96]
\end{Arabic}}
\flushleft{\begin{malayalam}
നീ പറയുക: എനിക്കും നിങ്ങള്‍ക്കുമിടയില്‍ സാക്ഷിയായി അല്ലാഹു മതി. തീര്‍ച്ചയായും അല്ലാഹു തന്‍റെ ദാസന്‍മാരെപ്പറ്റി സൂക്ഷ്മമായി അറിയുന്നവനും കാണുന്നവനുമാകുന്നു.
\end{malayalam}}
\flushright{\begin{Arabic}
\quranayah[17][97]
\end{Arabic}}
\flushleft{\begin{malayalam}
അല്ലാഹു ആരെ നേര്‍വഴിയിലാക്കുന്നുവോ അവനാണ് നേര്‍മാര്‍ഗം പ്രാപിച്ചവന്‍.അവന്‍ ആരെ ദുര്‍മാര്‍ഗത്തിലാക്കുന്നുവോ, അവര്‍ക്ക് അവന്നു പുറമെ രക്ഷാധികാരികളെയൊന്നും നീ കണ്ടെത്തുന്നതേയല്ല. ഉയിര്‍ത്തെഴുന്നേല്‍പിന്‍റെ നാളില്‍ മുഖം നിലത്ത് കുത്തിയവരായിക്കൊണ്ടും അന്ധരും ഊമകളും ബധിരരുമായിക്കൊണ്ടും നാം അവരെ ഒരുമിച്ചുകൂട്ടുന്നതാണ്‌. അവരുടെ സങ്കേതം നരകമത്രെ. അത് അണഞ്ഞ് പോകുമ്പോഴെല്ലാം നാം അവര്‍ക്ക് ജ്വാല കൂട്ടികൊടുക്കുന്നതാണ്‌.
\end{malayalam}}
\flushright{\begin{Arabic}
\quranayah[17][98]
\end{Arabic}}
\flushleft{\begin{malayalam}
അവര്‍ നമ്മുടെ ദൃഷ്ടാന്തങ്ങളെ നിഷേധിച്ചതിനും, ഞങ്ങള്‍ എല്ലുകളും ജീര്‍ണാവശിഷ്ടങ്ങളും ആയിക്കഴിഞ്ഞിട്ടാണോ പുതിയൊരു സൃഷ്ടിയായി ഞങ്ങള്‍ ഉയിര്‍ത്തെഴുന്നല്‍പിക്കപ്പെടുന്നത് എന്ന് അവര്‍ പറഞ്ഞതിനും അവര്‍ക്കുള്ള പ്രതിഫലമത്രെ അത്‌.
\end{malayalam}}
\flushright{\begin{Arabic}
\quranayah[17][99]
\end{Arabic}}
\flushleft{\begin{malayalam}
ആകാശങ്ങളും ഭൂമിയും സൃഷ്ടിച്ച അല്ലാഹു ഇവരെപ്പോലെയുള്ളവരെയും സൃഷ്ടിക്കാന്‍ ശക്തനാണ് എന്ന് ഇവര്‍ മനസ്സിലാക്കിയിട്ടില്ലേ? ഇവര്‍ക്ക് അവന്‍ ഒരു അവധി നിശ്ചയിച്ചിട്ടുണ്ട്‌. അതില്‍ സംശയമേ ഇല്ല. എന്നാല്‍ നന്ദികേട് കാണിക്കാനല്ലാതെ ഈ അക്രമികള്‍ക്ക് മനസ്സ് വന്നില്ല.
\end{malayalam}}
\flushright{\begin{Arabic}
\quranayah[17][100]
\end{Arabic}}
\flushleft{\begin{malayalam}
(നബിയേ,) പറയുക: എന്‍റെ രക്ഷിതാവിന്‍റെ കാരുണ്യത്തിന്‍റെ ഖജനാവുകള്‍ നിങ്ങളുടെ ഉടമസ്ഥതയിലായിരുന്നെങ്കില്‍ ചെലവഴിച്ച് തീര്‍ന്ന് പോകുമെന്ന് ഭയന്ന് നിങ്ങള്‍ പിശുക്കിപ്പിടിക്കുക തന്നെ ചെയ്യുമായിരുന്നു. മനുഷ്യന്‍ കടുത്ത ലുബ്ധനാകുന്നു.
\end{malayalam}}
\flushright{\begin{Arabic}
\quranayah[17][101]
\end{Arabic}}
\flushleft{\begin{malayalam}
തീര്‍ച്ചയായും മൂസായ്ക്ക് നാം പ്രത്യക്ഷമായ ഒമ്പതു ദൃഷ്ടാന്തങ്ങള്‍ നല്‍കുകയുണ്ടായി. അദ്ദേഹം അവരുടെ അടുത്ത് ചെല്ലുകയും, മൂസാ! തീര്‍ച്ചയായും നിന്നെ ഞാന്‍ മാരണം ബാധിച്ച ഒരാളായിട്ടാണ് കരുതുന്നത് എന്ന് ഫിര്‍ഔന്‍ അദ്ദേഹത്തോട് പറയുകയും ചെയ്ത സന്ദര്‍ഭത്തെപ്പറ്റി ഇസ്രായീല്‍ സന്തതികളോട് നീ ചോദിച്ച് നോക്കുക.
\end{malayalam}}
\flushright{\begin{Arabic}
\quranayah[17][102]
\end{Arabic}}
\flushleft{\begin{malayalam}
അദ്ദേഹം (ഫിര്‍ഔനോട്‌) പറഞ്ഞു: കണ്ണുതുറപ്പിക്കുന്ന ദൃഷ്ടാന്തങ്ങളായിക്കൊണ്ട് ഇവ ഇറക്കിയത് ആകാശങ്ങളുടെയും ഭൂമിയുടെയും രക്ഷിതാവ് തന്നെയാണ് എന്ന് തീര്‍ച്ചയായും നീ മനസ്സിലാക്കിയിട്ടുണ്ട്‌. ഫിര്‍ഔനേ, തീര്‍ച്ചയായും നീ നാശമടഞ്ഞവന്‍ തന്നെ എന്നാണ് ഞാന്‍ കരുതുന്നത്‌.
\end{malayalam}}
\flushright{\begin{Arabic}
\quranayah[17][103]
\end{Arabic}}
\flushleft{\begin{malayalam}
അപ്പോള്‍ അവരെ (ഇസ്രായീല്യരെ) നാട്ടില്‍ നിന്ന് വിരട്ടിയോടിക്കുവാനാണ് അവന്‍ ഉദ്ദേശിച്ചത്‌. അതിനാല്‍ അവനെയും അവന്‍റെ കൂടെയുള്ളവരെയും മുഴുവന്‍ നാം മുക്കിനശിപ്പിച്ചു.
\end{malayalam}}
\flushright{\begin{Arabic}
\quranayah[17][104]
\end{Arabic}}
\flushleft{\begin{malayalam}
അവന്‍റെ (നാശത്തിനു) ശേഷം നാം ഇസ്രായീല്‍ സന്തതികളോട് ഇപ്രകാരം പറയുകയും ചെയ്തു: നിങ്ങള്‍ ഈ നാട്ടില്‍ താമസിച്ച് കൊള്ളുക. അനന്തരം പരലോകത്തിന്‍റെ വാഗ്ദാനം വന്നെത്തിയാല്‍ നിങ്ങളെയെല്ലാം കൂട്ടത്തോടെ നാം കൊണ്ടു വരുന്നതാണ്‌.
\end{malayalam}}
\flushright{\begin{Arabic}
\quranayah[17][105]
\end{Arabic}}
\flushleft{\begin{malayalam}
സത്യത്തോടുകൂടിയാണ് നാം അത് (ഖുര്‍ആന്‍) അവതരിപ്പിച്ചത്‌. സത്യത്തോട് കൂടിത്തന്നെ അത് അവതരിക്കുകയും ചെയ്തിരിക്കുന്നു. സന്തോഷവാര്‍ത്ത അറിയിക്കുന്നവനും താക്കീത് നല്‍കുന്നവനുമായിക്കൊണ്ടല്ലാതെ നിന്നെ നാം അയച്ചിട്ടില്ല.
\end{malayalam}}
\flushright{\begin{Arabic}
\quranayah[17][106]
\end{Arabic}}
\flushleft{\begin{malayalam}
നീ ജനങ്ങള്‍ക്ക് സാവകാശത്തില്‍ ഓതികൊടുക്കേണ്ടതിനായി ഖുര്‍ആനിനെ നാം (പല ഭാഗങ്ങളായി) വേര്‍തിരിച്ചിരിക്കുന്നു. നാം അതിനെ ക്രമേണയായി ഇറക്കുകയും ചെയ്തിരിക്കുന്നു.
\end{malayalam}}
\flushright{\begin{Arabic}
\quranayah[17][107]
\end{Arabic}}
\flushleft{\begin{malayalam}
(നബിയേ,) പറയുക: നിങ്ങള്‍ ഇതില്‍ (ഖുര്‍ആനില്‍) വിശ്വസിച്ച് കൊള്ളുക. അല്ലെങ്കില്‍ വിശ്വസിക്കാതിരിക്കുക. തീര്‍ച്ചയായും ഇതിന് മുമ്പ് (ദിവ്യ) ജ്ഞാനം നല്‍കപ്പെട്ടവരാരോ അവര്‍ക്ക് ഇത് വായിച്ചുകേള്‍പിക്കപ്പെട്ടാല്‍ അവര്‍ പ്രണമിച്ച് കൊണ്ട് മുഖം കുത്തി വീഴുന്നതാണ്‌.
\end{malayalam}}
\flushright{\begin{Arabic}
\quranayah[17][108]
\end{Arabic}}
\flushleft{\begin{malayalam}
അവര്‍ പറയും: ഞങ്ങളുടെ രക്ഷിതാവ് എത്ര പരിശുദ്ധന്‍! തീര്‍ച്ചയായും ഞങ്ങളുടെ രക്ഷിതാവിന്‍റെ വാഗ്ദാനം നടപ്പിലാക്കപ്പെടുന്നതു തന്നെയാകുന്നു.
\end{malayalam}}
\flushright{\begin{Arabic}
\quranayah[17][109]
\end{Arabic}}
\flushleft{\begin{malayalam}
അവര്‍ കരഞ്ഞുകൊണ്ട് മുഖം കുത്തി വീഴുകയും അതവര്‍ക്ക് വിനയം വര്‍ദ്ധിപ്പിക്കുകയും ചെയ്യും.
\end{malayalam}}
\flushright{\begin{Arabic}
\quranayah[17][110]
\end{Arabic}}
\flushleft{\begin{malayalam}
(നബിയേ,) പറയുക: നിങ്ങള്‍ അല്ലാഹു എന്ന് വിളിച്ചുകൊള്ളുക. അല്ലെങ്കില്‍ റഹ്മാന്‍ എന്ന് വിളിച്ചുകൊള്ളുക. ഏതു തന്നെ നിങ്ങള്‍ വിളിക്കുകയാണെങ്കിലും അവന്നുള്ളതാകുന്നു ഏറ്റവും ഉല്‍കൃഷ്ടമായ നാമങ്ങള്‍. നിന്‍റെ പ്രാര്‍ത്ഥന നീ ഉച്ചത്തിലാക്കരുത്‌. അത് പതുക്കെയുമാക്കരുത്‌. അതിന്നിടയിലുള്ള ഒരു മാര്‍ഗം നീ തേടിക്കൊള്ളുക.
\end{malayalam}}
\flushright{\begin{Arabic}
\quranayah[17][111]
\end{Arabic}}
\flushleft{\begin{malayalam}
സന്താനത്തെ സ്വീകരിച്ചിട്ടില്ലാത്തവനും, ആധിപത്യത്തില്‍ പങ്കാളിയില്ലാത്തവനും നിന്ദ്യതയില്‍ നിന്ന് രക്ഷിക്കാന്‍ ഒരു രക്ഷകന്‍ ആവശ്യമില്ലാത്തവനുമായ അല്ലാഹുവിന് സ്തുതി! എന്ന് നീ പറയുകയും അവനെ ശരിയാംവണ്ണം മഹത്വപ്പെടുത്തുകയും ചെയ്യുക.
\end{malayalam}}
\chapter{\textmalayalam{അല്‍ കഹഫ് ( ഗുഹ‍ )}}
\begin{Arabic}
\Huge{\centerline{\basmalah}}\end{Arabic}
\flushright{\begin{Arabic}
\quranayah[18][1]
\end{Arabic}}
\flushleft{\begin{malayalam}
തന്‍റെ ദാസന്‍റെ മേല്‍ വേദഗ്രന്ഥമവതരിപ്പിക്കുകയും, അതിന് ഒരു വക്രതയും വരുത്താതിരിക്കുകയും ചെയ്ത അല്ലാഹുവിന് സ്തുതി.
\end{malayalam}}
\flushright{\begin{Arabic}
\quranayah[18][2]
\end{Arabic}}
\flushleft{\begin{malayalam}
ചൊവ്വായ നിലയില്‍. തന്‍റെപക്കല്‍ നിന്നുള്ള കഠിനമായ ശിക്ഷയെപ്പറ്റി താക്കീത് നല്‍കുവാനും, സല്‍കര്‍മ്മങ്ങള്‍ പ്രവര്‍ത്തിക്കുന്ന സത്യവിശ്വാസികള്‍ക്ക് ഉത്തമമായ പ്രതിഫലമുണ്ട് എന്ന് സന്തോഷവാര്‍ത്ത അറിയിക്കുവാനും വേണ്ടിയത്രെ അത്‌.
\end{malayalam}}
\flushright{\begin{Arabic}
\quranayah[18][3]
\end{Arabic}}
\flushleft{\begin{malayalam}
അത് (പ്രതിഫലം) അനുഭവിച്ച് കൊണ്ട് അവര്‍ എന്നെന്നും കഴിഞ്ഞുകൂടുന്നതായിരിക്കും.
\end{malayalam}}
\flushright{\begin{Arabic}
\quranayah[18][4]
\end{Arabic}}
\flushleft{\begin{malayalam}
അല്ലാഹു സന്താനത്തെ സ്വീകരിച്ചിരിക്കുന്നു എന്ന് പറഞ്ഞവര്‍ക്ക് താക്കീത് നല്‍കുവാന്‍ വേണ്ടിയുമാകുന്നു.
\end{malayalam}}
\flushright{\begin{Arabic}
\quranayah[18][5]
\end{Arabic}}
\flushleft{\begin{malayalam}
അവര്‍ക്കാകട്ടെ, അവരുടെ പിതാക്കള്‍ക്കാകട്ടെ അതിനെപ്പറ്റി യാതൊരു അറിവുമില്ല. അവരുടെ വായില്‍ നിന്ന് പുറത്ത് വരുന്ന ആ വാക്ക് ഗുരുതരമായിരിക്കുന്നു. അവര്‍ കള്ളമല്ലാതെ പറയുന്നില്ല.
\end{malayalam}}
\flushright{\begin{Arabic}
\quranayah[18][6]
\end{Arabic}}
\flushleft{\begin{malayalam}
അതിനാല്‍ ഈ സന്ദേശത്തില്‍ അവര്‍ വിശ്വസിച്ചില്ലെങ്കില്‍ അവര്‍ പിന്തിരിഞ്ഞ് പോയതിനെത്തുടര്‍ന്ന് (അതിലുള്ള) ദുഃഖത്താല്‍ നീ ജീവനൊടുക്കുന്നവനായേക്കാം.
\end{malayalam}}
\flushright{\begin{Arabic}
\quranayah[18][7]
\end{Arabic}}
\flushleft{\begin{malayalam}
തീര്‍ച്ചയായും ഭൂമുഖത്തുള്ളതിനെ നാം അതിന് ഒരു അലങ്കാരമാക്കിയിരിക്കുന്നു. മനുഷ്യരില്‍ ആരാണ് ഏറ്റവും നല്ല നിലയില്‍ പ്രവര്‍ത്തിക്കുന്നത് എന്ന് നാം പരീക്ഷിക്കുവാന്‍ വേണ്ടി.
\end{malayalam}}
\flushright{\begin{Arabic}
\quranayah[18][8]
\end{Arabic}}
\flushleft{\begin{malayalam}
തീര്‍ച്ചയായും അതിന്‍മേലുള്ളതെല്ലാം നശിപ്പിച്ച് നാം തന്നെ അതൊരു മൊട്ടയായ ഭൂപ്രദേശമാക്കി മാറ്റിക്കളയുന്നതുമാണ്‌.
\end{malayalam}}
\flushright{\begin{Arabic}
\quranayah[18][9]
\end{Arabic}}
\flushleft{\begin{malayalam}
അതല്ല, ഗുഹയുടെയും റഖീമിന്‍റെയും ആളുകള്‍ നമ്മുടെ ദൃഷ്ടാന്തങ്ങളുടെ കൂട്ടത്തില്‍ ഒരു അത്ഭുതമായിരുന്നുവെന്ന് നീ വിചാരിച്ചിരിക്കുകയാണോ ?
\end{malayalam}}
\flushright{\begin{Arabic}
\quranayah[18][10]
\end{Arabic}}
\flushleft{\begin{malayalam}
ആ യുവാക്കള്‍ ഗുഹയില്‍ അഭയം പ്രാപിച്ച സന്ദര്‍ഭം: അവര്‍ പറഞ്ഞു: ഞങ്ങളുടെ രക്ഷിതാവേ, നിന്‍റെ പക്കല്‍ നിന്നുള്ള കാരുണ്യം ഞങ്ങള്‍ക്ക് നീ നല്‍കുകയും ഞങ്ങളുടെ കാര്യം നേരാംവണ്ണം നിര്‍വഹിക്കുവാന്‍ നീ സൌകര്യം നല്‍കുകയും ചെയ്യേണമേ.
\end{malayalam}}
\flushright{\begin{Arabic}
\quranayah[18][11]
\end{Arabic}}
\flushleft{\begin{malayalam}
അങ്ങനെ കുറെയേറെ വര്‍ഷങ്ങള്‍ ആ ഗുഹയില്‍ വെച്ച് നാം അവരുടെ കാതുകള്‍ അടച്ചു (ഉറക്കിക്കളഞ്ഞു)
\end{malayalam}}
\flushright{\begin{Arabic}
\quranayah[18][12]
\end{Arabic}}
\flushleft{\begin{malayalam}
പിന്നെ അവര്‍ (ഗുഹയില്‍) താമസിച്ച കാലത്തെപ്പറ്റി കൃത്യമായി അറിയുന്നവര്‍ ഇരുകക്ഷികളില്‍ ആരാണെന്ന് അറിയാന്‍ തക്കവണ്ണം അവരെ നാം എഴുന്നേല്‍പിച്ചു.
\end{malayalam}}
\flushright{\begin{Arabic}
\quranayah[18][13]
\end{Arabic}}
\flushleft{\begin{malayalam}
അവരുടെ വര്‍ത്തമാനം നാം നിനക്ക് യഥാര്‍ത്ഥ രൂപത്തില്‍ വിവരിച്ചുതരാം. തങ്ങളുടെ രക്ഷിതാവില്‍ വിശ്വസിച്ച ഏതാനും യുവാക്കളായിരുന്നു അവര്‍. അവര്‍ക്കു നാം സന്‍മാര്‍ഗബോധം വര്‍ദ്ധിപ്പിക്കുകയും ചെയ്തു.
\end{malayalam}}
\flushright{\begin{Arabic}
\quranayah[18][14]
\end{Arabic}}
\flushleft{\begin{malayalam}
ഞങ്ങളുടെ രക്ഷിതാവ് ആകാശഭൂമികളുടെ രക്ഷിതാവ് ആകുന്നു. അവന്നു പുറമെ യാതൊരു ദൈവത്തോടും ഞങ്ങള്‍ പ്രാര്‍ത്ഥിക്കുന്നതേയല്ല, എങ്കില്‍ (അങ്ങനെ ഞങ്ങള്‍ ചെയ്യുന്ന പക്ഷം) തീര്‍ച്ചയായും ഞങ്ങള്‍ അന്യായമായ വാക്ക് പറഞ്ഞവരായി പോകും. എന്ന് അവര്‍ എഴുന്നേറ്റ് നിന്ന് പ്രഖ്യാപിച്ച സന്ദര്‍ഭത്തില്‍ അവരുടെ ഹൃദയങ്ങള്‍ക്കു നാം കെട്ടുറപ്പ് നല്‍കുകയും ചെയ്തു.
\end{malayalam}}
\flushright{\begin{Arabic}
\quranayah[18][15]
\end{Arabic}}
\flushleft{\begin{malayalam}
ഞങ്ങളുടെ ഈ ജനത അവന്നു പുറമെ പല ദൈവങ്ങളെയും സ്വീകരിച്ചിരിക്കുന്നു. അവരെ (ദൈവങ്ങളെ) സംബന്ധിച്ച് വ്യക്തമായ യാതൊരു പ്രമാണവും ഇവര്‍ കൊണ്ടുവരാത്തതെന്താണ്‌? അപ്പോള്‍ അല്ലാഹുവിന്‍റെ പേരില്‍ കള്ളം കെട്ടിച്ചമച്ചവനെക്കാള്‍ അക്രമിയായി ആരുണ്ട് ?
\end{malayalam}}
\flushright{\begin{Arabic}
\quranayah[18][16]
\end{Arabic}}
\flushleft{\begin{malayalam}
(അവര്‍ അന്യോന്യം പറഞ്ഞു:) അവരെയും അല്ലാഹു ഒഴികെ അവര്‍ ആരാധിച്ച് കൊണ്ടിരിക്കുന്നതിനെയും നിങ്ങള്‍ വിട്ടൊഴിഞ്ഞ സ്ഥിതിക്ക് നിങ്ങള്‍ ആ ഗുഹയില്‍ അഭയം പ്രാപിച്ച് കൊള്ളുക. നിങ്ങളുടെ രക്ഷിതാവ് അവന്‍റെ കാരുണ്യത്തില്‍ നിന്ന് നിങ്ങള്‍ക്ക് വിശാലമായി നല്‍കുകയും, നിങ്ങളുടെ കാര്യത്തില്‍ സൌകര്യമേര്‍പ്പെടുത്തിത്തരികയും ചെയ്യുന്നതാണ്‌.
\end{malayalam}}
\flushright{\begin{Arabic}
\quranayah[18][17]
\end{Arabic}}
\flushleft{\begin{malayalam}
സൂര്യന്‍ ഉദിക്കുമ്പോള്‍ അതവരുടെ ഗുഹവിട്ട് വലതുഭാഗത്തേക്ക് മാറിപ്പോകുന്നതായും, അത് അസ്തമിക്കുമ്പോള്‍ അതവരെ വിട്ട് കടന്ന് ഇടത് ഭാഗത്തേക്ക് പോകുന്നതായും നിനക്ക് കാണാം. അവരാകട്ടെ അതിന്‍റെ (ഗുഹയുടെ) വിശാലമായ ഒരു ഭാഗത്താകുന്നു. അത് അല്ലാഹുവിന്‍റെ ദൃഷ്ടാന്തങ്ങളില്‍ പെട്ടതത്രെ. അല്ലാഹു ആരെ നേര്‍വഴിയിലാക്കുന്നുവോ അവനാണ് സന്‍മാര്‍ഗം പ്രാപിച്ചവന്‍. അവന്‍ ആരെ ദുര്‍മാര്‍ഗത്തിലാക്കുന്നുവോ അവനെ നേര്‍വഴിയിലേക്ക് നയിക്കുന്ന ഒരു രക്ഷാധികാരിയെയും നീ കണ്ടെത്തുന്നതല്ല തന്നെ.
\end{malayalam}}
\flushright{\begin{Arabic}
\quranayah[18][18]
\end{Arabic}}
\flushleft{\begin{malayalam}
അവര്‍ ഉണര്‍ന്നിരിക്കുന്നവരാണ് എന്ന് നീ ധരിച്ച് പോകും.(വാസ്തവത്തില്‍) അവര്‍ ഉറങ്ങുന്നവരത്രെ. നാമവരെ വലത്തോട്ടും ഇടത്തോട്ടും മറിച്ച് കൊണ്ടിരിക്കുന്നു. അവരുടെ നായ ഗുഹാമുഖത്ത് അതിന്‍റെ രണ്ട് കൈകളും നീട്ടിവെച്ചിരിക്കുകയാണ്‌. അവരുടെ നേര്‍ക്ക് നീ എത്തി നോക്കുന്ന പക്ഷം നീ അവരില്‍ നിന്ന് പിന്തിരിഞ്ഞോടുകയും, അവരെപ്പറ്റി നീ ഭീതി പൂണ്ടവനായിത്തീരുകയും ചെയ്യും.
\end{malayalam}}
\flushright{\begin{Arabic}
\quranayah[18][19]
\end{Arabic}}
\flushleft{\begin{malayalam}
അപ്രകാരം-അവര്‍ അന്യോന്യം ചോദ്യം നടത്തുവാന്‍ തക്കവണ്ണം -നാം അവരെ എഴുന്നേല്‍പിച്ചു. അവരില്‍ ഒരാള്‍ ചോദിച്ചു: നിങ്ങളെത്ര കാലം (ഗുഹയില്‍) കഴിച്ചുകൂട്ടി? മറ്റുള്ളവര്‍ പറഞ്ഞു: നാം ഒരു ദിവസമോ ഒരു ദിവസത്തിന്‍റെ അല്‍പഭാഗമോ കഴിച്ചുകൂട്ടിയിരിക്കും. മറ്റു ചിലര്‍ പറഞ്ഞു: നിങ്ങളുടെ രക്ഷിതാവാകുന്നു നിങ്ങള്‍ കഴിച്ചുകൂട്ടിയതിനെപ്പറ്റി ശരിയായി അറിയുന്നവന്‍. എന്നാല്‍ നിങ്ങളില്‍ ഒരാളെ നിങ്ങളുടെ ഈ വെള്ളിനാണയവും കൊണ്ട് പട്ടണത്തിലേക്ക് അയക്കുക. അവിടെ ആരുടെ പക്കലാണ് ഏറ്റവും നല്ല ഭക്ഷണമുള്ളത് എന്ന് നോക്കിയിട്ട് അവിടെ നിന്ന് നിങ്ങള്‍ക്ക് അവന്‍ വല്ല ആഹാരവും കൊണ്ടുവരട്ടെ. അവന്‍ കരുതലോടെ പെരുമാറട്ടെ. നിങ്ങളെപ്പറ്റി അവന്‍ യാതൊരാളെയും അറിയിക്കാതിരിക്കട്ടെ.
\end{malayalam}}
\flushright{\begin{Arabic}
\quranayah[18][20]
\end{Arabic}}
\flushleft{\begin{malayalam}
തീര്‍ച്ചയായും നിങ്ങളെപ്പറ്റി അവര്‍ക്ക് അറിവ് ലഭിച്ചാല്‍ അവര്‍ നിങ്ങളെ എറിഞ്ഞ് കൊല്ലുകയോ, അവരുടെ മതത്തിലേക്ക് മടങ്ങാന്‍ നിങ്ങളെ നിര്‍ബന്ധിക്കുകയോ ചെയ്യും. എങ്കില്‍ (അങ്ങനെ നിങ്ങള്‍ മടങ്ങുന്ന പക്ഷം) നിങ്ങളൊരിക്കലും വിജയിക്കുകയില്ല തന്നെ.
\end{malayalam}}
\flushright{\begin{Arabic}
\quranayah[18][21]
\end{Arabic}}
\flushleft{\begin{malayalam}
അല്ലാഹുവിന്‍റെ വാഗ്ദാനം സത്യമാണെന്നും, അന്ത്യസമയത്തിന്‍റെ കാര്യത്തില്‍ യാതൊരു സംശയവുമില്ലെന്നും അവര്‍ (ജനങ്ങള്‍) മനസ്സിലാക്കുവാന്‍ വേണ്ടി നാം അവരെ (ഗുഹാവാസികളെ) കണ്ടെത്താന്‍ അപ്രകാരം അവസരം നല്‍കി. അവര്‍ അന്യോന്യം അവരുടെ (ഗുഹാവാസികളുടെ) കാര്യത്തില്‍ തര്‍ക്കിച്ചുകൊണ്ടിരുന്ന സന്ദര്‍ഭം (ശ്രദ്ധേയമാകുന്നു.) അവര്‍ (ഒരു വിഭാഗം) പറഞ്ഞു: നിങ്ങള്‍ അവരുടെ മേല്‍ ഒരു കെട്ടിടം നിര്‍മിക്കുക-അവരുടെ രക്ഷിതാവ് അവരെപ്പറ്റി നല്ലവണ്ണം അറിയുന്നവനത്രെ- അവരുടെ കാര്യത്തില്‍ പ്രാബല്യം നേടിയവര്‍ പറഞ്ഞു: നമുക്ക് അവരുടെ മേല്‍ ഒരു പള്ളി നിര്‍മിക്കുക തന്നെ ചെയ്യാം.
\end{malayalam}}
\flushright{\begin{Arabic}
\quranayah[18][22]
\end{Arabic}}
\flushleft{\begin{malayalam}
അവര്‍ (ജനങ്ങളില്‍ ഒരു വിഭാഗം) പറയും; (ഗുഹാവാസികള്‍) മൂന്ന് പേരാണ്‌, നാലാമത്തെത് അവരുടെ നായയാണ് എന്ന്‌. ചിലര്‍ പറയും: അവര്‍ അഞ്ചുപേരാണ്‌; ആറാമത്തെത് അവരുടെ നായയാണ് എന്ന്‌. അദൃശ്യകാര്യത്തെപ്പറ്റിയുള്ള ഊഹം പറയല്‍ മാത്രമാണത്‌. ചിലര്‍ പറയും: അവര്‍ ഏഴു പേരാണ്‌. എട്ടാമത്തെത് അവരുടെ നായയാണ് എന്ന് (നബിയേ) പറയുക; എന്‍റെ രക്ഷിതാവ് അവരുടെ എണ്ണത്തെപ്പറ്റി നല്ലവണ്ണം അറിയുന്നവനാണ്‌. ചുരുക്കം പേരല്ലാതെ അവരെപ്പറ്റി അറിയുകയില്ല. അതിനാല്‍ വ്യക്തമായ അറിവിന്‍റെ അടിസ്ഥാനത്തിലല്ലാതെ അവരുടെ വിഷയത്തില്‍ തര്‍ക്കിക്കരുത്‌. അവരില്‍ (ജനങ്ങളില്‍) ആരോടും അവരുടെ കാര്യത്തില്‍ നീ അഭിപ്രായം ആരായുകയും ചെയ്യരുത്‌.
\end{malayalam}}
\flushright{\begin{Arabic}
\quranayah[18][23]
\end{Arabic}}
\flushleft{\begin{malayalam}
യാതൊരു കാര്യത്തെപ്പറ്റിയും നാളെ ഞാനത് തീര്‍ച്ചയായും ചെയ്യാം എന്ന് നീ പറഞ്ഞുപോകരുത്‌.
\end{malayalam}}
\flushright{\begin{Arabic}
\quranayah[18][24]
\end{Arabic}}
\flushleft{\begin{malayalam}
അല്ലാഹു ഉദ്ദേശിക്കുന്നവെങ്കില്‍ (ചെയ്യാമെന്ന്‌) അല്ലാതെ. നീ മറന്നുപോകുന്ന പക്ഷം (ഓര്‍മവരുമ്പോള്‍) നിന്‍റെ രക്ഷിതാവിനെ അനുസ്മരിക്കുക. എന്‍റെ രക്ഷിതാവ് എന്നെ ഇതിനെക്കാള്‍ സന്‍മാര്‍ഗത്തോട് അടുത്ത ഒരു ജീവിതത്തിലേക്ക് നയിച്ചേക്കാം എന്ന് പറയുകയും ചെയ്യുക.
\end{malayalam}}
\flushright{\begin{Arabic}
\quranayah[18][25]
\end{Arabic}}
\flushleft{\begin{malayalam}
അവര്‍ അവരുടെ ഗുഹയില്‍ മുന്നൂറ് വര്‍ഷം താമസിച്ചു. അവര്‍ ഒമ്പതു വര്‍ഷം കൂടുതലാക്കുകയും ചെയ്തു.
\end{malayalam}}
\flushright{\begin{Arabic}
\quranayah[18][26]
\end{Arabic}}
\flushleft{\begin{malayalam}
നീ പറയുക: അവര്‍ താമസിച്ചതിനെപ്പറ്റി അല്ലാഹു നല്ലവണ്ണം അറിയുന്നവനാകുന്നു. ആകാശങ്ങളിലെയും ഭൂമിയിലെയും അദൃശ്യജ്ഞാനം അവന്നാണുള്ളത്‌. അവന്‍ എത്ര കാഴ്ചയുള്ളവന്‍. എത്ര കേള്‍വിയുള്ളവന്‍! അവന്നു പുറമെ അവര്‍ക്ക് (മനുഷ്യര്‍ക്ക്‌) യാതൊരു രക്ഷാധികാരിയുമില്ല. തന്‍റെ തീരുമാനാധികാരത്തില്‍ യാതൊരാളെയും അവന്‍ പങ്കുചേര്‍ക്കുകയുമില്ല.
\end{malayalam}}
\flushright{\begin{Arabic}
\quranayah[18][27]
\end{Arabic}}
\flushleft{\begin{malayalam}
നിനക്ക് ബോധനം നല്‍കപ്പെട്ട നിന്‍റെ രക്ഷിതാവിന്‍റെ ഗ്രന്ഥം നീ പാരായണം ചെയ്യുക. അവന്‍റെ വചനങ്ങള്‍ക്ക് ഭേദഗതി വരുത്താനാരുമില്ല. അവന്നു പുറമെ യാതൊരു അഭയസ്ഥാനവും നീ ഒരിക്കലും കണ്ടെത്തുകയുമില്ല.
\end{malayalam}}
\flushright{\begin{Arabic}
\quranayah[18][28]
\end{Arabic}}
\flushleft{\begin{malayalam}
തങ്ങളുടെ രക്ഷിതാവിന്‍റെ മുഖം ലക്ഷ്യമാക്കിക്കൊണ്ട് കാലത്തും വൈകുന്നേരവും അവനോട് പ്രാര്‍ത്ഥിച്ച് കൊണ്ടിരിക്കുന്നവരുടെ കൂടെ നീ നിന്‍റെ മനസ്സിനെ അടക്കി നിര്‍ത്തുക. ഇഹലോകജീവിതത്തിന്‍റെ അലങ്കാരം ലക്ഷ്യമാക്കിക്കൊണ്ട് നിന്‍റെ കണ്ണുകള്‍ അവരെ വിട്ടുമാറിപ്പോകാതിരിക്കട്ടെ. ഏതൊരുവന്‍റെ ഹൃദയത്തെ നമ്മുടെ സ്മരണയെ വിട്ടു നാം അശ്രദ്ധമാക്കിയിരിക്കുന്നുവോ, ഏതൊരുവന്‍ തന്നിഷ്ടത്തെ പിന്തുടരുകയും അവന്‍റെ കാര്യം അതിരുകവിഞ്ഞതായിരിക്കുകയും ചെയ്തുവോ, അവനെ നീ അനുസരിച്ചു പോകരുത്‌.
\end{malayalam}}
\flushright{\begin{Arabic}
\quranayah[18][29]
\end{Arabic}}
\flushleft{\begin{malayalam}
പറയുക: സത്യം നിങ്ങളുടെ രക്ഷിതാവിങ്കല്‍ നിന്നുള്ളതാകുന്നു. അതിനാല്‍ ഇഷ്ടമുള്ളവര്‍ വിശ്വസിക്കട്ടെ. ഇഷ്ടമുള്ളവര്‍ അവിശ്വസിക്കട്ടെ. അക്രമികള്‍ക്ക് നാം നരകാഗ്നി ഒരുക്കി വെച്ചിട്ടുണ്ട്‌. അതിന്‍റെ കൂടാരം അവരെ വലയം ചെയ്തിരിക്കുന്നു. അവര്‍ വെള്ളത്തിനപേക്ഷിക്കുന്ന പക്ഷം ഉരുക്കിയ ലോഹം പോലുള്ള ഒരു വെള്ളമായിരിക്കും. അവര്‍ക്ക് കുടിക്കാന്‍ നല്‍കപ്പെടുന്നത്‌. അത് മുഖങ്ങളെ എരിച്ച് കളയും. വളരെ ദുഷിച്ച പാനീയം തന്നെ. അത് (നരകം) വളരെ ദുഷിച്ച വിശ്രമ സ്ഥലം തന്നെ.
\end{malayalam}}
\flushright{\begin{Arabic}
\quranayah[18][30]
\end{Arabic}}
\flushleft{\begin{malayalam}
തീര്‍ച്ചയായും വിശ്വസിക്കുകയും സല്‍കര്‍മ്മങ്ങള്‍ പ്രവര്‍ത്തിക്കുകയും ചെയ്തവരാരോ അത്തരം സല്‍പ്രവര്‍ത്തനം നടത്തുന്ന യാതൊരാളുടെയും പ്രതിഫലം നാം തീര്‍ച്ചയായും പാഴാക്കുന്നതല്ല.
\end{malayalam}}
\flushright{\begin{Arabic}
\quranayah[18][31]
\end{Arabic}}
\flushleft{\begin{malayalam}
അക്കൂട്ടര്‍ക്കാകുന്നു സ്ഥിരവാസത്തിനുള്ള സ്വര്‍ഗത്തോപ്പുകള്‍. അവരുടെ താഴ്ഭാഗത്ത്കൂടി അരുവികള്‍ ഒഴുകിക്കൊണ്ടിരിക്കുന്നതാണ്‌. അവര്‍ക്കവിടെ സ്വര്‍ണം കൊണ്ടുള്ള വളകള്‍ അണിയിക്കപ്പെടുന്നതാണ്‌. നേരിയതും കട്ടിയുള്ളതുമായ പച്ചപ്പട്ടു വസ്ത്രങ്ങള്‍ അവര്‍ ധരിക്കുകയും ചെയ്യും. അവിടെ അവര്‍ അലങ്കരിച്ച കട്ടിലുകളില്‍ ചാരിയിരുന്ന് വിശ്രമിക്കുന്നവരായിരിക്കും. എത്ര വിശിഷ്ടമായ പ്രതിഫലം, എത്ര ഉത്തമമായ വിശ്രമസ്ഥലം!
\end{malayalam}}
\flushright{\begin{Arabic}
\quranayah[18][32]
\end{Arabic}}
\flushleft{\begin{malayalam}
നീ അവര്‍ക്ക് ഒരു ഉപമ വിവരിച്ചുകൊടുക്കുക. രണ്ട് പുരുഷന്‍മാര്‍. അവരില്‍ ഒരാള്‍ക്ക് നാം രണ്ട് മുന്തിരിത്തോട്ടങ്ങള്‍ നല്‍കി. അവയെ (തോട്ടങ്ങളെ) നാം ഈന്തപ്പനകൊണ്ട് വലയം ചെയ്തു. അവയ്ക്കിടയില്‍ (തോട്ടങ്ങള്‍ക്കിടയില്‍) ധാന്യകൃഷിയിടവും നാം നല്‍കി.
\end{malayalam}}
\flushright{\begin{Arabic}
\quranayah[18][33]
\end{Arabic}}
\flushleft{\begin{malayalam}
ഇരു തോട്ടങ്ങളും അവയുടെ ഫലങ്ങള്‍ നല്‍കി വന്നു. അതില്‍ യാതൊരു ക്രമക്കേടും വരുത്തിയില്ല. അവയ്ക്കിടയിലൂടെ നാം ഒരു നദി ഒഴുക്കുകയും ചെയ്തു.
\end{malayalam}}
\flushright{\begin{Arabic}
\quranayah[18][34]
\end{Arabic}}
\flushleft{\begin{malayalam}
അവന്നു പല വരുമാനവുമുണ്ടായിരുന്നു. അങ്ങനെ അവന്‍ തന്‍റെ ചങ്ങാതിയോട് സംവാദം നടത്തിക്കൊണ്ടിരിക്കെ പറയുകയുണ്ടായി: ഞാനാണ് നിന്നെക്കാള്‍ കൂടുതല്‍ ധനമുള്ളവനും കൂടുതല്‍ സംഘബലമുള്ളവനും.
\end{malayalam}}
\flushright{\begin{Arabic}
\quranayah[18][35]
\end{Arabic}}
\flushleft{\begin{malayalam}
സ്വന്തത്തോട് തന്നെ അന്യായം പ്രവര്‍ത്തിച്ച് കൊണ്ട് അവന്‍ തന്‍റെ തോട്ടത്തില്‍ പ്രവേശിച്ചു. അവന്‍ പറഞ്ഞു: ഒരിക്കലും ഇതൊന്നും നശിച്ച് പോകുമെന്ന് ഞാന്‍ വിചാരിക്കുന്നില്ല.
\end{malayalam}}
\flushright{\begin{Arabic}
\quranayah[18][36]
\end{Arabic}}
\flushleft{\begin{malayalam}
അന്ത്യസമയം നിലവില്‍ വരും എന്നും ഞാന്‍ വിചാരിക്കുന്നില്ല. ഇനി ഞാന്‍ എന്‍റെ രക്ഷിതാവിങ്കലേക്ക് മടക്കപ്പെടുകയാണെങ്കിലോ, തീര്‍ച്ചയായും, മടങ്ങിച്ചെല്ലുന്നതിന് ഇതിനേക്കാള്‍ ഉത്തമമായ ഒരു സ്ഥലം എനിക്ക് ലഭിക്കുക തന്നെ ചെയ്യും.
\end{malayalam}}
\flushright{\begin{Arabic}
\quranayah[18][37]
\end{Arabic}}
\flushleft{\begin{malayalam}
അവന്‍റെ ചങ്ങാതി അവനുമായി സംവാദം നടത്തിക്കൊണ്ടിരിക്കെ പറഞ്ഞു: മണ്ണില്‍ നിന്നും അനന്തരം ബീജത്തില്‍ നിന്നും നിന്നെ സൃഷ്ടിക്കുകയും, പിന്നീട് നിന്നെ ഒരു പുരുഷനായി സംവിധാനിക്കുകയും ചെയ്തവനില്‍ നീ അവിശ്വസിച്ചിരിക്കുകയാണോ?
\end{malayalam}}
\flushright{\begin{Arabic}
\quranayah[18][38]
\end{Arabic}}
\flushleft{\begin{malayalam}
എന്നാല്‍ (എന്‍റെ വിശ്വാസമിതാണ്‌.) അവന്‍ അഥവാ അല്ലാഹുവാകുന്നു എന്‍റെ രക്ഷിതാവ്‌. എന്‍റെ രക്ഷിതാവിനോട് യാതൊന്നിനെയും ഞാന്‍ പങ്കുചേര്‍ക്കുകയില്ല.
\end{malayalam}}
\flushright{\begin{Arabic}
\quranayah[18][39]
\end{Arabic}}
\flushleft{\begin{malayalam}
നീ നിന്‍റെ തോട്ടത്തില്‍ കടന്ന സമയത്ത്‌, ഇത് അല്ലാഹു ഉദ്ദേശിച്ചതത്രെ, അല്ലാഹുവെക്കൊണ്ടല്ലാതെ യാതൊരു ശക്തിയും ഇല്ല എന്ന് നിനക്ക് പറഞ്ഞ് കൂടായിരുന്നോ? നിന്നെക്കാള്‍ ധനവും സന്താനവും കുറഞ്ഞവനായി നീ എന്നെ കാണുന്നുവെങ്കില്‍.
\end{malayalam}}
\flushright{\begin{Arabic}
\quranayah[18][40]
\end{Arabic}}
\flushleft{\begin{malayalam}
എന്‍റെ രക്ഷിതാവ് എനിക്ക് നിന്‍റെ തോട്ടത്തെക്കാള്‍ നല്ലത് നല്‍കി എന്ന് വരാം. നിന്‍റെ തോട്ടത്തിന്‍റെ നേരെ അവന്‍ ആകാശത്ത് നിന്ന് ശിക്ഷ അയക്കുകയും, അങ്ങനെ അത് ചതുപ്പുനിലമായിത്തീരുകയും ചെയ്തു എന്ന് വരാം.
\end{malayalam}}
\flushright{\begin{Arabic}
\quranayah[18][41]
\end{Arabic}}
\flushleft{\begin{malayalam}
അല്ലെങ്കില്‍ അതിലെ വെള്ളം നിനക്ക് ഒരിക്കലും തേടിപ്പിടിച്ച് കൊണ്ട് വരുവാന്‍ കഴിയാത്ത വിധം വറ്റിപ്പോയെന്നും വരാം.
\end{malayalam}}
\flushright{\begin{Arabic}
\quranayah[18][42]
\end{Arabic}}
\flushleft{\begin{malayalam}
അവന്‍റെ ഫലസമൃദ്ധി (നാശത്താല്‍) വലയം ചെയ്യപ്പെട്ടു. അവ (തോട്ടങ്ങള്‍) അവയുടെ പന്തലുകളോടെ വീണടിഞ്ഞ് കിടക്കവെ താന്‍ അതില്‍ ചെലവഴിച്ചതിന്‍റെ പേരില്‍ അവന്‍ (നഷ്ടബോധത്താല്‍) കൈ മലര്‍ത്തുന്നവനായിത്തീര്‍ന്നു. എന്‍റെ രക്ഷിതാവിനോട് ആരെയും ഞാന്‍ പങ്കുചേര്‍ക്കാതിരുന്നെങ്കില്‍ എത്ര നന്നായിരുന്നേനെ എന്ന് അവന്‍ പറയുകയും ചെയ്ത്കൊണ്ടിരുന്നു.
\end{malayalam}}
\flushright{\begin{Arabic}
\quranayah[18][43]
\end{Arabic}}
\flushleft{\begin{malayalam}
അല്ലാഹുവിന് പുറമെ യാതൊരു കക്ഷിയും അവന്ന് സഹായം നല്‍കുവാനുണ്ടായില്ല. അവന്ന് (സ്വയം) അതിജയിക്കുവാന്‍ കഴിഞ്ഞതുമില്ല.
\end{malayalam}}
\flushright{\begin{Arabic}
\quranayah[18][44]
\end{Arabic}}
\flushleft{\begin{malayalam}
യഥാര്‍ത്ഥ ദൈവമായ അല്ലാഹുവിന്നത്രെ അവിടെ രക്ഷാധികാരം. നല്ല പ്രതിഫലം നല്‍കുന്നവനും നല്ല പര്യവസാനത്തിലെത്തുക്കുന്നവനും അവനത്രെ.
\end{malayalam}}
\flushright{\begin{Arabic}
\quranayah[18][45]
\end{Arabic}}
\flushleft{\begin{malayalam}
(നബിയേ,) നീ അവര്‍ക്ക് ഐഹികജീവിതത്തിന്‍റെ ഉപമ വിവരിച്ചുകൊടുക്കുക: ആകാശത്ത് നിന്ന് നാം വെള്ളം ഇറക്കി. അതുമൂലം ഭൂമിയില്‍ സസ്യങ്ങള്‍ ഇടകലര്‍ന്ന് വളര്‍ന്നു. താമസിയാതെ അത് കാറ്റുകള്‍ പറത്തിക്കളയുന്ന തുരുമ്പായിത്തീര്‍ന്നു. (അതുപോലെയത്രെ ഐഹികജീവിതം.) അല്ലാഹു ഏത് കാര്യത്തിനും കഴിവുള്ളവനാകുന്നു.
\end{malayalam}}
\flushright{\begin{Arabic}
\quranayah[18][46]
\end{Arabic}}
\flushleft{\begin{malayalam}
സ്വത്തും സന്താനങ്ങളും ഐഹികജീവിതത്തിന്‍റെ അലങ്കാരമാകുന്നു. എന്നാല്‍ നിലനില്‍ക്കുന്ന സല്‍കര്‍മ്മങ്ങളാണ് നിന്‍റെ രക്ഷിതാവിങ്കല്‍ ഉത്തമമായ പ്രതിഫലമുള്ളതും ഉത്തമമായ പ്രതീക്ഷ നല്‍കുന്നതും.
\end{malayalam}}
\flushright{\begin{Arabic}
\quranayah[18][47]
\end{Arabic}}
\flushleft{\begin{malayalam}
പര്‍വ്വതങ്ങളെ നാം സഞ്ചരിപ്പിക്കുകയും തെളിഞ്ഞ് നിരപ്പായ നിലയില്‍ ഭൂമി നിനക്ക് കാണുമാറാകുകയും, തുടര്‍ന്ന് അവരില്‍ നിന്ന് (മനുഷ്യരില്‍ നിന്ന്‌) ഒരാളെയും വിട്ടുകളയാതെ നാം അവരെ ഒരുമിച്ചുകൂട്ടുകയും ചെയ്യുന്ന ദിവസം (ശ്രദ്ധേയമാകുന്നു.)
\end{malayalam}}
\flushright{\begin{Arabic}
\quranayah[18][48]
\end{Arabic}}
\flushleft{\begin{malayalam}
നിന്‍റെ രക്ഷിതാവിന്‍റെ മുമ്പാകെ അവര്‍ അണിയണിയായി പ്രദര്‍ശിപ്പിക്കപ്പെടുകയും ചെയ്യും. (അന്നവന്‍ പറയും:) നിങ്ങളെ നാം ആദ്യതവണ സൃഷ്ടിച്ച പ്രകാരം നിങ്ങളിതാ നമ്മുടെ അടുത്ത് വന്നിരിക്കുന്നു. എന്നാല്‍ നിങ്ങള്‍ക്ക് നാം ഒരു നിശ്ചിത സമയം ഏര്‍പെടുത്തുകയേയില്ല എന്ന് നിങ്ങള്‍ ജല്‍പിക്കുകയാണ് ചെയ്തത്‌.
\end{malayalam}}
\flushright{\begin{Arabic}
\quranayah[18][49]
\end{Arabic}}
\flushleft{\begin{malayalam}
(കര്‍മ്മങ്ങളുടെ) രേഖ വെക്കപ്പെടും. അപ്പോള്‍ കുറ്റവാളികളെ, അതിലുള്ളതിനെപ്പറ്റി ഭയവിഹ്വലരായ നിലയില്‍ നിനക്ക് കാണാം. അവര്‍ പറയും: അയ്യോ! ഞങ്ങള്‍ക്ക് നാശം. ഇതെന്തൊരു രേഖയാണ്‌? ചെറുതോ വലുതോ ആയ ഒന്നും തന്നെ അത് കൃത്യമായി രേഖപ്പെടുത്താതെ വിട്ടുകളയുന്നില്ലല്ലോ! തങ്ങള്‍ പ്രവര്‍ത്തിച്ചതൊക്കെ (രേഖയില്‍) നിലവിലുള്ളതായി അവര്‍ കണ്ടെത്തും. നിന്‍റെ രക്ഷിതാവ് യാതൊരാളോടും അനീതി കാണിക്കുകയില്ല.
\end{malayalam}}
\flushright{\begin{Arabic}
\quranayah[18][50]
\end{Arabic}}
\flushleft{\begin{malayalam}
നാം മലക്കുകളോട് നിങ്ങള്‍ ആദമിന് പ്രണാമം ചെയ്യുക എന്ന് പറഞ്ഞ സന്ദര്‍ഭം (ശ്രദ്ധേയമത്രെ.) അവര്‍ പ്രണാമം ചെയ്തു. ഇബ്ലീസ് ഒഴികെ. അവന്‍ ജിന്നുകളില്‍ പെട്ടവനായിരുന്നു. അങ്ങനെ തന്‍റെ രക്ഷിതാവിന്‍റെ കല്‍പന അവന്‍ ധിക്കരിച്ചു. എന്നിരിക്കെ നിങ്ങള്‍ എന്നെ വിട്ട് അവനെയും അവന്‍റെ സന്തതികളെയും രക്ഷാധികാരികളാക്കുകയാണോ? അവര്‍ നിങ്ങളുടെ ശത്രുക്കളത്രെ. അക്രമികള്‍ക്ക് (അല്ലാഹുവിന്‌) പകരം കിട്ടിയത് വളരെ ചീത്ത തന്നെ.
\end{malayalam}}
\flushright{\begin{Arabic}
\quranayah[18][51]
\end{Arabic}}
\flushleft{\begin{malayalam}
ആകാശങ്ങളുടെയും ഭൂമിയുടെയും സൃഷ്ടിപ്പിനാകട്ടെ, അവരുടെ തന്നെ സൃഷ്ടിപ്പിനാകട്ടെ നാം അവരെ സാക്ഷികളാക്കിയിട്ടില്ല. വഴിപിഴപ്പിക്കുന്നവരെ ഞാന്‍ സഹായികളായി സ്വീകരിക്കുന്നവനല്ലതാനും.
\end{malayalam}}
\flushright{\begin{Arabic}
\quranayah[18][52]
\end{Arabic}}
\flushleft{\begin{malayalam}
എന്‍റെ പങ്കാളികളെന്ന് നിങ്ങള്‍ ജല്‍പിച്ച് കൊണ്ടിരുന്നവരെ നിങ്ങള്‍ വിളിച്ച് നോക്കൂ എന്ന് അവന്‍ (അല്ലാഹു) പറയുന്ന ദിവസം (ശ്രദ്ധേയമത്രെ.) അപ്പോള്‍ ഇവര്‍ അവരെ വിളിച്ച് നോക്കുന്നതാണ്‌. എന്നാല്‍ അവര്‍ ഇവര്‍ക്ക് ഉത്തരം നല്‍കുന്നതല്ല. അവര്‍ക്കിടയില്‍ നാം ഒരു നാശഗര്‍ത്തം ഉണ്ടാക്കുകയും ചെയ്യും.
\end{malayalam}}
\flushright{\begin{Arabic}
\quranayah[18][53]
\end{Arabic}}
\flushleft{\begin{malayalam}
കുറ്റവാളികള്‍ നരകം നേരില്‍ കാണും. അപ്പോള്‍ തങ്ങള്‍ അതില്‍ അകപ്പെടാന്‍ പോകുകയാണെന്ന് അവര്‍ മനസ്സിലാക്കും. അതില്‍ നിന്ന് വിട്ടുമാറിപ്പോകാന്‍ ഒരു മാര്‍ഗവും അവര്‍ കണ്ടെത്തുകയുമില്ല.
\end{malayalam}}
\flushright{\begin{Arabic}
\quranayah[18][54]
\end{Arabic}}
\flushleft{\begin{malayalam}
തീര്‍ച്ചയായും ജനങ്ങള്‍ക്കുവേണ്ടി എല്ലാവിധ ഉപമകളും ഈ ഖുര്‍ആനില്‍ നാം വിവിധ തരത്തില്‍ വിവരിച്ചിരിക്കുന്നു. എന്നാല്‍ മനുഷ്യന്‍ അത്യധികം തര്‍ക്കസ്വഭാവമുള്ളവനത്രെ.
\end{malayalam}}
\flushright{\begin{Arabic}
\quranayah[18][55]
\end{Arabic}}
\flushleft{\begin{malayalam}
തങ്ങള്‍ക്കു മാര്‍ഗദര്‍ശനം വന്നുകിട്ടിയപ്പോള്‍ അതില്‍ വിശ്വസിക്കുകയും, തങ്ങളുടെ രക്ഷിതാവിനോട് പാപമോചനം തേടുകയും ചെയ്യുന്നതിന് ജനങ്ങള്‍ക്ക് തടസ്സമായത് പൂര്‍വ്വികന്‍മാരുടെ കാര്യത്തിലുണ്ടായ അതേ നടപടി അവര്‍ക്കും വരണം. അല്ലെങ്കില്‍ അവര്‍ക്ക് നേരിട്ട് ശിക്ഷ വരണം എന്ന അവരുടെ നിലപാട് മാത്രമാകുന്നു.
\end{malayalam}}
\flushright{\begin{Arabic}
\quranayah[18][56]
\end{Arabic}}
\flushleft{\begin{malayalam}
സന്തോഷവാര്‍ത്ത അറിയിക്കുന്നവരായിക്കൊണ്ടും, താക്കീത് നല്‍കുന്നവരായിക്കൊണ്ടും മാത്രമാണ് നാം ദൂതന്‍മാരെ നിയോഗിക്കുന്നത്‌. അവിശ്വസിച്ചവര്‍ മിഥ്യാവാദവുമായി തര്‍ക്കിച്ച് കൊണ്ടിരിക്കുന്നു; അത് മൂലം സത്യത്തെ തകര്‍ത്ത് കളയുവാന്‍ വേണ്ടി. എന്‍റെ ദൃഷ്ടാന്തങ്ങളെയും അവര്‍ക്ക് നല്‍കപ്പെട്ട താക്കീതുകളെയും അവര്‍ പരിഹാസ്യമാക്കിത്തീര്‍ക്കുകയും ചെയ്തിരിക്കുന്നു.
\end{malayalam}}
\flushright{\begin{Arabic}
\quranayah[18][57]
\end{Arabic}}
\flushleft{\begin{malayalam}
തന്‍റെ രക്ഷിതാവിന്‍റെ ദൃഷ്ടാന്തങ്ങളെപ്പറ്റി ഓര്‍മിപ്പിക്കപ്പെട്ടിട്ട് അതില്‍ നിന്ന് തിരിഞ്ഞുകളയുകയും, തന്‍റെ കൈകള്‍ മുന്‍കൂട്ടി ചെയ്തത് (ദുഷ്കര്‍മ്മങ്ങള്‍) മറന്നുകളയുകയും ചെയ്തവനെക്കാള്‍ അക്രമിയായി ആരുണ്ട്‌? തീര്‍ച്ചയായും അവരത് ഗ്രഹിക്കുന്നതിന് (തടസ്സമായി) നാം അവരുടെ ഹൃദയങ്ങളില്‍ മൂടികളും, അവരുടെ കാതുകളില്‍ ഭാര (അടപ്പ്‌) വും ഏര്‍പെടുത്തിയിരിക്കുന്നു. (അങ്ങനെയിരിക്കെ) നീ അവരെ സന്‍മാര്‍ഗത്തിലേക്ക് ക്ഷണിക്കുന്ന പക്ഷം അവര്‍ ഒരിക്കലും സന്‍മാര്‍ഗം സ്വീകരിക്കുകയില്ല.
\end{malayalam}}
\flushright{\begin{Arabic}
\quranayah[18][58]
\end{Arabic}}
\flushleft{\begin{malayalam}
നിന്‍റെ രക്ഷിതാവ് ഏറെ പൊറുക്കുന്നവനും കരുണയുള്ളവനുമാകുന്നു. അവര്‍ ചെയ്ത് കൂട്ടിയതിന് അവന്‍ അവര്‍ക്കെതിരില്‍ നടപടി എടുക്കുകയായിരുന്നെങ്കില്‍ അവര്‍ക്കവന്‍ ഉടന്‍ തന്നെ ശിക്ഷ നല്‍കുമായിരുന്നു. പക്ഷെ അവര്‍ക്കൊരു നിശ്ചിത അവധിയുണ്ട്‌. അതിനെ മറികടന്ന് കൊണ്ട് രക്ഷപ്രാപിക്കാവുന്ന ഒരു സ്ഥാനവും അവര്‍ കണ്ടെത്തുകയേയില്ല.
\end{malayalam}}
\flushright{\begin{Arabic}
\quranayah[18][59]
\end{Arabic}}
\flushleft{\begin{malayalam}
ആ രാജ്യങ്ങള്‍ അക്രമത്തില്‍ ഏര്‍പെട്ടപ്പോള്‍ അവരെ നാം നശിപ്പിച്ച് കളഞ്ഞു. അവരുടെ നാശത്തിന് നാം ഒരു നിശ്ചിത അവധി വെച്ചിട്ടുണ്ട്‌.
\end{malayalam}}
\flushright{\begin{Arabic}
\quranayah[18][60]
\end{Arabic}}
\flushleft{\begin{malayalam}
മൂസാ തന്‍റെ ഭൃത്യനോട് ഇപ്രകാരം പറഞ്ഞ സന്ദര്‍ഭം (ശ്രദ്ധേയമാകുന്നു:) ഞാന്‍ രണ്ട് കടലുകള്‍ കൂടിച്ചേരുന്നിടത്ത് എത്തുകയോ, അല്ലെങ്കില്‍ സുദീര്‍ഘമായ ഒരു കാലഘട്ടം മുഴുവന്‍ നടന്ന് കഴിയുകയോ ചെയ്യുന്നത് വരെ ഞാന്‍ (ഈ യാത്ര) തുടര്‍ന്ന് കൊണേ്ടയിരിക്കും.
\end{malayalam}}
\flushright{\begin{Arabic}
\quranayah[18][61]
\end{Arabic}}
\flushleft{\begin{malayalam}
അങ്ങനെ അവര്‍ അവ (കടലുകള്‍) രണ്ടും കൂടിച്ചേരുന്നിടത്തെത്തിയപ്പോള്‍ തങ്ങളുടെ മത്സ്യത്തിന്‍റെ കാര്യം മറന്നുപോയി. അങ്ങനെ അത് കടലില്‍ (ചാടി) അത് പോയ മാര്‍ഗം ഒരു തുരങ്കം (പോലെ) ആക്കിത്തീര്‍ത്തു.
\end{malayalam}}
\flushright{\begin{Arabic}
\quranayah[18][62]
\end{Arabic}}
\flushleft{\begin{malayalam}
അങ്ങനെ അവര്‍ ആ സ്ഥലം വിട്ട് മുന്നോട്ട് പോയിക്കഴിഞ്ഞപ്പോള്‍ മൂസാ തന്‍റെ ഭൃത്യനോട് പറഞ്ഞു: നീ നമുക്ക് നമ്മുടെ ഭക്ഷണം കൊണ്ട് വാ. നമ്മുടെ ഈ യാത്ര നിമിത്തം നമുക്ക് ക്ഷീണം നേരിട്ടിരിക്കുന്നു.
\end{malayalam}}
\flushright{\begin{Arabic}
\quranayah[18][63]
\end{Arabic}}
\flushleft{\begin{malayalam}
അവന്‍ പറഞ്ഞു: താങ്കള്‍ കണ്ടുവോ? നാം ആ പാറക്കല്ലില്‍ അഭയം പ്രാപിച്ച സന്ദര്‍ഭത്തില്‍ ഞാന്‍ ആ മത്സ്യത്തെ മറന്നുപോകുക തന്നെ ചെയ്തു. അത് പറയാന്‍ എന്നെ മറപ്പിച്ചത് പിശാചല്ലാതെ മറ്റാരുമല്ല. അത് കടലിലൂടെ സഞ്ചരിച്ച വഴി ഒരു അത്ഭുതമാക്കിത്തീര്‍ക്കുകയും ചെയ്തിരിക്കുന്നു.
\end{malayalam}}
\flushright{\begin{Arabic}
\quranayah[18][64]
\end{Arabic}}
\flushleft{\begin{malayalam}
അദ്ദേഹം (മൂസാ) പറഞ്ഞു: അതുതന്നെയാണ് നാം തേടിക്കൊണ്ടിരുന്നത്‌. ഉടനെ അവര്‍ രണ്ട് പേരും തങ്ങളുടെ കാല്‍പാടുകള്‍ നോക്കിക്കൊണ്ട് മടങ്ങി.
\end{malayalam}}
\flushright{\begin{Arabic}
\quranayah[18][65]
\end{Arabic}}
\flushleft{\begin{malayalam}
അപ്പോള്‍ അവര്‍ രണ്ടുപേരും നമ്മുടെ ദാസന്‍മാരില്‍ ഒരാളെ കണ്ടെത്തി. അദ്ദേഹത്തിന് നാം നമ്മുടെ പക്കല്‍ നിന്നുള്ള കാരുണ്യം നല്‍കുകയും, നമ്മുടെ പക്കല്‍ നിന്നുള്ള ജ്ഞാനം നാം അദ്ദേഹത്തെ പഠിപ്പിക്കുകയും ചെയ്തിട്ടുണ്ട്‌.
\end{malayalam}}
\flushright{\begin{Arabic}
\quranayah[18][66]
\end{Arabic}}
\flushleft{\begin{malayalam}
മൂസാ അദ്ദേഹത്തോട് പറഞ്ഞു: താങ്കള്‍ക്ക് പഠിപ്പിക്കപ്പെട്ട സന്‍മാര്‍ഗജ്ഞാനത്തില്‍ നിന്ന് എനിക്ക് താങ്കള്‍ പഠിപ്പിച്ചുതരുന്നതിന്നായി ഞാന്‍ താങ്കളെ അനുഗമിക്കട്ടെ?
\end{malayalam}}
\flushright{\begin{Arabic}
\quranayah[18][67]
\end{Arabic}}
\flushleft{\begin{malayalam}
അദ്ദേഹം പറഞ്ഞു: തീര്‍ച്ചയായും താങ്കള്‍ക്ക് എന്‍റെ കൂടെ ക്ഷമിച്ച് കഴിയാന്‍ സാധിക്കുകയേ ഇല്ല.
\end{malayalam}}
\flushright{\begin{Arabic}
\quranayah[18][68]
\end{Arabic}}
\flushleft{\begin{malayalam}
താങ്കള്‍ സൂക്ഷ്മമായി അറിഞ്ഞിട്ടില്ലാത്ത ഒരു വിഷയത്തില്‍ താങ്കള്‍ക്കെങ്ങനെ ക്ഷമിക്കാനാകും.?
\end{malayalam}}
\flushright{\begin{Arabic}
\quranayah[18][69]
\end{Arabic}}
\flushleft{\begin{malayalam}
അദ്ദേഹം പറഞ്ഞു: അല്ലാഹു ഉദ്ദേശിക്കുന്ന പക്ഷം ക്ഷമയുള്ളവനായി താങ്കള്‍ എന്നെ കണ്ടെത്തുന്നതാണ്‌. ഞാന്‍ താങ്കളുടെ ഒരു കല്‍പനയ്ക്കും എതിര്‍ പ്രവര്‍ത്തിക്കുന്നതല്ല.
\end{malayalam}}
\flushright{\begin{Arabic}
\quranayah[18][70]
\end{Arabic}}
\flushleft{\begin{malayalam}
അദ്ദേഹം പറഞ്ഞു: എന്നാല്‍ താങ്കള്‍ എന്നെ അനുഗമിക്കുന്ന പക്ഷം യാതൊരു കാര്യത്തെപ്പറ്റിയും താങ്കള്‍ എന്നോട് ചോദിക്കരുത്‌: അതിനെപ്പറ്റിയുള്ള വിവരം ഞാന്‍ തന്നെ താങ്കള്‍ക്കു പറഞ്ഞുതരുന്നത് വരെ.
\end{malayalam}}
\flushright{\begin{Arabic}
\quranayah[18][71]
\end{Arabic}}
\flushleft{\begin{malayalam}
തുടര്‍ന്ന് അവര്‍ രണ്ട് പേരും കപ്പലില്‍ കയറിയപ്പോള്‍ അദ്ദേഹം അത് ഓട്ടയാക്കിക്കളഞ്ഞു. മൂസാ പറഞ്ഞു: അതിലുള്ളവരെ മുക്കിക്കളയുവാന്‍ വേണ്ടി താങ്കളത് ഓട്ടയാക്കിയിരിക്കുകയാണോ? തീര്‍ച്ചയായും ഗുരുതരമായ ഒരു കാര്യം തന്നെയാണ് താങ്കള്‍ ചെയ്തത്‌.
\end{malayalam}}
\flushright{\begin{Arabic}
\quranayah[18][72]
\end{Arabic}}
\flushleft{\begin{malayalam}
അദ്ദേഹം പറഞ്ഞു: തീര്‍ച്ചയായും താങ്കള്‍ക്ക് എന്‍റെ കൂടെ ക്ഷമിച്ചുകഴിയാന്‍ സാധിക്കില്ല എന്ന് ഞാന്‍ പറഞ്ഞിട്ടില്ലേ?
\end{malayalam}}
\flushright{\begin{Arabic}
\quranayah[18][73]
\end{Arabic}}
\flushleft{\begin{malayalam}
അദ്ദേഹം പറഞ്ഞു: ഞാന്‍ മറന്നുപോയതിന് താങ്കള്‍ എന്‍റെ പേരില്‍ നടപടി എടുക്കരുത്‌. എന്‍റെ കാര്യത്തില്‍ വിഷമകരമായ യാതൊന്നിനും താങ്കള്‍ എന്നെ നിര്‍ബന്ധിക്കുകയും ചെയ്യരുത്‌.
\end{malayalam}}
\flushright{\begin{Arabic}
\quranayah[18][74]
\end{Arabic}}
\flushleft{\begin{malayalam}
അനന്തരം അവര്‍ ഇരുവരും പോയി. അങ്ങനെ ഒരു ബാലനെ അവര്‍ കണ്ടുമുട്ടിയപ്പോള്‍ അദ്ദേഹം അവനെ കൊന്നുകളഞ്ഞു. മൂസാ പറഞ്ഞു: നിര്‍ദോഷിയായ ഒരാളെ മറ്റൊരാള്‍ക്കു പകരമായിട്ടല്ലാതെ താങ്കള്‍ കൊന്നുവോ? തീര്‍ച്ചയായും നിഷിദ്ധമായ ഒരു കാര്യം തന്നെയാണ് താങ്കള്‍ ചെയ്തിട്ടുള്ളത്‌.
\end{malayalam}}
\flushright{\begin{Arabic}
\quranayah[18][75]
\end{Arabic}}
\flushleft{\begin{malayalam}
അദ്ദേഹം പറഞ്ഞു: തീര്‍ച്ചയായും താങ്കള്‍ക്കു എന്‍റെ കൂടെ ക്ഷമിച്ച് കഴിയുവാന്‍ സാധിക്കുകയേ ഇല്ല എന്ന് ഞാന്‍ താങ്കളോട് പറഞ്ഞിട്ടില്ലേ?
\end{malayalam}}
\flushright{\begin{Arabic}
\quranayah[18][76]
\end{Arabic}}
\flushleft{\begin{malayalam}
മൂസാ പറഞ്ഞു: ഇതിന് ശേഷം വല്ലതിനെപ്പറ്റിയും ഞാന്‍ താങ്കളോട് ചോദിക്കുകയാണെങ്കില്‍ പിന്നെ താങ്കള്‍ എന്നെ സഹവാസിയാക്കേണ്ടതില്ല. എന്നില്‍ നിന്ന് താങ്കള്‍ക്ക് ന്യായമായ കാരണം കിട്ടിക്കഴിഞ്ഞു.
\end{malayalam}}
\flushright{\begin{Arabic}
\quranayah[18][77]
\end{Arabic}}
\flushleft{\begin{malayalam}
അനന്തരം അവര്‍ ഇരുവരും പോയി. അങ്ങനെ അവര്‍ ഇരുവരും ഒരു രാജ്യക്കാരുടെ അടുക്കല്‍ ചെന്നപ്പോള്‍ ആ രാജ്യക്കാരോട് അവര്‍ ഭക്ഷണം ആവശ്യപ്പെട്ടു. എന്നാല്‍ ഇവരെ സല്‍ക്കരിക്കുവാന്‍ അവര്‍ വൈമനസ്യം കാണിക്കുകയാണ് ചെയ്തത്‌. അപ്പോള്‍ പൊളിഞ്ഞുവീഴാനൊരുങ്ങുന്ന ഒരു മതില്‍ അവര്‍ അവിടെ കണ്ടെത്തി. ഉടനെ അദ്ദേഹം അത് നേരെയാക്കി. മൂസാ പറഞ്ഞു: താങ്കള്‍ ഉദ്ദേശിച്ചിരുന്നെങ്കില്‍ അതിന്‍റെ പേരില്‍ താങ്കള്‍ക്ക് വല്ല പ്രതിഫലവും വാങ്ങാമായിരുന്നു.
\end{malayalam}}
\flushright{\begin{Arabic}
\quranayah[18][78]
\end{Arabic}}
\flushleft{\begin{malayalam}
അദ്ദേഹം പറഞ്ഞു: ഇത് ഞാനും താങ്കളും തമ്മിലുള്ള വേര്‍പാടാകുന്നു. ഏതൊരു കാര്യത്തിന്‍റെ പേരില്‍ താങ്കള്‍ക്ക് ക്ഷമിക്കാന്‍ കഴിയാതിരുന്നുവോ അതിന്‍റെ പൊരുള്‍ ഞാന്‍ താങ്കള്‍ക്ക് അറിയിച്ച് തരാം.
\end{malayalam}}
\flushright{\begin{Arabic}
\quranayah[18][79]
\end{Arabic}}
\flushleft{\begin{malayalam}
എന്നാല്‍ ആ കപ്പല്‍ കടലില്‍ ജോലിചെയ്യുന്ന ഏതാനും ദരിദ്രന്‍മാരുടെതായിരുന്നു. അതിനാല്‍ ഞാനത് കേടുവരുത്തണമെന്ന് ഉദ്ദേശിച്ചു. (കാരണം) അവരുടെ പുറകെ എല്ലാ (നല്ല) കപ്പലും ബലാല്‍ക്കാരമായി പിടിച്ചെടുക്കുന്ന ഒരു രാജാവുണ്ടായിരുന്നു.
\end{malayalam}}
\flushright{\begin{Arabic}
\quranayah[18][80]
\end{Arabic}}
\flushleft{\begin{malayalam}
എന്നാല്‍ ആ ബാലനാകട്ടെ അവന്‍റെ മാതാപിതാക്കള്‍ സത്യവിശ്വാസികളായിരുന്നു. എന്നാല്‍ അവന്‍ അവരെ അതിക്രമത്തിനും അവിശ്വാസത്തിനും നിര്‍ബന്ധിതരാക്കിത്തീര്‍ക്കുമെന്ന് നാം ഭയപ്പെട്ടു.
\end{malayalam}}
\flushright{\begin{Arabic}
\quranayah[18][81]
\end{Arabic}}
\flushleft{\begin{malayalam}
അതിനാല്‍ അവര്‍ക്ക് അവരുടെ രക്ഷിതാവ് അവനെക്കാള്‍ സ്വഭാവശുദ്ധിയില്‍ മെച്ചപ്പെട്ടവനും, കാരുണ്യത്താല്‍ കൂടുതല്‍ അടുപ്പമുള്ളവനുമായ ഒരു സന്താനത്തെ പകരം നല്‍കണം എന്നു നാം ആഗ്രഹിച്ചു.
\end{malayalam}}
\flushright{\begin{Arabic}
\quranayah[18][82]
\end{Arabic}}
\flushleft{\begin{malayalam}
ആ മതിലാണെങ്കിലോ, അത് ആ പട്ടണത്തിലെ അനാഥരായ രണ്ട് ബാലന്‍മാരുടെതായിരുന്നു. അതിനു ചുവട്ടില്‍ അവര്‍ക്കായുള്ള ഒരു നിധിയുണ്ടായിരുന്നു. അവരുടെ പിതാവ് ഒരു നല്ല മനുഷ്യനായിരുന്നു. അതിനാല്‍ അവര്‍ ഇരുവരും യൌവ്വനം പ്രാപിക്കുകയും, എന്നിട്ടവരുടെ നിധി പുറത്തെടുക്കുകയും ചെയ്യണമെന്ന് താങ്കളുടെ രക്ഷിതാവ് ഉദ്ദേശിച്ചു താങ്കളുടെ രക്ഷിതാവിന്‍റെ കാരുണ്യം എന്ന നിലയിലത്രെ അത്‌. അതൊന്നും എന്‍റെ അഭിപ്രയപ്രകാരമല്ല ഞാന്‍ ചെയ്തത്‌. താങ്കള്‍ക്ക് ഏത് കാര്യത്തില്‍ ക്ഷമിക്കാന്‍ കഴിയാതിരുന്നുവോ അതിന്‍റെ പൊരുളാകുന്നു അത്‌.
\end{malayalam}}
\flushright{\begin{Arabic}
\quranayah[18][83]
\end{Arabic}}
\flushleft{\begin{malayalam}
അവര്‍ നിന്നോട് ദുല്‍ഖര്‍നൈനിയെപ്പറ്റി ചോദിക്കുന്നു. നീ പറയുക: അദ്ദേഹത്തെപ്പറ്റിയുള്ള വിവരം ഞാന്‍ നിങ്ങള്‍ക്ക് ഓതികേള്‍പിച്ച് തരാം.
\end{malayalam}}
\flushright{\begin{Arabic}
\quranayah[18][84]
\end{Arabic}}
\flushleft{\begin{malayalam}
തീര്‍ച്ചയായും നാം അദ്ദേഹത്തിന് ഭൂമിയില്‍ സ്വാധീനം നല്‍കുകയും, എല്ലാകാര്യത്തിനുമുള്ള മാര്‍ഗം നാം അദ്ദേഹത്തിന് സൌകര്യപ്പെടുത്തികൊടുക്കുകയും ചെയ്തു.
\end{malayalam}}
\flushright{\begin{Arabic}
\quranayah[18][85]
\end{Arabic}}
\flushleft{\begin{malayalam}
അങ്ങനെ അദ്ദേഹം ഒരു മാര്‍ഗം പിന്തുടര്‍ന്നു.
\end{malayalam}}
\flushright{\begin{Arabic}
\quranayah[18][86]
\end{Arabic}}
\flushleft{\begin{malayalam}
അങ്ങനെ അദ്ദേഹം സൂര്യാസ്തമനസ്ഥാനത്തെത്തിയപ്പോള്‍ അത് ചെളിവെള്ളമുള്ള ഒരു ജലാശയത്തില്‍ മറഞ്ഞ് പോകുന്നതായി അദ്ദേഹം കണ്ടു. അതിന്‍റെ അടുത്ത് ഒരു ജനവിഭാഗത്തെയും അദ്ദേഹം കണ്ടെത്തി.(അദ്ദേഹത്തോട്‌) നാം പറഞ്ഞു: ഹേ, ദുല്‍ഖര്‍നൈന്‍, ഒന്നുകില്‍ നിനക്ക് ഇവരെ ശിക്ഷിക്കാം. അല്ലെങ്കില്‍ നിനക്ക് അവരില്‍ നന്‍മയുണ്ടാക്കാം.
\end{malayalam}}
\flushright{\begin{Arabic}
\quranayah[18][87]
\end{Arabic}}
\flushleft{\begin{malayalam}
അദ്ദേഹം (ദുല്‍ഖര്‍നൈന്‍) പറഞ്ഞു: എന്നാല്‍ ആര്‍ അക്രമം പ്രവര്‍ത്തിച്ചുവോ അവനെ നാം ശിക്ഷിക്കുന്നതാണ്‌. പിന്നീട് അവന്‍ തന്‍റെ രക്ഷിതാവിങ്കലേക്ക് മടക്കപ്പെടുകയും അപ്പോള്‍ അവന്‍ ഗുരുതരമായ ശിക്ഷ അവന്ന് നല്‍കുകയും ചെയ്യുന്നതാണ്‌.
\end{malayalam}}
\flushright{\begin{Arabic}
\quranayah[18][88]
\end{Arabic}}
\flushleft{\begin{malayalam}
എന്നാല്‍ ആര്‍ വിശ്വസിക്കുകയും സല്‍കര്‍മ്മം പ്രവര്‍ത്തിക്കുകയും ചെയ്തുവോ അവന്നാണ് പ്രതിഫലമായി അതിവിശിഷ്ടമായ സ്വര്‍ഗമുള്ളത്‌. അവനോട് നാം നിര്‍ദേശിക്കുന്നത് നമ്മുടെ കല്‍പനയില്‍ നിന്ന് എളുപ്പമുള്ളതായി രിക്കുകയും ചെയ്യും.
\end{malayalam}}
\flushright{\begin{Arabic}
\quranayah[18][89]
\end{Arabic}}
\flushleft{\begin{malayalam}
പിന്നെ അദ്ദേഹം മറ്റൊരു മാര്‍ഗം പിന്തുടര്‍ന്നു.
\end{malayalam}}
\flushright{\begin{Arabic}
\quranayah[18][90]
\end{Arabic}}
\flushleft{\begin{malayalam}
അങ്ങനെ അദ്ദേഹം സൂര്യോദയസ്ഥാനത്തെത്തിയപ്പോള്‍ അത് ഒരു ജനതയുടെ മേല്‍ ഉദിച്ചുയരുന്നതായി അദ്ദേഹം കണ്ടെത്തി. അതിന്‍റെ (സൂര്യന്‍റെ) മുമ്പില്‍ അവര്‍ക്കു നാം യാതൊരു മറയും ഉണ്ടാക്കികൊടുത്തിട്ടില്ല.
\end{malayalam}}
\flushright{\begin{Arabic}
\quranayah[18][91]
\end{Arabic}}
\flushleft{\begin{malayalam}
അപ്രകാരം തന്നെ (അദ്ദേഹം പ്രവര്‍ത്തിച്ചു) അദ്ദേഹത്തിന്‍റെ പക്കലുള്ളതിനെപ്പറ്റി (നമ്മുടെ) സൂക്ഷ്മജ്ഞാനം കൊണ്ട് നാം പൂര്‍ണ്ണമായി അറിഞ്ഞിട്ടുണ്ട് താനും.
\end{malayalam}}
\flushright{\begin{Arabic}
\quranayah[18][92]
\end{Arabic}}
\flushleft{\begin{malayalam}
പിന്നെ അദ്ദേഹം മറ്റൊരു മാര്‍ഗം പിന്തുടര്‍ന്നു.
\end{malayalam}}
\flushright{\begin{Arabic}
\quranayah[18][93]
\end{Arabic}}
\flushleft{\begin{malayalam}
അങ്ങനെ അദ്ദേഹം രണ്ട് പര്‍വ്വതനിരകള്‍ക്കിടയിലെത്തിയപ്പോള്‍ അവയുടെ ഇപ്പുറത്തുണ്ടായിരുന്ന ഒരു ജനതയെ അദ്ദേഹം കാണുകയുണ്ടായി. പറയുന്നതൊന്നും മിക്കവാറും അവര്‍ക്ക് മനസ്സിലാക്കാനാവുന്നില്ല.
\end{malayalam}}
\flushright{\begin{Arabic}
\quranayah[18][94]
\end{Arabic}}
\flushleft{\begin{malayalam}
അവര്‍ പറഞ്ഞു: ഹേ, ദുല്‍ഖര്‍നൈന്‍, തീര്‍ച്ചയായും യഅ്ജൂജ് - മഅ്ജൂജ് വിഭാഗങ്ങള്‍ ഭൂമിയില്‍ കുഴപ്പമുണ്ടാക്കുന്നവരാകുന്നു. ഞങ്ങള്‍ക്കും അവര്‍ക്കുമിടയില്‍ താങ്കള്‍ ഒരു മതില്‍കെട്ട് ഉണ്ടാക്കിത്തരണമെന്ന വ്യവസ്ഥയില്‍ ഞങ്ങള്‍ താങ്കള്‍ക്ക് ഒരു കരം നിശ്ചയിച്ച് തരട്ടെയോ?
\end{malayalam}}
\flushright{\begin{Arabic}
\quranayah[18][95]
\end{Arabic}}
\flushleft{\begin{malayalam}
അദ്ദേഹം പറഞ്ഞു: എന്‍റെ രക്ഷിതാവ് എനിക്ക് അധീനപ്പെടുത്തിത്തന്നിട്ടുള്ളത് (അധികാരവും, ഐശ്വര്യവും) (നിങ്ങള്‍ നല്‍കുന്നതിനെക്കാളും) ഉത്തമമത്രെ. എന്നാല്‍ (നിങ്ങളുടെ ശാരീരിക) ശക്തികൊണ്ട് നിങ്ങളെന്നെ സഹായിക്കുവിന്‍. നിങ്ങള്‍ക്കും അവര്‍ക്കുമിടയില്‍ ഞാന്‍ ബലവത്തായ ഒരു മതിലുണ്ടാക്കിത്തരാം.
\end{malayalam}}
\flushright{\begin{Arabic}
\quranayah[18][96]
\end{Arabic}}
\flushleft{\begin{malayalam}
നിങ്ങള്‍ എനിക്ക് ഇരുമ്പുകട്ടികള്‍ കൊണ്ട് വന്ന് തരൂ. അങ്ങനെ ആ രണ്ട് പര്‍വ്വതപാര്‍ശ്വങ്ങളുടെ ഇട സമമാക്കിത്തീര്‍ത്തിട്ട് അദ്ദേഹം പറഞ്ഞു: നിങ്ങള്‍ കാറ്റൂതുക. അങ്ങനെ അത് (പഴുപ്പിച്ച്‌) തീ പോലെയാക്കിയപ്പോള്‍ അദ്ദേഹം പറഞ്ഞു: നിങ്ങളെനിക്ക് ഉരുക്കിയ ചെമ്പ് കൊണ്ട് വന്നു തരൂ ഞാനത് അതിന്‍മേല്‍ ഒഴിക്കട്ടെ.
\end{malayalam}}
\flushright{\begin{Arabic}
\quranayah[18][97]
\end{Arabic}}
\flushleft{\begin{malayalam}
പിന്നെ, ആ മതില്‍ക്കെട്ട് കയറിമറിയുവാന്‍ അവര്‍ക്ക് (യഅ്ജൂജ് - മഅ്ജൂജിന്ന്‌) സാധിച്ചില്ല. അതിന്ന് തുളയുണ്ടാക്കുവാനും അവര്‍ക്ക് സാധിച്ചില്ല.
\end{malayalam}}
\flushright{\begin{Arabic}
\quranayah[18][98]
\end{Arabic}}
\flushleft{\begin{malayalam}
അദ്ദേഹം (ദുല്‍ഖര്‍നൈന്‍) പറഞ്ഞു: ഇത് എന്‍റെ രക്ഷിതാവിങ്കല്‍ നിന്നുള്ള കാരുണ്യമത്രെ. എന്നാല്‍ എന്‍റെ രക്ഷിതാവിന്‍റെ വാഗ്ദത്ത സമയം വന്നാല്‍ അവന്‍ അതിനെ തകര്‍ത്ത് നിരപ്പാക്കിക്കളയുന്നതാണ്‌. എന്‍റെ രക്ഷിതാവിന്‍റെ വാഗ്ദാനം യാഥാര്‍ത്ഥ്യമാകുന്നു.
\end{malayalam}}
\flushright{\begin{Arabic}
\quranayah[18][99]
\end{Arabic}}
\flushleft{\begin{malayalam}
അന്ന്‌) അവരില്‍ ചിലര്‍ മറ്റുചിലരുടെ മേല്‍ തിരമാലകള്‍ പോലെ തള്ളിക്കയറുന്ന രൂപത്തില്‍ നാം വിട്ടേക്കുന്നതാണ്‌. കാഹളത്തില്‍ ഊതപ്പെടുകയും അപ്പോള്‍ നാം അവരെ ഒന്നിച്ച് ഒരുമിച്ചുകൂട്ടുകയും ചെയ്യും.
\end{malayalam}}
\flushright{\begin{Arabic}
\quranayah[18][100]
\end{Arabic}}
\flushleft{\begin{malayalam}
അവിശ്വാസികള്‍ക്ക് അന്നേ ദിവസം നാം നരകത്തെ ശരിയാംവണ്ണം കാണിച്ചുകൊടുക്കുന്നതാണ്‌.
\end{malayalam}}
\flushright{\begin{Arabic}
\quranayah[18][101]
\end{Arabic}}
\flushleft{\begin{malayalam}
എന്‍റെ സന്ദേശത്തിന്‍റെ മുമ്പില്‍ ആരുടെ കണ്ണുകള്‍ക്ക് മൂടിവീണ് പോകുകയും അതുകേട്ട് ഗ്രഹിക്കാന്‍ ആര്‍ക്ക് സാധിക്കാതാവുകയും ചെയ്തിരുന്നുവോ അവരത്രെ(ആ അവിശ്വാസികള്‍) .
\end{malayalam}}
\flushright{\begin{Arabic}
\quranayah[18][102]
\end{Arabic}}
\flushleft{\begin{malayalam}
എനിക്ക് പുറമെ എന്‍റെ ദാസന്‍മാരെ രക്ഷാകര്‍ത്താക്കളായി സ്വീകരിക്കാമെന്ന് അവിശ്വാസികള്‍ വിചാരിച്ചിരിക്കുകയാണോ? തീര്‍ച്ചയായും അവിശ്വാസികള്‍ക്ക് സല്‍ക്കാരം നല്‍കുവാനായി നാം നരകത്തെ ഒരുക്കിവെച്ചിരിക്കുന്നു.
\end{malayalam}}
\flushright{\begin{Arabic}
\quranayah[18][103]
\end{Arabic}}
\flushleft{\begin{malayalam}
(നബിയേ,) പറയുക: കര്‍മ്മങ്ങള്‍ ഏറ്റവും നഷ്ടകരമായി തീര്‍ന്നവരെ സംബന്ധിച്ച് നാം നിങ്ങള്‍ക്ക് പറഞ്ഞുതരട്ടെയോ?
\end{malayalam}}
\flushright{\begin{Arabic}
\quranayah[18][104]
\end{Arabic}}
\flushleft{\begin{malayalam}
ഐഹികജീവിതത്തിലെ തങ്ങളുടെ പ്രയത്നം പിഴച്ചുപോയവരത്രെ അവര്‍. അവര്‍ വിചാരിക്കുന്നതാകട്ടെ തങ്ങള്‍ നല്ല പ്രവര്‍ത്തനം നടത്തിക്കൊണ്ടിരിക്കുന്നു എന്നാണ്‌.
\end{malayalam}}
\flushright{\begin{Arabic}
\quranayah[18][105]
\end{Arabic}}
\flushleft{\begin{malayalam}
തങ്ങളുടെ രക്ഷിതാവിന്‍റെ ദൃഷ്ടാന്തങ്ങളിലും അവനുമായി കണ്ടുമുട്ടുന്നതിലും വിശ്വസിക്കാത്തവരത്രെ അവര്‍. അതിനാല്‍ അവരുടെ കര്‍മ്മങ്ങള്‍ നിഷ്ഫലമായിപ്പോയിരിക്കുന്നു. അതിനാല്‍ നാം അവര്‍ക്ക് ഉയിര്‍ത്തെഴുന്നേല്‍പിന്‍റെ നാളില്‍ യാതൊരു തൂക്കവും (സ്ഥാനവും) നിലനിര്‍ത്തുകയില്ല.
\end{malayalam}}
\flushright{\begin{Arabic}
\quranayah[18][106]
\end{Arabic}}
\flushleft{\begin{malayalam}
അതത്രെ അവര്‍ക്കുള്ള പ്രതിഫലം. അവിശ്വസിക്കുകയും, എന്‍റെ ദൃഷ്ടാന്തങ്ങളെയും, ദൂതന്‍മാരെയും പരിഹാസ്യമാക്കുകയും ചെയ്തതിന്നുള്ള (ശിക്ഷയായ) നരകം.
\end{malayalam}}
\flushright{\begin{Arabic}
\quranayah[18][107]
\end{Arabic}}
\flushleft{\begin{malayalam}
തീര്‍ച്ചയായും വിശ്വസിക്കുകയും സല്‍കര്‍മ്മങ്ങള്‍ പ്രവര്‍ത്തിക്കുകയും ചെയ്തവരാരോ അവര്‍ക്ക് സല്‍ക്കാരം നല്‍കാനുള്ളതാകുന്നു സ്വര്‍ഗത്തോപ്പുകള്‍.
\end{malayalam}}
\flushright{\begin{Arabic}
\quranayah[18][108]
\end{Arabic}}
\flushleft{\begin{malayalam}
അവരതില്‍ നിത്യവാസികളായിരിക്കും. അതില്‍ നിന്ന് വിട്ട് മാറാന്‍ അവര്‍ ആഗ്രഹിക്കുകയില്ല.
\end{malayalam}}
\flushright{\begin{Arabic}
\quranayah[18][109]
\end{Arabic}}
\flushleft{\begin{malayalam}
(നബിയേ,) പറയുക: സമുദ്രജലം എന്‍റെ രക്ഷിതാവിന്‍റെ വചനങ്ങളെഴുതാനുള്ള മഷിയായിരുന്നെങ്കില്‍ എന്‍റെ രക്ഷിതാവിന്‍റെ വചനങ്ങള്‍ തീരുന്നതിന് മുമ്പായി സമുദ്രജലം തീര്‍ന്ന് പോകുക തന്നെ ചെയ്യുമായിരുന്നു. അതിന് തുല്യമായ മറ്റൊരു സമുദ്രം കൂടി നാം സഹായത്തിനു കൊണ്ട് വന്നാലും ശരി.
\end{malayalam}}
\flushright{\begin{Arabic}
\quranayah[18][110]
\end{Arabic}}
\flushleft{\begin{malayalam}
(നബിയേ,) പറയുക: ഞാന്‍ നിങ്ങളെപ്പോലെയുള്ള ഒരു മനുഷ്യന്‍ മാത്രമാകുന്നു. നിങ്ങളുടെ ദൈവം ഏകദൈവം മാത്രമാണെന്ന് എനിക്ക് ബോധനം നല്‍കപ്പെടുന്നു. അതിനാല്‍ വല്ലവനും തന്‍റെ രക്ഷിതാവുമായി കണ്ടുമുട്ടണമെന്ന് ആഗ്രഹിക്കുന്നുവെങ്കില്‍ അവന്‍ സല്‍കര്‍മ്മം പ്രവര്‍ത്തിക്കുകയും, തന്‍റെ രക്ഷിതാവിനുള്ള ആരാധനയില്‍ യാതൊന്നിനെയും പങ്കുചേര്‍ക്കാതിരിക്കുകയും ചെയ്തുകൊള്ളട്ടെ.
\end{malayalam}}
\chapter{\textmalayalam{മര്‍യം}}
\begin{Arabic}
\Huge{\centerline{\basmalah}}\end{Arabic}
\flushright{\begin{Arabic}
\quranayah[19][1]
\end{Arabic}}
\flushleft{\begin{malayalam}
കാഫ്‌-ഹാ-യാ-ഐന്‍-സ്വാദ്‌.
\end{malayalam}}
\flushright{\begin{Arabic}
\quranayah[19][2]
\end{Arabic}}
\flushleft{\begin{malayalam}
നിന്‍റെ രക്ഷിതാവ് തന്‍റെ ദാസനായ സകരിയ്യായ്ക്ക് ചെയ്ത അനുഗ്രഹത്തെ സംബന്ധിച്ചുള്ള വിവരണമത്രെ ഇത്‌.
\end{malayalam}}
\flushright{\begin{Arabic}
\quranayah[19][3]
\end{Arabic}}
\flushleft{\begin{malayalam}
(അതായത്‌) അദ്ദേഹം തന്‍റെ രക്ഷിതാവിനെ പതുക്കെ വിളിച്ച് പ്രാര്‍ത്ഥിച്ച സന്ദര്‍ഭം.
\end{malayalam}}
\flushright{\begin{Arabic}
\quranayah[19][4]
\end{Arabic}}
\flushleft{\begin{malayalam}
അദ്ദേഹം പറഞ്ഞു: എന്‍റെ രക്ഷിതാവേ, എന്‍റെ എല്ലുകള്‍ ബലഹീനമായിക്കഴിഞ്ഞിരിക്കുന്നു. തലയാണെങ്കില്‍ നരച്ചു തിളങ്ങുന്നതായിരിക്കുന്നു. എന്‍റെ രക്ഷിതാവേ, നിന്നോട് പ്രാര്‍ത്ഥിച്ചിട്ട് ഞാന്‍ ഭാഗ്യം കെട്ടവനായിട്ടില്ല.
\end{malayalam}}
\flushright{\begin{Arabic}
\quranayah[19][5]
\end{Arabic}}
\flushleft{\begin{malayalam}
എനിക്ക് പുറകെ വരാനുള്ള ബന്ധുമിത്രാദികളെപ്പറ്റി എനിക്ക് ഭയമാകുന്നു. എന്‍റെ ഭാര്യയാണെങ്കില്‍ വന്ധ്യയുമാകുന്നു. അതിനാല്‍ നിന്‍റെ പക്കല്‍ നിന്ന് നീ എനിക്ക് ഒരു ബന്ധുവെ (അവകാശിയെ) നല്‍കേണമേ.
\end{malayalam}}
\flushright{\begin{Arabic}
\quranayah[19][6]
\end{Arabic}}
\flushleft{\begin{malayalam}
എനിക്ക് അവന്‍ അനന്തരാവകാശിയായിരിക്കും. യഅ്ഖൂബ് കുടുംബത്തിനും അവന്‍ അനന്തരാവകാശിയായിരിക്കും. എന്‍റെ രക്ഷിതാവേ, അവനെ നീ (ഏവര്‍ക്കും) തൃപ്തിപ്പെട്ടവനാക്കുകയും ചെയ്യേണമേ.
\end{malayalam}}
\flushright{\begin{Arabic}
\quranayah[19][7]
\end{Arabic}}
\flushleft{\begin{malayalam}
ഹേ, സകരിയ്യാ, തീര്‍ച്ചയായും നിനക്ക് നാം ഒരു ആണ്‍കുട്ടിയെപ്പറ്റി സന്തോഷവാര്‍ത്ത അറിയിക്കുന്നു. അവന്‍റെ പേര്‍ യഹ്‌യാ എന്നാകുന്നു. മുമ്പ് നാം ആരെയും അവന്‍റെ പേര് ഉള്ളവരാക്കിയിട്ടില്ല.
\end{malayalam}}
\flushright{\begin{Arabic}
\quranayah[19][8]
\end{Arabic}}
\flushleft{\begin{malayalam}
അദ്ദേഹം പറഞ്ഞു: എന്‍റെ രക്ഷിതാവേ, എനിക്കെങ്ങനെ ഒരു ആണ്‍കുട്ടിയുണ്ടാകും? എന്‍റെ ഭാര്യ വന്ധ്യയാകുന്നു. ഞാനാണെങ്കില്‍ വാര്‍ദ്ധക്യത്താല്‍ ചുക്കിച്ചുളിഞ്ഞ അവസ്ഥയിലെത്തിയിരിക്കുന്നു.
\end{malayalam}}
\flushright{\begin{Arabic}
\quranayah[19][9]
\end{Arabic}}
\flushleft{\begin{malayalam}
അവന്‍ (അല്ലാഹു) പറഞ്ഞു: അങ്ങനെ തന്നെ. മുമ്പ് നീ യാതൊന്നുമല്ലാതിരുന്നപ്പോള്‍ നിന്നെ ഞാന്‍ സൃഷ്ടിച്ചിരിക്കെ, ഇത് എന്നെ സംബന്ധിച്ചിടത്തോളം ഒരു നിസ്സാര കാര്യം മാത്രമാണ് എന്ന് നിന്‍റെ രക്ഷിതാവ് പ്രഖ്യാപിച്ചിരിക്കുന്നു.
\end{malayalam}}
\flushright{\begin{Arabic}
\quranayah[19][10]
\end{Arabic}}
\flushleft{\begin{malayalam}
അദ്ദേഹം (സകരിയ്യാ) പറഞ്ഞു: നീ എനിക്ക് ഒരു ദൃഷ്ടാന്തം ഏര്‍പെടുത്തിത്തരേണമേ. അവന്‍ (അല്ലാഹു) പറഞ്ഞു: നിനക്കുള്ള ദൃഷ്ടാന്തം വൈകല്യമൊന്നും ഇല്ലാത്തവനായിരിക്കെത്തന്നെ ജനങ്ങളോട് മൂന്ന് രാത്രി (ദിവസം) നീ സംസാരിക്കാതിരിക്കലാകുന്നു.
\end{malayalam}}
\flushright{\begin{Arabic}
\quranayah[19][11]
\end{Arabic}}
\flushleft{\begin{malayalam}
അങ്ങനെ അദ്ദേഹം പ്രാര്‍ത്ഥനാമണ്ഡപത്തില്‍ നിന്ന് തന്‍റെ ജനങ്ങളുടെ അടുക്കലേക്ക് പുറപ്പെട്ടു. എന്നിട്ട്‌, നിങ്ങള്‍ രാവിലെയും വൈകുന്നേരവും അല്ലാഹുവിന്‍റെ പരിശുദ്ധിയെ പ്രകീര്‍ത്തിക്കുക എന്ന് അവരോട് ആംഗ്യം കാണിച്ചു.
\end{malayalam}}
\flushright{\begin{Arabic}
\quranayah[19][12]
\end{Arabic}}
\flushleft{\begin{malayalam}
ഹേ, യഹ്‌യാ വേദഗ്രന്ഥം ബലമായി സ്വീകരിച്ച് കൊള്ളുക. (എന്ന് നാം പറഞ്ഞു:) കുട്ടിയായിരിക്കെത്തന്നെ അദ്ദേഹത്തിന് നാം ജ്ഞാനം നല്‍കുകയും ചെയ്തു.
\end{malayalam}}
\flushright{\begin{Arabic}
\quranayah[19][13]
\end{Arabic}}
\flushleft{\begin{malayalam}
നമ്മുടെ പക്കല്‍ നിന്നുള്ള അനുകമ്പയും പരിശുദ്ധിയും (നല്‍കി.) അദ്ദേഹം ധര്‍മ്മനിഷ്ഠയുള്ളവനുമായിരുന്നു.
\end{malayalam}}
\flushright{\begin{Arabic}
\quranayah[19][14]
\end{Arabic}}
\flushleft{\begin{malayalam}
തന്‍റെ മാതാപിതാക്കള്‍ക്ക് നന്‍മചെയ്യുന്നവനുമായിരുന്നു. നിഷ്ഠൂരനും അനുസരണം കെട്ടവനുമായിരുന്നില്ല.
\end{malayalam}}
\flushright{\begin{Arabic}
\quranayah[19][15]
\end{Arabic}}
\flushleft{\begin{malayalam}
അദ്ദേഹം ജനിച്ച ദിവസവും മരിക്കുന്ന ദിവസവും ജീവനോടെ എഴുന്നേല്‍പിക്കപ്പെടുന്ന ദിവസവും അദ്ദേഹത്തിന് സമാധാനം.
\end{malayalam}}
\flushright{\begin{Arabic}
\quranayah[19][16]
\end{Arabic}}
\flushleft{\begin{malayalam}
വേദഗ്രന്ഥത്തില്‍ മര്‍യമിനെപ്പറ്റിയുള്ള വിവരം നീ പറഞ്ഞുകൊടുക്കുക. അവള്‍ തന്‍റെ വീട്ടുകാരില്‍ നിന്നകന്ന് കിഴക്ക് ഭാഗത്തുള്ള ഒരു സ്ഥലത്തേക്ക് മാറിത്താമസിച്ച സന്ദര്‍ഭം.
\end{malayalam}}
\flushright{\begin{Arabic}
\quranayah[19][17]
\end{Arabic}}
\flushleft{\begin{malayalam}
എന്നിട്ട് അവര്‍ കാണാതിരിക്കാന്‍ അവള്‍ ഒരു മറയുണ്ടാക്കി. അപ്പോള്‍ നമ്മുടെ ആത്മാവിനെ (ജിബ്‌രീലിനെ) നാം അവളുടെ അടുത്തേക്ക് നിയോഗിച്ചു. അങ്ങനെ അദ്ദേഹം അവളുടെ മുമ്പില്‍ തികഞ്ഞ മനുഷ്യരൂപത്തില്‍ പ്രത്യക്ഷപ്പെട്ടു.
\end{malayalam}}
\flushright{\begin{Arabic}
\quranayah[19][18]
\end{Arabic}}
\flushleft{\begin{malayalam}
അവള്‍ പറഞ്ഞു: തീര്‍ച്ചയായും നിന്നില്‍ നിന്ന് ഞാന്‍ പരമകാരുണികനില്‍ ശരണം പ്രാപിക്കുന്നു. നീ ധര്‍മ്മനിഷ്ഠയുള്ളവനാണെങ്കില്‍ (എന്നെ വിട്ട് മാറിപ്പോകൂ.)
\end{malayalam}}
\flushright{\begin{Arabic}
\quranayah[19][19]
\end{Arabic}}
\flushleft{\begin{malayalam}
അദ്ദേഹം (ജിബ്‌രീല്‍) പറഞ്ഞു: പരിശുദ്ധനായ ഒരു ആണ്‍കുട്ടിയെ നിനക്ക് ദാനം ചെയ്യുന്നതിന് വേണ്ടി നിന്‍റെ രക്ഷിതാവ് അയച്ച ദൂതന്‍ മാത്രമാകുന്നു ഞാന്‍.
\end{malayalam}}
\flushright{\begin{Arabic}
\quranayah[19][20]
\end{Arabic}}
\flushleft{\begin{malayalam}
അവള്‍ പറഞ്ഞു: എനിക്കെങ്ങനെ ഒരു ആണ്‍കുട്ടിയുണ്ടാകും? യാതൊരു മനുഷ്യനും എന്നെ സ്പര്‍ശിച്ചിട്ടില്ല. ഞാന്‍ ഒരു ദുര്‍നടപടിക്കാരിയായിട്ടുമില്ല.
\end{malayalam}}
\flushright{\begin{Arabic}
\quranayah[19][21]
\end{Arabic}}
\flushleft{\begin{malayalam}
അദ്ദേഹം പറഞ്ഞു: (കാര്യം) അങ്ങനെതന്നെയാകുന്നു. അത് തന്നെ സംബന്ധിച്ചിടത്തോളം നിസ്സാരമായ ഒരു കാര്യമാണെന്ന് നിന്‍റെ രക്ഷിതാവ് പറഞ്ഞിരിക്കുന്നു. അവനെ (ആ കുട്ടിയെ) മനുഷ്യര്‍ക്കൊരു ദൃഷ്ടാന്തവും, നമ്മുടെ പക്കല്‍ നിന്നുള്ള കാരുണ്യവും ആക്കാനും (നാം ഉദ്ദേശിക്കുന്നു.) അത് തീരുമാനിക്കപ്പെട്ട ഒരു കാര്യമാകുന്നു.
\end{malayalam}}
\flushright{\begin{Arabic}
\quranayah[19][22]
\end{Arabic}}
\flushleft{\begin{malayalam}
അങ്ങനെ അവനെ ഗര്‍ഭം ധരിക്കുകയും, എന്നിട്ട് അതുമായി അവള്‍ അകലെ ഒരു സ്ഥലത്ത് മാറിത്താമസിക്കുകയും ചെയ്തു.
\end{malayalam}}
\flushright{\begin{Arabic}
\quranayah[19][23]
\end{Arabic}}
\flushleft{\begin{malayalam}
അങ്ങനെ പ്രസവവേദന അവളെ ഒരു ഈന്തപ്പന മരത്തിന്‍റെ അടുത്തേക്ക് കൊണ്ട് വന്നു. അവള്‍ പറഞ്ഞു: ഞാന്‍ ഇതിന് മുമ്പ് തന്നെ മരിക്കുകയും, പാടെ വിസ്മരിച്ച് തള്ളപ്പെട്ടവളാകുകയും ചെയ്തിരുന്നെങ്കില്‍ എത്ര നന്നായിരുന്നേനേ!
\end{malayalam}}
\flushright{\begin{Arabic}
\quranayah[19][24]
\end{Arabic}}
\flushleft{\begin{malayalam}
ഉടനെ അവളുടെ താഴ്ഭാഗത്ത് നിന്ന് (ഒരാള്‍) വിളിച്ചുപറഞ്ഞു: നീ വ്യസനിക്കേണ്ട, നിന്‍റെ താഴ്ഭാഗത്ത് ഒരു അരുവി ഉണ്ടാക്കി തന്നിരിക്കുന്നു.
\end{malayalam}}
\flushright{\begin{Arabic}
\quranayah[19][25]
\end{Arabic}}
\flushleft{\begin{malayalam}
നീ ഈന്തപ്പനമരം നിന്‍റെ അടുക്കലേക്ക് പിടിച്ചുകുലുക്കിക്കൊള്ളുക. അത് നിനക്ക് പാകമായ ഈന്തപ്പഴം വീഴ്ത്തിത്തരുന്നതാണ്‌.
\end{malayalam}}
\flushright{\begin{Arabic}
\quranayah[19][26]
\end{Arabic}}
\flushleft{\begin{malayalam}
അങ്ങനെ നീ തിന്നുകയും കുടിക്കുകയും കണ്ണുകുളിര്‍ത്തിരിക്കുകയും ചെയ്യുക. ഇനി നീ മനുഷ്യരില്‍ ആരെയെങ്കിലും കാണുന്ന പക്ഷം ഇപ്രകാരം പറഞ്ഞേക്കുക: പരമകാരുണികന്ന് വേണ്ടി ഞാന്‍ ഒരു വ്രതം നേര്‍ന്നിരിക്കയാണ് അതിനാല്‍ ഇന്നു ഞാന്‍ ഒരു മനുഷ്യനോടും സംസാരിക്കുകയില്ല തന്നെ.
\end{malayalam}}
\flushright{\begin{Arabic}
\quranayah[19][27]
\end{Arabic}}
\flushleft{\begin{malayalam}
അനന്തരം അവനെ (കുട്ടിയെ) യും വഹിച്ചുകൊണ്ട് അവള്‍ തന്‍റെ ആളുകളുടെ അടുത്ത് ചെന്നു. അവര്‍ പറഞ്ഞു: മര്‍യമേ, ആക്ഷേപകരമായ ഒരു കാര്യം തന്നെയാകുന്നു നീ ചെയ്തിരിക്കുന്നത്‌.
\end{malayalam}}
\flushright{\begin{Arabic}
\quranayah[19][28]
\end{Arabic}}
\flushleft{\begin{malayalam}
ഹേ; ഹാറൂന്‍റെ സഹോദരീ, നിന്‍റെ പിതാവ് ഒരു ചീത്ത മനുഷ്യനായിരുന്നില്ല. നിന്‍റെ മാതാവ് ഒരു ദുര്‍നടപടിക്കാരിയുമായിരുന്നില്ല.
\end{malayalam}}
\flushright{\begin{Arabic}
\quranayah[19][29]
\end{Arabic}}
\flushleft{\begin{malayalam}
അപ്പോള്‍ അവള്‍ അവന്‍റെ (കുട്ടിയുടെ) നേരെ ചൂണ്ടിക്കാണിച്ചു. അവര്‍ പറഞ്ഞു: തൊട്ടിലിലുള്ള ഒരു കുട്ടിയോട് ഞങ്ങള്‍ എങ്ങനെ സംസാരിക്കും?
\end{malayalam}}
\flushright{\begin{Arabic}
\quranayah[19][30]
\end{Arabic}}
\flushleft{\begin{malayalam}
അവന്‍ (കുട്ടി) പറഞ്ഞു: ഞാന്‍ അല്ലാഹുവിന്‍റെ ദാസനാകുന്നു. അവന്‍ എനിക്ക് വേദഗ്രന്ഥം നല്‍കുകയും എന്നെ അവന്‍ പ്രവാചകനാക്കുകയും ചെയ്തിരിക്കുന്നു.
\end{malayalam}}
\flushright{\begin{Arabic}
\quranayah[19][31]
\end{Arabic}}
\flushleft{\begin{malayalam}
ഞാന്‍ എവിടെയായിരുന്നാലും എന്നെ അവന്‍ അനുഗൃഹീതനാക്കിയിരിക്കുന്നു. ഞാന്‍ ജീവിച്ചിരിക്കുന്ന കാലമത്രയും നമസ്കരിക്കുവാനും സകാത്ത് നല്‍കുവാനും അവന്‍ എന്നോട് അനുശാസിക്കുകയും ചെയ്തിരിക്കുന്നു.
\end{malayalam}}
\flushright{\begin{Arabic}
\quranayah[19][32]
\end{Arabic}}
\flushleft{\begin{malayalam}
(അവന്‍ എന്നെ) എന്‍റെ മാതാവിനോട് നല്ല നിലയില്‍ പെരുമാറുന്നവനും (ആക്കിയിരിക്കുന്നു.) അവന്‍ എന്നെ നിഷ്ഠൂരനും ഭാഗ്യം കെട്ടവനുമാക്കിയിട്ടില്ല.
\end{malayalam}}
\flushright{\begin{Arabic}
\quranayah[19][33]
\end{Arabic}}
\flushleft{\begin{malayalam}
ഞാന്‍ ജനിച്ച ദിവസവും മരിക്കുന്ന ദിവസവും ജീവനോടെ എഴുന്നേല്‍പിക്കപ്പെടുന്ന ദിവസവും എന്‍റെ മേല്‍ ശാന്തി ഉണ്ടായിരിക്കും.
\end{malayalam}}
\flushright{\begin{Arabic}
\quranayah[19][34]
\end{Arabic}}
\flushleft{\begin{malayalam}
അതത്രെ മര്‍യമിന്‍റെ മകനായ ഈസാ അവര്‍ ഏതൊരു വിഷയത്തില്‍ തര്‍ക്കിച്ച് കൊണ്ടിരിക്കുന്നുവോ അതിനെപ്പറ്റിയുള്ള യഥാര്‍ത്ഥമായ വാക്കത്രെ ഇത്‌.
\end{malayalam}}
\flushright{\begin{Arabic}
\quranayah[19][35]
\end{Arabic}}
\flushleft{\begin{malayalam}
ഒരു സന്താനത്തെ സ്വീകരിക്കുക എന്നത് അല്ലാഹുവിന്നുണ്ടാകാവുന്നതല്ല. അവന്‍ എത്ര പരിശുദ്ധന്‍! അവന്‍ ഒരു കാര്യം തീരുമാനിച്ച് കഴിഞ്ഞാല്‍ അതിനോട് ഉണ്ടാകൂ എന്ന് പറയുക മാത്രംചെയ്യുന്നു. അപ്പോള്‍ അതുണ്ടാകുന്നു.
\end{malayalam}}
\flushright{\begin{Arabic}
\quranayah[19][36]
\end{Arabic}}
\flushleft{\begin{malayalam}
(ഈസാ പറഞ്ഞു:) തീര്‍ച്ചയായും അല്ലാഹു എന്‍റെയും നിങ്ങളുടെയും രക്ഷിതാവാകുന്നു. അതിനാല്‍ അവനെ നിങ്ങള്‍ ആരാധിക്കുക. ഇതത്രെ നേരെയുള്ള മാര്‍ഗം.
\end{malayalam}}
\flushright{\begin{Arabic}
\quranayah[19][37]
\end{Arabic}}
\flushleft{\begin{malayalam}
എന്നിട്ട് അവര്‍ക്കിടയില്‍ നിന്ന് കക്ഷികള്‍ ഭിന്നിച്ചുണ്ടായി. അപ്പോള്‍ അവിശ്വസിച്ചവര്‍ക്കത്രെ ഭയങ്കരമായ ഒരു ദിവസത്തിന്‍റെ സാന്നിദ്ധ്യത്താല്‍ വമ്പിച്ച നാശം.
\end{malayalam}}
\flushright{\begin{Arabic}
\quranayah[19][38]
\end{Arabic}}
\flushleft{\begin{malayalam}
അവര്‍ നമ്മുടെ അടുത്ത് വരുന്ന ദിവസം അവര്‍ക്ക് എന്തൊരു കേള്‍വിയും കാഴ്ചയുമായിരിക്കും! പക്ഷെ ഇന്ന് ആ അക്രമികള്‍ പ്രത്യക്ഷമായ വഴികേടിലാകുന്നു.
\end{malayalam}}
\flushright{\begin{Arabic}
\quranayah[19][39]
\end{Arabic}}
\flushleft{\begin{malayalam}
നഷ്ടബോധത്തിന്‍റെ ദിവസത്തെപ്പറ്റി അഥവാ കാര്യം (അന്തിമമായി) തീരുമാനിക്കപ്പെടുന്ന സന്ദര്‍ഭത്തെപ്പറ്റി നീ അവര്‍ക്ക് താക്കീത് നല്‍കുക. അവര്‍ അശ്രദ്ധയിലകപ്പെട്ടിരിക്കുകയാകുന്നു. അവര്‍ വിശ്വസിക്കുന്നില്ല.
\end{malayalam}}
\flushright{\begin{Arabic}
\quranayah[19][40]
\end{Arabic}}
\flushleft{\begin{malayalam}
തീര്‍ച്ചയായും നാം തന്നെയാണ് ഭൂമിയുടെയും അതിലുള്ളവയുടെയും അനന്തരാവകാശിയാകുന്നത്‌. നമ്മുടെ അടുക്കലേക്ക് തന്നെയായിരിക്കും അവര്‍ മടക്കപ്പെടുന്നത്‌.
\end{malayalam}}
\flushright{\begin{Arabic}
\quranayah[19][41]
\end{Arabic}}
\flushleft{\begin{malayalam}
വേദഗ്രന്ഥത്തില്‍ ഇബ്രാഹീമിനെപ്പറ്റിയുള്ള വിവരം നീ പറഞ്ഞുകൊടുക്കുക. തീര്‍ച്ചയായും അദ്ദേഹം സത്യവാനും പ്രവാചകനുമായിരുന്നു.
\end{malayalam}}
\flushright{\begin{Arabic}
\quranayah[19][42]
\end{Arabic}}
\flushleft{\begin{malayalam}
അദ്ദേഹം തന്‍റെ പിതാവിനോട് പറഞ്ഞ സന്ദര്‍ഭം (ശ്രദ്ധേയമാകുന്നു:) എന്‍റെ പിതാവേ, കേള്‍ക്കുകയോ, കാണുകയോ ചെയ്യാത്ത, താങ്കള്‍ക്ക് യാതൊരു ഉപകാരവും ചെയ്യാത്ത വസ്തുവെ താങ്കള്‍ എന്തിന് ആരാധിക്കുന്നു.?
\end{malayalam}}
\flushright{\begin{Arabic}
\quranayah[19][43]
\end{Arabic}}
\flushleft{\begin{malayalam}
എന്‍റെ പിതാവേ, തീര്‍ച്ചയായും താങ്കള്‍ക്ക് വന്നുകിട്ടിയിട്ടില്ലാത്ത അറിവ് എനിക്ക് വന്നുകിട്ടിയിട്ടുണ്ട്‌. ആകയാല്‍ താങ്കള്‍ എന്നെ പിന്തടരൂ. ഞാന്‍ താങ്കള്‍ക്ക് ശരിയായ മാര്‍ഗം കാണിച്ചുതരാം.
\end{malayalam}}
\flushright{\begin{Arabic}
\quranayah[19][44]
\end{Arabic}}
\flushleft{\begin{malayalam}
എന്‍റെ പിതാവേ, താങ്കള്‍ പിശാചിനെ ആരാധിക്കരുത്‌. തീര്‍ച്ചയായും പിശാച് പരമകാരുണികനോട് അനുസരണമില്ലാത്തവനാകുന്നു.
\end{malayalam}}
\flushright{\begin{Arabic}
\quranayah[19][45]
\end{Arabic}}
\flushleft{\begin{malayalam}
എന്‍റെ പിതാവേ, തീര്‍ച്ചയായും പരമകാരുണികനില്‍ നിന്നുള്ള വല്ല ശിക്ഷയും താങ്കളെ ബാധിക്കുമെന്ന് ഞാന്‍ ഭയപ്പെടുന്നു. അപ്പോള്‍ താങ്കള്‍ പിശാചിന്‍റെ മിത്രമായിരിക്കുന്നതാണ്‌.
\end{malayalam}}
\flushright{\begin{Arabic}
\quranayah[19][46]
\end{Arabic}}
\flushleft{\begin{malayalam}
അയാള്‍ പറഞ്ഞു: ഹേ; ഇബ്രാഹീം, നീ എന്‍റെ ദൈവങ്ങളെ വേണ്ടെന്ന് വെക്കുകയാണോ? നീ (ഇതില്‍ നിന്ന്‌) വിരമിക്കുന്നില്ലെങ്കില്‍ ഞാന്‍ നിന്നെ കല്ലെറിഞ്ഞോടിക്കുക തന്നെ ചെയ്യും. കുറെ കാലത്തേക്ക് നീ എന്നില്‍ നിന്ന് വിട്ടുമാറിക്കൊള്ളണം.
\end{malayalam}}
\flushright{\begin{Arabic}
\quranayah[19][47]
\end{Arabic}}
\flushleft{\begin{malayalam}
അദ്ദേഹം (ഇബ്രാഹീം) പറഞ്ഞു: താങ്കള്‍ക്ക് സലാം. താങ്കള്‍ക്ക് വേണ്ടി ഞാന്‍ എന്‍റെ രക്ഷിതാവിനോട് പാപമോചനം തേടാം. തീര്‍ച്ചയായും അവനെന്നോട് ദയയുള്ളവനാകുന്നു.
\end{malayalam}}
\flushright{\begin{Arabic}
\quranayah[19][48]
\end{Arabic}}
\flushleft{\begin{malayalam}
നിങ്ങളെയും അല്ലാഹുവിന് പുറമെ നിങ്ങള്‍ പ്രാര്‍ത്ഥിച്ചുവരുന്നവയെയും ഞാന്‍ വെടിയുന്നു. എന്‍റെ രക്ഷിതാവിനോട് ഞാന്‍ പ്രാര്‍ത്ഥിക്കുന്നു. എന്‍റെ രക്ഷിതാവിനോട് പ്രാര്‍ത്ഥിക്കുന്നത് മൂലം ഞാന്‍ ഭാഗ്യം കെട്ടവനാകാതിരുന്നേക്കാം.
\end{malayalam}}
\flushright{\begin{Arabic}
\quranayah[19][49]
\end{Arabic}}
\flushleft{\begin{malayalam}
അങ്ങനെ അവരെയും അല്ലാഹുവിന് പുറമെ അവര്‍ ആരാധിക്കുന്നവയെയും വെടിഞ്ഞ് അദ്ദേഹം പോയപ്പോള്‍ അദ്ദേഹത്തിന് നാം ഇഷാഖിനെയും (മകന്‍) യഅ്ഖൂബിനെയും (പൌത്രന്‍) നല്‍കി. അവരെയൊക്കെ നാം പ്രവാചകന്‍മാരാക്കുകയും ചെയ്തു.
\end{malayalam}}
\flushright{\begin{Arabic}
\quranayah[19][50]
\end{Arabic}}
\flushleft{\begin{malayalam}
നമ്മുടെ കാരുണ്യത്തില്‍ നിന്നും അവര്‍ക്ക് നാം നല്‍കുകയും, അവര്‍ക്ക് നാം ഉന്നതമായ സല്‍കീര്‍ത്തി ഉണ്ടാക്കുകയും ചെയ്തു.
\end{malayalam}}
\flushright{\begin{Arabic}
\quranayah[19][51]
\end{Arabic}}
\flushleft{\begin{malayalam}
വേദഗ്രന്ഥത്തില്‍ മൂസായെപ്പറ്റിയുള്ള വിവരവും നീ പറഞ്ഞുകൊടുക്കുക. തീര്‍ച്ചയായും അദ്ദേഹം നിഷ്കളങ്കനായിരുന്നു. അദ്ദേഹം ദൂതനും പ്രവാചകനുമായിരുന്നു.
\end{malayalam}}
\flushright{\begin{Arabic}
\quranayah[19][52]
\end{Arabic}}
\flushleft{\begin{malayalam}
പര്‍വ്വതത്തിന്‍റെ വലതുഭാഗത്ത് നിന്ന് നാം അദ്ദേഹത്തെ വിളിക്കുകയും, രഹസ്യസംഭാഷണത്തിനായി നാം അദ്ദേഹത്തിന് സാമീപ്യം നല്‍കുകയും ചെയ്തു.
\end{malayalam}}
\flushright{\begin{Arabic}
\quranayah[19][53]
\end{Arabic}}
\flushleft{\begin{malayalam}
നമ്മുടെ കാരുണ്യത്താല്‍ തന്‍റെ സഹോദരനായ ഹാറൂനിനെ ഒരു പ്രവാചകനായി, അദ്ദേഹത്തിന് (സഹായത്തിനായി) നാം നല്‍കുകയും ചെയ്തു.
\end{malayalam}}
\flushright{\begin{Arabic}
\quranayah[19][54]
\end{Arabic}}
\flushleft{\begin{malayalam}
വേദഗ്രന്ഥത്തില്‍ ഇസ്മാഈലിനെപ്പറ്റിയുള്ള വിവരം നീ പറഞ്ഞുകൊടുക്കുക. തീര്‍ച്ചയായും അദ്ദേഹം വാഗ്ദാനം പാലിക്കുന്നവനായിരുന്നു. അദ്ദേഹം ദൂതനും പ്രവാചകനുമായിരുന്നു.
\end{malayalam}}
\flushright{\begin{Arabic}
\quranayah[19][55]
\end{Arabic}}
\flushleft{\begin{malayalam}
തന്‍റെ ആളുകളോട് നമസ്കരിക്കുവാനും സകാത്ത് നല്‍കുവാനും അദ്ദേഹം കല്‍പിക്കുമായിരുന്നു. തന്‍റെ രക്ഷിതാവിന്‍റെ അടുക്കല്‍ അദ്ദേഹം പ്രീതി ലഭിച്ചവനായിരുന്നു.
\end{malayalam}}
\flushright{\begin{Arabic}
\quranayah[19][56]
\end{Arabic}}
\flushleft{\begin{malayalam}
വേദഗ്രന്ഥത്തില്‍ ഇദ്‌രീസിനെപ്പറ്റിയുള്ള വിവരം നീ പറഞ്ഞുകൊടുക്കുക. തീര്‍ച്ചയായും അദ്ദേഹം സത്യവാനും പ്രവാചകനുമായിരുന്നു.
\end{malayalam}}
\flushright{\begin{Arabic}
\quranayah[19][57]
\end{Arabic}}
\flushleft{\begin{malayalam}
അദ്ദേഹത്തെ നാം ഉന്നതമായ സ്ഥാനത്തേക്ക് ഉയര്‍ത്തുകയും ചെയ്തിരിക്കുന്നു.
\end{malayalam}}
\flushright{\begin{Arabic}
\quranayah[19][58]
\end{Arabic}}
\flushleft{\begin{malayalam}
അല്ലാഹു അനുഗ്രഹം നല്‍കിയിട്ടുള്ള പ്രവാചകന്‍മാരത്രെ അവര്‍. ആദമിന്‍റെ സന്തതികളില്‍ പെട്ടവരും, നൂഹിനോടൊപ്പെം നാം കപ്പലില്‍ കയറ്റിയവരില്‍പെട്ടവരും ഇബ്രാഹീമിന്‍റെയും ഇസ്രായീലിന്‍റെയും സന്തതികളില്‍ പെട്ടവരും, നാം നേര്‍വഴിയിലാക്കുകയും പ്രത്യേകം തെരഞ്ഞെടുക്കുകയും ചെയ്തവരില്‍ പെട്ടവരുമത്രെ അവര്‍. പരമകാരുണികന്‍റെ തെളിവുകള്‍ അവര്‍ക്ക് വായിച്ചുകേള്‍പിക്കപ്പെടുന്ന പക്ഷം പ്രണമിക്കുന്നവരും കരയുന്നവരുമായി ക്കൊണ്ട് അവര്‍ താഴെ വീഴുന്നതാണ്‌.
\end{malayalam}}
\flushright{\begin{Arabic}
\quranayah[19][59]
\end{Arabic}}
\flushleft{\begin{malayalam}
എന്നിട്ട് അവര്‍ക്ക് ശേഷം അവരുടെ സ്ഥാനത്ത് ഒരു പിന്‍തലമുറ വന്നു. അവര്‍ നമസ്കാരം പാഴാക്കുകയും തന്നിഷ്ടങ്ങളെ പിന്തുടരുകയും ചെയ്തു. തന്‍മൂലം ദുര്‍മാര്‍ഗത്തിന്‍റെ ഫലം അവര്‍ കണ്ടെത്തുന്നതാണ്‌.
\end{malayalam}}
\flushright{\begin{Arabic}
\quranayah[19][60]
\end{Arabic}}
\flushleft{\begin{malayalam}
എന്നാല്‍ പശ്ചാത്തപിക്കുകയും, വിശ്വസിക്കുകയും സല്‍കര്‍മ്മം പ്രവര്‍ത്തിക്കുകയും ചെയ്തവര്‍ ഇതില്‍ നിന്നൊഴിവാകുന്നു. അവര്‍ സ്വര്‍ഗത്തില്‍ പ്രവേശിക്കുന്നതാണ്‌. അവര്‍ ഒട്ടും അനീതിക്ക് വിധേയരാവുകയില്ല.
\end{malayalam}}
\flushright{\begin{Arabic}
\quranayah[19][61]
\end{Arabic}}
\flushleft{\begin{malayalam}
പരമകാരുണികന്‍ തന്‍റെ ദാസന്‍മാരോട് അദൃശ്യമായ നിലയില്‍ വാഗ്ദാനം ചെയ്തിട്ടുള്ള സ്ഥിരവാസത്തിനായുള്ള സ്വര്‍ഗത്തോപ്പുകളില്‍ (അവര്‍ പ്രവേശിക്കും.) തീര്‍ച്ചയായും അവന്‍റെ വാഗ്ദാനം നടപ്പില്‍ വരുന്നത് തന്നെയാകുന്നു.
\end{malayalam}}
\flushright{\begin{Arabic}
\quranayah[19][62]
\end{Arabic}}
\flushleft{\begin{malayalam}
സലാം അല്ലാതെ നിരര്‍ത്ഥകമായ യാതൊന്നും അവരവിടെ കേള്‍ക്കുകയില്ല. തങ്ങളുടെ ആഹാരം രാവിലെയും വൈകുന്നേരവും അവര്‍ക്കവിടെ ലഭിക്കുന്നതാണ്‌.
\end{malayalam}}
\flushright{\begin{Arabic}
\quranayah[19][63]
\end{Arabic}}
\flushleft{\begin{malayalam}
നമ്മുടെ ദാസന്‍മാരില്‍ നിന്ന് ആര്‍ ധര്‍മ്മനിഷ്ഠപുലര്‍ത്തുന്നവരായിരുന്നുവോ അവര്‍ക്കു നാം അവകാശപ്പെടുത്തികൊടുക്കുന്ന സ്വര്‍ഗമത്രെ അത്‌.
\end{malayalam}}
\flushright{\begin{Arabic}
\quranayah[19][64]
\end{Arabic}}
\flushleft{\begin{malayalam}
(നബിയോട് ജിബ്‌രീല്‍ പറഞ്ഞു:) താങ്കളുടെ രക്ഷിതാവിന്‍റെ കല്‍പനപ്രകാരമല്ലാതെ നാം ഇറങ്ങിവരുന്നതല്ല. നമ്മുടെ മുമ്പിലുള്ളതും നമ്മുടെ പിന്നിലുള്ളതും അതിന്നിടയിലുള്ളതും എല്ലാം അവന്‍റെതത്രെ. താങ്കളുടെ രക്ഷിതാവ് മറക്കുന്നവനായിട്ടില്ല.
\end{malayalam}}
\flushright{\begin{Arabic}
\quranayah[19][65]
\end{Arabic}}
\flushleft{\begin{malayalam}
ആകാശങ്ങളുടെയും ഭൂമിയുടെയും അവയ്ക്കിടയിലുള്ളതിന്‍റെയും രക്ഷിതാവത്രെ അവന്‍. അതിനാല്‍ അവനെ താങ്കള്‍ ആരാധിക്കുകയും അവന്നുള്ള ആരാധനയില്‍ ക്ഷമയോടെ ഉറച്ചുനില്‍ക്കുകയും ചെയ്യുക. അവന്ന് പേരൊത്ത ആരെയെങ്കിലും താങ്കള്‍ക്കറിയാമോ?
\end{malayalam}}
\flushright{\begin{Arabic}
\quranayah[19][66]
\end{Arabic}}
\flushleft{\begin{malayalam}
മനുഷ്യന്‍ പറയും: ഞാന്‍ മരിച്ചുകഴിഞ്ഞാല്‍ പിന്നീട് എന്നെ ജീവനുള്ളവനായി പുറത്ത് കൊണ്ട് വരുമോ?
\end{malayalam}}
\flushright{\begin{Arabic}
\quranayah[19][67]
\end{Arabic}}
\flushleft{\begin{malayalam}
മനുഷ്യന്‍ ഓര്‍മിക്കുന്നില്ലേ; അവന്‍ ഒന്നുമല്ലാതിരുന്ന ഒരു ഘട്ടത്തില്‍ നാമാണ് ആദ്യം അവനെ പടച്ചുണ്ടാക്കിയതെന്ന്‌?
\end{malayalam}}
\flushright{\begin{Arabic}
\quranayah[19][68]
\end{Arabic}}
\flushleft{\begin{malayalam}
എന്നാല്‍ നിന്‍റെ രക്ഷിതാവിനെ തന്നെയാണ! അവരെയും പിശാചുക്കളെയും നാം ഒരുമിച്ചുകൂട്ടുക തന്നെ ചെയ്യും. പിന്നീട് മുട്ടുകുത്തിയവരായിക്കൊണ്ട് നരകത്തിന് ചുറ്റും അവരെ നാം ഹാജരാക്കുക തന്നെ ചെയ്യും.
\end{malayalam}}
\flushright{\begin{Arabic}
\quranayah[19][69]
\end{Arabic}}
\flushleft{\begin{malayalam}
പിന്നീട് ഓരോ കക്ഷിയില്‍ നിന്നും പരമകാരുണികനോട് ഏറ്റവും കടുത്ത ധിക്കാരം കാണിച്ചിരുന്നവരെ നാം വേര്‍തിരിച്ച് നിര്‍ത്തുന്നതാണ്‌.
\end{malayalam}}
\flushright{\begin{Arabic}
\quranayah[19][70]
\end{Arabic}}
\flushleft{\begin{malayalam}
പിന്നീട് അതില്‍ (നരകത്തില്‍) എരിയുവാന്‍ അവരുടെ കൂട്ടത്തില്‍ ഏറ്റവും അര്‍ഹതയുള്ളവരെപ്പറ്റി നമുക്ക് നല്ലവണ്ണം അറിയാവുന്നതാണ്‌.
\end{malayalam}}
\flushright{\begin{Arabic}
\quranayah[19][71]
\end{Arabic}}
\flushleft{\begin{malayalam}
അതിനടുത്ത് (നരകത്തിനടുത്ത്‌) വരാത്തവരായി നിങ്ങളില്‍ ആരും തന്നെയില്ല. നിന്‍റെ രക്ഷിതാവിന്‍റെ ഖണ്ഡിതവും നടപ്പിലാക്കപ്പെടുന്നതുമായ ഒരു തീരുമാനമാകുന്നു അത്‌.
\end{malayalam}}
\flushright{\begin{Arabic}
\quranayah[19][72]
\end{Arabic}}
\flushleft{\begin{malayalam}
പിന്നീട് ധര്‍മ്മനിഷ്ഠ പാലിച്ചവരെ നാം രക്ഷപ്പെടുത്തുകയും, അക്രമികളെ മുട്ടുകുത്തിയവരായിക്കൊണ്ട് നാം അതില്‍ വിട്ടേക്കുകയും ചെയ്യുന്നതാണ്‌.
\end{malayalam}}
\flushright{\begin{Arabic}
\quranayah[19][73]
\end{Arabic}}
\flushleft{\begin{malayalam}
സ്പഷ്ടമായ നിലയില്‍ നമ്മുടെ ദൃഷ്ടാന്തങ്ങള്‍ അവര്‍ക്ക് വായിച്ചുകേള്‍പിക്കപ്പെട്ടാല്‍ അവിശ്വസിച്ചവര്‍ വിശ്വസിച്ചവരോട് പറയുന്നതാണ്‌: ഈ രണ്ട് വിഭാഗത്തില്‍ കൂടുതല്‍ ഉത്തമമായ സ്ഥാനമുള്ളവരും ഏറ്റവും മെച്ചപ്പെട്ട സംഘമുള്ളവരും ആരാണ് ?
\end{malayalam}}
\flushright{\begin{Arabic}
\quranayah[19][74]
\end{Arabic}}
\flushleft{\begin{malayalam}
സാധനസാമഗ്രികളിലും ബാഹ്യമോടിയിലും കൂടുതല്‍ മെച്ചപ്പെട്ടവരായ എത്ര തലമുറകളെയാണ് ഇവര്‍ക്ക് മുമ്പ് നാം നശിപ്പിച്ചിട്ടുള്ളത്‌!
\end{malayalam}}
\flushright{\begin{Arabic}
\quranayah[19][75]
\end{Arabic}}
\flushleft{\begin{malayalam}
(നബിയേ,) പറയുക: വല്ലവനും ദുര്‍മാര്‍ഗത്തിലായിക്കഴിഞ്ഞാല്‍ പരമകാരുണികന്‍ അവന്നു അവധി നീട്ടികൊടുക്കുന്നതാണ്‌. അങ്ങനെ തങ്ങള്‍ക്ക് മുന്നറിയിപ്പ് നല്‍കപ്പെടുന്ന കാര്യം അതായത് ഒന്നുകില്‍ ശിക്ഷ, അല്ലെങ്കില്‍ അന്ത്യസമയം -അവര്‍ കാണുമ്പോള്‍ അവര്‍ അറിഞ്ഞ് കൊള്ളും; കൂടുതല്‍ മോശമായ സ്ഥാനമുള്ളവരും, കുടുതല്‍ ദുര്‍ബലരായ സൈന്യവും ആരാണെന്ന്‌.
\end{malayalam}}
\flushright{\begin{Arabic}
\quranayah[19][76]
\end{Arabic}}
\flushleft{\begin{malayalam}
സന്‍മാര്‍ഗം സ്വീകരിച്ചവര്‍ക്ക് അല്ലാഹു സന്‍മാര്‍ഗനിഷ്ഠ വര്‍ദ്ധിപ്പിച്ച് കൊടുക്കുന്നതാണ്‌. നിലനില്‍ക്കുന്ന സല്‍കര്‍മ്മങ്ങളാണ് നിന്‍റെ രക്ഷിതാവിങ്കല്‍ ഉത്തമമായ പ്രതിഫലമുള്ളതും ഉത്തമമായ പരിണാമമുള്ളതും
\end{malayalam}}
\flushright{\begin{Arabic}
\quranayah[19][77]
\end{Arabic}}
\flushleft{\begin{malayalam}
എന്നാല്‍ നമ്മുടെ ദൃഷ്ടാന്തങ്ങളില്‍ അവിശ്വസിക്കുകയും എനിക്ക് സമ്പത്തും സന്താനവും നല്‍കപ്പെടുക തന്നെ ചെയ്യും. എന്ന് പറയുകയും ചെയ്തവനെ നീ കണ്ടുവോ?
\end{malayalam}}
\flushright{\begin{Arabic}
\quranayah[19][78]
\end{Arabic}}
\flushleft{\begin{malayalam}
അദൃശ്യകാര്യം അവന്‍ കണ്ടറിഞ്ഞിട്ടുണ്ടോ? അതല്ലെങ്കില്‍ പരമകാരുണികന്‍റെ അടുത്ത് അവന്‍ വല്ല കരാറുമുണ്ടാക്കിയിട്ടുണ്ടോ?
\end{malayalam}}
\flushright{\begin{Arabic}
\quranayah[19][79]
\end{Arabic}}
\flushleft{\begin{malayalam}
അല്ല, അവന്‍ പറയുന്നത് നാം രേഖപ്പെടുത്തുകയും, അവന്നു ശിക്ഷ കൂട്ടികൊടുക്കുകയും ചെയ്യും.
\end{malayalam}}
\flushright{\begin{Arabic}
\quranayah[19][80]
\end{Arabic}}
\flushleft{\begin{malayalam}
അവന്‍ ആ പറയുന്നതിനെല്ലാം (സ്വത്തിനും സന്താനത്തിനുമെല്ലാം) നാമായിരിക്കും അനന്തരാവകാശിയാകുന്നത്‌. അവന്‍ ഏകനായിക്കൊണ്ട് നമ്മുടെ അടുത്ത് വരികയും ചെയ്യും.
\end{malayalam}}
\flushright{\begin{Arabic}
\quranayah[19][81]
\end{Arabic}}
\flushleft{\begin{malayalam}
അല്ലാഹുവിന് പുറമെ അവര്‍ ദൈവങ്ങളെ സ്വീകരിച്ചിരിക്കുകയാണ്‌. അവര്‍ ഇവര്‍ക്ക് പിന്‍ബലമാകുന്നതിന് വേണ്ടി.
\end{malayalam}}
\flushright{\begin{Arabic}
\quranayah[19][82]
\end{Arabic}}
\flushleft{\begin{malayalam}
അല്ല, ഇവര്‍ ആരാധന നടത്തിയ കാര്യം തന്നെ അവര്‍ നിഷേധിക്കുകയും, അവര്‍ ഇവര്‍ക്ക് എതിരായിത്തീരുകയും ചെയ്യുന്നതാണ്‌.
\end{malayalam}}
\flushright{\begin{Arabic}
\quranayah[19][83]
\end{Arabic}}
\flushleft{\begin{malayalam}
സത്യനിഷേധികളുടെ നേര്‍ക്ക,് അവരെ ശക്തിയായി ഇളക്കിവിടാന്‍ വേണ്ടി നാം പിശാചുക്കളെ അയച്ചുവിട്ടിരിക്കുകയാണെന്ന് നീ കണ്ടില്ലേ?
\end{malayalam}}
\flushright{\begin{Arabic}
\quranayah[19][84]
\end{Arabic}}
\flushleft{\begin{malayalam}
അതിനാല്‍ അവരുടെ കാര്യത്തില്‍ നീ തിടുക്കം കാണിക്കേണ്ട. അവര്‍ക്കായി നാം (നാളുകള്‍) എണ്ണി എണ്ണിക്കൊണ്ടിരിക്കുക മാത്രമാകുന്നു.
\end{malayalam}}
\flushright{\begin{Arabic}
\quranayah[19][85]
\end{Arabic}}
\flushleft{\begin{malayalam}
ധര്‍മ്മനിഷ്ഠയുള്ളവരെ വിശിഷ്ടാതിഥികള്‍ എന്ന നിലയില്‍ പരമകാരുണികന്‍റെ അടുത്തേക്ക് നാം വിളിച്ചുകൂട്ടുന്ന ദിവസം.
\end{malayalam}}
\flushright{\begin{Arabic}
\quranayah[19][86]
\end{Arabic}}
\flushleft{\begin{malayalam}
കുറ്റവാളികളെ ദാഹാര്‍ത്തരായ നിലയില്‍ നരകത്തിലേക്ക് നാം തെളിച്ച് കൊണ്ട് പോകുകയും ചെയ്യുന്ന ദിവസം.
\end{malayalam}}
\flushright{\begin{Arabic}
\quranayah[19][87]
\end{Arabic}}
\flushleft{\begin{malayalam}
ആര്‍ക്കും ശുപാര്‍ശ ചെയ്യാന്‍ അധികാരമുണ്ടായിരിക്കുകയില്ല. പരമകാരുണികനുമായി കരാറുണ്ടാക്കിയിട്ടുള്ളവനൊഴികെ.
\end{malayalam}}
\flushright{\begin{Arabic}
\quranayah[19][88]
\end{Arabic}}
\flushleft{\begin{malayalam}
പരമകാരുണികന്‍ ഒരു സന്താനത്തെ സ്വീകരിച്ചിട്ടുണ്ടെന്ന് അവര്‍ പറഞ്ഞിരിക്കുന്നു.
\end{malayalam}}
\flushright{\begin{Arabic}
\quranayah[19][89]
\end{Arabic}}
\flushleft{\begin{malayalam}
(അപ്രകാരം പറയുന്നവരേ,) തീര്‍ച്ചയായും നിങ്ങള്‍ ചെയ്തിരിക്കുന്നത് ഗുരുതരമായ ഒരു കാര്യമാകുന്നു.
\end{malayalam}}
\flushright{\begin{Arabic}
\quranayah[19][90]
\end{Arabic}}
\flushleft{\begin{malayalam}
അത് നിമിത്തം ആകാശങ്ങള്‍ പൊട്ടിപ്പിളരുകയും, ഭൂമി വിണ്ടുകീറുകയും, പര്‍വ്വതങ്ങള്‍ തകര്‍ന്ന് വീഴുകയും ചെയ്യുമാറാകും.
\end{malayalam}}
\flushright{\begin{Arabic}
\quranayah[19][91]
\end{Arabic}}
\flushleft{\begin{malayalam}
(അതെ,) പരമകാരുണികന് സന്താനമുണ്ടെന്ന് അവര്‍ വാദിച്ചത് നിമിത്തം.
\end{malayalam}}
\flushright{\begin{Arabic}
\quranayah[19][92]
\end{Arabic}}
\flushleft{\begin{malayalam}
സന്താനത്തെ സ്വീകരിക്കുക എന്നത് പരമകാരുണികന് അനുയോജ്യമാവുകയില്ല.
\end{malayalam}}
\flushright{\begin{Arabic}
\quranayah[19][93]
\end{Arabic}}
\flushleft{\begin{malayalam}
ആകാശങ്ങളിലും ഭൂമിയിലുമുള്ള ഏതൊരാളും ഒരു ദാസനെന്ന നിലയില്‍ പരമകാരുണികന്‍റെ അടുത്ത് വരുന്നവന്‍ മാത്രമായിരിക്കും.
\end{malayalam}}
\flushright{\begin{Arabic}
\quranayah[19][94]
\end{Arabic}}
\flushleft{\begin{malayalam}
തീര്‍ച്ചയായും അവരെ അവന്‍ തിട്ടപ്പെടുത്തുകയും എണ്ണികണക്കാക്കുകയും ചെയ്തിരിക്കുന്നു.
\end{malayalam}}
\flushright{\begin{Arabic}
\quranayah[19][95]
\end{Arabic}}
\flushleft{\begin{malayalam}
അവരോരോരുത്തരും ഉയിര്‍ത്തെഴുന്നേല്‍പിന്‍റെ നാളില്‍ ഏകാകിയായിക്കൊണ്ട് അവന്‍റെ അടുക്കല്‍ വരുന്നതാണ്‌.
\end{malayalam}}
\flushright{\begin{Arabic}
\quranayah[19][96]
\end{Arabic}}
\flushleft{\begin{malayalam}
വിശ്വസിക്കുകയും സല്‍കര്‍മ്മങ്ങള്‍ പ്രവര്‍ത്തിക്കുകയും ചെയ്തവരാരോ അവര്‍ക്ക് പരമകാരുണികന്‍ സ്നേഹമുണ്ടാക്കികൊടുക്കുന്നതാണ്‌; തീര്‍ച്ച.
\end{malayalam}}
\flushright{\begin{Arabic}
\quranayah[19][97]
\end{Arabic}}
\flushleft{\begin{malayalam}
ഇത് (ഖുര്‍ആന്‍) നിന്‍റെ ഭാഷയില്‍ നാം ലളിതമാക്കിതന്നിരിക്കുന്നത് ധര്‍മ്മനിഷ്ഠയുള്ളവര്‍ക്ക് ഇത് മുഖേന നീ സന്തോഷവാര്‍ത്ത നല്‍കുവാനും, മര്‍ക്കടമുഷ്ടിക്കാരായ ആളുകള്‍ക്ക് ഇത് മുഖേന നീ താക്കീത് നല്‍കുവാനും വേണ്ടി മാത്രമാകുന്നു.
\end{malayalam}}
\flushright{\begin{Arabic}
\quranayah[19][98]
\end{Arabic}}
\flushleft{\begin{malayalam}
ഇവര്‍ക്ക് മുമ്പ് എത്ര തലമുറകളെ നാം നശിപ്പിച്ചിട്ടുണ്ട്‌. അവരില്‍ നിന്ന് ആരെയെങ്കിലും നീ കാണുന്നുണ്ടോ? അഥവാ അവരുടെ നേരിയ ശബ്ദമെങ്കിലും നീ കേള്‍ക്കുന്നുണ്ടോ?
\end{malayalam}}
\chapter{\textmalayalam{ത്വാഹാ}}
\begin{Arabic}
\Huge{\centerline{\basmalah}}\end{Arabic}
\flushright{\begin{Arabic}
\quranayah[20][1]
\end{Arabic}}
\flushleft{\begin{malayalam}
ത്വാഹാ
\end{malayalam}}
\flushright{\begin{Arabic}
\quranayah[20][2]
\end{Arabic}}
\flushleft{\begin{malayalam}
നിനക്ക് നാം ഖുര്‍ആന്‍ അവതരിപ്പിച്ച് തന്നത് നീ കഷ്ടപ്പെടാന്‍ വേണ്ടിയല്ല.
\end{malayalam}}
\flushright{\begin{Arabic}
\quranayah[20][3]
\end{Arabic}}
\flushleft{\begin{malayalam}
ഭയപ്പെടുന്നവര്‍ക്ക് ഉല്‍ബോധനം നല്‍കാന്‍ വേണ്ടി മാത്രമാണത്‌.
\end{malayalam}}
\flushright{\begin{Arabic}
\quranayah[20][4]
\end{Arabic}}
\flushleft{\begin{malayalam}
ഭൂമിയും ഉന്നതമായ ആകാശങ്ങളും സൃഷ്ടിച്ചവന്‍റെ പക്കല്‍ നിന്ന് അവതരിപ്പിക്കപ്പെട്ടതത്രെ അത്‌.
\end{malayalam}}
\flushright{\begin{Arabic}
\quranayah[20][5]
\end{Arabic}}
\flushleft{\begin{malayalam}
പരമകാരുണികന്‍ സിംഹാസനസ്ഥനായിരിക്കുന്നു.
\end{malayalam}}
\flushright{\begin{Arabic}
\quranayah[20][6]
\end{Arabic}}
\flushleft{\begin{malayalam}
അവന്നുള്ളതാകുന്നു ആകാശങ്ങളിലുള്ളതും ഭൂമിയിലുള്ളതും, അവയ്ക്കിടയിലുള്ളതും, മണ്ണിനടിയിലുള്ളതുമെല്ലാം.
\end{malayalam}}
\flushright{\begin{Arabic}
\quranayah[20][7]
\end{Arabic}}
\flushleft{\begin{malayalam}
നീ വാക്ക് ഉച്ചത്തിലാക്കുകയാണെങ്കില്‍ തീര്‍ച്ചയായും അവന്‍ (അല്ലാഹു) രഹസ്യമായതും, അത്യന്തം നിഗൂഢമായതും അറിയും (എന്ന് നീ മനസ്സിലാക്കുക)
\end{malayalam}}
\flushright{\begin{Arabic}
\quranayah[20][8]
\end{Arabic}}
\flushleft{\begin{malayalam}
അല്ലാഹു- അവനല്ലാതെ ഒരു ദൈവവുമില്ല. അവന്‍റെതാകുന്നു ഏറ്റവും ഉല്‍കൃഷ്ടമായ നാമങ്ങള്‍.
\end{malayalam}}
\flushright{\begin{Arabic}
\quranayah[20][9]
\end{Arabic}}
\flushleft{\begin{malayalam}
മൂസായുടെ വര്‍ത്തമാനം നിനക്ക് വന്നുകിട്ടിയോ?
\end{malayalam}}
\flushright{\begin{Arabic}
\quranayah[20][10]
\end{Arabic}}
\flushleft{\begin{malayalam}
അതായത് അദ്ദേഹം ഒരു തീ കണ്ട സന്ദര്‍ഭം. അപ്പോള്‍ തന്‍റെ കുടുംബത്തോട് അദ്ദേഹം പറഞ്ഞു: നിങ്ങള്‍ നില്‍ക്കൂ; ഞാന്‍ ഒരു തീ കണ്ടിരിക്കുന്നു. ഞാന്‍ അതില്‍ നിന്ന് കത്തിച്ചെടുത്തുകൊണ്ട് നിങ്ങളുടെ അടുത്ത് വന്നേക്കാം. അല്ലെങ്കില്‍ തീയുടെ അടുത്ത് വല്ല വഴികാട്ടിയെയും ഞാന്‍ കണ്ടേക്കും.
\end{malayalam}}
\flushright{\begin{Arabic}
\quranayah[20][11]
\end{Arabic}}
\flushleft{\begin{malayalam}
അങ്ങനെ അദ്ദേഹം അതിനടുത്ത് ചെന്നപ്പോള്‍ (ഇപ്രകാരം) വിളിച്ചുപറയപ്പെട്ടു ഹേ; മൂസാ.
\end{malayalam}}
\flushright{\begin{Arabic}
\quranayah[20][12]
\end{Arabic}}
\flushleft{\begin{malayalam}
തീര്‍ച്ചയായും ഞാനാണ് നിന്‍റെ രക്ഷിതാവ്‌. അതിനാല്‍ നീ നിന്‍റെ ചെരിപ്പുകള്‍ അഴിച്ച് വെക്കുക. നീ ത്വുവാ എന്ന പരിശുദ്ധ താഴ്‌വരയിലാകുന്നു.
\end{malayalam}}
\flushright{\begin{Arabic}
\quranayah[20][13]
\end{Arabic}}
\flushleft{\begin{malayalam}
ഞാന്‍ നിന്നെ തെരഞ്ഞെടുത്തിരിക്കുന്നു. അതിനാല്‍ ബോധനം നല്‍കപ്പെടുന്നത് നീ ശ്രദ്ധിച്ച് കേട്ടുകൊള്ളുക.
\end{malayalam}}
\flushright{\begin{Arabic}
\quranayah[20][14]
\end{Arabic}}
\flushleft{\begin{malayalam}
തീര്‍ച്ചയായും ഞാനാകുന്നു അല്ലാഹു. ഞാനല്ലാതെ ഒരു ദൈവവുമില്ല. അതിനാല്‍ എന്നെ നീ ആരാധിക്കുകയും, എന്നെ ഓര്‍മിക്കുന്നതിനായി നമസ്കാരം മുറപോലെ നിര്‍വഹിക്കുകയും ചെയ്യുക.
\end{malayalam}}
\flushright{\begin{Arabic}
\quranayah[20][15]
\end{Arabic}}
\flushleft{\begin{malayalam}
തീര്‍ച്ചയായും അന്ത്യസമയം വരിക തന്നെ ചെയ്യും. ഓരോ വ്യക്തിക്കും താന്‍ പ്രയത്നിക്കുന്നതിനനുസൃതമായി പ്രതിഫലം നല്‍കപ്പെടാന്‍ വേണ്ടി ഞാനത് ഗോപ്യമാക്കി വെച്ചേക്കാം.
\end{malayalam}}
\flushright{\begin{Arabic}
\quranayah[20][16]
\end{Arabic}}
\flushleft{\begin{malayalam}
ആകയാല്‍ അതില്‍ (അന്ത്യസമയത്തില്‍) വിശ്വസിക്കാതിരിക്കുകയും തന്നിഷ്ടത്തെ പിന്‍പറ്റുകയും ചെയ്തവര്‍ അതില്‍ (വിശ്വസിക്കുന്നതില്‍) നിന്ന് നിന്നെ തടയാതിരിക്കട്ടെ. അങ്ങനെ സംഭവിക്കുന്ന പക്ഷം നീയും നാശമടയുന്നതാണ്‌.
\end{malayalam}}
\flushright{\begin{Arabic}
\quranayah[20][17]
\end{Arabic}}
\flushleft{\begin{malayalam}
അല്ലാഹു പറഞ്ഞു:) ഹേ; മൂസാ, നിന്‍റെ വലതുകയ്യിലുള്ള ആ വസ്തു എന്താകുന്നു?
\end{malayalam}}
\flushright{\begin{Arabic}
\quranayah[20][18]
\end{Arabic}}
\flushleft{\begin{malayalam}
അദ്ദേഹം പറഞ്ഞു: ഇത് എന്‍റെ വടിയാകുന്നു. ഞാനതിന്‍മേല്‍ ഊന്നി നില്‍ക്കുകയും, അത് കൊണ്ട് എന്‍റെ ആടുകള്‍ക്ക് (ഇല) അടിച്ചുവീഴ്ത്തി കൊടുക്കുകയും ചെയ്യുന്നു. അതുകൊണ്ട് എനിക്ക് വേറെയും ഉപയോഗങ്ങളുണ്ട്‌.
\end{malayalam}}
\flushright{\begin{Arabic}
\quranayah[20][19]
\end{Arabic}}
\flushleft{\begin{malayalam}
അവന്‍ (അല്ലാഹു) പറഞ്ഞു: ഹേ; മൂസാ, നീ ആ വടി താഴെയിടൂ.
\end{malayalam}}
\flushright{\begin{Arabic}
\quranayah[20][20]
\end{Arabic}}
\flushleft{\begin{malayalam}
അദ്ദേഹം അത് താഴെയിട്ടു. അപ്പോഴതാ അത് ഒരു പാമ്പായി ഓടുന്നു.
\end{malayalam}}
\flushright{\begin{Arabic}
\quranayah[20][21]
\end{Arabic}}
\flushleft{\begin{malayalam}
അവന്‍ പറഞ്ഞു: അതിനെ പിടിച്ച് കൊള്ളുക. പേടിക്കേണ്ട. നാം അതിനെ അതിന്‍റെ ആദ്യസ്ഥിതിയിലേക്ക് തന്നെ മടക്കുന്നതാണ്‌.
\end{malayalam}}
\flushright{\begin{Arabic}
\quranayah[20][22]
\end{Arabic}}
\flushleft{\begin{malayalam}
നീ നിന്‍റെ കൈ കക്ഷത്തിലേക്ക് ചേര്‍ത്ത് പിടിക്കുക. യാതൊരു ദൂഷ്യവും കൂടാതെ തെളിഞ്ഞ വെള്ളനിറമുള്ളതായി അത് പുറത്ത് വരുന്നതാണ്‌. അത് മറ്റൊരു ദൃഷ്ടാന്തമത്രെ.
\end{malayalam}}
\flushright{\begin{Arabic}
\quranayah[20][23]
\end{Arabic}}
\flushleft{\begin{malayalam}
നമ്മുടെ മഹത്തായ ദൃഷ്ടാന്തങ്ങളില്‍ ചിലത് നിനക്ക് നം കാണിച്ചുതരുവാന്‍ വേണ്ടിയത്രെ അത്‌.
\end{malayalam}}
\flushright{\begin{Arabic}
\quranayah[20][24]
\end{Arabic}}
\flushleft{\begin{malayalam}
നീ ഫിര്‍ഔന്‍റെ അടുത്തേക്ക് പോകുക. തീര്‍ച്ചയായും അവന്‍ അതിക്രമകാരിയായിരിക്കുന്നു.
\end{malayalam}}
\flushright{\begin{Arabic}
\quranayah[20][25]
\end{Arabic}}
\flushleft{\begin{malayalam}
അദ്ദേഹം പറഞ്ഞു: എന്‍റെ രക്ഷിതാവേ, നീ എനിക്ക് ഹൃദയവിശാലത നല്‍കേണമേ.
\end{malayalam}}
\flushright{\begin{Arabic}
\quranayah[20][26]
\end{Arabic}}
\flushleft{\begin{malayalam}
എനിക്ക് എന്‍റെ കാര്യം നീ എളുപ്പമാക്കിത്തരേണമേ.
\end{malayalam}}
\flushright{\begin{Arabic}
\quranayah[20][27]
\end{Arabic}}
\flushleft{\begin{malayalam}
എന്‍റെ നാവില്‍ നിന്ന് നീ കെട്ടഴിച്ച് തരേണമേ.
\end{malayalam}}
\flushright{\begin{Arabic}
\quranayah[20][28]
\end{Arabic}}
\flushleft{\begin{malayalam}
ജനങ്ങള്‍ എന്‍റെ സംസാരം മനസ്സിലാക്കേണ്ടതിന്‌.
\end{malayalam}}
\flushright{\begin{Arabic}
\quranayah[20][29]
\end{Arabic}}
\flushleft{\begin{malayalam}
എന്‍റെ കുടുംബത്തില്‍ നിന്ന് എനിക്ക് ഒരു സഹായിയെ നീ ഏര്‍പെടുത്തുകയും ചെയ്യേണമേ.
\end{malayalam}}
\flushright{\begin{Arabic}
\quranayah[20][30]
\end{Arabic}}
\flushleft{\begin{malayalam}
അതായത് എന്‍റെ സഹോദരന്‍ ഹാറൂനെ.
\end{malayalam}}
\flushright{\begin{Arabic}
\quranayah[20][31]
\end{Arabic}}
\flushleft{\begin{malayalam}
അവന്‍ മുഖേന എന്‍റെ ശക്തി നീ ദൃഢമാക്കുകയും,
\end{malayalam}}
\flushright{\begin{Arabic}
\quranayah[20][32]
\end{Arabic}}
\flushleft{\begin{malayalam}
എന്‍റെ കാര്യത്തില്‍ അവനെ നീ പങ്കാളിയാക്കുകയും ചെയ്യേണമേ.
\end{malayalam}}
\flushright{\begin{Arabic}
\quranayah[20][33]
\end{Arabic}}
\flushleft{\begin{malayalam}
ഞങ്ങള്‍ ധാരാളമായി നിന്‍റെ പരിശുദ്ധിയെ വാഴ്ത്തുവാനും,
\end{malayalam}}
\flushright{\begin{Arabic}
\quranayah[20][34]
\end{Arabic}}
\flushleft{\begin{malayalam}
ധാരാളമായി നിന്നെ ഞങ്ങള്‍ സ്മരിക്കുവാനും വേണ്ടി.
\end{malayalam}}
\flushright{\begin{Arabic}
\quranayah[20][35]
\end{Arabic}}
\flushleft{\begin{malayalam}
തീര്‍ച്ചയായും നീ ഞങ്ങളെപ്പറ്റി കണ്ടറിയുന്നവനാകുന്നു.
\end{malayalam}}
\flushright{\begin{Arabic}
\quranayah[20][36]
\end{Arabic}}
\flushleft{\begin{malayalam}
അവന്‍ (അല്ലാഹു) പറഞ്ഞു: ഹേ; മൂസാ, നീ ചോദിച്ചത് നിനക്ക് നല്‍കപ്പെട്ടിരിക്കുന്നു.
\end{malayalam}}
\flushright{\begin{Arabic}
\quranayah[20][37]
\end{Arabic}}
\flushleft{\begin{malayalam}
മറ്റൊരിക്കലും നിനക്ക് നാം അനുഗ്രഹം ചെയ്ത് തന്നിട്ടുണ്ട്‌.
\end{malayalam}}
\flushright{\begin{Arabic}
\quranayah[20][38]
\end{Arabic}}
\flushleft{\begin{malayalam}
അതായത് നിന്‍റെ മാതാവിന് ബോധനം നല്‍കപ്പെടേണ്ട കാര്യം നാം ബോധനം നല്‍കിയ സന്ദര്‍ഭത്തില്‍.
\end{malayalam}}
\flushright{\begin{Arabic}
\quranayah[20][39]
\end{Arabic}}
\flushleft{\begin{malayalam}
നീ അവനെ (കുട്ടിയെ) പെട്ടിയിലാക്കിയിട്ട് നദിയിലിട്ടേക്കുക. നദി ആ പെട്ടി കരയില്‍ തള്ളിക്കൊള്ളും. എനിക്കും അവന്നും ശത്രുവായിട്ടുള്ള ഒരാള്‍ അവനെ എടുത്ത് കൊള്ളും. (ഹേ; മൂസാ,) എന്‍റെ പക്കല്‍ നിന്നുള്ള സ്നേഹം നിന്‍റെ മേല്‍ ഞാന്‍ ഇട്ടുതരികയും ചെയ്തു. എന്‍റെ നോട്ടത്തിലായിക്കൊണ്ട നീ വളര്‍ത്തിയെടുക്കപ്പെടാന്‍ വേണ്ടിയും കൂടിയാണത്‌.
\end{malayalam}}
\flushright{\begin{Arabic}
\quranayah[20][40]
\end{Arabic}}
\flushleft{\begin{malayalam}
നിന്‍റെ സഹോദരി നടന്ന് ചെല്ലുകയും ഇവന്‍റെ (കുട്ടിയുടെ) സംരക്ഷണമേല്‍ക്കാന്‍ കഴിയുന്ന ഒരാളെപ്പറ്റി ഞാന്‍ നിങ്ങള്‍ക്ക് അറിയിച്ച് തരട്ടെയോ എന്ന് പറയുകയും ചെയ്യുന്ന സന്ദര്‍ഭം (ശ്രദ്ധേയമാകുന്നു.) അങ്ങനെ നിന്‍റെ മാതാവിങ്കലേക്ക് തന്നെ നിന്നെ നാം തിരിച്ചേല്‍പിച്ചു. അവളുടെ കണ്‍കുളിര്‍ക്കുവാനും, അവള്‍ ദുഃഖിക്കാതിരിക്കുവാനും വേണ്ടി. നീ ഒരാളെ കൊല്ലുകയുണ്ടായി. എന്നിട്ട് (അതു സംബന്ധിച്ച്‌) മനഃക്ലേശത്തില്‍ നിന്ന് നിന്നെ നാം രക്ഷിക്കുകയും ചെയ്തു. പല പരീക്ഷണങ്ങളിലൂടെയും നിന്നെ നാം പരീക്ഷിക്കുകയുണ്ടായി. അങ്ങനെ മദ്‌യങ്കാരുടെ കൂട്ടത്തില്‍ കൊല്ലങ്ങളോളം നീ താമസിച്ചു. പിന്നീട് ഹേ; മൂസാ, നീ (എന്‍റെ) ഒരു നിശ്ചയപ്രകാരം ഇതാ വന്നിരിക്കുന്നു.
\end{malayalam}}
\flushright{\begin{Arabic}
\quranayah[20][41]
\end{Arabic}}
\flushleft{\begin{malayalam}
എന്‍റെ സ്വന്തം കാര്യത്തിനായി നിന്നെ ഞാന്‍ വളര്‍ത്തിയെടുത്തിരിക്കുന്നു.
\end{malayalam}}
\flushright{\begin{Arabic}
\quranayah[20][42]
\end{Arabic}}
\flushleft{\begin{malayalam}
എന്‍റെ ദൃഷ്ടാന്തങ്ങളുമായി നീയും നിന്‍റെ സഹോദരനും പോയിക്കൊള്ളുക. എന്നെ സ്മരിക്കുന്നതില്‍ നിങ്ങള്‍ അമാന്തിക്കരുത്‌.
\end{malayalam}}
\flushright{\begin{Arabic}
\quranayah[20][43]
\end{Arabic}}
\flushleft{\begin{malayalam}
നിങ്ങള്‍ രണ്ടുപേരും ഫിര്‍ഔന്‍റെ അടുത്തേക്ക് പോകുക. തീര്‍ച്ചയായും അവന്‍ അതിക്രമകാരിയായിരിക്കുന്നു.
\end{malayalam}}
\flushright{\begin{Arabic}
\quranayah[20][44]
\end{Arabic}}
\flushleft{\begin{malayalam}
എന്നിട്ട് നിങ്ങള്‍ അവനോട് സൌമ്യമായ വാക്ക് പറയുക. അവന്‍ ഒരു വേള ചിന്തിച്ച് മനസ്സിലാക്കിയേക്കാം. അല്ലെങ്കില്‍ ഭയപ്പെട്ടുവെന്ന് വരാം.
\end{malayalam}}
\flushright{\begin{Arabic}
\quranayah[20][45]
\end{Arabic}}
\flushleft{\begin{malayalam}
അവര്‍ രണ്ടുപേരും പറഞ്ഞു: ഞങ്ങളുടെ രക്ഷിതാവേ, അവന്‍ (ഫിര്‍ഔന്‍) ഞങ്ങളുടെ നേര്‍ക്ക് എടുത്തുചാടുകയോ, അതിക്രമം കാണിക്കുകയോ ചെയ്യുമെന്ന് ഞാന്‍ ഭയപ്പെടുന്നു.
\end{malayalam}}
\flushright{\begin{Arabic}
\quranayah[20][46]
\end{Arabic}}
\flushleft{\begin{malayalam}
അവന്‍ (അല്ലാഹു) പറഞ്ഞു: നിങ്ങള്‍ ഭയപ്പെടേണ്ട. തീര്‍ച്ചയായും ഞാന്‍ നിങ്ങളുടെ കൂടെയുണ്ട്‌. ഞാന്‍ കേള്‍ക്കുകയും കാണുകയും ചെയ്യുന്നുണ്ട്‌.
\end{malayalam}}
\flushright{\begin{Arabic}
\quranayah[20][47]
\end{Arabic}}
\flushleft{\begin{malayalam}
അതിനാല്‍ നിങ്ങള്‍ ഇരുവരും അവന്‍റെ അടുത്ത് ചെന്നിട്ട് പറയുക: തീര്‍ച്ചയായും ഞങ്ങള്‍ നിന്‍റെ രക്ഷിതാവിന്‍റെ ദൂതന്‍മാരാകുന്നു. അതിനാല്‍ ഇസ്രായീല്‍ സന്തതികളെ ഞങ്ങളുടെ കുടെ വിട്ടുതരണം. അവരെ മര്‍ദ്ദിക്കരുത്‌. നിന്‍റെയടുത്ത് ഞങ്ങള്‍ വന്നിട്ടുള്ളത് നിന്‍റെ രക്ഷിതാവിങ്കല്‍ നിന്നുള്ള ദൃഷ്ടാന്തവും കൊണ്ടാകുന്നു. സന്‍മാര്‍ഗം പിന്തുടര്‍ന്നവര്‍ക്കായിരിക്കും സമാധാനം.
\end{malayalam}}
\flushright{\begin{Arabic}
\quranayah[20][48]
\end{Arabic}}
\flushleft{\begin{malayalam}
നിഷേധിച്ച് തള്ളുകയും പിന്‍മാറിക്കളയുകയും ചെയ്തവര്‍ക്കാണ് ശിക്ഷയുള്ളതെന്ന് തീര്‍ച്ചയായും ഞങ്ങള്‍ക്ക് ബോധനം നല്‍കപ്പെട്ടിരിക്കുന്നു.
\end{malayalam}}
\flushright{\begin{Arabic}
\quranayah[20][49]
\end{Arabic}}
\flushleft{\begin{malayalam}
അവന്‍ (ഫിര്‍ഔന്‍) ചോദിച്ചു: ഹേ; മൂസാ, അപ്പോള്‍ ആരാണ് നിങ്ങളുടെ രണ്ട് പേരുടെയും രക്ഷിതാവ്‌?
\end{malayalam}}
\flushright{\begin{Arabic}
\quranayah[20][50]
\end{Arabic}}
\flushleft{\begin{malayalam}
അദ്ദേഹം (മൂസാ) പറഞ്ഞു: ഓരോ വസ്തുവിനും അതിന്‍റെ പ്രകൃതം നല്‍കുകയും, എന്നിട്ട് (അതിന്‌) വഴി കാണിക്കുകയും ചെയ്തവനാരോ അവനത്രെ ഞങ്ങളുടെ രക്ഷിതാവ്‌.
\end{malayalam}}
\flushright{\begin{Arabic}
\quranayah[20][51]
\end{Arabic}}
\flushleft{\begin{malayalam}
അവന്‍ പറഞ്ഞു: അപ്പോള്‍ മുന്‍ തലമുറകളുടെ അവസ്ഥയെന്താണ് ?
\end{malayalam}}
\flushright{\begin{Arabic}
\quranayah[20][52]
\end{Arabic}}
\flushleft{\begin{malayalam}
അദ്ദേഹം പറഞ്ഞു: അവരെ സംബന്ധിച്ചുള്ള അറിവ് എന്‍റെ രക്ഷിതാവിങ്കല്‍ ഒരു രേഖയിലുണ്ട്‌. എന്‍റെ രക്ഷിതാവ് പിഴച്ച് പോകുകയില്ല. അവന്‍ മറന്നുപോകുകയുമില്ല.
\end{malayalam}}
\flushright{\begin{Arabic}
\quranayah[20][53]
\end{Arabic}}
\flushleft{\begin{malayalam}
നിങ്ങള്‍ക്ക് വേണ്ടി ഭൂമിയെ തൊട്ടിലാക്കുകയും, നിങ്ങള്‍ക്ക് അതില്‍ വഴികള്‍ ഏര്‍പെടുത്തിത്തരികയും, ആകാശത്ത് നിന്ന് വെള്ളം ഇറക്കിത്തരികയും ചെയ്തവനത്രെ അവന്‍. അങ്ങനെ അത് (വെള്ളം) മൂലം വ്യത്യസ്ത തരത്തിലുള്ള സസ്യങ്ങളുടെ ജോടികള്‍ നാം (അല്ലാഹു) ഉല്‍പാദിപ്പിക്കുകയും ചെയ്തിരിക്കുന്നു.
\end{malayalam}}
\flushright{\begin{Arabic}
\quranayah[20][54]
\end{Arabic}}
\flushleft{\begin{malayalam}
നിങ്ങള്‍ തിന്നുകയും, നിങ്ങളുടെ കന്നുകാലികളെ മേയ്ക്കുകയും ചെയ്തുകൊള്ളുക. ബുദ്ധിമാന്‍മാര്‍ക്ക് അതില്‍ ദൃഷ്ടാന്തങ്ങളുണ്ട്‌.
\end{malayalam}}
\flushright{\begin{Arabic}
\quranayah[20][55]
\end{Arabic}}
\flushleft{\begin{malayalam}
അതില്‍ (ഭൂമിയില്‍) നിന്നാണ് നിങ്ങളെ നാം സൃഷ്ടിച്ചത്‌. അതിലേക്ക് തന്നെ നിങ്ങളെ നാം മടക്കുന്നു. അതില്‍ നിന്ന് തന്നെ നിങ്ങളെ മറ്റൊരു പ്രാവശ്യം നാം പുറത്തുകൊണ്ട് വരികയും ചെയ്യും.
\end{malayalam}}
\flushright{\begin{Arabic}
\quranayah[20][56]
\end{Arabic}}
\flushleft{\begin{malayalam}
നമ്മുടെ ദൃഷ്ടാന്തങ്ങളോരോന്നും നാം അവന്ന് (ഫിര്‍ഔന്ന്‌) കാണിച്ചുകൊടുക്കുക തന്നെ ചെയ്തു. എന്നിട്ടും അവന്‍ നിഷേധിച്ച് തള്ളുകയും നിരസിക്കുകയുമാണ് ചെയ്തത്‌.
\end{malayalam}}
\flushright{\begin{Arabic}
\quranayah[20][57]
\end{Arabic}}
\flushleft{\begin{malayalam}
അവന്‍ പറഞ്ഞു: ഓ മൂസാ, നിന്‍റെ ജാലവിദ്യകൊണ്ട് ഞങ്ങളെ ഞങ്ങളുടെ നാട്ടില്‍ നിന്ന് പുറന്തള്ളാന്‍ വേണ്ടിയാണോ നീ ഞങ്ങളുടെ അടുത്ത് വന്നിരിക്കുന്നത്‌?
\end{malayalam}}
\flushright{\begin{Arabic}
\quranayah[20][58]
\end{Arabic}}
\flushleft{\begin{malayalam}
എന്നാല്‍ ഇത് പോലെയുള്ള ജാലവിദ്യ തീര്‍ച്ചയായും ഞങ്ങള്‍ നിന്‍റെ അടുത്ത് കൊണ്ട് വന്ന് കാണിക്കാം. അത് കൊണ്ട് ഞങ്ങള്‍ക്കും നിനക്കുമിടയില്‍ നീ ഒരു അവധി നിശ്ചയിക്കുക. ഞങ്ങളോ നീയോ അത് ലംഘിക്കാവുന്നതല്ല. മദ്ധ്യസ്ഥമായ ഒരു സ്ഥലത്തായിരിക്കട്ടെ അത്‌.
\end{malayalam}}
\flushright{\begin{Arabic}
\quranayah[20][59]
\end{Arabic}}
\flushleft{\begin{malayalam}
അദ്ദേഹം (മൂസാ) പറഞ്ഞു: നിങ്ങള്‍ക്കുള്ള അവധി ഉത്സവ ദിവസമാകുന്നു. പൂര്‍വ്വാഹ്നത്തില്‍ ജനങ്ങളെ ഒരുമിച്ചുകൂട്ടേണ്ടതാണ്‌.
\end{malayalam}}
\flushright{\begin{Arabic}
\quranayah[20][60]
\end{Arabic}}
\flushleft{\begin{malayalam}
എന്നിട്ട് ഫിര്‍ഔന്‍ പിരിഞ്ഞ് പോയി. തന്‍റെ തന്ത്രങ്ങള്‍ സംഘടിപ്പിച്ചു. എന്നിട്ടവന്‍ (നിശ്ചിത സമയത്ത്‌) വന്നു.
\end{malayalam}}
\flushright{\begin{Arabic}
\quranayah[20][61]
\end{Arabic}}
\flushleft{\begin{malayalam}
മൂസാ അവരോട് പറഞ്ഞു: നിങ്ങള്‍ക്ക് നാശം! നിങ്ങള്‍ അല്ലാഹുവിന്‍റെ പേരില്‍ കള്ളം കെട്ടിച്ചമയ്ക്കരുത്‌. ഏതെങ്കിലും ഒരു ശിക്ഷ മുഖേന അവന്‍ നിങ്ങളെ ഉന്‍മൂലനം ചെയ്തേക്കും. കള്ളം കെട്ടിച്ചമച്ചവനാരോ അവന്‍ തീര്‍ച്ചയായും പരാജയപ്പെട്ടിരിക്കുന്നു.
\end{malayalam}}
\flushright{\begin{Arabic}
\quranayah[20][62]
\end{Arabic}}
\flushleft{\begin{malayalam}
(ഇത് കേട്ടപ്പോള്‍) അവര്‍ (ആളുകള്‍) തമ്മില്‍ അവരുടെ കാര്യത്തില്‍ ഭിന്നതയിലായി. അവര്‍ രഹസ്യസംഭാഷണത്തില്‍ ഏര്‍പെടുകയും ചെയ്തു.
\end{malayalam}}
\flushright{\begin{Arabic}
\quranayah[20][63]
\end{Arabic}}
\flushleft{\begin{malayalam}
(ചര്‍ച്ചയ്ക്ക് ശേഷം) അവര്‍ പറഞ്ഞു: തീര്‍ച്ചയായും ഇവര്‍ രണ്ടുപേരും ജാലവിദ്യക്കാര്‍ തന്നെയാണ്‌. അവരുടെ ജാലവിദ്യകൊണ്ട് നിങ്ങളുടെ നാട്ടില്‍ നിന്ന് നിങ്ങളെ പുറന്തള്ളാനും നിങ്ങളുടെ മാതൃകാപരമായ മാര്‍ഗത്തെ നശിപ്പിച്ചുകളയാനും അവര്‍ ഉദ്ദേശിക്കുന്നു.
\end{malayalam}}
\flushright{\begin{Arabic}
\quranayah[20][64]
\end{Arabic}}
\flushleft{\begin{malayalam}
അതിനാല്‍ നിങ്ങളുടെ തന്ത്രം നിങ്ങള്‍ ഏകകണ്ഠമായി തീരുമാനിക്കുകയും എന്നിട്ട് നിങ്ങള്‍ ഒരൊറ്റ അണിയായി (രംഗത്ത്‌) വരുകയും ചെയ്യുക. മികവ് നേടിയവരാരോ അവരാണ് ഇന്ന് വിജയികളായിരിക്കുക.
\end{malayalam}}
\flushright{\begin{Arabic}
\quranayah[20][65]
\end{Arabic}}
\flushleft{\begin{malayalam}
അവര്‍ (ജാലവിദ്യക്കാര്‍) പറഞ്ഞു: ഹേ; മൂസാ, ഒന്നുകില്‍ നീ ഇടുക. അല്ലെങ്കില്‍ ഞങ്ങളാകാം ആദ്യമായി ഇടുന്നവര്‍.
\end{malayalam}}
\flushright{\begin{Arabic}
\quranayah[20][66]
\end{Arabic}}
\flushleft{\begin{malayalam}
അദ്ദേഹം പറഞ്ഞു: അല്ല, നിങ്ങളിട്ട് കൊള്ളുക. അപ്പോഴതാ അവരുടെ ജാലവിദ്യ നിമിത്തം അവരുടെ കയറുകളും വടികളുമെല്ലാം ഓടുകയാണെന്ന് അദ്ദേഹത്തിന് തോന്നുന്നു.
\end{malayalam}}
\flushright{\begin{Arabic}
\quranayah[20][67]
\end{Arabic}}
\flushleft{\begin{malayalam}
അപ്പോള്‍ മൂസായ്ക്ക് തന്‍റെ മനസ്സില്‍ ഒരു പേടി തോന്നി.
\end{malayalam}}
\flushright{\begin{Arabic}
\quranayah[20][68]
\end{Arabic}}
\flushleft{\begin{malayalam}
നാം പറഞ്ഞു: പേടിക്കേണ്ട. തീര്‍ച്ചയായും നീ തന്നെയാണ് കൂടുതല്‍ ഔന്നത്യം നേടുന്നവന്‍.
\end{malayalam}}
\flushright{\begin{Arabic}
\quranayah[20][69]
\end{Arabic}}
\flushleft{\begin{malayalam}
നിന്‍റെ വലതുകയ്യിലുള്ളത് (വടി) നീ ഇട്ടേക്കുക. അവര്‍ ഉണ്ടാക്കിയതെല്ലാം അത് വിഴുങ്ങിക്കൊള്ളും. അവരുണ്ടാക്കിയത് ജാലവിദ്യക്കാരന്‍റെ തന്ത്രം മാത്രമാണ്‌. ജാലവിദ്യക്കാരന്‍ എവിടെച്ചെന്നാലും വിജയിയാവുകയില്ല.
\end{malayalam}}
\flushright{\begin{Arabic}
\quranayah[20][70]
\end{Arabic}}
\flushleft{\begin{malayalam}
ഉടനെ ആ ജാലവിദ്യക്കാര്‍ പ്രണമിച്ചുകൊണ്ട് താഴെ വീണു. അവര്‍ പറഞ്ഞു: ഞങ്ങള്‍ ഹാറൂന്‍റെയും മൂസായുടെയും രക്ഷിതാവില്‍ വിശ്വസിച്ചിരിക്കുന്നു.
\end{malayalam}}
\flushright{\begin{Arabic}
\quranayah[20][71]
\end{Arabic}}
\flushleft{\begin{malayalam}
അവന്‍ (ഫിര്‍ഔന്‍) പറഞ്ഞു: ഞാന്‍ നിങ്ങള്‍ക്ക് സമ്മതം തരുന്നതിന് മുമ്പ് നിങ്ങള്‍ അവനെ വിശ്വസിച്ച് കഴിഞ്ഞെന്നോ? തീര്‍ച്ചയായും നിങ്ങള്‍ക്ക് ജാലവിദ്യ പഠിപ്പിച്ചുതന്ന നിങ്ങളുടെ നേതാവ് തന്നെയാണവന്‍. ആകയാല്‍ തീര്‍ച്ചയായും ഞാന്‍ നിങ്ങളുടെ കൈകളും കാലുകളും എതിര്‍വശങ്ങളില്‍ നിന്നായി മുറിച്ചുകളയുകയും, ഈന്തപ്പനത്തടികളില്‍ നിങ്ങളെ ക്രൂശിക്കുകയും ചെയ്യുന്നതാണ്‌. ഞങ്ങളില്‍ ആരാണ് ഏറ്റവും കഠിനമായതും നീണ്ടുനില്‍ക്കുന്നതുമായ ശിക്ഷ നല്‍കുന്നവന്‍ എന്ന് തീര്‍ച്ചയായും നിങ്ങള്‍ക്ക് മനസ്സിലാകുകയും ചെയ്യും.
\end{malayalam}}
\flushright{\begin{Arabic}
\quranayah[20][72]
\end{Arabic}}
\flushleft{\begin{malayalam}
അവര്‍ പറഞ്ഞു: ഞങ്ങള്‍ക്ക് വന്നുകിട്ടിയ പ്രത്യക്ഷമായ തെളിവുകളെക്കാളും, ഞങ്ങളെ സൃഷ്ടിച്ചവനെക്കാളും നിനക്ക് ഞങ്ങള്‍ മുന്‍ഗണന നല്‍കുകയില്ല തന്നെ. അതിനാല്‍ നീ വിധിക്കുന്നതെന്തോ അത് വിധിച്ച് കൊള്ളുക. ഈ ഐഹികജീവിതത്തില്‍ മാത്രമേ നീ വിധിക്കുകയുള്ളൂ.
\end{malayalam}}
\flushright{\begin{Arabic}
\quranayah[20][73]
\end{Arabic}}
\flushleft{\begin{malayalam}
ഞങ്ങള്‍ ചെയ്ത പാപങ്ങളും, നീ ഞങ്ങളെ നിര്‍ബന്ധിച്ച് ചെയ്യിച്ച ജാലവിദ്യയും അവന്‍ ഞങ്ങള്‍ക്ക് പൊറുത്തുതരേണ്ടതിനായി ഞങ്ങള്‍ ഞങ്ങളുടെ രക്ഷിതാവില്‍ വിശ്വസിച്ചിരിക്കുന്നു. അല്ലാഹുവാണ് ഏറ്റവും ഉത്തമനും എന്നും നിലനില്‍ക്കുന്നവനും
\end{malayalam}}
\flushright{\begin{Arabic}
\quranayah[20][74]
\end{Arabic}}
\flushleft{\begin{malayalam}
തീര്‍ച്ചയായും വല്ലവനും കുറ്റവാളിയായിക്കൊണ്ട് തന്‍റെ രക്ഷിതാവിന്‍റെ അടുത്ത് ചെല്ലുന്ന പക്ഷം അവന്നുള്ളത് നരകമത്രെ. അതിലവന്‍ മരിക്കുകയില്ല.ജീവിക്കുകയുമില്ല.
\end{malayalam}}
\flushright{\begin{Arabic}
\quranayah[20][75]
\end{Arabic}}
\flushleft{\begin{malayalam}
സത്യവിശ്വാസിയായിക്കൊണ്ട് സല്‍കര്‍മ്മങ്ങള്‍ പ്രവര്‍ത്തിച്ചിട്ടാണ് വല്ലവനും അവന്‍റെയടുത്ത് ചെല്ലുന്നതെങ്കില്‍ അത്തരക്കാര്‍ക്കുള്ളതാകുന്നു ഉന്നതമായ പദവികള്‍.
\end{malayalam}}
\flushright{\begin{Arabic}
\quranayah[20][76]
\end{Arabic}}
\flushleft{\begin{malayalam}
അതായത് താഴ്ഭാഗത്ത് കൂടി നദികള്‍ ഒഴുകുന്ന, സ്ഥിരവാസത്തിനുള്ള സ്വര്‍ഗത്തോപ്പുകള്‍. അവരതില്‍ നിത്യവാസികളായിരിക്കും. അതാണ് പരിശുദ്ധി നേടിയവര്‍ക്കുള്ള പ്രതിഫലം.
\end{malayalam}}
\flushright{\begin{Arabic}
\quranayah[20][77]
\end{Arabic}}
\flushleft{\begin{malayalam}
മൂസായ്ക്ക് നാം ഇപ്രകാരം ബോധനം നല്‍കുകയുണ്ടായി: എന്‍റെ ദാസന്‍മാരെയും കൊണ്ട് രാത്രിയില്‍ നീ പോകുക. എന്നിട്ട് അവര്‍ക്ക് വേണ്ടി സമുദ്രത്തിലൂടെ ഒരു ഉണങ്ങിയ വഴി നീ ഏര്‍പെടുത്തികൊടുക്കുക. (ശത്രുക്കള്‍) പിന്തുടര്‍ന്ന് എത്തുമെന്ന് നീ പേടിക്കേണ്ടതില്ല. (യാതൊന്നും) നീ ഭയപ്പെടേണ്ടതുമില്ല.
\end{malayalam}}
\flushright{\begin{Arabic}
\quranayah[20][78]
\end{Arabic}}
\flushleft{\begin{malayalam}
അപ്പോള്‍ ഫിര്‍ഔന്‍ തന്‍റെ സൈന്യങ്ങളോട് കൂടി അവരുടെ പിന്നാലെ ചെന്നു.അപ്പോള്‍ കടലില്‍ നിന്ന് അവരെ ബാധിച്ചതെല്ലാം അവരെ ബാധിച്ചു.
\end{malayalam}}
\flushright{\begin{Arabic}
\quranayah[20][79]
\end{Arabic}}
\flushleft{\begin{malayalam}
ഫിര്‍ഔന്‍ തന്‍റെ ജനതയെ ദുര്‍മാര്‍ഗത്തിലാക്കി. അവന്‍ നേര്‍വഴിയിലേക്ക് നയിച്ചില്ല.
\end{malayalam}}
\flushright{\begin{Arabic}
\quranayah[20][80]
\end{Arabic}}
\flushleft{\begin{malayalam}
ഇസ്രായീല്‍ സന്തതികളേ, നിങ്ങളുടെ ശത്രുവില്‍ നിന്ന് നിങ്ങളെ നാം രക്ഷപ്പെടുത്തുകയും, ത്വൂര്‍ പര്‍വ്വതത്തിന്‍റെ വലതുഭാഗം നിങ്ങള്‍ക്ക് നാം നിശ്ചയിച്ച് തരികയും, മന്നായും സല്‍വായും നിങ്ങള്‍ക്ക് നാം ഇറക്കിത്തരികയും ചെയ്തു.
\end{malayalam}}
\flushright{\begin{Arabic}
\quranayah[20][81]
\end{Arabic}}
\flushleft{\begin{malayalam}
നിങ്ങള്‍ക്ക് നാം തന്നിട്ടുള്ള വിശിഷ്ടമായ വസ്തുക്കളില്‍ നിന്ന് നിങ്ങള്‍ ഭക്ഷിച്ച് കൊള്ളുക. അതില്‍ നിങ്ങള്‍ അതിരുകവിയരുത്‌. (നിങ്ങള്‍ അതിരുകവിയുന്ന പക്ഷം) എന്‍റെ കോപം നിങ്ങളുടെ മേല്‍ വന്നിറങ്ങുന്നതാണ്‌. എന്‍റെ കോപം ആരുടെമേല്‍ വന്നിറങ്ങുന്നുവോ അവന്‍ നാശത്തില്‍ പതിച്ചു.
\end{malayalam}}
\flushright{\begin{Arabic}
\quranayah[20][82]
\end{Arabic}}
\flushleft{\begin{malayalam}
പശ്ചാത്തപിക്കുകയും, വിശ്വസിക്കുകയും, സല്‍കര്‍മ്മം പ്രവര്‍ത്തിക്കുകയും, പിന്നെ നേര്‍മാര്‍ഗത്തില്‍ നിലകൊള്ളുകയും ചെയ്തവര്‍ക്ക് തീര്‍ച്ചയായും ഞാന്‍ ഏറെ പൊറുത്തുകൊടുക്കുന്നവനത്രെ.
\end{malayalam}}
\flushright{\begin{Arabic}
\quranayah[20][83]
\end{Arabic}}
\flushleft{\begin{malayalam}
(അല്ലാഹു ചോദിച്ചു:) ഹേ; മൂസാ, നിന്‍റെ ജനങ്ങളെ വിട്ടേച്ച് നീ ധൃതിപ്പെട്ട് വരാന്‍ കാരണമെന്താണ്‌?
\end{malayalam}}
\flushright{\begin{Arabic}
\quranayah[20][84]
\end{Arabic}}
\flushleft{\begin{malayalam}
അദ്ദേഹം പറഞ്ഞു: അവരിതാ എന്‍റെ പിന്നില്‍ തന്നെയുണ്ട്‌. എന്‍റെ രക്ഷിതാവേ, നീ തൃപ്തിപ്പെടുന്നതിന് വേണ്ടിയാണ് ഞാന്‍ നിന്‍റെ അടുത്തേക്ക് ധൃതിപ്പെട്ട് വന്നിരിക്കുന്നത്‌.
\end{malayalam}}
\flushright{\begin{Arabic}
\quranayah[20][85]
\end{Arabic}}
\flushleft{\begin{malayalam}
അവന്‍ (അല്ലാഹു) പറഞ്ഞു: എന്നാല്‍ നീ പോന്ന ശേഷം നിന്‍റെ ജനതയെ നാം പരീക്ഷിച്ചിരിക്കുന്നു. സാമിരി അവരെ വഴിതെറ്റിച്ച് കളഞ്ഞിരിക്കുന്നു.
\end{malayalam}}
\flushright{\begin{Arabic}
\quranayah[20][86]
\end{Arabic}}
\flushleft{\begin{malayalam}
അപ്പോള്‍ മൂസാ തന്‍റെ ജനങ്ങളുടെ അടുത്തേക്ക് കുപിതനും, ദുഃഖിതനുമായിക്കൊണ്ട് തിരിച്ചുചെന്നു. അദ്ദേഹം പറഞ്ഞു: എന്‍റെ ജനങ്ങളേ, നിങ്ങളുടെ രക്ഷിതാവ് നിങ്ങള്‍ക്ക് ഉത്തമമായ ഒരു വാഗ്ദാനം നല്‍കിയില്ലേ? എന്നിട്ട് നിങ്ങള്‍ക്ക് കാലം ദീര്‍ഘമായിപ്പോയോ? അഥവാ നിങ്ങളുടെ രക്ഷിതാവിങ്കല്‍ നിന്നുള്ള കോപം നിങ്ങളില്‍ ഇറങ്ങണമെന്ന് ആഗ്രഹിച്ച് കൊണ്ട് തന്നെ എന്നോടുള്ള നിശ്ചയം നിങ്ങള്‍ ലംഘിച്ചതാണോ?
\end{malayalam}}
\flushright{\begin{Arabic}
\quranayah[20][87]
\end{Arabic}}
\flushleft{\begin{malayalam}
അവര്‍ പറഞ്ഞു: ഞങ്ങള്‍ ഞങ്ങളുടെ ഹിതമനുസരിച്ച് താങ്കളോടുള്ള നിശ്ചയം ലംഘിച്ചതല്ല. എന്നാല്‍ ആ ജനങ്ങളുടെ ആഭരണചുമടുകള്‍ ഞങ്ങള്‍ വഹിപ്പിക്കപ്പെട്ടിരുന്നു. അങ്ങനെ ഞങ്ങളത് (തീയില്‍) എറിഞ്ഞുകളഞ്ഞു. അപ്പോള്‍ സാമിരിയും അപ്രകാരം അത് (തീയില്‍) ഇട്ടു.
\end{malayalam}}
\flushright{\begin{Arabic}
\quranayah[20][88]
\end{Arabic}}
\flushleft{\begin{malayalam}
എന്നിട്ട് അവര്‍ക്ക് അവന്‍ (ആ ലോഹം കൊണ്ട്‌) ഒരു മുക്രയിടുന്ന കാളക്കുട്ടിയുടെ രൂപം ഉണ്ടാക്കികൊടുത്തു. അപ്പോള്‍ അവര്‍ (അന്യോന്യം) പറഞ്ഞു: നിങ്ങളുടെ ദൈവവും മൂസായുടെ ദൈവവും ഇതുതന്നെയാണ്‌. എന്നാല്‍ അദ്ദേഹം മറന്നുപോയിരിക്കുകയാണ്‌.
\end{malayalam}}
\flushright{\begin{Arabic}
\quranayah[20][89]
\end{Arabic}}
\flushleft{\begin{malayalam}
എന്നാല്‍ അത് ഒരു വാക്ക് പോലും അവരോട് മറുപടി പറയുന്നില്ലെന്നും, അവര്‍ക്ക് യാതൊരു ഉപദ്രവവും ഉപകാരവും ചെയ്യാന്‍ അതിന് കഴിയില്ലെന്നും അവര്‍ കാണുന്നില്ലേ?
\end{malayalam}}
\flushright{\begin{Arabic}
\quranayah[20][90]
\end{Arabic}}
\flushleft{\begin{malayalam}
മുമ്പ് തന്നെ ഹാറൂന്‍ അവരോട് പറഞ്ഞിട്ടുണ്ടായിരുന്നു: എന്‍റെ ജനങ്ങളേ, ഇത് (കാളക്കുട്ടി) മൂലം നിങ്ങള്‍ പരീക്ഷിക്കപ്പെടുക മാത്രമാണുണ്ടായത്‌. തീര്‍ച്ചയായും നിങ്ങളുടെ രക്ഷിതാവ് പരമകാരുണികനത്രെ. അതുകൊണ്ട് നിങ്ങളെന്നെ പിന്തുടരുകയും,എന്‍റെ കല്‍പനകള്‍ നിങ്ങള്‍ അനുസരിക്കുകയും ചെയ്യുക.
\end{malayalam}}
\flushright{\begin{Arabic}
\quranayah[20][91]
\end{Arabic}}
\flushleft{\begin{malayalam}
അവര്‍ പറഞ്ഞു: മൂസാ ഞങ്ങളുടെ അടുത്തേക്ക് മടങ്ങിവരുവോളം ഞങ്ങള്‍ ഇതിനുള്ള ആരാധനയില്‍ നിരതരായി തന്നെയിരിക്കുന്നതാണ്‌.
\end{malayalam}}
\flushright{\begin{Arabic}
\quranayah[20][92]
\end{Arabic}}
\flushleft{\begin{malayalam}
അദ്ദേഹം (മൂസാ) പറഞ്ഞു: ഹാറൂനേ, ഇവര്‍ പിഴച്ചുപോയതായി നീ കണ്ടപ്പോള്‍ നിനക്ക് എന്ത് തടസ്സമാണുണ്ടായത്‌?
\end{malayalam}}
\flushright{\begin{Arabic}
\quranayah[20][93]
\end{Arabic}}
\flushleft{\begin{malayalam}
എന്നെ നീ പിന്തുടരാതിരിക്കാന്‍. നീ എന്‍റെ കല്‍പനയ്ക്ക് എതിര് പ്രവര്‍ത്തിക്കുകയാണോ ചെയ്തത്‌?
\end{malayalam}}
\flushright{\begin{Arabic}
\quranayah[20][94]
\end{Arabic}}
\flushleft{\begin{malayalam}
അദ്ദേഹം (ഹാറൂന്‍) പറഞ്ഞു: എന്‍റെ ഉമ്മയുടെ മകനേ, നീ എന്‍റെ താടിയിലും തലയിലും പിടിക്കാതിരിക്കൂ. ഇസ്രായീല്‍ സന്തതികള്‍ക്കിടയില്‍ നീ ഭിന്നിപ്പുണ്ടാക്കിക്കളഞ്ഞു, എന്‍റെ വാക്കിന് നീ കാത്തുനിന്നില്ല. എന്ന് നീ പറയുമെന്ന് ഞാന്‍ ഭയപ്പെടുകയാണുണ്ടായത്‌.
\end{malayalam}}
\flushright{\begin{Arabic}
\quranayah[20][95]
\end{Arabic}}
\flushleft{\begin{malayalam}
(തുടര്‍ന്ന് സാമിരിയോട്‌) അദ്ദേഹം പറഞ്ഞു: ഹേ; സാമിരീ, നിന്‍റെ കാര്യം എന്താണ്‌?
\end{malayalam}}
\flushright{\begin{Arabic}
\quranayah[20][96]
\end{Arabic}}
\flushleft{\begin{malayalam}
അവന്‍ പറഞ്ഞു: അവര്‍ (ജനങ്ങള്‍) കണ്ടുമനസ്സിലാക്കാത്ത ഒരു കാര്യം ഞാന്‍ കണ്ടുമനസ്സിലാക്കി. അങ്ങനെ ദൈവദൂതന്‍റെ കാല്‍പാടില്‍ നിന്ന് ഞാനൊരു പിടിപിടിക്കുകയും എന്നിട്ടത് ഇട്ടുകളയുകയും ചെയ്തു. അപ്രകാരം ചെയ്യാനാണ് എന്‍റെ മനസ്സ് എന്നെ പ്രേരിപ്പിച്ചത്‌.
\end{malayalam}}
\flushright{\begin{Arabic}
\quranayah[20][97]
\end{Arabic}}
\flushleft{\begin{malayalam}
അദ്ദേഹം (മൂസാ) പറഞ്ഞു: എന്നാല്‍ നീ പോ. തീര്‍ച്ചയായും നിനക്ക് ഈ ജീവിതത്തിലുള്ളത് തൊട്ടുകൂടാ എന്ന് പറയലായിരിക്കും. തീര്‍ച്ചയായും നിനക്ക് നിശ്ചിതമായ ഒരു അവധിയുണ്ട്‌. അത് അതിലംഘിക്കപ്പെടുകയേ ഇല്ല. നീ പൂജിച്ച് കൊണേ്ടയിരിക്കുന്ന നിന്‍റെ ആ ദൈവത്തിന്‍റെ നേരെ നോക്കൂ. തീര്‍ച്ചയായും നാം അതിനെ ചുട്ടെരിക്കുകയും, എന്നിട്ട് നാം അത് പൊടിച്ച് കടലില്‍ വിതറിക്കളയുകയും ചെയ്യുന്നതാണ്‌.
\end{malayalam}}
\flushright{\begin{Arabic}
\quranayah[20][98]
\end{Arabic}}
\flushleft{\begin{malayalam}
നിങ്ങളുടെ ദൈവം അല്ലാഹു മാത്രമാകുന്നു. അവനല്ലാതെ യാതൊരു ദൈവവുമില്ല. അവന്‍റെ അറിവ് എല്ലാകാര്യത്തേയും ഉള്‍കൊള്ളാന്‍ മാത്രം വിശാലമായിരിക്കുന്നു.
\end{malayalam}}
\flushright{\begin{Arabic}
\quranayah[20][99]
\end{Arabic}}
\flushleft{\begin{malayalam}
അപ്രകാരം മുമ്പ് കഴിഞ്ഞുപോയ സംഭവങ്ങളെപ്പറ്റിയുള്ള വൃത്താന്തങ്ങളില്‍ നിന്ന് നാം നിനക്ക് വിവരിച്ചുതരുന്നു. തീര്‍ച്ചയായും നാം നിനക്ക് നമ്മുടെ പക്കല്‍ നിന്നുള്ള ബോധനം നല്‍കിയിരിക്കുന്നു.
\end{malayalam}}
\flushright{\begin{Arabic}
\quranayah[20][100]
\end{Arabic}}
\flushleft{\begin{malayalam}
ആരെങ്കിലും അതില്‍ നിന്ന് തിരിഞ്ഞുകളയുന്ന പക്ഷം തീര്‍ച്ചയായും ഉയിര്‍ത്തെഴുന്നേല്‍പിന്‍റെ നാളില്‍ അവന്‍ (പാപത്തിന്‍റെ) ഭാരം വഹിക്കുന്നതാണ്‌.
\end{malayalam}}
\flushright{\begin{Arabic}
\quranayah[20][101]
\end{Arabic}}
\flushleft{\begin{malayalam}
അതില്‍ അവര്‍ നിത്യവാസികളായിരിക്കും. ഉയിര്‍ത്തെഴുന്നേല്‍പിന്‍റെ നാളില്‍ ആ ഭാരം അവര്‍ക്കെത്ര ദുസ്സഹം!
\end{malayalam}}
\flushright{\begin{Arabic}
\quranayah[20][102]
\end{Arabic}}
\flushleft{\begin{malayalam}
അതായത് കാഹളത്തില്‍ ഈതപ്പെടുന്ന ദിവസം. കുറ്റവാളികളെ അന്നേദിവസം നീലവര്‍ണമുള്ളവരായിക്കൊണ്ട് നാം ഒരുമിച്ചുകൂട്ടുന്നതാണ്‌.
\end{malayalam}}
\flushright{\begin{Arabic}
\quranayah[20][103]
\end{Arabic}}
\flushleft{\begin{malayalam}
അവര്‍ അന്യോന്യം പതുക്കെ പറയും: പത്ത് ദിവസമല്ലാതെ നിങ്ങള്‍ ഭൂമിയില്‍ താമസിക്കുകയുണ്ടായിട്ടില്ല എന്ന്‌.
\end{malayalam}}
\flushright{\begin{Arabic}
\quranayah[20][104]
\end{Arabic}}
\flushleft{\begin{malayalam}
അവരില്‍ ഏറ്റവും ന്യായമായ നിലപാടുകാരന്‍ ഒരൊറ്റ ദിവസം മാത്രമേ നിങ്ങള്‍ താമസിച്ചിട്ടുള്ളു എന്ന് പറയുമ്പോള്‍ അവര്‍ പറയുന്നതിനെപ്പറ്റി നാം നല്ലവണ്ണം അറിയുന്നവനാകുന്നു.
\end{malayalam}}
\flushright{\begin{Arabic}
\quranayah[20][105]
\end{Arabic}}
\flushleft{\begin{malayalam}
പര്‍വ്വതങ്ങളെ സംബന്ധിച്ച് അവര്‍ നിന്നോട് ചോദിക്കുന്നു. പറയുക: എന്‍റെ രക്ഷിതാവ് അവയെ പൊടിച്ച് പാറ്റിക്കളയുന്നതാണ്‌.
\end{malayalam}}
\flushright{\begin{Arabic}
\quranayah[20][106]
\end{Arabic}}
\flushleft{\begin{malayalam}
എന്നിട്ട് അവന്‍ അതിനെ സമനിരപ്പായ മൈതാനമാക്കി വിടുന്നതാണ്‌.
\end{malayalam}}
\flushright{\begin{Arabic}
\quranayah[20][107]
\end{Arabic}}
\flushleft{\begin{malayalam}
ഇറക്കമോ കയറ്റമോ നീ അവിടെ കാണുകയില്ല.
\end{malayalam}}
\flushright{\begin{Arabic}
\quranayah[20][108]
\end{Arabic}}
\flushleft{\begin{malayalam}
അന്നേ ദിവസം വിളിക്കുന്നവന്‍റെ പിന്നാലെ അവനോട് യാതൊരു വക്രതയും കാണിക്കാതെ അവര്‍ പോകുന്നതാണ്‌. എല്ലാ ശബ്ദങ്ങളും പരമകാരുണികന് കീഴടങ്ങുന്നതുമാണ്‌. അതിനാല്‍ ഒരു നേര്‍ത്ത ശബ്ദമല്ലാതെ നീ കേള്‍ക്കുകയില്ല.
\end{malayalam}}
\flushright{\begin{Arabic}
\quranayah[20][109]
\end{Arabic}}
\flushleft{\begin{malayalam}
അന്നേ ദിവസം പരമകാരുണികന്‍ ആരുടെ കാര്യത്തില്‍ അനുമതി നല്‍കുകയും ആരുടെ വാക്ക് തൃപ്തിപ്പെടുകയും ചെയ്തിരിക്കുന്നുവോ അവന്നല്ലാതെ ശുപാര്‍ശ പ്രയോജനപ്പെടുകയില്ല.
\end{malayalam}}
\flushright{\begin{Arabic}
\quranayah[20][110]
\end{Arabic}}
\flushleft{\begin{malayalam}
അവരുടെ മുമ്പിലുള്ളതും പിന്നിലുള്ളതും അവന്‍ അറിയുന്നു. അവര്‍ക്കാകട്ടെ അതിനെപ്പറ്റിയൊന്നും പരിപൂര്‍ണ്ണമായി അറിയാനാവുകയില്ല.
\end{malayalam}}
\flushright{\begin{Arabic}
\quranayah[20][111]
\end{Arabic}}
\flushleft{\begin{malayalam}
എന്നെന്നും ജീവിച്ചിരിക്കുന്നവനും എല്ലാം നിയന്ത്രിക്കുന്നവനും ആയുള്ളവന്ന് മുഖങ്ങള്‍ കീഴൊതുങ്ങിയിരിക്കുന്നു. അക്രമത്തിന്‍റെ ഭാരം ചുമന്നവന്‍ പരാജയമടയുകയും ചെയ്തിരിക്കുന്നു.
\end{malayalam}}
\flushright{\begin{Arabic}
\quranayah[20][112]
\end{Arabic}}
\flushleft{\begin{malayalam}
ആരെങ്കിലും സത്യവിശ്വാസിയായിക്കൊണ്ട് സല്‍കര്‍മ്മങ്ങളില്‍ വല്ലതും പ്രവര്‍ത്തിക്കുന്ന പക്ഷം അവന്‍ അക്രമത്തെയോ അനീതിയെയോ ഭയപ്പെടേണ്ടി വരില്ല.
\end{malayalam}}
\flushright{\begin{Arabic}
\quranayah[20][113]
\end{Arabic}}
\flushleft{\begin{malayalam}
അപ്രകാരം അറബിയില്‍ പാരായണം ചെയ്യപ്പെടുന്ന ഒരു ഗ്രന്ഥമായി നാം ഇതിനെ അവതരിപ്പിച്ചിരിക്കുന്നു. ഇതില്‍ നാം താക്കീത് വിവിധ തരത്തില്‍ വിവരിച്ചിരിക്കുന്നു. അവര്‍ സൂക്ഷ്മതയുള്ളവരാകുകയോ, അവര്‍ക്ക് ബോധമുളവാക്കുകയോ ചെയ്യുന്നതിനുവേണ്ടി.
\end{malayalam}}
\flushright{\begin{Arabic}
\quranayah[20][114]
\end{Arabic}}
\flushleft{\begin{malayalam}
സാക്ഷാല്‍ രാജാവായ അല്ലാഹു അത്യുന്നതനായിരിക്കുന്നു. ഖുര്‍ആന്‍- അത് നിനക്ക് ബോധനം നല്‍കപ്പെട്ടുകഴിയുന്നതിനുമുമ്പായി - പാരായണം ചെയ്യുന്നതിനു നീ ധൃതി കാണിക്കരുത്‌. എന്‍റെ രക്ഷിതാവേ, എനിക്കു നീ ജ്ഞാനം വര്‍ദ്ധിപ്പിച്ചു തരേണമേ എന്ന് നീ പറയുകയും ചെയ്യുക.
\end{malayalam}}
\flushright{\begin{Arabic}
\quranayah[20][115]
\end{Arabic}}
\flushleft{\begin{malayalam}
മുമ്പ് നാം ആദമിനോട് കരാര്‍ ചെയ്യുകയുണ്ടായി. എന്നാല്‍ അദ്ദേഹം അതു മറന്നുകളഞ്ഞു. അദ്ദേഹത്തിന് നിശ്ചയദാര്‍ഢ്യമുള്ളതായി നാം കണ്ടില്ല.
\end{malayalam}}
\flushright{\begin{Arabic}
\quranayah[20][116]
\end{Arabic}}
\flushleft{\begin{malayalam}
നിങ്ങള്‍ ആദമിന് സുജൂദ് ചെയ്യൂ എന്ന് നാം മലക്കുകളോട് പറഞ്ഞ സന്ദര്‍ഭം (ശ്രദ്ധേയമത്രെ.) അപ്പോള്‍ അവര്‍ സുജൂദ് ചെയ്തു. ഇബ്ലീസൊഴികെ. അവന്‍ വിസമ്മതിച്ചു.
\end{malayalam}}
\flushright{\begin{Arabic}
\quranayah[20][117]
\end{Arabic}}
\flushleft{\begin{malayalam}
അപ്പോള്‍ നാം പറഞ്ഞു: ആദമേ, തീര്‍ച്ചയായും ഇവന്‍ നിന്‍റെയും നിന്‍റെ ഇണയുടെയും ശത്രുവാകുന്നു. അതിനാല്‍ നിങ്ങളെ രണ്ട് പേരെയും അവന്‍ സ്വര്‍ഗത്തില്‍ നിന്ന് പുറം തള്ളാതിരിക്കട്ടെ (അങ്ങനെ സംഭവിക്കുന്ന പക്ഷം) നീ കഷ്ടപ്പെടും.
\end{malayalam}}
\flushright{\begin{Arabic}
\quranayah[20][118]
\end{Arabic}}
\flushleft{\begin{malayalam}
തീര്‍ച്ചയായും നിനക്ക് ഇവിടെ വിശക്കാതെയും നഗ്നനാകാതെയും കഴിയാം.
\end{malayalam}}
\flushright{\begin{Arabic}
\quranayah[20][119]
\end{Arabic}}
\flushleft{\begin{malayalam}
നിനക്കിവിടെ ദാഹിക്കാതെയും വെയിലുകൊള്ളാതെയും കഴിയാം.
\end{malayalam}}
\flushright{\begin{Arabic}
\quranayah[20][120]
\end{Arabic}}
\flushleft{\begin{malayalam}
അപ്പോള്‍ പിശാച് അദ്ദേഹത്തിന് ദുര്‍ബോധനം നല്‍കി: ആദമേ, അനശ്വരത നല്‍കുന്ന ഒരു വൃക്ഷത്തെപ്പറ്റിയും, ക്ഷയിച്ച് പോകാത്ത ആധിപത്യത്തെപ്പറ്റിയും ഞാന്‍ നിനക്ക് അറിയിച്ച് തരട്ടെയോ?
\end{malayalam}}
\flushright{\begin{Arabic}
\quranayah[20][121]
\end{Arabic}}
\flushleft{\begin{malayalam}
അങ്ങനെ അവര്‍ (ആദമും ഭാര്യയും) ആ വൃക്ഷത്തില്‍ നിന്ന് ഭക്ഷിച്ചു. അപ്പോള്‍ അവര്‍ ഇരുവര്‍ക്കും തങ്ങളുടെ ഗുഹ്യഭാഗങ്ങള്‍ വെളിപ്പെടുകയും, സ്വര്‍ഗത്തിലെ ഇലകള്‍ കൂട്ടിചേര്‍ത്ത് തങ്ങളുടെ ദേഹം അവര്‍ പൊതിയാന്‍ തുടങ്ങുകയും ചെയ്തു. ആദം തന്‍റെ രക്ഷിതാവിനോട് അനുസരണക്കേട് കാണിക്കുകയും, അങ്ങനെ പിഴച്ച് പോകുകയും ചെയ്തു.
\end{malayalam}}
\flushright{\begin{Arabic}
\quranayah[20][122]
\end{Arabic}}
\flushleft{\begin{malayalam}
അനന്തരം അദ്ദേഹത്തിന്‍റെ രക്ഷിതാവ് അദ്ദേഹത്തെ ഉല്‍കൃഷ്ടനായി തെരഞ്ഞെടുക്കുകയും, അദ്ദേഹത്തിന്‍റെ പശ്ചാത്താപം സ്വീകരിക്കുകയും, മാര്‍ഗദര്‍ശനം നല്‍കുകയും ചെയ്തു.
\end{malayalam}}
\flushright{\begin{Arabic}
\quranayah[20][123]
\end{Arabic}}
\flushleft{\begin{malayalam}
അവന്‍ (അല്ലാഹു) പറഞ്ഞു: നിങ്ങള്‍ രണ്ട് പേരും ഒന്നിച്ച് ഇവിടെ നിന്ന് ഇറങ്ങിപ്പോകുകണിങ്ങളില്‍ ചിലര്‍ ചിലര്‍ക്ക് ശത്രുക്കളാകുന്നു. എന്നാല്‍ എന്‍റെ പക്കല്‍ നിന്നുള്ള വല്ല മാര്‍ഗദര്‍ശനവും നിങ്ങള്‍ക്ക് വന്നുകിട്ടുന്ന പക്ഷം, അപ്പോള്‍ എന്‍റെ മാര്‍ഗദര്‍ശനം ആര്‍ പിന്‍പറ്റുന്നുവോ അവന്‍ പിഴച്ച് പോകുകയില്ല, കഷ്ടപ്പെടുകയുമില്ല.
\end{malayalam}}
\flushright{\begin{Arabic}
\quranayah[20][124]
\end{Arabic}}
\flushleft{\begin{malayalam}
എന്‍റെ ഉല്‍ബോധനത്തെ വിട്ട് വല്ലവനും തിരിഞ്ഞുകളയുന്ന പക്ഷം തീര്‍ച്ചയായും അവന്ന് ഇടുങ്ങിയ ഒരു ജീവിതമാണുണ്ടായിരിക്കുക. ഉയിര്‍ത്തെഴുന്നേല്‍പിന്‍റെ നാളില്‍ അവനെ നാം അന്ധനായ നിലയില്‍ എഴുന്നേല്‍പിച്ച് കൊണ്ട് വരുന്നതുമാണ്‌.
\end{malayalam}}
\flushright{\begin{Arabic}
\quranayah[20][125]
\end{Arabic}}
\flushleft{\begin{malayalam}
അവന്‍ പറയും: എന്‍റെ രക്ഷിതാവേ, നീ എന്തിനാണെന്നെ അന്ധനായ നിലയില്‍ എഴുന്നേല്‍പിച്ച് കൊണ്ട് വന്നത്‌? ഞാന്‍ കാഴ്ചയുള്ളവനായിരുന്നല്ലോ!
\end{malayalam}}
\flushright{\begin{Arabic}
\quranayah[20][126]
\end{Arabic}}
\flushleft{\begin{malayalam}
അല്ലാഹു പറയും: അങ്ങനെതന്നെയാകുന്നു. നിനക്ക് നമ്മുടെ ദൃഷ്ടാന്തങ്ങള്‍ വന്നെത്തുകയുണ്ടായി. എന്നിട്ട് നീ അത് മറന്നുകളഞ്ഞു. അത് പോലെ ഇന്ന് നീയും വിസ്മരിക്കപ്പെടുന്നു.
\end{malayalam}}
\flushright{\begin{Arabic}
\quranayah[20][127]
\end{Arabic}}
\flushleft{\begin{malayalam}
അതിരുകവിയുകയും, തന്‍റെ രക്ഷിതാവിന്‍റെ ദൃഷ്ടാന്തങ്ങളില്‍ വിശ്വസിക്കാതിരിക്കുകയും ചെയ്തവര്‍ക്ക് അപ്രകാരമാണ് നാം പ്രതിഫലം നല്‍കുന്നത്‌. പരലോകത്തെ ശിക്ഷ കൂടുതല്‍ കഠിനമായതും നിലനില്‍ക്കുന്നതും തന്നെയാകുന്നു.
\end{malayalam}}
\flushright{\begin{Arabic}
\quranayah[20][128]
\end{Arabic}}
\flushleft{\begin{malayalam}
അവര്‍ക്ക് മുമ്പ് നാം എത്രയോ തലമുറകളെ നശിപ്പിച്ച് കളഞ്ഞിട്ടുണ്ട് എന്ന വസ്തുത അവര്‍ക്ക് മാര്‍ഗദര്‍ശകമായിട്ടില്ലേ ? അവരുടെ വാസസ്ഥലങ്ങളില്‍ കൂടി ഇവര്‍ സഞ്ചരിച്ച് കൊണ്ടിരിക്കുന്നുണ്ട്‌. ബുദ്ധിമാന്‍മാര്‍ക്ക് തീര്‍ച്ചയായും അതില്‍ ദൃഷ്ടാന്തങ്ങളുണ്ട്‌.
\end{malayalam}}
\flushright{\begin{Arabic}
\quranayah[20][129]
\end{Arabic}}
\flushleft{\begin{malayalam}
നിന്‍റെ രക്ഷിതാവിങ്കല്‍ നിന്ന് ഒരു വാക്കും നിശ്ചിതമായ ഒരു അവധിയും മുമ്പേ പ്രഖ്യാപിക്കപ്പെട്ടില്ലായിരുന്നുവെങ്കില്‍ അത് (ശിക്ഷാനടപടി ഇവര്‍ക്കും) അനിവാര്യമാകുമായിരുന്നു.
\end{malayalam}}
\flushright{\begin{Arabic}
\quranayah[20][130]
\end{Arabic}}
\flushleft{\begin{malayalam}
ആയതിനാല്‍ ഇവര്‍ പറയുന്നതിനെ പറ്റി ക്ഷമിക്കുക. സൂര്യോദയത്തിനു മുമ്പും, സൂര്യാസ്തമയത്തിന് മുമ്പും നിന്‍റെ രക്ഷിതാവിനെ സ്തുതിക്കുന്നതോടൊപ്പം അവന്‍റെ പരിശുദ്ധിയെ നീ പ്രകീര്‍ത്തിക്കുകയും ചെയ്യുക. രാത്രിയില്‍ ചില നാഴികകളിലും, പകലിന്‍റെ ചില ഭാഗങ്ങളിലും നീ അവന്‍റെ പരിശുദ്ധിയെ പ്രകീര്‍ത്തിക്കുക. നിനക്ക് സംതൃപ്തി കൈവന്നേക്കാം.
\end{malayalam}}
\flushright{\begin{Arabic}
\quranayah[20][131]
\end{Arabic}}
\flushleft{\begin{malayalam}
അവരില്‍ (മനുഷ്യരില്‍) പല വിഭാഗങ്ങള്‍ക്ക് നാം ഐഹികജീവിതാലങ്കാരം അനുഭവിപ്പിച്ചതിലേക്ക് നിന്‍റെ ദൃഷ്ടികള്‍ നീ പായിക്കരുത്‌. അതിലൂടെ നാം അവരെ പരീക്ഷിക്കാന്‍ (ഉദ്ദേശിക്കുന്നു.) നിന്‍റെ രക്ഷിതാവ് നല്‍കുന്ന ഉപജീവനമാകുന്നു കൂടുതല്‍ ഉത്തമവും നിലനില്‍ക്കുന്നതും.
\end{malayalam}}
\flushright{\begin{Arabic}
\quranayah[20][132]
\end{Arabic}}
\flushleft{\begin{malayalam}
നിന്‍റെ കുടുംബത്തോട് നീ നമസ്കരിക്കാന്‍ കല്‍പിക്കുകയും, അതില്‍(നമസ്കാരത്തില്‍) നീ ക്ഷമാപൂര്‍വ്വം ഉറച്ചുനില്‍ക്കുകയും ചെയ്യുക. നിന്നോട് നാം ഉപജീവനം ചോദിക്കുന്നില്ല. നാം നിനക്ക് ഉപജീവനം നല്‍കുകയാണ് ചെയ്യുന്നത്‌. ധര്‍മ്മനിഷ്ഠയ്ക്കാകുന്നു ശുഭപര്യവസാനം.
\end{malayalam}}
\flushright{\begin{Arabic}
\quranayah[20][133]
\end{Arabic}}
\flushleft{\begin{malayalam}
അവര്‍ പറഞ്ഞു: അദ്ദേഹം (പ്രവാചകന്‍) എന്തുകൊണ്ട് ഞങ്ങള്‍ക്ക് തന്‍റെ രക്ഷിതാവിങ്കല്‍ നിന്ന് ഒരു ദൃഷ്ടാന്തം കൊണ്ട് വന്ന് തരുന്നില്ല? പൂര്‍വ്വഗ്രന്ഥങ്ങളിലെ പ്രത്യക്ഷമായ തെളിവ് അവര്‍ക്ക് വന്നുകിട്ടിയില്ലേ?
\end{malayalam}}
\flushright{\begin{Arabic}
\quranayah[20][134]
\end{Arabic}}
\flushleft{\begin{malayalam}
ഇതിനു മുമ്പ് വല്ല ശിക്ഷ കൊണ്ടും നാം അവരെ നശിപ്പിച്ചിരുന്നുവെങ്കില്‍ അവര്‍ പറയുമായിരുന്നു: ഞങ്ങളുടെ രക്ഷിതാവേ, നീ എന്തുകൊണ്ട് ഞങ്ങളുടെ അടുത്തേക്ക് ഒരു ദൂതനെ അയച്ചുതന്നില്ല? എങ്കില്‍ ഞങ്ങള്‍ അപമാനിതരും നിന്ദിതരുമായിത്തീരുന്നതിന് മുമ്പ് നിന്‍റെ ദൃഷ്ടാന്തങ്ങളെ ഞങ്ങള്‍ പിന്തുടരുമായിരുന്നു.
\end{malayalam}}
\flushright{\begin{Arabic}
\quranayah[20][135]
\end{Arabic}}
\flushleft{\begin{malayalam}
(നബിയേ,) പറയുക: എല്ലാവരും കാത്തിരിക്കുന്നവരാകുന്നു. നിങ്ങളും കാത്തിരിക്കുക. നേരായ പാതയുടെ ഉടമകളാരെന്നും സന്‍മാര്‍ഗം പ്രാപിച്ചവരാരെന്നും അപ്പോള്‍ നിങ്ങള്‍ക്ക് അറിയാറാകും.
\end{malayalam}}
\chapter{\textmalayalam{അന്‍ബിയാഅ് ( പ്രവാചകന്മാര്‍ )}}
\begin{Arabic}
\Huge{\centerline{\basmalah}}\end{Arabic}
\flushright{\begin{Arabic}
\quranayah[21][1]
\end{Arabic}}
\flushleft{\begin{malayalam}
ജനങ്ങള്‍ക്ക് അവരുടെ വിചാരണ ആസന്നമായിരിക്കുന്നു. അവരാകട്ടെ അശ്രദ്ധയിലായിക്കൊണ്ട് തിരിഞ്ഞുകളയുന്നവരാകുന്നു.
\end{malayalam}}
\flushright{\begin{Arabic}
\quranayah[21][2]
\end{Arabic}}
\flushleft{\begin{malayalam}
അവരുടെ രക്ഷിതാവിങ്കല്‍ നിന്ന് പുതുതായി ഏതൊരു ഉല്‍ബോധനം അവര്‍ക്ക് വന്നെത്തിയാലും കളിയാക്കുന്നവരായിക്കൊണ്ട് മാത്രമേ അവരത് കേള്‍ക്കുകയുള്ളൂ.
\end{malayalam}}
\flushright{\begin{Arabic}
\quranayah[21][3]
\end{Arabic}}
\flushleft{\begin{malayalam}
ഹൃദയങ്ങള്‍ അശ്രദ്ധമായിക്കൊണ്ട് (അവരിലെ) അക്രമികള്‍ അന്യോന്യം രഹസ്യമായി ഇപ്രകാരം മന്ത്രിച്ചു; നിങ്ങളെപ്പോലെയുള്ള ഒരു മനുഷ്യന്‍ മാത്രമല്ലേ ഇത്‌? എന്നിട്ട് നിങ്ങള്‍ കണ്ടറിഞ്ഞ് കൊണ്ട് തന്നെ ഈ ജാലവിദ്യയുടെ അടുത്തേക്ക് ചെല്ലുകയാണോ?
\end{malayalam}}
\flushright{\begin{Arabic}
\quranayah[21][4]
\end{Arabic}}
\flushleft{\begin{malayalam}
അദ്ദേഹം (നബി) പറഞ്ഞു: എന്‍റെ രക്ഷിതാവ് ആകാശത്തും ഭൂമിയിലും പറയപ്പെടുന്നതെല്ലാം അറിയുന്നു. അവനാണ് എല്ലാം കേള്‍ക്കുന്നവനും അറിയുന്നവനും.
\end{malayalam}}
\flushright{\begin{Arabic}
\quranayah[21][5]
\end{Arabic}}
\flushleft{\begin{malayalam}
എന്നാല്‍ അവര്‍ പറഞ്ഞു: പാഴ്കിനാവുകള്‍ കണ്ട വിവരമാണ് (മുഹമ്മദ് പറയുന്നത്‌) (മറ്റൊരിക്കല്‍ അവര്‍ പറഞ്ഞു:) അല്ല, അതവന്‍ കെട്ടിച്ചമച്ചുണ്ടാക്കിയതാണ്‌. (മറ്റൊരിക്കല്‍ അവര്‍ പറഞ്ഞു:) അല്ല; അവനൊരു കവിയാണ്‌. എന്നാല്‍ (അവന്‍ പ്രവാചകനാണെങ്കില്‍) മുന്‍ പ്രവാചകന്‍മാര്‍ ഏതൊരു ദൃഷ്ടാന്തവുമായാണോ അയക്കപ്പെട്ടത് അതുപോലൊന്ന് അവന്‍ നമുക്ക് കൊണ്ട് വന്നു കാണിക്കട്ടെ.
\end{malayalam}}
\flushright{\begin{Arabic}
\quranayah[21][6]
\end{Arabic}}
\flushleft{\begin{malayalam}
ഇവരുടെ മുമ്പ് നാം നശിപ്പിച്ച ഒരു നാട്ടുകാരും വിശ്വസിക്കുകയുണ്ടായില്ല. എന്നിരിക്കെ ഇവര്‍ വിശ്വസിക്കുമോ ?
\end{malayalam}}
\flushright{\begin{Arabic}
\quranayah[21][7]
\end{Arabic}}
\flushleft{\begin{malayalam}
നിനക്ക് മുമ്പ് പുരുഷന്‍മാരെ (ആളുകളെ) യല്ലാതെ നാം ദൂതന്‍മാരായി നിയോഗിച്ചിട്ടില്ല. അവര്‍ക്ക് നാം ബോധനം നല്‍കുന്നു. നിങ്ങള്‍ (ഈ കാര്യം) അറിയാത്തവരാണെങ്കില്‍ വേദക്കാരോട് ചോദിച്ച് നോക്കുക.
\end{malayalam}}
\flushright{\begin{Arabic}
\quranayah[21][8]
\end{Arabic}}
\flushleft{\begin{malayalam}
അവരെ (പ്രവാചകന്‍മാരെ) നാം ഭക്ഷണം കഴിക്കാത്ത ശരീരങ്ങളാക്കിയിട്ടില്ല. അവര്‍ നിത്യജീവികളായിരുന്നതുമില്ല.
\end{malayalam}}
\flushright{\begin{Arabic}
\quranayah[21][9]
\end{Arabic}}
\flushleft{\begin{malayalam}
അനന്തരം അവരോടുള്ള വാഗ്ദാനത്തില്‍ നാം സത്യസന്ധത പാലിച്ചു. അങ്ങനെ അവരെയും നാം ഉദ്ദേശിക്കുന്നവരെയും നാം രക്ഷപ്പെടുത്തി. അതിരുകവിഞ്ഞവരെ നാം നശിപ്പിക്കുകയും ചെയ്തു.
\end{malayalam}}
\flushright{\begin{Arabic}
\quranayah[21][10]
\end{Arabic}}
\flushleft{\begin{malayalam}
തീര്‍ച്ചയായും നിങ്ങള്‍ക്ക് നാം ഒരു ഗ്രന്ഥം അവതരിപ്പിച്ച് തന്നിട്ടുണ്ട്‌. നിങ്ങള്‍ക്കുള്ള ഉല്‍ബോധനം അതിലുണ്ട്‌. എന്നിട്ടും നിങ്ങള്‍ ചിന്തിക്കുന്നില്ലേ?
\end{malayalam}}
\flushright{\begin{Arabic}
\quranayah[21][11]
\end{Arabic}}
\flushleft{\begin{malayalam}
അക്രമത്തില്‍ ഏര്‍പെട്ടിരുന്ന എത്ര നാടുകളെ നാം നിശ്ശേഷം തകര്‍ത്തുകളയുകയും, അതിന് ശേഷം നാം മറ്റൊരു ജനവിഭാഗത്തെ വളര്‍ത്തിയെടുക്കുകയും ചെയ്തിട്ടുണ്ട്‌.!
\end{malayalam}}
\flushright{\begin{Arabic}
\quranayah[21][12]
\end{Arabic}}
\flushleft{\begin{malayalam}
അങ്ങനെ നമ്മുടെ ശിക്ഷ അവര്‍ക്ക് അനുഭവപ്പെട്ടപ്പോള്‍ അവരതാ അവിടെനിന്ന് ഓടിരക്ഷപ്പെടാന്‍ നോക്കുന്നു.
\end{malayalam}}
\flushright{\begin{Arabic}
\quranayah[21][13]
\end{Arabic}}
\flushleft{\begin{malayalam}
(അപ്പോള്‍ അവരോട് പറയപ്പെട്ടു.) നിങ്ങള്‍ ഓടിപ്പോകേണ്ട. നിങ്ങള്‍ക്ക് നല്‍കപ്പെട്ട സുഖാഡംബരങ്ങളിലേക്കും, നിങ്ങളുടെ വസതികളിലേക്കും നിങ്ങള്‍ തിരിച്ചുപോയിക്കൊള്ളുക. നിങ്ങള്‍ക്ക് വല്ല അപേക്ഷയും നല്‍കപ്പെടാനുണ്ടായേക്കാം.
\end{malayalam}}
\flushright{\begin{Arabic}
\quranayah[21][14]
\end{Arabic}}
\flushleft{\begin{malayalam}
അവര്‍ പറഞ്ഞു: അയ്യോ; ഞങ്ങള്‍ക്ക് നാശം! തീര്‍ച്ചയായും ഞങ്ങള്‍ അക്രമികളായിപ്പോയി.
\end{malayalam}}
\flushright{\begin{Arabic}
\quranayah[21][15]
\end{Arabic}}
\flushleft{\begin{malayalam}
അങ്ങനെ അവരെ നാം കൊയ്തിട്ട വിള പോലെ ചലനമറ്റ നിലയിലാക്കിത്തീര്‍ക്കുവോളം അവരുടെ മുറവിളി അതു തന്നെയായിക്കൊണ്ടിരുന്നു.
\end{malayalam}}
\flushright{\begin{Arabic}
\quranayah[21][16]
\end{Arabic}}
\flushleft{\begin{malayalam}
ആകാശത്തെയും, ഭൂമിയെയും, അവ രണ്ടിനുമിടയിലുള്ളതിനെയും നാം കളിയായിക്കൊണ്ട് സൃഷ്ടിച്ചതല്ല.
\end{malayalam}}
\flushright{\begin{Arabic}
\quranayah[21][17]
\end{Arabic}}
\flushleft{\begin{malayalam}
നാം ഒരു വിനോദമുണ്ടാക്കാന്‍ ഉദ്ദേശിച്ചിരുന്നുവെങ്കില്‍ നമ്മുടെ അടുക്കല്‍ നിന്നു തന്നെ നാമത് ഉണ്ടാക്കുമായിരുന്നു. (എന്നാല്‍) നാം (അത്‌) ചെയ്യുന്നതല്ല.
\end{malayalam}}
\flushright{\begin{Arabic}
\quranayah[21][18]
\end{Arabic}}
\flushleft{\begin{malayalam}
എന്നാല്‍ നാം സത്യത്തെ എടുത്ത് അസത്യത്തിന്‍റെ നേര്‍ക്ക് എറിയുന്നു. അങ്ങനെ അസത്യത്തെ അത് തകര്‍ത്ത് കളയുന്നു. അതോടെ അസത്യം നാശമടയുകയായി. നിങ്ങള്‍ (അല്ലാഹുവെപ്പറ്റി) പറഞ്ഞുണ്ടാക്കുന്നത് നിമിത്തം നിങ്ങള്‍ക്ക് നാശം.
\end{malayalam}}
\flushright{\begin{Arabic}
\quranayah[21][19]
\end{Arabic}}
\flushleft{\begin{malayalam}
അവന്റേതാകുന്നു ആകാശങ്ങളിലും, ഭൂമിയിയും ഉള്ളവരെല്ലാം. അവന്‍റെ അടുക്കലുള്ളവര്‍ (മലക്കുകള്‍) അവനെ ആരാധിക്കുന്നത് വിട്ട് അഹങ്കരിക്കുകയില്ല. അവര്‍ക്ക് ക്ഷീണം തോന്നുകയുമില്ല.
\end{malayalam}}
\flushright{\begin{Arabic}
\quranayah[21][20]
\end{Arabic}}
\flushleft{\begin{malayalam}
അവര്‍ രാവും പകലും (അല്ലാഹുവിന്‍റെ പരിശുദ്ധിയെ) പ്രകീര്‍ത്തിച്ചു കൊണ്ടിരിക്കുന്നു. അവര്‍ തളരുകയില്ല.
\end{malayalam}}
\flushright{\begin{Arabic}
\quranayah[21][21]
\end{Arabic}}
\flushleft{\begin{malayalam}
അതല്ല, അവര്‍ ഭൂമിയില്‍ നിന്നുതന്നെ (മരിച്ചവരെ) ജീവിപ്പിക്കാന്‍ കഴിവുള്ള വല്ല ദൈവങ്ങളെയും സ്വീകരിച്ചിരിക്കുകയാണോ?
\end{malayalam}}
\flushright{\begin{Arabic}
\quranayah[21][22]
\end{Arabic}}
\flushleft{\begin{malayalam}
ആകാശഭൂമികളില്‍ അല്ലാഹുവല്ലാത്ത വല്ല ദൈവങ്ങളുമുണ്ടായിരുന്നുവെങ്കില്‍ അത് രണ്ടും തകരാറാകുമായിരുന്നു. അപ്പോള്‍ സിംഹാസനത്തിന്‍റെ നാഥനായ അല്ലാഹു, അവര്‍ പറഞ്ഞുണ്ടാക്കുന്നതില്‍ നിന്നെല്ലാം എത്ര പരിശുദ്ധനാകുന്നു!
\end{malayalam}}
\flushright{\begin{Arabic}
\quranayah[21][23]
\end{Arabic}}
\flushleft{\begin{malayalam}
അവന്‍ പ്രവര്‍ത്തിക്കുന്നതിനെപ്പറ്റി ചോദ്യം ചെയ്യപ്പെടുകയില്ല. അവരാകട്ടെ ചോദ്യം ചെയ്യപ്പെടുന്നതുമാണ്‌.
\end{malayalam}}
\flushright{\begin{Arabic}
\quranayah[21][24]
\end{Arabic}}
\flushleft{\begin{malayalam}
അതല്ല, അവന്ന് പുറമെ അവര്‍ ദൈവങ്ങളെ സ്വീകരിച്ചിരിക്കുകയാണോ? പറയുക: എങ്കില്‍ നിങ്ങള്‍ക്കതിനുള്ള പ്രമാണം കൊണ്ട് വരിക. ഇതു തന്നെയാകുന്നു എന്‍റെ കൂടെയുള്ളവര്‍ക്കുള്ള ഉല്‍ബോധനവും എന്‍റെ മുമ്പുള്ളവര്‍ക്കുള്ള ഉല്‍ബോധനവും. പക്ഷെ, അവരില്‍ അധികപേരും സത്യം അറിയുന്നില്ല. അതിനാല്‍ അവര്‍ തിരിഞ്ഞുകളയുകയാകുന്നു.
\end{malayalam}}
\flushright{\begin{Arabic}
\quranayah[21][25]
\end{Arabic}}
\flushleft{\begin{malayalam}
ഞാനല്ലാതെ യാതൊരു ദൈവവുമില്ല. അതിനാല്‍ എന്നെ നിങ്ങള്‍ ആരാധിക്കൂ എന്ന് ബോധനം നല്‍കിക്കൊണ്ടല്ലാതെ നിനക്ക് മുമ്പ് ഒരു ദൂതനെയും നാം അയച്ചിട്ടില്ല.
\end{malayalam}}
\flushright{\begin{Arabic}
\quranayah[21][26]
\end{Arabic}}
\flushleft{\begin{malayalam}
പരമകാരുണികന്‍ സന്താനത്തെ സ്വീകരിച്ചിരിക്കുന്നു എന്നവര്‍ പറഞ്ഞു.അവന്‍ എത്ര പരിശുദ്ധന്‍! എന്നാല്‍ (അവര്‍ - മലക്കുകള്‍) അവന്‍റെ ആദരണീയരായ ദാസന്‍മാര്‍ മാത്രമാകുന്നു.
\end{malayalam}}
\flushright{\begin{Arabic}
\quranayah[21][27]
\end{Arabic}}
\flushleft{\begin{malayalam}
അവര്‍ അവനെ മറികടന്നു സംസാരിക്കുകയില്ല. അവന്‍റെ കല്‍പനയനുസരിച്ച് മാത്രം അവര്‍ പ്രവര്‍ത്തിക്കുന്നു
\end{malayalam}}
\flushright{\begin{Arabic}
\quranayah[21][28]
\end{Arabic}}
\flushleft{\begin{malayalam}
അവരുടെ മുമ്പിലുള്ളതും പിന്നിലുള്ളതും അവന്‍ അറിയുന്നു. അവന്‍ തൃപ്തിപ്പെട്ടവര്‍ക്കല്ലാതെ അവര്‍ ശുപാര്‍ശ ചെയ്യുകയില്ല. അവരാകട്ടെ, അവനെപ്പറ്റിയുള്ള ഭയത്താല്‍ നടുങ്ങുന്നവരാകുന്നു.
\end{malayalam}}
\flushright{\begin{Arabic}
\quranayah[21][29]
\end{Arabic}}
\flushleft{\begin{malayalam}
അവരുടെ കൂട്ടത്തില്‍ ആരെങ്കിലും ഞാന്‍ അവന്ന് (അല്ലാഹുവിന്‌) പുറമെയുള്ള ദൈവമാണെന്ന് പറയുന്ന പക്ഷം അവന്ന് നാം നരകം പ്രതിഫലമായി നല്‍കുന്നതാണ്‌. അപ്രകാരമത്രെ അക്രമികള്‍ക്ക് നാം പ്രതിഫലം നല്‍കുന്നത്‌.
\end{malayalam}}
\flushright{\begin{Arabic}
\quranayah[21][30]
\end{Arabic}}
\flushleft{\begin{malayalam}
ആകാശങ്ങളും ഭൂമിയും ഒട്ടിച്ചേര്‍ന്നതായിരുന്നു വെന്നും, എന്നിട്ട് നാം അവയെ വേര്‍പെടുത്തുകയാണുണ്ടായതെന്നും സത്യനിഷേധികള്‍ കണ്ടില്ലേ? വെള്ളത്തില്‍ നിന്ന് എല്ലാ ജീവവസ്തുക്കളും നാം ഉണ്ടാക്കുകയും ചെയ്തു. എന്നിട്ടും അവര്‍ വിശ്വസിക്കുന്നില്ലേ?
\end{malayalam}}
\flushright{\begin{Arabic}
\quranayah[21][31]
\end{Arabic}}
\flushleft{\begin{malayalam}
ഭൂമി അവരെയും കൊണ്ട് ഇളകാതിരിക്കുവാനായി അതില്‍ നാം ഉറച്ചുനില്‍ക്കുന്ന പര്‍വ്വതങ്ങളുണ്ടാക്കുകയും ചെയ്തിരിക്കുന്നു. അവര്‍ വഴി കണ്ടെത്തേണ്ടതിനായി അവയില്‍ (പര്‍വ്വതങ്ങളില്‍) നാം വിശാലമായ പാതകള്‍ ഏര്‍പെടുത്തുകയും ചെയ്തിരിക്കുന്നു.
\end{malayalam}}
\flushright{\begin{Arabic}
\quranayah[21][32]
\end{Arabic}}
\flushleft{\begin{malayalam}
ആകാശത്തെ നാം സംരക്ഷിതമായ ഒരു മേല്‍പുരയാക്കിയിട്ടുമുണ്ട്‌. അവരാകട്ടെ അതിലെ (ആകാശത്തിലെ) ദൃഷ്ടാന്തങ്ങള്‍ ശ്രദ്ധിക്കാതെ തിരിഞ്ഞുകളയുന്നവരാകുന്നു.
\end{malayalam}}
\flushright{\begin{Arabic}
\quranayah[21][33]
\end{Arabic}}
\flushleft{\begin{malayalam}
അവനത്രെ രാത്രി, പകല്‍, സൂര്യന്‍, ചന്ദ്രന്‍ എന്നിവയെ സൃഷ്ടിച്ചത്‌. ഓരോന്നും ഓരോ ഭ്രമണപഥത്തിലൂടെ നീന്തി (സഞ്ചരിച്ചു) ക്കൊണ്ടിരിക്കുന്നു.
\end{malayalam}}
\flushright{\begin{Arabic}
\quranayah[21][34]
\end{Arabic}}
\flushleft{\begin{malayalam}
(നബിയേ,) നിനക്ക് മുമ്പ് ഒരു മനുഷ്യന്നും നാം അനശ്വരത നല്‍കിയിട്ടില്ല. എന്നിരിക്കെ നീ മരിച്ചെങ്കില്‍ അവര്‍ നിത്യജീവികളായിരിക്കുമോ?
\end{malayalam}}
\flushright{\begin{Arabic}
\quranayah[21][35]
\end{Arabic}}
\flushleft{\begin{malayalam}
ഓരോ വ്യക്തിയും മരണം ആസ്വദിക്കുകതന്നെ ചെയ്യും. ഒരു പരീക്ഷണം എന്ന നിലയില്‍ തിന്‍മ നല്‍കിക്കൊണ്ടും നന്‍മ നല്‍കിക്കൊണ്ടും നിങ്ങളെ നാം പരിശോധിക്കുന്നതാണ്‌. നമ്മുടെ അടുത്തേക്ക് തന്നെ നിങ്ങള്‍ മടക്കപ്പെടുകയും ചെയ്യും.
\end{malayalam}}
\flushright{\begin{Arabic}
\quranayah[21][36]
\end{Arabic}}
\flushleft{\begin{malayalam}
സത്യനിഷേധികള്‍ നിന്നെ കണ്ടാല്‍, ഇവനാണോ നിങ്ങളുടെ ദൈവങ്ങളെ ആക്ഷേപിച്ച് സംസാരിക്കുന്നവന്‍ എന്ന് പറഞ്ഞ് കൊണ്ട് നിന്നെ തമാശയാക്കുക മാത്രമായിരിക്കും ചെയ്യുന്നത്‌. അവര്‍ തന്നെയാണ് പരമകാരുണികന്‍റെ ഉല്‍ബോധനത്തില്‍ അവിശ്വസിക്കുന്നവര്‍.
\end{malayalam}}
\flushright{\begin{Arabic}
\quranayah[21][37]
\end{Arabic}}
\flushleft{\begin{malayalam}
ധൃതികൂട്ടുന്നവനായിട്ടാകുന്നു മനുഷ്യന്‍ സൃഷ്ടിക്കപ്പെട്ടിട്ടുള്ളത്‌. എന്‍റെ ദൃഷ്ടാന്തങ്ങള്‍ വഴിയെ ഞാന്‍ നിങ്ങള്‍ക്ക് കാണിച്ചുതരുന്നതാണ്‌. അതിനാല്‍ നിങ്ങള്‍ എന്നോട് ധൃതികൂട്ടരുത്‌.
\end{malayalam}}
\flushright{\begin{Arabic}
\quranayah[21][38]
\end{Arabic}}
\flushleft{\begin{malayalam}
അവര്‍ ചോദിക്കുന്നു; നിങ്ങള്‍ സത്യവാന്‍മാരാണെങ്കില്‍ ഈ വാഗ്ദാനം എപ്പോഴാണ് (പുലരുക) എന്ന്‌.
\end{malayalam}}
\flushright{\begin{Arabic}
\quranayah[21][39]
\end{Arabic}}
\flushleft{\begin{malayalam}
ആ അവിശ്വാസികള്‍, അവര്‍ക്ക് തങ്ങളുടെ മുഖങ്ങളില്‍ നിന്നും മുതുകുകളില്‍ നിന്നും നരകാഗ്നിയെ തടുക്കാനാവാത്ത, അവര്‍ക്ക് ഒരു സഹായവും സിദ്ധിക്കാത്ത ഒരു സന്ദര്‍ഭത്തെപ്പറ്റി മനസ്സിലാക്കിയിരുന്നെങ്കില്‍!
\end{malayalam}}
\flushright{\begin{Arabic}
\quranayah[21][40]
\end{Arabic}}
\flushleft{\begin{malayalam}
അല്ല, പെട്ടന്നായിരിക്കും അത് (അന്ത്യസമയം) അവര്‍ക്ക് വന്നെത്തുന്നത് . അങ്ങനെ അതവരെ അമ്പരപ്പിച്ച് കളയും. അതിനെ തടുത്ത് നിര്‍ത്താന്‍ അവര്‍ക്ക് സാധിക്കുകയില്ല. അവര്‍ക്ക് ഇടകൊടുക്കപ്പെടുകയുമില്ല.
\end{malayalam}}
\flushright{\begin{Arabic}
\quranayah[21][41]
\end{Arabic}}
\flushleft{\begin{malayalam}
നിനക്ക് മുമ്പ് പല ദൈവദൂതന്‍മാരും പരിഹസിക്കപ്പെട്ടിട്ടുണ്ട്‌. എന്നിട്ട് അവരെ പുച്ഛിച്ച് തള്ളിയവര്‍ക്ക് തങ്ങള്‍ പരിഹസിച്ച് കൊണ്ടിരുന്നത് (ശിക്ഷ) വന്നെത്തുക തന്നെ ചെയ്തു.
\end{malayalam}}
\flushright{\begin{Arabic}
\quranayah[21][42]
\end{Arabic}}
\flushleft{\begin{malayalam}
(നബിയേ,) പറയുക: പരമകാരുണികനില്‍ നിന്ന് രാത്രിയും പകലും നിങ്ങള്‍ക്ക് രക്ഷനല്‍കാനാരുണ്ട്‌? അല്ല, അവര്‍ (ജനങ്ങള്‍) തങ്ങളുടെ രക്ഷിതാവിന്‍റെ ഉല്‍ബോധനത്തില്‍ നിന്ന് തിരിഞ്ഞുകളയുന്നവരാകുന്നു.
\end{malayalam}}
\flushright{\begin{Arabic}
\quranayah[21][43]
\end{Arabic}}
\flushleft{\begin{malayalam}
അതല്ല, നമുക്ക് പുറമെ അവരെ സംരക്ഷിക്കുന്ന വല്ല ദൈവങ്ങളും അവര്‍ക്കുണ്ടോ? സ്വദേഹങ്ങള്‍ക്ക് തന്നെ സഹായം ചെയ്യാന്‍ അവര്‍ക്ക് (ദൈവങ്ങള്‍ക്ക്‌) സാധിക്കുകയില്ല. നമ്മുടെ ഭാഗത്ത് നിന്നും അവര്‍ തുണക്കപ്പെടുകയുമില്ല.
\end{malayalam}}
\flushright{\begin{Arabic}
\quranayah[21][44]
\end{Arabic}}
\flushleft{\begin{malayalam}
അല്ല, ഇവര്‍ക്കും ഇവരുടെ പിതാക്കള്‍ക്കും നാം ജീവിതസുഖം നല്‍കി. അങ്ങനെ അവര്‍ ദീര്‍ഘകാലം ജീവിച്ചു. എന്നാല്‍ ആ ഭൂപ്രദേശത്തെ അതിന്‍റെ നാനാ ഭാഗങ്ങളില്‍ നിന്നും നാം ചുരുക്കിക്കൊണ്ട് വരുന്നത് ഇവര്‍ കാണുന്നില്ലേ ? എന്നിട്ടും ഇവര്‍ തന്നെയാണോ വിജയം പ്രാപിക്കുന്നവര്‍?
\end{malayalam}}
\flushright{\begin{Arabic}
\quranayah[21][45]
\end{Arabic}}
\flushleft{\begin{malayalam}
(നബിയേ,) പറയുക: ദിവ്യസന്ദേശ പ്രകാരം മാത്രമാണ് ഞാന്‍ നിങ്ങള്‍ക്ക് താക്കീത് നല്‍കുന്നത്‌. താക്കീത് നല്‍കപ്പെടുമ്പോള്‍ ബധിരന്‍മാര്‍ ആ വിളികേള്‍ക്കുകയില്ല.
\end{malayalam}}
\flushright{\begin{Arabic}
\quranayah[21][46]
\end{Arabic}}
\flushleft{\begin{malayalam}
നിന്‍റെ രക്ഷിതാവിന്‍റെ ശിക്ഷയില്‍ നിന്ന് ഒരു നേരിയ കാറ്റ് അവരെ സ്പര്‍ശിക്കുന്ന പക്ഷം തീര്‍ച്ചയായും അവര്‍ പറയും: ഞങ്ങളുടെ നാശമേ! തീര്‍ച്ചയായും ഞങ്ങള്‍ അക്രമികളായിപ്പോയല്ലോ!
\end{malayalam}}
\flushright{\begin{Arabic}
\quranayah[21][47]
\end{Arabic}}
\flushleft{\begin{malayalam}
ഉയിര്‍ത്തെഴുന്നേല്‍പിന്‍റെ നാളില്‍ നീതിപൂര്‍ണ്ണമായ തുലാസുകള്‍ നാം സ്ഥാപിക്കുന്നതാണ്‌. അപ്പോള്‍ ഒരാളോടും ഒട്ടും അനീതി കാണിക്കപ്പെടുകയില്ല. അത് (കര്‍മ്മം) ഒരു കടുക്മണിത്തൂക്കമുള്ളതാണെങ്കിലും നാമത് കൊണ്ട് വരുന്നതാണ്‌. കണക്ക് നോക്കുവാന്‍ നാം തന്നെ മതി.
\end{malayalam}}
\flushright{\begin{Arabic}
\quranayah[21][48]
\end{Arabic}}
\flushleft{\begin{malayalam}
മൂസായ്ക്കും ഹാറൂന്നും സത്യാസത്യവിവേചനത്തിനുള്ള പ്രമാണവും, പ്രകാശവും, ധര്‍മ്മനിഷ്ഠപുലര്‍ത്തുന്നവര്‍ക്കുള്ള ഉല്‍ബോധനവും നാം നല്‍കിയിട്ടുണ്ട്‌.
\end{malayalam}}
\flushright{\begin{Arabic}
\quranayah[21][49]
\end{Arabic}}
\flushleft{\begin{malayalam}
തങ്ങളുടെ രക്ഷിതാവിനെ അദൃശ്യാവസ്ഥയില്‍ ഭയപ്പെടുന്നവരും, അന്ത്യനാളിനെപ്പറ്റി ഉല്‍ക്കണ്ഠയുള്ളവരുമാരോ (അവര്‍ക്കുള്ള ഉല്‍ബോധനം.)
\end{malayalam}}
\flushright{\begin{Arabic}
\quranayah[21][50]
\end{Arabic}}
\flushleft{\begin{malayalam}
ഇത് (ഖുര്‍ആന്‍) നാം അവതരിപ്പിച്ച അനുഗ്രഹപൂര്‍ണ്ണമായ ഒരു ഉല്‍ബോധനമാകുന്നു. എന്നിരിക്കെ നിങ്ങള്‍ അതിനെ നിഷേധിക്കുകയാണോ?
\end{malayalam}}
\flushright{\begin{Arabic}
\quranayah[21][51]
\end{Arabic}}
\flushleft{\begin{malayalam}
മുമ്പ് ഇബ്രാഹീമിന് തന്‍റെതായ വിവേകം നാം നല്‍കുകയുണ്ടായി. അദ്ദേഹത്തെ പറ്റി നമുക്കറിയാമായിരുന്നു.
\end{malayalam}}
\flushright{\begin{Arabic}
\quranayah[21][52]
\end{Arabic}}
\flushleft{\begin{malayalam}
തന്‍റെ പിതാവിനോടും തന്‍റെ ജനങ്ങളോടും അദ്ദേഹം ഇപ്രകാരം ചോദിച്ച സന്ദര്‍ഭം (ശ്രദ്ധേയമത്രെ:) നിങ്ങള്‍ പൂജിച്ചുകൊണേ്ടയിരിക്കുന്ന ഈ പ്രതിമകള്‍ എന്താകുന്നു?
\end{malayalam}}
\flushright{\begin{Arabic}
\quranayah[21][53]
\end{Arabic}}
\flushleft{\begin{malayalam}
അവര്‍ പറഞ്ഞു: ഞങ്ങളുടെ പിതാക്കള്‍ ഇവയെ ആരാധിച്ച് വരുന്നതായിട്ടാണ് ഞങ്ങള്‍ കണ്ടത്‌.
\end{malayalam}}
\flushright{\begin{Arabic}
\quranayah[21][54]
\end{Arabic}}
\flushleft{\begin{malayalam}
അദ്ദേഹം പറഞ്ഞു: തീര്‍ച്ചയായും നിങ്ങളും നിങ്ങളുടെ പിതാക്കളും വ്യക്തമായ വഴികേടിലായിരിക്കുന്നു.
\end{malayalam}}
\flushright{\begin{Arabic}
\quranayah[21][55]
\end{Arabic}}
\flushleft{\begin{malayalam}
അവര്‍ പറഞ്ഞു: നീ ഞങ്ങളുടെ അടുത്ത് സത്യവും കൊണ്ട് വന്നിരിക്കുകയാണോ? അതല്ല, നീ കളിപറയുന്നവരുടെ കൂട്ടത്തിലാണോ?
\end{malayalam}}
\flushright{\begin{Arabic}
\quranayah[21][56]
\end{Arabic}}
\flushleft{\begin{malayalam}
അദ്ദേഹം പറഞ്ഞു: അല്ല, നിങ്ങളുടെ രക്ഷിതാവ് ആകാശങ്ങളുടെയും ഭൂമിയുടെയും രക്ഷിതാവാകുന്നു. അവയെ സൃഷ്ടിച്ചുണ്ടാക്കിയവന്‍. ഞാന്‍ അതിന് സാക്ഷ്യം വഹിക്കുന്നവരുടെ കൂട്ടത്തിലാകുന്നു.
\end{malayalam}}
\flushright{\begin{Arabic}
\quranayah[21][57]
\end{Arabic}}
\flushleft{\begin{malayalam}
അല്ലാഹുവെ തന്നെയാണ, തീര്‍ച്ചയായും നിങ്ങള്‍ പിന്നിട്ട് പോയതിന് ശേഷം ഞാന്‍ നിങ്ങളുടെ വിഗ്രഹങ്ങളുടെ കാര്യത്തില്‍ ഒരു തന്ത്രം പ്രയോഗിക്കുന്നതാണ്‌.
\end{malayalam}}
\flushright{\begin{Arabic}
\quranayah[21][58]
\end{Arabic}}
\flushleft{\begin{malayalam}
അങ്ങനെ അദ്ദേഹം അവരെ (ദൈവങ്ങളെ) തുണ്ടം തുണ്ടമാക്കിക്കളഞ്ഞു. അവരില്‍ ഒരാളെ ഒഴികെ. അവര്‍ക്ക് (വിവരമറിയാനായി) അയാളുടെ അടുത്തേക്ക് തിരിച്ചുചെല്ലാമല്ലോ?
\end{malayalam}}
\flushright{\begin{Arabic}
\quranayah[21][59]
\end{Arabic}}
\flushleft{\begin{malayalam}
അവര്‍ പറഞ്ഞു: നമ്മുടെ ദൈവങ്ങളെക്കൊണ്ട് ഇത് ചെയ്തവന്‍ ആരാണ്‌? തീര്‍ച്ചയായും അവന്‍ അക്രമികളില്‍ പെട്ടവന്‍ തന്നെയാണ്‌.
\end{malayalam}}
\flushright{\begin{Arabic}
\quranayah[21][60]
\end{Arabic}}
\flushleft{\begin{malayalam}
ചിലര്‍ പറഞ്ഞു: ഇബ്രാഹീം എന്ന് വിളിക്കപ്പെടുന്ന ഒരു ചെറുപ്പക്കാരന്‍ ആ ദൈവങ്ങളെപ്പറ്റി പരാമര്‍ശിക്കുന്നത് ഞങ്ങള്‍ കേട്ടിണ്ട്‌.
\end{malayalam}}
\flushright{\begin{Arabic}
\quranayah[21][61]
\end{Arabic}}
\flushleft{\begin{malayalam}
അവര്‍ പറഞ്ഞു: എന്നാല്‍ നിങ്ങള്‍ അവനെ ജനങ്ങളുടെ കണ്‍മുമ്പില്‍ കൊണ്ട് വരൂ. അവര്‍ സാക്ഷ്യം വഹിച്ചേക്കാം.
\end{malayalam}}
\flushright{\begin{Arabic}
\quranayah[21][62]
\end{Arabic}}
\flushleft{\begin{malayalam}
അവര്‍ ചോദിച്ചു: ഇബ്രാഹീമേ, നീയാണോ ഞങ്ങളുടെ ദൈവങ്ങളെക്കൊണ്ട് ഇതു ചെയ്തത്‌?
\end{malayalam}}
\flushright{\begin{Arabic}
\quranayah[21][63]
\end{Arabic}}
\flushleft{\begin{malayalam}
അദ്ദേഹം പറഞ്ഞു: എന്നാല്‍ അവരുടെ കൂട്ടത്തിലെ ഈ വലിയവനാണ് അത് ചെയ്തത്‌. അവര്‍ സംസാരിക്കുമെങ്കില്‍ നിങ്ങള്‍ അവരോട് ചോദിച്ച് നോക്കൂ!
\end{malayalam}}
\flushright{\begin{Arabic}
\quranayah[21][64]
\end{Arabic}}
\flushleft{\begin{malayalam}
അപ്പോള്‍ അവര്‍ സ്വമനസ്സകളിലേക്ക് തന്നെ മടങ്ങി. എന്നിട്ടവര്‍ (അന്യോന്യം) പറഞ്ഞു: തീര്‍ച്ചയായും നിങ്ങള്‍ തന്നെയാണ് അക്രമകാരികള്‍.
\end{malayalam}}
\flushright{\begin{Arabic}
\quranayah[21][65]
\end{Arabic}}
\flushleft{\begin{malayalam}
പിന്നെ അവര്‍ തലകുത്തനെ മറിഞ്ഞു. (അവര്‍ പറഞ്ഞു:) ഇവര്‍ സംസാരിക്കുകയില്ലെന്ന് നിനക്കറിയാമല്ലോ.
\end{malayalam}}
\flushright{\begin{Arabic}
\quranayah[21][66]
\end{Arabic}}
\flushleft{\begin{malayalam}
അദ്ദേഹം പറഞ്ഞു: അപ്പോള്‍ നിങ്ങള്‍ക്ക് യാതൊരു ഉപകാരമോ ഉപദ്രവമോ ചെയ്യാത്ത വസ്തുക്കളെ അല്ലാഹുവിന് പുറമെ നിങ്ങള്‍ ആരാധിക്കുകയാണോ?
\end{malayalam}}
\flushright{\begin{Arabic}
\quranayah[21][67]
\end{Arabic}}
\flushleft{\begin{malayalam}
നിങ്ങളുടെയും, അല്ലാഹുവിന് പുറമെ നിങ്ങള്‍ ആരാധിക്കുന്നവരുടെയും കാര്യം അപഹാസ്യം തന്നെ. നിങ്ങള്‍ ചിന്തിക്കുന്നില്ലേ?
\end{malayalam}}
\flushright{\begin{Arabic}
\quranayah[21][68]
\end{Arabic}}
\flushleft{\begin{malayalam}
അവര്‍ പറഞ്ഞു: നിങ്ങള്‍ക്ക് (വല്ലതും) ചെയ്യാനാകുമെങ്കില്‍ നിങ്ങള്‍ ഇവനെ ചുട്ടെരിച്ച് കളയുകയും, നിങ്ങളുടെ ദൈവങ്ങളെ സഹായിക്കുകയും ചെയ്യുക.
\end{malayalam}}
\flushright{\begin{Arabic}
\quranayah[21][69]
\end{Arabic}}
\flushleft{\begin{malayalam}
നാം പറഞ്ഞു: തീയേ, നീ ഇബ്രാഹീമിന് തണുപ്പും സമാധാനവുമായിരിക്കുക.
\end{malayalam}}
\flushright{\begin{Arabic}
\quranayah[21][70]
\end{Arabic}}
\flushleft{\begin{malayalam}
അദ്ദേഹത്തിന്‍റെ കാര്യത്തില്‍ ഒരു തന്ത്രം പ്രയോഗിക്കുവാന്‍ അവര്‍ ഉദ്ദേശിച്ചു. എന്നാല്‍ അവരെ ഏറ്റവും നഷ്ടം പറ്റിയവരാക്കുകയാണ് നാം ചെയ്തത്‌.
\end{malayalam}}
\flushright{\begin{Arabic}
\quranayah[21][71]
\end{Arabic}}
\flushleft{\begin{malayalam}
ലോകര്‍ക്ക് വേണ്ടി നാം അനുഗൃഹീതമാക്കിവെച്ചിട്ടുള്ള ഒരു ഭൂപ്രദേശത്തേക്ക് അദ്ദേഹത്തേയും ലൂത്വിനേയും നാം രക്ഷപ്പെടുത്തിക്കൊണ്ട് പോകുകയും ചെയ്തു.
\end{malayalam}}
\flushright{\begin{Arabic}
\quranayah[21][72]
\end{Arabic}}
\flushleft{\begin{malayalam}
അദ്ദേഹത്തിന് നാം ഇഷാഖിനെ പ്രദാനം ചെയ്തു. പുറമെ (പൌത്രന്‍) യഅ്ഖൂബിനെയും. അവരെയെല്ലാം നാം സദ്‌വൃത്തരാക്കിയിരിക്കുന്നു.
\end{malayalam}}
\flushright{\begin{Arabic}
\quranayah[21][73]
\end{Arabic}}
\flushleft{\begin{malayalam}
അവരെ നാം നമ്മുടെ കല്‍പനപ്രകാരം മാര്‍ഗദര്‍ശനം നല്‍കുന്ന നേതാക്കളാക്കുകയും ചെയ്തു. നല്ല കാര്യങ്ങള്‍ ചെയ്യണമെന്നും, നമസ്കാരം മുറപോലെ നിര്‍വഹിക്കണമെന്നും, സകാത്ത് നല്‍കണമെന്നും നാം അവര്‍ക്ക് ബോധനം നല്‍കുകയും ചെയ്തു. നമ്മെയായിരുന്നു അവര്‍ ആരാധിച്ചിരുന്നത്‌.
\end{malayalam}}
\flushright{\begin{Arabic}
\quranayah[21][74]
\end{Arabic}}
\flushleft{\begin{malayalam}
ലൂത്വിന് നാം വിധികര്‍ത്തൃത്വവും വിജ്ഞാനവും നല്‍കുകയുണ്ടായി. ദുര്‍വൃത്തികള്‍ ചെയ്തുകൊണ്ടിരുന്ന ആ നാട്ടില്‍ നിന്ന് അദ്ദേഹത്തെ നാം രക്ഷപ്പെടുത്തുകയും ചെയ്തു. തീര്‍ച്ചയായും അവര്‍ (നാട്ടുകാര്‍) ധിക്കാരികളായ ഒരു ദുഷിച്ച ജനതയായിരുന്നു.
\end{malayalam}}
\flushright{\begin{Arabic}
\quranayah[21][75]
\end{Arabic}}
\flushleft{\begin{malayalam}
നമ്മുടെ കാരുണ്യത്തില്‍ അദ്ദേഹത്തെ നാം ഉള്‍പെടുത്തുകയും ചെയ്തു. തീര്‍ച്ചയായും അദ്ദേഹം സദ്‌വൃത്തരുടെ കൂട്ടത്തിലാകുന്നു.
\end{malayalam}}
\flushright{\begin{Arabic}
\quranayah[21][76]
\end{Arabic}}
\flushleft{\begin{malayalam}
നൂഹിനെയും (ഓര്‍ക്കുക). മുമ്പ് അദ്ദേഹം വിളിച്ച് പ്രാര്‍ത്ഥിച്ച സന്ദര്‍ഭം. അദ്ദേഹത്തിന് നാം ഉത്തരം നല്‍കി. അങ്ങനെ അദ്ദേഹത്തെയും, അദ്ദേഹത്തിന്‍റെ കുടുംബത്തെയും നാം മഹാ ദുഃഖത്തില്‍ നിന്ന് രക്ഷപ്പെടുത്തി.
\end{malayalam}}
\flushright{\begin{Arabic}
\quranayah[21][77]
\end{Arabic}}
\flushleft{\begin{malayalam}
നമ്മുടെ ദൃഷ്ടാന്തങ്ങളെ നിഷേധിച്ചു തള്ളിയ ജനങ്ങളില്‍ നിന്ന് അദ്ദേഹത്തിന് നാം രക്ഷനല്‍കുകയും ചെയ്തു. തീര്‍ച്ചയായും അവര്‍ ദുഷിച്ച ഒരു ജനവിഭാഗമായിരുന്നു.അതിനാല്‍ അവരെ മുഴുവന്‍ നാം മുക്കി നശിപ്പിച്ചു കളഞ്ഞു.
\end{malayalam}}
\flushright{\begin{Arabic}
\quranayah[21][78]
\end{Arabic}}
\flushleft{\begin{malayalam}
ദാവൂദിനെയും (പുത്രന്‍) സുലൈമാനെയും (ഓര്‍ക്കുക.) ഒരു ജനവിഭാഗത്തിന്‍റെ ആടുകള്‍ വിളയില്‍ കടന്ന് മേഞ്ഞ പ്രശ്നത്തില്‍ അവര്‍ രണ്ട് പേരും വിധികല്‍പിക്കുന്ന സന്ദര്‍ഭം. അവരുടെ വിധിക്ക് നാം സാക്ഷ്യം വഹിക്കുന്നുണ്ടായിരിന്നു.
\end{malayalam}}
\flushright{\begin{Arabic}
\quranayah[21][79]
\end{Arabic}}
\flushleft{\begin{malayalam}
അപ്പോള്‍ സുലൈമാന്ന് നാം അത് (പ്രശ്നം) ഗ്രഹിപ്പിച്ചു അവര്‍ ഇരുവര്‍ക്കും നാം വിധികര്‍ത്തൃത്വവും വിജ്ഞാനവും നല്‍കിയിരുന്നു. ദാവൂദിനോടൊപ്പം കീര്‍ത്തനം ചെയ്തുകൊണ്ടിരിക്കുന്ന നിലയില്‍ പര്‍വ്വതങ്ങളെയും പക്ഷികളെയും നാം കീഴ്പെടുത്തികൊടുത്തു. നാമായിരുന്നു (അതെല്ലാം) നടപ്പാക്കിക്കൊണ്ടിരുന്നത്‌.
\end{malayalam}}
\flushright{\begin{Arabic}
\quranayah[21][80]
\end{Arabic}}
\flushleft{\begin{malayalam}
നിങ്ങള്‍ നേരിടുന്ന യുദ്ധ വിപത്തുകളില്‍ നിന്ന് നിങ്ങള്‍ക്ക് സംരക്ഷണം നല്‍കുവാനായി നിങ്ങള്‍ക്ക് വേണ്ടിയുള്ള പടയങ്കിയുടെ നിര്‍മാണവും അദ്ദേഹത്തെ നാം പഠിപ്പിച്ചു. എന്നിട്ട് നിങ്ങള്‍ നന്ദിയുള്ളവരാണോ?
\end{malayalam}}
\flushright{\begin{Arabic}
\quranayah[21][81]
\end{Arabic}}
\flushleft{\begin{malayalam}
സുലൈമാന്ന് ശക്തിയായി വീശുന്ന കാറ്റിനെയും (നാം കീഴ്പെടുത്തികൊടുത്തു.) നാം അനുഗ്രഹം നല്‍കിയിട്ടുള്ള ഭൂപ്രദേശത്തേക്ക് അദ്ദേഹത്തിന്‍റെ കല്‍പനപ്രകാരം അത് (കാറ്റ്‌) സഞ്ചരിച്ച് കൊണ്ടിരുന്നു. എല്ലാകാര്യത്തെപറ്റിയും നാം അറിവുള്ളവനാകുന്നു.
\end{malayalam}}
\flushright{\begin{Arabic}
\quranayah[21][82]
\end{Arabic}}
\flushleft{\begin{malayalam}
പിശാചുക്കളുടെ കൂട്ടത്തില്‍ നിന്ന് അദ്ദേഹത്തിന് വേണ്ടി (കടലില്‍) മുങ്ങുന്ന ചിലരെയും (നാം കീഴ്പെടുത്തികൊടുത്തു.) അതു കൂടാതെ മറ്റു ചില പ്രവൃത്തികളും അവര്‍ ചെയ്തിരുന്നു. നാമായിരുന്നു അവരെ കാത്തുസൂക്ഷിച്ച് കൊണ്ടിരുന്നത്‌.
\end{malayalam}}
\flushright{\begin{Arabic}
\quranayah[21][83]
\end{Arabic}}
\flushleft{\begin{malayalam}
അയ്യൂബിനെയും (ഓര്‍ക്കുക.) തന്‍റെ രക്ഷിതാവിനെ വിളിച്ച് കൊണ്ട് അദ്ദേഹം ഇപ്രകാരം പ്രാര്‍ത്ഥിച്ച സന്ദര്‍ഭം: എനിക്കിതാ കഷ്ടപ്പാട് ബാധിച്ചിരിക്കുന്നു. നീ കാരുണികരില്‍ വെച്ച് ഏറ്റവും കരുണയുള്ളവനാണല്ലോ.
\end{malayalam}}
\flushright{\begin{Arabic}
\quranayah[21][84]
\end{Arabic}}
\flushleft{\begin{malayalam}
അപ്പോള്‍ അദ്ദേഹത്തിന് നാം ഉത്തരം നല്‍കുകയും, അദ്ദേഹത്തിന് നേരിട്ട കഷ്ടപ്പാട് നാം അകറ്റിക്കളയുകയും ചെയ്തു. അദ്ദേഹത്തിന്‍റെ കുടുംബാംഗങ്ങളെയും, അവരോടൊപ്പം അവരുടെ അത്രയും പേരെ വേറെയും നാം അദ്ദേഹത്തിന് നല്‍കുകയും ചെയ്തു. നമ്മുടെ പക്കല്‍ നിന്നുള്ള ഒരു കാരുണ്യവും, ആരാധനാനിരതരായിട്ടുള്ളവര്‍ക്ക് ഒരു സ്മരണയുമാണത്‌.
\end{malayalam}}
\flushright{\begin{Arabic}
\quranayah[21][85]
\end{Arabic}}
\flushleft{\begin{malayalam}
ഇസ്മാഈലിനെയും, ഇദ്‌രീസിനെയും, ദുല്‍കിഫ്ലിനെയും (ഓര്‍ക്കുക) അവരെല്ലാം ക്ഷമാശീലരുടെ കൂട്ടത്തിലാകുന്നു.
\end{malayalam}}
\flushright{\begin{Arabic}
\quranayah[21][86]
\end{Arabic}}
\flushleft{\begin{malayalam}
അവരെ നാം നമ്മുടെ കാരുണ്യത്തില്‍ ഉള്‍പെടുത്തുകയും ചെയ്തിരിക്കുന്നു. തീര്‍ച്ചയായും അവര്‍ സദ്‌വൃത്തരുടെ കൂട്ടത്തിലാകുന്നു.
\end{malayalam}}
\flushright{\begin{Arabic}
\quranayah[21][87]
\end{Arabic}}
\flushleft{\begin{malayalam}
ദുന്നൂനി നെയും (ഓര്‍ക്കുക.) അദ്ദേഹം കുപിതനായി പോയിക്കളഞ്ഞ സന്ദര്‍ഭം. നാം ഒരിക്കലും അദ്ദേഹത്തിന് ഞെരുക്കമുണ്ടാക്കുകയില്ലെന്ന് അദ്ദേഹം ധരിച്ചു. അനന്തരം ഇരുട്ടുകള്‍ക്കുള്ളില്‍ നിന്ന് അദ്ദേഹം വിളിച്ചുപറഞ്ഞു: നീയല്ലാതെ യാതൊരു ദൈവവുമില്ല. നീ എത്ര പരിശുദ്ധന്‍! തീര്‍ച്ചയായും ഞാന്‍ അക്രമികളുടെ കൂട്ടത്തില്‍ പെട്ടവനായിരിക്കുന്നു.
\end{malayalam}}
\flushright{\begin{Arabic}
\quranayah[21][88]
\end{Arabic}}
\flushleft{\begin{malayalam}
അപ്പോള്‍ നാം അദ്ദേഹത്തിന് ഉത്തരം നല്‍കുകയും ദുഃഖത്തില്‍ നിന്ന് അദ്ദേഹത്തെ നാം രക്ഷപ്പെടുത്തുകയും ചെയ്തു. സത്യവിശ്വാസികളെ അപ്രകാരം നാം രക്ഷിക്കുന്നു.
\end{malayalam}}
\flushright{\begin{Arabic}
\quranayah[21][89]
\end{Arabic}}
\flushleft{\begin{malayalam}
സകരിയ്യായെയും (ഓര്‍ക്കുക.) അദ്ദേഹം തന്‍റെ രക്ഷിതാവിനെ വിളിച്ച് ഇപ്രകാരം പ്രാര്‍ത്ഥിച്ച സന്ദര്‍ഭം: എന്‍റെ രക്ഷിതാവേ, നീ എന്നെ ഏകനായി (പിന്തുടര്‍ച്ചക്കാരില്ലാതെ) വിടരുതേ. നീയാണല്ലോ അനന്തരാവകാശമെടുക്കുന്നവരില്‍ ഏറ്റവും ഉത്തമന്‍.
\end{malayalam}}
\flushright{\begin{Arabic}
\quranayah[21][90]
\end{Arabic}}
\flushleft{\begin{malayalam}
അപ്പോള്‍ നാം അദ്ദേഹത്തിന് ഉത്തരം നല്‍കുകയും, അദ്ദേഹത്തിന് (മകന്‍) യഹ്‌യായെ നാം പ്രദാനം ചെയ്യുകയും, അദ്ദേഹത്തിന്‍റെ ഭാര്യയെ നാം (ഗര്‍ഭധാരണത്തിന്‌) പ്രാപ്തയാക്കുകയും ചെയ്തു. തീര്‍ച്ചയായും അവര്‍ (പ്രവാചകന്‍മാര്‍) ഉത്തമകാര്യങ്ങള്‍ക്ക് ധൃതികാണിക്കുകയും, ആശിച്ച് കൊണ്ടും, പേടിച്ചുകൊണ്ടും നമ്മോട് പ്രാര്‍ത്ഥിക്കുകയും ചെയ്യുന്നവരായിരുന്നു. അവര്‍ നമ്മോട് താഴ്മ കാണിക്കുന്നവരുമായിരുന്നു.
\end{malayalam}}
\flushright{\begin{Arabic}
\quranayah[21][91]
\end{Arabic}}
\flushleft{\begin{malayalam}
തന്‍റെ ഗുഹ്യസ്ഥാനം സൂക്ഷിച്ച ഒരുവളെ (മര്‍യം) യും ഓര്‍ക്കുക. അങ്ങനെ അവളില്‍ നമ്മുടെ ആത്മാവില്‍ നിന്ന് നാം ഊതുകയും, അവളെയും അവളുടെ മകനെയും നാം ലോകര്‍ക്ക് ദൃഷ്ടാന്തമാക്കുകയും ചെയ്തു.
\end{malayalam}}
\flushright{\begin{Arabic}
\quranayah[21][92]
\end{Arabic}}
\flushleft{\begin{malayalam}
(മനുഷ്യരേ,) തീര്‍ച്ചയായും ഇതാണ് നിങ്ങളുടെ സമുദായം. ഏകസമുദായം. ഞാന്‍ നിങ്ങളുടെ രക്ഷിതാവും. അതിനാല്‍ നിങ്ങള്‍ എന്നെ ആരാധിക്കുവിന്‍.
\end{malayalam}}
\flushright{\begin{Arabic}
\quranayah[21][93]
\end{Arabic}}
\flushleft{\begin{malayalam}
എന്നാല്‍ അവര്‍ക്കിടയില്‍ അവരുടെ കാര്യം അവര്‍ ശിഥിലമാക്കിക്കളഞ്ഞിരിക്കയാണ്‌. എല്ലാവരും നമ്മുടെ അടുത്തേക്ക് തന്നെ മടങ്ങിവരുന്നവരത്രെ.
\end{malayalam}}
\flushright{\begin{Arabic}
\quranayah[21][94]
\end{Arabic}}
\flushleft{\begin{malayalam}
വല്ലവനും സത്യവിശ്വാസിയായിക്കൊണ്ട് സല്‍കര്‍മ്മങ്ങളില്‍ വല്ലതും ചെയ്യുന്ന പക്ഷം അവന്‍റെ പ്രയത്നത്തിന്‍റെ ഫലം നിഷേധിക്കപ്പെടുകയേയില്ല. തീര്‍ച്ചയായും നാം അത് എഴുതിവെക്കുന്നതാണ്‌.
\end{malayalam}}
\flushright{\begin{Arabic}
\quranayah[21][95]
\end{Arabic}}
\flushleft{\begin{malayalam}
നാം നശിപ്പിച്ച് കളഞ്ഞിട്ടുള്ള ഏതൊരു നാട്ടുകാരെ സംബന്ധിച്ചിടത്തോളവും അവര്‍ നമ്മുടെ അടുത്തേക്ക് തിരിച്ചുവരാതിരിക്കുക എന്നത് അസംഭവ്യമാകുന്നു.
\end{malayalam}}
\flushright{\begin{Arabic}
\quranayah[21][96]
\end{Arabic}}
\flushleft{\begin{malayalam}
അങ്ങനെ യഅ്ജൂജ് - മഅ്ജൂജ് ജനവിഭാഗങ്ങള്‍ തുറന്നുവിടപ്പെടുകയും, അവര്‍ എല്ലാ കുന്നുകളില്‍ നിന്നും കുതിച്ചിറങ്ങി വരികയും.
\end{malayalam}}
\flushright{\begin{Arabic}
\quranayah[21][97]
\end{Arabic}}
\flushleft{\begin{malayalam}
ആ സത്യവാഗ്ദാനം ആസന്നമാകുകയും ചെയ്താല്‍ അപ്പോഴതാ അവിശ്വസിച്ചവരുടെ കണ്ണുകള്‍ ഇമവെട്ടാതെ നിന്നു പോകന്നു. ഞങ്ങളുടെ നാശമേ! ഞങ്ങള്‍ ഈ കാര്യത്തെപ്പറ്റി അശ്രദ്ധയിലായിപ്പോയല്ലോ. അല്ല; ഞങ്ങള്‍ അക്രമകാരികളായിപ്പോയല്ലോ (എന്നായിരിക്കും അവര്‍ പറയുന്നത്‌.)
\end{malayalam}}
\flushright{\begin{Arabic}
\quranayah[21][98]
\end{Arabic}}
\flushleft{\begin{malayalam}
തീര്‍ച്ചയായും നിങ്ങളും അല്ലാഹുവിന് പുറമെ നിങ്ങള്‍ ആരാധിക്കുന്നവയും നരകത്തിലെ ഇന്ധനമാകുന്നു. നിങ്ങള്‍ അതിലേക്ക് വന്നുചേരുക തന്നെ ചെയ്യുന്നതാണ്‌.
\end{malayalam}}
\flushright{\begin{Arabic}
\quranayah[21][99]
\end{Arabic}}
\flushleft{\begin{malayalam}
ഇക്കൂട്ടര്‍ ദൈവങ്ങളായിരുന്നുവെങ്കില്‍ ഇവര്‍ അതില്‍ (നരകത്തില്‍) വന്നുചേരുകയില്ലായിരുന്നു. അവരെല്ലാം അതില്‍ നിത്യവാസികളായിരിക്കും.
\end{malayalam}}
\flushright{\begin{Arabic}
\quranayah[21][100]
\end{Arabic}}
\flushleft{\begin{malayalam}
അവര്‍ക്ക് അവിടെ ഒരു തേങ്ങലുണ്ടായിരിക്കും. അവര്‍ അതില്‍ വെച്ച് (യാതൊന്നും) കേള്‍ക്കുകയുമില്ല.
\end{malayalam}}
\flushright{\begin{Arabic}
\quranayah[21][101]
\end{Arabic}}
\flushleft{\begin{malayalam}
തീര്‍ച്ചയായും നമ്മുടെ പക്കല്‍ നിന്നു മുമ്പേ നന്‍മ ലഭിച്ചവരാരോ അവര്‍ അതില്‍ (നരകത്തില്‍) നിന്ന് അകറ്റിനിര്‍ത്തപ്പെടുന്നവരാകുന്നു.
\end{malayalam}}
\flushright{\begin{Arabic}
\quranayah[21][102]
\end{Arabic}}
\flushleft{\begin{malayalam}
അതിന്‍റെ നേരിയ ശബ്ദം പോലും അവര്‍ കേള്‍ക്കുകയില്ല. തങ്ങളുടെ മനസ്സുകള്‍ക്ക് ഇഷ്ടപ്പെട്ട സുഖാനുഭവങ്ങളില്‍ അവര്‍ നിത്യവാസികളായിരിക്കും.
\end{malayalam}}
\flushright{\begin{Arabic}
\quranayah[21][103]
\end{Arabic}}
\flushleft{\begin{malayalam}
ഏറ്റവും വലിയ ആ സംഭ്രമം അവര്‍ക്ക് ദുഃഖമുണ്ടാക്കുകയില്ല. നിങ്ങള്‍ക്ക് വാഗ്ദാനം ചെയ്യപ്പെട്ടിരുന്ന നിങ്ങളുടേതായ ദിവസമാണിത് എന്ന് പറഞ്ഞ് കൊണ്ട് മലക്കുകള്‍ അവരെ സ്വാഗതം ചെയ്യുന്നതാണ്‌.
\end{malayalam}}
\flushright{\begin{Arabic}
\quranayah[21][104]
\end{Arabic}}
\flushleft{\begin{malayalam}
ഗ്രന്ഥങ്ങളുടെ ഏടുകള്‍ ചുരുട്ടുന്ന പ്രകാരം ആകാശത്തെ നാം ചുരുട്ടിക്കളയുന്ന ദിവസം! ആദ്യമായി സൃഷ്ടി ആരംഭിച്ചത് പോലെത്തന്നെ നാം അത് ആവര്‍ത്തിക്കുന്നതുമാണ്‌. നാം ബാധ്യതയേറ്റ ഒരു വാഗ്ദാനമത്രെ അത്‌. നാം (അത്‌) നടപ്പിലാക്കുക തന്നെ ചെയ്യുന്നതാണ്‌.
\end{malayalam}}
\flushright{\begin{Arabic}
\quranayah[21][105]
\end{Arabic}}
\flushleft{\begin{malayalam}
ഭൂമിയുടെ അനന്തരാവകാശമെടുക്കുന്നത് എന്‍റെ സദ്‌വൃത്തരായ ദാസന്‍മാരായിരിക്കും എന്ന് ഉല്‍ബോധനത്തിന് ശേഷം നാം സബൂറില്‍ രേഖപ്പെടുത്തിയിട്ടുണ്ട്‌.
\end{malayalam}}
\flushright{\begin{Arabic}
\quranayah[21][106]
\end{Arabic}}
\flushleft{\begin{malayalam}
തീര്‍ച്ചയായും ഇതില്‍ ആരാധനാ നിരതരായ ആളുകള്‍ക്ക് ഒരു സന്ദേശമുണ്ട്‌.
\end{malayalam}}
\flushright{\begin{Arabic}
\quranayah[21][107]
\end{Arabic}}
\flushleft{\begin{malayalam}
ലോകര്‍ക്ക് കാരുണ്യമായിക്കൊണ്ടല്ലാതെ നിന്നെ നാം അയച്ചിട്ടില്ല.
\end{malayalam}}
\flushright{\begin{Arabic}
\quranayah[21][108]
\end{Arabic}}
\flushleft{\begin{malayalam}
പറയുക: നിങ്ങളുടെ ദൈവം ഏകദൈവം മാത്രമാണ് എന്നത്രെ എനിക്ക് ബോധനം നല്‍കപ്പെടുന്നത്‌. അതിനാല്‍ നിങ്ങള്‍ മുസ്ലിംകളാകുന്നുണ്ടോ?
\end{malayalam}}
\flushright{\begin{Arabic}
\quranayah[21][109]
\end{Arabic}}
\flushleft{\begin{malayalam}
എന്നിട്ട് അവര്‍ തിരിഞ്ഞുകളയുകയാണെങ്കില്‍ നീ പറയുക: നിങ്ങളോട് ഞാന്‍ പ്രഖ്യാപിച്ചിട്ടുള്ളത് തുല്യമായ വിധത്തിലാകുന്നു. നിങ്ങളോട് വാഗ്ദാനം ചെയ്യപ്പെടുന്ന കാര്യം ആസന്നമാണോ അതല്ല വിദൂരമാണോ എന്നെനിക്കറിഞ്ഞ് കൂടാ.
\end{malayalam}}
\flushright{\begin{Arabic}
\quranayah[21][110]
\end{Arabic}}
\flushleft{\begin{malayalam}
തീര്‍ച്ചയായും സംസാരത്തില്‍ നിന്ന് പരസ്യമായിട്ടുള്ളത് അവന്‍ അറിയും. നിങ്ങള്‍ ഒളിച്ച് വെക്കുന്നതും അവന്‍ അറിയും.
\end{malayalam}}
\flushright{\begin{Arabic}
\quranayah[21][111]
\end{Arabic}}
\flushleft{\begin{malayalam}
എനിക്കറിഞ്ഞ് കൂടാ, ഇത് ഒരു വേള നിങ്ങള്‍ക്കൊരു പരീക്ഷണവും, അല്‍പസമയത്തേക്ക് മാത്രമുള്ള ഒരു സുഖാനുഭവവും ആയേക്കാം.
\end{malayalam}}
\flushright{\begin{Arabic}
\quranayah[21][112]
\end{Arabic}}
\flushleft{\begin{malayalam}
അദ്ദേഹം (നബി) പറഞ്ഞു: എന്‍റെ രക്ഷിതാവേ, നീ യാഥാര്‍ത്ഥ്യമനുസരിച്ച് വിധികല്‍പിക്കേണമേ. നമ്മുടെ രക്ഷിതാവ് പരമകാരുണികനും നിങ്ങള്‍ പറഞ്ഞുണ്ടാക്കുന്നതിനെതിരില്‍ സഹായമര്‍ത്ഥിക്കപ്പെടാവുന്നവനുമത്രെ.
\end{malayalam}}
\chapter{\textmalayalam{ഹജ്ജ് ( തീര്‍ത്ഥാടനം )}}
\begin{Arabic}
\Huge{\centerline{\basmalah}}\end{Arabic}
\flushright{\begin{Arabic}
\quranayah[22][1]
\end{Arabic}}
\flushleft{\begin{malayalam}
മനുഷ്യരേ, നിങ്ങള്‍ നിങ്ങളുടെ രക്ഷിതാവിനെ സൂക്ഷിക്കുക, തീര്‍ച്ചയായും ആ അന്ത്യസമയത്തെ പ്രകമ്പനം ഭയങ്കരമായ ഒരു കാര്യം തന്നെയാകുന്നു.
\end{malayalam}}
\flushright{\begin{Arabic}
\quranayah[22][2]
\end{Arabic}}
\flushleft{\begin{malayalam}
നിങ്ങള്‍ അത് കാണുന്ന ദിവസം ഏതൊരു മുലകൊടുക്കുന്ന മാതാവും താന്‍ മുലയൂട്ടുന്ന കുഞ്ഞിനെപ്പറ്റി അശ്രദ്ധയിലായിപ്പോകും. ഗര്‍ഭവതിയായ ഏതൊരു സ്ത്രീയും തന്‍റെ ഗര്‍ഭത്തിലുള്ളത് പ്രസവിച്ചു പോകുകയും ചെയ്യും. ജനങ്ങളെ മത്തുപിടിച്ചവരായി നിനക്ക് കാണുകയും ചെയ്യാം. (യഥാര്‍ത്ഥത്തില്‍) അവര്‍ ലഹരി ബാധിച്ചവരല്ല.പക്ഷെ, അല്ലാഹുവിന്‍റെ ശിക്ഷ കഠിനമാകുന്നു.
\end{malayalam}}
\flushright{\begin{Arabic}
\quranayah[22][3]
\end{Arabic}}
\flushleft{\begin{malayalam}
യാതൊരു അറിവുമില്ലാതെ അല്ലാഹുവിന്‍റെ കാര്യത്തില്‍ തര്‍ക്കിക്കുകയും, ധിക്കാരിയായ ഏത് പിശാചിനെയും പിന്‍പറ്റുകയും ചെയ്യുന്ന ചിലര്‍ മനുഷ്യരുടെ കൂട്ടത്തിലുണ്ട്‌.
\end{malayalam}}
\flushright{\begin{Arabic}
\quranayah[22][4]
\end{Arabic}}
\flushleft{\begin{malayalam}
അവനെ (പിശാചിനെ) വല്ലവനും മിത്രമായി സ്വീകരിക്കുന്ന പക്ഷം അവന്‍ (പിശാച്‌) തീര്‍ച്ചയായും അവനെ പിഴപ്പിക്കുകയും, ജ്വലിക്കുന്ന നരകശിക്ഷയിലേക്ക് അവനെ നയിക്കുകയും ചെയ്യുന്നതാണ് എന്ന് അവനെ സംബന്ധിച്ച് എഴുതപ്പെട്ടിരിക്കുന്നു.
\end{malayalam}}
\flushright{\begin{Arabic}
\quranayah[22][5]
\end{Arabic}}
\flushleft{\begin{malayalam}
മനുഷ്യരേ, ഉയിര്‍ത്തെഴുന്നേല്‍പിനെ പറ്റി നിങ്ങള്‍ സംശയത്തിലാണെങ്കില്‍ (ആലോചിച്ച് നോക്കുക:) തീര്‍ച്ചയായും നാമാണ് നിങ്ങളെ മണ്ണില്‍ നിന്നും,പിന്നീട് ബീജത്തില്‍ നിന്നും, പിന്നീട് ഭ്രൂണത്തില്‍ നിന്നും, അനന്തരം രൂപം നല്‍കപ്പെട്ടതും രൂപം നല്‍കപ്പെടാത്തതുമായ മാംസപിണ്ഡത്തില്‍ നിന്നും സൃഷ്ടിച്ചത്‌. നാം നിങ്ങള്‍ക്ക് കാര്യങ്ങള്‍ വിശദമാക്കിത്തരാന്‍ വേണ്ടി (പറയുകയാകുന്നു.) നാം ഉദ്ദേശിക്കുന്നതിനെ നിശ്ചിതമായ ഒരു അവധിവരെ നാം ഗര്‍ഭാശയങ്ങളില്‍ താമസിപ്പിക്കുന്നു. പിന്നീട് നിങ്ങളെ നാം ശിശുക്കളായി പുറത്ത് കൊണ്ടു വരുന്നു. അനന്തരം നിങ്ങള്‍ നിങ്ങളുടെ പൂര്‍ണ്ണ ശക്തി പ്രാപിക്കുന്നതു വരെ (നാം നിങ്ങളെ വളര്‍ത്തുന്നു.) (നേരത്തെ) ജീവിതം അവസാനിപ്പിക്കപ്പെടുന്നവരും നിങ്ങളുടെ കൂട്ടത്തിലുണ്ട്‌. അറിവുണ്ടായിരുന്നതിന് ശേഷം യാതൊന്നും അറിയാതാകും വിധം ഏറ്റവും അവശമായ പ്രായത്തിലേക്ക് മടക്കപ്പെടുന്നവരും നിങ്ങളുടെ കൂട്ടത്തിലുണ്ട്‌. ഭൂമി വരണ്ടു നിര്‍ജീവമായി കിടക്കുന്നതായി നിനക്ക് കാണാം. എന്നിട്ട് അതിന്‍മേല്‍ നാം വെള്ളം ചൊരിഞ്ഞാല്‍ അത് ഇളകുകയും വികസിക്കുകയും, കൌതുകമുള്ള എല്ലാതരം ചെടികളേയും അത് മുളപ്പിക്കുകയും ചെയ്യുന്നു.
\end{malayalam}}
\flushright{\begin{Arabic}
\quranayah[22][6]
\end{Arabic}}
\flushleft{\begin{malayalam}
അതെന്തുകൊണ്ടെന്നാല്‍ അല്ലാഹു തന്നെയാണ് സത്യമായുള്ളവന്‍. അവന്‍ മരിച്ചവരെ ജീവിപ്പിക്കും. അവന്‍ ഏത് കാര്യത്തിനും കഴിവുള്ളവനാണ്‌.
\end{malayalam}}
\flushright{\begin{Arabic}
\quranayah[22][7]
\end{Arabic}}
\flushleft{\begin{malayalam}
അന്ത്യസമയം വരിക തന്നെചെയ്യും. അതില്‍ യാതൊരു സംശയവുമില്ല. ഖബ്‌റുകളിലുള്ളവരെ അല്ലാഹു ഉയിര്‍ത്തെഴുന്നേല്‍പിക്കുകയും ചെയ്യും.
\end{malayalam}}
\flushright{\begin{Arabic}
\quranayah[22][8]
\end{Arabic}}
\flushleft{\begin{malayalam}
യാതൊരു അറിവോ, മാര്‍ഗദര്‍ശനമോ, വെളിച്ചം നല്‍കുന്ന ഗ്രന്ഥമോ ഇല്ലാതെ, അല്ലാഹുവിന്‍റെ കാര്യത്തില്‍ തര്‍ക്കിക്കുന്നവനും മനുഷ്യരുടെ കൂട്ടത്തിലുണ്ട്‌.
\end{malayalam}}
\flushright{\begin{Arabic}
\quranayah[22][9]
\end{Arabic}}
\flushleft{\begin{malayalam}
അഹങ്കാരത്തോടെ തിരിഞ്ഞു കൊണ്ട് അല്ലാഹുവിന്‍റെ മാര്‍ഗത്തില്‍ നിന്ന് (ജനങ്ങളെ) തെറ്റിച്ചുകളയാന്‍ വേണ്ടിയത്രെ (അവന്‍ അങ്ങനെ ചെയ്യുന്നത്‌.) ഇഹലോകത്ത് അവന്ന് നിന്ദ്യതയാണുള്ളത്‌. ഉയിര്‍ത്തെഴുന്നേല്‍പിന്‍റെ നാളില്‍ ചുട്ടെരിക്കുന്ന ശിക്ഷ അവന്ന് നാം ആസ്വദിപ്പിക്കുകയും ചെയ്യും.
\end{malayalam}}
\flushright{\begin{Arabic}
\quranayah[22][10]
\end{Arabic}}
\flushleft{\begin{malayalam}
(അന്നവനോട് ഇപ്രകാരം പറയപ്പെടും:) നിന്‍റെ കൈകള്‍ മുന്‍കൂട്ടി ചെയ്തത് നിമിത്തവും, അല്ലാഹു (തന്‍റെ) ദാസന്‍മാരോട് ഒട്ടും അനീതി ചെയ്യുന്നവനല്ല എന്നതിനാലുമത്രെ അത്‌.
\end{malayalam}}
\flushright{\begin{Arabic}
\quranayah[22][11]
\end{Arabic}}
\flushleft{\begin{malayalam}
ഒരു വക്കിലിരുന്നുകൊണ്ട് അല്ലാഹുവെ ആരാധിച്ചു കൊണ്ടിരിക്കുന്നവനും ജനങ്ങളുടെ കൂട്ടത്തിലുണ്ട്‌. അവന്ന് വല്ല ഗുണവും വന്നെത്തുന്ന പക്ഷം അതില്‍ അവന്‍ സമാധാനമടഞ്ഞു കൊള്ളും. അവന്ന് വല്ല പരീക്ഷണവും നേരിട്ടാലോ, അവന്‍ അവന്‍റെ പാട്ടിലേക്കുതന്നെ മറിഞ്ഞു കളയുന്നതാണ്‌. ഇഹലോകവും പരലോകവും അവന്ന് നഷ്ടപ്പെട്ടു. അതു തന്നെയാണ് വ്യക്തമായ നഷ്ടം.
\end{malayalam}}
\flushright{\begin{Arabic}
\quranayah[22][12]
\end{Arabic}}
\flushleft{\begin{malayalam}
അല്ലാഹുവിന് പുറമെ അവന്ന് ഉപദ്രവമോ ഉപകാരമോ ചെയ്യാത്ത വസ്തുക്കളെ അവന്‍ വിളിച്ചു പ്രാര്‍ത്ഥിക്കുന്നു. അതു തന്നെയാണ് വിദൂരമായ വഴികേട്‌.
\end{malayalam}}
\flushright{\begin{Arabic}
\quranayah[22][13]
\end{Arabic}}
\flushleft{\begin{malayalam}
ഏതൊരുത്തനെക്കൊണ്ടുള്ള ഉപദ്രവം അവനെക്കൊണ്ടുള്ള ഉപകാരത്തേക്കാള്‍ അടുത്ത് നില്‍ക്കുന്നുവോ അങ്ങനെയുള്ളവനെത്തന്നെ അവന്‍ വിളിച്ച് പ്രാര്‍ത്ഥിക്കുന്നു. അവന്‍ എത്ര ചീത്ത സഹായി! എത്ര ചീത്ത കൂട്ടുകാരന്‍!
\end{malayalam}}
\flushright{\begin{Arabic}
\quranayah[22][14]
\end{Arabic}}
\flushleft{\begin{malayalam}
വിശ്വസിക്കുകയും, സല്‍കര്‍മ്മങ്ങള്‍ പ്രവര്‍ത്തിക്കുകയും ചെയ്തവരെ താഴ്ഭാഗത്തുകൂടി നദികള്‍ ഒഴുകിക്കൊണ്ടിരിക്കുന്ന സ്വര്‍ഗത്തോപ്പുകളില്‍ അല്ലാഹു പ്രവേശിപ്പിക്കുക തന്നെ ചെയ്യുന്നതാണ്‌. തീര്‍ച്ചയായും അല്ലാഹു താന്‍ ഉദ്ദേശിക്കുന്നത് പ്രവര്‍ത്തിക്കുന്നു.
\end{malayalam}}
\flushright{\begin{Arabic}
\quranayah[22][15]
\end{Arabic}}
\flushleft{\begin{malayalam}
ഇഹലോകത്തും പരലോകത്തും അദ്ദേഹത്തെ (നബിയെ) അല്ലാഹു സഹായിക്കുന്നതേ അല്ല എന്ന് വല്ലവനും വിചാരിക്കുന്നുവെങ്കില്‍ അവന്‍ ആകാശത്തേക്ക് ഒരു കയര്‍ നീട്ടിക്കെട്ടിയിട്ട് (നബിക്ക് കിട്ടുന്ന സഹായം) വിച്ഛേദിച്ചുകൊള്ളട്ടെ. എന്നിട്ട് തന്നെ രോഷം കൊള്ളിക്കുന്ന കാര്യത്തെ (നബിയുടെ വിജയത്തെ) തന്‍റെ തന്ത്രം കൊണ്ട് ഇല്ലാതാക്കാന്‍ കഴിയുമോ എന്ന് അവന്‍ നോക്കട്ടെ.
\end{malayalam}}
\flushright{\begin{Arabic}
\quranayah[22][16]
\end{Arabic}}
\flushleft{\begin{malayalam}
അപ്രകാരം വ്യക്തമായ ദൃഷ്ടാന്തങ്ങളായിക്കൊണ്ട് നാം ഇത് (ഗ്രന്ഥം) അവതരിപ്പിച്ചിരിക്കുന്നു. അല്ലാഹു താന്‍ ഉദ്ദേശിക്കുന്നവരെ നേര്‍വഴിയിലേക്ക് നയിക്കുന്നതുമാണ്‌.
\end{malayalam}}
\flushright{\begin{Arabic}
\quranayah[22][17]
\end{Arabic}}
\flushleft{\begin{malayalam}
സത്യവിശ്വാസികള്‍, യഹൂദന്‍മാര്‍, സാബീമതക്കാര്‍, ക്രിസ്ത്യാനികള്‍, മജൂസികള്‍, ബഹുദൈവവിശ്വാസികള്‍ എന്നിവര്‍ക്കിടയില്‍ ഉയിര്‍ത്തെഴുന്നേല്‍പിന്‍റെ നാളില്‍ തീര്‍ച്ചയായും അല്ലാഹു തീര്‍പ്പുകല്‍പിക്കുന്നതാണ്‌. തീര്‍ച്ചയായും അല്ലാഹു എല്ലാകാര്യത്തിനും സാക്ഷിയാകുന്നു.
\end{malayalam}}
\flushright{\begin{Arabic}
\quranayah[22][18]
\end{Arabic}}
\flushleft{\begin{malayalam}
ആകാശങ്ങളിലുള്ളവരും ഭൂമിയിലുള്ളവരും, സൂര്യനും ചന്ദ്രനും നക്ഷത്രങ്ങളും, പര്‍വ്വതങ്ങളും വൃക്ഷങ്ങളും ജന്തുക്കളും, മനുഷ്യരില്‍ കുറെപേരും അല്ലാഹുവിന് പ്രണാമം അര്‍പ്പിച്ചു കൊണ്ടിരിക്കുന്നു എന്ന് നീ കണ്ടില്ലേ? (വേറെ) കുറെ പേരുടെ കാര്യത്തില്‍ ശിക്ഷ സ്ഥിരപ്പെടുകയും ചെയ്തിരിക്കുന്നു. അല്ലാഹു വല്ലവനെയും അപമാനിതനാക്കുന്ന പക്ഷം അവനെ ബഹുമാനിക്കുവാന്‍ ആരും തന്നെയില്ല. തീര്‍ച്ചയായും അല്ലാഹു താന്‍ ഉദ്ദേശിക്കുന്നത് ചെയ്യുന്നു.
\end{malayalam}}
\flushright{\begin{Arabic}
\quranayah[22][19]
\end{Arabic}}
\flushleft{\begin{malayalam}
ഈ രണ്ടു വിഭാഗം രണ്ട് എതിര്‍കക്ഷികളാകുന്നു. തങ്ങളുടെ രക്ഷിതാവിന്‍റെ കാര്യത്തില്‍ അവര്‍ എതിര്‍വാദക്കാരായി. എന്നാല്‍ അവിശ്വസിച്ചവരാരോ അവര്‍ക്ക് അഗ്നികൊണ്ടുള്ള വസ്ത്രങ്ങള്‍ മുറിച്ചുകൊടുക്കപ്പെടുന്നതാണ്‌. അവരുടെ തലയ്ക്കുമീതെ തിളയ്ക്കുന്ന വെള്ളം ചൊരിയപ്പെടുന്നതാണ്‌.
\end{malayalam}}
\flushright{\begin{Arabic}
\quranayah[22][20]
\end{Arabic}}
\flushleft{\begin{malayalam}
അതു നിമിത്തം അവരുടെ വയറുകളിലുള്ളതും ചര്‍മ്മങ്ങളും ഉരുക്കപ്പെടും.
\end{malayalam}}
\flushright{\begin{Arabic}
\quranayah[22][21]
\end{Arabic}}
\flushleft{\begin{malayalam}
അവര്‍ക്ക് ഇരുമ്പിന്‍റെ ദണ്ഡുകളുമുണ്ടായിരിക്കും.
\end{malayalam}}
\flushright{\begin{Arabic}
\quranayah[22][22]
\end{Arabic}}
\flushleft{\begin{malayalam}
അതില്‍ നിന്ന് കഠിനക്ലേശം നിമിത്തം പുറത്ത് പോകാന്‍ അവര്‍ ഉദ്ദേശിക്കുമ്പോഴെല്ലാം അതിലേക്ക് തന്നെ അവര്‍ മടക്കപ്പെടുന്നതാണ്‌. എരിച്ച് കളയുന്ന ശിക്ഷ നിങ്ങള്‍ ആസ്വദിച്ചു കൊള്ളുക. (എന്ന് അവരോട് പറയപ്പെടുകയും ചെയ്യും.)
\end{malayalam}}
\flushright{\begin{Arabic}
\quranayah[22][23]
\end{Arabic}}
\flushleft{\begin{malayalam}
വിശ്വസിക്കുകയും സല്‍കര്‍മ്മങ്ങള്‍ പ്രവര്‍ത്തിക്കുകയും ചെയ്തവരെ, താഴ്ഭാഗത്തുകൂടി നദികള്‍ ഒഴുകുന്ന സ്വര്‍ഗത്തോപ്പുകളില്‍ തീര്‍ച്ചയായും അല്ലാഹു പ്രവേശിപ്പിക്കുന്നതാണ്‌. അവര്‍ക്കവിടെ സ്വര്‍ണവളകളും മുത്തും അണിയിക്കപ്പെടുന്നതാണ്‌. പട്ടായിരിക്കും അവര്‍ക്ക് അവിടെയുള്ള വസ്ത്രം.
\end{malayalam}}
\flushright{\begin{Arabic}
\quranayah[22][24]
\end{Arabic}}
\flushleft{\begin{malayalam}
വാക്കുകളില്‍ വെച്ച് ഉത്തമമായതിലേക്കാണ് അവര്‍ക്ക് മാര്‍ഗദര്‍ശനം നല്‍കപ്പെട്ടത്‌. സ്തുത്യര്‍ഹനായ അല്ലാഹുവിന്‍റെ പാതയിലേക്കാണ് അവര്‍ക്ക് മാര്‍ഗദര്‍ശനം നല്‍കപ്പെട്ടത്‌.
\end{malayalam}}
\flushright{\begin{Arabic}
\quranayah[22][25]
\end{Arabic}}
\flushleft{\begin{malayalam}
തീര്‍ച്ചയായും സത്യത്തെ നിഷേധിക്കുകയും, അല്ലാഹുവിന്‍റെ മാര്‍ഗത്തില്‍ നിന്നും, മനുഷ്യര്‍ക്ക് -സ്ഥിരവാസിക്കും പരദേശിക്കും - സമാവകാശമുള്ളതായി നാം നിശ്ചയിച്ചിട്ടുള്ള മസ്ജിദുല്‍ ഹറാമില്‍ നിന്നും ജനങ്ങളെ തടഞ്ഞു കൊണ്ടിരിക്കുകയും ചെയ്യുന്നവരാരോ അവര്‍ (കരുതിയിരിക്കട്ടെ). അവിടെ വെച്ച് വല്ലവനും അന്യായമായി ധര്‍മ്മവിരുദ്ധമായ വല്ലതും ചെയ്യാന്‍ ഉദ്ദേശിക്കുന്ന പക്ഷം അവന്ന് വേദനയേറിയ ശിക്ഷയില്‍ നിന്നും നാം ആസ്വദിപ്പിക്കുന്നതാണ്‌.
\end{malayalam}}
\flushright{\begin{Arabic}
\quranayah[22][26]
\end{Arabic}}
\flushleft{\begin{malayalam}
ഇബ്രാഹീമിന് ആ ഭവനത്തിന്‍റെ (കഅ്ബയുടെ) സ്ഥാനം നാം സൌകര്യപ്പെടുത്തികൊടുത്ത സന്ദര്‍ഭം (ശ്രദ്ധേയമത്രെ.) യാതൊരു വസ്തുവെയും എന്നോട് നീ പങ്കുചേര്‍ക്കരുത് എന്നും, ത്വവാഫ് (പ്രദിക്ഷിണം) ചെയ്യുന്നവര്‍ക്ക് വേണ്ടിയും, നിന്നും കുനിഞ്ഞും സാഷ്ടാംഗത്തിലായിക്കൊണ്ടും പ്രാര്‍ത്ഥിക്കുന്നവര്‍ക്ക് വേണ്ടിയും എന്‍റെ ഭവനം ശുദ്ധമാക്കിവെക്കണം എന്നും (നാം അദ്ദേഹത്തോട് നിര്‍ദേശിച്ചു.)
\end{malayalam}}
\flushright{\begin{Arabic}
\quranayah[22][27]
\end{Arabic}}
\flushleft{\begin{malayalam}
(നാം അദ്ദേഹത്തോട് പറഞ്ഞു:) ജനങ്ങള്‍ക്കിടയില്‍ നീ തീര്‍ത്ഥാടനത്തെപറ്റി വിളംബരം ചെയ്യുക. നടന്നുകൊണ്ടും, വിദൂരമായ സകല മലമ്പാതകളിലൂടെയും വരുന്ന എല്ലാ വിധ മെലിഞ്ഞ ഒട്ടകങ്ങളുടെ പുറത്ത് കയറിയും അവര്‍ നിന്‍റെയടുത്ത് വന്നു കൊള്ളും.
\end{malayalam}}
\flushright{\begin{Arabic}
\quranayah[22][28]
\end{Arabic}}
\flushleft{\begin{malayalam}
അവര്‍ക്ക് പ്രയോജനകരമായ രംഗങ്ങളില്‍ അവര്‍ സന്നിഹിതരാകുവാനും, അല്ലാഹു അവര്‍ക്ക് നല്‍കിയിട്ടുള്ള നാല്‍കാലി മൃഗങ്ങളെ നിശ്ചിത ദിവസങ്ങളില്‍ അവന്‍റെ നാമം ഉച്ചരിച്ചു കൊണ്ട് ബലികഴിക്കാനും വേണ്ടിയത്രെ അത്‌. അങ്ങനെ അവയില്‍ നിന്ന് നിങ്ങള്‍ തിന്നുകയും, പരവശനും ദരിദ്രനുമായിട്ടുള്ളവന് ഭക്ഷിക്കാന്‍ കൊടുക്കുകയും ചെയ്യുക.
\end{malayalam}}
\flushright{\begin{Arabic}
\quranayah[22][29]
\end{Arabic}}
\flushleft{\begin{malayalam}
പിന്നെ അവര്‍ തങ്ങളുടെ അഴുക്ക് നീക്കികളയുകയും, തങ്ങളുടെ നേര്‍ച്ചകള്‍ നിറവേറ്റുകയും, പുരാതനമായ ആ ഭവനത്തെ പ്രദക്ഷിണം വെക്കുകയും ചെയ്തുകൊള്ളട്ടെ.
\end{malayalam}}
\flushright{\begin{Arabic}
\quranayah[22][30]
\end{Arabic}}
\flushleft{\begin{malayalam}
അത് (നിങ്ങള്‍ ഗ്രഹിക്കുക.) അല്ലാഹു പവിത്രത നല്‍കിയ വസ്തുക്കളെ വല്ലവനും ബഹുമാനിക്കുന്ന പക്ഷം അത് തന്‍റെ രക്ഷിതാവിന്‍റെ അടുക്കല്‍ അവന്ന് ഗുണകരമായിരിക്കും. നിങ്ങള്‍ക്ക് ഓതികേള്‍പിക്കപ്പെടുന്നതൊഴിച്ചുള്ള കന്നുകാലികള്‍ നിങ്ങള്‍ക്ക് അനുവദിക്കപ്പെട്ടിരിക്കുന്നു. ആകയാല്‍ വിഗ്രഹങ്ങളാകുന്ന മാലിന്യത്തില്‍ നിന്നും നിങ്ങള്‍ അകന്ന് നില്‍ക്കുക. വ്യാജവാക്കില്‍ നിന്നും നിങ്ങള്‍ അകന്ന് നില്‍ക്കുക.
\end{malayalam}}
\flushright{\begin{Arabic}
\quranayah[22][31]
\end{Arabic}}
\flushleft{\begin{malayalam}
വക്രതയില്ലാതെ (ഋജുമാനസരായി) അല്ലാഹുവിലേക്ക് തിരിഞ്ഞവരും, അവനോട് യാതൊന്നും പങ്കുചേര്‍ക്കാത്തവരുമായിരിക്കണം (നിങ്ങള്‍.) അല്ലാഹുവോട് വല്ലവനും പങ്കുചേര്‍ക്കുന്ന പക്ഷം അവന്‍ ആകാശത്തു നിന്ന് വീണത് പോലെയാകുന്നു. അങ്ങനെ പക്ഷികള്‍ അവനെ റാഞ്ചിക്കൊണ്ടു പോകുന്നു. അല്ലെങ്കില്‍ കാറ്റ് അവനെ വിദൂരസ്ഥലത്തേക്ക് കൊണ്ടു പോയി തള്ളുന്നു.
\end{malayalam}}
\flushright{\begin{Arabic}
\quranayah[22][32]
\end{Arabic}}
\flushleft{\begin{malayalam}
അത് (നിങ്ങള്‍ ഗ്രഹിക്കുക.) വല്ലവനും അല്ലാഹുവിന്‍റെ മതചിഹ്നങ്ങളെ ആദരിക്കുന്ന പക്ഷം തീര്‍ച്ചയായും അത് ഹൃദയങ്ങളിലെ ധര്‍മ്മനിഷ്ഠയില്‍ നിന്നുണ്ടാകുന്നതത്രെ.
\end{malayalam}}
\flushright{\begin{Arabic}
\quranayah[22][33]
\end{Arabic}}
\flushleft{\begin{malayalam}
അവയില്‍ നിന്ന് ഒരു നിശ്ചിത അവധിവരെ നിങ്ങള്‍ക്ക് പ്രയോജനങ്ങളെടുക്കാം. പിന്നെ അവയെ ബലികഴിക്കേണ്ട സ്ഥലം ആ പുരാതന ഭവന (കഅ്ബഃ) ത്തിങ്കലാകുന്നു.
\end{malayalam}}
\flushright{\begin{Arabic}
\quranayah[22][34]
\end{Arabic}}
\flushleft{\begin{malayalam}
ഓരോ സമുദായത്തിനും നാം ഓരോ ആരാധനാകര്‍മ്മം നിശ്ചയിച്ചിട്ടുണ്ട്‌. അവര്‍ക്ക് ഉപജീവനത്തിനായി അല്ലാഹു അവര്‍ക്ക് നല്‍കിയിട്ടുള്ള കന്നുകാലിമൃഗങ്ങളെ അവന്‍റെ നാമം ഉച്ചരിച്ചു കൊണ്ട് അവര്‍ അറുക്കേണ്ടതിനു വേണ്ടിയത്രെ അത്‌. നിങ്ങളുടെ ദൈവം ഏകദൈവമാകുന്നു. അതിനാല്‍ അവന്നു മാത്രം നിങ്ങള്‍ കീഴ്പെടുക. (നബിയേ,) വിനീതര്‍ക്ക് നീ സന്തോഷവാര്‍ത്ത അറിയിക്കുക.
\end{malayalam}}
\flushright{\begin{Arabic}
\quranayah[22][35]
\end{Arabic}}
\flushleft{\begin{malayalam}
അല്ലാഹുവെപ്പറ്റി പരാമര്‍ശിക്കപ്പെട്ടാല്‍ ഹൃദയങ്ങള്‍ കിടിലം കൊള്ളുന്നവരും, തങ്ങളെ ബാധിച്ച ആപത്തിനെ ക്ഷമാപൂര്‍വ്വം തരണം ചെയ്യുന്നവരും, നമസ്കാരം മുറപോലെ നിര്‍വഹിക്കുന്നവരും, നാം നല്‍കിയിട്ടുള്ളതില്‍ നിന്ന് ചെലവ് ചെയ്യുന്നവരുമത്രെ അവര്‍.
\end{malayalam}}
\flushright{\begin{Arabic}
\quranayah[22][36]
\end{Arabic}}
\flushleft{\begin{malayalam}
ബലി ഒട്ടകങ്ങളെ നാം നിങ്ങള്‍ക്ക് അല്ലാഹുവിന്‍റെ ചിഹ്നങ്ങളില്‍ പെട്ടതാക്കിയിരിക്കുന്നു. നിങ്ങള്‍ക്കവയില്‍ ഗുണമുണ്ട്‌. അതിനാല്‍ അവയെ വരിവരിയായി നിര്‍ത്തിക്കൊണ്ട് അവയുടെ മേല്‍ നിങ്ങള്‍ അല്ലാഹുവിന്‍റെ നാമം ഉച്ചരി(ച്ചുകൊണ്ട് ബലിയര്‍പ്പി)ക്കുക. അങ്ങനെ അവ പാര്‍ശ്വങ്ങളില്‍ വീണ് കഴിഞ്ഞാല്‍ അവയില്‍ നിന്നെടുത്ത് നിങ്ങള്‍ ഭക്ഷിക്കുകയും, (യാചിക്കാതെ) സംതൃപ്തിയടയുന്നവന്നും, ആവശ്യപ്പെട്ടു വരുന്നവന്നും നിങ്ങള്‍ ഭക്ഷിക്കാന്‍ കൊടുക്കുകയും ചെയ്യുക. നിങ്ങള്‍ നന്ദികാണിക്കുവാന്‍ വേണ്ടി അവയെ നിങ്ങള്‍ക്ക് അപ്രകാരം നാം കീഴ്പെടുത്തിത്തന്നിരിക്കുന്നു.
\end{malayalam}}
\flushright{\begin{Arabic}
\quranayah[22][37]
\end{Arabic}}
\flushleft{\begin{malayalam}
അവയുടെ മാംസമോ രക്തമോ അല്ലാഹുവിങ്കല്‍ എത്തുന്നതേയില്ല. എന്നാല്‍ നിങ്ങളുടെ ധര്‍മ്മനിഷ്ഠയാണ് അവങ്കല്‍ എത്തുന്നത്‌. അല്ലാഹു നിങ്ങള്‍ക്ക് മാര്‍ഗദര്‍ശനം നല്‍കിയതിന്‍റെ പേരില്‍ നിങ്ങള്‍ അവന്‍റെ മഹത്വം പ്രകീര്‍ത്തിക്കേണ്ടതിനായി അപ്രകാരം അവന്‍ അവയെ നിങ്ങള്‍ക്ക് കീഴ്പെടുത്തി തന്നിരിക്കുന്നു. (നബിയേ,) സദ്‌വൃത്തര്‍ക്ക് നീ സന്തോഷവാര്‍ത്ത അറിയിക്കുക.
\end{malayalam}}
\flushright{\begin{Arabic}
\quranayah[22][38]
\end{Arabic}}
\flushleft{\begin{malayalam}
തീര്‍ച്ചയായും സത്യവിശ്വാസികള്‍ക്ക് വേണ്ടി അല്ലാഹു പ്രതിരോധം ഏര്‍പെടുത്തുന്നതാണ്‌. നന്ദികെട്ട വഞ്ചകരെയൊന്നും അല്ലാഹു ഇഷ്ടപ്പെടുകയില്ല; തീര്‍ച്ച
\end{malayalam}}
\flushright{\begin{Arabic}
\quranayah[22][39]
\end{Arabic}}
\flushleft{\begin{malayalam}
യുദ്ധത്തിന്ന് ഇരയാകുന്നവര്‍ക്ക്‌, അവര്‍ മര്‍ദ്ദിതരായതിനാല്‍ (തിരിച്ചടിക്കാന്‍) അനുവാദം നല്‍കപ്പെട്ടിരിക്കുന്നു. തീര്‍ച്ചയായും അല്ലാഹു അവരെ സഹായിക്കാന്‍ കഴിവുള്ളവന്‍ തന്നെയാകുന്നു.
\end{malayalam}}
\flushright{\begin{Arabic}
\quranayah[22][40]
\end{Arabic}}
\flushleft{\begin{malayalam}
യാതൊരു ന്യായവും കൂടാതെ, ഞങ്ങളുടെ രക്ഷിതാവ് അല്ലാഹുവാണ് എന്ന് പറയുന്നതിന്‍റെ പേരില്‍ മാത്രം തങ്ങളുടെ ഭവനങ്ങളില്‍ നിന്ന് പുറത്താക്കപ്പെട്ടവരത്രെ അവര്‍. മനുഷ്യരില്‍ ചിലരെ മറ്റുചിലരെക്കൊണ്ട് അല്ലാഹു തടുക്കുന്നില്ലായിരുന്നുവെങ്കില്‍ സന്യാസിമഠങ്ങളും, ക്രിസ്തീയദേവാലയങ്ങളും, യഹൂദദേവാലയങ്ങളും, അല്ലാഹുവിന്‍റെ നാമം ധാരാളമായി പ്രകീര്‍ത്തിക്കപ്പെടുന്ന മുസ്ലിം പള്ളികളും തകര്‍ക്കപ്പെടുമായിരുന്നു. തന്നെ സഹായിക്കുന്നതാരോ അവനെ തീര്‍ച്ചയായും അല്ലാഹു സഹായിക്കും. തീര്‍ച്ചയായും അല്ലാഹു ശക്തനും പ്രതാപിയും തന്നെയാകുന്നു.
\end{malayalam}}
\flushright{\begin{Arabic}
\quranayah[22][41]
\end{Arabic}}
\flushleft{\begin{malayalam}
ഭൂമിയില്‍ നാം സ്വാധീനം നല്‍കിയാല്‍ നമസ്കാരം മുറപോലെ നിര്‍വഹിക്കുകയും, സകാത്ത് നല്‍കുകയും, സദാചാരം സ്വീകരിക്കാന്‍ കല്‍പിക്കുകയും, ദുരാചാരത്തില്‍ നിന്ന് വിലക്കുകയും ചെയ്യുന്നവരത്രെ അവര്‍ (ആ മര്‍ദ്ദിതര്‍). കാര്യങ്ങളുടെ പര്യവസാനം അല്ലാഹുവിന്നുള്ളതാകുന്നു.
\end{malayalam}}
\flushright{\begin{Arabic}
\quranayah[22][42]
\end{Arabic}}
\flushleft{\begin{malayalam}
(നബിയേ,) നിന്നെ ഇവര്‍ നിഷേധിച്ചു തള്ളുന്ന പക്ഷം ഇവര്‍ക്ക് മുമ്പ് നൂഹിന്‍റെ ജനതയും, ആദും, ഥമൂദും (പ്രവാചകന്‍മാരെ) നിഷേധിച്ച് തള്ളിയിട്ടുണ്ട്‌.
\end{malayalam}}
\flushright{\begin{Arabic}
\quranayah[22][43]
\end{Arabic}}
\flushleft{\begin{malayalam}
ഇബ്രാഹീമിന്‍റെ ജനതയും, ലൂത്വിന്‍റെ ജനതയും.
\end{malayalam}}
\flushright{\begin{Arabic}
\quranayah[22][44]
\end{Arabic}}
\flushleft{\begin{malayalam}
മദ്‌യന്‍ നിവാസികളും (നിഷേധിച്ചിട്ടുണ്ട്‌.) മൂസായും അവിശ്വസിക്കപ്പെട്ടിട്ടുണ്ട്‌. എന്നാല്‍ അവിശ്വാസികള്‍ക്ക് ഞാന്‍ സമയം നീട്ടികൊടുക്കുകയും, പിന്നെ ഞാനവരെ പിടികൂടുകയുമാണ് ചെയ്തത്‌. അപ്പോള്‍ എന്‍റെ പ്രതിഷേധം എങ്ങനെയുണ്ടായിരുന്നു.?
\end{malayalam}}
\flushright{\begin{Arabic}
\quranayah[22][45]
\end{Arabic}}
\flushleft{\begin{malayalam}
എത്രയെത്ര നാടുകള്‍ അവിടത്തുകാര്‍ അക്രമത്തില്‍ ഏര്‍പെട്ടിരിക്കെ നാം നശിപ്പിച്ചു കളഞ്ഞു! അങ്ങനെ അവയതാ മേല്‍പുരകളോടെ വീണടിഞ്ഞ് കിടക്കുന്നു. ഉപയോഗശൂന്യമായിത്തീര്‍ന്ന എത്രയെത്ര കിണറുകള്‍! പടുത്തുയര്‍ത്തിയ എത്രയെത്ര കോട്ടകള്‍!
\end{malayalam}}
\flushright{\begin{Arabic}
\quranayah[22][46]
\end{Arabic}}
\flushleft{\begin{malayalam}
ഇവര്‍ ഭൂമിയിലൂടെ സഞ്ചരിക്കുന്നില്ലേ? എങ്കില്‍ ചിന്തിച്ച് മനസ്സിലാക്കാനുതകുന്ന ഹൃദയങ്ങളോ, കേട്ടറിയാനുതകുന്ന കാതുകളോ അവര്‍ക്കുണ്ടാകുമായിരുന്നു. തീര്‍ച്ചയായും കണ്ണുകളെയല്ല അന്ധത ബാധിക്കുന്നത്‌. പക്ഷെ, നെഞ്ചുകളിലുള്ള ഹൃദയങ്ങളെയാണ് അന്ധത ബാധിക്കുന്നത്‌.
\end{malayalam}}
\flushright{\begin{Arabic}
\quranayah[22][47]
\end{Arabic}}
\flushleft{\begin{malayalam}
(നബിയേ,) നിന്നോട് അവര്‍ ശിക്ഷയുടെ കാര്യത്തില്‍ ധൃതികൂട്ടികൊണ്ടിരിക്കുന്നു. അല്ലാഹു തന്‍റെ വാഗ്ദാനം ലംഘിക്കുകയേ ഇല്ല. തീര്‍ച്ചയായും നിന്‍റെ രക്ഷിതാവിന്‍റെ അടുക്കല്‍ ഒരു ദിവസമെന്നാല്‍ നിങ്ങള്‍ എണ്ണിവരുന്ന തരത്തിലുള്ള ആയിരം കൊല്ലം പോലെയാകുന്നു.)
\end{malayalam}}
\flushright{\begin{Arabic}
\quranayah[22][48]
\end{Arabic}}
\flushleft{\begin{malayalam}
എത്രയോ നാടുകള്‍ക്ക് അവിടത്തുകാര്‍ അക്രമികളായിരിക്കെതന്നെ ഞാന്‍ സമയം നീട്ടികൊടുക്കുകയും, പിന്നീട് ഞാന്‍ അവരെ പിടികൂടുകയും ചെയ്തിട്ടുണ്ട്‌. എന്‍റെ അടുത്തേക്കാകുന്നു (എല്ലാറ്റിന്‍റെയും) മടക്കം.
\end{malayalam}}
\flushright{\begin{Arabic}
\quranayah[22][49]
\end{Arabic}}
\flushleft{\begin{malayalam}
(നബിയേ,) പറയുക: മനുഷ്യരേ, ഞാന്‍ നിങ്ങള്‍ക്ക് വ്യക്തമായ ഒരു താക്കീതുകാരന്‍ മാത്രമാകുന്നു.
\end{malayalam}}
\flushright{\begin{Arabic}
\quranayah[22][50]
\end{Arabic}}
\flushleft{\begin{malayalam}
എന്നാല്‍ വിശ്വസിക്കുകയും, സല്‍കര്‍മ്മങ്ങള്‍ പ്രവര്‍ത്തിക്കുകയും ചെയ്തവരാരോ അവര്‍ക്ക് പാപമോചനവും മാന്യമായ ഉപജീവനവും ഉണ്ടായിരിക്കുന്നതാണ്‌.
\end{malayalam}}
\flushright{\begin{Arabic}
\quranayah[22][51]
\end{Arabic}}
\flushleft{\begin{malayalam}
(നമ്മെ) തോല്‍പിച്ച് കളയാമെന്ന ഭാവത്തില്‍ നമ്മുടെ ദൃഷ്ടാന്തങ്ങളെ വളച്ചൊടിക്കാന്‍ ശ്രമിക്കുന്നവരാരോ അവരത്രെ നരകാവകാശികള്‍.
\end{malayalam}}
\flushright{\begin{Arabic}
\quranayah[22][52]
\end{Arabic}}
\flushleft{\begin{malayalam}
നിനക്ക് മുമ്പ് ഏതൊരു ദൂതനെയും പ്രവാചകനെയും നാം അയച്ചിട്ട്‌, അദ്ദേഹം ഓതികേള്‍പിക്കുന്ന സമയത്ത് ആ ഓതികേള്‍പിക്കുന്ന കാര്യത്തില്‍ പിശാച് (തന്‍റെ ദുര്‍ബോധനം) ചെലുത്തിവിടാതിരുന്നിട്ടില്ല. എന്നാല്‍ പിശാച് ചെലുത്തിവിടുന്നത് അല്ലാഹു മായ്ച്ചുകളയുകയും, എന്നിട്ട് അല്ലാഹു തന്‍റെ വചനങ്ങളെ പ്രബലമാക്കുകയും ചെയ്യും. അല്ലാഹു എല്ലാം അറിയുന്നവനും യുക്തിമാനുമാകുന്നു.
\end{malayalam}}
\flushright{\begin{Arabic}
\quranayah[22][53]
\end{Arabic}}
\flushleft{\begin{malayalam}
ആ പിശാച് കുത്തിച്ചെലുത്തുന്ന കാര്യത്തെ ഹൃദയങ്ങളില്‍ രോഗമുള്ളവര്‍ക്കും, ഹൃദയങ്ങള്‍ കടുത്തുപോയവര്‍ക്കും ഒരു പരീക്ഷണമാക്കിത്തീര്‍ക്കുവാന്‍ വേണ്ടിയത്രെ അത്‌. തീര്‍ച്ചയായും അക്രമികള്‍ (സത്യത്തില്‍ നിന്ന്‌) വിദൂരമായ കക്ഷിമാത്സര്യത്തിലാകുന്നു.
\end{malayalam}}
\flushright{\begin{Arabic}
\quranayah[22][54]
\end{Arabic}}
\flushleft{\begin{malayalam}
വിജ്ഞാനം നല്‍കപ്പെട്ടിട്ടുള്ളവര്‍ക്കാകട്ടെ ഇത് നിന്‍റെ രക്ഷിതാവിങ്കല്‍ നിന്നുള്ള സത്യം തന്നെയാണെന്ന് മനസ്സിലാക്കിയിട്ട് ഇതില്‍ വിശ്വസിക്കുവാനും, അങ്ങനെ അവരുടെ ഹൃദയങ്ങള്‍ ഇതിന്ന് കീഴ്പെടുവാനുമാണ് (അത് ഇടയാക്കുക.) തീര്‍ച്ചയായും അല്ലാഹു സത്യവിശ്വാസികളെ നേരായ പാതയിലേക്ക് നയിക്കുന്നവനാകുന്നു.
\end{malayalam}}
\flushright{\begin{Arabic}
\quranayah[22][55]
\end{Arabic}}
\flushleft{\begin{malayalam}
തങ്ങള്‍ക്ക് അന്ത്യസമയം പെട്ടെന്ന് വന്നെത്തുകയോ, വിനാശകരമായ ഒരു ദിവസത്തെ ശിക്ഷ തങ്ങള്‍ക്ക് വന്നെത്തുകയോ ചെയ്യുന്നത് വരെ ആ അവിശ്വാസികള്‍ ഇതിനെ (സത്യത്തെ) പ്പറ്റി സംശയത്തിലായിക്കൊണേ്ടയിരിക്കും.
\end{malayalam}}
\flushright{\begin{Arabic}
\quranayah[22][56]
\end{Arabic}}
\flushleft{\begin{malayalam}
അന്നേദിവസം ആധിപത്യം അല്ലാഹുവിനായിരിക്കും. അവന്‍ അവര്‍ക്കിടയില്‍ വിധികല്‍പിക്കും. എന്നാല്‍ വിശ്വസിക്കുകയും സല്‍കര്‍മ്മങ്ങള്‍ പ്രവര്‍ത്തിക്കുകയും ചെയ്തവരാരോ അവര്‍ സുഖാനുഭവത്തിന്‍റെ സ്വര്‍ഗത്തോപ്പുകളിലായിരിക്കും.
\end{malayalam}}
\flushright{\begin{Arabic}
\quranayah[22][57]
\end{Arabic}}
\flushleft{\begin{malayalam}
അവിശ്വസിക്കുകയും നമ്മുടെ ദൃഷ്ടാന്തങ്ങളെ നിഷേധിച്ച് തള്ളുകയും ചെയ്തവരാരോ അവര്‍ക്കാണ് അപമാനകരമായ ശിക്ഷയുള്ളത്‌.
\end{malayalam}}
\flushright{\begin{Arabic}
\quranayah[22][58]
\end{Arabic}}
\flushleft{\begin{malayalam}
അല്ലാഹുവിന്‍റെ മാര്‍ഗത്തില്‍ സ്വദേശം വെടിഞ്ഞതിന് ശേഷം കൊല്ലപ്പെടുകയോ, മരിക്കുകയോ ചെയ്തവര്‍ക്ക് തീര്‍ച്ചയായും അല്ലാഹു ഉത്തമമായ ഉപജീവനം നല്‍കുന്നതാണ്‌. തീര്‍ച്ചയായും അല്ലാഹു തന്നെയാണ് ഉപജീവനം നല്‍കുന്നവരില്‍ ഏറ്റവും ഉത്തമന്‍.
\end{malayalam}}
\flushright{\begin{Arabic}
\quranayah[22][59]
\end{Arabic}}
\flushleft{\begin{malayalam}
അവര്‍ക്ക് തൃപ്തികരമായ ഒരു സ്ഥലത്ത് തീര്‍ച്ചയായും അല്ലാഹു അവരെ പ്രവേശിപ്പിക്കുന്നതാണ്‌. തീര്‍ച്ചയായും അല്ലാഹു സര്‍വ്വജ്ഞനും ക്ഷമാശീലനുമാകുന്നു.
\end{malayalam}}
\flushright{\begin{Arabic}
\quranayah[22][60]
\end{Arabic}}
\flushleft{\begin{malayalam}
അത് (അങ്ങനെതന്നെയാകുന്നു.) താന്‍ ശിക്ഷിക്കപ്പെട്ടതിന് തുല്യമായ ശിക്ഷയിലൂടെ വല്ലവനും പ്രതികാരം ചെയ്യുകയും, പിന്നീട് അവന്‍ അതിക്രമത്തിന് ഇരയാവുകയും ചെയ്യുന്ന പക്ഷം തീര്‍ച്ചയായും അല്ലാഹു അവനെ സഹായിക്കുന്നതാണ്‌. തീര്‍ച്ചയായും അല്ലാഹു ഏറെ മാപ്പ് ചെയ്യുന്നവനും പൊറുക്കുന്നവനുമത്രെ.
\end{malayalam}}
\flushright{\begin{Arabic}
\quranayah[22][61]
\end{Arabic}}
\flushleft{\begin{malayalam}
അതെന്തുകൊണ്ടെന്നാല്‍ അല്ലാഹുവാണ് രാവിനെ പകലില്‍ പ്രവേശിപ്പിക്കുകയും, പകലിനെ രാവില്‍ പ്രവേശിപ്പിക്കുകയും ചെയ്യുന്നത്‌. അല്ലാഹുവാണ് എല്ലാം കേള്‍ക്കുകയും കാണുകയും ചെയ്യുന്നവന്‍.
\end{malayalam}}
\flushright{\begin{Arabic}
\quranayah[22][62]
\end{Arabic}}
\flushleft{\begin{malayalam}
അതെന്തുകൊണ്ടെന്നാല്‍ അല്ലാഹുവാണ് സത്യമായിട്ടുള്ളവന്‍. അവനു പുറമെ അവര്‍ ഏതൊന്നിനെ വിളിച്ച് പ്രാര്‍ത്ഥിക്കുന്നുവോ അതുതന്നെയാണ് നിരര്‍ത്ഥകമായിട്ടുള്ളത്‌. അല്ലാഹു തന്നെയാണ് ഉന്നതനും മഹാനുമായിട്ടുള്ളവന്‍.
\end{malayalam}}
\flushright{\begin{Arabic}
\quranayah[22][63]
\end{Arabic}}
\flushleft{\begin{malayalam}
അല്ലാഹു ആകാശത്ത് നിന്ന് വെള്ളമിറക്കിയിട്ട് അതുകൊണ്ടാണ് ഭൂമി പച്ചപിടിച്ചതായിത്തീരുന്നത് എന്ന് നീ മനസ്സിലാക്കിയിട്ടില്ലേ? തീര്‍ച്ചയായും അല്ലാഹു നയജ്ഞനും സൂക്ഷ്മജ്ഞനുമാകുന്നു.
\end{malayalam}}
\flushright{\begin{Arabic}
\quranayah[22][64]
\end{Arabic}}
\flushleft{\begin{malayalam}
അവന്റേതാകുന്നു ആകാശങ്ങളിലുള്ളതും ഭൂമിയിലുള്ളതും. തീര്‍ച്ചയായും അല്ലാഹു പരാശ്രയമുക്തനും സ്തുത്യര്‍ഹനുമാകുന്നു.
\end{malayalam}}
\flushright{\begin{Arabic}
\quranayah[22][65]
\end{Arabic}}
\flushleft{\begin{malayalam}
അല്ലാഹു നിങ്ങള്‍ക്ക് ഭൂമിയിലുള്ളതെല്ലാം കീഴ്പെടുത്തി തന്നിരിക്കുന്നു എന്ന് നീ മനസ്സിലാക്കിയില്ലേ? അവന്‍റെ കല്‍പന പ്രകാരം കടലിലൂടെ സഞ്ചരിക്കുന്ന കപ്പലിനെയും (അവന്‍ കീഴ്പെടുത്തി തന്നിരിക്കുന്നു.) അവന്‍റെ അനുമതി കൂടാതെ ഭൂമിയില്‍ വീണുപോകാത്ത വിധം ഉപരിലോകത്തെ അവന്‍ പിടിച്ചു നിര്‍ത്തുകയും ചെയ്യുന്നു. തീര്‍ച്ചയായും അല്ലാഹു മനുഷ്യരോട് ഏറെ ദയയുള്ളവനും കരുണയുള്ളവനുമാകുന്നു.
\end{malayalam}}
\flushright{\begin{Arabic}
\quranayah[22][66]
\end{Arabic}}
\flushleft{\begin{malayalam}
അവനാണ് നിങ്ങളെ ജീവിപ്പിച്ചവന്‍. പിന്നെ അവന്‍ നിങ്ങളെ മരിപ്പിക്കും. പിന്നെയും അവന്‍ നിങ്ങളെ ജീവിപ്പിക്കുംഠീര്‍ച്ചയായും മനുഷ്യന്‍ ഏറെ നന്ദികെട്ടവന്‍ തന്നെയാകുന്നു.
\end{malayalam}}
\flushright{\begin{Arabic}
\quranayah[22][67]
\end{Arabic}}
\flushleft{\begin{malayalam}
ഓരോ സമുദായത്തിനും നാം ഓരോ ആരാധനാക്രമം നിശ്ചയിച്ചു കൊടുത്തിട്ടുണ്ട്‌. അവര്‍ അതാണ് അനുഷ്ഠിച്ചു വരുന്നത്‌. അതിനാല്‍ ഈ കാര്യത്തില്‍ അവര്‍ നിന്നോട് വഴക്കിടാതിരിക്കട്ടെ. നീ നിന്‍റെ രക്ഷിതാവിങ്കലേക്ക് ക്ഷണിച്ചു കൊള്ളുക. തീര്‍ച്ചയായും നീ വക്രതയില്ലാത്ത സന്‍മാര്‍ഗത്തിലാകുന്നു.
\end{malayalam}}
\flushright{\begin{Arabic}
\quranayah[22][68]
\end{Arabic}}
\flushleft{\begin{malayalam}
അവര്‍ നിന്നോട് തര്‍ക്കിക്കുകയാണെങ്കില്‍ നീ പറഞ്ഞേക്കുക: നിങ്ങള്‍ പ്രവര്‍ത്തിക്കുന്നതിനെപ്പറ്റി അല്ലാഹു നല്ലവണ്ണം അറിയുന്നവനാകുന്നു.
\end{malayalam}}
\flushright{\begin{Arabic}
\quranayah[22][69]
\end{Arabic}}
\flushleft{\begin{malayalam}
നിങ്ങള്‍ ഭിന്നിച്ചു കൊണ്ടിരിക്കുന്ന വിഷയത്തില്‍ ഉയിര്‍ത്തെഴുന്നേല്‍പിന്‍റെ നാളില്‍ അല്ലാഹു നിങ്ങള്‍ക്കിടയില്‍ വിധികല്‍പിച്ചു കൊള്ളും.
\end{malayalam}}
\flushright{\begin{Arabic}
\quranayah[22][70]
\end{Arabic}}
\flushleft{\begin{malayalam}
ആകാശത്തിലും ഭൂമിയിലുമുള്ളത് അല്ലാഹു അറിയുന്നുണ്ടെന്ന് നിനക്ക് അറിഞ്ഞ്കൂടേ? തീര്‍ച്ചയായും അത് ഒരു രേഖയിലുണ്ട്‌. തീര്‍ച്ചയായും അത് അല്ലാഹുവിന് എളുപ്പമുള്ള കാര്യമത്രെ.
\end{malayalam}}
\flushright{\begin{Arabic}
\quranayah[22][71]
\end{Arabic}}
\flushleft{\begin{malayalam}
അല്ലാഹു യാതൊരു പ്രമാണവും അവതരിപ്പിച്ചിട്ടില്ലാത്തതും, അവര്‍ക്ക് തന്നെ യാതൊരു അറിവുമില്ലാത്തതുമായ വസ്തുക്കളെ അവന്ന് പുറമെ അവര്‍ ആരാധിച്ചു കൊണ്ടിരിക്കുന്നു. അക്രമകാരികള്‍ക്ക് യാതൊരു സഹായിയും ഇല്ല.
\end{malayalam}}
\flushright{\begin{Arabic}
\quranayah[22][72]
\end{Arabic}}
\flushleft{\begin{malayalam}
വ്യക്തമായ നിലയില്‍ നമ്മുടെ ദൃഷ്ടാന്തങ്ങള്‍ അവര്‍ക്കു വായിച്ചുകേള്‍പിക്കപ്പെടുകയാണെങ്കില്‍ അവിശ്വാസികളുടെ മുഖങ്ങളില്‍ അനിഷ്ടം (പ്രകടമാകുന്നത്‌) നിനക്ക് മനസ്സിലാക്കാം. നമ്മുടെ ദൃഷ്ടാന്തങ്ങള്‍ അവര്‍ക്ക് വായിച്ചുകേള്‍പിക്കുന്നവരെ കയ്യേറ്റം ചെയ്യാന്‍ തന്നെ അവര്‍ മുതിര്‍ന്നേക്കാം. പറയുക: അതിനെക്കാളെല്ലാം ദോഷകരമായ കാര്യം ഞാന്‍ നിങ്ങള്‍ക്ക് അറിയിച്ച് തരട്ടെയോ? നരകാഗ്നിയത്രെ അത്‌. അവിശ്വാസികള്‍ക്ക് അതാണ് അല്ലാഹു വാഗ്ദാനം ചെയ്തിട്ടുള്ളത്‌. ചെന്നുചേരാനുള്ള ആ സ്ഥലം എത്ര ചീത്ത!
\end{malayalam}}
\flushright{\begin{Arabic}
\quranayah[22][73]
\end{Arabic}}
\flushleft{\begin{malayalam}
മനുഷ്യരേ, ഒരു ഉദാഹരണമിതാ വിവരിക്കപ്പെടുന്നു. നിങ്ങള്‍ അത് ശ്രദ്ധിച്ചു കേള്‍ക്കുക. തീര്‍ച്ചയായും അല്ലാഹുവിന് പുറമെ നിങ്ങള്‍ വിളിച്ചു പ്രാര്‍ത്ഥിക്കുന്നവര്‍ ഒരു ഈച്ചയെപ്പോലും സൃഷ്ടിക്കുകയില്ല. അതിന്നായി അവരെല്ലാവരും ഒത്തുചേര്‍ന്നാല്‍ പോലും. ഈച്ച അവരുടെ പക്കല്‍ നിന്ന് വല്ലതും തട്ടിയെടുത്താല്‍ അതിന്‍റെ പക്കല്‍ നിന്ന് അത് മോചിപ്പിച്ചെടുക്കാനും അവര്‍ക്ക് കഴിയില്ല. അപേക്ഷിക്കുന്നവനും അപേക്ഷിക്കപ്പെടുന്നവനും ദുര്‍ബലര്‍ തന്നെ.
\end{malayalam}}
\flushright{\begin{Arabic}
\quranayah[22][74]
\end{Arabic}}
\flushleft{\begin{malayalam}
അല്ലാഹുവെ കണക്കാക്കേണ്ട മുറപ്രകാരം അവര്‍ കണക്കാക്കിയിട്ടില്ല. തീര്‍ച്ചയായും അല്ലാഹു ശക്തനും പ്രതാപിയും തന്നെയാകുന്നു.
\end{malayalam}}
\flushright{\begin{Arabic}
\quranayah[22][75]
\end{Arabic}}
\flushleft{\begin{malayalam}
മലക്കുകളില്‍ നിന്നും മനുഷ്യരില്‍ നിന്നും അല്ലാഹു ദൂതന്‍മാരെ തെരഞ്ഞെടുക്കുന്നു. തീര്‍ച്ചയായും അല്ലാഹു കേള്‍ക്കുന്നവനും കാണുന്നവനുമത്രെ.
\end{malayalam}}
\flushright{\begin{Arabic}
\quranayah[22][76]
\end{Arabic}}
\flushleft{\begin{malayalam}
അവരുടെ മുമ്പിലുള്ളതും പിന്നിലുള്ളതും അവന്‍ അറിയുന്നു. അല്ലാഹുവിങ്കലേക്കാകുന്നു കാര്യങ്ങള്‍ മടക്കപ്പെടുന്നത്‌.
\end{malayalam}}
\flushright{\begin{Arabic}
\quranayah[22][77]
\end{Arabic}}
\flushleft{\begin{malayalam}
സത്യവിശ്വാസികളേ, നിങ്ങള്‍ കുമ്പിടുകയും, സാഷ്ടാംഗം ചെയ്യുകയും, നിങ്ങളുടെ രക്ഷിതാവിനെ ആരാധിക്കുകയും, നന്‍മ പ്രവര്‍ത്തിക്കുകയും ചെയ്യുക. നിങ്ങള്‍ വിജയം പ്രാപിച്ചേക്കാം.
\end{malayalam}}
\flushright{\begin{Arabic}
\quranayah[22][78]
\end{Arabic}}
\flushleft{\begin{malayalam}
അല്ലാഹുവിന്‍റെ മാര്‍ഗത്തില്‍ സമരം ചെയ്യേണ്ട മുറപ്രകാരം നിങ്ങള്‍ സമരം ചെയ്യുക. അവന്‍ നിങ്ങളെ ഉല്‍കൃഷ്ടരായി തെരഞ്ഞെടുത്തിരിക്കുന്നു. മതകാര്യത്തില്‍ യാതൊരു പ്രയാസവും നിങ്ങളുടെ മേല്‍ അവന്‍ ചുമത്തിയിട്ടില്ല. നിങ്ങളുടെ പിതാവായ ഇബ്രാഹീമിന്‍റെ മാര്‍ഗമത്രെ അത്‌. മുമ്പും (മുന്‍വേദങ്ങളിലും) ഇതിലും (ഈ വേദത്തിലും) അവന്‍ (അല്ലാഹു) നിങ്ങള്‍ക്ക് മുസ്ലിംകളെന്ന് പേര് നല്‍കിയിരിക്കുന്നു. റസൂല്‍ നിങ്ങള്‍ക്ക് സാക്ഷിയായിരിക്കുവാനും, നിങ്ങള്‍ ജനങ്ങള്‍ക്ക് സാക്ഷികളായിരിക്കുവാനും വേണ്ടി. ആകയാല്‍ നിങ്ങള്‍ നമസ്കാരം മുറപോലെ നിര്‍വഹിക്കുകയും, സകാത്ത് നല്‍കുകയും, അല്ലാഹുവെ മുറുകെപിടിക്കുകയും ചെയ്യുക. അവനാണ് നിങ്ങളുടെ രക്ഷാധികാരി. എത്ര നല്ല രക്ഷാധികാരി! എത്ര നല്ല സഹായി!
\end{malayalam}}
\chapter{\textmalayalam{അല്‍ മുഅ്മിനൂന്‍ ( സത്യവിശ്വാസികള്‍ )}}
\begin{Arabic}
\Huge{\centerline{\basmalah}}\end{Arabic}
\flushright{\begin{Arabic}
\quranayah[23][1]
\end{Arabic}}
\flushleft{\begin{malayalam}
സത്യവിശ്വാസികള്‍ വിജയം പ്രാപിച്ചിരിക്കുന്നു.
\end{malayalam}}
\flushright{\begin{Arabic}
\quranayah[23][2]
\end{Arabic}}
\flushleft{\begin{malayalam}
തങ്ങളുടെ നമസ്കാരത്തില്‍ ഭക്തിയുള്ളവരായ,
\end{malayalam}}
\flushright{\begin{Arabic}
\quranayah[23][3]
\end{Arabic}}
\flushleft{\begin{malayalam}
അനാവശ്യകാര്യത്തില്‍ നിന്ന് തിരിഞ്ഞുകളയുന്നവരുമായ,
\end{malayalam}}
\flushright{\begin{Arabic}
\quranayah[23][4]
\end{Arabic}}
\flushleft{\begin{malayalam}
സകാത്ത് നിര്‍വഹിക്കുന്നവരുമായ.
\end{malayalam}}
\flushright{\begin{Arabic}
\quranayah[23][5]
\end{Arabic}}
\flushleft{\begin{malayalam}
തങ്ങളുടെ ഗുഹ്യാവയവങ്ങളെ കാത്തുസൂക്ഷിക്കുന്നവരുമത്രെ അവര്‍.
\end{malayalam}}
\flushright{\begin{Arabic}
\quranayah[23][6]
\end{Arabic}}
\flushleft{\begin{malayalam}
തങ്ങളുടെ ഭാര്യമാരുമായോ, തങ്ങളുടെ അധീനത്തിലുള്ള അടിമസ്ത്രീകളുമായോ ഉള്ള ബന്ധം ഒഴികെ. അപ്പോള്‍ അവര്‍ ആക്ഷേപാര്‍ഹരല്ല.
\end{malayalam}}
\flushright{\begin{Arabic}
\quranayah[23][7]
\end{Arabic}}
\flushleft{\begin{malayalam}
എന്നാല്‍ അതിന്നപ്പുറം ആരെങ്കിലും ആഗ്രഹിക്കുന്ന പക്ഷം അവര്‍ തന്നെയാണ് അതിക്രമകാരികള്‍.
\end{malayalam}}
\flushright{\begin{Arabic}
\quranayah[23][8]
\end{Arabic}}
\flushleft{\begin{malayalam}
തങ്ങളുടെ അനാമത്തുകളും കരാറുകളും പാലിക്കുന്നവരും,
\end{malayalam}}
\flushright{\begin{Arabic}
\quranayah[23][9]
\end{Arabic}}
\flushleft{\begin{malayalam}
തങ്ങളുടെ നമസ്കാരങ്ങള്‍ കൃത്യമായി അനുഷ്ഠിച്ചു പോരുന്നവരുമത്രെ (ആ വിശ്വാസികള്‍.)
\end{malayalam}}
\flushright{\begin{Arabic}
\quranayah[23][10]
\end{Arabic}}
\flushleft{\begin{malayalam}
അവര്‍ തന്നെയാകുന്നു അനന്തരാവകാശികള്‍.
\end{malayalam}}
\flushright{\begin{Arabic}
\quranayah[23][11]
\end{Arabic}}
\flushleft{\begin{malayalam}
അതായത് ഉന്നതമായ സ്വര്‍ഗം അനന്തരാവകാശമായി നേടുന്നവര്‍. അവരതില്‍ നിത്യവാസികളായിരിക്കും.
\end{malayalam}}
\flushright{\begin{Arabic}
\quranayah[23][12]
\end{Arabic}}
\flushleft{\begin{malayalam}
തീര്‍ച്ചയായും മനുഷ്യനെ കളിമണ്ണിന്‍റെ സത്തില്‍ നിന്ന് നാം സൃഷ്ടിച്ചിരിക്കുന്നു.
\end{malayalam}}
\flushright{\begin{Arabic}
\quranayah[23][13]
\end{Arabic}}
\flushleft{\begin{malayalam}
പിന്നീട് ഒരു ബീജമായിക്കൊണ്ട് അവനെ നാം ഭദ്രമായ ഒരു സ്ഥാനത്ത് വെച്ചു.
\end{malayalam}}
\flushright{\begin{Arabic}
\quranayah[23][14]
\end{Arabic}}
\flushleft{\begin{malayalam}
പിന്നെ ആ ബീജത്തെ നാം ഒരു ഭ്രൂണമായി രൂപപ്പെടുത്തി. അനന്തരം ആ ഭ്രൂണത്തെ നാം ഒരു മാംസപിണ്ഡമായി രൂപപ്പെടുത്തി. തുടര്‍ന്ന് നം ആ മാംസപിണ്ഡത്തെ അസ്ഥികൂടമായി രൂപപ്പെടുത്തി. എന്നിട്ട് നാം അസ്ഥികൂടത്തെ മാംസം കൊണ്ട് പൊതിഞ്ഞു. പിന്നീട് മറ്റൊരു സൃഷ്ടിയായി നാം അവനെ വളര്‍ത്തിയെടുത്തു. അപ്പോള്‍ ഏറ്റവും നല്ല സൃഷ്ടികര്‍ത്താവായ അല്ലാഹു അനുഗ്രഹപൂര്‍ണ്ണനായിരിക്കുന്നു.
\end{malayalam}}
\flushright{\begin{Arabic}
\quranayah[23][15]
\end{Arabic}}
\flushleft{\begin{malayalam}
പിന്നീട് തീര്‍ച്ചയായും നിങ്ങള്‍ അതിനു ശേഷം മരിക്കുന്നവരാകുന്നു.
\end{malayalam}}
\flushright{\begin{Arabic}
\quranayah[23][16]
\end{Arabic}}
\flushleft{\begin{malayalam}
പിന്നീട് ഉയിര്‍ത്തെഴുന്നേല്‍പിന്‍റെ നാളില്‍ തീര്‍ച്ചയായും നിങ്ങള്‍ എഴുന്നേല്‍പിക്കപ്പെടുന്നതാണ്‌.
\end{malayalam}}
\flushright{\begin{Arabic}
\quranayah[23][17]
\end{Arabic}}
\flushleft{\begin{malayalam}
തീര്‍ച്ചയായും നിങ്ങള്‍ക്ക് മീതെ നാം ഏഴുപഥങ്ങള്‍ സൃഷ്ടിച്ചിട്ടുണ്ട്‌. സൃഷ്ടിയെപ്പറ്റി നാം അശ്രദ്ധനായിരുന്നിട്ടില്ല.
\end{malayalam}}
\flushright{\begin{Arabic}
\quranayah[23][18]
\end{Arabic}}
\flushleft{\begin{malayalam}
ആകാശത്തു നിന്ന് നാം ഒരു നിശ്ചിത അളവില്‍ വെള്ളം ചൊരിയുകയും, എന്നിട്ട് നാം അതിനെ ഭൂമിയില്‍ തങ്ങിനില്‍ക്കുന്നതാക്കുകയും ചെയ്തിരിക്കുന്നു. അത് വറ്റിച്ചു കളയാന്‍ തീര്‍ച്ചയായും നാം ശക്തനാകുന്നു.
\end{malayalam}}
\flushright{\begin{Arabic}
\quranayah[23][19]
\end{Arabic}}
\flushleft{\begin{malayalam}
അങ്ങനെ അത് (വെള്ളം) കൊണ്ട് നാം നിങ്ങള്‍ക്ക് ഈന്തപ്പനകളുടെയും, മുന്തിരിവള്ളികളുടെയും തോട്ടങ്ങള്‍ വളര്‍ത്തിത്തന്നു. . അവയില്‍ നിങ്ങള്‍ക്ക് ധാരാളം പഴങ്ങളുണ്ട്‌. അവയില്‍ നിന്ന് നിങ്ങള്‍ തിന്നു കൊണ്ടിരിക്കുകയും ചെയ്യുന്നു.
\end{malayalam}}
\flushright{\begin{Arabic}
\quranayah[23][20]
\end{Arabic}}
\flushleft{\begin{malayalam}
സീനാപര്‍വ്വതത്തില്‍ മുളച്ചു വരുന്ന ഒരു മരവും (നാം സൃഷ്ടിച്ചു തന്നിരിക്കുന്നു.) എണ്ണയും, ഭക്ഷണം കഴിക്കുന്നവര്‍ക്ക് കറിയും അത് ഉല്‍പാദിപ്പിക്കുന്നു.
\end{malayalam}}
\flushright{\begin{Arabic}
\quranayah[23][21]
\end{Arabic}}
\flushleft{\begin{malayalam}
തീര്‍ച്ചയായും നിങ്ങള്‍ക്ക് കന്നുകാലികളില്‍ ഒരു ഗുണപാഠമുണ്ട്‌. അവയുടെ ഉദരങ്ങളിലുള്ളതില്‍ നിന്ന് നിങ്ങള്‍ക്ക് നാം കുടിക്കാന്‍ തരുന്നു. നിങ്ങള്‍ക്ക് അവയില്‍ ധാരാളം പ്രയോജനങ്ങളുണ്ട്‌. അവയില്‍ നിന്ന് (മാംസം) നിങ്ങള്‍ ഭക്ഷിക്കുകയും ചെയ്യുന്നു.
\end{malayalam}}
\flushright{\begin{Arabic}
\quranayah[23][22]
\end{Arabic}}
\flushleft{\begin{malayalam}
അവയുടെ പുറത്തും കപ്പലുകളിലും നിങ്ങള്‍ വഹിക്കപ്പെടുകയും ചെയ്യുന്നു.
\end{malayalam}}
\flushright{\begin{Arabic}
\quranayah[23][23]
\end{Arabic}}
\flushleft{\begin{malayalam}
നൂഹിനെ നാം അദ്ദേഹത്തിന്‍റെ ജനതയിലേക്ക് ദൂതനായി അയക്കുകയുണ്ടായി. എന്നിട്ട് അദ്ദേഹം പറഞ്ഞു: എന്‍റെ ജനങ്ങളേ, നിങ്ങള്‍ അല്ലാഹുവെ ആരാധിക്കുക. നിങ്ങള്‍ക്ക് അവനല്ലാതെ യാതൊരു ദൈവവുമില്ല. അതിനാല്‍ നിങ്ങള്‍ സൂക്ഷ്മത പാലിക്കുന്നില്ലേ?
\end{malayalam}}
\flushright{\begin{Arabic}
\quranayah[23][24]
\end{Arabic}}
\flushleft{\begin{malayalam}
അപ്പോള്‍ അദ്ദേഹത്തിന്‍റെ ജനതയിലെ സത്യനിഷേധികളായ പ്രമാണിമാര്‍ പറഞ്ഞു: ഇവന്‍ നിങ്ങളെപ്പോലെയുള്ള ഒരു മനുഷ്യന്‍ മാത്രമാകുന്നു. നിങ്ങളേക്കാളുപരിയായി അവന്‍ മഹത്വം നേടിയെടുക്കാന്‍ ആഗ്രഹിക്കുന്നു. അല്ലാഹു ഉദ്ദേശിച്ചിരുന്നെങ്കില്‍ അവന്‍ (ദൂതന്‍മാരായി) മലക്കുകളെ തന്നെ ഇറക്കുമായിരുന്നു. ഞങ്ങളുടെ പൂര്‍വ്വപിതാക്കള്‍ക്കിടയില്‍ ഇങ്ങനെയൊന്ന് ഞങ്ങള്‍ കേട്ടിട്ടില്ല.
\end{malayalam}}
\flushright{\begin{Arabic}
\quranayah[23][25]
\end{Arabic}}
\flushleft{\begin{malayalam}
ഇവന്‍ ഭ്രാന്ത് ബാധിച്ച ഒരു മനുഷ്യന്‍ മാത്രമാകുന്നു. അതിനാല്‍ കുറച്ചുകാലം വരെ ഇവന്‍റെ കാര്യത്തില്‍ നിങ്ങള്‍ കാത്തിരിക്കുവിന്‍.
\end{malayalam}}
\flushright{\begin{Arabic}
\quranayah[23][26]
\end{Arabic}}
\flushleft{\begin{malayalam}
അദ്ദേഹം പറഞ്ഞു: എന്‍റെ രക്ഷിതാവേ, ഇവരെന്നെ നിഷേധിച്ചു തള്ളിയിരിക്കയാല്‍ നീ എന്നെ സഹായിക്കേണമേ.
\end{malayalam}}
\flushright{\begin{Arabic}
\quranayah[23][27]
\end{Arabic}}
\flushleft{\begin{malayalam}
അപ്പോള്‍ നാം അദ്ദേഹത്തിന് ഇപ്രകാരം ബോധനം നല്‍കി: നമ്മുടെ മേല്‍നോട്ടത്തിലും, നമ്മുടെ നിര്‍ദേശമനുസരിച്ചും നീ കപ്പല്‍ നിര്‍മിച്ചു കൊള്ളുക. അങ്ങനെ നമ്മുടെ കല്‍പന വരുകയും, അടുപ്പില്‍ നിന്ന് ഉറവ് പൊട്ടുകയും ചെയ്താല്‍ എല്ലാ വസ്തുക്കളില്‍ നിന്നും രണ്ട് ഇണകളെയും, നിന്‍റെ കുടുംബത്തെയും നീ അതില്‍ കയറ്റികൊള്ളുക. അവരുടെ കൂട്ടത്തില്‍ ആര്‍ക്കെതിരില്‍ (ശിക്ഷയുടെ) വചനം മുന്‍കൂട്ടി ഉണ്ടായിട്ടുണ്ടോ അവരൊഴികെ. അക്രമം ചെയ്തവരുടെ കാര്യത്തില്‍ നീ എന്നോട് സംസാരിച്ചു പോകരുത്‌. തീര്‍ച്ചയായും അവര്‍ മുക്കി നശിപ്പിക്കപ്പെടുന്നതാണ്‌.
\end{malayalam}}
\flushright{\begin{Arabic}
\quranayah[23][28]
\end{Arabic}}
\flushleft{\begin{malayalam}
അങ്ങനെ നീയും നിന്‍റെ കൂടെയുള്ളവരും കപ്പലില്‍ കയറിക്കഴിഞ്ഞാല്‍ നീ പറയുക: അക്രമകാരികളില്‍ നിന്ന് ഞങ്ങളെ രക്ഷിച്ച അല്ലാഹുവിന് സ്തുതി.
\end{malayalam}}
\flushright{\begin{Arabic}
\quranayah[23][29]
\end{Arabic}}
\flushleft{\begin{malayalam}
എന്‍റെ രക്ഷിതാവേ, അനുഗൃഹീതമായ ഒരു താവളത്തില്‍ നീ എന്നെ ഇറക്കിത്തരേണമേ. നീയാണല്ലോ ഇറക്കിത്തരുന്നവരില്‍ ഏറ്റവും ഉത്തമന്‍ എന്നും പറയുക.
\end{malayalam}}
\flushright{\begin{Arabic}
\quranayah[23][30]
\end{Arabic}}
\flushleft{\begin{malayalam}
തീര്‍ച്ചയായും അതില്‍ (പ്രളയത്തില്‍) പല ദൃഷ്ടാന്തങ്ങളുമുണ്ട്‌. തീര്‍ച്ചയായും നാം പരീക്ഷണം നടത്തുന്നവന്‍ തന്നെയാകുന്നു.
\end{malayalam}}
\flushright{\begin{Arabic}
\quranayah[23][31]
\end{Arabic}}
\flushleft{\begin{malayalam}
പിന്നീട് അവര്‍ക്ക് ശേഷം നാം മറ്റൊരു തലമുറയെ വളര്‍ത്തിയെടുത്തു.
\end{malayalam}}
\flushright{\begin{Arabic}
\quranayah[23][32]
\end{Arabic}}
\flushleft{\begin{malayalam}
അപ്പോള്‍ അവരില്‍ നിന്ന് തന്നെയുള്ള ഒരു ദൂതനെ അവരിലേക്ക് നാം അയച്ചു. (അദ്ദേഹം പറഞ്ഞു:) നിങ്ങള്‍ അല്ലാഹുവെ ആരാധിക്കുക. നിങ്ങള്‍ക്ക് അവനല്ലാതെ ഒരു ദൈവവുമില്ല. അതിനാല്‍ നിങ്ങള്‍ സൂക്ഷ്മത പാലിക്കുന്നില്ലേ?
\end{malayalam}}
\flushright{\begin{Arabic}
\quranayah[23][33]
\end{Arabic}}
\flushleft{\begin{malayalam}
അദ്ദേഹത്തിന്‍റെ ജനതയില്‍ നിന്ന് അവിശ്വസിച്ചവരും, പരലോകത്തെ കണ്ടുമുട്ടുന്നതിനെ നിഷേധിച്ചു കളഞ്ഞവരും, ഐഹികജീവിതത്തില്‍ നാം സുഖാഡംബരങ്ങള്‍ നല്‍കിയവരുമായ പ്രമാണിമാര്‍ പറഞ്ഞു: ഇവന്‍ നിങ്ങളെപ്പോലെയുള്ള ഒരു മനുഷ്യന്‍ മാത്രമാകുന്നു. നിങ്ങള്‍ തിന്നുന്ന തരത്തിലുള്ളത് തന്നെയാണ് അവന്‍ തിന്നുന്നത്‌. നിങ്ങള്‍ കുടിക്കുന്ന തരത്തിലുള്ളത് തന്നെയാണ് അവനും കുടിക്കുന്നത്‌.
\end{malayalam}}
\flushright{\begin{Arabic}
\quranayah[23][34]
\end{Arabic}}
\flushleft{\begin{malayalam}
നിങ്ങളെപ്പോലെയുള്ള ഒരു മനുഷ്യനെ നിങ്ങള്‍ അനുസരിക്കുകയാണെങ്കില്‍ തീര്‍ച്ചയായും നിങ്ങളപ്പോള്‍ നഷ്ടക്കാര്‍ തന്നെയാകുന്നു.
\end{malayalam}}
\flushright{\begin{Arabic}
\quranayah[23][35]
\end{Arabic}}
\flushleft{\begin{malayalam}
നിങ്ങള്‍ മരിക്കുകയും, മണ്ണും അസ്ഥിശകലങ്ങളുമായിത്തീരുകയും ചെയ്താല്‍ നിങ്ങള്‍ (വീണ്ടും ജീവനോടെ) പുറത്ത് കൊണ്ടു വരപ്പെടും എന്നാണോ അവന്‍ നിങ്ങള്‍ക്ക് വാഗ്ദാനം നല്‍കുന്നത്‌?
\end{malayalam}}
\flushright{\begin{Arabic}
\quranayah[23][36]
\end{Arabic}}
\flushleft{\begin{malayalam}
നിങ്ങള്‍ക്ക് നല്‍കപ്പെടുന്ന ആ വാഗ്ദാനം എത്രയെത്ര വിദൂരം!
\end{malayalam}}
\flushright{\begin{Arabic}
\quranayah[23][37]
\end{Arabic}}
\flushleft{\begin{malayalam}
ജീവിതമെന്നത് നമ്മുടെ ഈ ഐഹികജീവിതം മാത്രമാകുന്നു. നാം മരിക്കുന്നു. നാം ജനിക്കുന്നു. നാം ഉയിര്‍ത്തെഴുന്നേല്‍പിക്കപ്പെടുന്നവരല്ല തന്നെ.
\end{malayalam}}
\flushright{\begin{Arabic}
\quranayah[23][38]
\end{Arabic}}
\flushleft{\begin{malayalam}
ഇവന്‍ അല്ലാഹുവിന്‍റെ മേല്‍ കള്ളം കെട്ടിച്ചമച്ച ഒരു പുരുഷന്‍ മാത്രമാകുന്നു. ഞങ്ങള്‍ അവനെ വിശ്വസിക്കുന്നവരേ അല്ല.
\end{malayalam}}
\flushright{\begin{Arabic}
\quranayah[23][39]
\end{Arabic}}
\flushleft{\begin{malayalam}
അദ്ദേഹം പറഞ്ഞു: എന്‍റെ രക്ഷിതാവേ, ഇവര്‍ എന്നെ നിഷേധിച്ചു തള്ളിയിരിക്കയാല്‍ നീ എന്നെ സഹായിക്കേണമേ.
\end{malayalam}}
\flushright{\begin{Arabic}
\quranayah[23][40]
\end{Arabic}}
\flushleft{\begin{malayalam}
അവന്‍ (അല്ലാഹു) പറഞ്ഞു: അടുത്തു തന്നെ അവര്‍ ഖേദിക്കുന്നവരായിത്തീരും.
\end{malayalam}}
\flushright{\begin{Arabic}
\quranayah[23][41]
\end{Arabic}}
\flushleft{\begin{malayalam}
അങ്ങനെ ഒരു കഠോര ശബ്ദം യഥാര്‍ത്ഥമായും അവരെ പിടികൂടി. എന്നിട്ട് നാം അവരെ വെറും ചവറാക്കിക്കളഞ്ഞു. അപ്പോള്‍ അക്രമികളായ ജനങ്ങള്‍ക്ക് നാശം!
\end{malayalam}}
\flushright{\begin{Arabic}
\quranayah[23][42]
\end{Arabic}}
\flushleft{\begin{malayalam}
പിന്നെ അവര്‍ക്ക് ശേഷം വേറെ തലമുറകളെ നാം വളര്‍ത്തിയെടുത്തു.
\end{malayalam}}
\flushright{\begin{Arabic}
\quranayah[23][43]
\end{Arabic}}
\flushleft{\begin{malayalam}
ഒരു സമുദായവും അതിന്‍റെ അവധി വിട്ട് മുന്നോട്ട് പോകുകയോ പിന്നോട്ട് പോകുകയോ ഇല്ല.
\end{malayalam}}
\flushright{\begin{Arabic}
\quranayah[23][44]
\end{Arabic}}
\flushleft{\begin{malayalam}
പിന്നെ നാം നമ്മുടെ ദൂതന്‍മാരെ തുടരെത്തുടരെ അയച്ചു കൊണ്ടിരുന്നു. ഓരോ സമുദായത്തിന്‍റെ അടുക്കലും അവരിലേക്കുള്ള ദൂതന്‍ ചെല്ലുമ്പോഴൊക്കെ അവര്‍ അദ്ദേഹത്തെ നിഷേധിച്ചു തള്ളുകയാണ് ചെയ്തത്‌. അപ്പോള്‍ അവരെ ഒന്നിനുപുറകെ മറ്റൊന്നായി നാം നശിപ്പിച്ചു. അവരെ നാം സംസാരവിഷയമാക്കിത്തീര്‍ക്കുകയും ചെയ്തു. ആകയാല്‍ വിശ്വസിക്കാത്ത ജനങ്ങള്‍ക്ക് നാശം!
\end{malayalam}}
\flushright{\begin{Arabic}
\quranayah[23][45]
\end{Arabic}}
\flushleft{\begin{malayalam}
പിന്നീട് മൂസായെയും അദ്ദേഹത്തിന്‍റെ സഹോദരന്‍ ഹാറൂനെയും നമ്മുടെ ദൃഷ്ടാന്തങ്ങളോടും, വ്യക്തമായ പ്രമാണത്തോടും കൂടി നാം അയക്കുകയുണ്ടായി.
\end{malayalam}}
\flushright{\begin{Arabic}
\quranayah[23][46]
\end{Arabic}}
\flushleft{\begin{malayalam}
ഫിര്‍ഔന്‍റെയും, അവന്‍റെ പ്രമാണിസംഘത്തിന്‍റെയും അടുത്തേക്ക്‌. അപ്പോള്‍ അവര്‍ അഹംഭാവം നടിക്കുകയാണ് ചെയ്തത്‌. അവര്‍ പൊങ്ങച്ചക്കാരായ ഒരു ജനതയായിരുന്നു.
\end{malayalam}}
\flushright{\begin{Arabic}
\quranayah[23][47]
\end{Arabic}}
\flushleft{\begin{malayalam}
അതിനാല്‍ അവര്‍ പറഞ്ഞു: നമ്മളെപ്പോലെയുള്ള രണ്ടുമനുഷ്യന്‍മാരെ നാം വിശ്വസിക്കുകയോ? അവരുടെ ജനതയാകട്ടെ നമുക്ക് കീഴ്‌വണക്കം ചെയ്യുന്നവരാണ് താനും.
\end{malayalam}}
\flushright{\begin{Arabic}
\quranayah[23][48]
\end{Arabic}}
\flushleft{\begin{malayalam}
അങ്ങനെ അവരെ രണ്ടുപേരെയും അവര്‍ നിഷേധിച്ചു തള്ളിക്കളഞ്ഞു. തന്നിമിത്തം അവര്‍ നശിപ്പിക്കപ്പെട്ടവരുടെ കൂട്ടത്തിലായിത്തീര്‍ന്നു.
\end{malayalam}}
\flushright{\begin{Arabic}
\quranayah[23][49]
\end{Arabic}}
\flushleft{\begin{malayalam}
അവര്‍ (ജനങ്ങള്‍) സന്‍മാര്‍ഗം കണ്ടെത്തുന്നതിന് വേണ്ടി മൂസായ്ക്ക് നാം വേദഗ്രന്ഥം നല്‍കുകയുണ്ടായി.
\end{malayalam}}
\flushright{\begin{Arabic}
\quranayah[23][50]
\end{Arabic}}
\flushleft{\begin{malayalam}
മര്‍യമിന്‍റെ പുത്രനെയും അവന്‍റെ മാതാവിനെയും നാം ഒരു ദൃഷ്ടാന്തമാക്കിയിരിക്കുന്നു. നിവാസയോഗ്യമായതും ഒരു നീരുറവുള്ളതുമായ ഒരു ഉയര്‍ന്ന പ്രദേശത്ത് അവര്‍ ഇരുവര്‍ക്കും നാം അഭയം നല്‍കുകയും ചെയ്തു.
\end{malayalam}}
\flushright{\begin{Arabic}
\quranayah[23][51]
\end{Arabic}}
\flushleft{\begin{malayalam}
ഹേ; ദൂതന്‍മാരേ, വിശിഷ്ടവസ്തുക്കളില്‍ നിന്ന് നിങ്ങള്‍ ഭക്ഷിക്കുകയും, സല്‍കര്‍മ്മം പ്രവര്‍ത്തിക്കുകയും ചെയ്യുവിന്‍. തീര്‍ച്ചയായും ഞാന്‍ നിങ്ങള്‍ പ്രവര്‍ത്തിക്കുന്നതിനെപ്പറ്റി അറിയുന്നവനാകുന്നു.
\end{malayalam}}
\flushright{\begin{Arabic}
\quranayah[23][52]
\end{Arabic}}
\flushleft{\begin{malayalam}
തീര്‍ച്ചയായും ഇതാണ് നിങ്ങളുടെ സമുദായം. ഏകസമുദായം. ഞാനാണ് നിങ്ങളുടെ രക്ഷിതാവ്‌. അതിനാല്‍ നിങ്ങള്‍ എന്നെ സൂക്ഷിച്ചു ജീവിക്കുവിന്‍.
\end{malayalam}}
\flushright{\begin{Arabic}
\quranayah[23][53]
\end{Arabic}}
\flushleft{\begin{malayalam}
എന്നാല്‍ അവര്‍ (ജനങ്ങള്‍) കക്ഷികളായിപിരിഞ്ഞു കൊണ്ട് തങ്ങളുടെ കാര്യത്തില്‍ പരസ്പരം ഭിന്നിക്കുകയാണുണ്ടായത്‌. ഓരോ കക്ഷിയും തങ്ങളുടെ പക്കലുള്ളതു കൊണ്ട് സംതൃപ്തി അടയുന്നവരാകുന്നു.
\end{malayalam}}
\flushright{\begin{Arabic}
\quranayah[23][54]
\end{Arabic}}
\flushleft{\begin{malayalam}
(നബിയേ,) അതിനാല്‍ ഒരു സമയം വരെ അവരെ അവരുടെ വഴികേടിലായിക്കൊണ്ട് വിട്ടേക്കുക.
\end{malayalam}}
\flushright{\begin{Arabic}
\quranayah[23][55]
\end{Arabic}}
\flushleft{\begin{malayalam}
അവര്‍ വിചാരിക്കുന്നുണ്ടോ; സ്വത്തും സന്താനങ്ങളും നല്‍കി നാം അവരെ സഹായിച്ചു കൊണ്ടിരിക്കുന്നത്‌
\end{malayalam}}
\flushright{\begin{Arabic}
\quranayah[23][56]
\end{Arabic}}
\flushleft{\begin{malayalam}
നാം അവര്‍ക്ക് നന്‍മകള്‍ നല്‍കാന്‍ ധൃതി കാണിക്കുന്നതാണെന്ന് ? അവര്‍ (യാഥാര്‍ത്ഥ്യം) ഗ്രഹിക്കുന്നില്ല.
\end{malayalam}}
\flushright{\begin{Arabic}
\quranayah[23][57]
\end{Arabic}}
\flushleft{\begin{malayalam}
തീര്‍ച്ചയായും തങ്ങളുടെ രക്ഷിതാവിനെപ്പറ്റിയുള്ള ഭയത്താല്‍ നടുങ്ങുന്നവര്‍,
\end{malayalam}}
\flushright{\begin{Arabic}
\quranayah[23][58]
\end{Arabic}}
\flushleft{\begin{malayalam}
തങ്ങളുടെ രക്ഷിതാവിന്‍റെ ദൃഷ്ടാന്തങ്ങളില്‍ വിശ്വസിക്കുന്നവരും,
\end{malayalam}}
\flushright{\begin{Arabic}
\quranayah[23][59]
\end{Arabic}}
\flushleft{\begin{malayalam}
തങ്ങളുടെ രക്ഷിതാവിനോട് പങ്കുചേര്‍ക്കാത്തവരും,
\end{malayalam}}
\flushright{\begin{Arabic}
\quranayah[23][60]
\end{Arabic}}
\flushleft{\begin{malayalam}
രക്ഷിതാവിങ്കലേക്ക് തങ്ങള്‍ മടങ്ങിച്ചെല്ലേണ്ടവരാണല്ലോ എന്ന് മനസ്സില്‍ ഭയമുള്ളതോടു കൂടി തങ്ങള്‍ ദാനം ചെയ്യുന്നതെല്ലാം ദാനം ചെയ്യുന്നവരും ആരോ
\end{malayalam}}
\flushright{\begin{Arabic}
\quranayah[23][61]
\end{Arabic}}
\flushleft{\begin{malayalam}
അവരത്രെ നന്‍മകളില്‍ ധൃതിപ്പെട്ട് മുന്നേറുന്നവര്‍. അവരത്രെ അവയില്‍ മുമ്പേ ചെന്നെത്തുന്നവരും.
\end{malayalam}}
\flushright{\begin{Arabic}
\quranayah[23][62]
\end{Arabic}}
\flushleft{\begin{malayalam}
ഒരാളോടും അയാളുടെ കഴിവില്‍ പെട്ടതല്ലാതെ നാം ശാസിക്കുകയില്ല. സത്യം തുറന്നുപറയുന്ന ഒരു രേഖ നമ്മുടെ പക്കലുണ്ട്‌. അവരോട് അനീതി കാണിക്കപ്പെടുന്നതല്ല.
\end{malayalam}}
\flushright{\begin{Arabic}
\quranayah[23][63]
\end{Arabic}}
\flushleft{\begin{malayalam}
പക്ഷെ, അവരുടെ ഹൃദയങ്ങള്‍ ഈ കാര്യത്തെപ്പറ്റി അശ്രദ്ധയിലാകുന്നു. അവര്‍ക്ക് അത് കൂടാതെയുള്ള ചില പ്രവൃത്തികളാണുള്ളത്‌. അവര്‍ അത് ചെയ്തുകൊണ്ടിരിക്കുകയാകുന്നു.
\end{malayalam}}
\flushright{\begin{Arabic}
\quranayah[23][64]
\end{Arabic}}
\flushleft{\begin{malayalam}
അങ്ങനെ അവരിലെ സുഖലോലുപന്‍മാരെ ശിക്ഷയിലൂടെ നാം പിടികൂടിയപ്പോള്‍ അവരതാ നിലവിളികൂട്ടുന്നു.
\end{malayalam}}
\flushright{\begin{Arabic}
\quranayah[23][65]
\end{Arabic}}
\flushleft{\begin{malayalam}
(നാം പറയും:) നിങ്ങളിന്ന് നിലവിളി കൂട്ടേണ്ട. തീര്‍ച്ചയായും നിങ്ങള്‍ക്ക് നമ്മുടെ പക്കല്‍ നിന്ന് സഹായം നല്‍കപ്പെടുകയില്ല.
\end{malayalam}}
\flushright{\begin{Arabic}
\quranayah[23][66]
\end{Arabic}}
\flushleft{\begin{malayalam}
എന്‍റെ തെളിവുകള്‍ നിങ്ങള്‍ക്ക് ഓതികേള്‍പിക്കപ്പെടാറുണ്ടായിരുന്നു. അപ്പോള്‍ നിങ്ങള്‍ പുറം തിരിഞ്ഞുപോകുകയായിരുന്നു.
\end{malayalam}}
\flushright{\begin{Arabic}
\quranayah[23][67]
\end{Arabic}}
\flushleft{\begin{malayalam}
പൊങ്ങച്ചം നടിച്ചുകൊണ്ട്‌, ഒരു രാക്കഥയെന്നോണം നിങ്ങള്‍ അതിനെപ്പറ്റി (ഖുര്‍ആനെപ്പറ്റി) അസംബന്ധങ്ങള്‍ പുലമ്പുകയായിരുന്നു.
\end{malayalam}}
\flushright{\begin{Arabic}
\quranayah[23][68]
\end{Arabic}}
\flushleft{\begin{malayalam}
ഈ വാക്കിനെ (ഖുര്‍ആനിനെ) പ്പറ്റി അവര്‍ ആലോചിച്ച് നോക്കിയിട്ടില്ലേ? അതല്ല, അവരുടെ പൂര്‍വ്വപിതാക്കള്‍ക്ക് വന്നിട്ടില്ലാത്ത ഒരു കാര്യമാണോ അവര്‍ക്ക് വന്നുകിട്ടിയിരിക്കുന്നത് ?
\end{malayalam}}
\flushright{\begin{Arabic}
\quranayah[23][69]
\end{Arabic}}
\flushleft{\begin{malayalam}
അതല്ല അവരുടെ ദൂതനെ അവര്‍ക്ക് പരിചയമില്ലാഞ്ഞിട്ടാണോ അവര്‍ അദ്ദേഹത്തെ നിഷേധിക്കുന്നത് ?
\end{malayalam}}
\flushright{\begin{Arabic}
\quranayah[23][70]
\end{Arabic}}
\flushleft{\begin{malayalam}
അതല്ല, അദ്ദേഹത്തിന് ഭ്രാന്തുണ്ടെന്നാണോ അവര്‍ പറയുന്നത്‌? അല്ല, അദ്ദേഹം അവരുടെയടുക്കല്‍ സത്യവും കൊണ്ട് വന്നിരിക്കയാണ്‌. എന്നാല്‍ അവരില്‍ അധികപേരും സത്യത്തെ വെറുക്കുന്നവരത്രെ.
\end{malayalam}}
\flushright{\begin{Arabic}
\quranayah[23][71]
\end{Arabic}}
\flushleft{\begin{malayalam}
സത്യം അവരുടെ തന്നിഷ്ടങ്ങളെ പിന്‍പറ്റിയിരുന്നെങ്കില്‍ ആകാശങ്ങളും ഭൂമിയും അവയിലുള്ളവരുമെല്ലാം കുഴപ്പത്തിലാകുമായിരുന്നു. അല്ല, അവര്‍ക്കുള്ള ഉല്‍ബോധനവും കൊണ്ടാണ് നാം അവരുടെ അടുത്ത് ചെന്നിരിക്കുന്നത്‌. എന്നിട്ട് അവര്‍ തങ്ങള്‍ക്കുള്ള ഉല്‍ബോധനത്തില്‍ നിന്ന് തിരിഞ്ഞുകളയുകയാകുന്നു.
\end{malayalam}}
\flushright{\begin{Arabic}
\quranayah[23][72]
\end{Arabic}}
\flushleft{\begin{malayalam}
അതല്ല, നീ അവരോട് വല്ല പ്രതിഫലവും ചോദിക്കുന്നുണ്ടോ? എന്നാല്‍ നിന്‍റെ രക്ഷിതാവിങ്കല്‍ നിന്നുള്ള പ്രതിഫലമാകുന്നു ഏറ്റവും ഉത്തമമായിട്ടുള്ളത്‌. അവന്‍ ഉപജീവനം നല്‍കുന്നവരുടെ കൂട്ടത്തില്‍ ഉത്തമനാകുന്നു.
\end{malayalam}}
\flushright{\begin{Arabic}
\quranayah[23][73]
\end{Arabic}}
\flushleft{\begin{malayalam}
തീര്‍ച്ചയായും നീ അവരെ നേരായ പാതയിലേക്കാകുന്നു ക്ഷണിക്കുന്നത്‌.
\end{malayalam}}
\flushright{\begin{Arabic}
\quranayah[23][74]
\end{Arabic}}
\flushleft{\begin{malayalam}
പരലോകത്തില്‍ വിശ്വസിക്കാത്തവര്‍ ആ പാതയില്‍ നിന്ന് തെറ്റിപ്പോകുന്നവരാകുന്നു.
\end{malayalam}}
\flushright{\begin{Arabic}
\quranayah[23][75]
\end{Arabic}}
\flushleft{\begin{malayalam}
നാം അവരോട് കരുണ കാണിക്കുകയും, അവരിലുള്ള കഷ്ടത നീക്കുകയും ചെയ്തിരുന്നുവെങ്കില്‍ അവര്‍ തങ്ങളുടെ ധിക്കാരത്തില്‍ വിഹരിക്കുന്ന അവസ്ഥയില്‍ തന്നെ ശഠിച്ചുനില്‍ക്കുമായിരുന്നു.
\end{malayalam}}
\flushright{\begin{Arabic}
\quranayah[23][76]
\end{Arabic}}
\flushleft{\begin{malayalam}
നാം അവരെ ശിക്ഷയുമായി പിടികൂടുകയുണ്ടായി. എന്നിട്ടവര്‍ തങ്ങളുടെ രക്ഷിതാവിന് കീഴൊതുങ്ങിയില്ല. അവര്‍ താഴ്മ കാണിക്കുന്നുമില്ല.
\end{malayalam}}
\flushright{\begin{Arabic}
\quranayah[23][77]
\end{Arabic}}
\flushleft{\begin{malayalam}
അങ്ങനെ നാം അവരുടെ നേരെ കഠിനശിക്ഷയുടെ ഒരു കവാടമങ്ങ് തുറന്നാല്‍ അവരതാ അതില്‍ നൈരാശ്യം പൂണ്ടവരായിക്കഴിയുന്നു.
\end{malayalam}}
\flushright{\begin{Arabic}
\quranayah[23][78]
\end{Arabic}}
\flushleft{\begin{malayalam}
അവനാണ് നിങ്ങള്‍ക്ക് കേള്‍വിയും കാഴ്ചകളും ഹൃദയങ്ങളും ഉണ്ടാക്കിതന്നിട്ടുള്ളവന്‍. കുറച്ചു മാത്രമേ നിങ്ങള്‍ നന്ദികാണിക്കുന്നുള്ളു.
\end{malayalam}}
\flushright{\begin{Arabic}
\quranayah[23][79]
\end{Arabic}}
\flushleft{\begin{malayalam}
അവനാകുന്നു ഭൂമിയില്‍ നിങ്ങളെ സൃഷ്ടിച്ചു വ്യാപിപ്പിച്ചവന്‍. അവന്‍റെ അടുക്കലേക്കാകുന്നു നിങ്ങള്‍ ഒരുമിച്ചുകൂട്ടപ്പെടുന്നതും.
\end{malayalam}}
\flushright{\begin{Arabic}
\quranayah[23][80]
\end{Arabic}}
\flushleft{\begin{malayalam}
അവന്‍ തന്നെയാണ് ജീവിപ്പിക്കുകയും മരിപ്പിക്കുകയും ചെയ്യുന്നവന്‍. രാപകലുകളുടെ വ്യത്യാസവും അവന്‍റെ നിയന്ത്രണത്തില്‍ തന്നെയാകുന്നു. അതിനാല്‍ നിങ്ങള്‍ ചിന്തിച്ചു മനസ്സിലാക്കുന്നില്ലേ?
\end{malayalam}}
\flushright{\begin{Arabic}
\quranayah[23][81]
\end{Arabic}}
\flushleft{\begin{malayalam}
അല്ല, പൂര്‍വ്വികന്‍മാര്‍ പറഞ്ഞതു പോലെ ഇവരും പറഞ്ഞിരിക്കുകയാണ്‌.
\end{malayalam}}
\flushright{\begin{Arabic}
\quranayah[23][82]
\end{Arabic}}
\flushleft{\begin{malayalam}
അവര്‍ പറഞ്ഞു: ഞങ്ങള്‍ മരിച്ചു മണ്ണും അസ്ഥിശകലങ്ങളും ആയിക്കഴിഞ്ഞാല്‍ ഞങ്ങള്‍ ഉയിര്‍ത്തെഴുന്നേല്‍പിക്കപ്പെടുമെന്നോ?
\end{malayalam}}
\flushright{\begin{Arabic}
\quranayah[23][83]
\end{Arabic}}
\flushleft{\begin{malayalam}
ഞങ്ങള്‍ക്കും, മുമ്പ് ഞങ്ങളുടെ പിതാക്കള്‍ക്കും ഈ വാഗ്ദാനം നല്‍കപ്പെട്ടിരുന്നു. ഇത് പൂര്‍വ്വികന്‍മാരുടെ കെട്ടുകഥകള്‍ മാത്രമാകുന്നു.
\end{malayalam}}
\flushright{\begin{Arabic}
\quranayah[23][84]
\end{Arabic}}
\flushleft{\begin{malayalam}
(നബിയേ,) ചോദിക്കുക: ഭൂമിയും അതിലുള്ളതും ആരുടെതാണ്‌? നിങ്ങള്‍ക്കറിയാമെങ്കില്‍ (പറയൂ.)
\end{malayalam}}
\flushright{\begin{Arabic}
\quranayah[23][85]
\end{Arabic}}
\flushleft{\begin{malayalam}
അവര്‍ പറയും; അല്ലാഹുവിന്റേതാണെന്ന്‌. നീ പറയുക: എന്നാല്‍ നിങ്ങള്‍ ആലോചിച്ച് മനസ്സിലാക്കുന്നില്ലേ?
\end{malayalam}}
\flushright{\begin{Arabic}
\quranayah[23][86]
\end{Arabic}}
\flushleft{\begin{malayalam}
നീ ചോദിക്കുക: ഏഴുആകാശങ്ങളുടെ രക്ഷിതാവും മഹത്തായ സിംഹാസനത്തിന്‍റെ രക്ഷിതാവും ആരാകുന്നു?
\end{malayalam}}
\flushright{\begin{Arabic}
\quranayah[23][87]
\end{Arabic}}
\flushleft{\begin{malayalam}
അവര്‍ പറയും: അല്ലാഹുവിന്നാകുന്നു (രക്ഷാകര്‍ത്തൃത്വം). നീ പറയുക: എന്നാല്‍ നിങ്ങള്‍ സൂക്ഷ്മത പാലിക്കുന്നില്ലേ?
\end{malayalam}}
\flushright{\begin{Arabic}
\quranayah[23][88]
\end{Arabic}}
\flushleft{\begin{malayalam}
നീ ചോദിക്കുക: എല്ലാ വസ്തുക്കളുടെയും ആധിപത്യം ഒരുവന്‍റെ കൈവശത്തിലാണ്‌. അവന്‍ അഭയം നല്‍കുന്നു. അവന്നെതിരായി (എവിടെ നിന്നും) അഭയം ലഭിക്കുകയില്ല. അങ്ങനെയുള്ളവന്‍ ആരാണ്‌? നിങ്ങള്‍ക്കറിയാമെങ്കില്‍ (പറയൂ.)
\end{malayalam}}
\flushright{\begin{Arabic}
\quranayah[23][89]
\end{Arabic}}
\flushleft{\begin{malayalam}
അവര്‍ പറയും: (അതെല്ലാം) അല്ലാഹുവിന്നുള്ളതാണ്‌. നീ ചോദിക്കുക: പിന്നെ എങ്ങനെയാണ് നിങ്ങള്‍ മായാവലയത്തില്‍ പെട്ടുപോകുന്നത്‌?
\end{malayalam}}
\flushright{\begin{Arabic}
\quranayah[23][90]
\end{Arabic}}
\flushleft{\begin{malayalam}
അല്ല. നാം അവരുടെ അടുത്ത് സത്യവും കൊണ്ട് ചെന്നിരിക്കുകയാണ്‌. അവരാകട്ടെ വ്യാജവാദികള്‍ തന്നെയാകുന്നു.
\end{malayalam}}
\flushright{\begin{Arabic}
\quranayah[23][91]
\end{Arabic}}
\flushleft{\begin{malayalam}
അല്ലാഹു യാതൊരു സന്താനത്തെയും സ്വീകരിച്ചിട്ടില്ല. അവനോടൊപ്പം യാതൊരു ദൈവവുമുണ്ടായിട്ടില്ല. അങ്ങനെയായിരുന്നുവെങ്കില്‍ ഓരോ ദൈവവും താന്‍ സൃഷ്ടിച്ചതുമായി പോയിക്കളയുകയും, അവരില്‍ ചിലര്‍ ചിലരെ അടിച്ചമര്‍ത്തുകയും ചെയ്യുമായിരുന്നു. അവര്‍ പറഞ്ഞുണ്ടാക്കുന്നതില്‍ നിന്നെല്ലാം അല്ലാഹു എത്ര പരിശുദ്ധന്‍!
\end{malayalam}}
\flushright{\begin{Arabic}
\quranayah[23][92]
\end{Arabic}}
\flushleft{\begin{malayalam}
അവന്‍ അദൃശ്യവും ദൃശ്യവും അറിയുന്നവനാകുന്നു. അതിനാല്‍ അവന്‍ അവര്‍ പങ്കുചേര്‍ക്കുന്നതിനെല്ലാം അതീതനായിരിക്കുന്നു.
\end{malayalam}}
\flushright{\begin{Arabic}
\quranayah[23][93]
\end{Arabic}}
\flushleft{\begin{malayalam}
(നബിയേ,) പറയുക: എന്‍റെ രക്ഷിതാവേ, ഇവര്‍ക്ക് താക്കീത് നല്‍കപ്പെടുന്ന ശിക്ഷ നീ എനിക്ക് കാണുമാറാക്കുകയാണെങ്കില്‍,
\end{malayalam}}
\flushright{\begin{Arabic}
\quranayah[23][94]
\end{Arabic}}
\flushleft{\begin{malayalam}
എന്‍റെ രക്ഷിതാവേ, നീ എന്നെ അക്രമികളായ ജനതയുടെ കൂട്ടത്തില്‍ പെടുത്തരുതേ.
\end{malayalam}}
\flushright{\begin{Arabic}
\quranayah[23][95]
\end{Arabic}}
\flushleft{\begin{malayalam}
നാം അവര്‍ക്ക് താക്കീത് നല്‍കുന്ന ശിക്ഷ നിനക്ക് കാണിച്ചുതരുവാന്‍ തീര്‍ച്ചയായും നാം കഴിവുള്ളവന്‍ തന്നെയാകുന്നു.
\end{malayalam}}
\flushright{\begin{Arabic}
\quranayah[23][96]
\end{Arabic}}
\flushleft{\begin{malayalam}
ഏറ്റവും നല്ലതേതോ അതുകൊണ്ട് നീ തിന്‍മയെ തടുത്തു കൊള്ളുക. അവര്‍ പറഞ്ഞുണ്ടാക്കുന്നതിനെപ്പറ്റി നാം നല്ലവണ്ണം അറിയുന്നവനാകുന്നു.
\end{malayalam}}
\flushright{\begin{Arabic}
\quranayah[23][97]
\end{Arabic}}
\flushleft{\begin{malayalam}
നീ പറയുക: എന്‍റെ രക്ഷിതാവേ, പിശാചുക്കളുടെ ദുര്‍ബോധനങ്ങളില്‍ നിന്ന് ഞാന്‍ നിന്നോട് രക്ഷതേടുന്നു.
\end{malayalam}}
\flushright{\begin{Arabic}
\quranayah[23][98]
\end{Arabic}}
\flushleft{\begin{malayalam}
അവര്‍ (പിശാചുക്കള്‍) എന്‍റെ അടുത്ത് സന്നിഹിതരാകുന്നതില്‍ നിന്നും എന്‍റെ രക്ഷിതാവേ, ഞാന്‍ നിന്നോട് രക്ഷതേടുന്നു.
\end{malayalam}}
\flushright{\begin{Arabic}
\quranayah[23][99]
\end{Arabic}}
\flushleft{\begin{malayalam}
അങ്ങനെ അവരില്‍ ഒരാള്‍ക്ക് മരണം വന്നെത്തുമ്പോള്‍ അവന്‍ പറയും: എന്‍റെ രക്ഷിതാവേ, എന്നെ (ജീവിതത്തിലേക്ക്‌) തിരിച്ചയക്കേണമേ
\end{malayalam}}
\flushright{\begin{Arabic}
\quranayah[23][100]
\end{Arabic}}
\flushleft{\begin{malayalam}
ഞാന്‍ ഉപേക്ഷ വരുത്തിയിട്ടുള്ള കാര്യത്തില്‍ എനിക്ക് നല്ല നിലയില്‍ പ്രവര്‍ത്തിക്കുവാന്‍ കഴിയത്തക്കവിധം. ഒരിക്കലുമില്ല! അതൊരു വെറും വാക്കാണ്‌. അതവന്‍ പറഞ്ഞു കൊണ്ടിരിക്കും. അവരുടെ പിന്നില്‍ അവര്‍ ഉയിര്‍ത്തെഴുന്നേല്‍പിക്കപ്പെടുന്ന ദിവസം വരെ ഒരു മറയുണ്ടായിരിക്കുന്നതാണ്‌.
\end{malayalam}}
\flushright{\begin{Arabic}
\quranayah[23][101]
\end{Arabic}}
\flushleft{\begin{malayalam}
എന്നിട്ട് കാഹളത്തില്‍ ഊതപ്പെട്ടാല്‍ അന്ന് അവര്‍ക്കിടയില്‍ കുടുംബബന്ധങ്ങളൊന്നുമുണ്ടായിരിക്കുകയില്ല. അവര്‍ അന്യോന്യം അന്വേഷിക്കുകയുമില്ല.
\end{malayalam}}
\flushright{\begin{Arabic}
\quranayah[23][102]
\end{Arabic}}
\flushleft{\begin{malayalam}
അപ്പോള്‍ ആരുടെ (സല്‍കര്‍മ്മങ്ങളുടെ) തൂക്കങ്ങള്‍ ഘനമുള്ളതായോ അവര്‍ തന്നെയാണ് വിജയികള്‍.
\end{malayalam}}
\flushright{\begin{Arabic}
\quranayah[23][103]
\end{Arabic}}
\flushleft{\begin{malayalam}
ആരുടെ (സല്‍കര്‍മ്മങ്ങളുടെ) തൂക്കങ്ങള്‍ ലഘുവായിപ്പോയോ അവരാണ് ആത്മനഷ്ടം പറ്റിയവര്‍, നരകത്തില്‍ നിത്യവാസികള്‍.
\end{malayalam}}
\flushright{\begin{Arabic}
\quranayah[23][104]
\end{Arabic}}
\flushleft{\begin{malayalam}
നരകാഗ്നി അവരുടെ മുഖങ്ങള്‍ കരിച്ചു കളയും. അവരതില്‍ പല്ലിളിച്ചവരായിരിക്കും.
\end{malayalam}}
\flushright{\begin{Arabic}
\quranayah[23][105]
\end{Arabic}}
\flushleft{\begin{malayalam}
അവരോട് പറയപ്പെടും:) എന്‍റെ ദൃഷ്ടാന്തങ്ങള്‍ നിങ്ങള്‍ക്ക് ഓതികേള്‍പിക്കപ്പെട്ടിരുന്നില്ലേ? അപ്പോള്‍ നിങ്ങള്‍ അവയെ നിഷേധിച്ചു തള്ളുകയായിരുന്നുവല്ലോ.
\end{malayalam}}
\flushright{\begin{Arabic}
\quranayah[23][106]
\end{Arabic}}
\flushleft{\begin{malayalam}
അവര്‍ പറയും: ഞങ്ങളുടെ രക്ഷിതാവേ, ഞങ്ങളുടെ നിര്‍ഭാഗ്യം ഞങ്ങളെ അതിജയിച്ചു കളഞ്ഞു. ഞങ്ങള്‍ വഴിപിഴച്ച ഒരു ജനവിഭാഗമായിപ്പോയി.
\end{malayalam}}
\flushright{\begin{Arabic}
\quranayah[23][107]
\end{Arabic}}
\flushleft{\begin{malayalam}
ഞങ്ങളുടെ രക്ഷിതാവേ, ഞങ്ങളെ നീ ഇതില്‍ നിന്ന് പുറത്തു കൊണ്ട് വരേണമേ. ഇനി ഞങ്ങള്‍ (ദുര്‍മാര്‍ഗത്തിലേക്ക് തന്നെ) മടങ്ങുകയാണെങ്കില്‍ തീര്‍ച്ചയായും ഞങ്ങള്‍ അക്രമികള്‍ തന്നെയായിരിക്കും.
\end{malayalam}}
\flushright{\begin{Arabic}
\quranayah[23][108]
\end{Arabic}}
\flushleft{\begin{malayalam}
അവന്‍ (അല്ലാഹു) പറയും: നിങ്ങള്‍ അവിടെത്തന്നെ നിന്ദ്യരായിക്കഴിയുക. നിങ്ങള്‍ എന്നോട് മിണ്ടിപ്പോകരുത്‌.
\end{malayalam}}
\flushright{\begin{Arabic}
\quranayah[23][109]
\end{Arabic}}
\flushleft{\begin{malayalam}
തീര്‍ച്ചയായും എന്‍റെ ദാസന്‍മാരില്‍ ഒരു വിഭാഗം ഇപ്രകാരം പറയാറുണ്ടായിരുന്നു. ഞങ്ങളുടെ രക്ഷിതാവേ, ഞങ്ങള്‍ വിശ്വസിച്ചിരിക്കുന്നു. അതിനാല്‍ ഞങ്ങള്‍ക്ക് നീ പൊറുത്തുതരികയും, ഞങ്ങളോട് കരുണ കാണിക്കുകയും ചെയ്യേണമേ. നീ കാരുണികരില്‍ ഉത്തമനാണല്ലോ.
\end{malayalam}}
\flushright{\begin{Arabic}
\quranayah[23][110]
\end{Arabic}}
\flushleft{\begin{malayalam}
അപ്പോള്‍ നിങ്ങള്‍ അവരെ പരിഹാസപാത്രമാക്കുകയാണ് ചെയ്തത്‌. അങ്ങനെ നിങ്ങള്‍ക്ക് എന്നെപ്പറ്റിയുള്ള ഓര്‍മ മറന്നുപോകാന്‍ അവര്‍ ഒരു കാരണമായിത്തീര്‍ന്നു. നിങ്ങള്‍ അവരെ പുച്ഛിച്ചു ചിരിച്ചു കൊണ്ടിരിക്കുകയായിരുന്നു.
\end{malayalam}}
\flushright{\begin{Arabic}
\quranayah[23][111]
\end{Arabic}}
\flushleft{\begin{malayalam}
അവര്‍ ക്ഷമിച്ചതു കൊണ്ട് ഇന്നിതാ ഞാനവര്‍ക്ക് പ്രതിഫലം നല്‍കിയിരിക്കുന്നു. അതെന്തെന്നാല്‍ അവര്‍ തന്നെയാകുന്നു ഭാഗ്യവാന്‍മാര്‍.
\end{malayalam}}
\flushright{\begin{Arabic}
\quranayah[23][112]
\end{Arabic}}
\flushleft{\begin{malayalam}
അവന്‍ (അല്ലാഹു) ചോദിക്കും: ഭൂമിയില്‍ നിങ്ങള്‍ താമസിച്ച കൊല്ലങ്ങളുടെ എണ്ണം എത്രയാകുന്നു?
\end{malayalam}}
\flushright{\begin{Arabic}
\quranayah[23][113]
\end{Arabic}}
\flushleft{\begin{malayalam}
അവര്‍ പറയും: ഞങ്ങള്‍ ഒരു ദിവസമോ, ഒരു ദിവസത്തിന്‍റെ അല്‍പഭാഗമോ താമസിച്ചിട്ടുണ്ടാകും. എണ്ണിത്തിട്ടപ്പെടുത്തിയവരോട് നീ ചോദിച്ചു നോക്കുക.
\end{malayalam}}
\flushright{\begin{Arabic}
\quranayah[23][114]
\end{Arabic}}
\flushleft{\begin{malayalam}
അവന്‍ പറയും: നിങ്ങള്‍ അല്‍പം മാത്രമേ താമസിച്ചിട്ടുള്ളൂ. നിങ്ങളത് മനസ്സിലാക്കുന്നവരായിരുന്നെങ്കില്‍(എത്ര നന്നായിരുന്നേനെ!)
\end{malayalam}}
\flushright{\begin{Arabic}
\quranayah[23][115]
\end{Arabic}}
\flushleft{\begin{malayalam}
അപ്പോള്‍ നാം നിങ്ങളെ വൃഥാ സൃഷ്ടിച്ചതാണെന്നും, നമ്മുടെ അടുക്കലേക്ക് നിങ്ങള്‍ മടക്കപ്പെടുകയില്ലെന്നും നിങ്ങള്‍ കണക്കാക്കിയിരിക്കുകയാണോ?
\end{malayalam}}
\flushright{\begin{Arabic}
\quranayah[23][116]
\end{Arabic}}
\flushleft{\begin{malayalam}
എന്നാല്‍ യഥാര്‍ത്ഥ രാജാവായ അല്ലാഹു ഉന്നതനായിരിക്കുന്നു. അവനല്ലാതെ യാതൊരു ദൈവവുമില്ല. മഹത്തായ സിംഹാസനത്തിന്‍റെ നാഥനത്രെ അവന്‍.
\end{malayalam}}
\flushright{\begin{Arabic}
\quranayah[23][117]
\end{Arabic}}
\flushleft{\begin{malayalam}
വല്ലവനും അല്ലാഹുവോടൊപ്പം മറ്റ് വല്ല ദൈവത്തെയും വിളിച്ച് പ്രാര്‍ത്ഥിക്കുന്ന പക്ഷം- അതിന് അവന്‍റെ പക്കല്‍ യാതൊരു പ്രമാണവും ഇല്ല തന്നെ - അവന്‍റെ വിചാരണ അവന്‍റെ രക്ഷിതാവിന്‍റെ അടുക്കല്‍ വെച്ചുതന്നെയായിരിക്കും. സത്യനിഷേധികള്‍ വിജയം പ്രാപിക്കുകയില്ല; തീര്‍ച്ച.
\end{malayalam}}
\flushright{\begin{Arabic}
\quranayah[23][118]
\end{Arabic}}
\flushleft{\begin{malayalam}
(നബിയേ,) പറയുക: എന്‍റെ രക്ഷിതാവേ, നീ പൊറുത്തുതരികയും കരുണ കാണിക്കുകയും ചെയ്യേണമേ. നീ കാരുണികരില്‍ ഏറ്റവും ഉത്തമനാണല്ലോ.
\end{malayalam}}
\chapter{\textmalayalam{നൂര്‍ ( പ്രകാശം )}}
\begin{Arabic}
\Huge{\centerline{\basmalah}}\end{Arabic}
\flushright{\begin{Arabic}
\quranayah[24][1]
\end{Arabic}}
\flushleft{\begin{malayalam}
നാം അവതരിപ്പിക്കുകയും നിയമമാക്കിവെക്കുകയും ചെയ്തിട്ടുള്ള ഒരു അദ്ധ്യായമത്രെ ഇത്‌. നിങ്ങള്‍ ആലോചിച്ചു മനസ്സിലാക്കുന്നതിനു വേണ്ടി വ്യക്തമായ ദൃഷ്ടാന്തങ്ങള്‍ നാം ഇതില്‍ അവതരിപ്പിച്ചിരിക്കുന്നു.
\end{malayalam}}
\flushright{\begin{Arabic}
\quranayah[24][2]
\end{Arabic}}
\flushleft{\begin{malayalam}
വ്യഭിചരിക്കുന്ന സ്ത്രീ പുരുഷന്‍മാരില്‍ ഓരോരുത്തരെയും നിങ്ങള്‍ നൂറ് അടി അടിക്കുക. നിങ്ങള്‍ അല്ലാഹുവിലും അന്ത്യദിനത്തിലും വിശ്വസിക്കുന്നവരാണെങ്കില്‍ അല്ലാഹുവിന്‍റെ മതനിയമത്തില്‍ (അത് നടപ്പാക്കുന്ന വിഷയത്തില്‍) അവരോടുള്ള ദയയൊന്നും നിങ്ങളെ ബാധിക്കാതിരിക്കട്ടെ. അവരുടെ ശിക്ഷ നടക്കുന്നേടത്ത് സത്യവിശ്വാസികളില്‍ നിന്നുള്ള ഒരു സംഘം സന്നിഹിതരാകുകയും ചെയ്യട്ടെ.
\end{malayalam}}
\flushright{\begin{Arabic}
\quranayah[24][3]
\end{Arabic}}
\flushleft{\begin{malayalam}
വ്യഭിചാരിയായ പുരുഷന്‍ വ്യഭിചാരിണിയെയോ ബഹുദൈവവിശ്വാസിനിയെയോ അല്ലാതെ വിവാഹം കഴിക്കാറില്ല. വ്യഭിചാരിണിയെ വ്യഭിചാരിയോ ബഹുദൈവവിശ്വാസിയോ അല്ലാതെ വിവാഹം കഴിക്കാറുമില്ല. സത്യവിശ്വാസികളുടെ മേല്‍ അത് നിഷിദ്ധമാക്കപ്പെട്ടിരിക്കുന്നു.
\end{malayalam}}
\flushright{\begin{Arabic}
\quranayah[24][4]
\end{Arabic}}
\flushleft{\begin{malayalam}
ചാരിത്രവതികളുടെ മേല്‍ (വ്യഭിചാരം) ആരോപിക്കുകയും, എന്നിട്ട് നാലു സാക്ഷികളെ കൊണ്ടു വരാതിരിക്കുകയും ചെയ്യുന്നവരെ നിങ്ങള്‍ എണ്‍പത് അടി അടിക്കുക. അവരുടെ സാക്ഷ്യം നിങ്ങള്‍ ഒരിക്കലും സ്വീകരിക്കുകയും ചെയ്യരുത്‌. അവര്‍ തന്നെയാകുന്നു അധര്‍മ്മകാരികള്‍.
\end{malayalam}}
\flushright{\begin{Arabic}
\quranayah[24][5]
\end{Arabic}}
\flushleft{\begin{malayalam}
അതിന് ശേഷം പശ്ചാത്തപിക്കുകയും നിലപാട് നന്നാക്കിത്തീര്‍ക്കുകയും ചെയ്തവരൊഴികെ. എന്നാല്‍ അല്ലാഹു ഏറെ പൊറുക്കുന്നവനും കരുണാനിധിയും തന്നെയാകുന്നു.
\end{malayalam}}
\flushright{\begin{Arabic}
\quranayah[24][6]
\end{Arabic}}
\flushleft{\begin{malayalam}
തങ്ങളുടെ ഭാര്യമാരുടെ മേല്‍ (വ്യഭിചാരം) ആരോപിക്കുകയും, അവരവര്‍ ഒഴികെ മറ്റു സാക്ഷികളൊന്നും തങ്ങള്‍ക്ക് ഇല്ലാതിരിക്കുകയും ചെയ്യുന്നവരാരോ അവരില്‍ ഓരോരുത്തരും നിര്‍വഹിക്കേണ്ട സാക്ഷ്യം തീര്‍ച്ചയായും താന്‍ സത്യവാന്‍മാരുടെ കൂട്ടത്തിലാകുന്നു എന്ന് അല്ലാഹുവിന്‍റെ പേരില്‍ നാലു പ്രാവശ്യം സാക്ഷ്യം വഹിക്കലാകുന്നു.
\end{malayalam}}
\flushright{\begin{Arabic}
\quranayah[24][7]
\end{Arabic}}
\flushleft{\begin{malayalam}
അഞ്ചാമതായി, താന്‍ കള്ളം പറയുന്നവരുടെ കൂട്ടത്തിലാണെങ്കില്‍ അല്ലാഹുവിന്‍റെ ശാപം തന്‍റെ മേല്‍ ഭവിക്കട്ടെ എന്ന് (പറയുകയും വേണം.)
\end{malayalam}}
\flushright{\begin{Arabic}
\quranayah[24][8]
\end{Arabic}}
\flushleft{\begin{malayalam}
തീര്‍ച്ചയായും അവന്‍ കളവ് പറയുന്നവരുടെ കൂട്ടത്തിലാകുന്നു എന്ന് അല്ലാഹുവിന്‍റെ പേരില്‍ അവള്‍ നാലു പ്രാവശ്യം സാക്ഷ്യം വഹിക്കുന്ന പക്ഷം, അതവളെ ശിക്ഷയില്‍ നിന്ന് ഒഴിവാക്കുന്നതാണ്‌.
\end{malayalam}}
\flushright{\begin{Arabic}
\quranayah[24][9]
\end{Arabic}}
\flushleft{\begin{malayalam}
അഞ്ചാമതായി അവന്‍ സത്യവാന്‍മാരുടെ കൂട്ടത്തിലാണെങ്കില്‍ അല്ലാഹുവിന്‍റെ കോപം തന്‍റെ മേല്‍ ഭവിക്കട്ടെ എന്ന് (പറയുകയും വേണം.)
\end{malayalam}}
\flushright{\begin{Arabic}
\quranayah[24][10]
\end{Arabic}}
\flushleft{\begin{malayalam}
അല്ലാഹുവിന്‍റെ അനുഗ്രഹവും കാരുണ്യവും നിങ്ങളുടെ മേല്‍ ഇല്ലാതിരിക്കുകയും, അല്ലാഹു ഏറെ പശ്ചാത്താപം സ്വീകരിക്കുന്നവനും, യുക്തിമാനും അല്ലാതിരിക്കുകയും ചെയ്തിരുന്നെങ്കില്‍ (നിങ്ങളുടെ സ്ഥിതി എന്താകുമായിരുന്നു?)
\end{malayalam}}
\flushright{\begin{Arabic}
\quranayah[24][11]
\end{Arabic}}
\flushleft{\begin{malayalam}
തീര്‍ച്ചയായും ആ കള്ള വാര്‍ത്തയും കൊണ്ട് വന്നവര്‍ നിങ്ങളില്‍ നിന്നുള്ള ഒരു സംഘം തന്നെയാകുന്നു. അത് നിങ്ങള്‍ക്ക് ദോഷകരമാണെന്ന് നിങ്ങള്‍ കണക്കാക്കേണ്ട. അല്ല, അത് നിങ്ങള്‍ക്ക് ഗുണകരം തന്നെയാകുന്നു. അവരില്‍ ഓരോ ആള്‍ക്കും താന്‍ സമ്പാദിച്ച പാപം ഉണ്ടായിരിക്കുന്നതാണ്‌. അവരില്‍ അതിന്‍റെ നേതൃത്വം ഏറ്റെടുത്തവനാരോ അവന്നാണ് ഭയങ്കര ശിക്ഷയുള്ളത്‌.
\end{malayalam}}
\flushright{\begin{Arabic}
\quranayah[24][12]
\end{Arabic}}
\flushleft{\begin{malayalam}
നിങ്ങള്‍ അത് കേട്ട സമയത്ത് സത്യവിശ്വാസികളായ സ്ത്രീകളും പുരുഷന്‍മാരും തങ്ങളുടെ സ്വന്തം ആളുകളെപ്പറ്റി എന്തുകൊണ്ട് നല്ലതു വിചാരിക്കുകയും, ഇതു വ്യക്തമായ നുണ തന്നെയാണ് എന്ന് പറയുകയും ചെയ്തില്ല?
\end{malayalam}}
\flushright{\begin{Arabic}
\quranayah[24][13]
\end{Arabic}}
\flushleft{\begin{malayalam}
അവര്‍ എന്തുകൊണ്ട് അതിനു നാലു സാക്ഷികളെ കൊണ്ടു വന്നില്ല.? എന്നാല്‍ അവര്‍ സാക്ഷികളെ കൊണ്ട് വരാത്തതിനാല്‍ അവര്‍ തന്നെയാകുന്നു അല്ലാഹുവിങ്കല്‍ വ്യാജവാദികള്‍.
\end{malayalam}}
\flushright{\begin{Arabic}
\quranayah[24][14]
\end{Arabic}}
\flushleft{\begin{malayalam}
ഇഹലോകത്തും പരലോകത്തും നിങ്ങളുടെ മേല്‍ അല്ലാഹുവിന്‍റെ അനുഗ്രഹവും കാരുണ്യവുമില്ലായിരുന്നുവെങ്കില്‍ നിങ്ങള്‍ ഈ സംസാരത്തില്‍ ഏര്‍പെട്ടതിന്‍റെ പേരില്‍ ഭയങ്കരമായ ശിക്ഷ നിങ്ങളെ ബാധിക്കുമായിരുന്നു.
\end{malayalam}}
\flushright{\begin{Arabic}
\quranayah[24][15]
\end{Arabic}}
\flushleft{\begin{malayalam}
നിങ്ങള്‍ നിങ്ങളുടെ നാവുകള്‍ കൊണ്ട് അതേറ്റു പറയുകയും, നിങ്ങള്‍ക്കൊരു വിവരവുമില്ലാത്തത് നിങ്ങളുടെ വായ്കൊണ്ട് മൊഴിയുകയും ചെയ്തിരുന്ന സന്ദര്‍ഭം. അതൊരു നിസ്സാരകാര്യമായി നിങ്ങള്‍ ഗണിക്കുന്നു. അല്ലാഹുവിന്‍റെ അടുക്കല്‍ അത് ഗുരുതരമാകുന്നു.
\end{malayalam}}
\flushright{\begin{Arabic}
\quranayah[24][16]
\end{Arabic}}
\flushleft{\begin{malayalam}
നിങ്ങള്‍ അത് കേട്ട സന്ദര്‍ഭത്തില്‍ ഞങ്ങള്‍ക്ക് ഇതിനെ പറ്റി സംസാരിക്കുവാന്‍ പാടുള്ളതല്ല. (അല്ലാഹുവേ,) നീ എത്ര പരിശുദ്ധന്‍! ഇത് ഭയങ്കരമായ ഒരു അപവാദം തന്നെയാകുന്നു എന്ന് നിങ്ങള്‍ എന്തുകൊണ്ട് പറഞ്ഞില്ല?
\end{malayalam}}
\flushright{\begin{Arabic}
\quranayah[24][17]
\end{Arabic}}
\flushleft{\begin{malayalam}
നിങ്ങള്‍ സത്യവിശ്വാസികളാണെങ്കില്‍ ഇതു പോലുള്ളത് ഒരിക്കലും നിങ്ങള്‍ ആവര്‍ത്തിക്കാതിരിക്കുന്നതിന് അല്ലാഹു നിങ്ങളെ ഉപദേശിക്കുന്നു.
\end{malayalam}}
\flushright{\begin{Arabic}
\quranayah[24][18]
\end{Arabic}}
\flushleft{\begin{malayalam}
അല്ലാഹു നിങ്ങള്‍ക്ക് ദൃഷ്ടാന്തങ്ങള്‍ വിവരിച്ചുതരികയും ചെയ്യുന്നു. അല്ലാഹു സര്‍വ്വജ്ഞനും യുക്തിമാനുമാകുന്നു.
\end{malayalam}}
\flushright{\begin{Arabic}
\quranayah[24][19]
\end{Arabic}}
\flushleft{\begin{malayalam}
തീര്‍ച്ചയായും സത്യവിശ്വാസികള്‍ക്കിടയില്‍ ദുര്‍വൃത്തി പ്രചരിക്കുന്നത് ഇഷ്ടപ്പെടുന്നവരാരോ അവര്‍ക്കാണ് ഇഹത്തിലും പരത്തിലും വേദനയേറിയ ശിക്ഷയുള്ളത്‌. അല്ലാഹു അറിയുന്നു. നിങ്ങള്‍ അറിയുന്നില്ല.
\end{malayalam}}
\flushright{\begin{Arabic}
\quranayah[24][20]
\end{Arabic}}
\flushleft{\begin{malayalam}
അല്ലാഹുവിന്‍റെ അനുഗ്രഹവും കാരുണ്യവും നിങ്ങളുടെ മേല്‍ ഇല്ലാതിരിക്കുകയും, അല്ലാഹു ദയാലുവും കരുണാനിധിയും അല്ലാതിരിക്കുകയും ചെയ്തിരുന്നെങ്കില്‍ (നിങ്ങളുടെ സ്ഥിതി എന്താകുമായിരുന്നു?)
\end{malayalam}}
\flushright{\begin{Arabic}
\quranayah[24][21]
\end{Arabic}}
\flushleft{\begin{malayalam}
സത്യവിശ്വാസികളേ, പിശാചിന്‍റെ കാല്‍പാടുകള്‍ പിന്‍പറ്റരുത്‌. വല്ലവനും പിശാചിന്‍റെ കാല്‍പാടുകള്‍ പിന്‍പറ്റുന്ന പക്ഷം തീര്‍ച്ചയായും അവന്‍ (പിശാച്‌) കല്‍പിക്കുന്നത് നീചവൃത്തിയും ദുരാചാരവും ചെയ്യാനായിരിക്കും. നിങ്ങളുടെ മേല്‍ അല്ലാഹുവിന്‍റെ അനുഗ്രഹവും കാരുണ്യവും ഇല്ലാതിരുന്നെങ്കില്‍ നിങ്ങളില്‍ ഒരാളും ഒരിക്കലും പരിശുദ്ധി പ്രാപിക്കുകയില്ലായിരുന്നു. പക്ഷെ, അല്ലാഹു താന്‍ ഉദ്ദേശിക്കുന്നവര്‍ക്ക് പരിശുദ്ധി നല്‍കുന്നു. അല്ലാഹു എല്ലാം കേള്‍ക്കുന്നവനും അറിയുന്നവനുമത്രെ.
\end{malayalam}}
\flushright{\begin{Arabic}
\quranayah[24][22]
\end{Arabic}}
\flushleft{\begin{malayalam}
നിങ്ങളുടെ കൂട്ടത്തില്‍ ശ്രേഷ്ഠതയും കഴിവുമുള്ളവര്‍ കുടുംബബന്ധമുള്ളവര്‍ക്കും സാധുക്കള്‍ക്കും അല്ലാഹുവിന്‍റെ മാര്‍ഗത്തില്‍ സ്വദേശം വെടിഞ്ഞു വന്നവര്‍ക്കും ഒന്നും കൊടുക്കുകയില്ലെന്ന് ശപഥം ചെയ്യരുത്‌. അവര്‍ മാപ്പുനല്‍കുകയും വിട്ടുവീഴ്ച കാണിക്കുകയും ചെയ്യട്ടെ. അല്ലാഹു നിങ്ങള്‍ക്ക് പൊറുത്തുതരാന്‍ നിങ്ങള്‍ ഇഷ്ടപ്പെടുന്നില്ലേ ? അല്ലാഹു ഏറെ പൊറുക്കുന്നവനും കരുണാനിധിയുമത്രെ.
\end{malayalam}}
\flushright{\begin{Arabic}
\quranayah[24][23]
\end{Arabic}}
\flushleft{\begin{malayalam}
പതിവ്രതകളും (ദുര്‍വൃത്തിയെപ്പറ്റി) ഓര്‍ക്കുക പോലും ചെയ്യാത്തവരുമായ സത്യവിശ്വാസിനികളെപ്പറ്റി ദുരാരോപണം നടത്തുന്നവരാരോ അവര്‍ ഇഹത്തിലും പരത്തിലും ശപിക്കപ്പെട്ടിരിക്കുന്നു; തീര്‍ച്ച. അവര്‍ക്ക് ഭയങ്കരമായ ശിക്ഷയുമുണ്ട്‌.
\end{malayalam}}
\flushright{\begin{Arabic}
\quranayah[24][24]
\end{Arabic}}
\flushleft{\begin{malayalam}
അവര്‍ പ്രവര്‍ത്തിച്ചു കൊണ്ടിരുന്നതിനെപ്പറ്റി അവരുടെ നാവുകളും കൈകളും കാലുകളും അവര്‍ക്കെതിരായി സാക്ഷിപറയുന്ന ദിവസത്തിലത്രെ അത് (ശിക്ഷ) .
\end{malayalam}}
\flushright{\begin{Arabic}
\quranayah[24][25]
\end{Arabic}}
\flushleft{\begin{malayalam}
അന്ന് അല്ലാഹു അവര്‍ക്ക് അവരുടെ യഥാര്‍ത്ഥ പ്രതിഫലം നിറവേറ്റികൊടുക്കുന്നതാണ്‌. അല്ലാഹു തന്നെയാണ് പ്രത്യക്ഷമായ സത്യമെന്ന് അവര്‍ അറിയുകയും ചെയ്യും.
\end{malayalam}}
\flushright{\begin{Arabic}
\quranayah[24][26]
\end{Arabic}}
\flushleft{\begin{malayalam}
ദുഷിച്ച സ്ത്രീകള്‍ ദുഷിച്ച പുരുഷന്‍മാര്‍ക്കും, ദുഷിച്ച പുരുഷന്‍മാര്‍ ദുഷിച്ച സ്ത്രീകള്‍ക്കുമാകുന്നു. നല്ല സ്ത്രീകള്‍ നല്ല പുരുഷന്‍മാര്‍ക്കും, നല്ല പുരുഷന്‍മാര്‍ നല്ല സ്ത്രീകള്‍ക്കുമാകുന്നു. ഇവര്‍ (ദുഷ്ടന്‍മാര്‍) പറഞ്ഞുണ്ടാക്കുന്ന കാര്യത്തില്‍ അവര്‍ (നല്ലവര്‍) നിരപരാധരാകുന്നു. അവര്‍ക്ക് പാപമോചനവും മാന്യമായ ഉപജീവനവും ഉണ്ടായിരിക്കും.
\end{malayalam}}
\flushright{\begin{Arabic}
\quranayah[24][27]
\end{Arabic}}
\flushleft{\begin{malayalam}
ഹേ; സത്യവിശ്വാസികളേ, നിങ്ങളുടെതല്ലാത്ത വീടുകളില്‍ നിങ്ങള്‍ കടക്കരുത്‌; നിങ്ങള്‍ അനുവാദം തേടുകയും ആ വീട്ടുകാര്‍ക്ക് സലാം പറയുകയും ചെയ്തിട്ടല്ലാതെ. അതാണ് നിങ്ങള്‍ക്ക് ഗുണകരം. നിങ്ങള്‍ ആലോചിച്ചു മനസ്സിലാക്കാന്‍ വേണ്ടിയത്രെ (ഇതു പറയുന്നത്‌) .
\end{malayalam}}
\flushright{\begin{Arabic}
\quranayah[24][28]
\end{Arabic}}
\flushleft{\begin{malayalam}
ഇനി നിങ്ങള്‍ അവിടെ ആരെയും കണ്ടെത്തിയില്ലെങ്കില്‍ നിങ്ങള്‍ക്ക് സമ്മതം കിട്ടുന്നത് വരെ നിങ്ങള്‍ അവിടെ കടക്കരുത്‌. നിങ്ങള്‍ തിരിച്ചുപോകൂ എന്ന് നിങ്ങളോട് പറയപ്പെട്ടാല്‍ നിങ്ങള്‍ തിരിച്ചുപോകണം. അതാണ് നിങ്ങള്‍ക്ക് ഏറെ പരിശുദ്ധമായിട്ടുള്ളത്‌. അല്ലാഹു നിങ്ങള്‍ പ്രവര്‍ത്തിക്കുന്നതിനെപ്പറ്റി അറിവുള്ളവനാകുന്നു.
\end{malayalam}}
\flushright{\begin{Arabic}
\quranayah[24][29]
\end{Arabic}}
\flushleft{\begin{malayalam}
ആള്‍ പാര്‍പ്പില്ലാത്തതും, നിങ്ങള്‍ക്ക് എന്തെങ്കിലും ഉപയോഗമുള്ളതുമായ ഭവനങ്ങളില്‍ നിങ്ങള്‍ പ്രവേശിക്കുന്നതിന് നിങ്ങള്‍ക്ക് കുറ്റമില്ല. നിങ്ങള്‍ വെളിപ്പെടുത്തുന്നതും ഒളിച്ചുവെക്കുന്നതും അല്ലാഹു അറിയുന്നു.
\end{malayalam}}
\flushright{\begin{Arabic}
\quranayah[24][30]
\end{Arabic}}
\flushleft{\begin{malayalam}
(നബിയേ,) നീ സത്യവിശ്വാസികളോട് അവരുടെ ദൃഷ്ടികള്‍ താഴ്ത്തുവാനും, ഗുഹ്യാവയവങ്ങള്‍ കാത്തുസൂക്ഷിക്കുവാനും പറയുക. അതാണ് അവര്‍ക്ക് ഏറെ പരിശുദ്ധമായിട്ടുള്ളത്‌. തീര്‍ച്ചയായും അല്ലാഹു അവര്‍ പ്രവര്‍ത്തിക്കുന്നതിനെപ്പറ്റി സൂക്ഷ്മമായി അറിയുന്നവനാകുന്നു.
\end{malayalam}}
\flushright{\begin{Arabic}
\quranayah[24][31]
\end{Arabic}}
\flushleft{\begin{malayalam}
സത്യവിശ്വാസിനികളോടും അവരുടെ ദൃഷ്ടികള്‍ താഴ്ത്തുവാനും അവരുടെ ഗുഹ്യാവയവങ്ങള്‍ കാത്തുസൂക്ഷിക്കുവാനും, അവരുടെ ഭംഗിയില്‍ നിന്ന് പ്രത്യക്ഷമായതൊഴിച്ച് മറ്റൊന്നും വെളിപ്പെടുത്താതിരിക്കുവാനും നീ പറയുക. അവരുടെ മക്കനകള്‍ കുപ്പായമാറുകള്‍ക്ക് മീതെ അവര്‍ താഴ്ത്തിയിട്ടുകൊള്ളട്ടെ. അവരുടെ ഭര്‍ത്താക്കന്‍മാര്‍, അവരുടെ പിതാക്കള്‍, അവരുടെ ഭര്‍തൃപിതാക്കള്‍, അവരുടെ പുത്രന്‍മാര്‍, അവരുടെ ഭര്‍തൃപുത്രന്‍മാര്‍, അവരുടെ സഹോദരന്‍മാര്‍, അവരുടെ സഹോദരപുത്രന്‍മാര്‍, അവരുടെ സഹോദരീ പുത്രന്‍മാര്‍, മുസ്ലിംകളില്‍ നിന്നുള്ള സ്ത്രീകള്‍, അവരുടെ വലംകൈകള്‍ ഉടമപ്പെടുത്തിയവര്‍ (അടിമകള്‍) , ലൈംഗികാസക്തി ഉള്ളവരല്ലാത്ത പുരുഷന്‍മാരായ പരിചാരകര്‍, സ്ത്രീകളുടെ രഹസ്യങ്ങള്‍ മനസ്സിലാക്കിയിട്ടില്ലാത്ത കുട്ടികള്‍ എന്നിവരൊഴിച്ച് മറ്റാര്‍ക്കും തങ്ങളുടെ ഭംഗി അവര്‍ വെളിപ്പെടുത്തരുത്‌. തങ്ങള്‍ മറച്ചു വെക്കുന്ന തങ്ങളുടെ അലങ്കാരം അറിയപ്പെടുവാന്‍ വേണ്ടി അവര്‍ കാലിട്ടടിക്കുകയും ചെയ്യരുത്‌. സത്യവിശ്വാസികളേ, നിങ്ങളെല്ലാവരും അല്ലാഹുവിങ്കലേക്ക് ഖേദിച്ചുമടങ്ങുക. നിങ്ങള്‍ വിജയം പ്രാപിച്ചേക്കാം.
\end{malayalam}}
\flushright{\begin{Arabic}
\quranayah[24][32]
\end{Arabic}}
\flushleft{\begin{malayalam}
നിങ്ങളിലുള്ള അവിവാഹിതരെയും, നിങ്ങളുടെ അടിമകളില്‍ നിന്നും അടിമസ്ത്രീകളില്‍ നിന്നും നല്ലവരായിട്ടുള്ളവരെയും നിങ്ങള്‍ വിവാഹബന്ധത്തില്‍ ഏര്‍പെടുത്തുക. അവര്‍ ദരിദ്രരാണെങ്കില്‍ അല്ലാഹു തന്‍റെ അനുഗ്രഹത്തില്‍ നിന്ന് അവര്‍ക്ക് ഐശ്വര്യം നല്‍കുന്നതാണ്‌. അല്ലാഹു വിപുലമായ കഴിവുള്ളവനും സര്‍വ്വജ്ഞനുമത്രെ.
\end{malayalam}}
\flushright{\begin{Arabic}
\quranayah[24][33]
\end{Arabic}}
\flushleft{\begin{malayalam}
വിവാഹം കഴിക്കാന്‍ കഴിവ് ലഭിക്കാത്തവര്‍ അവര്‍ക്ക് അല്ലാഹു തന്‍റെ അനുഗ്രഹത്തില്‍ നിന്ന് സ്വാശ്രയത്വം നല്‍കുന്നത് വരെ സന്‍മാര്‍ഗനിഷ്ഠ നിലനിര്‍ത്തട്ടെ. നിങ്ങളുടെ വലതുകൈകള്‍ ഉടമപ്പെടുത്തിയവരില്‍ (അടിമകളില്‍) നിന്ന് മോചനക്കരാറില്‍ ഏര്‍പെടാന്‍ ആഗ്രഹിക്കുന്നവരാരോ അവരുമായി നിങ്ങള്‍ മോചനക്കരാറില്‍ ഏര്‍പെടുക; അവരില്‍ നന്‍മയുള്ളതായി നിങ്ങള്‍ മനസ്സിലാക്കിയിട്ടുണ്ടെങ്കില്‍. അല്ലാഹു നിങ്ങള്‍ക്ക് നല്‍കിയിട്ടുള്ള സമ്പത്തില്‍ നിന്ന് അവര്‍ക്ക് നിങ്ങള്‍ നല്‍കി സഹായിക്കുകയും ചെയ്യുക. നിങ്ങളുടെ അടിമസ്ത്രീകള്‍ ചാരിത്രശുദ്ധിയോടെ ജീവിക്കാന്‍ അഗ്രഹിക്കുന്നുണ്ടെങ്കില്‍ ഐഹികജീവിതത്തിന്‍റെ വിഭവം ആഗ്രഹിച്ചു കൊണ്ട് നിങ്ങള്‍ അവരെ വേശ്യാവൃത്തിക്ക് നിര്‍ബന്ധിക്കരുത്‌. വല്ലവനും അവരെ നിര്‍ബന്ധിക്കുന്ന പക്ഷം അവര്‍ നിര്‍ബന്ധിതരായി തെറ്റുചെയ്തതിന് ശേഷം തീര്‍ച്ചയായും അല്ലാഹു ഏറെ പൊറുക്കുന്നവനും കരുണ കാണിക്കുന്നവനുമാകുന്നു.
\end{malayalam}}
\flushright{\begin{Arabic}
\quranayah[24][34]
\end{Arabic}}
\flushleft{\begin{malayalam}
തീര്‍ച്ചയായും നിങ്ങള്‍ക്ക് നാം വ്യക്തമായ ദൃഷ്ടാന്തങ്ങളും, നിങ്ങളുടെ മുമ്പ് കഴിഞ്ഞുപോയവരുടെ (ചരിത്രത്തില്‍ നിന്നുള്ള) ഉദാഹരണങ്ങളും, ധര്‍മ്മനിഷ്ഠപാലിക്കുന്നവര്‍ക്ക് വേണ്ടിയുള്ള ഉപദേശവും അവതരിപ്പിച്ചു തന്നിരിക്കുന്നു.
\end{malayalam}}
\flushright{\begin{Arabic}
\quranayah[24][35]
\end{Arabic}}
\flushleft{\begin{malayalam}
അല്ലാഹു ആകാശങ്ങളുടെയും ഭൂമിയുടെയും പ്രകാശമാകുന്നു. അവന്‍റെ പ്രകാശത്തിന്‍റെ ഉപമയിതാ: (ചുമരില്‍ വിളക്ക് വെക്കാനുള്ള) ഒരു മാടം അതില്‍ ഒരു വിളക്ക്‌. വിളക്ക് ഒരു സ്ഫടികത്തിനകത്ത് . സ്ഫടികം ഒരു ജ്വലിക്കുന്ന നക്ഷത്രം പോലെയിരിക്കുന്നു. അനുഗൃഹീതമായ ഒരു വൃക്ഷത്തില്‍ നിന്നാണ് അതിന് (വിളക്കിന്‌) ഇന്ധനം നല്‍കപ്പെടുന്നത്‌. അതായത് കിഴക്ക് ഭാഗത്തുള്ളതോ പടിഞ്ഞാറ് ഭാഗത്തുള്ളതോ അല്ലാത്ത ഒലീവ് വൃക്ഷത്തില്‍ നിന്ന്‌. അതിന്‍റെ എണ്ണ തീ തട്ടിയില്ലെങ്കില്‍ പോലും പ്രകാശിക്കുമാറാകുന്നു. (അങ്ങനെ) പ്രകാശത്തിന്‍മേല്‍ പ്രകാശം. അല്ലാഹു തന്‍റെ പ്രകാശത്തിലേക്ക് താന്‍ ഉദ്ദേശിക്കുന്നവരെ നയിക്കുന്നു. അല്ലാഹു ജനങ്ങള്‍ക്ക് വേണ്ടി ഉപമകള്‍ വിവരിച്ചുകൊടുക്കുന്നു. അല്ലാഹു ഏത് കാര്യത്തെപ്പറ്റിയും അറിവുള്ളവനത്രെ.
\end{malayalam}}
\flushright{\begin{Arabic}
\quranayah[24][36]
\end{Arabic}}
\flushleft{\begin{malayalam}
ചില ഭവനങ്ങളിലത്രെ (ആ വെളിച്ചമുള്ളത്‌.) അവ ഉയര്‍ത്തപ്പെടാനും അവയില്‍ തന്‍റെ നാമം സ്മരിക്കപ്പെടാനും അല്ലാഹു ഉത്തരവ് നല്‍കിയിരിക്കുന്നു. അവയില്‍ രാവിലെയും സന്ധ്യാസമയങ്ങളിലും അവന്‍റെ മഹത്വം പ്രകീര്‍ത്തിച്ചു കൊണ്ടിരിക്കുന്നു.
\end{malayalam}}
\flushright{\begin{Arabic}
\quranayah[24][37]
\end{Arabic}}
\flushleft{\begin{malayalam}
ചില ആളുകള്‍. അല്ലാഹുവെ സ്മരിക്കുന്നതില്‍ നിന്നും, നമസ്കാരം മുറപോലെ നിര്‍വഹിക്കുന്നതില്‍ നിന്നും, സകാത്ത് നല്‍കുന്നതില്‍ നിന്നും കച്ചവടമോ ക്രയവിക്രയമോ അവരുടെ ശ്രദ്ധതിരിച്ചുവിടുകയില്ല. ഹൃദയങ്ങളും കണ്ണുകളും ഇളകിമറിയുന്ന ഒരു ദിവസത്തെ അവര്‍ ഭയപ്പെട്ടു കൊണ്ടിരിക്കുന്നു.
\end{malayalam}}
\flushright{\begin{Arabic}
\quranayah[24][38]
\end{Arabic}}
\flushleft{\begin{malayalam}
അല്ലാഹു അവര്‍ക്ക് അവര്‍ പ്രവര്‍ത്തിച്ചതിനുള്ള ഏറ്റവും നല്ല പ്രതിഫലം നല്‍കുവാനും, അവന്‍റെ അനുഗ്രഹത്തില്‍ നിന്ന് അവര്‍ക്ക് കൂടുതലായി നല്‍കുവാനും വേണ്ടിയത്രെ അത്‌. അല്ലാഹു അവന്‍ ഉദ്ദേശിക്കുന്നവര്‍ക്ക് കണക്ക് നോക്കാതെ തന്നെ നല്‍കുന്നു.
\end{malayalam}}
\flushright{\begin{Arabic}
\quranayah[24][39]
\end{Arabic}}
\flushleft{\begin{malayalam}
അവിശ്വസിച്ചവരാകട്ടെ അവരുടെ കര്‍മ്മങ്ങള്‍ മരുഭൂമിയിലെ മരീചിക പോലെയാകുന്നു. ദാഹിച്ചവന്‍ അത് വെള്ളമാണെന്ന് വിചാരിക്കുന്നു. അങ്ങനെ അവന്‍ അതിന്നടുത്തേക്ക് ചെന്നാല്‍ അങ്ങനെ ഒന്ന് ഉള്ളതായി തന്നെ അവന്‍ കണ്ടെത്തുകയില്ല. എന്നാല്‍ തന്‍റെ അടുത്ത് അല്ലാഹുവെ അവന്‍ കണ്ടെത്തുന്നതാണ്‌. അപ്പോള്‍ (അല്ലാഹു) അവന്ന് അവന്‍റെ കണക്ക് തീര്‍ത്തു കൊടുക്കുന്നതാണ്‌. അല്ലാഹു അതിവേഗം കണക്ക് നോക്കുന്നവനത്രെ.
\end{malayalam}}
\flushright{\begin{Arabic}
\quranayah[24][40]
\end{Arabic}}
\flushleft{\begin{malayalam}
അല്ലെങ്കില്‍ ആഴക്കടലിലെ ഇരുട്ടുകള്‍ പോലെയാകുന്നു. (അവരുടെ പ്രവര്‍ത്തനങ്ങളുടെ ഉപമ) . തിരമാല അതിനെ (കടലിനെ) പൊതിയുന്നു. അതിനു മീതെ വീണ്ടും തിരമാല. അതിനു മീതെ കാര്‍മേഘം. അങ്ങനെ ഒന്നിനു മീതെ മറ്റൊന്നായി അനേകം ഇരുട്ടുകള്‍. അവന്‍റെ കൈ പുറത്തേക്ക് നീട്ടിയാല്‍ അതുപോലും അവന്‍ കാണുമാറാകില്ല. അല്ലാഹു ആര്‍ക്ക് പ്രകാശം നല്‍കിയിട്ടില്ലയോ അവന്ന് യാതൊരു പ്രകാശവുമില്ല.
\end{malayalam}}
\flushright{\begin{Arabic}
\quranayah[24][41]
\end{Arabic}}
\flushleft{\begin{malayalam}
ആകാശങ്ങളിലും ഭൂമിയിലുമുള്ളവരും, ചിറക് നിവര്‍ത്തിപ്പിടിച്ചു കൊണ്ട് പക്ഷികളും അല്ലാഹുവിന്‍റെ മഹത്വം പ്രകീര്‍ത്തിച്ചു കൊണ്ടിരിക്കുന്നു എന്ന് നീ കണ്ടില്ലേ? ഓരോരുത്തര്‍ക്കും തന്‍റെ പ്രാര്‍ത്ഥനയും കീര്‍ത്തനവും എങ്ങനെയെന്ന് അറിവുണ്ട്‌. അവര്‍ പ്രവര്‍ത്തിക്കുന്നതിനെപ്പറ്റി അല്ലാഹു അറിയുന്നവനത്രെ.
\end{malayalam}}
\flushright{\begin{Arabic}
\quranayah[24][42]
\end{Arabic}}
\flushleft{\begin{malayalam}
അല്ലാഹുവിന്നാകുന്നു ആകാശങ്ങളുടെയും ഭൂമിയുടെയും ആധിപത്യം. അല്ലാഹുവിങ്കലേക്ക് തന്നെയാണ് മടക്കവും.
\end{malayalam}}
\flushright{\begin{Arabic}
\quranayah[24][43]
\end{Arabic}}
\flushleft{\begin{malayalam}
അല്ലാഹു കാര്‍മേഘത്തെ തെളിച്ച് കൊണ്ട് വരികയും, എന്നിട്ട് അത് തമ്മില്‍ സംയോജിപ്പിക്കുകയും, എന്നിട്ടതിനെ അവന്‍ അട്ടിയാക്കുകയും ചെയ്യുന്നു. എന്ന് നീ കണ്ടില്ലേ? അപ്പോള്‍ അതിന്നിടയിലൂടെ മഴ പുറത്ത് വരുന്നതായി നിനക്ക് കാണാം. ആകാശത്ത് നിന്ന് -അവിടെ മലകള്‍ പോലുള്ള മേഘകൂമ്പാരങ്ങളില്‍ നിന്ന് -അവന്‍ ആലിപ്പഴം ഇറക്കുകയും എന്നിട്ട് താന്‍ ഉദ്ദേശിക്കുന്നവര്‍ക്ക് അത് അവന്‍ ബാധിപ്പിക്കുകയും താന്‍ ഉദ്ദേശിക്കുന്നവരില്‍ നിന്ന് അത് തിരിച്ചുവിടുകയും ചെയ്യുന്നു. അതിന്‍റെ മിന്നല്‍ വെളിച്ചം കാഴ്ചകള്‍ റാഞ്ചിക്കളയുമാറാകുന്നു.
\end{malayalam}}
\flushright{\begin{Arabic}
\quranayah[24][44]
\end{Arabic}}
\flushleft{\begin{malayalam}
അല്ലാഹു രാവും പകലും മാറ്റി മറിച്ചു കൊണ്ടിരിക്കുന്നു. തീര്‍ച്ചയായും അതില്‍ കണ്ണുള്ളവര്‍ക്ക് ഒരു ചിന്താവിഷയമുണ്ട്‌.
\end{malayalam}}
\flushright{\begin{Arabic}
\quranayah[24][45]
\end{Arabic}}
\flushleft{\begin{malayalam}
എല്ലാ ജന്തുക്കളെയും അല്ലാഹു വെള്ളത്തില്‍ നിന്ന് സൃഷ്ടിച്ചിരിക്കുന്നു. അവരുടെ കൂട്ടത്തില്‍ ഉദരത്തില്‍മേല്‍ ഇഴഞ്ഞ് നടക്കുന്നവരുണ്ട്‌. രണ്ട് കാലില്‍ നടക്കുന്നവരും അവരിലുണ്ട്‌. നാലുകാലില്‍ നടക്കുന്നവരും അവരിലുണ്ട്‌. അല്ലാഹു താന്‍ ഉദ്ദേശിക്കുന്നത് സൃഷ്ടിക്കുന്നു. തീര്‍ച്ചയായും അല്ലാഹു എല്ലാകാര്യത്തിനും കഴിവുള്ളവനാകുന്നു.
\end{malayalam}}
\flushright{\begin{Arabic}
\quranayah[24][46]
\end{Arabic}}
\flushleft{\begin{malayalam}
(യാഥാര്‍ത്ഥ്യം) വ്യക്തമാക്കുന്ന ദൃഷ്ടാന്തങ്ങള്‍ നാം അവതരിപ്പിച്ചിരിക്കുന്നു. അല്ലാഹു താന്‍ ഉദ്ദേശിക്കുന്നവരെ നേരായ പാതയിലേക്ക് നയിക്കുന്നു.
\end{malayalam}}
\flushright{\begin{Arabic}
\quranayah[24][47]
\end{Arabic}}
\flushleft{\begin{malayalam}
അവര്‍ പറയുന്നു; ഞങ്ങള്‍ അല്ലാഹുവിലും റസൂലിലും വിശ്വസിക്കുകയും, അനുസരിക്കുകയും ചെയ്തിരിക്കുന്നു എന്ന്‌. പിന്നെ അതിന് ശേഷം അവരില്‍ ഒരു വിഭാഗമതാ പിന്‍മാറിപ്പോകുന്നു. അവര്‍ വിശ്വാസികളല്ല തന്നെ.
\end{malayalam}}
\flushright{\begin{Arabic}
\quranayah[24][48]
\end{Arabic}}
\flushleft{\begin{malayalam}
അവര്‍ക്കിടയില്‍ (റസൂല്‍) തീര്‍പ്പുകല്‍പിക്കുന്നതിനായി അല്ലാഹുവിലേക്കും അവന്‍റെ റസൂലിലേക്കും അവര്‍ വിളിക്കപ്പെട്ടാല്‍ അപ്പോഴതാ അവരില്‍ ഒരു വിഭാഗം തിരിഞ്ഞുകളയുന്നു.
\end{malayalam}}
\flushright{\begin{Arabic}
\quranayah[24][49]
\end{Arabic}}
\flushleft{\begin{malayalam}
ന്യായം അവര്‍ക്ക് അനുകൂലമാണെങ്കിലോ അവര്‍ അദ്ദേഹത്തിന്‍റെ (റസൂലിന്‍റെ) അടുത്തേക്ക് വിധേയത്വത്തോട് കൂടി വരികയും ചെയ്യും.
\end{malayalam}}
\flushright{\begin{Arabic}
\quranayah[24][50]
\end{Arabic}}
\flushleft{\begin{malayalam}
അവരുടെ ഹൃദയങ്ങളില്‍ വല്ല രോഗവുമുണ്ടോ? അതല്ല അവര്‍ക്ക് സംശയം പിടിപെട്ടിരിക്കുകയാണോ? അതല്ല അല്ലാഹുവും അവന്‍റെ റസൂലും അവരോട് അനീതി പ്രവര്‍ത്തിക്കുമെന്ന് അവര്‍ ഭയപ്പെടുകയാണോ? അല്ല, അവര്‍ തന്നെയാകുന്നു അക്രമികള്‍.
\end{malayalam}}
\flushright{\begin{Arabic}
\quranayah[24][51]
\end{Arabic}}
\flushleft{\begin{malayalam}
തങ്ങള്‍ക്കിടയില്‍ (റസൂല്‍) തീര്‍പ്പുകല്‍പിക്കുന്നതിനായി അല്ലാഹുവിലേക്കും റസൂലിലേക്കും വിളിക്കപ്പെട്ടാല്‍ സത്യവിശ്വാസികളുടെ വാക്ക്‌, ഞങ്ങള്‍ കേള്‍ക്കുകയും അനുസരിക്കുകയും ചെയ്തിരിക്കുന്നു എന്ന് പറയുക മാത്രമായിരിക്കും. അവര്‍ തന്നെയാണ് വിജയികള്‍.
\end{malayalam}}
\flushright{\begin{Arabic}
\quranayah[24][52]
\end{Arabic}}
\flushleft{\begin{malayalam}
അല്ലാഹുവെയും റസൂലിനെയും അനുസരിക്കുകയും, അല്ലാഹുവെ ഭയപ്പെടുകയും അവനോട് സൂക്ഷ്മത പുലര്‍ത്തുകയും ചെയ്യുന്നവരാരോ അവര്‍ തന്നെയാണ് വിജയം നേടിയവര്‍.
\end{malayalam}}
\flushright{\begin{Arabic}
\quranayah[24][53]
\end{Arabic}}
\flushleft{\begin{malayalam}
നീ അവരോട് കല്‍പിക്കുകയാണെങ്കില്‍ അവര്‍ പുറപ്പെടുക തന്നെ ചെയ്യുമെന്ന് - അവര്‍ക്ക് സത്യം ചെയ്യാന്‍ കഴിയുന്ന വിധത്തിലെല്ലാം -അല്ലാഹുവിന്‍റെ പേരില്‍ അവര്‍ സത്യം ചെയ്ത് പറഞ്ഞു. നീ പറയുക: നിങ്ങള്‍ സത്യം ചെയ്യേണ്ടതില്ല. ന്യായമായ അനുസരണമാണ് വേണ്ടത്‌. തീര്‍ച്ചയായും അല്ലാഹു നിങ്ങള്‍ പ്രവര്‍ത്തിക്കുന്നതിനെപ്പറ്റി സൂക്ഷ്മമായി അറിയുന്നവനാകുന്നു.
\end{malayalam}}
\flushright{\begin{Arabic}
\quranayah[24][54]
\end{Arabic}}
\flushleft{\begin{malayalam}
നീ പറയുക: നിങ്ങള്‍ അല്ലാഹുവെ അനുസരിക്കുവിന്‍. റസൂലിനെയും നിങ്ങള്‍ അനുസരിക്കുവിന്‍. എന്നാല്‍ നിങ്ങള്‍ പിന്തിരിയുന്ന പക്ഷം അദ്ദേഹം (റസൂല്‍) ചുമതലപ്പെടുത്തപ്പെട്ട കാര്യത്തില്‍ മാത്രമാണ് അദ്ദേഹത്തിന് ബാധ്യതയുള്ളത്‌. നിങ്ങള്‍ക്ക് ബാധ്യതയുള്ളത് നിങ്ങള്‍ ചുമതല ഏല്‍പിക്കപ്പെട്ട കാര്യത്തിലാണ്‌. നിങ്ങള്‍ അദ്ദേഹത്തെ അനുസരിക്കുകയാണെങ്കില്‍ നിങ്ങള്‍ക്ക് സന്‍മാര്‍ഗം പ്രാപിക്കാം. റസൂലിന്‍റെ ബാധ്യത വ്യക്തമായ പ്രബോധനം മാത്രമാകുന്നു.
\end{malayalam}}
\flushright{\begin{Arabic}
\quranayah[24][55]
\end{Arabic}}
\flushleft{\begin{malayalam}
നിങ്ങളില്‍ നിന്ന് വിശ്വസിക്കുകയും സല്‍കര്‍മ്മങ്ങള്‍ പ്രവര്‍ത്തിക്കുകയും ചെയ്തവരോട് അല്ലാഹു വാഗ്ദാനം ചെയ്തിരിക്കുന്നു; അവരുടെ മുമ്പുള്ളവര്‍ക്ക് പ്രാതിനിധ്യം നല്‍കിയത് പോലെതന്നെ തീര്‍ച്ചയായും ഭൂമിയില്‍ അവന്‍ അവര്‍ക്ക് പ്രാതിനിധ്യം നല്‍കുകയും, അവര്‍ക്ക് അവന്‍ തൃപ്തിപ്പെട്ട് കൊടുത്ത അവരുടെ മതത്തിന്‍റെ കാര്യത്തില്‍ അവര്‍ക്ക് അവന്‍ സ്വാധീനം നല്‍കുകയും, അവരുടെ ഭയപ്പാടിന് ശേഷം അവര്‍ക്ക് നിര്‍ഭയത്വം പകരം നല്‍കുകയും ചെയ്യുന്നതാണെന്ന്‌. എന്നെയായിരിക്കും അവര്‍ ആരാധിക്കുന്നത്‌. എന്നോട് യാതൊന്നും അവര്‍ പങ്കുചേര്‍ക്കുകയില്ല. അതിന് ശേഷം ആരെങ്കിലും നന്ദികേട് കാണിക്കുന്ന പക്ഷം അവര്‍ തന്നെയാകുന്നു ധിക്കാരികള്‍.
\end{malayalam}}
\flushright{\begin{Arabic}
\quranayah[24][56]
\end{Arabic}}
\flushleft{\begin{malayalam}
നിങ്ങള്‍ നമസ്കാരം മുറപോലെ നിര്‍വഹിക്കുകയും സകാത്ത് നല്‍കുകയും, റസൂലിനെ അനുസരിക്കുകയും ചെയ്യുവിന്‍. നിങ്ങള്‍ക്ക് കാരുണ്യം ലഭിച്ചേക്കാം.
\end{malayalam}}
\flushright{\begin{Arabic}
\quranayah[24][57]
\end{Arabic}}
\flushleft{\begin{malayalam}
സത്യനിഷേധികള്‍ ഭൂമിയില്‍ (അല്ലാഹുവെ) തോല്‍പിച്ച് കളയുന്നവരാണെന്ന് നീ വിചാരിക്കരുത്‌. അവരുടെ വാസസ്ഥലം നരകമാകുന്നു. ചെന്നുചേരാനുള്ള ആ സ്ഥലം വളരെ ചീത്ത.
\end{malayalam}}
\flushright{\begin{Arabic}
\quranayah[24][58]
\end{Arabic}}
\flushleft{\begin{malayalam}
സത്യവിശ്വാസികളേ, നിങ്ങളുടെ വലതുകൈകള്‍ ഉടമപ്പെടുത്തിയവ (അടിമകള്‍) രും, നിങ്ങളില്‍ പ്രായപൂര്‍ത്തി എത്തിയിട്ടില്ലാത്തവരും മൂന്ന് സന്ദര്‍ഭങ്ങളില്‍ നിങ്ങളോട് (പ്രവേശനത്തിന്‌) അനുവാദം തേടിക്കൊള്ളട്ടെ. പ്രഭാതനമസ്കാരത്തിനു മുമ്പും, ഉച്ചസമയത്ത് (ഉറങ്ങുവാന്‍) നിങ്ങളുടെ വസ്ത്രങ്ങള്‍ മേറ്റീവ്ക്കുന്ന സമയത്തും, ഇശാ നമസ്കാരത്തിന് ശേഷവും. നിങ്ങളുടെ മൂന്ന് സ്വകാര്യ സന്ദര്‍ഭങ്ങളത്രെ ഇത്‌. ഈ സന്ദര്‍ഭങ്ങള്‍ക്ക് ശേഷം നിങ്ങള്‍ക്കോ അവര്‍ക്കോ (കൂടിക്കലര്‍ന്ന് ജീവിക്കുന്നതിന്‌) യാതൊരു കുറ്റവുമില്ല. അവര്‍ നിങ്ങളുടെ അടുത്ത് ചുറ്റി നടക്കുന്നവരത്രെ. നിങ്ങള്‍ അന്യോന്യം ഇടകലര്‍ന്ന് വര്‍ത്തിക്കുന്നു. അപ്രകാരം അല്ലാഹു നിങ്ങള്‍ക്ക് തെളിവുകള്‍ വിവരിച്ചുതരുന്നു. അല്ലാഹു സര്‍വ്വജ്ഞനും യുക്തിമാനുമാകുന്നു.
\end{malayalam}}
\flushright{\begin{Arabic}
\quranayah[24][59]
\end{Arabic}}
\flushleft{\begin{malayalam}
നിങ്ങളില്‍ നിന്നുള്ള കുട്ടികള്‍ പ്രായപൂര്‍ത്തിയെത്തിയാല്‍ അവരും അവര്‍ക്ക് മുമ്പുള്ളവര്‍ സമ്മതം ചോദിച്ചത് പോലെത്തന്നെ സമ്മതം ചോദിക്കേണ്ടതാണ്‌. അപ്രകാരം അല്ലാഹു നിങ്ങള്‍ക്ക് അവന്‍റെ തെളിവുകള്‍ വിവരിച്ചുതരുന്നു. അല്ലാഹു സര്‍വ്വജ്ഞനും യുക്തിമാനുമാകുന്നു.
\end{malayalam}}
\flushright{\begin{Arabic}
\quranayah[24][60]
\end{Arabic}}
\flushleft{\begin{malayalam}
വിവാഹ ജീവിതം പ്രതീക്ഷിക്കാത്ത കിഴവികളെ സംബന്ധിച്ചടത്തോളം സൌന്ദര്യം പ്രദര്‍ശിപ്പിക്കാത്തവരായിക്കൊണ്ട് തങ്ങളുടെ മേല്‍വസ്ത്രങ്ങള്‍ മാറ്റി വെക്കുന്നതില്‍ അവര്‍ക്ക് കുറ്റമില്ല. അവര്‍ മാന്യത കാത്തുസൂക്ഷിക്കുന്നതാണ് അവര്‍ക്ക് കൂടുതല്‍ നല്ലത്‌. അല്ലാഹു എല്ലാം കേള്‍ക്കുന്നവനും എല്ലാം അറിയുന്നവനുമാകുന്നു.
\end{malayalam}}
\flushright{\begin{Arabic}
\quranayah[24][61]
\end{Arabic}}
\flushleft{\begin{malayalam}
അന്ധന്‍റെ മേല്‍ കുറ്റമില്ല. മുടന്തന്‍റെ മേലും കുറ്റമില്ല. രോഗിയുടെമേലും കുറ്റമില്ല. നിങ്ങളുടെ വീടുകളില്‍ നിന്നോ, നിങ്ങളുടെ പിതാക്കളുടെ വീടുകളില്‍ നിന്നോ, നിങ്ങളുടെ മാതാക്കളുടെ വീടുകളില്‍ നിന്നോ, നിങ്ങളുടെ സഹോദരന്‍മാരുടെ വീടുകളില്‍ നിന്നോ, നിങ്ങളുടെ സഹോദരിമാരുടെ വീടുകളില്‍ നിന്നോ, നിങ്ങളുടെ പിതൃവ്യന്‍മാരുടെ വീടുകളില്‍ നിന്നോ, നിങ്ങളുടെ അമ്മായികളുടെ വീടുകളില്‍ നിന്നോ, നിങ്ങളുടെ അമ്മാവന്‍മാരുടെ വീടുകളില്‍ നിന്നോ, നിങ്ങളുടെ മാതൃസഹോദരികളുടെ വീടുകളില്‍ നിന്നോ, താക്കോലുകള്‍ നിങ്ങളുടെ കൈവശത്തിലിരിക്കുന്ന വീടുകളില്‍ നിന്നോ, നിങ്ങളുടെ സ്നേഹിതന്‍റെ വീട്ടില്‍ നിന്നോ നിങ്ങള്‍ ഭക്ഷണം കഴിക്കുന്ന കാര്യത്തില്‍ നിങ്ങള്‍ക്കും കുറ്റമില്ല. നിങ്ങള്‍ ഒരുമിച്ചോ വെവ്വേറെയോ ഭക്ഷണം കഴിക്കുന്നതിന് നിങ്ങള്‍ക്ക് കുറ്റമില്ല. എന്നാല്‍ നിങ്ങള്‍ വല്ല വീടുകളിലും പ്രവേശിക്കുകയാണെങ്കില്‍ അല്ലാഹുവിങ്കല്‍ നിന്നുള്ള അനുഗൃഹീതവും പാവനവുമായ ഒരു ഉപചാരമെന്ന നിലയില്‍ നിങ്ങള്‍ അന്യോന്യം സലാം പറയണം. നിങ്ങള്‍ ചിന്തിച്ചു ഗ്രഹിക്കുന്നതിന് വേണ്ടി അപ്രകാരം അല്ലാഹു നിങ്ങള്‍ക്ക് തെളിവുകള്‍ വിവരിച്ചുതരുന്നു.
\end{malayalam}}
\flushright{\begin{Arabic}
\quranayah[24][62]
\end{Arabic}}
\flushleft{\begin{malayalam}
അല്ലാഹുവിലും അവന്‍റെ റസൂലിലും വിശ്വസിച്ചവര്‍ മാത്രമാകുന്നു സത്യവിശ്വാസികള്‍. അദ്ദേഹത്തോടൊപ്പം അവര്‍ വല്ല പൊതുകാര്യത്തിലും ഏര്‍പെട്ടിരിക്കുകയാണെങ്കില്‍ അദ്ദേഹത്തോട് അനുവാദം ചോദിക്കാതെ അവര്‍ പിരിഞ്ഞു പോകുകയില്ല. തീര്‍ച്ചയായും നിന്നോട് അനുവാദം ചോദിക്കുന്നവരാരോ അവരാകുന്നു അല്ലാഹുവിലും അവന്‍റെ റസൂലിലും വിശ്വസിക്കുന്നവര്‍. അങ്ങനെ അവരുടെ എന്തെങ്കിലും ആവശ്യത്തിന് വേണ്ടി (പിരിഞ്ഞ് പോകാന്‍) അവര്‍ നിന്നോട് അനുവാദം ചോദിക്കുകയാണെങ്കില്‍ അവരില്‍ നീ ഉദ്ദേശിക്കുന്നവര്‍ക്ക് നീ അനുവാദം നല്‍കുകയും, അവര്‍ക്ക് വേണ്ടി നീ അല്ലാഹുവോട് പാപമോചനം തേടുകയും ചെയ്യുക. തീര്‍ച്ചയായും അല്ലാഹു ഏറെ പൊറുക്കുന്നവനും കരുണാനിധിയുമാകുന്നു.
\end{malayalam}}
\flushright{\begin{Arabic}
\quranayah[24][63]
\end{Arabic}}
\flushleft{\begin{malayalam}
നിങ്ങള്‍ക്കിടയില്‍ റസൂലിന്‍റെ വിളിയെ നിങ്ങളില്‍ ചിലര്‍ ചിലരെ വിളിക്കുന്നത് പോലെ നിങ്ങള്‍ ആക്കിത്തീര്‍ക്കരുത്‌. (മറ്റുള്ളവരുടെ) മറപിടിച്ചുകൊണ്ട് നിങ്ങളുടെ കൂട്ടത്തില്‍ നിന്ന് ചോര്‍ന്ന് പോകുന്നവരെ അല്ലാഹു അറിയുന്നുണ്ട്‌. ആകയാല്‍ അദ്ദേഹത്തിന്‍റെ കല്‍പനയ്ക്ക് എതിര്‍ പ്രവര്‍ത്തിക്കുന്നവര്‍ തങ്ങള്‍ക്ക് വല്ല ആപത്തും വന്നുഭവിക്കുകയോ, വേദനയേറിയ ശിക്ഷ ബാധിക്കുകയോ ചെയ്യുന്നത് സൂക്ഷിച്ചു കൊള്ളട്ടെ.
\end{malayalam}}
\flushright{\begin{Arabic}
\quranayah[24][64]
\end{Arabic}}
\flushleft{\begin{malayalam}
അറിയുക: തീര്‍ച്ചയായും അല്ലാഹുവിനുള്ളതാകുന്നു ആകാശങ്ങളിലും ഭൂമിയിലുമുള്ളതെല്ലാം. നിങ്ങള്‍ ഏതൊരു നിലപാടിലാണെന്ന് അവന്നറിയാം. അവങ്കലേക്ക് അവര്‍ മടക്കപ്പെടുന്ന ദിവസം അവന്നറിയാം. അപ്പോള്‍ അവര്‍ പ്രവര്‍ത്തിച്ചതിനെപ്പറ്റി അവര്‍ക്കവന്‍ പറഞ്ഞുകൊടുക്കുന്നതാണ്‌. അല്ലാഹു ഏതു കാര്യത്തെപ്പറ്റിയും അറിവുള്ളവനത്രെ.
\end{malayalam}}
\chapter{\textmalayalam{ഫുര്‍ഖാന്‍ ( സത്യാസത്യ വിവേചനം )}}
\begin{Arabic}
\Huge{\centerline{\basmalah}}\end{Arabic}
\flushright{\begin{Arabic}
\quranayah[25][1]
\end{Arabic}}
\flushleft{\begin{malayalam}
തന്‍റെ ദാസന്‍റെ മേല്‍ സത്യാസത്യവിവേചനത്തിനുള്ള പ്രമാണം (ഖുര്‍ആന്‍) അവതരിപ്പിച്ചവന്‍ അനുഗ്രഹപൂര്‍ണ്ണനാകുന്നു. അദ്ദേഹം (റസൂല്‍) ലോകര്‍ക്ക് ഒരു താക്കീതുകാരന്‍ ആയിരിക്കുന്നതിനു വേണ്ടിയത്രെ അത്‌.
\end{malayalam}}
\flushright{\begin{Arabic}
\quranayah[25][2]
\end{Arabic}}
\flushleft{\begin{malayalam}
ആകാശങ്ങളുടെയും ഭൂമിയുടെയും ആധിപത്യം ആര്‍ക്കാണോ അവനത്രെ (അത് അവതരിപ്പിച്ചവന്‍.) അവന്‍ സന്താനത്തെ സ്വീകരിച്ചിട്ടില്ല. ആധിപത്യത്തില്‍ അവന്ന് യാതൊരു പങ്കാളിയും ഉണ്ടായിട്ടുമില്ല. ഓരോ വസ്തുവെയും അവന്‍ സൃഷ്ടിക്കുകയും, അതിനെ അവന്‍ ശരിയാംവണ്ണം വ്യവസ്ഥപ്പെടുത്തുകയും ചെയ്തിരിക്കുന്നു.
\end{malayalam}}
\flushright{\begin{Arabic}
\quranayah[25][3]
\end{Arabic}}
\flushleft{\begin{malayalam}
അവന്ന് പുറമെ പല ദൈവങ്ങളേയും അവര്‍ സ്വീകരിച്ചിരിക്കുന്നു. അവര്‍ (ദൈവങ്ങള്‍) യാതൊന്നും സൃഷ്ടിക്കുന്നില്ല. അവര്‍ തന്നെയും സൃഷ്ടിക്കപ്പെടുകയാകുന്നു. തങ്ങള്‍ക്ക് തന്നെ ഉപദ്രവമോ ഉപകാരമോ അവര്‍ അധീനപ്പെടുത്തുന്നുമില്ല. മരണത്തെയോ ജീവിതത്തെയോ ഉയിര്‍ത്തെഴുന്നേല്‍പിനെയോ അവര്‍ അധീനപ്പെടുത്തുന്നില്ല.
\end{malayalam}}
\flushright{\begin{Arabic}
\quranayah[25][4]
\end{Arabic}}
\flushleft{\begin{malayalam}
സത്യനിഷേധികള്‍ പറഞ്ഞു: ഇത് (ഖുര്‍ആന്‍) അവന്‍ കെട്ടിച്ചമച്ച നുണ മാത്രമാകുന്നു. വേറെ ചില ആളുകള്‍ അവനെ അതിന് സഹായിച്ചിട്ടുമുണ്ട്‌. എന്നാല്‍ അന്യായത്തിലും വ്യാജത്തിലും തന്നെയാണ് ഈ കൂട്ടര്‍ വന്നെത്തിയിരിക്കുന്നത്‌.
\end{malayalam}}
\flushright{\begin{Arabic}
\quranayah[25][5]
\end{Arabic}}
\flushleft{\begin{malayalam}
ഇത് പൂര്‍വ്വികന്‍മാരുടെ കെട്ടുകഥകള്‍ മാത്രമാണ്‌. ഇവന്‍ അത് എഴുതിച്ചുവെച്ചിരിക്കുന്നു, എന്നിട്ടത് രാവിലെയും വൈകുന്നേരവും അവന്ന് വായിച്ചുകേള്‍പിക്കപ്പെടുന്നു എന്നും അവര്‍ പറഞ്ഞു.
\end{malayalam}}
\flushright{\begin{Arabic}
\quranayah[25][6]
\end{Arabic}}
\flushleft{\begin{malayalam}
(നബിയേ,) പറയുക: ആകാശങ്ങളിലെയും ഭൂമിയിലെയും രഹസ്യമറിയുന്നവനാണ് ഇത് അവതരിപ്പിച്ചിരിക്കുന്നത്‌. തീര്‍ച്ചയായും അവന്‍ ഏറെ പൊറുക്കുന്നവനും കരുണാനിധിയുമാകുന്നു.
\end{malayalam}}
\flushright{\begin{Arabic}
\quranayah[25][7]
\end{Arabic}}
\flushleft{\begin{malayalam}
അവര്‍ പറഞ്ഞു: ഈ ദൂതന്‍ എന്താണിങ്ങനെ? ഇയാള്‍ ഭക്ഷണം കഴിക്കുകയും, അങ്ങാടികളിലൂടെ നടക്കുകയും ചെയ്യുന്നല്ലോ. ഇയാളുടെ കൂടെ ഒരു താക്കീതുകാരനായിരിക്കത്തക്കവണ്ണം ഇയാളുടെ അടുത്തേക്ക് എന്ത് കൊണ്ട് ഒരു മലക്ക് ഇറക്കപ്പെടുന്നില്ല?
\end{malayalam}}
\flushright{\begin{Arabic}
\quranayah[25][8]
\end{Arabic}}
\flushleft{\begin{malayalam}
അല്ലെങ്കില്‍ എന്ത് കൊണ്ട് ഇയാള്‍ക്ക് ഒരു നിധി ഇട്ടുകൊടുക്കപ്പെടുന്നില്ല? അല്ലെങ്കില്‍ ഇയാള്‍ക്ക് (കായ്കനികള്‍) എടുത്ത് തിന്നാന്‍ പാകത്തില്‍ ഒരു തോട്ടമുണ്ടാകുന്നില്ല? (റസൂലിനെ പറ്റി) അക്രമികള്‍ പറഞ്ഞു: മാരണം ബാധിച്ച ഒരാളെ മാത്രമാകുന്നു നിങ്ങള്‍ പിന്‍പറ്റുന്നത്‌.
\end{malayalam}}
\flushright{\begin{Arabic}
\quranayah[25][9]
\end{Arabic}}
\flushleft{\begin{malayalam}
അവര്‍ നിന്നെക്കുറിച്ച് എങ്ങനെയാണ് ചിത്രീകരണങ്ങള്‍ നടത്തിയതെന്ന് നോക്കൂ. അങ്ങനെ അവര്‍ പിഴച്ചു പോയിരിക്കുന്നു. അതിനാല്‍ യാതൊരു മാര്‍ഗവും കണ്ടെത്താന്‍ അവര്‍ക്ക് സാധിക്കുകയില്ല.
\end{malayalam}}
\flushright{\begin{Arabic}
\quranayah[25][10]
\end{Arabic}}
\flushleft{\begin{malayalam}
താന്‍ ഉദ്ദേശിക്കുന്ന പക്ഷം അതിനെക്കാള്‍ ഉത്തമമായത് അഥവാ താഴ്ഭാഗത്തു കൂടി നദികള്‍ ഒഴുകുന്ന തോപ്പുകള്‍ നിനക്ക് നല്‍കുവാനും നിനക്ക് കൊട്ടാരങ്ങള്‍ ഉണ്ടാക്കിത്തരുവാനും കഴിവുള്ളവനാരോ അവന്‍ അനുഗ്രഹപൂര്‍ണ്ണനാകുന്നു.
\end{malayalam}}
\flushright{\begin{Arabic}
\quranayah[25][11]
\end{Arabic}}
\flushleft{\begin{malayalam}
അല്ല, അന്ത്യസമയത്തെ അവര്‍ നിഷേധിച്ചു തള്ളിയിരിക്കുന്നു. അന്ത്യസമയത്തെ നിഷേധിച്ച് തള്ളിയവര്‍ക്ക് കത്തിജ്വലിക്കുന്ന നരകം നാം ഒരുക്കിവെച്ചിരിക്കുന്നു.
\end{malayalam}}
\flushright{\begin{Arabic}
\quranayah[25][12]
\end{Arabic}}
\flushleft{\begin{malayalam}
ദൂരസ്ഥലത്ത് നിന്ന് തന്നെ അത് അവരെ കാണുമ്പോള്‍ ക്ഷോഭിച്ചിളകുന്നതും ഇരമ്പുന്നതും അവര്‍ക്ക് കേള്‍ക്കാവുന്നതാണ്‌.
\end{malayalam}}
\flushright{\begin{Arabic}
\quranayah[25][13]
\end{Arabic}}
\flushleft{\begin{malayalam}
അതില്‍ (നരകത്തില്‍) ഒരു ഇടുങ്ങിയ സ്ഥലത്ത് ചങ്ങലകളില്‍ ബന്ധിക്കപ്പെട്ട നിലയില്‍ അവരെ ഇട്ടാല്‍ അവിടെ വെച്ച് അവര്‍ നാശമേ, എന്ന് വിളിച്ചുകേഴുന്നതാണ്‌.
\end{malayalam}}
\flushright{\begin{Arabic}
\quranayah[25][14]
\end{Arabic}}
\flushleft{\begin{malayalam}
ഇന്ന് നിങ്ങള്‍ ഒരു നാശത്തെ വിളിക്കേണ്ടതില്ല. ധാരാളം നാശത്തെ വിളിച്ചുകൊള്ളുക. (എന്നായിരിക്കും അവര്‍ക്ക് കിട്ടുന്ന മറുപടി)
\end{malayalam}}
\flushright{\begin{Arabic}
\quranayah[25][15]
\end{Arabic}}
\flushleft{\begin{malayalam}
(നബിയേ,) പറയുക; അതാണോ ഉത്തമം, അതല്ല ധര്‍മ്മനിഷ്ഠപാലിക്കുന്നവര്‍ക്ക് വാഗ്ദാനം ചെയ്യപ്പെട്ടിട്ടുള്ള ശാശ്വത സ്വര്‍ഗമാണോ? അതായിരിക്കും അവര്‍ക്കുള്ള പ്രതിഫലവും ചെന്ന് ചേരാനുള്ള സ്ഥലവും.
\end{malayalam}}
\flushright{\begin{Arabic}
\quranayah[25][16]
\end{Arabic}}
\flushleft{\begin{malayalam}
തങ്ങള്‍ ഉദ്ദേശിക്കുന്നതെന്തും അവര്‍ക്കവിടെ ഉണ്ടായിരിക്കുന്നതാണ്‌. അവര്‍ നിത്യവാസികളായിരിക്കും. അത് നിന്‍റെ രക്ഷിതാവ് ബാധ്യത ഏറ്റിട്ടുള്ള വാഗ്ദാനമാകുന്നു. ചോദിക്കപ്പെടാവുന്നതുമാകുന്നു.
\end{malayalam}}
\flushright{\begin{Arabic}
\quranayah[25][17]
\end{Arabic}}
\flushleft{\begin{malayalam}
അവരെയും അല്ലാഹുവിന് പുറമെ അവര്‍ ആരാധിക്കുന്നവയെയും അവന്‍ ഒരുമിച്ചുകൂട്ടുന്ന ദിവസം (ശ്രദ്ധേയമാകുന്നു.) എന്നിട്ടവന്‍ (ആരാധ്യരോട്‌) പറയും: എന്‍റെ ഈ ദാസന്‍മാരെ നിങ്ങള്‍ വഴിപിഴപ്പിച്ചതാണോ അതല്ല അവര്‍ തന്നെ വഴിതെറ്റിപ്പോയതാണോ?
\end{malayalam}}
\flushright{\begin{Arabic}
\quranayah[25][18]
\end{Arabic}}
\flushleft{\begin{malayalam}
അവര്‍ (ആരാധ്യര്‍) പറയും: നീ എത്ര പരിശുദ്ധന്‍! നിനക്ക് പുറമെ വല്ല രക്ഷാധികാരികളെയും സ്വീകരിക്കുക എന്നത് ഞങ്ങള്‍ക്ക് യോജിച്ചതല്ല. പക്ഷെ, അവര്‍ക്കും അവരുടെ പിതാക്കള്‍ക്കും നീ സൌഖ്യം നല്‍കി. അങ്ങനെ അവര്‍ ഉല്‍ബോധനം മറന്നുകളയുകയും, നശിച്ച ഒരു ജനതയായിത്തീരുകയും ചെയ്തു.
\end{malayalam}}
\flushright{\begin{Arabic}
\quranayah[25][19]
\end{Arabic}}
\flushleft{\begin{malayalam}
അപ്പോള്‍ ബഹുദൈവാരാധകരോട് അല്ലാഹു പറയും:) നിങ്ങള്‍ പറയുന്നതില്‍ അവര്‍ നിങ്ങളെ നിഷേധിച്ചു തള്ളിക്കഴിഞ്ഞിരിക്കുന്നു. ഇനി (ശിക്ഷ) തിരിച്ചുവിടാനോ വല്ല സഹായവും നേടാനോ നിങ്ങള്‍ക്ക് സാധിക്കുന്നതല്ല. അതിനാല്‍ (മനുഷ്യരേ,) നിങ്ങളില്‍ നിന്ന് അക്രമം ചെയ്തവരാരോ അവന്ന് നാം ഗുരുതരമായ ശിക്ഷ ആസ്വദിപ്പിക്കുന്നതാണ്‌.
\end{malayalam}}
\flushright{\begin{Arabic}
\quranayah[25][20]
\end{Arabic}}
\flushleft{\begin{malayalam}
ഭക്ഷണം കഴിക്കുകയും അങ്ങാടിയിലൂടെ നടക്കുകയും ചെയ്യുന്നവരായിട്ടല്ലാതെ നിനക്ക് മുമ്പ് ദൂതന്‍മാരില്‍ ആരെയും നാം അയക്കുകയുണ്ടായിട്ടില്ല. നിങ്ങള്‍ ക്ഷമിക്കുമോ എന്ന് നോക്കാനായി നിങ്ങളില്‍ ചിലരെ ചിലര്‍ക്ക് നാം ഒരു പരീക്ഷണമാക്കിയിരിക്കുന്നു. (നിന്‍റെ രക്ഷിതാവ് (എല്ലാം) കണ്ടറിയുന്നവനാകുന്നു.
\end{malayalam}}
\flushright{\begin{Arabic}
\quranayah[25][21]
\end{Arabic}}
\flushleft{\begin{malayalam}
നമ്മെ കണ്ടുമുട്ടാന്‍ ആശിക്കാത്തവര്‍ പറഞ്ഞു: നമ്മുടെ മേല്‍ മലക്കുകള്‍ ഇറക്കപ്പെടുകയോ, നമ്മുടെ രക്ഷിതാവിനെ നാം (നേരില്‍) കാണുകയോ ചെയ്യാത്തതെന്താണ്‌? തീര്‍ച്ചയായും അവര്‍ സ്വയം ഗര്‍വ്വ് നടിക്കുകയും, വലിയ ധിക്കാരം കാണിക്കുകയും ചെയ്തിരിക്കുന്നു.
\end{malayalam}}
\flushright{\begin{Arabic}
\quranayah[25][22]
\end{Arabic}}
\flushleft{\begin{malayalam}
മലക്കുകളെ അവര്‍ കാണുന്ന ദിവസം(ശ്രദ്ധേയമാകുന്നു.) അന്നേ ദിവസം കുറ്റവാളികള്‍ക്ക് യാതൊരു സന്തോഷവാര്‍ത്തയുമില്ല. കര്‍ക്കശമായ വിലക്ക് കല്‍പിക്കപ്പെട്ടിരിക്കുകയാണ് എന്നായിരിക്കും അവര്‍ (മലക്കുകള്‍) പറയുക.
\end{malayalam}}
\flushright{\begin{Arabic}
\quranayah[25][23]
\end{Arabic}}
\flushleft{\begin{malayalam}
അവര്‍ പ്രവര്‍ത്തിച്ച കര്‍മ്മങ്ങളുടെ നേരെ നാം തിരിയുകയും, നാമതിനെ ചിതറിയ ധൂളിപോലെ ആക്കിത്തീര്‍ക്കുകയും ചെയ്യും.
\end{malayalam}}
\flushright{\begin{Arabic}
\quranayah[25][24]
\end{Arabic}}
\flushleft{\begin{malayalam}
അന്ന് സ്വര്‍ഗവാസികള്‍ ഉത്തമമായ വാസസ്ഥലവും ഏറ്റവും നല്ല വിശ്രമസ്ഥലവുമുള്ളവരായിരിക്കും.
\end{malayalam}}
\flushright{\begin{Arabic}
\quranayah[25][25]
\end{Arabic}}
\flushleft{\begin{malayalam}
ആകാശം പൊട്ടിപ്പിളര്‍ന്ന് വെണ്‍മേഘപടലം പുറത്ത് വരുകയും, മലക്കുകള്‍ ശക്തിയായി ഇറക്കപ്പെടുകയും ചെയ്യുന്ന ദിവസം.
\end{malayalam}}
\flushright{\begin{Arabic}
\quranayah[25][26]
\end{Arabic}}
\flushleft{\begin{malayalam}
അന്ന് യഥാര്‍ത്ഥമായ ആധിപത്യം പരമകാരുണികന്നായിരിക്കും. സത്യനിഷേധികള്‍ക്ക് വിഷമകരമായ ഒരു ദിവസമായിരിക്കും അത്‌.
\end{malayalam}}
\flushright{\begin{Arabic}
\quranayah[25][27]
\end{Arabic}}
\flushleft{\begin{malayalam}
അക്രമം ചെയ്തവന്‍ തന്‍റെ കൈകള്‍ കടിക്കുന്ന ദിവസം. അവന്‍ പറയും റസൂലിന്‍റെ കൂടെ ഞാനൊരു മാര്‍ഗം സ്വീകരിച്ചിരുന്നെങ്കില്‍ എത്ര നന്നായിരുന്നേനെ,
\end{malayalam}}
\flushright{\begin{Arabic}
\quranayah[25][28]
\end{Arabic}}
\flushleft{\begin{malayalam}
എന്‍റെ കഷ്ടമേ! ഇന്ന ആളെ ഞാന്‍ സുഹൃത്തായി സ്വീകരിച്ചിട്ടില്ലായിരുന്നെങ്കില്‍ എത്ര നന്നായിരുന്നേനെ.
\end{malayalam}}
\flushright{\begin{Arabic}
\quranayah[25][29]
\end{Arabic}}
\flushleft{\begin{malayalam}
എനിക്ക് ബോധനം വന്നുകിട്ടിയതിന് ശേഷം അതില്‍ നിന്നവന്‍ എന്നെ തെറ്റിച്ചുകളഞ്ഞുവല്ലോ. പിശാച് മനുഷ്യനെ കൈവിട്ടുകളയുന്നവനാകുന്നു.
\end{malayalam}}
\flushright{\begin{Arabic}
\quranayah[25][30]
\end{Arabic}}
\flushleft{\begin{malayalam}
(അന്ന്‌) റസൂല്‍ പറയും: എന്‍റെ രക്ഷിതാവേ, തീര്‍ച്ചയായും എന്‍റെ ജനത ഈ ഖുര്‍ആനിനെ അഗണ്യമാക്കിതള്ളിക്കളഞ്ഞിരിക്കുന്നു.
\end{malayalam}}
\flushright{\begin{Arabic}
\quranayah[25][31]
\end{Arabic}}
\flushleft{\begin{malayalam}
അപ്രകാരം തന്നെ ഓരോ പ്രവാചകന്നും കുറ്റവാളികളില്‍ പെട്ട ചില ശത്രുക്കളെ നാം ഏര്‍പെടുത്തിയിരിക്കുന്നു. മാര്‍ഗദര്‍ശകനായും സഹായിയായും നിന്‍റെ രക്ഷിതാവ് തന്നെ മതി.
\end{malayalam}}
\flushright{\begin{Arabic}
\quranayah[25][32]
\end{Arabic}}
\flushleft{\begin{malayalam}
സത്യനിഷേധികള്‍ പറഞ്ഞു; ഇദ്ദേഹത്തിന് ഖുര്‍ആന്‍ ഒറ്റതവണയായി ഇറക്കപ്പെടാത്തതെന്താണെന്ന്‌. അത് അപ്രകാരം (ഘട്ടങ്ങളിലായി അവതരിപ്പിക്കുക) തന്നെയാണ് വേണ്ടത്‌. അത് കൊണ്ട് നിന്‍റെ ഹൃദയത്തെ ഉറപ്പിച്ച് നിര്‍ത്തുവാന്‍ വേണ്ടിയാകുന്നു. ശരിയായ സാവകാശത്തോടെ നാമത് പാരായണം ചെയ്ത് കേള്‍പിക്കുകയും ചെയ്തിരിക്കുന്നു.
\end{malayalam}}
\flushright{\begin{Arabic}
\quranayah[25][33]
\end{Arabic}}
\flushleft{\begin{malayalam}
അവര്‍ ഏതൊരു പ്രശ്നവും കൊണ്ട് നിന്‍റെ അടുത്ത് വരികയാണെങ്കിലും അതിന്‍റെ യാഥാര്‍ത്ഥ്യവും ഏറ്റവും നല്ല വിവരണവും നിനക്ക് നാം കൊണ്ട് വന്ന് തരാതിരിക്കില്ല.
\end{malayalam}}
\flushright{\begin{Arabic}
\quranayah[25][34]
\end{Arabic}}
\flushleft{\begin{malayalam}
മുഖങ്ങള്‍ നിലത്ത് കുത്തിയ നിലയില്‍ നരകത്തിലേക്ക് തെളിച്ചു കൂട്ടപ്പെടുന്നവരാരോ അവരാണ് ഏറ്റവും മോശമായ സ്ഥാനത്ത് നില്‍ക്കുന്നവരും, ഏറ്റവും വഴിപിഴച്ചു പോയവരും.
\end{malayalam}}
\flushright{\begin{Arabic}
\quranayah[25][35]
\end{Arabic}}
\flushleft{\begin{malayalam}
മൂസായ്ക്ക് നാം വേദഗ്രന്ഥം നല്‍കുകയും, അദ്ദേഹത്തിന്‍റെ സഹോദരന്‍ ഹാറൂനെ അദ്ദേഹത്തോടൊപ്പം നാം സഹായിയായി നിശ്ചയിക്കുകയും ചെയ്തു.
\end{malayalam}}
\flushright{\begin{Arabic}
\quranayah[25][36]
\end{Arabic}}
\flushleft{\begin{malayalam}
എന്നിട്ട് നാം പറഞ്ഞു: നമ്മുടെ ദൃഷ്ടാന്തങ്ങളെ നിഷേധിച്ചു കളഞ്ഞ ജനതയുടെ അടുത്തേക്ക് നിങ്ങള്‍ പോകുക തുടര്‍ന്ന് നാം ആ ജനതയെ പാടെ തകര്‍ത്തു കളഞ്ഞു.
\end{malayalam}}
\flushright{\begin{Arabic}
\quranayah[25][37]
\end{Arabic}}
\flushleft{\begin{malayalam}
നൂഹിന്‍റെ ജനതയേയും (നാം നശിപ്പിച്ചു.) അവര്‍ ദൂതന്‍മാരെ നിഷേധിച്ചു കളഞ്ഞപ്പോള്‍ നാം അവരെ മുക്കി നശിപ്പിച്ചു. അവരെ നാം മനുഷ്യര്‍ക്ക് ഒരു ദൃഷ്ടാന്തമാക്കുകയും ചെയ്തു. അക്രമികള്‍ക്ക് (പരലോകത്ത്‌) വേദനയേറിയ ശിക്ഷ നാം ഒരുക്കിവെക്കുകയും ചെയ്തിരിക്കുന്നു.
\end{malayalam}}
\flushright{\begin{Arabic}
\quranayah[25][38]
\end{Arabic}}
\flushleft{\begin{malayalam}
ആദ് സമുദായത്തേയും, ഥമൂദ് സമുദായത്തെയും, റസ്സുകാരെയും അതിന്നിടയിലായി അനേകം തലമുറകളേയും (നാം നശിപ്പിച്ചിട്ടുണ്ട്‌.)
\end{malayalam}}
\flushright{\begin{Arabic}
\quranayah[25][39]
\end{Arabic}}
\flushleft{\begin{malayalam}
എല്ലാവര്‍ക്കും നാം ഉദാഹരണങ്ങള്‍ വിവരിച്ചുകൊടുത്തു. (അത് തള്ളിക്കളഞ്ഞപ്പോള്‍) എല്ലാവരെയും നാം നിശ്ശേഷം നശിപ്പിച്ചു കളയുകയും ചെയ്തു.
\end{malayalam}}
\flushright{\begin{Arabic}
\quranayah[25][40]
\end{Arabic}}
\flushleft{\begin{malayalam}
ആ ചീത്ത മഴ വര്‍ഷിക്കപ്പെട്ട നാട്ടിലൂടെ ഇവര്‍ കടന്നുവന്നിട്ടുണ്ടല്ലോ. അപ്പോള്‍ ഇവരത് കണ്ടിരുന്നില്ലേ? അല്ല, ഇവര്‍ ഉയിര്‍ത്തെഴുന്നേല്‍പ് പ്രതീക്ഷിക്കാത്തവരാകുന്നു.
\end{malayalam}}
\flushright{\begin{Arabic}
\quranayah[25][41]
\end{Arabic}}
\flushleft{\begin{malayalam}
നിന്നെ അവര്‍ കാണുമ്പോള്‍ നിന്നെ ഒരു പരിഹാസപാത്രമാക്കിക്കൊണ്ട്‌, അല്ലാഹു ദൂതനായി നിയോഗിച്ചിരിക്കുന്നത് ഇവനെയാണോ? എന്ന് ചോദിക്കുക മാത്രമായിരിക്കും അവര്‍ ചെയ്യുന്നത്‌.
\end{malayalam}}
\flushright{\begin{Arabic}
\quranayah[25][42]
\end{Arabic}}
\flushleft{\begin{malayalam}
നമ്മുടെ ദൈവങ്ങളുടെ കാര്യത്തില്‍ നാം ക്ഷമയോടെ ഉറച്ചുനിന്നിട്ടില്ലെങ്കില്‍ അവയില്‍ നിന്ന് ഇവന്‍ നമ്മെ തെറ്റിച്ചുകളയാനിടയാകുമായിരുന്നു (എന്നും അവര്‍ പറഞ്ഞു.) ശിക്ഷ നേരില്‍ കാണുന്ന സമയത്ത് അവര്‍ക്കറിയുമാറാകും; ആരാണ് ഏറ്റവും വഴിപിഴച്ചവന്‍ എന്ന്‌.
\end{malayalam}}
\flushright{\begin{Arabic}
\quranayah[25][43]
\end{Arabic}}
\flushleft{\begin{malayalam}
തന്‍റെ ദൈവത്തെ തന്‍റെ തന്നിഷ്ടമാക്കി മാറ്റിയവനെ നീ കണ്ടുവോ ? എന്നിരിക്കെ നീ അവന്‍റെ കാര്യത്തിന് ചുമതലപ്പെട്ടവനാകുമോ?
\end{malayalam}}
\flushright{\begin{Arabic}
\quranayah[25][44]
\end{Arabic}}
\flushleft{\begin{malayalam}
അതല്ല, അവരില്‍ അധികപേരും കേള്‍ക്കുകയോ ചിന്തിക്കുകയോ ചെയ്യുമെന്ന് നീ വിചാരിക്കുന്നുണ്ടോ? അവര്‍ കന്നുകാലികളെപ്പോലെ മാത്രമാകുന്നു. അല്ല, അവരാകുന്നു കൂടുതല്‍ വഴിപിഴച്ചവര്‍.
\end{malayalam}}
\flushright{\begin{Arabic}
\quranayah[25][45]
\end{Arabic}}
\flushleft{\begin{malayalam}
നിന്‍റെ രക്ഷിതാവിനെ സംബന്ധിച്ച് നീ ചിന്തിച്ച് നോക്കിയിട്ടില്ലേ? എങ്ങനെയാണ് അവന്‍ നിഴലിനെ നീട്ടിയത് എന്ന്‌. അവന്‍ ഉദ്ദേശിച്ചിരുന്നെങ്കില്‍ അതിനെ അവന്‍ നിശ്ചലമാക്കുമായിരുന്നു. എന്നിട്ട് നാം സൂര്യനെ അതിന്ന് തെളിവാക്കി.
\end{malayalam}}
\flushright{\begin{Arabic}
\quranayah[25][46]
\end{Arabic}}
\flushleft{\begin{malayalam}
പിന്നീട് നമ്മുടെ അടുത്തേക്ക് നാം അതിനെ അല്‍പാല്‍പമായി പിടിച്ചെടുത്തു.
\end{malayalam}}
\flushright{\begin{Arabic}
\quranayah[25][47]
\end{Arabic}}
\flushleft{\begin{malayalam}
അവനത്രെ നിങ്ങള്‍ക്ക് വേണ്ടി രാത്രിയെ ഒരു വസ്ത്രവും, ഉറക്കത്തെ ഒരു വിശ്രമവും ആക്കിത്തന്നവന്‍. പകലിനെ അവന്‍ എഴുന്നേല്‍പ് സമയമാക്കുകയും ചെയ്തിരിക്കുന്നു.
\end{malayalam}}
\flushright{\begin{Arabic}
\quranayah[25][48]
\end{Arabic}}
\flushleft{\begin{malayalam}
തന്‍റെ കാരുണ്യത്തിന്‍റെ മുമ്പില്‍ സന്തോഷസൂചകമായി കാറ്റുകളെ അയച്ചതും അവനത്രെ. ആകാശത്ത് നിന്ന് ശുദ്ധമായ ജലം നാം ഇറക്കുകയും ചെയ്തിരിക്കുന്നു.
\end{malayalam}}
\flushright{\begin{Arabic}
\quranayah[25][49]
\end{Arabic}}
\flushleft{\begin{malayalam}
നിര്‍ജീവമായ നാടിന് അത് മുഖേന നാം ജീവന്‍ നല്‍കുവാനും, നാം സൃഷ്ടിച്ചിട്ടുള്ള ധാരാളം കന്നുകാലികള്‍ക്കും മനുഷ്യര്‍ക്കും അത് കുടിപ്പിക്കുവാനും വേണ്ടി.
\end{malayalam}}
\flushright{\begin{Arabic}
\quranayah[25][50]
\end{Arabic}}
\flushleft{\begin{malayalam}
അവര്‍ ആലോചിച്ചു മനസ്സിലാക്കേണ്ടതിനായി അത് (മഴവെള്ളം) അവര്‍ക്കിടയില്‍ നാം വിതരണം ചെയ്തിരിക്കുന്നു. എന്നാല്‍ മനുഷ്യരില്‍ അധികപേര്‍ക്കും നന്ദികേട് കാണിക്കുവാനല്ലാതെ മനസ്സു വന്നില്ല.
\end{malayalam}}
\flushright{\begin{Arabic}
\quranayah[25][51]
\end{Arabic}}
\flushleft{\begin{malayalam}
നാം ഉദ്ദേശിച്ചിരുന്നെങ്കില്‍ എല്ലാ നാട്ടിലും ഓരോ താക്കീതുകാരനെ നാം നിയോഗിക്കുമായിരുന്നു.
\end{malayalam}}
\flushright{\begin{Arabic}
\quranayah[25][52]
\end{Arabic}}
\flushleft{\begin{malayalam}
അതിനാല്‍ സത്യനിഷേധികളെ നീ അനുസരിച്ചു പോകരുത്‌. ഇത് (ഖുര്‍ആന്‍) കൊണ്ട് നീ അവരോട് വലിയൊരു സമരം നടത്തിക്കൊള്ളുക.
\end{malayalam}}
\flushright{\begin{Arabic}
\quranayah[25][53]
\end{Arabic}}
\flushleft{\begin{malayalam}
രണ്ട് ജലാശയങ്ങളെ സ്വതന്ത്രമായി ഒഴുകാന്‍ വിട്ടവനാകുന്നു അവന്‍. ഒന്ന് സ്വച്ഛമായ ശുദ്ധജലം, മറ്റൊന്ന് അരോചകമായി തോന്നുന്ന ഉപ്പുവെള്ളവും. അവ രണ്ടിനുമിടയില്‍ ഒരു മറയും ശക്തിയായ ഒരു തടസ്സവും അവന്‍ ഏര്‍പെടുത്തുകയും ചെയ്തിരിക്കുന്നു.
\end{malayalam}}
\flushright{\begin{Arabic}
\quranayah[25][54]
\end{Arabic}}
\flushleft{\begin{malayalam}
അവന്‍ തന്നെയാണ് വെള്ളത്തില്‍ നിന്ന് മനുഷ്യനെ സൃഷ്ടിക്കുകയും, അവനെ രക്തബന്ധമുള്ളവനും വിവാഹബന്ധമുള്ളവനും ആക്കുകയും ചെയ്തിരിക്കുന്നത്‌. നിന്‍റെ രക്ഷിതാവ് കഴിവുള്ളവനാകുന്നു.
\end{malayalam}}
\flushright{\begin{Arabic}
\quranayah[25][55]
\end{Arabic}}
\flushleft{\begin{malayalam}
അല്ലാഹുവിന് പുറമെ അവര്‍ക്ക് ഉപകാരമുണ്ടാക്കുകയോ ഉപദ്രവമുണ്ടാക്കുകയോ ചെയ്യാത്തതിനെ അവര്‍ ആരാധിക്കുന്നു. സത്യനിഷേധി തന്‍റെ രക്ഷിതാവിനെതിരെ (ദുശ്ശക്തികള്‍ക്ക്‌) പിന്തുണ നല്‍കുന്നവനായിരിക്കുന്നു.
\end{malayalam}}
\flushright{\begin{Arabic}
\quranayah[25][56]
\end{Arabic}}
\flushleft{\begin{malayalam}
(നബിയേ,) ഒരു സന്തോഷവാര്‍ത്തക്കാരനായിക്കൊണ്ടും, താക്കീതുകാരനായിക്കൊണ്ടുമല്ലാതെ നിന്നെ നാം നിയോഗിച്ചിട്ടില്ല.
\end{malayalam}}
\flushright{\begin{Arabic}
\quranayah[25][57]
\end{Arabic}}
\flushleft{\begin{malayalam}
പറയുക: ഞാന്‍ ഇതിന്‍റെ പേരില്‍ നിങ്ങളോട് യാതൊരു പ്രതിഫലവും ചോദിക്കുന്നില്ല. വല്ലവനും നിന്‍റെ രക്ഷിതാവിങ്കലേക്കുള്ള മാര്‍ഗം സ്വീകരിക്കണം എന്ന് ഉദ്ദേശിക്കുന്നുവെങ്കില്‍ (അങ്ങനെ ചെയ്യാം എന്ന്‌) മാത്രം.
\end{malayalam}}
\flushright{\begin{Arabic}
\quranayah[25][58]
\end{Arabic}}
\flushleft{\begin{malayalam}
ഒരിക്കലും മരിക്കാതെ ജീവിച്ചിരിക്കുന്നവനെ നീ ഭരമേല്‍പിക്കുക. അവനെ സ്തുതിക്കുന്നതോടൊപ്പം പ്രകീര്‍ത്തിക്കുകയും ചെയ്യുക. തന്‍റെ ദാസന്‍മാരുടെ പാപങ്ങളെപ്പറ്റി സൂക്ഷ്മജ്ഞാനമുള്ളവനായിട്ട് അവന്‍ തന്നെ മതി.
\end{malayalam}}
\flushright{\begin{Arabic}
\quranayah[25][59]
\end{Arabic}}
\flushleft{\begin{malayalam}
ആകാശങ്ങളും ഭൂമിയും അവയ്ക്കിടയിലുള്ളതും ആറുദിവസങ്ങളില്‍ സൃഷ്ടിച്ചവനത്രെ അവന്‍. എന്നിട്ട് അവന്‍ സിംഹാസനസ്ഥനായിരിക്കുന്നു. പരമകാരുണികനത്രെ അവന്‍. ആകയാല്‍ ഇതിനെപ്പറ്റി സൂക്ഷ്മജ്ഞാനമുള്ളവനോട് തന്നെ ചോദിക്കുക.
\end{malayalam}}
\flushright{\begin{Arabic}
\quranayah[25][60]
\end{Arabic}}
\flushleft{\begin{malayalam}
പരമകാരുണികന് നിങ്ങള്‍ പ്രണാമം ചെയ്യുക എന്ന് അവരോട് പറയപ്പെട്ടാല്‍ അവര്‍ പറയും: എന്താണീ പരമകാരുണികന്‍ ? നീ ഞങ്ങളോട് കല്‍പിക്കുന്നതിന് ഞങ്ങള്‍ പ്രണാമം ചെയ്യുകയോ? അങ്ങനെ അത് അവരുടെ അകല്‍ച്ച വര്‍ദ്ധിപ്പിക്കുകയാണ് ചെയ്തത്‌.
\end{malayalam}}
\flushright{\begin{Arabic}
\quranayah[25][61]
\end{Arabic}}
\flushleft{\begin{malayalam}
ആകാശത്ത് നക്ഷത്രമണ്ഡലങ്ങള്‍ ഉണ്ടാക്കിയവന്‍ അനുഗ്രഹപൂര്‍ണ്ണനാകുന്നു. അവിടെ അവന്‍ ഒരു വിളക്കും (സൂര്യന്‍) വെളിച്ചം നല്‍കുന്ന ചന്ദ്രനും ഉണ്ടാക്കിയിരിക്കുന്നു.
\end{malayalam}}
\flushright{\begin{Arabic}
\quranayah[25][62]
\end{Arabic}}
\flushleft{\begin{malayalam}
അവന്‍ തന്നെയാണ് രാപകലുകളെ മാറി മാറി വരുന്നതാക്കിയവന്‍. ആലോചിച്ച് മനസ്സിലാക്കാന്‍ ഉദ്ദേശിക്കുകയോ, നന്ദികാണിക്കാന്‍ ഉദ്ദേശിക്കുകയോ ചെയ്യുന്നവര്‍ക്ക് (ദൃഷ്ടാന്തമായിരിക്കുവാനാണത്‌.)
\end{malayalam}}
\flushright{\begin{Arabic}
\quranayah[25][63]
\end{Arabic}}
\flushleft{\begin{malayalam}
പരമകാരുണികന്‍റെ ദാസന്‍മാര്‍ ഭൂമിയില്‍ കൂടി വിനയത്തോടെ നടക്കുന്നവരും, അവിവേകികള്‍ തങ്ങളോട് സംസാരിച്ചാല്‍ സമാധാനപരമായി മറുപടി നല്‍കുന്നവരുമാകുന്നു.
\end{malayalam}}
\flushright{\begin{Arabic}
\quranayah[25][64]
\end{Arabic}}
\flushleft{\begin{malayalam}
തങ്ങളുടെ രക്ഷിതാവിന് പ്രണാമം ചെയ്യുന്നവരായിക്കൊണ്ടും, നിന്ന് നമസ്കരിക്കുന്നവരായിക്കൊണ്ടും രാത്രി കഴിച്ചുകൂട്ടുന്നവരുമാകുന്നു അവര്‍.
\end{malayalam}}
\flushright{\begin{Arabic}
\quranayah[25][65]
\end{Arabic}}
\flushleft{\begin{malayalam}
ഇപ്രകാരം പറയുന്നുവരുമാകുന്നു അവര്‍ ഞങ്ങളുടെ രക്ഷിതാവേ, ഞങ്ങളില്‍ നിന്ന് നരകശിക്ഷ നീ ഒഴിവാക്കിത്തരേണമേ, തീര്‍ച്ചയായും അതിലെ ശിക്ഷ വിട്ടൊഴിയാത്ത വിപത്താകുന്നു.
\end{malayalam}}
\flushright{\begin{Arabic}
\quranayah[25][66]
\end{Arabic}}
\flushleft{\begin{malayalam}
തീര്‍ച്ചയായും അത് (നരകം) ചീത്തയായ ഒരു താവളവും പാര്‍പ്പിടവും തന്നെയാകുന്നു.
\end{malayalam}}
\flushright{\begin{Arabic}
\quranayah[25][67]
\end{Arabic}}
\flushleft{\begin{malayalam}
ചെലവുചെയ്യുകയാണെങ്കില്‍ അമിതവ്യയം നടത്തുകയോ, പിശുക്കിപ്പിടിക്കുകയോ ചെയ്യാതെ അതിനിടക്കുള്ള മിതമായ മാര്‍ഗം സ്വീകരിക്കുന്നവരുമാകുന്നു അവര്‍.
\end{malayalam}}
\flushright{\begin{Arabic}
\quranayah[25][68]
\end{Arabic}}
\flushleft{\begin{malayalam}
അല്ലാഹുവോടൊപ്പം വേറെയൊരു ദൈവത്തെയും വിളിച്ചു പ്രാര്‍ത്ഥിക്കാത്തവരും, അല്ലാഹു പവിത്രമാക്കി വെച്ചിട്ടുള്ള ജീവനെ ന്യായമായ കാരണത്താലല്ലാതെ ഹനിച്ചു കളയാത്തവരും, വ്യഭിചരിക്കാത്തവരുമാകുന്നു അവര്‍. ആ കാര്യങ്ങള്‍ വല്ലവനും ചെയ്യുന്ന പക്ഷം അവന്‍ പാപഫലം കണ്ടെത്തുക തന്നെ ചെയ്യും.
\end{malayalam}}
\flushright{\begin{Arabic}
\quranayah[25][69]
\end{Arabic}}
\flushleft{\begin{malayalam}
ഉയിര്‍ത്തെഴുന്നേല്‍പിന്‍റെ നാളില്‍ അവന്നു ശിക്ഷ ഇരട്ടിയാക്കപ്പെടുകയും, നിന്ദ്യനായിക്കൊണ്ട് അവന്‍ അതില്‍ എന്നെന്നും കഴിച്ചുകൂട്ടുകയും ചെയ്യും.
\end{malayalam}}
\flushright{\begin{Arabic}
\quranayah[25][70]
\end{Arabic}}
\flushleft{\begin{malayalam}
പശ്ചാത്തപിക്കുകയും, വിശ്വസിക്കുകയും സല്‍കര്‍മ്മം പ്രവര്‍ത്തിക്കുകയും ചെയ്തവരൊഴികെ. അത്തരക്കാര്‍ക്ക് അല്ലാഹു തങ്ങളുടെ തിന്‍മകള്‍ക്ക് പകരം നന്‍മകള്‍ മാറ്റികൊടുക്കുന്നതാണ്‌. അല്ലാഹു ഏറെ പൊറുക്കുന്നവനും കരുണാനിധിയുമായിരിക്കുന്നു.
\end{malayalam}}
\flushright{\begin{Arabic}
\quranayah[25][71]
\end{Arabic}}
\flushleft{\begin{malayalam}
വല്ലവനും പശ്ചാത്തപിക്കുകയും, സല്‍കര്‍മ്മം പ്രവര്‍ത്തിക്കുകയും ചെയ്യുന്ന പക്ഷം അല്ലാഹുവിങ്കലേക്ക് ശരിയായ നിലയില്‍ മടങ്ങുകയാണ് അവന്‍ ചെയ്യുന്നത്‌.
\end{malayalam}}
\flushright{\begin{Arabic}
\quranayah[25][72]
\end{Arabic}}
\flushleft{\begin{malayalam}
വ്യാജത്തിന് സാക്ഷി നില്‍ക്കാത്തവരും, അനാവശ്യവൃത്തികള്‍ നടക്കുന്നേടത്തു കൂടി പോകുകയാണെങ്കില്‍ മാന്യന്‍മാരായിക്കൊണ്ട് കടന്നുപോകുന്നവരുമാകുന്നു അവര്‍.
\end{malayalam}}
\flushright{\begin{Arabic}
\quranayah[25][73]
\end{Arabic}}
\flushleft{\begin{malayalam}
തങ്ങളുടെ രക്ഷിതാവിന്‍റെ വചനങ്ങള്‍ മുഖേന ഉല്‍ബോധനം നല്‍കപ്പെട്ടാല്‍ ബധിരന്‍മാരും അന്ധന്‍മാരുമായിക്കൊണ്ട് അതിന്‍മേല്‍ ചാടിവീഴാത്തവരുമാകുന്നു അവര്‍
\end{malayalam}}
\flushright{\begin{Arabic}
\quranayah[25][74]
\end{Arabic}}
\flushleft{\begin{malayalam}
ഞങ്ങളുടെ രക്ഷിതാവേ, ഞങ്ങളുടെ ഭാര്യമാരില്‍ നിന്നും സന്തതികളില്‍ നിന്നും ഞങ്ങള്‍ക്ക് നീ കണ്‍കുളിര്‍മ നല്‍കുകയും ധര്‍മ്മനിഷ്ഠ പാലിക്കുന്നവര്‍ക്ക് ഞങ്ങളെ നീ മാതൃകയാക്കുകയും ചെയ്യേണമേ എന്ന് പറയുന്നവരുമാകുന്നു അവര്‍.
\end{malayalam}}
\flushright{\begin{Arabic}
\quranayah[25][75]
\end{Arabic}}
\flushleft{\begin{malayalam}
അത്തരക്കാര്‍ക്ക് തങ്ങള്‍ ക്ഷമിച്ചതിന്‍റെ പേരില്‍ (സ്വര്‍ഗത്തില്‍) ഉന്നതമായ സ്ഥാനം പ്രതിഫലമായി നല്‍കപ്പെടുന്നതാണ്‌. അഭിവാദ്യത്തോടും സമാധാനാശംസയോടും കൂടി അവര്‍ അവിടെ സ്വീകരിക്കപ്പെടുന്നതുമാണ്‌.
\end{malayalam}}
\flushright{\begin{Arabic}
\quranayah[25][76]
\end{Arabic}}
\flushleft{\begin{malayalam}
അവര്‍ അതില്‍ നിത്യവാസികളായിരിക്കും. എത്ര നല്ല താവളവും പാര്‍പ്പിടവും!
\end{malayalam}}
\flushright{\begin{Arabic}
\quranayah[25][77]
\end{Arabic}}
\flushleft{\begin{malayalam}
(നബിയേ,) പറയുക: നിങ്ങളുടെ പ്രാര്‍ത്ഥനയില്ലെങ്കില്‍ എന്‍റെ രക്ഷിതാവ് നിങ്ങള്‍ക്ക് എന്ത് പരിഗണന നല്‍കാനാണ് ? എന്നാല്‍ നിങ്ങള്‍ നിഷേധിച്ച് തള്ളിയിരിക്കുകയാണ്‌. അതിനാല്‍ അതിനുള്ള ശിക്ഷ അനിവാര്യമായിരിക്കും.
\end{malayalam}}
\chapter{\textmalayalam{ശുഅറാ ( കവികള്‍ )}}
\begin{Arabic}
\Huge{\centerline{\basmalah}}\end{Arabic}
\flushright{\begin{Arabic}
\quranayah[26][1]
\end{Arabic}}
\flushleft{\begin{malayalam}
ത്വാ-സീന്‍-മീം
\end{malayalam}}
\flushright{\begin{Arabic}
\quranayah[26][2]
\end{Arabic}}
\flushleft{\begin{malayalam}
സുവ്യക്തമായ ഗ്രന്ഥത്തിലെ വചനങ്ങളാണിവ
\end{malayalam}}
\flushright{\begin{Arabic}
\quranayah[26][3]
\end{Arabic}}
\flushleft{\begin{malayalam}
അവര്‍ വിശ്വാസികളാകാത്തതിന്‍റെ പേരില്‍നീ നിന്‍റെ ജീവന്‍ നശിപ്പിച്ചേക്കാം
\end{malayalam}}
\flushright{\begin{Arabic}
\quranayah[26][4]
\end{Arabic}}
\flushleft{\begin{malayalam}
എന്നാല്‍ ‍നാം ഉദ്ദേശിക്കുന്ന പക്ഷം അവരുടെ മേല്‍ ആകാശത്ത് നിന്ന് നാം ഒരു ദൃഷ്ടാന്തം ഇറക്കികൊടുക്കുന്നതാണ് അന്നേരം അവരുടെ പിരടികള്‍ അതിന്ന് കീഴൊതുങ്ങുന്നതായിത്തീരുകയും ചെയ്യും
\end{malayalam}}
\flushright{\begin{Arabic}
\quranayah[26][5]
\end{Arabic}}
\flushleft{\begin{malayalam}
പരമകാരുണികന്‍റെ പക്കല്‍ ‍നിന്ന് ഏതൊരു പുതിയ ഉല്‍ബോധനം വന്നെത്തുമ്പോഴും അവര്‍ അതില്‍നിന്ന് തിരിഞ്ഞുകളയുന്നവരാകാതിരുന്നിട്ടില്ല
\end{malayalam}}
\flushright{\begin{Arabic}
\quranayah[26][6]
\end{Arabic}}
\flushleft{\begin{malayalam}
അങ്ങനെ അവര്‍ നിഷേധിച്ചു തള്ളിയിരിക്കയാണ് അതിനാല്‍ അവര്‍ ഏതൊന്നിനെ പരിഹസിക്കുന്നവരായിരിക്കുന്നുവോ അതിനെപ്പറ്റിയുള്ള വൃത്താന്തങ്ങള്‍ അവര്‍ക്ക് വന്നെത്തിക്കൊള്ളും
\end{malayalam}}
\flushright{\begin{Arabic}
\quranayah[26][7]
\end{Arabic}}
\flushleft{\begin{malayalam}
ഭൂമിയിലേക്ക് അവര്‍ നോക്കിയില്ലേ? എല്ലാ മികച്ച സസ്യവര്‍ഗങ്ങളില്‍നിന്നും എത്രയാണ് നാം അതില്‍ ‍മുളപ്പിച്ചിരിക്കുന്നത്‌?
\end{malayalam}}
\flushright{\begin{Arabic}
\quranayah[26][8]
\end{Arabic}}
\flushleft{\begin{malayalam}
തീര്‍ച്ചയായും അതില്‍ഒരു ദൃഷ്ടാന്തമുണ്ട് എന്നാല്‍ അവരില്‍ ‍അധികപേരും വിശ്വാസികളായില്ല
\end{malayalam}}
\flushright{\begin{Arabic}
\quranayah[26][9]
\end{Arabic}}
\flushleft{\begin{malayalam}
തീര്‍ച്ചയായും നിന്‍റെ രക്ഷിതാവ് തന്നെയാകുന്നു പ്രതാപിയും കരുണാനിധിയും
\end{malayalam}}
\flushright{\begin{Arabic}
\quranayah[26][10]
\end{Arabic}}
\flushleft{\begin{malayalam}
നിന്‍റെ രക്ഷിതാവ് മൂസായെ വിളിച്ചു കൊണ്ട് ഇപ്രകാരം പറഞ്ഞ സന്ദര്‍ഭം (ശ്രദ്ധേയമത്രെ,) നീ ആ അക്രമികളായ ജനങ്ങളുടെ അടുത്തേക്ക് ചെല്ലുക
\end{malayalam}}
\flushright{\begin{Arabic}
\quranayah[26][11]
\end{Arabic}}
\flushleft{\begin{malayalam}
അതായത്‌, ഫിര്‍ഔന്‍റെ ജനതയുടെ അടുക്കലേക്ക് അവര്‍ സൂക്ഷ്മത പാലിക്കുന്നില്ലേ? (എന്നു ചോദിക്കുക)
\end{malayalam}}
\flushright{\begin{Arabic}
\quranayah[26][12]
\end{Arabic}}
\flushleft{\begin{malayalam}
അദ്ദേഹം പറഞ്ഞു: എന്‍റെ രക്ഷിതാവേ, അവര്‍ എന്നെ നിഷേധിച്ചു തള്ളുമെന്ന് തീര്‍ച്ചയായും ഞാന്‍ ഭയപ്പെടുന്നു
\end{malayalam}}
\flushright{\begin{Arabic}
\quranayah[26][13]
\end{Arabic}}
\flushleft{\begin{malayalam}
എന്‍റെ ഹൃദയം ഞെരുങ്ങിപ്പോകും എന്‍റെ നാവിന് ഒഴുക്കുണ്ടാവുകയില്ല അതിനാല്‍ ‍ഹാറൂന്ന് കൂടി നീ സന്ദേശം അയക്കേണമേ
\end{malayalam}}
\flushright{\begin{Arabic}
\quranayah[26][14]
\end{Arabic}}
\flushleft{\begin{malayalam}
അവര്‍ക്ക് എന്‍റെ പേരില്‍ ഒരു കുറ്റം ആരോപിക്കാനുമുണ്ട് അതിനാല്‍ അവര്‍ എന്നെ കൊന്നേക്കുമെന്ന് ഞാന്‍ ഭയപ്പെടുന്നു
\end{malayalam}}
\flushright{\begin{Arabic}
\quranayah[26][15]
\end{Arabic}}
\flushleft{\begin{malayalam}
അല്ലാഹു പറഞ്ഞു: ഒരിക്കലുമില്ല, നമ്മുടെ ദൃഷ്ടാന്തങ്ങളും കൊണ്ട് നിങ്ങള്‍ ഇരുവരും പോയിക്കൊള്ളുക തീര്‍ച്ചയായും നിങ്ങളോടൊപ്പം നാം ശ്രദ്ധിച്ചു കേള്‍ക്കുന്നുണ്ട്‌
\end{malayalam}}
\flushright{\begin{Arabic}
\quranayah[26][16]
\end{Arabic}}
\flushleft{\begin{malayalam}
എന്നിട്ട് നിങ്ങള്‍ ഫിര്‍ഔന്‍റെ അടുക്കല്‍ചെന്ന് ഇപ്രകാരം പറയുക: തീര്‍ച്ചയായും ഞങ്ങള്‍ ലോകരക്ഷിതാവിങ്കല്‍നിന്ന് നിയോഗിക്കപ്പെട്ട ദൂതന്‍മാരാകുന്നു
\end{malayalam}}
\flushright{\begin{Arabic}
\quranayah[26][17]
\end{Arabic}}
\flushleft{\begin{malayalam}
ഇസ്രായീല്‍ ‍സന്തതികളെ ഞങ്ങളോടൊപ്പം അയച്ചുതരണം എന്ന നിര്‍ദേശവുമായിട്ട്‌
\end{malayalam}}
\flushright{\begin{Arabic}
\quranayah[26][18]
\end{Arabic}}
\flushleft{\begin{malayalam}
അവന്‍ (ഫിര്‍ഔന്‍) പറഞ്ഞു: കുട്ടിയായിരുന്നപ്പോള്‍ ഞങ്ങളുടെ കൂട്ടത്തില്‍ ‍നിന്നെ ഞങ്ങള്‍ വളര്‍ത്തിയില്ലേ? നിന്‍റെ ആയുസ്സില്‍ ‍കുറെ കൊല്ലങ്ങള്‍ ഞങ്ങളുടെ ഇടയില്‍ ‍നീ കഴിച്ചുകൂട്ടിയിട്ടുമുണ്ട്‌
\end{malayalam}}
\flushright{\begin{Arabic}
\quranayah[26][19]
\end{Arabic}}
\flushleft{\begin{malayalam}
നീ ചെയ്ത നിന്‍റെ ആ(ദുഷ്‌) പ്രവൃത്തി നീ ചെയ്യുകയുമുണ്ടായി നീ നന്ദികെട്ടവരുടെ കൂട്ടത്തില്‍തന്നെയാകുന്നു
\end{malayalam}}
\flushright{\begin{Arabic}
\quranayah[26][20]
\end{Arabic}}
\flushleft{\begin{malayalam}
അദ്ദേഹം (മൂസാ) പറഞ്ഞു: ഞാന്‍ അന്ന് അത് ചെയ്യുകയുണ്ടായി എന്നാല്‍ ‍ഞാന്‍ പിഴവ് പറ്റിയവരുടെ കൂട്ടത്തിലായിരുന്നു
\end{malayalam}}
\flushright{\begin{Arabic}
\quranayah[26][21]
\end{Arabic}}
\flushleft{\begin{malayalam}
അങ്ങനെ നിങ്ങളെപ്പറ്റി ഭയം തോന്നിയപ്പോള്‍ ഞാന്‍ നിങ്ങളില്‍നിന്ന് ഓടിപ്പോയി അനന്തരം എന്‍റെ രക്ഷിതാവ് എനിക്ക് തത്വജ്ഞാനം നല്‍കുകയും, അവന്‍ എന്നെ ദൂതന്‍മാരില്‍ ഒരാളാക്കുകയും ചെയ്തു
\end{malayalam}}
\flushright{\begin{Arabic}
\quranayah[26][22]
\end{Arabic}}
\flushleft{\begin{malayalam}
എനിക്ക് നീ ചെയ്തു തന്നതായി നീ എടുത്തുപറയുന്ന ആ അനുഗ്രഹം ഇസ്രായീല്‍സന്തതികളെ നീ അടിമകളാക്കി വെച്ചതിനാല്‍ ഉണ്ടായതത്രെ
\end{malayalam}}
\flushright{\begin{Arabic}
\quranayah[26][23]
\end{Arabic}}
\flushleft{\begin{malayalam}
ഫിര്‍ഔന്‍ പറഞ്ഞു: എന്താണ് ഈ ലോകരക്ഷിതാവ് എന്ന് പറയുന്നത്‌?
\end{malayalam}}
\flushright{\begin{Arabic}
\quranayah[26][24]
\end{Arabic}}
\flushleft{\begin{malayalam}
അദ്ദേഹം (മൂസാ) പറഞ്ഞു: ആകാശങ്ങളുടെയും ഭൂമിയുടെയും അവയ്ക്കിടയിലുള്ളതിന്‍റെയും രക്ഷിതാവാകുന്നു നിങ്ങള്‍ ദൃഢ വിശ്വാസമുള്ളവരാണെങ്കില്‍
\end{malayalam}}
\flushright{\begin{Arabic}
\quranayah[26][25]
\end{Arabic}}
\flushleft{\begin{malayalam}
അവന്‍ (ഫിര്‍ഔന്‍) തന്‍റെ ചുറ്റുമുള്ളവരോട് പറഞ്ഞു: എന്താ നിങ്ങള്‍ ശ്രദ്ധിച്ചു കേള്‍ക്കുന്നില്ലേ?
\end{malayalam}}
\flushright{\begin{Arabic}
\quranayah[26][26]
\end{Arabic}}
\flushleft{\begin{malayalam}
അദ്ദേഹം (മൂസാ) പറഞ്ഞു: നിങ്ങളുടെ രക്ഷിതാവും നിങ്ങളുടെ പൂര്‍വ്വ പിതാക്കളുടെ രക്ഷിതാവുമത്രെ (അവന്‍)
\end{malayalam}}
\flushright{\begin{Arabic}
\quranayah[26][27]
\end{Arabic}}
\flushleft{\begin{malayalam}
അവന്‍ (ഫിര്‍ഔന്‍) പറഞ്ഞു: നിങ്ങളുടെ അടുത്തേക്ക് നിയോഗിക്കപ്പെട്ട നിങ്ങളുടെ ഈ ദൂതനുണ്ടല്ലോ തീര്‍ച്ചയായും അവന്‍ ഒരു ഭ്രാന്തന്‍ തന്നെയാണ്‌
\end{malayalam}}
\flushright{\begin{Arabic}
\quranayah[26][28]
\end{Arabic}}
\flushleft{\begin{malayalam}
അദ്ദേഹം (മൂസാ) പറഞ്ഞു: ഉദയസ്ഥാനത്തിന്‍റെയും അസ്തമയസ്ഥാനത്തിന്‍റെയും അവയ്ക്കിടയിലുള്ളതിന്‍റെയും രക്ഷിതാവത്രെ (അവന്‍) നിങ്ങള്‍ ചിന്തിച്ചു മനസ്സിലാക്കുന്നവരാണെങ്കില്‍
\end{malayalam}}
\flushright{\begin{Arabic}
\quranayah[26][29]
\end{Arabic}}
\flushleft{\begin{malayalam}
അവന്‍ (ഫിര്‍ഔന്‍) പറഞ്ഞു: ഞാനല്ലാത്ത വല്ല ദൈവത്തേയും നീ സ്വീകരിക്കുകയാണെങ്കില്‍ ‍തീര്‍ച്ചയായും നിന്നെ ഞാന്‍ തടവുകാരുടെ കൂട്ടത്തിലാക്കുന്നതാണ്‌
\end{malayalam}}
\flushright{\begin{Arabic}
\quranayah[26][30]
\end{Arabic}}
\flushleft{\begin{malayalam}
അദ്ദേഹം (മൂസാ) പറഞ്ഞു: സ്പഷ്ടമായ എന്തെങ്കിലും തെളിവു ഞാന്‍ നിനക്ക് കൊണ്ടു വന്നു കാണിച്ചാലും (നീ സമ്മതിക്കുകയില്ലേ?)
\end{malayalam}}
\flushright{\begin{Arabic}
\quranayah[26][31]
\end{Arabic}}
\flushleft{\begin{malayalam}
അവന്‍ (ഫിര്‍ഔന്‍) പറഞ്ഞു: എന്നാല്‍ ‍നീ അത് കൊണ്ട് വരിക നീ സത്യവാന്‍മാരില്‍പെട്ടവനാണെങ്കില്‍
\end{malayalam}}
\flushright{\begin{Arabic}
\quranayah[26][32]
\end{Arabic}}
\flushleft{\begin{malayalam}
അപ്പോള്‍ അദ്ദേഹം (മൂസാ) തന്‍റെ വടി താഴെയിട്ടു അപ്പോഴതാ അത് പ്രത്യക്ഷമായ ഒരു സര്‍പ്പമായി മാറുന്നു
\end{malayalam}}
\flushright{\begin{Arabic}
\quranayah[26][33]
\end{Arabic}}
\flushleft{\begin{malayalam}
അദ്ദേഹം തന്‍റെ കൈ പുറത്തേക്കെടുക്കുകയും ചെയ്തു അപ്പോഴതാ അത് കാണികള്‍ക്ക് വെള്ളനിറമാകുന്നു
\end{malayalam}}
\flushright{\begin{Arabic}
\quranayah[26][34]
\end{Arabic}}
\flushleft{\begin{malayalam}
തന്‍റെ ചുറ്റുമുള്ള പ്രമുഖന്‍മാരോട് അവന്‍ (ഫിര്‍ഔന്‍) പറഞ്ഞു: തീര്‍ച്ചയായും ഇവന്‍ വിവരമുള്ള ഒരു ജാലവിദ്യക്കാരന്‍ തന്നെയാണ്‌
\end{malayalam}}
\flushright{\begin{Arabic}
\quranayah[26][35]
\end{Arabic}}
\flushleft{\begin{malayalam}
തന്‍റെ ജാലവിദ്യകൊണ്ട് നിങ്ങളുടെ നാട്ടില്‍നിന്ന് നിങ്ങളെ പുറത്താക്കാന്‍ അവന്‍ ഉദ്ദേശിക്കുന്നു അതിനാല്‍ ‍നിങ്ങള്‍ എന്ത് നിര്‍ദേശിക്കുന്നു?
\end{malayalam}}
\flushright{\begin{Arabic}
\quranayah[26][36]
\end{Arabic}}
\flushleft{\begin{malayalam}
അവര്‍ പറഞ്ഞു: അവന്നും അവന്‍റെ സഹോദരന്നും താങ്കള്‍ സാവകാശം നല്‍കുക ആളുകളെ വിളിച്ചുകൂട്ടാന്‍ നഗരങ്ങളിലേക്ക് താങ്കള്‍ ദൂതന്‍മാരെ നിയോഗിക്കുകയും ചെയ്യുക
\end{malayalam}}
\flushright{\begin{Arabic}
\quranayah[26][37]
\end{Arabic}}
\flushleft{\begin{malayalam}
എല്ലാ വിവരമുള്ള ജാലവിദ്യക്കാരെയും അവര്‍ താങ്കളുടെ അടുത്ത് കൊണ്ടു വരട്ടെ
\end{malayalam}}
\flushright{\begin{Arabic}
\quranayah[26][38]
\end{Arabic}}
\flushleft{\begin{malayalam}
അങ്ങനെ അറിയപ്പെട്ട ഒരു ദിവസം നിശ്ചിതമായ ഒരു സമയത്ത് ജാലവിദ്യക്കാര്‍ ഒരുമിച്ചുകൂട്ടപ്പെട്ടു
\end{malayalam}}
\flushright{\begin{Arabic}
\quranayah[26][39]
\end{Arabic}}
\flushleft{\begin{malayalam}
ജനങ്ങളോട് ചോദിക്കപ്പെട്ടു: നിങ്ങള്‍ സമ്മേളിക്കുന്നുണ്ടല്ലോ?
\end{malayalam}}
\flushright{\begin{Arabic}
\quranayah[26][40]
\end{Arabic}}
\flushleft{\begin{malayalam}
ജാലവിദ്യക്കാരാണ് വിജയികളാകുന്നതെങ്കില്‍ ‍നമുക്കവരെ പിന്തുടരാമല്ലോ!
\end{malayalam}}
\flushright{\begin{Arabic}
\quranayah[26][41]
\end{Arabic}}
\flushleft{\begin{malayalam}
അങ്ങനെ ജാലവിദ്യക്കാര്‍ വന്നെത്തിയപ്പോള്‍ ഫിര്‍ഔനോട് അവര്‍ ചോദിച്ചു: ഞങ്ങളാണ് വിജയികളാകുന്നതെങ്കില്‍ ‍തീര്‍ച്ചയായും ഞങ്ങള്‍ക്ക് പ്രതിഫലമുണ്ടായിരിക്കുമോ?
\end{malayalam}}
\flushright{\begin{Arabic}
\quranayah[26][42]
\end{Arabic}}
\flushleft{\begin{malayalam}
അവന്‍ (ഫിര്‍ഔന്‍) പറഞ്ഞു: അതെ, തീര്‍ച്ചയായും നിങ്ങള്‍ സാമീപ്യം നല്‍കപ്പെടുന്നവരുടെ കൂട്ടത്തിലായിരിക്കും
\end{malayalam}}
\flushright{\begin{Arabic}
\quranayah[26][43]
\end{Arabic}}
\flushleft{\begin{malayalam}
മൂസാ അവരോട് പറഞ്ഞു: നിങ്ങള്‍ക്ക് ഇടാനുള്ളതെല്ലാം നിങ്ങള്‍ ഇട്ടുകൊള്ളുക
\end{malayalam}}
\flushright{\begin{Arabic}
\quranayah[26][44]
\end{Arabic}}
\flushleft{\begin{malayalam}
അപ്പോള്‍ തങ്ങളുടെ കയറുകളും വടികളും അവര്‍ ഇട്ടു അവര്‍ പറയുകയും ചെയ്തു: ഫിര്‍ഔന്‍റെ പ്രതാപം തന്നെയാണ സത്യം! തീര്‍ച്ചയായും ഞങ്ങള്‍ തന്നെയായിരിക്കും വിജയികള്‍
\end{malayalam}}
\flushright{\begin{Arabic}
\quranayah[26][45]
\end{Arabic}}
\flushleft{\begin{malayalam}
അനന്തരം മൂസാ തന്‍റെ വടി താഴെയിട്ടു അപ്പോഴതാ അത് അവര്‍ വ്യാജമായി നിര്‍മിച്ചിരുന്നതിനെയെല്ലാം വിഴുങ്ങിക്കളയുന്നു
\end{malayalam}}
\flushright{\begin{Arabic}
\quranayah[26][46]
\end{Arabic}}
\flushleft{\begin{malayalam}
അപ്പോള്‍ ജാലവിദ്യക്കാര്‍ സാഷ്ടാംഗത്തിലായി വീണു
\end{malayalam}}
\flushright{\begin{Arabic}
\quranayah[26][47]
\end{Arabic}}
\flushleft{\begin{malayalam}
അവര്‍ പറഞ്ഞു: ലോകരക്ഷിതാവില്‍ ‍ഞങ്ങള്‍ വിശ്വസിച്ചിരിക്കുന്നു
\end{malayalam}}
\flushright{\begin{Arabic}
\quranayah[26][48]
\end{Arabic}}
\flushleft{\begin{malayalam}
അതായത് മൂസായുടെയും ഹാറൂന്‍റെയും രക്ഷിതാവില്‍
\end{malayalam}}
\flushright{\begin{Arabic}
\quranayah[26][49]
\end{Arabic}}
\flushleft{\begin{malayalam}
അവന്‍ (ഫിര്‍ഔന്‍) പറഞ്ഞു: ഞാന്‍ നിങ്ങള്‍ക്ക് അനുവാദം തരുന്നതിന് മുമ്പായി നിങ്ങള്‍ അവനില്‍ ‍വിശ്വസിച്ചുവെന്നോ? തീര്‍ച്ചയായും ഇവന്‍ നിങ്ങള്‍ക്ക് ജാലവിദ്യ പഠിപ്പിച്ച നിങ്ങളുടെ തലവന്‍ തന്നെയാണ് വഴിയെ നിങ്ങള്‍ അറിഞ്ഞു കൊള്ളും തീര്‍ച്ചയായും നിങ്ങളുടെ കൈകളും, നിങ്ങളുടെ കാലുകളും എതിര്‍ ‍വശങ്ങളില്‍നിന്നായിക്കൊണ്ട് ഞാന്‍ മുറിച്ചു കളയുകയും, നിങ്ങളെ മുഴുവന്‍ ഞാന്‍ ക്രൂശിക്കുകയും ചെയ്യുന്നതാണ്‌
\end{malayalam}}
\flushright{\begin{Arabic}
\quranayah[26][50]
\end{Arabic}}
\flushleft{\begin{malayalam}
അവര്‍ പറഞ്ഞു: കുഴപ്പമില്ല തീര്‍ച്ചയായും ഞങ്ങള്‍ ഞങ്ങളുടെ രക്ഷിതാവിങ്കലേക്ക് മടങ്ങിപ്പോകുന്നവരാകുന്നു
\end{malayalam}}
\flushright{\begin{Arabic}
\quranayah[26][51]
\end{Arabic}}
\flushleft{\begin{malayalam}
ഞങ്ങള്‍ ആദ്യമായി വിശ്വസിച്ചവരായതിനാല്‍ ‍ഞങ്ങളുടെ തെറ്റുകള്‍ ഞങ്ങളുടെ രക്ഷിതാവ് ഞങ്ങള്‍ക്ക് പൊറുത്തുതരുമെന്ന് ഞങ്ങള്‍ ആശിക്കുന്നു
\end{malayalam}}
\flushright{\begin{Arabic}
\quranayah[26][52]
\end{Arabic}}
\flushleft{\begin{malayalam}
മൂസായ്ക്ക് നാം ബോധനം നല്‍കി: എന്‍റെ ദാസന്‍മാരെയും കൊണ്ട് രാത്രിയില്‍ ‍നീ പുറപ്പെട്ടുകൊള്ളുക തീര്‍ച്ചയായും (ശത്രുക്കള്‍) നിങ്ങളെ പിന്തുടരാന്‍ പോകുകയാണ്‌
\end{malayalam}}
\flushright{\begin{Arabic}
\quranayah[26][53]
\end{Arabic}}
\flushleft{\begin{malayalam}
അപ്പോള്‍ ഫിര്‍ഔന്‍ ആളുകളെ വിളിച്ചുകൂട്ടാന്‍ പട്ടണങ്ങളിലേക്ക് ദൂതന്‍മാരെ അയച്ചു
\end{malayalam}}
\flushright{\begin{Arabic}
\quranayah[26][54]
\end{Arabic}}
\flushleft{\begin{malayalam}
തീര്‍ച്ചയായും ഇവര്‍ കുറച്ച് പേര്‍ മാത്രമുള്ള ഒരു സംഘമാകുന്നു
\end{malayalam}}
\flushright{\begin{Arabic}
\quranayah[26][55]
\end{Arabic}}
\flushleft{\begin{malayalam}
തീര്‍ച്ചയായും അവര്‍ നമ്മെ അരിശം കൊള്ളിക്കുന്നവരാകുന്നു
\end{malayalam}}
\flushright{\begin{Arabic}
\quranayah[26][56]
\end{Arabic}}
\flushleft{\begin{malayalam}
തീര്‍ച്ചയായും നാം സംഘടിതരും ജാഗരൂകരുമാകുന്നു (എന്നിങ്ങനെ വിളിച്ചുപറയാനാണ് ഫിര്‍ഔന്‍ നിര്‍ദേശിച്ചത്‌)
\end{malayalam}}
\flushright{\begin{Arabic}
\quranayah[26][57]
\end{Arabic}}
\flushleft{\begin{malayalam}
അങ്ങനെ തോട്ടങ്ങളില്‍നിന്നും നീരുറവകളില്‍നിന്നും നാം അവരെ പുറത്തിറക്കി
\end{malayalam}}
\flushright{\begin{Arabic}
\quranayah[26][58]
\end{Arabic}}
\flushleft{\begin{malayalam}
ഭണ്ഡാരങ്ങളില്‍നിന്നും മാന്യമായ വാസസ്ഥലങ്ങളില്‍നിന്നും
\end{malayalam}}
\flushright{\begin{Arabic}
\quranayah[26][59]
\end{Arabic}}
\flushleft{\begin{malayalam}
അപ്രകാരമത്രെ (നമ്മുടെ നടപടി) അതൊക്കെ ഇസ്രായീല്‍ ‍സന്തതികള്‍ക്ക് നാം അവകാശപ്പെടുത്തികൊടുക്കുകയും ചെയ്തു
\end{malayalam}}
\flushright{\begin{Arabic}
\quranayah[26][60]
\end{Arabic}}
\flushleft{\begin{malayalam}
എന്നിട്ട് അവര്‍ (ഫിര്‍ഔനും സംഘവും) ഉദയവേളയില്‍ അവരുടെ (ഇസ്രായീല്യരുടെ) പിന്നാലെ ചെന്നു
\end{malayalam}}
\flushright{\begin{Arabic}
\quranayah[26][61]
\end{Arabic}}
\flushleft{\begin{malayalam}
അങ്ങനെ രണ്ട് സംഘവും പരസ്പരം കണ്ടപ്പോള്‍ മൂസായുടെ അനുചരന്‍മാര്‍ പറഞ്ഞു: തീര്‍ച്ചയായും നാം പിടിയിലകപ്പെടാന്‍ പോകുകയാണ്‌
\end{malayalam}}
\flushright{\begin{Arabic}
\quranayah[26][62]
\end{Arabic}}
\flushleft{\begin{malayalam}
അദ്ദേഹം (മൂസാ) പറഞ്ഞു: ഒരിക്കലുമില്ല, തീര്‍ച്ചയായും എന്നോടൊപ്പം എന്‍റെ രക്ഷിതാവുണ്ട് അവന്‍ എനിക്ക് വഴി കാണിച്ചുതരും
\end{malayalam}}
\flushright{\begin{Arabic}
\quranayah[26][63]
\end{Arabic}}
\flushleft{\begin{malayalam}
അപ്പോള്‍ നാം മൂസായ്ക്ക് ബോധനം നല്‍കി; നീ നിന്‍റെ വടികൊണ്ട് കടലില്‍ ‍അടിക്കൂ എന്ന് അങ്ങനെ അത് (കടല്‍ ‍) പിളരുകയും എന്നിട്ട് (വെള്ളത്തിന്‍റെ) ഓരോ പൊളിയും വലിയ പര്‍വ്വതം പോലെ ആയിത്തീരുകയും ചെയ്തു
\end{malayalam}}
\flushright{\begin{Arabic}
\quranayah[26][64]
\end{Arabic}}
\flushleft{\begin{malayalam}
മറ്റവരെ (ഫിര്‍ഔന്‍റെ പക്ഷം) യും നാം അതിന്‍റെ അടുത്തെത്തിക്കുകയുണ്ടായി
\end{malayalam}}
\flushright{\begin{Arabic}
\quranayah[26][65]
\end{Arabic}}
\flushleft{\begin{malayalam}
മൂസായെയും അദ്ദേഹത്തോടൊപ്പമുള്ളവരെയും മുഴുവന്‍ നാം രക്ഷപ്പെടുത്തി
\end{malayalam}}
\flushright{\begin{Arabic}
\quranayah[26][66]
\end{Arabic}}
\flushleft{\begin{malayalam}
പിന്നെ മറ്റവരെ നാം മുക്കി നശിപ്പിച്ചു
\end{malayalam}}
\flushright{\begin{Arabic}
\quranayah[26][67]
\end{Arabic}}
\flushleft{\begin{malayalam}
തീര്‍ച്ചയായും അതില്‍ (സത്യനിഷേധികള്‍ക്ക്‌) ഒരു ദൃഷ്ടാന്തമുണ്ട് എന്നാല്‍ അവരില്‍ അധികപേരും വിശ്വസിക്കുന്നവരായില്ല
\end{malayalam}}
\flushright{\begin{Arabic}
\quranayah[26][68]
\end{Arabic}}
\flushleft{\begin{malayalam}
തീര്‍ച്ചയായും നിന്‍റെ രക്ഷിതാവ് തന്നെയാണ് പ്രതാപിയും കരുണാനിധിയും
\end{malayalam}}
\flushright{\begin{Arabic}
\quranayah[26][69]
\end{Arabic}}
\flushleft{\begin{malayalam}
ഇബ്രാഹീമിന്‍റെ വൃത്താന്തവും അവര്‍ക്ക് നീ വായിച്ചുകേള്‍പിക്കുക
\end{malayalam}}
\flushright{\begin{Arabic}
\quranayah[26][70]
\end{Arabic}}
\flushleft{\begin{malayalam}
അതായത് നിങ്ങള്‍ എന്തൊന്നിനെയാണ് ആരാധിച്ചു കൊണ്ടിരിക്കുന്നത് എന്ന് തന്‍റെ പിതാവിനോടും, തന്‍റെ ജനങ്ങളോടും അദ്ദേഹം ചോദിച്ച സന്ദര്‍ഭം
\end{malayalam}}
\flushright{\begin{Arabic}
\quranayah[26][71]
\end{Arabic}}
\flushleft{\begin{malayalam}
അവര്‍ പറഞ്ഞു: ഞങ്ങള്‍ ചില വിഗ്രഹങ്ങളെ ആരാധിക്കുകയും അവയുടെ മുമ്പില്‍ ‍ഭജനമിരിക്കുകയും ചെയ്യുന്നു
\end{malayalam}}
\flushright{\begin{Arabic}
\quranayah[26][72]
\end{Arabic}}
\flushleft{\begin{malayalam}
അദ്ദേഹം പറഞ്ഞു: നിങ്ങള്‍ പ്രാര്‍ത്ഥിക്കുമ്പോള്‍ അവരത് കേള്‍ക്കുമോ?
\end{malayalam}}
\flushright{\begin{Arabic}
\quranayah[26][73]
\end{Arabic}}
\flushleft{\begin{malayalam}
അഥവാ, അവര്‍ നിങ്ങള്‍ക്ക് ഉപകാരമോ ഉപദ്രവമോ ചെയ്യുമോ?
\end{malayalam}}
\flushright{\begin{Arabic}
\quranayah[26][74]
\end{Arabic}}
\flushleft{\begin{malayalam}
അവര്‍ പറഞ്ഞു: അല്ല, ഞങ്ങളുടെ പിതാക്കള്‍ അപ്രകാരം ചെയ്യുന്നതായി ഞങ്ങള്‍ കണ്ടിരിക്കുന്നു (എന്ന് മാത്രം)
\end{malayalam}}
\flushright{\begin{Arabic}
\quranayah[26][75]
\end{Arabic}}
\flushleft{\begin{malayalam}
അദ്ദേഹം പറഞ്ഞു: അപ്പോള്‍ നിങ്ങള്‍ ആരാധിച്ചു കൊണ്ടിരിക്കുന്നത് എന്തിനെയാണെന്ന് നിങ്ങള്‍ ചിന്തിച്ചു നോക്കിയിട്ടുണ്ടോ?
\end{malayalam}}
\flushright{\begin{Arabic}
\quranayah[26][76]
\end{Arabic}}
\flushleft{\begin{malayalam}
നിങ്ങളും നിങ്ങളുടെ പൂര്‍വ്വപിതാക്കളും
\end{malayalam}}
\flushright{\begin{Arabic}
\quranayah[26][77]
\end{Arabic}}
\flushleft{\begin{malayalam}
എന്നാല്‍ അവര്‍ (ദൈവങ്ങള്‍) എന്‍റെ ശത്രുക്കളാകുന്നു ലോകരക്ഷിതാവ് ഒഴികെ
\end{malayalam}}
\flushright{\begin{Arabic}
\quranayah[26][78]
\end{Arabic}}
\flushleft{\begin{malayalam}
അതായത് എന്നെ സൃഷ്ടിച്ച് എനിക്ക് മാര്‍ഗദര്‍ശനം നല്‍കിക്കൊണ്ടിരിക്കുന്നവന്‍
\end{malayalam}}
\flushright{\begin{Arabic}
\quranayah[26][79]
\end{Arabic}}
\flushleft{\begin{malayalam}
എനിക്ക് ആഹാരം തരികയും കുടിനീര്‍ തരികയും ചെയ്യുന്നവന്‍
\end{malayalam}}
\flushright{\begin{Arabic}
\quranayah[26][80]
\end{Arabic}}
\flushleft{\begin{malayalam}
എനിക്ക് രോഗം ബാധിച്ചാല്‍ അവനാണ് എന്നെ സുഖപ്പെടുത്തുന്നത്‌
\end{malayalam}}
\flushright{\begin{Arabic}
\quranayah[26][81]
\end{Arabic}}
\flushleft{\begin{malayalam}
എന്നെ മരിപ്പിക്കുകയും പിന്നീട് ജീവിപ്പിക്കുകയും ചെയ്യുന്നവന്‍
\end{malayalam}}
\flushright{\begin{Arabic}
\quranayah[26][82]
\end{Arabic}}
\flushleft{\begin{malayalam}
പ്രതിഫലത്തിന്‍റെ നാളില്‍ ഏതൊരുവന്‍ എന്‍റെ തെറ്റ് പൊറുത്തുതരുമെന്ന് ഞാന്‍ ആശിക്കുന്നുവോ അവന്‍
\end{malayalam}}
\flushright{\begin{Arabic}
\quranayah[26][83]
\end{Arabic}}
\flushleft{\begin{malayalam}
എന്‍റെ രക്ഷിതാവേ, എനിക്ക് നീ തത്വജ്ഞാനം നല്‍കുകയും എന്നെ നീ സജ്ജനങ്ങളോടൊപ്പം ചേര്‍ക്കുകയും ചെയ്യേണമേ
\end{malayalam}}
\flushright{\begin{Arabic}
\quranayah[26][84]
\end{Arabic}}
\flushleft{\begin{malayalam}
പില്‍ക്കാലക്കാര്‍ക്കിടയില്‍ എനിക്ക് നീ സല്‍കീര്‍ത്തി ഉണ്ടാക്കേണമേ
\end{malayalam}}
\flushright{\begin{Arabic}
\quranayah[26][85]
\end{Arabic}}
\flushleft{\begin{malayalam}
എന്നെ നീ സുഖസമ്പൂര്‍ണ്ണമായ സ്വര്‍ഗത്തിന്‍റെ അവകാശികളില്‍ ‍പെട്ടവനാക്കേണമേ
\end{malayalam}}
\flushright{\begin{Arabic}
\quranayah[26][86]
\end{Arabic}}
\flushleft{\begin{malayalam}
എന്‍റെ പിതാവിന് നീ പൊറുത്തുകൊടുക്കേണമേ തീര്‍ച്ചയായും അദ്ദേഹം വഴിപിഴച്ചവരുടെ കൂട്ടത്തിലായിരിക്കുന്നു
\end{malayalam}}
\flushright{\begin{Arabic}
\quranayah[26][87]
\end{Arabic}}
\flushleft{\begin{malayalam}
അവര്‍ (മനുഷ്യര്‍) ഉയിര്‍ത്തെഴുന്നേല്‍പിക്കപ്പെടുന്ന ദിവസം എന്നെ നീ അപമാനത്തിലാക്കരുതേ
\end{malayalam}}
\flushright{\begin{Arabic}
\quranayah[26][88]
\end{Arabic}}
\flushleft{\begin{malayalam}
അതായത് സ്വത്തോ സന്താനങ്ങളോ പ്രയോജനപ്പെടാത്ത ദിവസം
\end{malayalam}}
\flushright{\begin{Arabic}
\quranayah[26][89]
\end{Arabic}}
\flushleft{\begin{malayalam}
കുറ്റമറ്റ ഹൃദയവുമായി അല്ലാഹുവിങ്കല്‍ ‍ചെന്നവര്‍ക്കൊഴികെ
\end{malayalam}}
\flushright{\begin{Arabic}
\quranayah[26][90]
\end{Arabic}}
\flushleft{\begin{malayalam}
(അന്ന്‌) സൂക്ഷ്മത പാലിക്കുന്നവര്‍ക്ക് സ്വര്‍ഗം അടുപ്പിക്കപ്പെടുന്നതാണ്‌
\end{malayalam}}
\flushright{\begin{Arabic}
\quranayah[26][91]
\end{Arabic}}
\flushleft{\begin{malayalam}
ദുര്‍മാര്‍ഗികള്‍ക്ക് നരകം തുറന്നു കാണിക്കപ്പെടുന്നതുമാണ്‌
\end{malayalam}}
\flushright{\begin{Arabic}
\quranayah[26][92]
\end{Arabic}}
\flushleft{\begin{malayalam}
അവരോട് ചോദിക്കപ്പെടുകയും ചെയ്യും: നിങ്ങള്‍ ആരാധിച്ചിരുന്നതെല്ലാം എവിടെപ്പോയി?
\end{malayalam}}
\flushright{\begin{Arabic}
\quranayah[26][93]
\end{Arabic}}
\flushleft{\begin{malayalam}
അല്ലാഹുവിനു പുറമെ അവര്‍ നിങ്ങളെ സഹായിക്കുകയോ, സ്വയം സഹായം നേടുകയോ ചെയ്യുന്നുണ്ടോ?
\end{malayalam}}
\flushright{\begin{Arabic}
\quranayah[26][94]
\end{Arabic}}
\flushleft{\begin{malayalam}
തുര്‍ന്ന് അവരും (ആരാധ്യന്‍മാര്‍) ആ ദുര്‍മാര്‍ഗികളും അതില്‍ ‍(നരകത്തില്‍ ‍) മുഖം കുത്തി വീഴ്ത്തപ്പെടുന്നതാണ്‌
\end{malayalam}}
\flushright{\begin{Arabic}
\quranayah[26][95]
\end{Arabic}}
\flushleft{\begin{malayalam}
ഇബ്ലീസിന്‍റെ മുഴുവന്‍ സൈന്യങ്ങളും
\end{malayalam}}
\flushright{\begin{Arabic}
\quranayah[26][96]
\end{Arabic}}
\flushleft{\begin{malayalam}
അവിടെ വെച്ച് അന്യോന്യം വഴക്ക് കൂടിക്കൊണ്ടിരിക്കെ അവര്‍ പറയും:
\end{malayalam}}
\flushright{\begin{Arabic}
\quranayah[26][97]
\end{Arabic}}
\flushleft{\begin{malayalam}
അല്ലാഹുവാണ സത്യം! ഞങ്ങള്‍ വ്യക്തമായ വഴികേടില്‍ ‍തന്നെയായിരുന്നു
\end{malayalam}}
\flushright{\begin{Arabic}
\quranayah[26][98]
\end{Arabic}}
\flushleft{\begin{malayalam}
നിങ്ങള്‍ക്ക് ഞങ്ങള്‍ ലോകരക്ഷിതാവിനോട് തുല്യത കല്‍പിക്കുന്ന സമയത്ത്‌
\end{malayalam}}
\flushright{\begin{Arabic}
\quranayah[26][99]
\end{Arabic}}
\flushleft{\begin{malayalam}
ഞങ്ങളെ വഴിപിഴപ്പിച്ചത് ആ കുറ്റവാളികളല്ലാതെ മറ്റാരുമല്ല
\end{malayalam}}
\flushright{\begin{Arabic}
\quranayah[26][100]
\end{Arabic}}
\flushleft{\begin{malayalam}
ഇപ്പോള്‍ ഞങ്ങള്‍ക്ക് ശുപാര്‍ശക്കാരായി ആരുമില്ല
\end{malayalam}}
\flushright{\begin{Arabic}
\quranayah[26][101]
\end{Arabic}}
\flushleft{\begin{malayalam}
ഉറ്റ സുഹൃത്തുമില്ല
\end{malayalam}}
\flushright{\begin{Arabic}
\quranayah[26][102]
\end{Arabic}}
\flushleft{\begin{malayalam}
അതിനാല്‍ ‍ഞങ്ങള്‍ക്കൊന്നു മടങ്ങിപ്പോകാന്‍ കഴിഞ്ഞിരുന്നെങ്കില്‍ എങ്കില്‍ ‍ഞങ്ങള്‍ സത്യവിശ്വാസികളുടെ കൂട്ടത്തിലാകുമായിരുന്നു
\end{malayalam}}
\flushright{\begin{Arabic}
\quranayah[26][103]
\end{Arabic}}
\flushleft{\begin{malayalam}
തീര്‍ച്ചയായും അതില്‍ (മനുഷ്യര്‍ക്ക്‌) ഒരു ദൃഷ്ടാന്തമുണ്ട് എന്നാല്‍ അവരില്‍ അധികപേരും വിശ്വസിക്കുന്നവരായില്ല
\end{malayalam}}
\flushright{\begin{Arabic}
\quranayah[26][104]
\end{Arabic}}
\flushleft{\begin{malayalam}
തീര്‍ച്ചയായും നിന്‍റെ രക്ഷിതാവ് തന്നെയാകുന്നു പ്രതാപിയും കരുണാനിധിയും
\end{malayalam}}
\flushright{\begin{Arabic}
\quranayah[26][105]
\end{Arabic}}
\flushleft{\begin{malayalam}
നൂഹിന്‍റെ ജനത ദൈവദൂതന്‍മാരെ നിഷേധിച്ചു തള്ളി
\end{malayalam}}
\flushright{\begin{Arabic}
\quranayah[26][106]
\end{Arabic}}
\flushleft{\begin{malayalam}
അവരുടെ സഹോദരന്‍ നൂഹ് അവരോട് ഇപ്രകാരം പറഞ്ഞ സന്ദര്‍ഭം: നിങ്ങള്‍ സൂക്ഷ്മത പാലിക്കുന്നില്ലേ?
\end{malayalam}}
\flushright{\begin{Arabic}
\quranayah[26][107]
\end{Arabic}}
\flushleft{\begin{malayalam}
തീര്‍ച്ചയായും ഞാന്‍ നിങ്ങള്‍ക്ക് വിശ്വസ്തനായ ഒരു ദൂതനാകുന്നു
\end{malayalam}}
\flushright{\begin{Arabic}
\quranayah[26][108]
\end{Arabic}}
\flushleft{\begin{malayalam}
അതിനാല്‍ ‍നിങ്ങള്‍ അല്ലാഹുവെ സൂക്ഷിക്കുകയും, എന്നെ അനുസരിക്കുകയും ചെയ്യുവിന്‍
\end{malayalam}}
\flushright{\begin{Arabic}
\quranayah[26][109]
\end{Arabic}}
\flushleft{\begin{malayalam}
ഇതിന്‍റെ പേരില്‍ ‍യാതൊരു പ്രതിഫലവും ഞാന്‍ നിങ്ങളോട് ചോദിക്കുന്നില്ല എനിക്കുള്ള പ്രതിഫലം ലോകരക്ഷിതാവിങ്കല്‍ ‍നിന്ന് മാത്രമാകുന്നു
\end{malayalam}}
\flushright{\begin{Arabic}
\quranayah[26][110]
\end{Arabic}}
\flushleft{\begin{malayalam}
അതിനാല്‍ ‍നിങ്ങള്‍ അല്ലാഹുവെ സൂക്ഷിക്കുകയും, എന്നെ അനുസരിക്കുകയും ചെയ്യുക
\end{malayalam}}
\flushright{\begin{Arabic}
\quranayah[26][111]
\end{Arabic}}
\flushleft{\begin{malayalam}
അവര്‍ പറഞ്ഞു; നിന്നെ പിന്തുടര്‍ന്നിട്ടുള്ളത് ഏറ്റവും താഴ്ന്ന ആളുകളായിരിക്കെ ഞങ്ങള്‍ നിന്നെ വിശ്വസിക്കുകയോ?
\end{malayalam}}
\flushright{\begin{Arabic}
\quranayah[26][112]
\end{Arabic}}
\flushleft{\begin{malayalam}
അദ്ദേഹം പറഞ്ഞു: അവര്‍ പ്രവര്‍ത്തിച്ചു കൊണ്ടിരിക്കുന്നതിനെപ്പറ്റി എനിക്ക് എന്തറിയാം?
\end{malayalam}}
\flushright{\begin{Arabic}
\quranayah[26][113]
\end{Arabic}}
\flushleft{\begin{malayalam}
അവരെ വിചാരണ നടത്തുക എന്നത് എന്‍റെ രക്ഷിതാവിന്‍റെ ബാധ്യത മാത്രമാകുന്നു നിങ്ങള്‍ ബോധമുള്ളവരായെങ്കില്‍ !
\end{malayalam}}
\flushright{\begin{Arabic}
\quranayah[26][114]
\end{Arabic}}
\flushleft{\begin{malayalam}
സത്യവിശ്വാസികളെ ഞാന്‍ ഒരിക്കലും ആട്ടിക്കളയുന്നതല്ല
\end{malayalam}}
\flushright{\begin{Arabic}
\quranayah[26][115]
\end{Arabic}}
\flushleft{\begin{malayalam}
ഞാന്‍ വ്യക്തമായ ഒരു താക്കീതുകാരന്‍ മാത്രമാകുന്നു
\end{malayalam}}
\flushright{\begin{Arabic}
\quranayah[26][116]
\end{Arabic}}
\flushleft{\begin{malayalam}
അവര്‍ പറഞ്ഞു: നൂഹേ, നീ (ഇതില്‍നിന്നു) വിരമിക്കുന്നില്ലെങ്കില്‍തീര്‍ച്ചയായും നീ എറിഞ്ഞു കൊല്ലപ്പെടുന്നവരുടെ കൂട്ടത്തിലായിരിക്കും
\end{malayalam}}
\flushright{\begin{Arabic}
\quranayah[26][117]
\end{Arabic}}
\flushleft{\begin{malayalam}
അദ്ദേഹം പറഞ്ഞു: എന്‍റെ രക്ഷിതാവേ, തീര്‍ച്ചയായും എന്‍റെ ജനത എന്നെ നിഷേധിച്ചു തള്ളിയിരിക്കുന്നു
\end{malayalam}}
\flushright{\begin{Arabic}
\quranayah[26][118]
\end{Arabic}}
\flushleft{\begin{malayalam}
അതിനാല്‍ എനിക്കും അവര്‍ക്കുമിടയില്‍ ‍നീ ഒരു തുറന്ന തീരുമാനമെടുക്കുകയും, എന്നെയും എന്‍റെ കൂടെയുള്ള വിശ്വാസികളെയും നീ രക്ഷപ്പെടുത്തുകയും ചെയ്യേണമേ
\end{malayalam}}
\flushright{\begin{Arabic}
\quranayah[26][119]
\end{Arabic}}
\flushleft{\begin{malayalam}
അപ്പോള്‍ അദ്ദേഹത്തെയും അദ്ദേഹത്തിന്‍റെ കൂടെയുള്ളവരെയും ഭാരം നിറക്കപ്പെട്ട കപ്പലില്‍ ‍നാം രക്ഷപ്പെടുത്തി
\end{malayalam}}
\flushright{\begin{Arabic}
\quranayah[26][120]
\end{Arabic}}
\flushleft{\begin{malayalam}
പിന്നെ ബാക്കിയുള്ളവരെ അതിന് ശേഷം നാം മുക്കി നശിപ്പിക്കുകയും ചെയ്തു
\end{malayalam}}
\flushright{\begin{Arabic}
\quranayah[26][121]
\end{Arabic}}
\flushleft{\begin{malayalam}
തീര്‍ച്ചയായും അതില്‍ ‍(മനുഷ്യര്‍ക്ക്‌) ഒരു ദൃഷ്ടാന്തമുണ്ട് എന്നാല്‍ അവരില്‍അധികപേരും വിശ്വസിക്കുന്നവരായില്ല
\end{malayalam}}
\flushright{\begin{Arabic}
\quranayah[26][122]
\end{Arabic}}
\flushleft{\begin{malayalam}
തീര്‍ച്ചയായും നിന്‍റെ രക്ഷിതാവ് തന്നെയാകുന്നു പ്രതാപിയും കരുണാനിധിയും
\end{malayalam}}
\flushright{\begin{Arabic}
\quranayah[26][123]
\end{Arabic}}
\flushleft{\begin{malayalam}
ആദ് സമുദായം ദൈവദൂതന്‍മാരെ നിഷേധിച്ചു തള്ളി
\end{malayalam}}
\flushright{\begin{Arabic}
\quranayah[26][124]
\end{Arabic}}
\flushleft{\begin{malayalam}
അവരുടെ സഹോദരന്‍ ഹൂദ് അവരോട് പറഞ്ഞ സന്ദര്‍ഭം : നിങ്ങള്‍ സൂക്ഷ്മത പാലിക്കുന്നില്ലേ?
\end{malayalam}}
\flushright{\begin{Arabic}
\quranayah[26][125]
\end{Arabic}}
\flushleft{\begin{malayalam}
തീര്‍ച്ചയായും ഞാന്‍ നിങ്ങള്‍ക്ക് വിശ്വസ്തനായ ഒരു ദൂതനാകുന്നു
\end{malayalam}}
\flushright{\begin{Arabic}
\quranayah[26][126]
\end{Arabic}}
\flushleft{\begin{malayalam}
അതിനാല്‍ ‍നിങ്ങള്‍ അല്ലാഹുവെ സൂക്ഷിക്കുകയും, എന്നെ അനുസരിക്കുകയും ചെയ്യുവിന്‍
\end{malayalam}}
\flushright{\begin{Arabic}
\quranayah[26][127]
\end{Arabic}}
\flushleft{\begin{malayalam}
ഇതിന്‍റെ പേരില്‍ ‍ഞാന്‍ നിങ്ങളോട് യാതൊരു പ്രതിഫലവും ചോദിക്കുന്നില്ല എനിക്കുള്ള പ്രതിഫലം ലോകരക്ഷിതാവിങ്കല്‍നിന്ന് മാത്രമാകുന്നു
\end{malayalam}}
\flushright{\begin{Arabic}
\quranayah[26][128]
\end{Arabic}}
\flushleft{\begin{malayalam}
വൃഥാ പൊങ്ങച്ചം കാണിക്കുവാനായി എല്ലാ കുന്നിന്‍ പ്രദേശങ്ങളിലും നിങ്ങള്‍ പ്രതാപചിഹ്നങ്ങള്‍ (ഗോപുരങ്ങള്‍) കെട്ടിപൊക്കുകയാണോ?
\end{malayalam}}
\flushright{\begin{Arabic}
\quranayah[26][129]
\end{Arabic}}
\flushleft{\begin{malayalam}
നിങ്ങള്‍ക്ക് എന്നെന്നും താമസിക്കാമെന്ന ഭാവേന നിങ്ങള്‍ മഹാസൌധങ്ങള്‍ ഉണ്ടാക്കുകയുമാണോ?
\end{malayalam}}
\flushright{\begin{Arabic}
\quranayah[26][130]
\end{Arabic}}
\flushleft{\begin{malayalam}
നിങ്ങള്‍ ബലം പ്രയോഗിക്കുകയാണെങ്കില്‍ ‍നിഷ്ഠൂരന്‍മാരായിക്കൊണ്ട് നിങ്ങള്‍ ബലം പ്രയോഗിക്കുന്നു
\end{malayalam}}
\flushright{\begin{Arabic}
\quranayah[26][131]
\end{Arabic}}
\flushleft{\begin{malayalam}
ആകയാല്‍ ‍നിങ്ങള്‍ അല്ലാഹുവെ സൂക്ഷിക്കുകയും എന്നെ അനുസരിക്കുകയും ചെയ്യുക
\end{malayalam}}
\flushright{\begin{Arabic}
\quranayah[26][132]
\end{Arabic}}
\flushleft{\begin{malayalam}
നിങ്ങള്‍ക്ക് തന്നെ അറിയാവുന്നവ (സുഖസൌകര്യങ്ങള്‍) മുഖേന നിങ്ങളെ സഹായിച്ചവനെ നിങ്ങള്‍ സൂക്ഷിക്കുക
\end{malayalam}}
\flushright{\begin{Arabic}
\quranayah[26][133]
\end{Arabic}}
\flushleft{\begin{malayalam}
കന്നുകാലികളും സന്താനങ്ങളും മുഖേന അവന്‍ നിങ്ങളെ സഹായിച്ചിരിക്കുന്നു
\end{malayalam}}
\flushright{\begin{Arabic}
\quranayah[26][134]
\end{Arabic}}
\flushleft{\begin{malayalam}
തോട്ടങ്ങളും അരുവികളും മുഖേനയും
\end{malayalam}}
\flushright{\begin{Arabic}
\quranayah[26][135]
\end{Arabic}}
\flushleft{\begin{malayalam}
നിങ്ങളുടെ കാര്യത്തില്‍ ‍ഭയങ്കരമായ ഒരു ദിവസത്തെ ശിക്ഷ തീര്‍ച്ചയായും ഞാന്‍ ഭയപ്പെടുന്നു
\end{malayalam}}
\flushright{\begin{Arabic}
\quranayah[26][136]
\end{Arabic}}
\flushleft{\begin{malayalam}
അവര്‍ പറഞ്ഞു: നീ ഉപദേശം നല്‍കിയാലും, ഉപദേശിക്കുന്നവരുടെ കൂട്ടത്തില്‍ ആയില്ലെങ്കിലും ഞങ്ങളെ സംബന്ധിച്ചിടത്തോളം സമമാകുന്നു
\end{malayalam}}
\flushright{\begin{Arabic}
\quranayah[26][137]
\end{Arabic}}
\flushleft{\begin{malayalam}
ഇത് പൂര്‍വ്വികന്‍മാരുടെ സമ്പ്രദായം തന്നെയാകുന്നു
\end{malayalam}}
\flushright{\begin{Arabic}
\quranayah[26][138]
\end{Arabic}}
\flushleft{\begin{malayalam}
ഞങ്ങള്‍ ശിക്ഷിക്കപ്പെടുന്നവരല്ല
\end{malayalam}}
\flushright{\begin{Arabic}
\quranayah[26][139]
\end{Arabic}}
\flushleft{\begin{malayalam}
അങ്ങനെ അവര്‍ അദ്ദേഹത്തെ നിഷേധിച്ചു തള്ളുകയും, അതിനാല്‍ ‍നാം അവരെ നശിപ്പിക്കുകയും ചെയ്തു തീര്‍ച്ചയായും അതില്‍ (മനുഷ്യര്‍ക്ക്‌) ഒരു ദൃഷ്ടാന്തമുണ്ട് എന്നാല്‍ അവരില്‍ അധികപേരും വിശ്വസിക്കുന്നവരായില്ല
\end{malayalam}}
\flushright{\begin{Arabic}
\quranayah[26][140]
\end{Arabic}}
\flushleft{\begin{malayalam}
തീര്‍ച്ചയായും നിന്‍റെ രക്ഷിതാവ് തന്നെയാകുന്നു പ്രതാപിയും കരുണാനിധിയും
\end{malayalam}}
\flushright{\begin{Arabic}
\quranayah[26][141]
\end{Arabic}}
\flushleft{\begin{malayalam}
ഥമൂദ് സമുദായം ദൈവദൂതന്‍മാരെ നിഷേധിച്ചു തള്ളി
\end{malayalam}}
\flushright{\begin{Arabic}
\quranayah[26][142]
\end{Arabic}}
\flushleft{\begin{malayalam}
അവരുടെ സഹോദരന്‍ സ്വാലിഹ് അവരോട് പറഞ്ഞ സന്ദര്‍ഭം: നിങ്ങള്‍ സൂക്ഷ്മത പാലിക്കുന്നില്ലേ?
\end{malayalam}}
\flushright{\begin{Arabic}
\quranayah[26][143]
\end{Arabic}}
\flushleft{\begin{malayalam}
തീര്‍ച്ചയായും ഞാന്‍ നിങ്ങള്‍ക്ക് വിശ്വസ്തനായ ഒരു ദൂതനാകുന്നു
\end{malayalam}}
\flushright{\begin{Arabic}
\quranayah[26][144]
\end{Arabic}}
\flushleft{\begin{malayalam}
അതിനാല്‍ ‍നിങ്ങള്‍ അല്ലാഹുവെ സൂക്ഷിക്കുകയും, എന്നെ അനുസരിക്കുകയും ചെയ്യുവിന്‍
\end{malayalam}}
\flushright{\begin{Arabic}
\quranayah[26][145]
\end{Arabic}}
\flushleft{\begin{malayalam}
നിങ്ങളോട് ഞാന്‍ ഇതിന്‍റെ പേരില്‍ ‍യാതൊരു പ്രതിഫലവും ചോദിക്കുന്നില്ല എനിക്കുള്ള പ്രതിഫലം ലോകരക്ഷിതാവിങ്കല്‍ ‍നിന്ന് മാത്രമാകുന്നു
\end{malayalam}}
\flushright{\begin{Arabic}
\quranayah[26][146]
\end{Arabic}}
\flushleft{\begin{malayalam}
ഇവിടെയുള്ളതില്‍(സമൃദ്ധിയില്‍) നിര്‍ഭയരായിക്കഴിയാന്‍ നിങ്ങള്‍ വിട്ടേക്കപ്പെടുമോ?
\end{malayalam}}
\flushright{\begin{Arabic}
\quranayah[26][147]
\end{Arabic}}
\flushleft{\begin{malayalam}
അതായത് തോട്ടങ്ങളിലും അരുവികളിലും
\end{malayalam}}
\flushright{\begin{Arabic}
\quranayah[26][148]
\end{Arabic}}
\flushleft{\begin{malayalam}
വയലുകളിലും, കുല ഭാരം തൂങ്ങുന്ന ഈന്തപ്പനകളിലും
\end{malayalam}}
\flushright{\begin{Arabic}
\quranayah[26][149]
\end{Arabic}}
\flushleft{\begin{malayalam}
നിങ്ങള്‍ സന്തോഷപ്രമത്തരായിക്കൊണ്ട് പര്‍വ്വതങ്ങളില്‍ ‍വീടുകള്‍ തുരന്നുണ്ടാക്കുകയും ചെയ്യുന്നു
\end{malayalam}}
\flushright{\begin{Arabic}
\quranayah[26][150]
\end{Arabic}}
\flushleft{\begin{malayalam}
ആകയാല്‍നിങ്ങള്‍ അല്ലാഹുവെ സൂക്ഷിക്കുകയും, എന്നെ അനുസരിക്കുകയും ചെയ്യുവിന്‍
\end{malayalam}}
\flushright{\begin{Arabic}
\quranayah[26][151]
\end{Arabic}}
\flushleft{\begin{malayalam}
അതിക്രമകാരികളുടെ കല്‍പന നിങ്ങള്‍ അനുസരിച്ചു പോകരുത്‌
\end{malayalam}}
\flushright{\begin{Arabic}
\quranayah[26][152]
\end{Arabic}}
\flushleft{\begin{malayalam}
ഭൂമിയില്‍ ‍കുഴപ്പമുണ്ടാക്കുകയും, നന്‍മവരുത്താതിരിക്കുകയും ചെയ്യുന്നവരുടെ
\end{malayalam}}
\flushright{\begin{Arabic}
\quranayah[26][153]
\end{Arabic}}
\flushleft{\begin{malayalam}
അവര്‍ പറഞ്ഞു: നീ മാരണം ബാധിച്ചവരില്‍പെട്ട ഒരാള്‍ മാത്രമാകുന്നു
\end{malayalam}}
\flushright{\begin{Arabic}
\quranayah[26][154]
\end{Arabic}}
\flushleft{\begin{malayalam}
നീ ഞങ്ങളെപ്പോലെയുള്ള ഒരു മനുഷ്യന്‍ മാത്രമാണ് അതിനാല്‍ ‍നീ സത്യവാന്‍മാരില്‍പെട്ടവനാണെങ്കില്‍ ‍വല്ല ദൃഷ്ടാന്തവും കൊണ്ട് വരൂ
\end{malayalam}}
\flushright{\begin{Arabic}
\quranayah[26][155]
\end{Arabic}}
\flushleft{\begin{malayalam}
അദ്ദേഹം പറഞ്ഞു: ഇതാ ഒരു ഒട്ടകം അതിന്ന് വെള്ളം കുടിക്കാന്‍ ഒരു ഊഴമുണ്ട് നിങ്ങള്‍ക്കും ഒരു ഊഴമുണ്ട്‌; ഒരു നിശ്ചിത ദിവസത്തില്‍
\end{malayalam}}
\flushright{\begin{Arabic}
\quranayah[26][156]
\end{Arabic}}
\flushleft{\begin{malayalam}
നിങ്ങള്‍ അതിന് യാതൊരു ദ്രോഹവും ഏല്‍പിക്കരുത് (അങ്ങനെ ചെയ്യുന്ന പക്ഷം) ഭയങ്കരമായ ഒരു ദിവസത്തെ ശിക്ഷ നിങ്ങളെ പിടികൂടും
\end{malayalam}}
\flushright{\begin{Arabic}
\quranayah[26][157]
\end{Arabic}}
\flushleft{\begin{malayalam}
എന്നാല്‍ അവര്‍ അതിനെ വെട്ടിക്കൊന്നു അങ്ങനെ അവര്‍ ഖേദക്കാരായിത്തീര്‍ന്നു
\end{malayalam}}
\flushright{\begin{Arabic}
\quranayah[26][158]
\end{Arabic}}
\flushleft{\begin{malayalam}
ഉടനെ ശിക്ഷ അവരെ പിടികൂടി തീര്‍ച്ചയായും അതില്‍(മനുഷ്യര്‍ക്ക്‌) ഒരു ദൃഷ്ടാന്തമുണ്ട് എന്നാല്‍ അവരില്‍ അധികപേരും വിശ്വസിക്കുന്നവരായില്ല
\end{malayalam}}
\flushright{\begin{Arabic}
\quranayah[26][159]
\end{Arabic}}
\flushleft{\begin{malayalam}
തീര്‍ച്ചയായും നിന്‍റെ രക്ഷിതാവ് തന്നെയാകുന്നു പ്രതാപിയും കരുണാനിധിയും
\end{malayalam}}
\flushright{\begin{Arabic}
\quranayah[26][160]
\end{Arabic}}
\flushleft{\begin{malayalam}
ലൂത്വിന്‍റെ ജനത ദൈവദൂതന്‍മാരെ നിഷേധിച്ചു തള്ളി
\end{malayalam}}
\flushright{\begin{Arabic}
\quranayah[26][161]
\end{Arabic}}
\flushleft{\begin{malayalam}
അവരുടെ സഹോദരന്‍ ലൂത്വ് അവരോട് പറഞ്ഞ സന്ദര്‍ഭം: നിങ്ങള്‍ സൂക്ഷ്മത പാലിക്കുന്നില്ലേ?
\end{malayalam}}
\flushright{\begin{Arabic}
\quranayah[26][162]
\end{Arabic}}
\flushleft{\begin{malayalam}
തീര്‍ച്ചയായും ഞാന്‍ നിങ്ങള്‍ക്ക് വിശ്വസ്തനായ ഒരു ദൂതനാകുന്നു
\end{malayalam}}
\flushright{\begin{Arabic}
\quranayah[26][163]
\end{Arabic}}
\flushleft{\begin{malayalam}
അതിനാല്‍ ‍നിങ്ങള്‍ അല്ലാഹുവെ സൂക്ഷിക്കുകയും, എന്നെ അനുസരിക്കുകയും ചെയ്യുവിന്‍
\end{malayalam}}
\flushright{\begin{Arabic}
\quranayah[26][164]
\end{Arabic}}
\flushleft{\begin{malayalam}
ഇതിന്‍റെ പേരില്‍ ‍നിങ്ങളോട് ഞാന്‍ യാതൊരു പ്രതിഫലവും ചോദിക്കുന്നില്ല എനിക്കുള്ള പ്രതിഫലം ലോകരക്ഷിതാവിങ്കല്‍ ‍നിന്ന് മാത്രമാകുന്നു
\end{malayalam}}
\flushright{\begin{Arabic}
\quranayah[26][165]
\end{Arabic}}
\flushleft{\begin{malayalam}
നിങ്ങള്‍ ലോകരില്‍ ‍നിന്ന് ആണുങ്ങളുടെ അടുക്കല്‍ ‍ചെല്ലുകയാണോ?
\end{malayalam}}
\flushright{\begin{Arabic}
\quranayah[26][166]
\end{Arabic}}
\flushleft{\begin{malayalam}
നിങ്ങളുടെ രക്ഷിതാവ് നിങ്ങള്‍ക്ക് വേണ്ടി സൃഷ്ടിച്ചു തന്നിട്ടുള്ള നിങ്ങളുടെ ഇണകളെ വിട്ടുകളയുകയുമാണോ? അല്ല, നിങ്ങള്‍ അതിക്രമകാരികളായ ഒരു ജനത തന്നെ
\end{malayalam}}
\flushright{\begin{Arabic}
\quranayah[26][167]
\end{Arabic}}
\flushleft{\begin{malayalam}
അവര്‍ പറഞ്ഞു: ലൂത്വേ, നീ (ഇതില്‍നിന്ന്‌) വിരമിച്ചില്ലെങ്കില്‍ ‍തീര്‍ച്ചയായും നീ (നാട്ടില്‍നിന്ന്‌) പുറത്താക്കപ്പെടുന്നവരുടെ കൂട്ടത്തിലായിരിക്കും
\end{malayalam}}
\flushright{\begin{Arabic}
\quranayah[26][168]
\end{Arabic}}
\flushleft{\begin{malayalam}
അദ്ദേഹം പറഞ്ഞു: തീര്‍ച്ചയായും ഞാന്‍ നിങ്ങളുടെ പ്രവൃത്തിയെ വെറുക്കുന്നവരുടെ കൂട്ടത്തിലാകുന്നു
\end{malayalam}}
\flushright{\begin{Arabic}
\quranayah[26][169]
\end{Arabic}}
\flushleft{\begin{malayalam}
അദ്ദേഹം (പ്രാര്‍ത്ഥിച്ചു:) എന്‍റെ രക്ഷിതാവേ, എന്നെയും എന്‍റെ കുടുംബത്തേയും ഇവര്‍ പ്രവര്‍ത്തിച്ചു കൊണ്ടിരിക്കുന്നതില്‍ ‍നിന്ന് നീ രക്ഷപ്പെടുത്തേണമേ
\end{malayalam}}
\flushright{\begin{Arabic}
\quranayah[26][170]
\end{Arabic}}
\flushleft{\begin{malayalam}
അപ്പോള്‍ അദ്ദേഹത്തെയും അദ്ദേഹത്തിന്‍റെ കുടുംബത്തേയും മുഴുവന്‍ നാം രക്ഷപ്പെടുത്തി
\end{malayalam}}
\flushright{\begin{Arabic}
\quranayah[26][171]
\end{Arabic}}
\flushleft{\begin{malayalam}
പിന്‍മാറി നിന്നവരില്‍ ഒരു കിഴവി ഒഴികെ
\end{malayalam}}
\flushright{\begin{Arabic}
\quranayah[26][172]
\end{Arabic}}
\flushleft{\begin{malayalam}
പിന്നീട് മറ്റുള്ളവരെ നാം തകര്‍ത്തുകളഞ്ഞു
\end{malayalam}}
\flushright{\begin{Arabic}
\quranayah[26][173]
\end{Arabic}}
\flushleft{\begin{malayalam}
അവരുടെ മേല്‍ ‍നാം ഒരു തരം മഴ വര്‍ഷിപ്പിക്കുകയും ചെയ്തു താക്കീത് നല്‍കപ്പെട്ടവര്‍ക്ക് ലഭിച്ച ആ മഴ എത്ര മോശം!
\end{malayalam}}
\flushright{\begin{Arabic}
\quranayah[26][174]
\end{Arabic}}
\flushleft{\begin{malayalam}
തീര്‍ച്ചയായും അതില്‍(മനുഷ്യര്‍ക്ക്‌) ഒരു ദൃഷ്ടാന്തമുണ്ട് എന്നാല്‍ അവരില്‍ ‍അധികപേരും വിശ്വസിക്കുന്നവരായില്ല
\end{malayalam}}
\flushright{\begin{Arabic}
\quranayah[26][175]
\end{Arabic}}
\flushleft{\begin{malayalam}
തീര്‍ച്ചയായും നിന്‍റെ രക്ഷിതാവ് തന്നെയാണ് പ്രതാപിയും കരുണാനിധിയും
\end{malayalam}}
\flushright{\begin{Arabic}
\quranayah[26][176]
\end{Arabic}}
\flushleft{\begin{malayalam}
ഐക്കത്തില്‍(മരക്കൂട്ടങ്ങള്‍ക്കിടയില്‍) താമസിച്ചിരുന്നവരും ദൈവദൂതന്‍ ‍മാരെ നിഷേധിച്ചുതള്ളി
\end{malayalam}}
\flushright{\begin{Arabic}
\quranayah[26][177]
\end{Arabic}}
\flushleft{\begin{malayalam}
അവരോട് ശുഐബ് പറഞ്ഞ സന്ദര്‍ഭം: നിങ്ങള്‍ സൂക്ഷ്മത പാലിക്കുന്നില്ലേ?
\end{malayalam}}
\flushright{\begin{Arabic}
\quranayah[26][178]
\end{Arabic}}
\flushleft{\begin{malayalam}
തീര്‍ച്ചയായും ഞാന്‍ നിങ്ങള്‍ക്ക് വിശ്വസ്തനായ ഒരു ദൂതനാകുന്നു
\end{malayalam}}
\flushright{\begin{Arabic}
\quranayah[26][179]
\end{Arabic}}
\flushleft{\begin{malayalam}
അതിനാല്‍ ‍നിങ്ങള്‍ അല്ലാഹുവെ സൂക്ഷിക്കുകയും, എന്നെ അനുസരിക്കുകയും ചെയ്യുവിന്‍
\end{malayalam}}
\flushright{\begin{Arabic}
\quranayah[26][180]
\end{Arabic}}
\flushleft{\begin{malayalam}
ഇതിന്‍റെ പേരില്‍ ‍യാതൊരു പ്രതിഫലവും ഞാന്‍ നിങ്ങളോട് ചോദിക്കുന്നില്ല എനിക്കുള്ള പ്രതിഫലം ലോകരക്ഷിതാവിങ്കല്‍ ‍നിന്ന് മാത്രമാകുന്നു
\end{malayalam}}
\flushright{\begin{Arabic}
\quranayah[26][181]
\end{Arabic}}
\flushleft{\begin{malayalam}
നിങ്ങള്‍ അളവു പൂര്‍ത്തിയാക്കികൊടുക്കുക നിങ്ങള്‍ (ജനങ്ങള്‍ക്ക്‌) നഷ്ടമുണ്ടാക്കുന്നവരുടെ കൂട്ടത്തിലാകരുത്‌
\end{malayalam}}
\flushright{\begin{Arabic}
\quranayah[26][182]
\end{Arabic}}
\flushleft{\begin{malayalam}
കൃത്രിമമില്ലാത്ത തുലാസ് കൊണ്ട് നിങ്ങള്‍ തൂക്കുക
\end{malayalam}}
\flushright{\begin{Arabic}
\quranayah[26][183]
\end{Arabic}}
\flushleft{\begin{malayalam}
ജനങ്ങള്‍ക്ക് അവരുടെ സാധനങ്ങളില്‍ ‍നിങ്ങള്‍ കമ്മിവരുത്തരുത് നാശകാരികളായിക്കൊണ്ട് നിങ്ങള്‍ ഭൂമിയില്‍ അതിക്രമം പ്രവര്‍ത്തിക്കരുത്‌
\end{malayalam}}
\flushright{\begin{Arabic}
\quranayah[26][184]
\end{Arabic}}
\flushleft{\begin{malayalam}
നിങ്ങളെയും പൂര്‍വ്വതലമുറകളെയും സൃഷ്ടിച്ചവനെ നിങ്ങള്‍ സൂക്ഷിക്കുകയും ചെയ്യുക
\end{malayalam}}
\flushright{\begin{Arabic}
\quranayah[26][185]
\end{Arabic}}
\flushleft{\begin{malayalam}
അവര്‍ പറഞ്ഞു: നീ മാരണം ബാധിച്ചവരില്‍ ഒരാള്‍ മാത്രമാകുന്നു
\end{malayalam}}
\flushright{\begin{Arabic}
\quranayah[26][186]
\end{Arabic}}
\flushleft{\begin{malayalam}
നീ ഞങ്ങളെപ്പോലെയുള്ള ഒരു മനുഷ്യന്‍ മാത്രമാകുന്നു തീര്‍ച്ചയായും നീ വ്യാജവാദികളില്‍പെട്ടവനാണെന്നാണ് ഞങ്ങള്‍ വിചാരിക്കുന്നത്‌
\end{malayalam}}
\flushright{\begin{Arabic}
\quranayah[26][187]
\end{Arabic}}
\flushleft{\begin{malayalam}
അതുകൊണ്ട് നീ സത്യവാന്‍മാരില്‍പെട്ടവനാണെങ്കില്‍ ‍ആകാശത്ത് നിന്നുള്ള കഷ്ണങ്ങള്‍ ഞങ്ങളുടെ മേല്‍ ‍നീ വീഴ്ത്തുക
\end{malayalam}}
\flushright{\begin{Arabic}
\quranayah[26][188]
\end{Arabic}}
\flushleft{\begin{malayalam}
അദ്ദേഹം പറഞ്ഞു; നിങ്ങള്‍ പ്രവര്‍ത്തിക്കുന്നതിനെ പറ്റി എന്‍റെ രക്ഷിതാവ് നല്ലവണ്ണം അറിയുന്നവനാകുന്നു
\end{malayalam}}
\flushright{\begin{Arabic}
\quranayah[26][189]
\end{Arabic}}
\flushleft{\begin{malayalam}
അങ്ങനെ അവര്‍ അദ്ദേഹത്തെ നിഷേധിച്ചുതള്ളി അതിനാല്‍ ‍മേഘത്തണല്‍മൂടിയ ദിവസത്തെ ശിക്ഷ അവരെ പിടികൂടി തീര്‍ച്ചയായും അത് ഭയങ്കരമായ ഒരു ദിവസത്തെ ശിക്ഷ തന്നെയായിരുന്നു
\end{malayalam}}
\flushright{\begin{Arabic}
\quranayah[26][190]
\end{Arabic}}
\flushleft{\begin{malayalam}
തീര്‍ച്ചയായും അതില്‍(മനുഷ്യര്‍ക്ക്‌) ഒരു ദൃഷ്ടാന്തമുണ്ട് എന്നാല്‍ അവരില്‍അധികപേരും വിശ്വസിക്കുന്നവരായില്ല
\end{malayalam}}
\flushright{\begin{Arabic}
\quranayah[26][191]
\end{Arabic}}
\flushleft{\begin{malayalam}
തീര്‍ച്ചയായും നിന്‍റെ രക്ഷിതാവ് തന്നെയാകുന്നു പ്രതാപിയും കരുണാനിധിയും
\end{malayalam}}
\flushright{\begin{Arabic}
\quranayah[26][192]
\end{Arabic}}
\flushleft{\begin{malayalam}
തീര്‍ച്ചയായും ഇത് (ഖുര്‍ആന്‍) ലോകരക്ഷിതാവ് അവതരിപ്പിച്ചത് തന്നെയാകുന്നു
\end{malayalam}}
\flushright{\begin{Arabic}
\quranayah[26][193]
\end{Arabic}}
\flushleft{\begin{malayalam}
വിശ്വസ്താത്മാവ് (ജിബ്‌രീല്‍) അതും കൊണ്ട് ഇറങ്ങിയിരിക്കുന്നു
\end{malayalam}}
\flushright{\begin{Arabic}
\quranayah[26][194]
\end{Arabic}}
\flushleft{\begin{malayalam}
നിന്‍റെ ഹൃദയത്തില്‍ ‍നീ താക്കീത് നല്‍കുന്നവരുടെ കൂട്ടത്തിലായിരിക്കുവാന്‍ വേണ്ടിയത്രെ അത്‌
\end{malayalam}}
\flushright{\begin{Arabic}
\quranayah[26][195]
\end{Arabic}}
\flushleft{\begin{malayalam}
സ്പഷ്ടമായ അറബി ഭാഷയിലാണ് (അത് അവതരിപ്പിച്ചത്‌)
\end{malayalam}}
\flushright{\begin{Arabic}
\quranayah[26][196]
\end{Arabic}}
\flushleft{\begin{malayalam}
തീര്‍ച്ചയായും അത് മുന്‍ഗാമികളുടെ വേദഗ്രന്ഥങ്ങളിലുണ്ട്‌
\end{malayalam}}
\flushright{\begin{Arabic}
\quranayah[26][197]
\end{Arabic}}
\flushleft{\begin{malayalam}
ഇസ്രായീല്‍സന്തതികളിലെ പണ്ഡിതന്‍മാര്‍ക്ക് അത് അറിയാം എന്ന കാര്യം ഇവര്‍ക്ക് (അവിശ്വാസികള്‍ക്ക്‌) ഒരു ദൃഷ്ടാന്തമായിരിക്കുന്നില്ലേ ?
\end{malayalam}}
\flushright{\begin{Arabic}
\quranayah[26][198]
\end{Arabic}}
\flushleft{\begin{malayalam}
നാം അത് അനറബികളില്‍ ഒരാളുടെ മേല്‍അവതരിപ്പിക്കുകയും,
\end{malayalam}}
\flushright{\begin{Arabic}
\quranayah[26][199]
\end{Arabic}}
\flushleft{\begin{malayalam}
എന്നിട്ട് അയാള്‍ അത് അവര്‍ക്ക് ഓതികേള്‍പിക്കുകയും ചെയ്തിരുന്നുവെങ്കില്‍ അവരതില്‍ ‍വിശ്വസിക്കുമായിരുന്നില്ല
\end{malayalam}}
\flushright{\begin{Arabic}
\quranayah[26][200]
\end{Arabic}}
\flushleft{\begin{malayalam}
അപ്രകാരം കുറ്റവാളികളുടെ ഹൃദയങ്ങളില്‍ ‍നാം അത് (അവിശ്വാസം) കടത്തിവിട്ടിരിക്കയാണ്‌
\end{malayalam}}
\flushright{\begin{Arabic}
\quranayah[26][201]
\end{Arabic}}
\flushleft{\begin{malayalam}
വേദനയേറിയ ശിക്ഷ കാണുന്നത് വരേക്കും അവരതില്‍ ‍വിശ്വസിക്കുകയില്ല
\end{malayalam}}
\flushright{\begin{Arabic}
\quranayah[26][202]
\end{Arabic}}
\flushleft{\begin{malayalam}
അവര്‍ ഓര്‍ക്കാത്ത നിലയില്‍ ‍പെട്ടെന്നായിരിക്കും അതവര്‍ക്ക് വന്നെത്തുന്നത്‌
\end{malayalam}}
\flushright{\begin{Arabic}
\quranayah[26][203]
\end{Arabic}}
\flushleft{\begin{malayalam}
ഞങ്ങള്‍ക്ക് (ഒരല്‍പം) അവധി നല്‍കപ്പെടുമോ? എന്ന് അപ്പോള്‍ അവര്‍ ചോദിച്ചേക്കും
\end{malayalam}}
\flushright{\begin{Arabic}
\quranayah[26][204]
\end{Arabic}}
\flushleft{\begin{malayalam}
എന്നാല്‍ ‍നമ്മുടെ ശിക്ഷയെപ്പറ്റിയാണോ അവര്‍ ധൃതികൂട്ടികൊണ്ടിരിക്കുന്നത്‌?
\end{malayalam}}
\flushright{\begin{Arabic}
\quranayah[26][205]
\end{Arabic}}
\flushleft{\begin{malayalam}
എന്നാല്‍ ‍നീ ആലോചിച്ചിട്ടുണ്ടോ? നാം അവര്‍ക്ക് കുറെ കൊല്ലങ്ങളോളം സുഖസൌകര്യം നല്‍കുകയും,
\end{malayalam}}
\flushright{\begin{Arabic}
\quranayah[26][206]
\end{Arabic}}
\flushleft{\begin{malayalam}
അനന്തരം അവര്‍ക്ക് താക്കീത് നല്‍കപ്പെട്ടു കൊണ്ടിരിക്കുന്ന ശിക്ഷ അവര്‍ക്ക് വരികയും ചെയ്തുവെന്ന് വെക്കുക
\end{malayalam}}
\flushright{\begin{Arabic}
\quranayah[26][207]
\end{Arabic}}
\flushleft{\begin{malayalam}
(എന്നാലും) അവര്‍ക്ക് നല്‍കപ്പെട്ടിരുന്ന ആ സുഖസൌകര്യങ്ങള്‍ അവര്‍ക്കൊരു പ്രയോജനവും ചെയ്യുമായിരുന്നില്ല
\end{malayalam}}
\flushright{\begin{Arabic}
\quranayah[26][208]
\end{Arabic}}
\flushleft{\begin{malayalam}
ഒരു രാജ്യവും നാം നശിപ്പിച്ചിട്ടില്ല; അതിന് താക്കീതുകാര്‍ ഉണ്ടായിട്ടല്ലാതെ
\end{malayalam}}
\flushright{\begin{Arabic}
\quranayah[26][209]
\end{Arabic}}
\flushleft{\begin{malayalam}
ഓര്‍മപ്പെടുത്തുവാന്‍ വേണ്ടിയത്രെ അത് നാം അക്രമം ചെയ്യുന്നവനായിട്ടില്ല
\end{malayalam}}
\flushright{\begin{Arabic}
\quranayah[26][210]
\end{Arabic}}
\flushleft{\begin{malayalam}
ഇതുമായി (ഖുര്‍ആനുമായി) പിശാചുക്കള്‍ ഇറങ്ങി വന്നിട്ടില്ല
\end{malayalam}}
\flushright{\begin{Arabic}
\quranayah[26][211]
\end{Arabic}}
\flushleft{\begin{malayalam}
അതവര്‍ക്ക് അനുയോജ്യമാവുകയുമില്ല അതവര്‍ക്ക് സാധിക്കുന്നതുമല്ല
\end{malayalam}}
\flushright{\begin{Arabic}
\quranayah[26][212]
\end{Arabic}}
\flushleft{\begin{malayalam}
തീര്‍ച്ചയായും അവര്‍ (ദിവ്യസന്ദേശം) കേള്‍ക്കുന്നതില്‍നിന്ന് അകറ്റപെട്ടവരാകുന്നു
\end{malayalam}}
\flushright{\begin{Arabic}
\quranayah[26][213]
\end{Arabic}}
\flushleft{\begin{malayalam}
ആകയാല്‍ അല്ലാഹുവോടൊപ്പം മറ്റൊരു ദൈവത്തേയും നീ വിളിച്ചു പ്രാര്‍ത്ഥിക്കരുത് എങ്കില്‍ ‍നീ ശിക്ഷിക്കപ്പെടുന്നവരുടെ കൂട്ടത്തിലായിരിക്കും
\end{malayalam}}
\flushright{\begin{Arabic}
\quranayah[26][214]
\end{Arabic}}
\flushleft{\begin{malayalam}
നിന്‍റെ അടുത്ത ബന്ധുക്കള്‍ക്ക് നീ താക്കീത് നല്‍കുക
\end{malayalam}}
\flushright{\begin{Arabic}
\quranayah[26][215]
\end{Arabic}}
\flushleft{\begin{malayalam}
നിന്നെ പിന്തുടര്‍ന്ന സത്യവിശ്വാസികള്‍ക്ക് നിന്‍റെ ചിറക് താഴ്ത്തികൊടുക്കുകയും ചെയ്യുക
\end{malayalam}}
\flushright{\begin{Arabic}
\quranayah[26][216]
\end{Arabic}}
\flushleft{\begin{malayalam}
ഇനി അവര്‍ നിന്നെ അനുസരിക്കാതിരിക്കുന്ന പക്ഷം, നിങ്ങള്‍ പ്രവര്‍ത്തിക്കുന്നതിനൊന്നും ഞാന്‍ ഉത്തരവാദിയല്ലെന്ന് നീ പറഞ്ഞേക്കുക
\end{malayalam}}
\flushright{\begin{Arabic}
\quranayah[26][217]
\end{Arabic}}
\flushleft{\begin{malayalam}
പ്രതാപിയും കരുണാനിധിയുമായിട്ടുള്ളവനെ നീ ഭരമേല്‍പിക്കുകയും ചെയ്യുക
\end{malayalam}}
\flushright{\begin{Arabic}
\quranayah[26][218]
\end{Arabic}}
\flushleft{\begin{malayalam}
നീ നിന്നു പ്രാര്‍ത്ഥിക്കുന്ന സമയത്ത് നിന്നെ കാണുന്നവനത്രെ അവന്‍
\end{malayalam}}
\flushright{\begin{Arabic}
\quranayah[26][219]
\end{Arabic}}
\flushleft{\begin{malayalam}
സാഷ്ടാംഗംചെയ്യുന്നവരുടെ കൂട്ടത്തിലുള്ള നിന്‍റെ ചലനവും (കാണുന്നവന്‍)
\end{malayalam}}
\flushright{\begin{Arabic}
\quranayah[26][220]
\end{Arabic}}
\flushleft{\begin{malayalam}
തീര്‍ച്ചയായും അവന്‍ എല്ലാം കേള്‍ക്കുന്നവനും അറിയുന്നവനുമത്രെ
\end{malayalam}}
\flushright{\begin{Arabic}
\quranayah[26][221]
\end{Arabic}}
\flushleft{\begin{malayalam}
നബിയേ, പറയുക:) ആരുടെ മേലാണ് പിശാചുക്കള്‍ ഇറങ്ങുന്നതെന്ന് ഞാന്‍ നിങ്ങള്‍ക്ക് അറിയിച്ചു തരട്ടെയോ?
\end{malayalam}}
\flushright{\begin{Arabic}
\quranayah[26][222]
\end{Arabic}}
\flushleft{\begin{malayalam}
പെരും നുണയന്‍മാരും പാപികളുമായ എല്ലാവരുടെ മേലും അവര്‍ (പിശാചുക്കള്‍) ഇറങ്ങുന്നു
\end{malayalam}}
\flushright{\begin{Arabic}
\quranayah[26][223]
\end{Arabic}}
\flushleft{\begin{malayalam}
അവര്‍ ചെവികൊടുത്ത് കേള്‍ക്കുന്നു അവരില്‍ അധികപേരും കള്ളം പറയുന്നവരാകുന്നു
\end{malayalam}}
\flushright{\begin{Arabic}
\quranayah[26][224]
\end{Arabic}}
\flushleft{\begin{malayalam}
കവികളാകട്ടെ, ദുര്‍മാര്‍ഗികളാകുന്നു അവരെ പിന്‍പറ്റുന്നത്.‌
\end{malayalam}}
\flushright{\begin{Arabic}
\quranayah[26][225]
\end{Arabic}}
\flushleft{\begin{malayalam}
അവര്‍ എല്ലാ താഴ്‌വരകളിലും അലഞ്ഞു നടക്കുന്നവരാണെന്ന് നീ കണ്ടില്ലേ?
\end{malayalam}}
\flushright{\begin{Arabic}
\quranayah[26][226]
\end{Arabic}}
\flushleft{\begin{malayalam}
പ്രവര്‍ത്തിക്കാത്തത് പറയുന്നവരാണ് അവരെന്നും
\end{malayalam}}
\flushright{\begin{Arabic}
\quranayah[26][227]
\end{Arabic}}
\flushleft{\begin{malayalam}
വിശ്വസിക്കുകയും സല്‍കര്‍മ്മങ്ങള്‍ പ്രവര്‍ത്തിക്കുകയും, അല്ലാഹുവെ ധാരാളമായി സ്മരിക്കുകയും, അക്രമത്തിന് ഇരയായതിനെത്തുടര്‍ന്ന് ആത്മരക്ഷയ്ക്ക് നടപടി എടുക്കുകയും ചെയ്തവര്‍ ഇതില്‍നിന്ന് ഒഴിവാകുന്നു അക്രമകാരികള്‍ അറിഞ്ഞു കൊള്ളും; തങ്ങള്‍ തിരിഞ്ഞുമറിഞ്ഞ് എങ്ങനെയുള്ള പര്യവസാനത്തിലാണ് എത്തുകയെന്ന്.
\end{malayalam}}
\chapter{\textmalayalam{നംല്‍ ( ഉറുമ്പ് )}}
\begin{Arabic}
\Huge{\centerline{\basmalah}}\end{Arabic}
\flushright{\begin{Arabic}
\quranayah[27][1]
\end{Arabic}}
\flushleft{\begin{malayalam}
ത്വാ-സീന്‍. ഖുര്‍ആനിലെ, അഥവാ കാര്യങ്ങള്‍ സ്പഷ്ടമാക്കുന്ന ഗ്രന്ഥത്തിലെ വചനങ്ങളത്രെ അവ.
\end{malayalam}}
\flushright{\begin{Arabic}
\quranayah[27][2]
\end{Arabic}}
\flushleft{\begin{malayalam}
സത്യവിശ്വാസികള്‍ക്ക് മാര്‍ഗദര്‍ശനവും സന്തോഷവാര്‍ത്തയുമത്രെ അത്‌.
\end{malayalam}}
\flushright{\begin{Arabic}
\quranayah[27][3]
\end{Arabic}}
\flushleft{\begin{malayalam}
നമസ്കാരം മുറപോലെ നിവ്വഹിക്കുകയും, സകാത്ത് നല്‍കുകയും, പരലോകത്തില്‍ ദൃഢമായി വിശ്വസിക്കുകയും ചെയ്യുന്നവര്‍ക്ക്‌.
\end{malayalam}}
\flushright{\begin{Arabic}
\quranayah[27][4]
\end{Arabic}}
\flushleft{\begin{malayalam}
പരലോകത്തില്‍ വിശ്വസിക്കാത്തതാരോ അവര്‍ക്ക് തങ്ങളുടെ പ്രവര്‍ത്തനങ്ങള്‍ നാം ഭംഗിയായി തോന്നിച്ചിരിക്കുന്നു. അങ്ങനെ അവര്‍ വിഹരിച്ചുകൊണ്ടിരിക്കുന്നു.
\end{malayalam}}
\flushright{\begin{Arabic}
\quranayah[27][5]
\end{Arabic}}
\flushleft{\begin{malayalam}
അവരത്രെ കഠിനശിക്ഷയുള്ളവര്‍. പരലോകത്താകട്ടെ അവര്‍ തന്നെയായിരിക്കും ഏറ്റവും നഷ്ടം നേരിടുന്നവര്‍.
\end{malayalam}}
\flushright{\begin{Arabic}
\quranayah[27][6]
\end{Arabic}}
\flushleft{\begin{malayalam}
തീര്‍ച്ചയായും യുക്തിമാനും സര്‍വ്വജ്ഞനുമായിട്ടുള്ളവന്‍റെ പക്കല്‍ നിന്നാകുന്നു നിനക്ക് ഖുര്‍ആന്‍ നല്‍കപ്പെടുന്നത്‌.
\end{malayalam}}
\flushright{\begin{Arabic}
\quranayah[27][7]
\end{Arabic}}
\flushleft{\begin{malayalam}
മൂസാ തന്‍റെ കുടുംബത്തോട് പറഞ്ഞ സന്ദര്‍ഭം: തീര്‍ച്ചയായും ഞാന്‍ ഒരു തീ കണ്ടിരിക്കുന്നു. അതിന്‍റെ അടുത്ത് നിന്ന് ഞാന്‍ നിങ്ങള്‍ക്ക് വല്ല വിവരവും കൊണ്ട് വരാം. അല്ലെങ്കില്‍ അതില്‍ നിന്ന് ഒരു തീ നാളം കൊളുത്തി എടുത്ത് ഞാന്‍ നിങ്ങള്‍ക്ക് കൊണ്ട് വരാം. നിങ്ങള്‍ക്ക് തീ കായാമല്ലോ.
\end{malayalam}}
\flushright{\begin{Arabic}
\quranayah[27][8]
\end{Arabic}}
\flushleft{\begin{malayalam}
അങ്ങനെ അദ്ദേഹം അതിനടുത്ത് ചെന്നപ്പോള്‍ ഇപ്രകാരം വിളിച്ചുപറയപ്പെട്ടു; തീയിലുള്ളവരും അതിനു ചുറ്റുമുള്ളവരും അനുഗ്രഹിക്കപ്പെട്ടിരിക്കുന്നു. ലോകരക്ഷിതാവായ അല്ലാഹു എത്രയോ പരിശുദ്ധനാകുന്നു.
\end{malayalam}}
\flushright{\begin{Arabic}
\quranayah[27][9]
\end{Arabic}}
\flushleft{\begin{malayalam}
ഹേ; മൂസാ, തീര്‍ച്ചയായും പ്രതാപിയും യുക്തിമാനുമായ അല്ലാഹുവാണ് ഞാന്‍.
\end{malayalam}}
\flushright{\begin{Arabic}
\quranayah[27][10]
\end{Arabic}}
\flushleft{\begin{malayalam}
നീ നിന്‍റെ വടി താഴെയിടൂ. അങ്ങനെ അത് ഒരു സര്‍പ്പമെന്നോണം ചലിക്കുന്നത് കണ്ടപ്പോള്‍ അദ്ദേഹം പിന്തിരിഞ്ഞോടി. അദ്ദേഹം തിരിഞ്ഞ് നോക്കിയില്ല. അല്ലാഹു പറഞ്ഞു: ഹേ; മൂസാ, നീ ഭയപ്പെടരുത്‌. ദൂതന്‍മാര്‍ എന്‍റെ അടുക്കല്‍ പേടിക്കേണ്ടതില്ല; തീര്‍ച്ച.
\end{malayalam}}
\flushright{\begin{Arabic}
\quranayah[27][11]
\end{Arabic}}
\flushleft{\begin{malayalam}
പക്ഷെ, വല്ലവനും അക്രമം പ്രവര്‍ത്തിക്കുകയും, പിന്നീട് തിന്‍മയ്ക്ക് ശേഷം നന്‍മയെ പകരം കൊണ്ട് വരികയും ചെയ്താല്‍ തീര്‍ച്ചയായും ഞാന്‍ ഏറെ പൊറുക്കുന്നവനും കരുണാനിധിയുമാകുന്നു.
\end{malayalam}}
\flushright{\begin{Arabic}
\quranayah[27][12]
\end{Arabic}}
\flushleft{\begin{malayalam}
നീ നിന്‍റെ കൈ കുപ്പായമാറിലേക്ക് പ്രവേശിപ്പിക്കുക. യാതൊരു കളങ്കവും കൂടാതെ വെളുപ്പുനിറമുള്ളതായിക്കൊണ്ട് അത് പുറത്ത് വരും. ഫിര്‍ഔന്‍റെയും അവന്‍റെ ജനതയുടെയും അടുത്തേക്കുള്ള ഒമ്പത് ദൃഷ്ടാന്തങ്ങളില്‍ പെട്ടതത്രെ ഇവ. തീര്‍ച്ചയായും അവര്‍ ധിക്കാരികളായ ഒരു ജനതയായിരിക്കുന്നു.
\end{malayalam}}
\flushright{\begin{Arabic}
\quranayah[27][13]
\end{Arabic}}
\flushleft{\begin{malayalam}
അങ്ങനെ കണ്ണുതുറപ്പിക്കത്തക്ക നിലയില്‍ നമ്മുടെ ദൃഷ്ടാന്തങ്ങള്‍ അവര്‍ക്ക് വന്നെത്തിയപ്പോള്‍ അവര്‍ പറഞ്ഞു: ഇതു സ്പഷ്ടമായ ജാലവിദ്യതന്നെയാകുന്നു.
\end{malayalam}}
\flushright{\begin{Arabic}
\quranayah[27][14]
\end{Arabic}}
\flushleft{\begin{malayalam}
അവയെപ്പറ്റി അവരുടെ മനസ്സുകള്‍ക്ക് ദൃഢമായ ബോധ്യം വന്നിട്ടും അക്രമവും അഹങ്കാരവും മൂലം അവരതിനെ നിഷേധിച്ചുകളഞ്ഞു. അപ്പോള്‍ ആ കുഴപ്പക്കാരുടെ പര്യവസാനം എങ്ങനെയായിരുന്നു എന്ന് നോക്കുക.
\end{malayalam}}
\flushright{\begin{Arabic}
\quranayah[27][15]
\end{Arabic}}
\flushleft{\begin{malayalam}
ദാവൂദിനും സുലൈമാന്നും നാം വിജ്ഞാനം നല്‍കുകയുണ്ടായി. തന്‍റെ വിശ്വാസികളായ ദാസന്‍മാരില്‍ മിക്കവരെക്കാളും ഞങ്ങള്‍ക്ക് ശ്രേഷ്ഠത നല്‍കിയ അല്ലാഹുവിന് സ്തുതി എന്ന് അവര്‍ ഇരുവരും പറയുകയും ചെയ്തു.
\end{malayalam}}
\flushright{\begin{Arabic}
\quranayah[27][16]
\end{Arabic}}
\flushleft{\begin{malayalam}
സുലൈമാന്‍ ദാവൂദിന്‍റെ അനന്തരാവകാശിയായി. അദ്ദേഹം പറഞ്ഞു: ജനങ്ങളേ, പക്ഷികളുടെ ഭാഷ നമുക്ക് പഠിപ്പിക്കപ്പെട്ടിരിക്കുന്നു. എല്ലാ കാര്യങ്ങളില്‍ നിന്നും നമുക്ക് നല്‍കപ്പെടുകയും ചെയ്തിരിക്കുന്നു. തീര്‍ച്ചയായും ഇത് തന്നെയാകുന്നു പ്രത്യക്ഷമായ അനുഗ്രഹം.
\end{malayalam}}
\flushright{\begin{Arabic}
\quranayah[27][17]
\end{Arabic}}
\flushleft{\begin{malayalam}
സുലൈമാന്ന് വേണ്ടി ജിന്നിലും മനുഷ്യരിലും പക്ഷികളിലും പെട്ട തന്‍റെ സൈന്യങ്ങള്‍ ശേഖരിക്കപ്പെട്ടു. അങ്ങനെ അവരതാ ക്രമപ്രകാരം നിര്‍ത്തപ്പെടുന്നു.
\end{malayalam}}
\flushright{\begin{Arabic}
\quranayah[27][18]
\end{Arabic}}
\flushleft{\begin{malayalam}
അങ്ങനെ അവര്‍ ഉറുമ്പിന്‍ താഴ്‌വരയിലൂടെ ചെന്നപ്പോള്‍ ഒരു ഉറുമ്പ് പറഞ്ഞു: ഹേ, ഉറുമ്പുകളേ, നിങ്ങള്‍ നിങ്ങളുടെ പാര്‍പ്പിടങ്ങളില്‍ പ്രവേശിച്ചു കൊള്ളുക. സുലൈമാനും അദ്ദേഹത്തിന്‍റെ സൈന്യങ്ങളും അവര്‍ ഓര്‍ക്കാത്ത വിധത്തില്‍ നിങ്ങളെ ചവിട്ടിതേച്ചു കളയാതിരിക്കട്ടെ.
\end{malayalam}}
\flushright{\begin{Arabic}
\quranayah[27][19]
\end{Arabic}}
\flushleft{\begin{malayalam}
അപ്പോള്‍ അതിന്‍റെ വാക്ക് കേട്ട് അദ്ദേഹം നന്നായൊന്ന് പുഞ്ചിരിച്ചു. എന്‍റെ രക്ഷിതാവേ, എനിക്കും എന്‍റെ മാതാപിതാക്കള്‍ക്കും നീ ചെയ്ത് തന്നിട്ടുള്ള നിന്‍റെ അനുഗ്രഹത്തിന് നന്ദികാണിക്കുവാനും നീ തൃപ്തിപ്പെടുന്ന സല്‍കര്‍മ്മം ചെയ്യുവാനും എനിക്ക് നീ പ്രചോദനം നല്‍കേണമേ. നിന്‍റെ കാരുണ്യത്താല്‍ നിന്‍റെ സദ്‌വൃത്തരായ ദാസന്‍മാരുടെ കൂട്ടത്തില്‍ എന്നെ നീ ഉള്‍പെടുത്തുകയും ചെയ്യേണമേ.
\end{malayalam}}
\flushright{\begin{Arabic}
\quranayah[27][20]
\end{Arabic}}
\flushleft{\begin{malayalam}
അദ്ദേഹം പക്ഷികളെ പരിശോധിക്കുകയുണ്ടായി. എന്നിട്ട് അദ്ദേഹം പറഞ്ഞു; എന്തുപറ്റി? മരംകൊത്തിയെ ഞാന്‍ കാണുന്നില്ലല്ലോ, അഥവാ അത് സ്ഥലം വിട്ടു പോയ കൂട്ടത്തിലാണോ?
\end{malayalam}}
\flushright{\begin{Arabic}
\quranayah[27][21]
\end{Arabic}}
\flushleft{\begin{malayalam}
ഞാനതിന് കഠിനശിക്ഷ നല്‍കുകയോ, അല്ലെങ്കില്‍ അതിനെ അറുക്കുകയോ തന്നെ ചെയ്യും. അല്ലെങ്കില്‍ വ്യക്തമായ വല്ല ന്യായവും അതെനിക്ക് ബോധിപ്പിച്ചു തരണം.
\end{malayalam}}
\flushright{\begin{Arabic}
\quranayah[27][22]
\end{Arabic}}
\flushleft{\begin{malayalam}
എന്നാല്‍ അത് എത്തിച്ചേരാന്‍ അധികം താമസിച്ചില്ല. എന്നിട്ട് അത് പറഞ്ഞു: താങ്കള്‍ സൂക്ഷ്മമായി അറിഞ്ഞിട്ടില്ലാത്ത ഒരു കാര്യം ഞാന്‍ സൂക്ഷ്മമായി മനസ്സിലാക്കിയിട്ടുണ്ട്‌. സബഇല്‍ നിന്ന് യഥാര്‍ത്ഥമായ ഒരു വാര്‍ത്തയും കൊണ്ടാണ് ഞാന്‍ വന്നിരിക്കുന്നത്‌.
\end{malayalam}}
\flushright{\begin{Arabic}
\quranayah[27][23]
\end{Arabic}}
\flushleft{\begin{malayalam}
ഒരു സ്ത്രീ അവരെ ഭരിക്കുന്നതായി ഞാന്‍ കണ്ടെത്തുകയുണ്ടായി. എല്ലാകാര്യങ്ങളില്‍ നിന്നും അവള്‍ക്ക് നല്‍കപ്പെട്ടിട്ടുണ്ട്‌. അവള്‍ക്ക് ഗംഭീരമായ ഒരു സിംഹാസനവുമുണ്ട്‌.
\end{malayalam}}
\flushright{\begin{Arabic}
\quranayah[27][24]
\end{Arabic}}
\flushleft{\begin{malayalam}
അവളും അവളുടെ ജനതയും അല്ലാഹുവിന് പുറമെ സൂര്യന് പ്രണാമം ചെയ്യുന്നതായിട്ടാണ് ഞാന്‍ കണ്ടെത്തിയത്‌. പിശാച് അവര്‍ക്ക് തങ്ങളുടെ പ്രവര്‍ത്തനങ്ങള്‍ ഭംഗിയായി തോന്നിക്കുകയും, അവരെ നേര്‍മാര്‍ഗത്തില്‍ നിന്ന് തടയുകയും ചെയ്തിരിക്കുന്നു. അതിനാല്‍ അവര്‍ നേര്‍വഴി പ്രാപിക്കുന്നില്ല.
\end{malayalam}}
\flushright{\begin{Arabic}
\quranayah[27][25]
\end{Arabic}}
\flushleft{\begin{malayalam}
ആകാശങ്ങളിലും ഭൂമിയിലും ഒളിഞ്ഞു കിടക്കുന്നത് പുറത്ത് കൊണ്ട് വരികയും, നിങ്ങള്‍ രഹസ്യമാക്കുന്നതും പരസ്യമാക്കുന്നതും അറിയുകയും ചെയ്യുന്നവനായ അല്ലാഹുവിന് അവര്‍ പ്രണാമം ചെയ്യാതിരിക്കുവാന്‍ വേണ്ടി (പിശാച് അങ്ങനെ ചെയ്യുന്നു.)
\end{malayalam}}
\flushright{\begin{Arabic}
\quranayah[27][26]
\end{Arabic}}
\flushleft{\begin{malayalam}
മഹത്തായ സിംഹാസനത്തിന്‍റെ നാഥനായ അല്ലാഹു അല്ലാതെ യാതൊരു ദൈവവുമില്ല.
\end{malayalam}}
\flushright{\begin{Arabic}
\quranayah[27][27]
\end{Arabic}}
\flushleft{\begin{malayalam}
സുലൈമാന്‍ പറഞ്ഞു: നീ സത്യം പറഞ്ഞതാണോ അതല്ല നീ കള്ളം പറയുന്നവരുടെ കൂട്ടത്തിലായിരിക്കുന്നുവോ എന്ന് നാം നോക്കാം.
\end{malayalam}}
\flushright{\begin{Arabic}
\quranayah[27][28]
\end{Arabic}}
\flushleft{\begin{malayalam}
നീ എന്‍റെ ഈ എഴുത്ത് കൊണ്ടു പോയി അവര്‍ക്ക് ഇട്ടുകൊടുക്കുക. പിന്നീട് നീ അവരില്‍ നിന്ന് മാറി നിന്ന് അവര്‍ എന്ത് മറുപടി നല്‍കുന്നു എന്ന് നോക്കുക.
\end{malayalam}}
\flushright{\begin{Arabic}
\quranayah[27][29]
\end{Arabic}}
\flushleft{\begin{malayalam}
അവള്‍ പറഞ്ഞു: ഹേ; പ്രമുഖന്‍മാരേ, എനിക്കിതാ മാന്യമായ ഒരു എഴുത്ത് നല്‍കപ്പെട്ടിരിക്കുന്നു.
\end{malayalam}}
\flushright{\begin{Arabic}
\quranayah[27][30]
\end{Arabic}}
\flushleft{\begin{malayalam}
അത് സുലൈമാന്‍റെ പക്കല്‍ നിന്നുള്ളതാണ്‌. ആ കത്ത് ഇപ്രകാരമത്രെ: പരമകാരുണികനും കരുണാനിധിയുമായ അല്ലാഹുവിന്‍റെ നാമത്തില്‍.
\end{malayalam}}
\flushright{\begin{Arabic}
\quranayah[27][31]
\end{Arabic}}
\flushleft{\begin{malayalam}
എനിക്കെതിരില്‍ നിങ്ങള്‍ അഹങ്കാരം കാണിക്കാതിരിക്കുകയും, കീഴൊതുങ്ങിയവരായിക്കൊണ്ട് നിങ്ങള്‍ എന്‍റെ അടുത്ത് വരികയും ചെയ്യുക.
\end{malayalam}}
\flushright{\begin{Arabic}
\quranayah[27][32]
\end{Arabic}}
\flushleft{\begin{malayalam}
അവള്‍ പറഞ്ഞു: ഹേ; പ്രമുഖന്‍മാരേ, എന്‍റെ ഈ കാര്യത്തില്‍ നിങ്ങള്‍ എനിക്ക് നിര്‍ദേശം നല്‍കുക. നിങ്ങള്‍ എന്‍റെ അടുക്കല്‍ സന്നിഹിതരായിട്ടല്ലാതെ യാതൊരു കാര്യവും ഖണ്ഡിതമായി തീരുമാനിക്കുന്നവളല്ല ഞാന്‍.
\end{malayalam}}
\flushright{\begin{Arabic}
\quranayah[27][33]
\end{Arabic}}
\flushleft{\begin{malayalam}
അവര്‍ പറഞ്ഞു: നാം ശക്തിയുള്ളവരും ഉഗ്രമായ സമരവീര്യമുള്ളവരുമാണ്‌. അധികാരം അങ്ങേക്കാണല്ലോ, അതിനാല്‍ എന്താണ് കല്‍പിച്ചരുളേണ്ടതെന്ന് ആലോചിച്ചു നോക്കുക.
\end{malayalam}}
\flushright{\begin{Arabic}
\quranayah[27][34]
\end{Arabic}}
\flushleft{\begin{malayalam}
അവള്‍ പറഞ്ഞു: തീര്‍ച്ചയായും രാജാക്കന്‍മാര്‍ ഒരു നാട്ടില്‍ കടന്നാല്‍ അവര്‍ അവിടെ നാശമുണ്ടാക്കുകയും, അവിടത്തുകാരിലെ പ്രതാപികളെ നിന്ദ്യന്‍മാരാക്കുകയും ചെയ്യുന്നതാണ്‌. അപ്രകാരമാകുന്നു അവര്‍ ചെയ്തു കൊണ്ടിരിക്കുന്നത്‌.
\end{malayalam}}
\flushright{\begin{Arabic}
\quranayah[27][35]
\end{Arabic}}
\flushleft{\begin{malayalam}
ഞാന്‍ അവര്‍ക്ക് ഒരു പാരിതോഷികം കൊടുത്തയച്ചിട്ട് എന്തൊരു വിവരവും കൊണ്ടാണ് ദൂതന്‍മാര്‍ മടങ്ങിവരുന്നതെന്ന് നോക്കാന്‍ പോകുകയാണ്‌.
\end{malayalam}}
\flushright{\begin{Arabic}
\quranayah[27][36]
\end{Arabic}}
\flushleft{\begin{malayalam}
അങ്ങനെ അവന്‍ (ദൂതന്‍) സുലൈമാന്‍റെ അടുത്ത് ചെന്നപ്പോള്‍ അദ്ദേഹം പറഞ്ഞു: നിങ്ങള്‍ സമ്പത്ത് തന്ന് എന്നെ സഹായിക്കുകയാണോ? എന്നാല്‍ എനിക്ക് അല്ലാഹു നല്‍കിയിട്ടുള്ളതാണ് നിങ്ങള്‍ക്കവന്‍ നല്‍കിയിട്ടുള്ളതിനെക്കാള്‍ ഉത്തമം. പക്ഷെ, നിങ്ങള്‍ നിങ്ങളുടെ പാരിതോഷികം കൊണ്ട് സന്തോഷം കൊള്ളുകയാകുന്നു.
\end{malayalam}}
\flushright{\begin{Arabic}
\quranayah[27][37]
\end{Arabic}}
\flushleft{\begin{malayalam}
നീ അവരുടെ അടുത്തേക്ക് തന്നെ മടങ്ങിച്ചെല്ലുക. തീര്‍ച്ചയായും അവര്‍ക്ക് നേരിടുവാന്‍ കഴിയാത്ത സൈന്യങ്ങളെയും കൊണ്ട് നാം അവരുടെ അടുത്ത് ചെല്ലുകയും, നിന്ദ്യരും അപമാനിതരുമായ നിലയില്‍ അവരെ നാം അവിടെ നിന്ന് പുറത്താക്കുകയും ചെയ്യുന്നതാണ്‌.
\end{malayalam}}
\flushright{\begin{Arabic}
\quranayah[27][38]
\end{Arabic}}
\flushleft{\begin{malayalam}
അദ്ദേഹം (സുലൈമാന്‍) പറഞ്ഞു: ഹേ; പ്രമുഖന്‍മാരേ, അവര്‍ കീഴൊതുങ്ങിക്കൊണ്ട് എന്‍റെ അടുത്ത് വരുന്നതിന് മുമ്പായി നിങ്ങളില്‍ ആരാണ് അവളുടെ സിംഹാസനം എനിക്ക് കൊണ്ടു വന്ന് തരിക.?
\end{malayalam}}
\flushright{\begin{Arabic}
\quranayah[27][39]
\end{Arabic}}
\flushleft{\begin{malayalam}
ജിന്നുകളുടെ കൂട്ടത്തിലുള്ള ഒരു മല്ലന്‍ പറഞ്ഞു: അങ്ങ് അങ്ങയുടെ ഈ സദസ്സില്‍ നിന്ന് എഴുന്നേല്‍ക്കുന്നതിനുമുമ്പായി ഞാനത് അങ്ങേക്ക് കൊണ്ടുവന്നുതരാം. തീര്‍ച്ചയായും ഞാനതിന് കഴിവുള്ളവനും വിശ്വസ്തനുമാകുന്നു.
\end{malayalam}}
\flushright{\begin{Arabic}
\quranayah[27][40]
\end{Arabic}}
\flushleft{\begin{malayalam}
വേദത്തില്‍ നിന്നുള്ള വിജ്ഞാനം കരഗതമാക്കിയിട്ടുള്ള ആള്‍ പറഞ്ഞു; താങ്കളുടെ ദൃഷ്ടി താങ്കളിലേക്ക് തിരിച്ചുവരുന്നതിന് മുമ്പായി ഞാനത് താങ്കള്‍ക്ക് കൊണ്ടു വന്ന് തരാം. അങ്ങനെ അത് (സിംഹാസനം) തന്‍റെ അടുക്കല്‍ സ്ഥിതി ചെയ്യുന്നതായി കണ്ടപ്പോള്‍ അദ്ദേഹം പറഞ്ഞു: ഞാന്‍ നന്ദികാണിക്കുമോ, അതല്ല നന്ദികേട് കാണിക്കുമോ എന്ന് എന്നെ പരീക്ഷിക്കുവാനായി എന്‍റെ രക്ഷിതാവ് എനിക്ക് നല്‍കിയ അനുഗ്രഹത്തില്‍പെട്ടതാകുന്നു ഇത്‌. വല്ലവനും നന്ദികാണിക്കുന്ന പക്ഷം തന്‍റെ ഗുണത്തിനായിട്ട് തന്നെയാകുന്നു അവന്‍ നന്ദികാണിക്കുന്നത്‌. വല്ലവനും നന്ദികേട് കാണിക്കുന്ന പക്ഷം തീര്‍ച്ചയായും എന്‍റെ രക്ഷിതാവ് പരാശ്രയമുക്തനും, ഉല്‍കൃഷ്ടനുമാകുന്നു.
\end{malayalam}}
\flushright{\begin{Arabic}
\quranayah[27][41]
\end{Arabic}}
\flushleft{\begin{malayalam}
അദ്ദേഹം (സുലൈമാന്‍) പറഞ്ഞു: നിങ്ങള്‍ അവളുടെ സിംഹാസനം അവള്‍ക്ക് തിരിച്ചറിയാത്ത വിധത്തില്‍ മാറ്റുക. അവള്‍ യാഥാര്‍ത്ഥ്യം മനസ്സിലാക്കുമോ, അതല്ല അവള്‍ യാഥാര്‍ത്ഥ്യം കണ്ടെത്താത്തവരുടെ കൂട്ടത്തിലായിരിക്കുമോ എന്ന് നമുക്ക് നോക്കാം.
\end{malayalam}}
\flushright{\begin{Arabic}
\quranayah[27][42]
\end{Arabic}}
\flushleft{\begin{malayalam}
അങ്ങനെ അവള്‍ വന്നപ്പോള്‍ (അവളോട്‌) ചോദിക്കപ്പെട്ടു: താങ്കളുടെ സിംഹാസനം ഇതു പോലെയാണോ? അവള്‍ പറഞ്ഞു: ഇത് അത് തന്നെയാണെന്ന് തോന്നുന്നു. ഇതിനു മുമ്പ് തന്നെ ഞങ്ങള്‍ക്ക് അറിവ് നല്‍കപ്പെട്ടിരുന്നു. ഞങ്ങള്‍ മുസ്ലിംകളാവുകയും ചെയ്തിരുന്നു.
\end{malayalam}}
\flushright{\begin{Arabic}
\quranayah[27][43]
\end{Arabic}}
\flushleft{\begin{malayalam}
അല്ലാഹുവിന് പുറമെ അവള്‍ ആരാധിച്ചിരുന്നതില്‍നിന്ന് അദ്ദേഹം അവളെ തടയുകയും ചെയ്തു. തീര്‍ച്ചയായും അവള്‍ സത്യനിഷേധികളായ ജനതയില്‍ പെട്ടവളായിരുന്നു.
\end{malayalam}}
\flushright{\begin{Arabic}
\quranayah[27][44]
\end{Arabic}}
\flushleft{\begin{malayalam}
കൊട്ടാരത്തില്‍ പ്രവേശിച്ചു കൊള്ളുക എന്ന് അവളോട് പറയപ്പെട്ടു. എന്നാല്‍ അവളതു കണ്ടപ്പോള്‍ അതൊരു ജലാശയമാണെന്ന് വിചാരിക്കുകയും, തന്‍റെ കണങ്കാലുകളില്‍ നിന്ന് വസ്ത്രം മേലോട്ട് നീക്കുകയും ചെയ്തു. സുലൈമാന്‍ പറഞ്ഞു: ഇത് സ്ഫടികകഷ്ണങ്ങള്‍ പാകിമിനുക്കിയ ഒരു കൊട്ടാരമാകുന്നു. അവള്‍ പറഞ്ഞു: എന്‍റെ രക്ഷിതാവേ, ഞാന്‍ എന്നോട് തന്നെ അന്യായം ചെയ്തിരിക്കുന്നു. ഞാനിതാ സുലൈമാനോടൊപ്പം ലോകരക്ഷിതാവായ അല്ലാഹുവിന് കീഴ്പെട്ടിരിക്കുന്നു.
\end{malayalam}}
\flushright{\begin{Arabic}
\quranayah[27][45]
\end{Arabic}}
\flushleft{\begin{malayalam}
നിങ്ങള്‍ അല്ലാഹുവെ (മാത്രം) ആരാധിക്കുക എന്ന ദൌത്യവുമായി ഥമൂദ് സമുദായത്തിലേക്ക് അവരുടെ സഹോദരനായ സ്വാലിഹിനെയും നാം അയക്കുകയുണ്ടായി. അപ്പോഴതാ അവര്‍ അന്യോന്യം വഴക്കടിക്കുന്ന രണ്ട് കക്ഷികളായിത്തീരുന്നു.
\end{malayalam}}
\flushright{\begin{Arabic}
\quranayah[27][46]
\end{Arabic}}
\flushleft{\begin{malayalam}
അദ്ദേഹം പറഞ്ഞു: എന്‍റെ ജനങ്ങളേ, നിങ്ങള്‍ എന്തിനാണ് നന്‍മയെക്കാള്‍ മുമ്പായി തിന്‍മയ്ക്ക് തിടുക്കം കൂട്ടുന്നത്‌.? നിങ്ങള്‍ക്ക് അല്ലാഹുവോട് പാപമോചനം തേടിക്കൂടേ? എങ്കില്‍ നിങ്ങള്‍ക്കു കാരുണ്യം നല്‍കപ്പെട്ടേക്കാം.
\end{malayalam}}
\flushright{\begin{Arabic}
\quranayah[27][47]
\end{Arabic}}
\flushleft{\begin{malayalam}
അവര്‍ പറഞ്ഞു: നീ മൂലവും, നിന്‍റെ കൂടെയുള്ളവര്‍ മൂലവും ഞങ്ങള്‍ ശകുനപ്പിഴയിലായിരിക്കുന്നു. അദ്ദേഹം പറഞ്ഞു: നിങ്ങളുടെ ശകുനം അല്ലാഹുവിങ്കല്‍ രേഖപ്പെട്ടതത്രെ. അല്ല, നിങ്ങള്‍ (ദൈവികമായ) പരീക്ഷണത്തിന് വിധേയരാകുന്ന ഒരു ജനതയാകുന്നു.
\end{malayalam}}
\flushright{\begin{Arabic}
\quranayah[27][48]
\end{Arabic}}
\flushleft{\begin{malayalam}
ഭൂമിയില്‍ കുഴപ്പമുണ്ടാക്കുന്നവരും, ഒരു നന്‍മയുമുണ്ടാക്കാത്തവരുമായ ഒമ്പതു പേര്‍ ആ പട്ടണത്തിലുണ്ടായിരുന്നു.
\end{malayalam}}
\flushright{\begin{Arabic}
\quranayah[27][49]
\end{Arabic}}
\flushleft{\begin{malayalam}
അവനെ (സ്വാലിഹിനെ) യും അവന്‍റെ ആളുകളെയും നമുക്ക് രാത്രിയില്‍ കൊന്നുകളയണമെന്നും പിന്നീട് അവന്‍റെ അവകാശിയോട്‌, തന്‍റെ ആളുകളുടെ നാശത്തിന് ഞങ്ങള്‍ സാക്ഷ്യം വഹിച്ചിട്ടില്ല, തീര്‍ച്ചയായും ഞങ്ങള്‍ സത്യം പറയുന്നവരാകുന്നു എന്ന് നാം പറയണമെന്നും നിങ്ങള്‍ അല്ലാഹുവിന്‍റെ പേരില്‍ സത്യം ചെയ്യണം എന്ന് അവര്‍ തമ്മില്‍ പറഞ്ഞുറച്ചു.
\end{malayalam}}
\flushright{\begin{Arabic}
\quranayah[27][50]
\end{Arabic}}
\flushleft{\begin{malayalam}
അവര്‍ ഒരു തന്ത്രം പ്രയോഗിച്ചു. അവര്‍ ഓര്‍ക്കാതിരിക്കെ നാമും ഒരു തന്ത്രം പ്രയോഗിച്ചു.
\end{malayalam}}
\flushright{\begin{Arabic}
\quranayah[27][51]
\end{Arabic}}
\flushleft{\begin{malayalam}
എന്നിട്ട് അവരുടെ തന്ത്രത്തിന്‍റെ പര്യവസാനം എങ്ങനെയായിരുന്നു എന്ന് നോക്കുക. അതെ, അവരെയും അവരുടെ ജനങ്ങളെയും മുഴുവന്‍ നാം തകര്‍ത്തു കളഞ്ഞു.
\end{malayalam}}
\flushright{\begin{Arabic}
\quranayah[27][52]
\end{Arabic}}
\flushleft{\begin{malayalam}
അങ്ങനെ അവര്‍ അക്രമം പ്രവര്‍ത്തിച്ചതിന്‍റെ ഫലമായി അവരുടെ വീടുകളതാ (ശൂന്യമായി) വീണടിഞ്ഞ് കിടക്കുന്നു. തീര്‍ച്ചയായും മനസ്സിലാക്കുന്ന ജനങ്ങള്‍ക്കു അതില്‍ ദൃഷ്ടാന്തമുണ്ട്‌.
\end{malayalam}}
\flushright{\begin{Arabic}
\quranayah[27][53]
\end{Arabic}}
\flushleft{\begin{malayalam}
വിശ്വസിക്കുകയും, ധര്‍മ്മനിഷ്ഠ പാലിച്ചു വരികയും ചെയ്തവരെ നാം രക്ഷപ്പെടുത്തുകയും ചെയ്തു.
\end{malayalam}}
\flushright{\begin{Arabic}
\quranayah[27][54]
\end{Arabic}}
\flushleft{\begin{malayalam}
ലൂത്വിനെയും (ഓര്‍ക്കുക.) അദ്ദേഹം തന്‍റെ ജനതയോട് പറഞ്ഞ സന്ദര്‍ഭം: നിങ്ങള്‍ കണ്ടറിഞ്ഞു കൊണ്ട് നീചവൃത്തി ചെയ്യുകയാണോ?
\end{malayalam}}
\flushright{\begin{Arabic}
\quranayah[27][55]
\end{Arabic}}
\flushleft{\begin{malayalam}
നിങ്ങള്‍ കാമനിവൃത്തിക്കായി സ്ത്രീകളെ വിട്ട് പുരുഷന്‍മാരുടെ അടുക്കല്‍ ചെല്ലുകയാണോ? അല്ല. നിങ്ങള്‍ അവിവേകം കാണിക്കുന്ന ഒരു ജനതയാകുന്നു.
\end{malayalam}}
\flushright{\begin{Arabic}
\quranayah[27][56]
\end{Arabic}}
\flushleft{\begin{malayalam}
ലൂത്വിന്‍റെ അനുയായികളെ നിങ്ങളുടെ രാജ്യത്ത് നിന്നും പുറത്താക്കുക, അവര്‍ ശുദ്ധിപാലിക്കുന്ന കുറെ ആളുകളാകുന്നു എന്നു പറഞ്ഞത് മാത്രമായിരുന്നു അദ്ദേഹത്തിന്‍റെ ജനതയുടെ മറുപടി.
\end{malayalam}}
\flushright{\begin{Arabic}
\quranayah[27][57]
\end{Arabic}}
\flushleft{\begin{malayalam}
അപ്പോള്‍ അദ്ദേഹത്തെയും അദ്ദേഹത്തിന്‍റെ ആളുകളെയും നാം രക്ഷപ്പെടുത്തി; അദ്ദേഹത്തിന്‍റെ ഭാര്യ ഒഴികെ. പിന്‍മാറി നിന്നവരുടെ കൂട്ടത്തിലാണ് നാം അവളെ കണക്കാക്കിയത്‌.
\end{malayalam}}
\flushright{\begin{Arabic}
\quranayah[27][58]
\end{Arabic}}
\flushleft{\begin{malayalam}
അവരുടെ മേല്‍ നാം ഒരു മഴ വര്‍ഷിക്കുകയും ചെയ്തു. താക്കീത് നല്‍കപ്പെട്ടവര്‍ക്ക് ലഭിച്ച ആ മഴ എത്രമോശം!
\end{malayalam}}
\flushright{\begin{Arabic}
\quranayah[27][59]
\end{Arabic}}
\flushleft{\begin{malayalam}
(നബിയേ,) പറയുക: അല്ലാഹുവിന് സ്തുതി. അവന്‍ തെരഞ്ഞെടുത്ത അവന്‍റെ ദാസന്‍മാര്‍ക്ക് സമാധാനം. അല്ലാഹുവാണോ ഉത്തമന്‍; അതല്ല, (അവനോട്‌) അവര്‍ പങ്കുചേര്‍ക്കുന്നവയോ?
\end{malayalam}}
\flushright{\begin{Arabic}
\quranayah[27][60]
\end{Arabic}}
\flushleft{\begin{malayalam}
അഥവാ, ആകാശങ്ങളും ഭൂമിയും സൃഷ്ടിക്കുകയും നിങ്ങള്‍ക്ക് ആകാശത്ത് നിന്ന് വെള്ളം ചൊരിഞ്ഞുതരികയും ചെയ്തവനോ? (അതോ അവരുടെ ദൈവങ്ങളോ!) എന്നിട്ട് അത് (വെള്ളം) മൂലം കൌതുകമുള്ള ചില തോട്ടങ്ങള്‍ നാം മുളപ്പിച്ചുണ്ടാക്കിത്തരികയും ചെയ്തു. അവയിലെ വൃക്ഷങ്ങള്‍ മുളപ്പിക്കുവാന്‍ നിങ്ങള്‍ക്ക് കഴിയുമായിരുന്നില്ല. അല്ലാഹുവോടൊപ്പം വേറെ വല്ല ദൈവവുമുണ്ടോ? അല്ല, അവര്‍ വ്യതിചലിച്ചു പോകുന്ന ഒരു ജനതയാകുന്നു.
\end{malayalam}}
\flushright{\begin{Arabic}
\quranayah[27][61]
\end{Arabic}}
\flushleft{\begin{malayalam}
അഥവാ, ഭൂമിയെ നിവാസയോഗ്യമാക്കുകയും, അതിനിടയില്‍ നദികളുണ്ടാക്കുകയും, അതിന് ഉറപ്പ് നല്‍കുന്ന പര്‍വ്വതങ്ങള്‍ ഉണ്ടാക്കുകയും, രണ്ടുതരം ജലാശയങ്ങള്‍ക്കിടയില്‍ ഒരു തടസ്സം ഉണ്ടാക്കുകയും ചെയ്തവനോ? (അതോ അവരുടെ ദൈവങ്ങളോ?) അല്ലാഹുവോടൊപ്പം മറ്റു വല്ല ദൈവവുമുണ്ടോ? അല്ല, അവരില്‍ അധികപേരും അറിയുന്നില്ല.
\end{malayalam}}
\flushright{\begin{Arabic}
\quranayah[27][62]
\end{Arabic}}
\flushleft{\begin{malayalam}
അഥവാ, കഷ്ടപ്പെട്ടവന്‍ വിളിച്ചു പ്രാര്‍ത്ഥിച്ചാല്‍ അവന്നു ഉത്തരം നല്‍കുകയും വിഷമം നീക്കികൊടുക്കുകയും, നിങ്ങളെ ഭൂമിയില്‍ പ്രതിനിധികളാക്കുകയും ചെയ്യുന്നവനോ (അതല്ല, അവരുടെ ദൈവങ്ങളോ?) അല്ലാഹുവോടൊപ്പം വേറെ വല്ല ദൈവവുമുണ്ടോ? കുറച്ച് മാത്രമേ നിങ്ങള്‍ ആലോചിച്ച് മനസ്സിലാക്കുന്നുള്ളൂ.
\end{malayalam}}
\flushright{\begin{Arabic}
\quranayah[27][63]
\end{Arabic}}
\flushleft{\begin{malayalam}
അഥവാ കരയിലെയും കടലിലെയും അന്ധകാരങ്ങളില്‍ നിങ്ങള്‍ക്ക് വഴി കാണിക്കുകയും, തന്‍റെ കാരുണ്യത്തിന് മുമ്പില്‍ സന്തോഷസൂചകമായി കാറ്റുകള്‍ അയക്കുകയും ചെയ്യുന്നവനോ? (അതല്ല, നിങ്ങളുടെദൈവങ്ങളോ?) അല്ലാഹുവോടൊപ്പം മറ്റു വല്ല ദൈവവുമുണ്ടോ? അവര്‍ പങ്കുചേര്‍ക്കുന്നതിനെല്ലാം അല്ലാഹു അതീതനായിരിക്കുന്നു.
\end{malayalam}}
\flushright{\begin{Arabic}
\quranayah[27][64]
\end{Arabic}}
\flushleft{\begin{malayalam}
അഥവാ, സൃഷ്ടി ആരംഭിക്കുകയും പിന്നീട് അത് ആവര്‍ത്തിക്കുകയും, ആകാശത്തു നിന്നും ഭൂമിയില്‍ നിന്നും നിങ്ങള്‍ക്ക് ഉപജീവനം നല്‍കുകയും ചെയ്യുന്നവനോ? (അതല്ല, അവരുടെ ദൈവങ്ങളോ?) അല്ലാഹുവോടൊപ്പം വേറെ വല്ല ദൈവവുമുണ്ടോ? (നബിയേ,) പറയുക: നിങ്ങള്‍ സത്യവാന്‍മാരാണെങ്കില്‍ നിങ്ങള്‍ക്കുള്ള തെളിവ് നിങ്ങള്‍ കൊണ്ട് വരിക.
\end{malayalam}}
\flushright{\begin{Arabic}
\quranayah[27][65]
\end{Arabic}}
\flushleft{\begin{malayalam}
(നബിയേ,) പറയുക; ആകാശങ്ങളിലും ഭൂമിയിലും ഉള്ളവരാരും അദൃശ്യകാര്യം അറിയുകയില്ല; അല്ലാഹുവല്ലാതെ. തങ്ങള്‍ എന്നാണ് ഉയിര്‍ത്തെഴുന്നേല്‍പിക്കപ്പെടുക എന്നും അവര്‍ക്കറിയില്ല.
\end{malayalam}}
\flushright{\begin{Arabic}
\quranayah[27][66]
\end{Arabic}}
\flushleft{\begin{malayalam}
അല്ല, അവരുടെ അറിവ് പരലോകത്തില്‍ എത്തി നില്‍ക്കുകയാണ്‌. അല്ല, അവര്‍ അതിനെപ്പറ്റി സംശയത്തിലാകുന്നു. അല്ല, അവര്‍ അതിനെപ്പറ്റി അന്ധതയില്‍ കഴിയുന്നവരത്രെ.
\end{malayalam}}
\flushright{\begin{Arabic}
\quranayah[27][67]
\end{Arabic}}
\flushleft{\begin{malayalam}
അവിശ്വസിച്ചവര്‍ പറഞ്ഞു: ഞങ്ങളും ഞങ്ങളുടെ പിതാക്കളുമൊക്കെ മണ്ണായിക്കഴിഞ്ഞാല്‍ ഞങ്ങള്‍ (ശവകുടീരങ്ങളില്‍ നിന്ന്‌) പുറത്ത് കൊണ്ടുവരപ്പെടുന്നവരാണെന്നോ?
\end{malayalam}}
\flushright{\begin{Arabic}
\quranayah[27][68]
\end{Arabic}}
\flushleft{\begin{malayalam}
ഞങ്ങളോടും മുമ്പ് ഞങ്ങളുടെ പിതാക്കളോടും ഇപ്രകാരം വാഗ്ദാനം ചെയ്യപ്പെടുകയുണ്ടായിട്ടുണ്ട്‌. പൂര്‍വ്വികന്‍മാരുടെ ഇതിഹാസങ്ങള്‍ മാത്രമാകുന്നു ഇത്‌.
\end{malayalam}}
\flushright{\begin{Arabic}
\quranayah[27][69]
\end{Arabic}}
\flushleft{\begin{malayalam}
(നബിയേ,) പറയുക: നിങ്ങള്‍ ഭൂമിയില്‍ കൂടി സഞ്ചരിച്ചിട്ട് കുറ്റവാളികളുടെ പര്യവസാനം എപ്രകാരമായിരുന്നു എന്ന് നോക്കുക.
\end{malayalam}}
\flushright{\begin{Arabic}
\quranayah[27][70]
\end{Arabic}}
\flushleft{\begin{malayalam}
നീ അവരുടെ പേരില്‍ ദുഃഖിക്കേണ്ട. അവര്‍ നടത്തിക്കൊണ്ടിരിക്കുന്ന കുതന്ത്രത്തെപ്പറ്റി നീ മനഃപ്രയാസത്തിലാവുകയും വേണ്ട.
\end{malayalam}}
\flushright{\begin{Arabic}
\quranayah[27][71]
\end{Arabic}}
\flushleft{\begin{malayalam}
അവര്‍ പറയുന്നു: എപ്പോഴാണ് ഈ വാഗ്ദാനം നടപ്പില്‍ വരിക? നിങ്ങള്‍ സത്യവാന്‍മാരാണെങ്കില്‍ (പറഞ്ഞുതരൂ.)
\end{malayalam}}
\flushright{\begin{Arabic}
\quranayah[27][72]
\end{Arabic}}
\flushleft{\begin{malayalam}
നീ പറയുക: നിങ്ങള്‍ ധൃതികൂട്ടിക്കൊണ്ടിരിക്കുന്ന കാര്യങ്ങളില്‍ ചിലത് ഒരു പക്ഷെ നിങ്ങളുടെ തൊട്ടു പുറകില്‍ എത്തിയിട്ടുണ്ടായിരിക്കാം.
\end{malayalam}}
\flushright{\begin{Arabic}
\quranayah[27][73]
\end{Arabic}}
\flushleft{\begin{malayalam}
തീര്‍ച്ചയായും നിന്‍റെ രക്ഷിതാവ് മനുഷ്യരോട് ഔദാര്യമുള്ളവന്‍ തന്നെയാകുന്നു. പക്ഷെ അവരില്‍ അധികപേരും നന്ദികാണിക്കുന്നില്ല.
\end{malayalam}}
\flushright{\begin{Arabic}
\quranayah[27][74]
\end{Arabic}}
\flushleft{\begin{malayalam}
അവരുടെ ഹൃദയങ്ങള്‍ ഒളിച്ച് വെക്കുന്നതും അവര്‍ പരസ്യമാക്കുന്നതും എല്ലാം നിന്‍റെ രക്ഷിതാവ് അറിയുന്നു.
\end{malayalam}}
\flushright{\begin{Arabic}
\quranayah[27][75]
\end{Arabic}}
\flushleft{\begin{malayalam}
ആകാശത്തിലോ ഭൂമിയിലോ മറഞ്ഞു കിടക്കുന്ന യാതൊരു കാര്യവും സ്പഷ്ടമായ ഒരു രേഖയില്‍ രേഖപ്പെടുത്താതിരുന്നിട്ടില്ല.
\end{malayalam}}
\flushright{\begin{Arabic}
\quranayah[27][76]
\end{Arabic}}
\flushleft{\begin{malayalam}
ഇസ്രായീല്‍ സന്തതികള്‍ അഭിപ്രായഭിന്നത പുലര്‍ത്തിക്കൊണ്ടിരിക്കുന്ന വിഷയങ്ങളില്‍ മിക്കതും ഈ ഖുര്‍ആന്‍ അവര്‍ക്ക് വിവരിച്ചുകൊടുക്കുന്നു.
\end{malayalam}}
\flushright{\begin{Arabic}
\quranayah[27][77]
\end{Arabic}}
\flushleft{\begin{malayalam}
തീര്‍ച്ചയായും ഇത് സത്യവിശ്വാസികള്‍ക്ക് മാര്‍ഗദര്‍ശനവും കാരുണ്യവുമാകുന്നു.
\end{malayalam}}
\flushright{\begin{Arabic}
\quranayah[27][78]
\end{Arabic}}
\flushleft{\begin{malayalam}
തീര്‍ച്ചയായും നിന്‍റെ രക്ഷിതാവ് തന്‍റെ വിധിയിലൂടെ അവര്‍ക്കിടയില്‍ തീര്‍പ്പുകല്‍പിക്കുന്നതാണ്‌. അവനത്രെ പ്രതാപിയും സര്‍വ്വജ്ഞനും.
\end{malayalam}}
\flushright{\begin{Arabic}
\quranayah[27][79]
\end{Arabic}}
\flushleft{\begin{malayalam}
അതിനാല്‍ നീ അല്ലാഹുവെ ഭരമേല്‍പിച്ചു കൊള്ളുക. തീര്‍ച്ചയായും നീ സ്പഷ്ടമായ സത്യത്തില്‍ തന്നെയാകുന്നു.
\end{malayalam}}
\flushright{\begin{Arabic}
\quranayah[27][80]
\end{Arabic}}
\flushleft{\begin{malayalam}
മരണപ്പെട്ടവരെ നിനക്ക് കേള്‍പിക്കാനാവുകയില്ല; തീര്‍ച്ച. ബധിരന്‍മാര്‍ പുറംതിരിച്ചു മാറിപോയാല്‍ അവരെയും നിനക്ക് വിളികേള്‍പിക്കാനാവില്ല.
\end{malayalam}}
\flushright{\begin{Arabic}
\quranayah[27][81]
\end{Arabic}}
\flushleft{\begin{malayalam}
അന്ധന്‍മാരെ അവരുടെ ദുര്‍മാര്‍ഗത്തില്‍ നിന്നും നേര്‍വഴിക്ക് കൊണ്ടുവരാനും നിനക്ക് കഴിയില്ല. നമ്മുടെ ദൃഷ്ടാന്തങ്ങളില്‍ വിശ്വസിക്കുകയും തന്നിമിത്തം കീഴൊതുങ്ങുന്നവരായിരിക്കുകയും ചെയ്യുന്നവരെയല്ലാതെ നിനക്ക് കേള്‍പിക്കാനാവില്ല.
\end{malayalam}}
\flushright{\begin{Arabic}
\quranayah[27][82]
\end{Arabic}}
\flushleft{\begin{malayalam}
ആ വാക്ക് അവരുടെ മേല്‍ വന്നുഭവിച്ചാല്‍ ഭൂമിയില്‍ നിന്ന് ഒരു ജന്തുവെ നാം അവരുടെ നേരെ പുറപ്പെടുവിക്കുന്നതാണ്‌. മനുഷ്യര്‍ നമ്മുടെ ദൃഷ്ടാന്തങ്ങളില്‍ ദൃഢവിശ്വാസം കൊള്ളാതിരിക്കുകയാകുന്നു എന്ന വിഷയം അത് അവരോട് സംസാരിക്കുന്നതാണ്‌.
\end{malayalam}}
\flushright{\begin{Arabic}
\quranayah[27][83]
\end{Arabic}}
\flushleft{\begin{malayalam}
ഓരോ സമുദായത്തില്‍ നിന്നും നമ്മുടെ ദൃഷ്ടാന്തങ്ങളെ നിഷേധിച്ച് തള്ളുന്ന ഓരോ സംഘത്തെ നാം ഒരുമിച്ചുകൂട്ടുകയും, അങ്ങനെ അവര്‍ ക്രമത്തില്‍ നിര്‍ത്തപ്പെടുകയും ചെയ്യുന്ന ദിവസത്തെ (ഓര്‍ക്കുക.)
\end{malayalam}}
\flushright{\begin{Arabic}
\quranayah[27][84]
\end{Arabic}}
\flushleft{\begin{malayalam}
അങ്ങനെ അവര്‍ വന്നു കഴിഞ്ഞാല്‍ അവന്‍ പറയും: എന്‍റെ ദൃഷ്ടാന്തങ്ങളെപ്പറ്റി തികച്ചും മനസ്സിലാക്കാതെ നിങ്ങള്‍ അവയെ നിഷേധിച്ച് തള്ളുകയാണോ ചെയ്തത്‌? അതല്ല, എന്താണ് നിങ്ങള്‍ ചെയ്തു കൊണ്ടിരുന്നത്‌.?
\end{malayalam}}
\flushright{\begin{Arabic}
\quranayah[27][85]
\end{Arabic}}
\flushleft{\begin{malayalam}
അവര്‍ അക്രമം പ്രവര്‍ത്തിച്ചതു നിമിത്തം (ശിക്ഷയെപ്പറ്റിയുള്ള) വാക്ക് അവരുടെ മേല്‍ വന്നു ഭവിച്ചു. അപ്പോള്‍ അവര്‍ (യാതൊന്നും) ഉരിയാടുകയില്ല.
\end{malayalam}}
\flushright{\begin{Arabic}
\quranayah[27][86]
\end{Arabic}}
\flushleft{\begin{malayalam}
രാത്രിയെ നാം അവര്‍ക്ക് സമാധാനമടയാനുള്ളതാക്കുകയും, പകലിനെ പ്രകാശമുള്ളതാക്കുകയും ചെയ്തിരിക്കുന്നു എന്നവര്‍ കണ്ടില്ലേ? വിശ്വസിക്കുന്ന ജനങ്ങള്‍ക്കു തീര്‍ച്ചയായും അതില്‍ ദൃഷ്ടാന്തങ്ങളുണ്ട്‌.
\end{malayalam}}
\flushright{\begin{Arabic}
\quranayah[27][87]
\end{Arabic}}
\flushleft{\begin{malayalam}
കാഹളത്തില്‍ ഊതപ്പെടുന്ന ദിവസത്തെ (ഓര്‍ക്കുക). അപ്പോള്‍ ആകാശങ്ങളിലുള്ളവരും, ഭൂമിയിലുള്ളവരും ഭയവിഹ്വലരായിപ്പോകും; അല്ലാഹു ഉദ്ദേശിച്ചവരൊഴികെ. എല്ലാവരും എളിയവരായിക്കൊണ്ട് അവന്‍റെ അടുത്ത് ചെല്ലുകയും ചെയ്യും.
\end{malayalam}}
\flushright{\begin{Arabic}
\quranayah[27][88]
\end{Arabic}}
\flushleft{\begin{malayalam}
പര്‍വ്വതങ്ങളെ നീ കാണുമ്പോള്‍ അവ ഉറച്ചുനില്‍ക്കുന്നതാണ് എന്ന് നീ ധരിച്ച് പോകും. എന്നാല്‍ അവ മേഘങ്ങള്‍ ചലിക്കുന്നത് പോലെ ചലിക്കുന്നതാണ്‌. എല്ലാകാര്യവും കുറ്റമറ്റതാക്കിത്തീര്‍ത്ത അല്ലാഹുവിന്‍റെ പ്രവര്‍ത്തനമത്രെ അത്‌. തീര്‍ച്ചയായും അവന്‍ നിങ്ങള്‍ പ്രവര്‍ത്തിക്കുന്നതിനെപ്പറ്റി സൂക്ഷ്മമായി അറിയുന്നവനാകുന്നു.
\end{malayalam}}
\flushright{\begin{Arabic}
\quranayah[27][89]
\end{Arabic}}
\flushleft{\begin{malayalam}
ആര്‍ നന്‍മയും കൊണ്ട് വന്നോ അവന് (അന്ന്‌) അതിനെക്കാള്‍ ഉത്തമമായത് ഉണ്ടായിരിക്കും. അന്ന് ഭയവിഹ്വലതയില്‍ നിന്ന് അവര്‍ സുരക്ഷിതരായിരിക്കുകയും ചെയ്യും.
\end{malayalam}}
\flushright{\begin{Arabic}
\quranayah[27][90]
\end{Arabic}}
\flushleft{\begin{malayalam}
ആര്‍ തിന്‍മയും കൊണ്ട് വന്നുവോ അവര്‍ നരകത്തില്‍ മുഖം കുത്തി വീഴ്ത്തപ്പെടുന്നതാണ്‌. നിങ്ങള്‍ പ്രവര്‍ത്തിച്ചു കൊണ്ടിരുന്നതിനല്ലാതെ നിങ്ങള്‍ക്ക് പ്രതിഫലം നല്‍കപ്പെടുമോ?
\end{malayalam}}
\flushright{\begin{Arabic}
\quranayah[27][91]
\end{Arabic}}
\flushleft{\begin{malayalam}
(നീ പറയുക:) ഈ രാജ്യത്തെ പവിത്രമാക്കിത്തീര്‍ത്ത ഇതിന്‍റെ രക്ഷിതാവിനെ ആരാധിക്കുവാന്‍ മാത്രമാണ് ഞാന്‍ കല്‍പിക്കപ്പെട്ടിട്ടുള്ളത്‌. എല്ലാ വസ്തുവും അവന്‍റെതത്രെ. ഞാന്‍ കീഴ്പെടുന്നവരുടെ കൂട്ടത്തിലായിരിക്കണമെന്നും കല്‍പിക്കപ്പെട്ടിരിക്കുന്നു.
\end{malayalam}}
\flushright{\begin{Arabic}
\quranayah[27][92]
\end{Arabic}}
\flushleft{\begin{malayalam}
ഖുര്‍ആന്‍ ഓതികേള്‍പിക്കുവാനും (ഞാന്‍ കല്‍പിക്കപ്പെട്ടിരിക്കുന്നു.) ആകയാല്‍ വല്ലവരും സന്‍മാര്‍ഗം സ്വീകരിക്കുന്ന പക്ഷം സ്വന്തം ഗുണത്തിനായി തന്നെയാണ് അവന്‍ സന്‍മാര്‍ഗം സ്വീകരിക്കുന്നത്‌. വല്ലവനും വ്യതിചലിച്ചു പോകുന്ന പക്ഷം നീ പറഞ്ഞേക്കുക: ഞാന്‍ മുന്നറിയിപ്പുകാരില്‍ ഒരാള്‍ മാത്രമാകുന്നു.
\end{malayalam}}
\flushright{\begin{Arabic}
\quranayah[27][93]
\end{Arabic}}
\flushleft{\begin{malayalam}
പറയുക: അല്ലാഹുവിന് സ്തുതി. തന്‍റെ ദൃഷ്ടാന്തങ്ങള്‍ അവന്‍ നിങ്ങള്‍ക്ക് കാണിച്ചുതരുന്നതാണ്‌. അപ്പോള്‍ നിങ്ങള്‍ക്കവ മനസ്സിലാകും. നിങ്ങള്‍ പ്രവര്‍ത്തിച്ചുകൊണ്ടിരിക്കുന്ന യാതൊന്നിനെപ്പറ്റിയും നിന്‍റെ രക്ഷിതാവ് അശ്രദ്ധനല്ല.
\end{malayalam}}
\chapter{\textmalayalam{ഖസസ് ( കഥാകഥനം‍ )}}
\begin{Arabic}
\Huge{\centerline{\basmalah}}\end{Arabic}
\flushright{\begin{Arabic}
\quranayah[28][1]
\end{Arabic}}
\flushleft{\begin{malayalam}
ത്വാ-സീന്‍-മീം.
\end{malayalam}}
\flushright{\begin{Arabic}
\quranayah[28][2]
\end{Arabic}}
\flushleft{\begin{malayalam}
സ്പഷ്ടമായ വേദഗ്രന്ഥത്തിലെ വചനങ്ങളത്രെ അവ.
\end{malayalam}}
\flushright{\begin{Arabic}
\quranayah[28][3]
\end{Arabic}}
\flushleft{\begin{malayalam}
വിശ്വസിക്കുന്ന ജനങ്ങള്‍ക്ക് വേണ്ടി മൂസായുടെയും ഫിര്‍ഔന്‍റെയും വൃത്താന്തത്തില്‍ നിന്നും സത്യപ്രകാരം നിനക്ക് നാം ഓതികേള്‍പിക്കുന്നു.
\end{malayalam}}
\flushright{\begin{Arabic}
\quranayah[28][4]
\end{Arabic}}
\flushleft{\begin{malayalam}
തീര്‍ച്ചയായും ഫിര്‍ഔന്‍ നാട്ടില്‍ ഔന്നത്യം നടിച്ചു. അവിടത്തുകാരെ അവന്‍ വ്യത്യസ്ത കക്ഷികളാക്കിത്തീര്‍ക്കുകയും ചെയ്തു. അവരില്‍ ഒരു വിഭാഗത്തെ ദുര്‍ബലരാക്കിയിട്ട് അവരുടെ ആണ്‍മക്കളെ അറുകൊല നടത്തുകയും അവരുടെ പെണ്‍മക്കളെ ജീവിക്കാന്‍ അനുവദിക്കുകയും ചെയ്തുകൊണ്ട്‌. തീര്‍ച്ചയായും അവന്‍ നാശകാരികളില്‍ പെട്ടവനായിരുന്നു.
\end{malayalam}}
\flushright{\begin{Arabic}
\quranayah[28][5]
\end{Arabic}}
\flushleft{\begin{malayalam}
നാമാകട്ടെ ഭൂമിയില്‍ അടിച്ചമര്‍ത്തപ്പെട്ട ദുര്‍ബലരോട് ഔദാര്യം കാണിക്കുവാനും, അവരെ നേതാക്കളാക്കുവാനും, അവരെ (നാടിന്‍റെ) അനന്തരാവകാശികളാക്കാനുമാണ് ഉദ്ദേശിക്കുന്നത്‌.
\end{malayalam}}
\flushright{\begin{Arabic}
\quranayah[28][6]
\end{Arabic}}
\flushleft{\begin{malayalam}
അവര്‍ക്ക് (ആ മര്‍ദ്ദിതര്‍ക്ക്‌) ഭൂമിയില്‍ സ്വാധീനം നല്‍കുവാനും, ഫിര്‍ഔന്നും ഹാമാന്നും അവരുടെ സൈന്യങ്ങള്‍ക്കും അവരില്‍ നിന്ന് തങ്ങള്‍ ആശങ്കിച്ചിരുന്നതെന്തോ അത് കാണിച്ചുകൊടുക്കുവാനും (നാം ഉദ്ദേശിക്കുന്നു.)
\end{malayalam}}
\flushright{\begin{Arabic}
\quranayah[28][7]
\end{Arabic}}
\flushleft{\begin{malayalam}
മൂസായുടെ മാതാവിന് നാം ബോധനം നല്‍കി: അവന്ന് നീ മുലകൊടുത്തു കൊള്ളുക. ഇനി അവന്‍റെ കാര്യത്തില്‍ നിനക്ക് ഭയം തോന്നുകയാണെങ്കില്‍ അവനെ നീ നദിയില്‍ ഇട്ടേക്കുക. നീ ഭയപ്പെടുകയും ദുഃഖിക്കുകയും വേണ്ട. തീര്‍ച്ചയായും അവനെ നാം നിന്‍റെ അടുത്തേക്ക് തിരിച്ച് കൊണ്ട് വരുന്നതും , അവനെ ദൈവദൂതന്‍മാരില്‍ ഒരാളാക്കുന്നതുമാണ്‌.
\end{malayalam}}
\flushright{\begin{Arabic}
\quranayah[28][8]
\end{Arabic}}
\flushleft{\begin{malayalam}
എന്നിട്ട് ഫിര്‍ഔന്‍റെ ആളുകള്‍ അവനെ (നദിയില്‍ നിന്ന്‌) കണ്ടെടുത്തു. അവന്‍ അവരുടെ ശത്രുവും ദുഃഖഹേതുവും ആയിരിക്കാന്‍ വേണ്ടി. തീര്‍ച്ചയായും ഫിര്‍ഔനും ഹാമാനും അവരുടെ സൈന്യങ്ങളും അബദ്ധം പറ്റിയവരായിരുന്നു.
\end{malayalam}}
\flushright{\begin{Arabic}
\quranayah[28][9]
\end{Arabic}}
\flushleft{\begin{malayalam}
ഫിര്‍ഔന്‍റെ ഭാര്യ പറഞ്ഞു: എനിക്കും അങ്ങേക്കും കണ്ണിന് കുളിര്‍മയത്രെ (ഈ കുട്ടി.) അതിനാല്‍ ഇവനെ നിങ്ങള്‍ കൊല്ലരുത്‌. ഇവന്‍ നമുക്ക് ഉപകരിച്ചേക്കാം. അല്ലെങ്കില്‍ ഇവനെ നമുക്ക് ഒരു മകനായി സ്വീകരിക്കാം. അവര്‍ യാഥാര്‍ത്ഥ്യം ഗ്രഹിച്ചിരുന്നില്ല.
\end{malayalam}}
\flushright{\begin{Arabic}
\quranayah[28][10]
\end{Arabic}}
\flushleft{\begin{malayalam}
മൂസായുടെ മാതാവിന്‍റെ മനസ്സ് (അന്യ ചിന്തകളില്‍ നിന്ന്‌) ഒഴിവായതായിത്തീര്‍ന്നു. അവളുടെ മനസ്സിനെ നാം ഉറപ്പിച്ചു നിര്‍ത്തിയില്ലായിരുന്നുവെങ്കില്‍ അവന്‍റെ കാര്യം അവള്‍ വെളിപ്പെടുത്തിയേക്കുമായിരുന്നു. അവള്‍ സത്യവിശ്വാസികളുടെ കൂട്ടത്തിലായിരിക്കാന്‍ വേണ്ടിയത്രെ (നാം അങ്ങനെ ചെയ്തത്‌.)
\end{malayalam}}
\flushright{\begin{Arabic}
\quranayah[28][11]
\end{Arabic}}
\flushleft{\begin{malayalam}
അവള്‍ അവന്‍റെ (മൂസായുടെ) സഹോദരിയോട് പറഞ്ഞു: നീ അവന്‍റെ പിന്നാലെ പോയി അന്വേഷിച്ചു നോക്കൂ. അങ്ങനെ ദൂരെ നിന്ന് അവള്‍ അവനെ നിരീക്ഷിച്ചു. അവര്‍ അതറിഞ്ഞിരുന്നില്ല.
\end{malayalam}}
\flushright{\begin{Arabic}
\quranayah[28][12]
\end{Arabic}}
\flushleft{\begin{malayalam}
അതിനു മുമ്പ് മുലയൂട്ടുന്ന സ്ത്രീകള്‍ അവന്ന് മുലകൊടുക്കുന്നതിന് നാം തടസ്സമുണ്ടാക്കിയിരുന്നു. അപ്പോള്‍ അവള്‍ (സഹോദരി) പറഞ്ഞു: നിങ്ങള്‍ക്ക് വേണ്ടി ഇവനെ സംരക്ഷിക്കുന്ന ഒരു വീട്ടുകാരെപ്പറ്റി ഞാന്‍ നിങ്ങള്‍ക്ക് അറിവ് തരട്ടെയോ? അവര്‍ ഇവന്‍റെ ഗുണകാംക്ഷികളായിരിക്കുകയും ചെയ്യും.
\end{malayalam}}
\flushright{\begin{Arabic}
\quranayah[28][13]
\end{Arabic}}
\flushleft{\begin{malayalam}
അങ്ങനെ അവന്‍റെ മാതാവിന്‍റെ കണ്ണ് കുളിര്‍ക്കുവാനും, അവള്‍ ദുഃഖിക്കാതിരിക്കുവാനും, അല്ലാഹുവിന്‍റെ വാഗ്ദാനം സത്യമാണെന്ന് അവള്‍ മനസ്സിലാക്കുവാനും വേണ്ടി അവനെ നാം അവള്‍ക്ക് തിരിച്ചേല്‍പിച്ചു. പക്ഷെ അവരില്‍ അധികപേരും (കാര്യം) മനസ്സിലാക്കുന്നില്ല.
\end{malayalam}}
\flushright{\begin{Arabic}
\quranayah[28][14]
\end{Arabic}}
\flushleft{\begin{malayalam}
അങ്ങനെ അദ്ദേഹം (മൂസാ) ശക്തി പ്രാപിക്കുകയും, പാകത എത്തുകയും ചെയ്തപ്പോള്‍ അദ്ദേഹത്തിന് നാം വിവേകവും വിജ്ഞാനവും നല്‍കി. അപ്രകാരമാണ് സദ്‌വൃത്തര്‍ക്ക് നാം പ്രതിഫലം നല്‍കുന്നത്‌.
\end{malayalam}}
\flushright{\begin{Arabic}
\quranayah[28][15]
\end{Arabic}}
\flushleft{\begin{malayalam}
പട്ടണവാസികള്‍ അശ്രദ്ധരായിരുന്ന സമയത്ത് മൂസാ അവിടെ കടന്നു ചെന്നു. അപ്പോള്‍ അവിടെ രണ്ടുപുരുഷന്‍മാര്‍ പരസ്പരം പൊരുതുന്നതായി അദ്ദേഹം കണ്ടു. ഒരാള്‍ തന്‍റെ കക്ഷിയില്‍ പെട്ടവന്‍. മറ്റൊരാള്‍ തന്‍റെ ശത്രുവിഭാഗത്തില്‍ പെട്ടവനും. അപ്പോള്‍ തന്‍റെ കക്ഷിയില്‍ പെട്ടവന്‍ തന്‍റെ ശത്രുവിഭാഗത്തില്‍ പെട്ടവന്നെതിരില്‍ അദ്ദേഹത്തോട് സഹായം തേടി. അപ്പോള്‍ മൂസാ അവനെ മുഷ്ടിചുരുട്ടി ഇടിച്ചു. അതവന്‍റെ കഥ കഴിച്ചു. മൂസാ പറഞ്ഞു: ഇത് പിശാചിന്‍റെ പ്രവര്‍ത്തനത്തില്‍ പെട്ടതാകുന്നു. അവന്‍ വ്യക്തമായും വഴിപിഴപ്പിക്കുന്ന ശത്രു തന്നെയാകുന്നു.
\end{malayalam}}
\flushright{\begin{Arabic}
\quranayah[28][16]
\end{Arabic}}
\flushleft{\begin{malayalam}
അദ്ദേഹം പറഞ്ഞു: എന്‍റെ രക്ഷിതാവേ, തീര്‍ച്ചയായും ഞാന്‍ എന്നോട് തന്നെ അന്യായം ചെയ്തിരിക്കുന്നു. അതിനാല്‍ നീ എനിക്ക് പൊറുത്തുതരേണമേ. അപ്പോള്‍ അദ്ദേഹത്തിന് അവന്‍ പൊറുത്തുകൊടുത്തു. തീര്‍ച്ചയായും അവന്‍ ഏറെ പൊറുക്കുന്നവനും കരുണാനിധിയുമാകുന്നു.
\end{malayalam}}
\flushright{\begin{Arabic}
\quranayah[28][17]
\end{Arabic}}
\flushleft{\begin{malayalam}
അദ്ദേഹം പറഞ്ഞു: എന്‍റെ രക്ഷിതാവേ, നീ എനിക്ക് അനുഗ്രഹം നല്‍കിയിട്ടുള്ളതു കൊണ്ട് ഇനി ഒരിക്കലും ഞാന്‍ കുറ്റവാളികള്‍ക്കു സഹായം നല്‍കുന്നവനാവുകയില്ല.
\end{malayalam}}
\flushright{\begin{Arabic}
\quranayah[28][18]
\end{Arabic}}
\flushleft{\begin{malayalam}
അങ്ങനെ അദ്ദേഹം പട്ടണത്തില്‍ ഭയപ്പാടോടും കരുതലോടും കൂടി വര്‍ത്തിച്ചു. അപ്പോഴതാ തലേദിവസം തന്നോട് സഹായം തേടിയവന്‍ വീണ്ടും തന്നോട് സഹായത്തിനു മുറവിളികൂട്ടുന്നു. മൂസാ അവനോട് പറഞ്ഞു: നീ വ്യക്തമായും ഒരു ദുര്‍മാര്‍ഗി തന്നെയാകുന്നു.
\end{malayalam}}
\flushright{\begin{Arabic}
\quranayah[28][19]
\end{Arabic}}
\flushleft{\begin{malayalam}
എന്നിട്ട് അവര്‍ ഇരുവര്‍ക്കും ശത്രുവായിട്ടുള്ളവനെ പിടികൂടാന്‍ അദ്ദേഹം ഉദ്ദേശിച്ചപ്പോള്‍ അവന്‍ പറഞ്ഞു: ഹേ മൂസാ, ഇന്നലെ നീ ഒരാളെ കൊന്നത് പോലെ നീ എന്നെയും കൊല്ലാന്‍ ഉദ്ദേശിക്കുകയാണോ? നാട്ടില്‍ ഒരു പോക്കിരിയാകാന്‍ മാത്രമാണ് നീ ഉദ്ദേശിക്കുന്നത്‌. നന്‍മയുണ്ടാക്കുന്നവരുടെ കൂട്ടത്തിലാകാന്‍ നീ ഉദ്ദേശിക്കുന്നില്ല.
\end{malayalam}}
\flushright{\begin{Arabic}
\quranayah[28][20]
\end{Arabic}}
\flushleft{\begin{malayalam}
പട്ടണത്തിന്‍റെ അങ്ങേ അറ്റത്തു നിന്ന് ഒരു പുരുഷന്‍ ഓടിവന്നു. അയാള്‍ പറഞ്ഞു: ഹേ; മൂസാ, താങ്കളെ കൊല്ലാന്‍ വേണ്ടി പ്രമുഖവ്യക്തികള്‍ ആലോചന നടത്തികൊണ്ടിരിക്കുന്നുണ്ട്‌. അതിനാല്‍ താങ്കള്‍ (ഈജിപ്തില്‍ നിന്ന്‌) പുറത്ത് പോയിക്കൊള്ളുക. തീര്‍ച്ചയായും ഞാന്‍ താങ്കളുടെ ഗുണകാംക്ഷികളുടെ കൂട്ടത്തിലാകുന്നു.
\end{malayalam}}
\flushright{\begin{Arabic}
\quranayah[28][21]
\end{Arabic}}
\flushleft{\begin{malayalam}
അങ്ങനെ ഭയപ്പാടോടും കരുതലോടും കൂടി അദ്ദേഹം അവിടെ നിന്ന് പുറപ്പെട്ടു. അദ്ദേഹം പറഞ്ഞു: എന്‍റെ രക്ഷിതാവേ, അക്രമികളായ ജനതയില്‍ നിന്ന് എന്നെ നീ രക്ഷപ്പെടുത്തേണമേ.
\end{malayalam}}
\flushright{\begin{Arabic}
\quranayah[28][22]
\end{Arabic}}
\flushleft{\begin{malayalam}
മദ്‌യന്‍റെ നേര്‍ക്ക് യാത്ര തിരിച്ചപ്പോള്‍ അദ്ദേഹം പറഞ്ഞു: എന്‍റെ രക്ഷിതാവ് ശരിയായ മാര്‍ഗത്തിലേക്ക് എന്നെ നയിച്ചേക്കാം.
\end{malayalam}}
\flushright{\begin{Arabic}
\quranayah[28][23]
\end{Arabic}}
\flushleft{\begin{malayalam}
മദ്‌യനിലെ ജലാശയത്തിങ്കല്‍ അദ്ദേഹം ചെന്നെത്തിയപ്പോള്‍ ആടുകള്‍ക്ക് വെള്ളം കൊടുത്ത് കൊണ്ടിരിക്കുന്ന ഒരു കൂട്ടം ആളുകളെ അതിന്നടുത്ത് അദ്ദേഹം കണ്ടെത്തി. അവരുടെ ഇപ്പുറത്തായി (തങ്ങളുടെ ആട്ടിന്‍ പറ്റത്തെ) തടഞ്ഞു നിര്‍ത്തിക്കൊണ്ടിരിക്കുന്ന രണ്ട് സ്ത്രീകളെയും അദ്ദേഹം കണ്ടു. അദ്ദേഹം ചോദിച്ചു: എന്താണ് നിങ്ങളുടെ പ്രശ്നം? അവര്‍ പറഞ്ഞു: ഇടയന്‍മാര്‍ (ആടുകള്‍ക്ക് വെള്ളം കൊടുത്ത്‌) തിരിച്ചു കൊണ്ടു പോകുന്നത് വരെ ഞങ്ങള്‍ക്ക് വെള്ളം കൊടുക്കാനാവില്ല. ഞങ്ങളുടെ പിതാവാകട്ടെ വലിയൊരു വൃദ്ധനുമാണ്‌.
\end{malayalam}}
\flushright{\begin{Arabic}
\quranayah[28][24]
\end{Arabic}}
\flushleft{\begin{malayalam}
അങ്ങനെ അവര്‍ക്കു വേണ്ടി അദ്ദേഹം (അവരുടെ കാലികള്‍ക്ക്‌) വെള്ളം കൊടുത്തു. പിന്നീടദ്ദേഹം തണലിലേക്ക് മാറിയിരുന്നിട്ട് ഇപ്രകാരം പ്രാര്‍ത്ഥിച്ചു: എന്‍റെ രക്ഷിതാവേ, നീ എനിക്ക് ഇറക്കിത്തരുന്ന ഏതൊരു നന്‍മയ്ക്കും ഞാന്‍ ആവശ്യക്കാരനാകുന്നു.
\end{malayalam}}
\flushright{\begin{Arabic}
\quranayah[28][25]
\end{Arabic}}
\flushleft{\begin{malayalam}
അപ്പോള്‍ ആ രണ്ട് സ്ത്രീകളില്‍ ഒരാള്‍ നാണിച്ചു കൊണ്ട് അദ്ദേഹത്തിന്‍റെ അടുത്ത് നടന്നു ചെന്നിട്ട് പറഞ്ഞു: താങ്കള്‍ ഞങ്ങള്‍ക്കു വേണ്ടി (ആടുകള്‍ക്ക്‌) വെള്ളം കൊടുത്തതിനുള്ള പ്രതിഫലം താങ്കള്‍ക്കു നല്‍കുവാനായി എന്‍റെ പിതാവ് താങ്കളെ വിളിക്കുന്നു. അങ്ങനെ മൂസാ അദ്ദേഹത്തിന്‍റെ അടുത്ത് ചെന്നിട്ട് തന്‍റെ കഥ അദ്ദേഹത്തിന് വിവരിച്ചുകൊടുത്തപ്പോള്‍ അദ്ദേഹം പറഞ്ഞു: ഭയപ്പെടേണ്ട. അക്രമികളായ ആ ജനതയില്‍നിന്ന് നീ രക്ഷപ്പെട്ടിരിക്കുന്നു.
\end{malayalam}}
\flushright{\begin{Arabic}
\quranayah[28][26]
\end{Arabic}}
\flushleft{\begin{malayalam}
ആ രണ്ടുസ്ത്രീകളിലൊരാള്‍ പറഞ്ഞു: എന്‍റെ പിതാവേ, ഇദ്ദേഹത്തെ താങ്കള്‍ കൂലിക്കാരനായി നിര്‍ത്തുക. തീര്‍ച്ചയായും താങ്കള്‍ കൂലിക്കാരായി എടുക്കുന്നവരില്‍ ഏറ്റവും ഉത്തമന്‍ ശക്തനും വിശ്വസ്തനുമായിട്ടുള്ളവനത്രെ.
\end{malayalam}}
\flushright{\begin{Arabic}
\quranayah[28][27]
\end{Arabic}}
\flushleft{\begin{malayalam}
അദ്ദേഹം (പിതാവ്‌) പറഞ്ഞു: നീ എട്ടു വര്‍ഷം എനിക്ക് കൂലിവേല ചെയ്യണം എന്ന വ്യവസ്ഥയില്‍ എന്‍റെ ഈ രണ്ടു പെണ്‍മക്കളില്‍ ഒരാളെ നിനക്ക് വിവാഹം ചെയ്തു തരാന്‍ ഞാന്‍ ഉദ്ദേശിക്കുന്നു. ഇനി പത്തുവര്‍ഷം നീ പൂര്‍ത്തിയാക്കുകയാണെങ്കില്‍ അത് നിന്‍റെ ഇഷ്ടം. നിനക്ക് പ്രയാസമുണ്ടാക്കാന്‍ ഞാന്‍ ഉദ്ദേശിക്കുന്നില്ല. അല്ലാഹു ഉദ്ദേശിക്കുന്ന പക്ഷം നല്ല നിലയില്‍ വര്‍ത്തിക്കുന്നവരില്‍ ഒരാളായി നിനക്ക് എന്നെ കാണാം.
\end{malayalam}}
\flushright{\begin{Arabic}
\quranayah[28][28]
\end{Arabic}}
\flushleft{\begin{malayalam}
അദ്ദേഹം (മൂസാ) പറഞ്ഞു; ഞാനും താങ്കളും തമ്മിലുള്ള നിശ്ചയം അതു തന്നെ. ഈ രണ്ട് അവധികളില്‍ ഏത് ഞാന്‍ നിറവേറ്റിയാലും എന്നോട് വിരോധമുണ്ടാകരുത്‌. നാം പറയുന്നതിന് അല്ലാഹു സാക്ഷിയാകുന്നു.
\end{malayalam}}
\flushright{\begin{Arabic}
\quranayah[28][29]
\end{Arabic}}
\flushleft{\begin{malayalam}
അങ്ങനെ മൂസാ ആ അവധി നിറവേറ്റുകയും, തന്‍റെ കുടുംബവും കൊണ്ട് യാത്രപോകുകയും ചെയ്തപ്പോള്‍ പര്‍വ്വതത്തിന്‍റെ ഭാഗത്ത് നിന്ന് അദ്ദേഹം ഒരു തീ കണ്ടു. അദ്ദേഹം തന്‍റെ കുടുംബത്തോട് പറഞ്ഞു: നിങ്ങള്‍ നില്‍ക്കൂ. ഞാനൊരു തീ കണ്ടിരിക്കുന്നു. അവിടെ നിന്ന് വല്ല വിവരമോ, അല്ലെങ്കില്‍ ഒരു തീക്കൊള്ളിയോ ഞാന്‍ നിങ്ങള്‍ക്ക് കൊണ്ടുവന്ന് തന്നേക്കാം. നിങ്ങള്‍ക്ക് തീ കായാമല്ലോ?
\end{malayalam}}
\flushright{\begin{Arabic}
\quranayah[28][30]
\end{Arabic}}
\flushleft{\begin{malayalam}
അങ്ങനെ അദ്ദേഹം അതിന്നടുത്ത് ചെന്നപ്പോള്‍ അനുഗൃഹീതമായ പ്രദേശത്തുള്ള താഴ്‌വരയുടെ വലതുഭാഗത്ത് നിന്ന്‌, ഒരു വൃക്ഷത്തില്‍ നിന്ന് അദ്ദേഹത്തോട് വിളിച്ചുപറയപ്പെട്ടു: ഹേ; മൂസാ, തീര്‍ച്ചയായും ഞാനാകുന്നു ലോകരക്ഷിതാവായ അല്ലാഹു.
\end{malayalam}}
\flushright{\begin{Arabic}
\quranayah[28][31]
\end{Arabic}}
\flushleft{\begin{malayalam}
നീ നിന്‍റെ വടി താഴെയിടൂ! എന്നിട്ടത് ഒരു സര്‍പ്പമെന്നോണം പിടയുന്നത് കണ്ടപ്പോള്‍ അദ്ദേഹം പിന്നോക്കം തിരിഞ്ഞോടി. അദ്ദേഹം തിരിഞ്ഞ് നോക്കിയത് പോലുമില്ല. അല്ലാഹു പറഞ്ഞു: മൂസാ! നീ മുന്നോട്ട് വരിക. പേടിക്കേണ്ട. തീര്‍ച്ചയായും നീ സുരക്ഷിതരുടെ കൂട്ടത്തിലാകുന്നു.
\end{malayalam}}
\flushright{\begin{Arabic}
\quranayah[28][32]
\end{Arabic}}
\flushleft{\begin{malayalam}
നീ നിന്‍റെ കൈ കുപ്പായമാറിലേക്ക് പ്രവേശിപ്പിക്കുക. യാതൊരു കെടുതിയും കൂടാതെ വെളുത്തതായി അതുപുറത്ത് വരുന്നതാണ്‌. ഭയത്തില്‍ നിന്ന് മോചനത്തിനായ് നിന്‍റെ പാര്‍ശ്വഭാഗം നീ ശരീരത്തിലേക്ക് ചേര്‍ത്ത് പിടിക്കുകയും ചെയ്യുക. അങ്ങനെ അത് രണ്ടും ഫിര്‍ഔന്‍റെയും, അവന്‍റെ പ്രമുഖന്‍മാരുടെയും അടുത്തേക്ക് നിന്‍റെ രക്ഷിതാവിങ്കല്‍ നിന്നുള്ള രണ്ടു തെളിവുകളാകുന്നു. തീര്‍ച്ചയായും അവര്‍ ധിക്കാരികളായ ഒരു ജനതയായിരിക്കുന്നു.
\end{malayalam}}
\flushright{\begin{Arabic}
\quranayah[28][33]
\end{Arabic}}
\flushleft{\begin{malayalam}
അദ്ദേഹം പറഞ്ഞു: എന്‍റെ രക്ഷിതാവേ, അവരുടെ കൂട്ടത്തില്‍ ഒരാളെ ഞാന്‍ കൊന്നുപോയിട്ടുണ്ട്‌. അതിനാല്‍ അവര്‍ എന്നെ കൊല്ലുമെന്ന് ഞാന്‍ ഭയപ്പെടുന്നു.
\end{malayalam}}
\flushright{\begin{Arabic}
\quranayah[28][34]
\end{Arabic}}
\flushleft{\begin{malayalam}
എന്‍റെ സഹോദരന്‍ ഹാറൂന്‍ എന്നെക്കാള്‍ വ്യക്തമായി സംസാരിക്കാന്‍ കഴിവുള്ളവനാകുന്നു. അതു കൊണ്ട് എന്നോടൊപ്പം എന്‍റെ സത്യത സ്ഥാപിക്കുന്ന ഒരു സഹായിയായി കൊണ്ട് അവനെ നീ നിയോഗിക്കേണമേ. അവര്‍ എന്നെ നിഷേധിച്ച് കളയുമെന്ന് തീര്‍ച്ചയായും ഞാന്‍ ഭയപ്പെടുന്നു.
\end{malayalam}}
\flushright{\begin{Arabic}
\quranayah[28][35]
\end{Arabic}}
\flushleft{\begin{malayalam}
അവന്‍ (അല്ലാഹു) പറഞ്ഞു: നിന്‍റെ സഹോദരന്‍ മുഖേന നിന്‍റെ കൈക്ക് നാം ബലം നല്‍കുകയും, നിങ്ങള്‍ക്ക് ഇരുവര്‍ക്കും നാം ഒരു ആധികാരിക ശക്തി നല്‍കുകയും ചെയ്യുന്നതാണ്‌. അതിനാല്‍ അവര്‍ നിങ്ങളുടെ അടുത്ത് എത്തുകയില്ല. നമ്മുടെ ദൃഷ്ടാന്തങ്ങള്‍ നിമിത്തം നിങ്ങളും നിങ്ങളെ പിന്തുടരുന്നവരും തന്നെയായിരിക്കും വിജയികള്‍.
\end{malayalam}}
\flushright{\begin{Arabic}
\quranayah[28][36]
\end{Arabic}}
\flushleft{\begin{malayalam}
അങ്ങനെ നമ്മുടെ വ്യക്തമായ ദൃഷ്ടാന്തങ്ങളും കൊണ്ട് മൂസാ അവരുടെ അടുത്ത് ചെന്നപ്പോള്‍ അവര്‍ പറഞ്ഞു: ഇത് വ്യാജനിര്‍മിതമായ ഒരു ജാലവിദ്യയല്ലാതെ മറ്റൊന്നുമല്ല. നമ്മുടെ പൂര്‍വ്വ പിതാക്കളില്‍ ഇങ്ങനെ ഒരു കാര്യത്തെപ്പറ്റി നാം കേട്ടിട്ടുമില്ല.
\end{malayalam}}
\flushright{\begin{Arabic}
\quranayah[28][37]
\end{Arabic}}
\flushleft{\begin{malayalam}
മൂസാ പറഞ്ഞു: തന്‍റെ പക്കല്‍ നിന്ന് സന്‍മാര്‍ഗവും കൊണ്ട് വന്നിട്ടുള്ളവനാരെന്നും, ഈ ലോകത്തിന്‍റെ പര്യവസാനം ആര്‍ക്ക് അനുകൂലമായിരിക്കുമെന്നും എന്‍റെ രക്ഷിതാവിന് നല്ലപോലെ അറിയാം. അക്രമികള്‍ വിജയം പ്രാപിക്കുകയില്ല; തീര്‍ച്ച.
\end{malayalam}}
\flushright{\begin{Arabic}
\quranayah[28][38]
\end{Arabic}}
\flushleft{\begin{malayalam}
ഫിര്‍ഔന്‍ പറഞ്ഞു: പ്രമുഖന്‍മാരെ, ഞാനല്ലാതെ യാതൊരു ദൈവവും നിങ്ങള്‍ക്കുള്ളതായി ഞാന്‍ അറിഞ്ഞിട്ടില്ല. അതുകൊണ്ട് ഹാമാനേ, എനിക്കു വേണ്ടി കളിമണ്ണുകൊണ്ട് (ഇഷ്ടിക) ചുട്ടെടുക്കുക. എന്നിട്ട് എനിക്ക് നീ ഒരു ഉന്നത സൌധം ഉണ്ടാക്കിത്തരിക. മൂസായുടെ ദൈവത്തിങ്കലേക്ക് എനിക്കൊന്നു എത്തിനോക്കാമല്ലോ. തീര്‍ച്ചയായും അവന്‍ വ്യാജം പറയുന്നവരുടെ കൂട്ടത്തിലാണെന്നാണ് ഞാന്‍ വിചാരിക്കുന്നത്‌.
\end{malayalam}}
\flushright{\begin{Arabic}
\quranayah[28][39]
\end{Arabic}}
\flushleft{\begin{malayalam}
അവനും അവന്‍റെ സൈന്യങ്ങളും ഭൂമിയില്‍ അന്യായമായി അഹങ്കരിക്കുകയും, നമ്മുടെ അടുക്കലേക്ക് അവര്‍ മടക്കപ്പെടുകയില്ലെന്ന് അവര്‍ വിചാരിക്കുകയും ചെയ്തു.
\end{malayalam}}
\flushright{\begin{Arabic}
\quranayah[28][40]
\end{Arabic}}
\flushleft{\begin{malayalam}
അതിനാല്‍ അവനെയും അവന്‍റെ സൈന്യങ്ങളെയും നാം പിടികൂടി കടലില്‍ എറിഞ്ഞ് കളഞ്ഞു. അപ്പോള്‍ ആ അക്രമികളുടെ പര്യവസാനം എങ്ങനെയായിരുന്നു എന്ന് നോക്കൂ.
\end{malayalam}}
\flushright{\begin{Arabic}
\quranayah[28][41]
\end{Arabic}}
\flushleft{\begin{malayalam}
അവരെ നാം നരകത്തിലേക്ക് ക്ഷണിക്കുന്ന നേതാക്കന്‍മാരാക്കി. ഉയിര്‍ത്തെഴുന്നേല്‍പിന്‍റെ നാളില്‍ അവര്‍ക്കൊരു സഹായവും നല്‍കപ്പെടുന്നതല്ല.
\end{malayalam}}
\flushright{\begin{Arabic}
\quranayah[28][42]
\end{Arabic}}
\flushleft{\begin{malayalam}
ഈ ഐഹികജീവിതത്തില്‍ അവരുടെ പിന്നാലെ നാം ശാപം അയക്കുകയും ചെയ്തു. ഉയിര്‍ത്തെഴുന്നേല്‍പിന്‍റെ നാളില്‍ അവര്‍ വെറുക്കപ്പെട്ടവരുടെ കൂട്ടത്തിലായിരിക്കുകയും ചെയ്യും.
\end{malayalam}}
\flushright{\begin{Arabic}
\quranayah[28][43]
\end{Arabic}}
\flushleft{\begin{malayalam}
പൂര്‍വ്വതലമുറകളെ നാം നശിപ്പിച്ചതിന് ശേഷം, ജനങ്ങള്‍ക്കു ഉള്‍കാഴ്ച നല്‍കുന്ന തെളിവുകളും മാര്‍ഗദര്‍ശനവും കാരുണ്യവുമായിക്കൊണ്ട് മൂസായ്ക്ക് നാം വേദഗ്രന്ഥം നല്‍കുകയുണ്ടായി. അവര്‍ ചിന്തിച്ച് ഗ്രഹിച്ചേക്കാമല്ലോ.
\end{malayalam}}
\flushright{\begin{Arabic}
\quranayah[28][44]
\end{Arabic}}
\flushleft{\begin{malayalam}
(നബിയേ,) മൂസായ്ക്ക് നാം കല്‍പന ഏല്‍പിച്ച് കൊടുത്ത സമയത്ത് ആ പടിഞ്ഞാറെ മലയുടെ പാര്‍ശ്വത്തില്‍ നീ ഉണ്ടായിരുന്നില്ല. (ആ സംഭവത്തിന്‌) സാക്ഷ്യം വഹിച്ചവരുടെ കൂട്ടത്തില്‍ നീ ഉണ്ടായിരുന്നതുമില്ല.
\end{malayalam}}
\flushright{\begin{Arabic}
\quranayah[28][45]
\end{Arabic}}
\flushleft{\begin{malayalam}
പക്ഷെ നാം (പിന്നീട്‌) പല തലമുറകളെയും വളര്‍ത്തിയെടുത്തു. അങ്ങനെ അവരിലൂടെ യുഗങ്ങള്‍ ദീര്‍ഘിച്ചു. മദ്‌യങ്കാര്‍ക്ക് നമ്മുടെ ദൃഷ്ടാന്തങ്ങള്‍ ഓതികേള്‍പിച്ചു കൊടുത്തു കൊണ്ട് നീ അവര്‍ക്കിടയില്‍ താമസിച്ചിരുന്നില്ല.പക്ഷെ നാം ദൂതന്‍മാരെ നിയോഗിക്കുന്നവനായിരിക്കുന്നു.
\end{malayalam}}
\flushright{\begin{Arabic}
\quranayah[28][46]
\end{Arabic}}
\flushleft{\begin{malayalam}
നാം (മൂസായെ) വിളിച്ച സമയത്ത് ആ പര്‍വ്വതത്തിന്‍റെ പാര്‍ശ്വത്തില്‍ നീ ഉണ്ടായിരുന്നുമില്ല. പക്ഷെ, നിന്‍റെ രക്ഷിതാവിങ്കല്‍ നിന്നുള്ള കാരുണ്യത്താല്‍ (ഇതെല്ലാം അറിയിച്ച് തരികയാകുന്നു.) നിനക്ക് മുമ്പ് ഒരു താക്കീതുകാരനും വന്നിട്ടില്ലാത്ത ഒരു ജനതയ്ക്ക് നീ താക്കീത് നല്‍കുവാന്‍ വേണ്ടിയത്രെ ഇത്‌. അവര്‍ ആലോചിച്ച് മനസ്സിലാക്കിയേക്കാം.
\end{malayalam}}
\flushright{\begin{Arabic}
\quranayah[28][47]
\end{Arabic}}
\flushleft{\begin{malayalam}
തങ്ങളുടെ കൈകള്‍ മുന്‍കൂട്ടിചെയ്തു വെച്ചതിന്‍റെ ഫലമായി അവര്‍ക്കു വല്ല വിപത്തും നേരിടുകയും അപ്പോള്‍ ഞങ്ങളുടെ രക്ഷിതാവേ, ഞങ്ങളുടെ അടുത്തേക്ക് നിനക്ക് ഒരു ദൂതനെ അയച്ചുകൂടായിരുന്നോ, എങ്കില്‍ ഞങ്ങള്‍ നിന്‍റെ തെളിവുകള്‍ പിന്തുടരുകയും, ഞങ്ങള്‍ സത്യവിശ്വാസികളുടെ കൂട്ടത്തിലാവുകയും ചെയ്തേനെ എന്ന് അവര്‍ പറയുകയും ചെയ്യില്ലായിരുന്നുവെങ്കില്‍ (നാം നിന്നെ ദൂതനായി അയക്കുമായിരുന്നില്ല.)
\end{malayalam}}
\flushright{\begin{Arabic}
\quranayah[28][48]
\end{Arabic}}
\flushleft{\begin{malayalam}
എന്നാല്‍ നമ്മുടെ പക്കല്‍ നിന്നുള്ള സത്യം (മുഹമ്മദ് നബി മുഖേന) അവര്‍ക്ക് വന്നെത്തിയപ്പോള്‍ അവര്‍ പറയുകയാണ്‌; മൂസായ്ക്ക് നല്‍കപ്പെട്ടത് പോലെയുള്ള ദൃഷ്ടാന്തങ്ങള്‍ ഇവന്ന് നല്‍കപ്പെടാത്തത് എന്താണ് എന്ന്‌. എന്നാല്‍ മുമ്പ് മൂസായ്ക്ക് നല്‍കപ്പെട്ടതില്‍ അവര്‍ അവിശ്വസിക്കുകയുണ്ടായില്ലേ? അവര്‍ പറഞ്ഞു: പരസ്പരം പിന്തുണ നല്‍കിയ രണ്ടു ജാലവിദ്യകളാണിവ. ഞങ്ങള്‍ ഇതൊക്കെ അവിശ്വസിക്കുന്നവരാണ് എന്നും അവര്‍ പറഞ്ഞു.
\end{malayalam}}
\flushright{\begin{Arabic}
\quranayah[28][49]
\end{Arabic}}
\flushleft{\begin{malayalam}
(നബിയേ,) പറയുക: എന്നാല്‍ അവ രണ്ടിനെക്കാളും നേര്‍വഴി കാണിക്കുന്നതായ ഒരു ഗ്രന്ഥം അല്ലാഹുവിന്‍റെ പക്കല്‍ നിന്ന് നിങ്ങള്‍ കൊണ്ട് വരൂ; ഞാനത് പിന്‍പറ്റിക്കൊള്ളാം. നിങ്ങള്‍ സത്യവാന്‍മാരാണെങ്കില്‍.
\end{malayalam}}
\flushright{\begin{Arabic}
\quranayah[28][50]
\end{Arabic}}
\flushleft{\begin{malayalam}
ഇനി നിനക്ക് അവര്‍ ഉത്തരം നല്‍കിയില്ലെങ്കില്‍ തങ്ങളുടെ തന്നിഷ്ടങ്ങളെ മാത്രമാണ് അവര്‍ പിന്തുടരുന്നത് എന്ന് നീ അറിഞ്ഞേക്കുക. അല്ലാഹുവിങ്കല്‍ നിന്നുള്ള യാതൊരു മാര്‍ഗദര്‍ശനവും കൂടാതെ തന്നിഷ്ടത്തെ പിന്തുടര്‍ന്നവനെക്കാള്‍ വഴിപിഴച്ചവന്‍ ആരുണ്ട്‌? അല്ലാഹു അക്രമികളായ ജനങ്ങളെ നേര്‍വഴിയിലാക്കുകയില്ല; തീര്‍ച്ച.
\end{malayalam}}
\flushright{\begin{Arabic}
\quranayah[28][51]
\end{Arabic}}
\flushleft{\begin{malayalam}
അവര്‍ ആലോചിച്ച് മനസ്സിലാക്കേണ്ടതിന്നായി വചനം അവര്‍ക്ക് നാം നിരന്തരമായി എത്തിച്ചുകൊടുത്തിട്ടുണ്ട്‌.
\end{malayalam}}
\flushright{\begin{Arabic}
\quranayah[28][52]
\end{Arabic}}
\flushleft{\begin{malayalam}
ഇതിന് മുമ്പ് നാം ആര്‍ക്ക് വേദഗ്രന്ഥം നല്‍കിയോ അവര്‍ ഇതില്‍ വിശ്വസിക്കുന്നു.
\end{malayalam}}
\flushright{\begin{Arabic}
\quranayah[28][53]
\end{Arabic}}
\flushleft{\begin{malayalam}
ഇതവര്‍ക്ക് ഓതികേള്‍പിക്കപ്പെടുമ്പോള്‍ അവര്‍ പറയും: ഞങ്ങള്‍ ഇതില്‍ വിശ്വസിച്ചിരിക്കുന്നു. തീര്‍ച്ചയായും ഇത് ഞങ്ങളുടെ രക്ഷിതാവിങ്കല്‍ നിന്നുള്ള സത്യമാകുന്നു. ഇതിനു മുമ്പു തന്നെ തീര്‍ച്ചയായും ഞങ്ങള്‍ കീഴ്പെടുന്നവരായിരിക്കുന്നു.
\end{malayalam}}
\flushright{\begin{Arabic}
\quranayah[28][54]
\end{Arabic}}
\flushleft{\begin{malayalam}
അത്തരക്കാര്‍ക്ക് അവര്‍ ക്ഷമിച്ചതിന്‍റെ ഫലമായി അവരുടെ പ്രതിഫലം രണ്ടുമടങ്ങായി നല്‍കപ്പെടുന്നതാണ്‌. അവര്‍ നന്‍മകൊണ്ട് തിന്‍മയെ തടുക്കുകയും, നാം അവര്‍ക്ക് നല്‍കിയിട്ടുള്ളതില്‍ നിന്ന് ചെലവഴിക്കുകയും ചെയ്യും.
\end{malayalam}}
\flushright{\begin{Arabic}
\quranayah[28][55]
\end{Arabic}}
\flushleft{\begin{malayalam}
വ്യര്‍ത്ഥമായ വാക്കുകള്‍ അവര്‍ കേട്ടാല്‍ അതില്‍ നിന്നവര്‍ തിരിഞ്ഞുകളയുകയും ഇപ്രകാരം പറയുകയും ചെയ്യും: ഞങ്ങള്‍ക്കുള്ളത് ഞങ്ങളുടെ കര്‍മ്മങ്ങളാണ്‌. നിങ്ങള്‍ക്കുള്ളത് നിങ്ങളുടെ കര്‍മ്മങ്ങളും. നിങ്ങള്‍ക്കു സലാം. മൂഢന്‍മാരെ ഞങ്ങള്‍ക്ക് ആാ‍വശ്യമില്ല.
\end{malayalam}}
\flushright{\begin{Arabic}
\quranayah[28][56]
\end{Arabic}}
\flushleft{\begin{malayalam}
തീര്‍ച്ചയായും നിനക്ക് ഇഷ്ടപ്പെട്ടവരെ നിനക്ക് നേര്‍വഴിയിലാക്കാനാവില്ല. പക്ഷെ, അല്ലാഹു താന്‍ ഉദ്ദേശിക്കുന്നവരെ നേര്‍വഴിയിലാക്കുന്നു. സന്‍മാര്‍ഗം പ്രാപിക്കുന്നവരെപ്പറ്റി അവന്‍ (അല്ലാഹു) നല്ലവണ്ണം അറിയുന്നവനാകുന്നു.
\end{malayalam}}
\flushright{\begin{Arabic}
\quranayah[28][57]
\end{Arabic}}
\flushleft{\begin{malayalam}
നിന്നോടൊപ്പം ഞങ്ങള്‍ സന്‍മാര്‍ഗം പിന്തുടരുന്ന പക്ഷം ഞങ്ങളുടെ നാട്ടില്‍ നിന്ന് ഞങ്ങള്‍ എടുത്തെറിയപ്പെടും. എന്ന് അവര്‍ പറഞ്ഞു. നിര്‍ഭയമായ ഒരു പവിത്രസങ്കേതം നാം അവര്‍ക്ക് അധീനപ്പെടുത്തികൊടുത്തിട്ടില്ലേ? എല്ലാ വസ്തുക്കളുടെയും ഫലങ്ങള്‍ അവിടേക്ക് ശേഖരിച്ച് കൊണ്ടു വരപ്പെടുന്നു. നമ്മുടെ പക്കല്‍ നിന്നുള്ള ഉപജീവനമത്രെ അത്‌. പക്ഷെ അവരില്‍ അധികപേരും (കാര്യം) മനസ്സിലാക്കുന്നില്ല.
\end{malayalam}}
\flushright{\begin{Arabic}
\quranayah[28][58]
\end{Arabic}}
\flushleft{\begin{malayalam}
സ്വന്തം ജീവിതസുഖത്തില്‍ മതിമറന്ന് അഹങ്കരിച്ച എത്രരാജ്യങ്ങള്‍ നാം നശിപ്പിച്ചിട്ടുണ്ട്‌. അവരുടെ വാസസ്ഥലങ്ങളതാ, അവര്‍ക്കു ശേഷം അപൂര്‍വ്വമായല്ലാതെ അവിടെ ജനവാസമുണ്ടായിട്ടില്ല. നാം തന്നെയായി (അവയുടെ) അവകാശി.
\end{malayalam}}
\flushright{\begin{Arabic}
\quranayah[28][59]
\end{Arabic}}
\flushleft{\begin{malayalam}
രാജ്യങ്ങളുടെ കേന്ദ്രത്തില്‍ നമ്മുടെ ദൃഷ്ടാന്തങ്ങള്‍ ജനങ്ങള്‍ക്ക് ഓതികേള്‍പിക്കുന്ന ഒരു ദൂതനെ അയക്കുന്നത് വരേക്കും നിന്‍റെ രക്ഷിതാവ് ആ രാജ്യങ്ങളെ നശിപ്പിക്കുന്നവനല്ല. രാജ്യക്കാര്‍ അക്രമികളായിരിക്കുമ്പോഴല്ലാതെ നാം രാജ്യങ്ങളെ നശിപ്പിക്കുന്നതുമല്ല.
\end{malayalam}}
\flushright{\begin{Arabic}
\quranayah[28][60]
\end{Arabic}}
\flushleft{\begin{malayalam}
നിങ്ങള്‍ക്ക് വല്ല വസ്തുവും നല്‍കപ്പെട്ടിട്ടുണ്ടെങ്കില്‍ അത് ഐഹികജീവിതത്തിന്‍റെ സുഖഭോഗവും, അതിന്‍റെ അലങ്കാരവും മാത്രമാകുന്നു. അല്ലാഹുവിങ്കലുള്ളത് കൂടുതല്‍ ഉത്തമവും നീണ്ടുനില്‍ക്കുന്നതുമത്രെ. എന്നിരിക്കെ നിങ്ങള്‍ ചിന്തിച്ച് മനസ്സിലാക്കുന്നില്ലേ?
\end{malayalam}}
\flushright{\begin{Arabic}
\quranayah[28][61]
\end{Arabic}}
\flushleft{\begin{malayalam}
അപ്പോള്‍ നാം ഏതൊരുവന് നല്ലൊരു വാഗ്ദാനം നല്‍കുകയും എന്നിട്ട് അവന്‍ അത് (നിറവേറിയതായി) കണ്ടെത്തുകയും ചെയ്തുവോ അവന്‍ ഐഹികജീവിതത്തിന്‍റെ സുഖാനുഭവം നാം അനുഭവിപ്പിക്കുകയും, പിന്നീട് ഉയിര്‍ത്തെഴുന്നേല്‍പിന്‍റെ നാളില്‍ (ശിക്ഷയ്ക്ക്‌) ഹാജരാക്കപ്പെടുന്നവരുടെ കൂട്ടത്തിലാവുകയും ചെയ്തവനെപ്പോലെയാണോ?
\end{malayalam}}
\flushright{\begin{Arabic}
\quranayah[28][62]
\end{Arabic}}
\flushleft{\begin{malayalam}
അവന്‍ (അല്ലാഹു) അവരെ വിളിക്കുകയും, എന്‍റെ പങ്കുകാര്‍ എന്ന് നിങ്ങള്‍ ജല്‍പിച്ചിരുന്നവര്‍ എവിടെ? എന്ന് ചോദിക്കുകയും ചെയ്യുന്ന ദിവസം (ശ്രദ്ധേയമത്രെ.)
\end{malayalam}}
\flushright{\begin{Arabic}
\quranayah[28][63]
\end{Arabic}}
\flushleft{\begin{malayalam}
(ശിക്ഷയെപ്പറ്റിയുള്ള) വാക്ക് ആരുടെ മേല്‍ സ്ഥിരപ്പെട്ടിരിക്കുന്നുവോ അവര്‍ (അന്ന്‌) ഇപ്രകാരം പറയുന്നതാണ്‌: ഞങ്ങളുടെ രക്ഷിതാവേ, ഇവരെയാണ് ഞങ്ങള്‍ വഴിപിഴപ്പിച്ചത്‌. ഞങ്ങള്‍ വഴിപിഴച്ചത് പോലെ അവരെയും വഴിപിഴപ്പിച്ചതാണ്‌. ഞങ്ങള്‍ നിന്‍റെ മുമ്പാകെ ഉത്തരവാദിത്തം ഒഴിഞ്ഞിരിക്കുന്നു. ഞങ്ങളെയല്ല അവര്‍ ആരാധിച്ചിരുന്നത്‌.
\end{malayalam}}
\flushright{\begin{Arabic}
\quranayah[28][64]
\end{Arabic}}
\flushleft{\begin{malayalam}
നിങ്ങള്‍ നിങ്ങളുടെ പങ്കാളികളെ വിളിക്കൂ എന്ന് (ബഹുദൈവവാദികളോട്‌) പറയപ്പെടും. അപ്പോള്‍ ഇവര്‍ അവരെ വിളിക്കും. എന്നാല്‍ അവര്‍ (പങ്കാളികള്‍) ഇവര്‍ക്കു ഉത്തരം നല്‍കുന്നതല്ല. ശിക്ഷ ഇവര്‍ നേരില്‍ കാണുകയും ചെയ്യും. ഇവര്‍ സന്‍മാര്‍ഗം പ്രാപിച്ചിരുന്നെങ്കില്‍.
\end{malayalam}}
\flushright{\begin{Arabic}
\quranayah[28][65]
\end{Arabic}}
\flushleft{\begin{malayalam}
അവന്‍ (അല്ലാഹു) അവരെ വിളിക്കുകയും, ദൈവദൂതന്‍മാര്‍ക്ക് എന്ത് ഉത്തരമാണ് നിങ്ങള്‍ നല്‍കിയത് എന്ന് ചോദിക്കുകയും ചെയ്യുന്ന ദിവസം.(ശ്രദ്ധേയമാകുന്നു.)
\end{malayalam}}
\flushright{\begin{Arabic}
\quranayah[28][66]
\end{Arabic}}
\flushleft{\begin{malayalam}
അന്നത്തെ ദിവസം വര്‍ത്തമാനങ്ങള്‍ അവര്‍ക്ക് അവ്യക്തമായിത്തീരുന്നതാണ്‌. അപ്പോള്‍ അവര്‍ അന്യോന്യം ചോദിച്ചറിയുകയില്ല.
\end{malayalam}}
\flushright{\begin{Arabic}
\quranayah[28][67]
\end{Arabic}}
\flushleft{\begin{malayalam}
എന്നാല്‍ ഖേദിച്ചുമടങ്ങുകയും വിശ്വസിക്കുകയും സല്‍കര്‍മ്മം പ്രവര്‍ത്തിക്കുകയും ചെയ്തവനാരോ, അവന്‍ വിജയികളുടെ കൂട്ടത്തിലായേക്കാം.
\end{malayalam}}
\flushright{\begin{Arabic}
\quranayah[28][68]
\end{Arabic}}
\flushleft{\begin{malayalam}
നിന്‍റെ രക്ഷിതാവ് താന്‍ ഉദ്ദേശിക്കുന്നത് സൃഷ്ടിക്കുകയും, (ഇഷ്ടമുള്ളത്‌) തെരഞ്ഞെടുക്കുകയും ചെയ്യുന്നു. അവര്‍ക്ക് തെരഞ്ഞെടുക്കുവാന്‍ അര്‍ഹതയില്ല. അല്ലാഹു എത്രയോ പരിശുദ്ധനും, അവര്‍ പങ്കുചേര്‍ക്കുന്നതിനെല്ലാം അതീതനുമായിരിക്കുന്നു.
\end{malayalam}}
\flushright{\begin{Arabic}
\quranayah[28][69]
\end{Arabic}}
\flushleft{\begin{malayalam}
അവരുടെ മനസ്സുകള്‍ ഒളിച്ചുവെക്കുന്നതും അവര്‍ പരസ്യമാക്കുന്നതും നിന്‍റെ രക്ഷിതാവ് അറിയുന്നു.
\end{malayalam}}
\flushright{\begin{Arabic}
\quranayah[28][70]
\end{Arabic}}
\flushleft{\begin{malayalam}
അവനത്രെ അല്ലാഹു. അവനല്ലാതെ യാതൊരു ദൈവവുമില്ല. ഈ ലോകത്തും പരലോകത്തും അവന്നാകുന്നു സ്തുതി. അവന്നാണ് വിധികര്‍ത്തൃത്വവും. അവങ്കലേക്ക് തന്നെ നിങ്ങള്‍ മടക്കപ്പെടുന്നതുമാണ്‌.
\end{malayalam}}
\flushright{\begin{Arabic}
\quranayah[28][71]
\end{Arabic}}
\flushleft{\begin{malayalam}
(നബിയേ,) പറയുക: നിങ്ങള്‍ ചിന്തിച്ച് നോക്കിയിട്ടുണ്ടോ? ഉയിര്‍ത്തെഴുന്നേല്‍പിന്‍റെ നാളുവരെ അല്ലാഹു നിങ്ങളുടെ മേല്‍ രാത്രിയെ ശാശ്വതമാക്കിത്തീര്‍ത്തിരുന്നെങ്കില്‍ അല്ലാഹു അല്ലാത്ത ഏതൊരു ദൈവമാണ് നിങ്ങള്‍ക്ക് വെളിച്ചം കൊണ്ട് വന്നു തരിക? എന്നിരിക്കെ നിങ്ങള്‍ കേട്ടുമനസ്സിലാക്കുന്നില്ലേ?
\end{malayalam}}
\flushright{\begin{Arabic}
\quranayah[28][72]
\end{Arabic}}
\flushleft{\begin{malayalam}
പറയുക: നിങ്ങള്‍ ചിന്തിച്ച് നോക്കിയിട്ടുണ്ടോ? ഉയിര്‍ത്തെഴുന്നേല്‍പിന്‍റെ നാളുവരെ അല്ലാഹു നിങ്ങളുടെ മേല്‍ പകലിനെ ശാശ്വതമാക്കിയിരുന്നുവെങ്കില്‍ അല്ലാഹുവല്ലാത്ത ഏതൊരു ദൈവമാണ് നിങ്ങള്‍ക്ക് വിശ്രമിക്കുവാന്‍ ഒരു രാത്രികൊണ്ട് വന്ന് തരിക? എന്നിരിക്കെ നിങ്ങള്‍ കണ്ടുമനസ്സിലാക്കുന്നില്ലേ?
\end{malayalam}}
\flushright{\begin{Arabic}
\quranayah[28][73]
\end{Arabic}}
\flushleft{\begin{malayalam}
അവന്‍റെ കാരുണ്യത്താല്‍ അവന്‍ നിങ്ങള്‍ക്ക് രാവും പകലും ഉണ്ടാക്കിതന്നിരിക്കുന്നു, രാത്രിയില്‍ നിങ്ങള്‍ വിശ്രമിക്കുവാനും (പകല്‍ സമയത്ത്‌) അവന്‍റെ അനുഗ്രഹത്തില്‍ നിന്ന് നിങ്ങള്‍ തേടിക്കൊണ്ട് വരാനും, നിങ്ങള്‍ നന്ദികാണിക്കുവാനും വേണ്ടി.
\end{malayalam}}
\flushright{\begin{Arabic}
\quranayah[28][74]
\end{Arabic}}
\flushleft{\begin{malayalam}
അവന്‍ (അല്ലാഹു) അവരെ വിളിക്കുകയും എന്‍റെ പങ്കാളികളെന്ന് നിങ്ങള്‍ ജല്‍പിച്ചു കൊണ്ടിരുന്നവര്‍ എവിടെ? എന്ന് ചോദിക്കുകയും ചെയ്യുന്ന ദിവസം (ശ്രദ്ധേയമാകുന്നു.)
\end{malayalam}}
\flushright{\begin{Arabic}
\quranayah[28][75]
\end{Arabic}}
\flushleft{\begin{malayalam}
ഓരോ സമുദായത്തില്‍ നിന്നും ഓരോ സാക്ഷിയെ (അന്ന്‌) നാം പുറത്ത് കൊണ്ട് വരുന്നതാണ്‌. എന്നിട്ട് (ആ സമുദായങ്ങളോട്‌) നിങ്ങളുടെ തെളിവ് നിങ്ങള്‍ കൊണ്ട് വരൂ എന്ന് നാം പറയും. ന്യായം അല്ലാഹുവിനാണുള്ളതെന്ന് അപ്പോള്‍ അവര്‍ മനസ്സിലാക്കും. അവര്‍ കെട്ടിച്ചമച്ചു കൊണ്ടിരുന്നതെല്ലാം അവരെ വിട്ടുമാറിപ്പോകുകയും ചെയ്യും.
\end{malayalam}}
\flushright{\begin{Arabic}
\quranayah[28][76]
\end{Arabic}}
\flushleft{\begin{malayalam}
തീര്‍ച്ചയായും ഖാറൂന്‍ മൂസായുടെ ജനതയില്‍ പെട്ടവനായിരുന്നു. എന്നിട്ട് അവന്‍ അവരുടെ നേരെ അതിക്രമം കാണിച്ചു. തന്‍റെ ഖജനാവുകള്‍ ശക്തന്‍മാരായ ഒരു സംഘത്തിനുപോലും ഭാരമാകാന്‍ തക്കവണ്ണമുള്ള നിക്ഷേപങ്ങള്‍ നാം അവന് നല്‍കിയിരുന്നു. അവനോട് അവന്‍റെ ജനത ഇപ്രകാരം പറഞ്ഞ സന്ദര്‍ഭം (ശ്രദ്ധേയമത്രെ:) നീ പുളകം കൊള്ളേണ്ട. പുളകം കൊള്ളുന്നവരെ അല്ലാഹു തീര്‍ച്ചയായും ഇഷ്ടപ്പെടുകയില്ല.
\end{malayalam}}
\flushright{\begin{Arabic}
\quranayah[28][77]
\end{Arabic}}
\flushleft{\begin{malayalam}
അല്ലാഹു നിനക്ക് നല്‍കിയിട്ടുള്ളതിലൂടെ നീ പരലോകവിജയം തേടുക. ഐഹികജീവിതത്തില്‍ നിന്ന് നിനക്കുള്ള ഓഹരി നീ വിസ്മരിക്കുകയും വേണ്ട. അല്ലാഹു നിനക്ക് നന്‍മ ചെയ്തത് പോലെ നീയും നന്‍മചെയ്യുക. നീ നാട്ടില്‍ കുഴപ്പത്തിന് മുതിരരുത്‌. കുഴപ്പമുണ്ടാക്കുന്നവരെ അല്ലാഹു തീര്‍ച്ചയായും ഇഷ്ടപ്പെടുന്നതല്ല.
\end{malayalam}}
\flushright{\begin{Arabic}
\quranayah[28][78]
\end{Arabic}}
\flushleft{\begin{malayalam}
ഖാറൂന്‍ പറഞ്ഞു: എന്‍റെ കൈവശമുള്ള വിദ്യകൊണ്ട് മാത്രമാണ് എനിക്കിതു ലഭിച്ചത്‌. എന്നാല്‍ അവന്നു മുമ്പ് അവനേക്കാള്‍ കടുത്ത ശക്തിയുള്ളവരും, കൂടുതല്‍ സംഘബലമുള്ളവരുമായിരുന്ന തലമുറകളെ അല്ലാഹു നശിപ്പിച്ചിട്ടുണ്ടെന്ന് അവന്‍ മനസ്സിലാക്കിയിട്ടില്ലേ? തങ്ങളുടെ പാപങ്ങളെ പറ്റി കുറ്റവാളികളോട് അന്വേഷിക്കപ്പെടുന്നതല്ല.
\end{malayalam}}
\flushright{\begin{Arabic}
\quranayah[28][79]
\end{Arabic}}
\flushleft{\begin{malayalam}
അങ്ങനെ അവന്‍ ജനമദ്ധ്യത്തിലേക്ക് ആര്‍ഭാടത്തോടെ ഇറങ്ങി പുറപ്പെട്ടു. ഐഹികജീവിതം ലക്ഷ്യമാക്കുന്നവര്‍ അത് കണ്ടിട്ട് ഇപ്രകാരം പറഞ്ഞു: ഖാറൂന് ലഭിച്ചത് പോലുള്ളത് ഞങ്ങള്‍ക്കുമുണ്ടായിരുന്നെങ്കില്‍ എത്ര നന്നായിരുന്നേനെ. തീര്‍ച്ചയായും അവന്‍ വലിയ ഭാഗ്യമുള്ളവന്‍ തന്നെ!
\end{malayalam}}
\flushright{\begin{Arabic}
\quranayah[28][80]
\end{Arabic}}
\flushleft{\begin{malayalam}
ജ്ഞാനം നല്‍കപ്പെട്ടിട്ടുള്ളവര്‍ പറഞ്ഞു: നിങ്ങള്‍ക്ക് നാശം! വിശ്വസിക്കുകയും സല്‍കര്‍മ്മം പ്രവര്‍ത്തിക്കുകയും ചെയ്തിട്ടുള്ളവര്‍ക്ക് അല്ലാഹുവിന്‍റെ പ്രതിഫലമാണ് കൂടുതല്‍ ഉത്തമം. ക്ഷമാശീലമുള്ളവര്‍ക്കല്ലാതെ അത് നല്‍കപ്പെടുകയില്ല.
\end{malayalam}}
\flushright{\begin{Arabic}
\quranayah[28][81]
\end{Arabic}}
\flushleft{\begin{malayalam}
അങ്ങനെ അവനെയും അവന്‍റെ ഭവനത്തേയും നാം ഭൂമിയില്‍ ആഴ്ത്തികളഞ്ഞു. അപ്പോള്‍ അല്ലാഹുവിന് പുറമെ തന്നെ സഹായിക്കുന്ന ഒരു കക്ഷിയും അവന്നുണ്ടായില്ല. അവന്‍ സ്വയം രക്ഷിക്കുന്നവരുടെ കൂട്ടത്തിലുമായില്ല.
\end{malayalam}}
\flushright{\begin{Arabic}
\quranayah[28][82]
\end{Arabic}}
\flushleft{\begin{malayalam}
ഇന്നലെ അവന്‍റെ സ്ഥാനം കൊതിച്ചിരുന്നവര്‍ (ഇന്ന്‌) ഇപ്രകാരം പറയുന്നവരായിത്തീര്‍ന്നു: അഹോ! കഷ്ടം! തന്‍റെ ദാസന്‍മാരില്‍ നിന്ന് താന്‍ ഉദ്ദേശിക്കുന്നവര്‍ക്ക് അല്ലാഹു ഉപജീവനം വിശാലമാക്കികൊടുക്കുകയും, (താന്‍ ഉദ്ദേശിക്കുന്നവര്‍ക്ക് അതു) ഇടുങ്ങിയതാക്കുകയും ചെയ്യുന്നു. ഞങ്ങളോട് അല്ലാഹു ഔദാര്യം കാണിച്ചിരുന്നില്ലെങ്കില്‍ ഞങ്ങളെയും അവന്‍ ആഴ്ത്തിക്കളയുമായിരുന്നു. അഹോ, കഷ്ടം! സത്യനിഷേധികള്‍ വിജയം പ്രാപിക്കുകയില്ല.
\end{malayalam}}
\flushright{\begin{Arabic}
\quranayah[28][83]
\end{Arabic}}
\flushleft{\begin{malayalam}
ഭൂമിയില്‍ ഔന്നത്യമോ കുഴപ്പമോ ആഗ്രഹിക്കാത്തവര്‍ക്കാകുന്നു ആ പാരത്രിക ഭവനം നാം ഏര്‍പെടുത്തികൊടുക്കുന്നത്‌. അന്ത്യഫലം സൂക്ഷ്മത പാലിക്കുന്നവര്‍ക്ക് അനുകൂലമായിരിക്കും.
\end{malayalam}}
\flushright{\begin{Arabic}
\quranayah[28][84]
\end{Arabic}}
\flushleft{\begin{malayalam}
ആര്‍ നന്‍മയും കൊണ്ട് വന്നുവോ അവന്ന് അതിനേക്കാള്‍ ഉത്തമമായതുണ്ടായിരിക്കും. വല്ലവനും തിന്‍മയും കൊണ്ടാണ് വരുന്നതെങ്കില്‍ തിന്‍മ പ്രവര്‍ത്തിച്ചവര്‍ക്ക് തങ്ങള്‍ പ്രവര്‍ത്തിച്ചിരുന്നതിന്‍റെ ഫലമല്ലാതെ നല്‍കപ്പെടുകയില്ല.
\end{malayalam}}
\flushright{\begin{Arabic}
\quranayah[28][85]
\end{Arabic}}
\flushleft{\begin{malayalam}
തീര്‍ച്ചയായും നിനക്ക് ഈ ഖുര്‍ആന്‍ നിയമമായി നല്‍കിയവന്‍ തിരിച്ചെത്തേണ്ട സ്ഥാനത്തേക്ക് നിന്നെ തിരിച്ചു കൊണ്ട് വരിക തന്നെ ചെയ്യും. പറയുക: സന്‍മാര്‍ഗവും കൊണ്ട് വന്നതാരെന്നും, സ്പഷ്ടമായ ദുര്‍മാര്‍ഗത്തിലകപ്പെട്ടത് ആരെന്നും എന്‍റെ രക്ഷിതാവ് നല്ലവണ്ണം അറിയുന്നവനാണ്‌.
\end{malayalam}}
\flushright{\begin{Arabic}
\quranayah[28][86]
\end{Arabic}}
\flushleft{\begin{malayalam}
നിനക്ക് വേദഗ്രന്ഥം നല്‍കപ്പെടണമെന്ന് നീ ആഗ്രഹിച്ചിരുന്നില്ല. പക്ഷെ നിന്‍റെ രക്ഷിതാവിങ്കല്‍ നിന്നുള്ള കാരുണ്യത്താല്‍ (അതു ലഭിച്ചു) ആകയാല്‍ നീ സത്യനിഷേധികള്‍ക്കു സഹായിയായിരിക്കരുത്‌.
\end{malayalam}}
\flushright{\begin{Arabic}
\quranayah[28][87]
\end{Arabic}}
\flushleft{\begin{malayalam}
അല്ലാഹുവിന്‍റെ വചനങ്ങള്‍ നിനക്ക് അവതരിപ്പിക്കപ്പെട്ടതിന് ശേഷം അവര്‍ നിന്നെ അതില്‍ നിന്ന് തടയാതിരിക്കട്ടെ. നിന്‍റെ രക്ഷിതാവിങ്കലേക്ക് നീ ക്ഷണിക്കുക. നീ ബഹുദൈവവിശ്വാസികളുടെ കൂട്ടത്തിലായിപ്പോകരുത്‌.
\end{malayalam}}
\flushright{\begin{Arabic}
\quranayah[28][88]
\end{Arabic}}
\flushleft{\begin{malayalam}
അല്ലാഹുവോടൊപ്പം വേറെ യാതൊരു ദൈവത്തെയും നീ വിളിച്ച് പ്രാര്‍ത്ഥിക്കുകയും ചെയ്യരുത്‌. അവനല്ലാതെ യാതൊരു ദൈവവുമില്ല. അവന്‍റെ തിരുമുഖം ഒഴികെ എല്ലാ വസ്തുക്കളും നാശമടയുന്നതാണ്‌. അവന്നുള്ളതാണ് വിധികര്‍ത്തൃത്വം. അവങ്കലേക്ക് തന്നെ നിങ്ങള്‍ മടക്കപ്പെടുകയും ചെയ്യും.
\end{malayalam}}
\chapter{\textmalayalam{അങ്കബൂത് ( എട്ടുകാലി )}}
\begin{Arabic}
\Huge{\centerline{\basmalah}}\end{Arabic}
\flushright{\begin{Arabic}
\quranayah[29][1]
\end{Arabic}}
\flushleft{\begin{malayalam}
അലിഫ്‌-ലാം-മീം.
\end{malayalam}}
\flushright{\begin{Arabic}
\quranayah[29][2]
\end{Arabic}}
\flushleft{\begin{malayalam}
ഞങ്ങള്‍ വിശ്വസിച്ചിരിക്കുന്നു എന്ന് പറയുന്നത് കൊണ്ട് മാത്രം തങ്ങള്‍ പരീക്ഷണത്തിന് വിധേയരാകാതെ വിട്ടേക്കപ്പെടുമെന്ന് മനുഷ്യര്‍ വിചാരിച്ചിരിക്കയാണോ?
\end{malayalam}}
\flushright{\begin{Arabic}
\quranayah[29][3]
\end{Arabic}}
\flushleft{\begin{malayalam}
അവരുടെ മുമ്പുണ്ടായിരുന്നവരെ നാം പരീക്ഷിച്ചിട്ടുണ്ട്‌. അപ്പോള്‍ സത്യം പറഞ്ഞവര്‍ ആരെന്ന് അല്ലാഹു അറിയുകതന്നെ ചെയ്യും. കള്ളം പറയുന്നവരെയും അവനറിയും.
\end{malayalam}}
\flushright{\begin{Arabic}
\quranayah[29][4]
\end{Arabic}}
\flushleft{\begin{malayalam}
അതല്ല, തിന്‍മചെയ്ത് കൊണ്ടിരിക്കുന്നവര്‍ നമ്മെ മറികടന്ന് കളയാം എന്ന് വിചാരിച്ചിരിക്കുകയാണോ? അവന്‍ തീരുമാനിക്കുന്നത് വളരെ മോശം തന്നെ.
\end{malayalam}}
\flushright{\begin{Arabic}
\quranayah[29][5]
\end{Arabic}}
\flushleft{\begin{malayalam}
വല്ലവനും അല്ലാഹുവുമായി കണ്ടുമുട്ടണമെന്ന് ആഗ്രഹിക്കുന്നുവെങ്കില്‍ തീര്‍ച്ചയായും അല്ലാഹു നിശ്ചയിച്ച അവധി വരിക തന്നെ ചെയ്യും. അവന്‍ എല്ലാം കേള്‍ക്കുന്നവനും അറിയുന്നവനുമത്രെ.
\end{malayalam}}
\flushright{\begin{Arabic}
\quranayah[29][6]
\end{Arabic}}
\flushleft{\begin{malayalam}
വല്ലവനും (അല്ലാഹുവിന്‍റെ മാര്‍ഗത്തില്‍) സമരം ചെയ്യുകയാണെങ്കില്‍ തന്‍റെ സ്വന്തം ഗുണത്തിനായിത്തന്നെയാണ് അവന്‍ സമരം ചെയ്യുന്നത്‌. തീര്‍ച്ചയായും അല്ലാഹു ലോകരെ ആശ്രയിക്കുന്നതില്‍ നിന്ന് മുക്തനത്രെ.
\end{malayalam}}
\flushright{\begin{Arabic}
\quranayah[29][7]
\end{Arabic}}
\flushleft{\begin{malayalam}
വിശ്വസിക്കുകയും സല്‍കര്‍മ്മങ്ങള്‍ പ്രവര്‍ത്തിക്കുകയും ചെയ്തവരാരോ അവരുടെ തിന്‍മകള്‍ അവരില്‍ നിന്ന് നാം മായ്ച്ചുകളയുക തന്നെ ചെയ്യും. അവര്‍ പ്രവര്‍ത്തിച്ച് കൊണ്ടിരുന്നതില്‍ ഏറ്റവും നല്ലതിനുള്ള പ്രതിഫലം അവര്‍ക്ക് നാം നല്‍കുന്നതുമാണ്‌.
\end{malayalam}}
\flushright{\begin{Arabic}
\quranayah[29][8]
\end{Arabic}}
\flushleft{\begin{malayalam}
തന്‍റെ മാതാപിതാക്കളോട് നല്ല നിലയില്‍ വര്‍ത്തിക്കാന്‍ മനുഷ്യനോട് നാം അനുശാസിച്ചിരിക്കുന്നു. നിനക്ക് യാതൊരു അറിവുമില്ലാത്ത ഒന്നിനെ എന്നോട് പങ്കുചേര്‍ക്കുവാന്‍ അവര്‍ (മാതാപിതാക്കള്‍) നിന്നോട് നിര്‍ബന്ധപൂര്‍വ്വം ആവശ്യപ്പെട്ടാല്‍ അവരെ നീ അനുസരിച്ച് പോകരുത്‌. എന്‍റെ അടുത്തേക്കാണ് നിങ്ങളുടെ മടക്കം. അപ്പോള്‍ നിങ്ങള്‍ പ്രവര്‍ത്തിച്ചിരുന്നതിനെപ്പറ്റി ഞാന്‍ നിങ്ങളെ വിവരമറിയിക്കുന്നതാണ്‌.
\end{malayalam}}
\flushright{\begin{Arabic}
\quranayah[29][9]
\end{Arabic}}
\flushleft{\begin{malayalam}
വിശ്വസിക്കുകയും, സല്‍കര്‍മ്മങ്ങളള്‍ പ്രവര്‍ത്തിക്കുകയും ചെയ്തവരാരോ അവരെ നാം സദ്‌വൃത്തരുടെ കൂട്ടത്തില്‍ ഉള്‍പെടുത്തുക തന്നെ ചെയ്യും.
\end{malayalam}}
\flushright{\begin{Arabic}
\quranayah[29][10]
\end{Arabic}}
\flushleft{\begin{malayalam}
ഞങ്ങള്‍ അല്ലാഹുവില്‍ വിശ്വസിച്ചിരിക്കുന്നു. എന്ന് പറയുന്ന ചിലര്‍ മനുഷ്യരുടെ കൂട്ടത്തിലുണ്ട്‌. എന്നാല്‍ അല്ലാഹുവിന്‍റെ മാര്‍ഗത്തില്‍ അവര്‍ പീഡിപ്പിക്കപ്പെട്ടാല്‍ ജനങ്ങളുടെ മര്‍ദ്ദനത്തെ അല്ലാഹുവിന്‍റെ ശിക്ഷയെപ്പോലെ അവര്‍ ഗണിക്കുന്നു. നിന്‍റെ രക്ഷിതാവിങ്കല്‍ നിന്ന് വല്ല സഹായവും വന്നാല്‍ (സത്യവിശ്വാസികളോട്‌) അവര്‍ പറയും: തീര്‍ച്ചയായും ഞങ്ങള്‍ നിങ്ങളോടൊപ്പം തന്നെയായിരുന്നു. ലോകരുടെ ഹൃദയങ്ങളിലുള്ളതിനെപ്പറ്റി അല്ലാഹു നല്ലവണ്ണം അറിയുന്നവനല്ലയോ?
\end{malayalam}}
\flushright{\begin{Arabic}
\quranayah[29][11]
\end{Arabic}}
\flushleft{\begin{malayalam}
വിശ്വസിച്ചിട്ടുള്ളവരാരെന്ന് അല്ലാഹു അറിയുക തന്നെ ചെയ്യും. കപടന്‍മാരെയും അല്ലാഹു അറിയും.
\end{malayalam}}
\flushright{\begin{Arabic}
\quranayah[29][12]
\end{Arabic}}
\flushleft{\begin{malayalam}
നിങ്ങള്‍ ഞങ്ങളുടെ മാര്‍ഗം പിന്തുടരൂ, നിങ്ങളുടെ തെറ്റുകുറ്റങ്ങള്‍ ഞങ്ങള്‍ വഹിച്ചുകൊള്ളാം എന്ന് സത്യനിഷേധികള്‍ സത്യവിശ്വാസികളോട് പറഞ്ഞു. എന്നാല്‍ അവരുടെ തെറ്റുകുറ്റങ്ങളില്‍ നിന്ന് യാതൊന്നും തന്നെ ഇവര്‍ വഹിക്കുന്നതല്ല. തീര്‍ച്ചയായും ഇവര്‍ കള്ളം പറയുന്നവരാകുന്നു.
\end{malayalam}}
\flushright{\begin{Arabic}
\quranayah[29][13]
\end{Arabic}}
\flushleft{\begin{malayalam}
തങ്ങളുടെ പാപഭാരങ്ങളും സ്വന്തം പാപഭാരങ്ങളോടൊപ്പം വെറെയും പാപഭാരങ്ങളും അവര്‍ വഹിക്കേണ്ടിവരും. അവര്‍ കെട്ടിച്ചമച്ചിരുന്നതിനെപ്പറ്റി ഉയിര്‍ത്തെഴുന്നേല്‍പിന്‍റെ നാളില്‍ ചോദ്യം ചെയ്യപ്പെടുന്നതുമാണ്‌.
\end{malayalam}}
\flushright{\begin{Arabic}
\quranayah[29][14]
\end{Arabic}}
\flushleft{\begin{malayalam}
നൂഹിനെ നാം അദ്ദേഹത്തിന്‍റെ ജനതയിലേക്ക് അയക്കുകയുണ്ടായി. അമ്പതുകൊല്ലം ഒഴിച്ചാല്‍ ആയിരം വര്‍ഷം തന്നെ അദ്ദേഹം അവര്‍ക്കിടയില്‍ കഴിച്ചുകൂട്ടി. അങ്ങനെ അവര്‍ അക്രമികളായിരിക്കെ പ്രളയം അവരെ പിടികൂടി.
\end{malayalam}}
\flushright{\begin{Arabic}
\quranayah[29][15]
\end{Arabic}}
\flushleft{\begin{malayalam}
എന്നിട്ട് നാം അദ്ദേഹത്തെയും കപ്പലിലുള്ളവരെയും രക്ഷപ്പെടുത്തുകയും അതിനെ ലോകര്‍ക്ക് ഒരു ദൃഷ്ടാന്തമാക്കുകയും ചെയ്തു.
\end{malayalam}}
\flushright{\begin{Arabic}
\quranayah[29][16]
\end{Arabic}}
\flushleft{\begin{malayalam}
ഇബ്രാഹീമിനെയും (നാം അയച്ചു,) അദ്ദേഹം തന്‍റെ ജനതയോട് ഇപ്രകാരം പറഞ്ഞ സന്ദര്‍ഭം (ശ്രദ്ധേയമത്രെ.): നിങ്ങള്‍ അല്ലാഹുവെ ആരാധിക്കുകയും, അവനെ സൂക്ഷിക്കുകയും ചെയ്യുക. അതാണ് നിങ്ങള്‍ക്ക് ഉത്തമം. നിങ്ങള്‍ മനസ്സിലാക്കുന്നുവെങ്കില്‍.
\end{malayalam}}
\flushright{\begin{Arabic}
\quranayah[29][17]
\end{Arabic}}
\flushleft{\begin{malayalam}
നിങ്ങള്‍ അല്ലാഹുവിന് പുറമെ ചില വിഗ്രഹങ്ങളെ ആരാധിക്കുകയും കള്ളം കെട്ടിയുണ്ടാക്കുകയുമാണ് ചെയ്യുന്നത്‌. അല്ലാഹുവിന് പുറമെ നിങ്ങള്‍ ആരാധിക്കുന്നത് ആരെയാണോ അവര്‍ നിങ്ങള്‍ക്കുള്ള ഉപജീവനം അധീനമാക്കുന്നില്ല. അതിനാല്‍ നിങ്ങള്‍ അല്ലാഹുവിങ്കല്‍ ഉപജീവനം തേടുകയും അവനെ ആരാധിക്കുകയും അവനോട് നന്ദികാണിക്കുകയും ചെയ്യുക. അവങ്കലേക്കാണ് നിങ്ങള്‍ മടക്കപ്പെടുന്നത്‌.
\end{malayalam}}
\flushright{\begin{Arabic}
\quranayah[29][18]
\end{Arabic}}
\flushleft{\begin{malayalam}
നിങ്ങള്‍ നിഷേധിച്ച് തള്ളുകയാണെങ്കില്‍ നിങ്ങള്‍ക്കുമുമ്പുള്ള പല സമുദായങ്ങളും നിഷേധിച്ച് തള്ളുകയുണ്ടായിട്ടുണ്ട്‌. ദൈവദൂതന്‍റെ ബാധ്യത വ്യക്തമായ പ്രബോധനം മാത്രമാകുന്നു.
\end{malayalam}}
\flushright{\begin{Arabic}
\quranayah[29][19]
\end{Arabic}}
\flushleft{\begin{malayalam}
അല്ലാഹു എങ്ങനെ സൃഷ്ടി ആരംഭിക്കുകയും, പിന്നെ അത് ആവര്‍ത്തിക്കുകയും ചെയ്യുന്നു എന്ന് അവര്‍ ചിന്തിച്ച് നോക്കിയില്ലേ? തീര്‍ച്ചയായും അത് അല്ലാഹുവെ സംബന്ധിച്ചിടത്തോളം എളുപ്പമുള്ളതത്രെ.
\end{malayalam}}
\flushright{\begin{Arabic}
\quranayah[29][20]
\end{Arabic}}
\flushleft{\begin{malayalam}
പറയുക: നിങ്ങള്‍ ഭൂമിയിലൂടെ സഞ്ചരിച്ചിട്ട് അവന്‍ എപ്രകാരം സൃഷ്ടി ആരംഭിച്ചിരിക്കുന്നു എന്ന് നോക്കൂ. പിന്നീട് അല്ലാഹു അവസാനം മറ്റൊരിക്കല്‍കൂടി സൃഷ്ടിക്കുന്നതാണ്‌. തീര്‍ച്ചയായും അല്ലാഹു ഏത് കാര്യത്തിനും കഴിവുള്ളവനത്രെ.
\end{malayalam}}
\flushright{\begin{Arabic}
\quranayah[29][21]
\end{Arabic}}
\flushleft{\begin{malayalam}
താന്‍ ഉദ്ദേശിക്കുന്നവരെ അവന്‍ ശിക്ഷിക്കുകയും, താന്‍ ഉദ്ദേശിക്കുന്നവരോട് അവന്‍ കരുണ കാണിക്കുകയും ചെയ്യുന്നു. അവങ്കലേക്ക് തന്നെ നിങ്ങള്‍ തിരിച്ച് കൊണ്ടുവരപ്പെടുകയും ചെയ്യും.
\end{malayalam}}
\flushright{\begin{Arabic}
\quranayah[29][22]
\end{Arabic}}
\flushleft{\begin{malayalam}
ഭൂമിയിലാകട്ടെ ആകാശത്താകട്ടെ നിങ്ങള്‍ക്കു (അവനെ) തോല്‍പിക്കാനാവില്ല. നിങ്ങള്‍ക്കു അല്ലാഹുവിന് പുറമെ ഒരു രക്ഷാധികാരയോ സഹായിയോ ഇല്ല.
\end{malayalam}}
\flushright{\begin{Arabic}
\quranayah[29][23]
\end{Arabic}}
\flushleft{\begin{malayalam}
അല്ലാഹുവിന്‍റെ ദൃഷ്ടാന്തങ്ങളിലും, അവനെ കണ്ടുമുട്ടുന്നതിലും അവിശ്വസിച്ചവരാരോ അവര്‍ എന്‍റെ കാരുണ്യത്തെപറ്റി നിരാശപ്പെട്ടിരിക്കുകയാണ്‌. അക്കൂട്ടര്‍ക്കത്രെ വേദനയേറിയ ശിക്ഷയുള്ളത്‌.
\end{malayalam}}
\flushright{\begin{Arabic}
\quranayah[29][24]
\end{Arabic}}
\flushleft{\begin{malayalam}
നിങ്ങള്‍ അവനെ കൊന്നുകളയുകയോ ചുട്ടെരിക്കുകയോ ചെയ്യൂ. എന്ന് പറഞ്ഞതല്ലാതെ അപ്പോള്‍ അദ്ദേഹത്തിന്‍റെ (ഇബ്രാഹീമിന്‍റെ) ജനത മറുപടിയൊന്നും നല്‍കിയില്ല. എന്നാല്‍ അല്ലാഹു അദ്ദേഹത്തെ അഗ്നിയില്‍ നിന്ന് രക്ഷിച്ചു. വിശ്വസിക്കുന്ന ജനങ്ങള്‍ക്ക് തീര്‍ച്ചയായും അതില്‍ ദൃഷ്ടാന്തങ്ങളുണ്ട്‌.
\end{malayalam}}
\flushright{\begin{Arabic}
\quranayah[29][25]
\end{Arabic}}
\flushleft{\begin{malayalam}
അദ്ദേഹം (ഇബ്രാഹീം) പറഞ്ഞു: അല്ലാഹുവിന് പുറമെ നിങ്ങള്‍ വിഗ്രഹങ്ങളെ സ്വീകരിച്ചിട്ടുള്ളത് ഐഹികജീവിതത്തില്‍ നിങ്ങള്‍ തമ്മിലുള്ള സ്നേഹബന്ധത്തിന്‍റെ പേരില്‍ മാത്രമാകുന്നു. പിന്നീട് ഉയിര്‍ത്തെഴുന്നേല്‍പിന്‍റെ നാളില്‍ നിങ്ങളില്‍ ചിലര്‍ ചിലരെ നിഷേധിക്കുകയും, ചിലര്‍ ചിലരെ ശപിക്കുകയും ചെയ്യുന്നതാണ്‌. നിങ്ങളുടെ സങ്കേതം നരകമായിരിക്കുകയും ചെയ്യും. നിങ്ങള്‍ക്ക് സഹായികളാരുമുണ്ടാകുകയില്ല.
\end{malayalam}}
\flushright{\begin{Arabic}
\quranayah[29][26]
\end{Arabic}}
\flushleft{\begin{malayalam}
അപ്പോള്‍ ലൂത്വ് അദ്ദേഹത്തില്‍ വിശ്വസിച്ചു. അദ്ദേഹം (ഇബ്രാഹീം) പറഞ്ഞു: തീര്‍ച്ചയായും ഞാന്‍ സ്വദേശം വെടിഞ്ഞ് എന്‍റെ രക്ഷിതാവിങ്കലേക്ക് പോകുകയാണ്‌. തീര്‍ച്ചയായും അവനാകുന്നു പ്രതാപിയും യുക്തിമാനും.
\end{malayalam}}
\flushright{\begin{Arabic}
\quranayah[29][27]
\end{Arabic}}
\flushleft{\begin{malayalam}
അദ്ദേഹത്തിന് (പുത്രന്‍) ഇഷാഖിനെയും (പൌത്രന്‍) യഅ്ഖൂബിനെയും നാം പ്രദാനം ചെയ്യുകയുണ്ടായി. അദ്ദേഹത്തിന്‍റെ സന്തതിപരമ്പരയില്‍ പ്രവാചകത്വവും വേദവും നാം നല്‍കുകയും ചെയ്തു. ഇഹലോകത്ത് അദ്ദേഹത്തിന് നാം പ്രതിഫലം നല്‍കിയിട്ടുണ്ട്‌. പരലോകത്ത് തീര്‍ച്ചയായും അദ്ദേഹം സജ്ജനങ്ങളുടെ കൂട്ടത്തിലായിരിക്കുകയും ചെയ്യും.
\end{malayalam}}
\flushright{\begin{Arabic}
\quranayah[29][28]
\end{Arabic}}
\flushleft{\begin{malayalam}
ലൂത്വിനെയും (ദൂതനായി അയച്ചു) തന്‍റെ ജനതയോട് അദ്ദേഹം ഇപ്രകാരം പറഞ്ഞ സന്ദര്‍ഭം (ശ്രദ്ധേയമാകുന്നു:) തീര്‍ച്ചയായും നിങ്ങള്‍ നീചകൃത്യമാണ് ചെയ്തു കൊണ്ടിരിക്കുന്നത്‌. നിങ്ങള്‍ക്കു മുമ്പ് ലോകരില്‍ ഒരാളും അതുചെയ്യുകയുണ്ടായിട്ടില്ല.
\end{malayalam}}
\flushright{\begin{Arabic}
\quranayah[29][29]
\end{Arabic}}
\flushleft{\begin{malayalam}
നിങ്ങള്‍ കാമനിവൃത്തിക്കായി പുരുഷന്‍മാരുടെ അടുത്ത് ചെല്ലുകയും (പ്രകൃതിപരമായ) മാര്‍ഗം ലംഘിക്കുകയും നിങ്ങളുടെ സദസ്സില്‍ വെച്ച് നിഷിദ്ധവൃത്തി ചെയ്യുകയുമാണോ? അപ്പോള്‍ അദ്ദേഹത്തിന്‍റെ ജനത മറുപടിയൊന്നും നല്‍കുകയുണ്ടായില്ല; നീ സത്യവാന്‍മാരുടെ കൂട്ടത്തിലാണെങ്കില്‍ ഞങ്ങള്‍ക്ക് അല്ലാഹുവിന്‍റെ ശിക്ഷ നീ കൊണ്ടുവാ എന്ന് അവര്‍ പറഞ്ഞതല്ലാതെ.
\end{malayalam}}
\flushright{\begin{Arabic}
\quranayah[29][30]
\end{Arabic}}
\flushleft{\begin{malayalam}
അദ്ദേഹം പറഞ്ഞു: എന്‍റെ രക്ഷിതാവേ, കുഴപ്പക്കാരായ ഈ ജനതക്കെതിരില്‍ എന്നെ നീ സഹായിക്കണമേ.
\end{malayalam}}
\flushright{\begin{Arabic}
\quranayah[29][31]
\end{Arabic}}
\flushleft{\begin{malayalam}
നമ്മുടെ ദൂതന്‍മാര്‍ ഇബ്രാഹീമിന്‍റെ അടുത്ത് സന്തോഷവാര്‍ത്തയും കൊണ്ട് ചെന്നപ്പോള്‍ അവര്‍ പറഞ്ഞു: തീര്‍ച്ചയായും ഞങ്ങള്‍ ഈ നാട്ടുകാരെ നശിപ്പിക്കാന്‍ പോകുന്നവരാകുന്നു. തീര്‍ച്ചയായും ഈ നാട്ടുകാര്‍ അക്രമികളായിരിക്കുന്നു.
\end{malayalam}}
\flushright{\begin{Arabic}
\quranayah[29][32]
\end{Arabic}}
\flushleft{\begin{malayalam}
ഇബ്രാഹീം പറഞ്ഞു: ലൂത്വ് അവിടെ ഉണ്ടല്ലോ. അവര്‍ (ദൂതന്‍മാര്‍) പറഞ്ഞു: അവിടെയുള്ളവരെപ്പറ്റി നമുക്ക് നല്ലവണ്ണം അറിയാം. അദ്ദേഹത്തെയും അദ്ദേഹത്തിന്‍റെ കുടുംബത്തെയും നാം രക്ഷപ്പെടുത്തുക തന്നെ ചെയ്യും. അദ്ദേഹത്തിന്‍റെ ഭാര്യയൊഴികെ. അവള്‍ ശിക്ഷയില്‍ അകപ്പെടുന്നവരുടെ കൂട്ടത്തിലായിരിക്കുന്നു.
\end{malayalam}}
\flushright{\begin{Arabic}
\quranayah[29][33]
\end{Arabic}}
\flushleft{\begin{malayalam}
നമ്മുടെ ദൂതന്‍മാര്‍ ലൂത്വിന്‍റെ അടുത്ത് ചെന്നപ്പോള്‍ അവരുടെ കാര്യത്തില്‍ അദ്ദേഹം ദുഃഖിതനാകുകയും, അവരുടെ കാര്യത്തില്‍ അദ്ദേഹത്തിന് മനഃപ്രയാസമുണ്ടാകുകയും ചെയ്തു. അവര്‍ പറഞ്ഞു: താങ്കള്‍ ഭയപ്പെടുകയോ ദുഃഖിക്കുകയോ വേണ്ട. തങ്കളെയും കുടുംബത്തെയും തീര്‍ച്ചയായും ഞങ്ങള്‍ രക്ഷപ്പെടുത്തുന്നതാണ്‌. താങ്കളുടെ ഭാര്യ ഒഴികെ. അവള്‍ ശിക്ഷയില്‍ അകപ്പെടുന്നവരുടെ കൂട്ടത്തിലായിരിക്കുന്നു.
\end{malayalam}}
\flushright{\begin{Arabic}
\quranayah[29][34]
\end{Arabic}}
\flushleft{\begin{malayalam}
ഈ നാട്ടുകാരുടെ മേല്‍ അവര്‍ ചെയ്തുകൊണ്ടിരുന്ന അധര്‍മ്മത്തിന്‍റെ ഫലമായി ആകാശത്തു നിന്ന് ഞങ്ങള്‍ ഒരു ശിക്ഷ ഇറക്കുന്നതാണ്‌.
\end{malayalam}}
\flushright{\begin{Arabic}
\quranayah[29][35]
\end{Arabic}}
\flushleft{\begin{malayalam}
തീര്‍ച്ചയായും അതില്‍ ചിന്തിക്കുന്ന ആളുകള്‍ക്ക് വ്യക്തമായ ഒരു ദൃഷ്ടാന്തം നാം അവശേഷിപ്പിച്ചിട്ടുണ്ട്‌.
\end{malayalam}}
\flushright{\begin{Arabic}
\quranayah[29][36]
\end{Arabic}}
\flushleft{\begin{malayalam}
മദ്‌യങ്കാരിലേക്ക് അവരുടെ സഹോദരനായ ശുഐബിനേയും (നാം അയച്ചു) അദ്ദേഹം പറഞ്ഞു: എന്‍റെ ജനങ്ങളേ, നിങ്ങള്‍ അല്ലാഹുവെ ആരാധിക്കുകയും, അന്ത്യദിനത്തെ പ്രതീക്ഷിക്കുകയും ചെയ്യുവിന്‍. നാശകാരികളായിക്കൊണ്ട് നിങ്ങള്‍ ഭൂമിയില്‍ കുഴപ്പമുണ്ടാക്കരുത്‌.
\end{malayalam}}
\flushright{\begin{Arabic}
\quranayah[29][37]
\end{Arabic}}
\flushleft{\begin{malayalam}
അപ്പോള്‍ അവര്‍ അദ്ദേഹത്തെ നിഷേധിച്ചുതള്ളി. അതിനാല്‍ ഭൂകമ്പം അവരെ പിടികൂടി. അങ്ങനെ അവര്‍ തങ്ങളുടെ വീടുകളില്‍ വീണടിഞ്ഞവരായിത്തീര്‍ന്നു.
\end{malayalam}}
\flushright{\begin{Arabic}
\quranayah[29][38]
\end{Arabic}}
\flushleft{\begin{malayalam}
ആദ്‌, ഥമൂദ് സമുദായങ്ങളെയും (നാം നശിപ്പിക്കുകയുണ്ടായി.) അവരുടെ വാസസ്ഥലങ്ങളില്‍ നിന്ന് നിങ്ങള്‍ക്കത് വ്യക്തമായി മനസ്സിലായിട്ടുണ്ട്‌. പിശാച് അവര്‍ക്ക് അവരുടെ പ്രവര്‍ത്തനങ്ങള്‍ ഭംഗിയായി തോന്നിക്കുകയും അവരെ ശരിയായ മാര്‍ഗത്തില്‍ നിന്ന് തടയുകയും ചെയ്തു. (വാസ്തവത്തില്‍) അവര്‍ കണ്ടറിയുവാന്‍ കഴിവുള്ളരായിരുന്നു.
\end{malayalam}}
\flushright{\begin{Arabic}
\quranayah[29][39]
\end{Arabic}}
\flushleft{\begin{malayalam}
ഖാറൂനെയും, ഫിര്‍ഔനെയും ഹാമാനെയും (നാം നശിപ്പിച്ചു.) വ്യക്തമായ തെളിവുകളും കൊണ്ട് മൂസാ അവരുടെ അടുത്ത് ചെല്ലുകയുണ്ടായി. അപ്പോള്‍ അവര്‍ നാട്ടില്‍ അഹങ്കരിച്ച് നടന്നു. അവര്‍ (നമ്മെ) മറികടക്കുന്നവരായില്ല.
\end{malayalam}}
\flushright{\begin{Arabic}
\quranayah[29][40]
\end{Arabic}}
\flushleft{\begin{malayalam}
അങ്ങനെ എല്ലാവരെയും അവരവരുടെ കുറ്റത്തിന് നാം പിടികൂടി. അവരില്‍ ചിലരുടെ നേരെ നാം ചരല്‍കാറ്റ് അയക്കുകയാണ് ചെയ്തത്‌. അവരില്‍ ചിലരെ ഘോരശബ്ദം പിടികൂടി. അവരില്‍ ചിലരെ നാം ഭൂമിയില്‍ ആഴ്ത്തികളഞ്ഞു. അവരില്‍ ചിലരെ നാം മുക്കിനശിപ്പിച്ചു.അല്ലാഹു അവരോട് അക്രമം ചെയ്യുകയായിരുന്നില്ല. പക്ഷെ അവര്‍ അവരോട് തന്നെ അക്രമം ചെയ്യുകയായിരുന്നു.
\end{malayalam}}
\flushright{\begin{Arabic}
\quranayah[29][41]
\end{Arabic}}
\flushleft{\begin{malayalam}
അല്ലാഹുവിന് പുറമെ വല്ല രക്ഷാധികാരികളെയും സ്വീകരിച്ചവരുടെ ഉപമ എട്ടുകാലിയുടേത് പോലെയാകുന്നു. അത് ഒരു വീടുണ്ടാക്കി. വീടുകളില്‍ വെച്ച് ഏറ്റവും ദുര്‍ബലമായത് എട്ടുകാലിയുടെ വീട് തന്നെ. അവര്‍ കാര്യം മനസ്സിലാക്കിയിരുന്നുവെങ്കില്‍!
\end{malayalam}}
\flushright{\begin{Arabic}
\quranayah[29][42]
\end{Arabic}}
\flushleft{\begin{malayalam}
തനിക്ക് പുറമെ അവര്‍ വിളിച്ച് പ്രാര്‍ത്ഥിക്കുന്ന ഏതൊരു വസ്തുവെയും തീര്‍ച്ചയായും അല്ലാഹു അറിയുന്നു. അവനാകുന്നു പ്രതാപിയും യുക്തിമാനും.
\end{malayalam}}
\flushright{\begin{Arabic}
\quranayah[29][43]
\end{Arabic}}
\flushleft{\begin{malayalam}
ആ ഉപമകള്‍ നാം മനുഷ്യര്‍ക്ക് വേണ്ടി വിവരിക്കുകയാണ്‌. അറിവുള്ളവരല്ലാതെ അവയെപ്പറ്റി ചിന്തിച്ച് മനസ്സിലാക്കുകയില്ല.
\end{malayalam}}
\flushright{\begin{Arabic}
\quranayah[29][44]
\end{Arabic}}
\flushleft{\begin{malayalam}
ആകാശങ്ങളും ഭൂമിയും മുറപ്രകാരം അല്ലാഹു സൃഷ്ടിച്ചിരിക്കുന്നു. തീര്‍ച്ചയായും അതില്‍ സത്യവിശ്വാസികള്‍ക്ക് ദൃഷ്ടാന്തമുണ്ട്‌.
\end{malayalam}}
\flushright{\begin{Arabic}
\quranayah[29][45]
\end{Arabic}}
\flushleft{\begin{malayalam}
(നബിയേ,) വേദഗ്രന്ഥത്തില്‍ നിന്നും നിനക്ക് ബോധനം നല്‍കപ്പെട്ടത് ഓതികേള്‍പിക്കുകയും, നമസ്കാരം മുറപോലെ നിര്‍വഹിക്കുകയും ചെയ്യുക. തീര്‍ച്ചയായും നമസ്കാരം നീചവൃത്തിയില്‍ നിന്നും നിഷിദ്ധകര്‍മ്മത്തില്‍ നിന്നും തടയുന്നു. അല്ലാഹുവെ ഓര്‍മിക്കുക എന്നത് ഏറ്റവും മഹത്തായ കാര്യം തന്നെയാകുന്നു. നിങ്ങള്‍ പ്രവര്‍ത്തിക്കുന്നതെന്തോ അത് അല്ലാഹു അറിയുന്നു.
\end{malayalam}}
\flushright{\begin{Arabic}
\quranayah[29][46]
\end{Arabic}}
\flushleft{\begin{malayalam}
വേദക്കാരോട് ഏറ്റവും നല്ല രീതിയിലല്ലാതെ നിങ്ങള്‍ സംവാദം നടത്തരുത്‌- അവരില്‍ നിന്ന് അക്രമം പ്രവര്‍ത്തിച്ചവരോടൊഴികെ. നിങ്ങള്‍ (അവരോട്‌) പറയുക: ഞങ്ങള്‍ക്ക് അവതരിപ്പിക്കപ്പെട്ടതിലും നിങ്ങള്‍ക്ക് അവതരിപ്പിക്കപ്പെട്ടതിലും ഞങ്ങള്‍ വിശ്വസിച്ചിരിക്കുന്നു. ഞങ്ങളുടെ ദൈവവും നിങ്ങളുടെ ദൈവവും ഒരുവനാകുന്നു. ഞങ്ങള്‍ അവന് കീഴ്പെട്ടവരുമാകുന്നു.
\end{malayalam}}
\flushright{\begin{Arabic}
\quranayah[29][47]
\end{Arabic}}
\flushleft{\begin{malayalam}
അതുപോലെ നിനക്കും നാം വേദഗ്രന്ഥം അവതരിപ്പിച്ച് തന്നിരിക്കുന്നു. അപ്പോള്‍ നാം (മുമ്പ്‌) വേദഗ്രന്ഥം നല്‍കിയിട്ടുള്ളവര്‍ ഇതില്‍ വിശ്വസിക്കുന്നതാണ്‌. ഈ കൂട്ടരിലും അതില്‍ വിശ്വസിക്കുന്നവരുണ്ട്‌. അവിശ്വാസികളല്ലാതെ നമ്മുടെ ദൃഷ്ടാന്തങ്ങളെ നിഷേധിക്കുകയില്ല.
\end{malayalam}}
\flushright{\begin{Arabic}
\quranayah[29][48]
\end{Arabic}}
\flushleft{\begin{malayalam}
ഇതിന് മുമ്പ് നീ വല്ല ഗ്രന്ഥവും പാരായണം ചെയ്യുകയോ, നിന്‍റെ വലതുകൈ കൊണ്ട് അത് എഴുതുകയോ ചെയ്തിരുന്നില്ല. അങ്ങനെയാണെങ്കില്‍ ഈ സത്യനിഷേധികള്‍ക്കു സംശയിക്കാമായിരുന്നു.
\end{malayalam}}
\flushright{\begin{Arabic}
\quranayah[29][49]
\end{Arabic}}
\flushleft{\begin{malayalam}
എന്നാല്‍ ജ്ഞാനം നല്‍കപ്പെട്ടവരുടെ ഹൃദയങ്ങളില്‍ അത് സുവ്യക്തമായ ദൃഷ്ടാന്തങ്ങളാകുന്നു. അക്രമികളല്ലാതെ നമ്മുടെ ദൃഷ്ടാന്തങ്ങളെ നിഷേധിക്കുകയില്ല.
\end{malayalam}}
\flushright{\begin{Arabic}
\quranayah[29][50]
\end{Arabic}}
\flushleft{\begin{malayalam}
അവര്‍ (അവിശ്വാസികള്‍) പറഞ്ഞു: ഇവന്നു ഇവന്‍റെ രക്ഷിതാവിങ്കല്‍ നിന്ന് എന്തുകൊണ്ട് ദൃഷ്ടാന്തങ്ങള്‍ ഇറക്കികൊടുക്കപ്പെടുന്നില്ല? നീ പറയുക: ദൃഷ്ടാന്തങ്ങള്‍ അല്ലാഹുവിങ്കല്‍ മാത്രമാകുന്നു. ഞാന്‍ വ്യക്തമായഒരു താക്കീതുകാരന്‍ മാത്രമാകുന്നു.
\end{malayalam}}
\flushright{\begin{Arabic}
\quranayah[29][51]
\end{Arabic}}
\flushleft{\begin{malayalam}
നാം നിനക്ക് വേദഗ്രന്ഥം ഇറക്കിത്തന്നിരിക്കുന്നു. എന്നതു തന്നെ അവര്‍ക്കു (തെളിവിന്‌) മതിയായിട്ടില്ലേ? അതവര്‍ക്ക് ഓതികേള്‍പിക്കപ്പെട്ട് കൊണ്ടിരിക്കുന്നു. വിശ്വസിക്കുന്ന ജനങ്ങള്‍ക്ക് തീര്‍ച്ചയായും അതില്‍ അനുഗ്രഹവും ഉല്‍ബോധനവുമുണ്ട്‌.
\end{malayalam}}
\flushright{\begin{Arabic}
\quranayah[29][52]
\end{Arabic}}
\flushleft{\begin{malayalam}
(നബിയേ,) പറയുക: എനിക്കും നിങ്ങള്‍ക്കുമിടയില്‍ സാക്ഷിയായി അല്ലാഹു മതി. ആകാശങ്ങളിലും ഭൂമിയിലുമുള്ളത് അവന്‍ അറിയുന്നു. അസത്യത്തില്‍ വിശ്വസിക്കുകയും അല്ലാഹുവില്‍ അവിശ്വസിക്കുകയും ചെയ്തവരാരോ അവര്‍ തന്നെയാണ് നഷ്ടം പറ്റിയവര്‍.
\end{malayalam}}
\flushright{\begin{Arabic}
\quranayah[29][53]
\end{Arabic}}
\flushleft{\begin{malayalam}
ശിക്ഷയുടെ കാര്യത്തില്‍ അര്‍ നിന്നോട് ധൃതികൂട്ടുന്നു. നിര്‍ണയിക്കപ്പെട്ട ഒരു അവധി ഉണ്ടായിരുന്നില്ലെങ്കില്‍ അവര്‍ക്ക് ശിക്ഷ വന്നുകഴിഞ്ഞിട്ടുണ്ടാകുമായിരുന്നു. അവര്‍ ഓര്‍ക്കാതിരിക്കെ പെട്ടെന്ന് അതവര്‍ക്ക് വന്നെത്തുക തന്നെ ചെയ്യും.
\end{malayalam}}
\flushright{\begin{Arabic}
\quranayah[29][54]
\end{Arabic}}
\flushleft{\begin{malayalam}
ശിക്ഷയുടെ കാര്യത്തില്‍ അവര്‍ നിന്നോട് ധൃതികൂട്ടികൊണ്ടിരിക്കുന്നു. തീര്‍ച്ചയായും നരകം സത്യനിഷേധികളെ വലയം ചെയ്യുന്നതാകുന്നു.
\end{malayalam}}
\flushright{\begin{Arabic}
\quranayah[29][55]
\end{Arabic}}
\flushleft{\begin{malayalam}
അവരുടെ മുകള്‍ഭാഗത്തു നിന്നും അവരുടെ കാലുകള്‍ക്കിടയില്‍ നിന്നും ശിക്ഷ അവരെ മൂടിക്കളയുന്ന ദിവസത്തില്‍. (അന്ന്‌) അവന്‍ (അല്ലാഹു) പറയും: നിങ്ങള്‍ പ്രവര്‍ത്തിച്ചിരുന്നതിന്‍റെ ഫലം നിങ്ങള്‍ ആസ്വദിച്ച് കൊള്ളുക.
\end{malayalam}}
\flushright{\begin{Arabic}
\quranayah[29][56]
\end{Arabic}}
\flushleft{\begin{malayalam}
വിശ്വസിച്ചവരായ എന്‍റെ ദാസന്‍മാരെ, തീര്‍ച്ചയായും എന്‍റെ ഭൂമി വിശാലമാകുന്നു. അതിനാല്‍ എന്നെ മാത്രം നിങ്ങള്‍ ആരാധിക്കുവിന്‍.
\end{malayalam}}
\flushright{\begin{Arabic}
\quranayah[29][57]
\end{Arabic}}
\flushleft{\begin{malayalam}
ഏതൊരാളും മരണത്തെ ആസ്വദിക്കുന്നതാണ്‌. പിന്നീട് നമ്മുടെ അടുക്കലേക്ക് തന്നെ നിങ്ങള്‍ മടക്കപ്പെടുകയും ചെയ്യും.
\end{malayalam}}
\flushright{\begin{Arabic}
\quranayah[29][58]
\end{Arabic}}
\flushleft{\begin{malayalam}
വിശ്വസിക്കുകയും സല്‍കര്‍മ്മങ്ങള്‍ പ്രവര്‍ത്തിക്കുകയും ചെയ്തവരാരോ അവര്‍ക്ക് നാം സ്വര്‍ഗത്തില്‍ താഴ്ഭാഗത്ത് കൂടി നദികള്‍ ഒഴുകുന്ന ഉന്നത സൌധങ്ങളില്‍ താമസസൌകര്യം നല്‍കുന്നതാണ്‌. അവര്‍ അവിടെ നിത്യവാസികളായിരിക്കും. പ്രവര്‍ത്തിക്കുന്നവര്‍ക്കുള്ള പ്രതിഫലം എത്ര വിശിഷ്ടം!
\end{malayalam}}
\flushright{\begin{Arabic}
\quranayah[29][59]
\end{Arabic}}
\flushleft{\begin{malayalam}
ക്ഷമ കൈക്കൊള്ളുകയും, തങ്ങളുടെ രക്ഷിതാവിനെ ഭരമേല്‍പിച്ചു കൊണ്ടിരിക്കുകയും ചെയ്തവരത്രെ അവര്‍.
\end{malayalam}}
\flushright{\begin{Arabic}
\quranayah[29][60]
\end{Arabic}}
\flushleft{\begin{malayalam}
സ്വന്തം ഉപജീവനത്തിന്‍റെ ചുമതല വഹിക്കാത്ത എത്രയെത്ര ജീവികളുണ്ട്‌. അല്ലാഹുവാണ് അവയ്ക്കും നിങ്ങള്‍ക്കും ഉപജീവനം നല്‍കുന്നത്‌. അവനാണ് എല്ലാം കേള്‍ക്കുകയും അറിയുകയും ചെയ്യുന്നവന്‍.
\end{malayalam}}
\flushright{\begin{Arabic}
\quranayah[29][61]
\end{Arabic}}
\flushleft{\begin{malayalam}
ആകാശങ്ങളും ഭൂമിയും സൃഷ്ടിക്കുകയും സൂര്യനെയും ചന്ദ്രനെയും കീഴ്പെടുത്തുകയും ചെയ്തത് ആരാണെന്ന് നീ അവരോട് (ബഹുദൈവവിശ്വാസികളോട്‌) ചോദിക്കുന്ന പക്ഷം തീര്‍ച്ചയായും അവര്‍ പറയും: അല്ലാഹുവാണെന്ന്‌. അപ്പോള്‍ എങ്ങനെയാണ് അവര്‍ (സത്യത്തില്‍ നിന്ന്‌) തെറ്റിക്കപ്പെടുന്നത്‌?
\end{malayalam}}
\flushright{\begin{Arabic}
\quranayah[29][62]
\end{Arabic}}
\flushleft{\begin{malayalam}
അല്ലാഹുവാണ് തന്‍റെ ദാസന്‍മാരില്‍ നിന്ന് താന്‍ ഉദ്ദേശിക്കുന്നവര്‍ക്ക് ഉപജീവനമാര്‍ഗം വിശാലമാക്കുന്നതും, താന്‍ ഉദ്ദേശിക്കുന്നവര്‍ക്ക് അതു ഇടുങ്ങിയതാക്കുന്നതും. തീര്‍ച്ചയായും അല്ലാഹു ഏത് കാര്യത്തെപ്പറ്റിയും അറിവുള്ളവനത്രെ.
\end{malayalam}}
\flushright{\begin{Arabic}
\quranayah[29][63]
\end{Arabic}}
\flushleft{\begin{malayalam}
ആകാശത്ത് നിന്ന് വെള്ളം ചൊരിയുകയും, ഭൂമി നിര്‍ജീവമായി കിടന്നതിനു ശേഷം അതുമൂലം അതിന് ജീവന്‍ നല്‍കുകയും ചെയ്താരെന്ന് നീ അവരോട് ചോദിക്കുന്ന പക്ഷം തീര്‍ച്ചയായും അവര്‍ പറയും; അല്ലാഹുവാണെന്ന്‌. പറയുക: അല്ലാഹുവിന് സ്തുതി! പക്ഷെ അവരില്‍ അധികപേരും ചിന്തിച്ച് മനസ്സിലാക്കുന്നില്ല.
\end{malayalam}}
\flushright{\begin{Arabic}
\quranayah[29][64]
\end{Arabic}}
\flushleft{\begin{malayalam}
ഈ ഐഹികജീവിതം വിനോദവും കളിയുമല്ലാതെ മറ്റൊന്നുമല്ല. തീര്‍ച്ചയായും പരലോകം തന്നെയാണ് യഥാര്‍ത്ഥ ജീവിതം. അവര്‍ മനസ്സിലാക്കിയിരുന്നെങ്കില്‍!
\end{malayalam}}
\flushright{\begin{Arabic}
\quranayah[29][65]
\end{Arabic}}
\flushleft{\begin{malayalam}
എന്നാല്‍ അവര്‍ (ബഹുദൈവാരാധകര്‍) കപ്പലില്‍ കയറിയാല്‍ കീഴ്‌വണക്കം അല്ലാഹുവിന് നിഷ്കളങ്കമാക്കികൊണ്ട് അവനെ വിളിച്ച് പ്രാര്‍ത്ഥിക്കും. എന്നിട്ട് അവരെ അവന്‍ കരയിലേക്ക് രക്ഷപ്പെടുത്തിയപ്പോഴോ അവരതാ (അവനോട്‌) പങ്കുചേര്‍ക്കുന്നു.
\end{malayalam}}
\flushright{\begin{Arabic}
\quranayah[29][66]
\end{Arabic}}
\flushleft{\begin{malayalam}
അങ്ങനെ നാം അവര്‍ക്ക് നല്‍കിയതില്‍ അവര്‍ നന്ദികേട് കാണിക്കുകയും, അവര്‍ സുഖം അനുഭവിക്കുകയും ചെയ്യുന്നവരായിത്തീര്‍ന്നു. എന്നാല്‍ വഴിയെ അവര്‍ (കാര്യം) മനസ്സിലാക്കികൊള്ളും.
\end{malayalam}}
\flushright{\begin{Arabic}
\quranayah[29][67]
\end{Arabic}}
\flushleft{\begin{malayalam}
നിര്‍ഭയമായ ഒരു പവിത്രസങ്കേതം നാം ഏര്‍പെടുത്തിയിരിക്കുന്നു എന്ന് അവര്‍ കണ്ടില്ലേ? അവരുടെ ചുറ്റുഭാഗത്തു നിന്നാകട്ടെ ആളുകള്‍ റാഞ്ചിയെടുക്കപ്പെടുന്നു. എന്നിട്ടും അസത്യത്തില്‍ അവര്‍ വിശ്വസിക്കുകയും അല്ലാഹുവിന്‍റെ അനുഗ്രഹത്തോട് അവര്‍ നന്ദികേട് കാണിക്കുകയുമാണോ?
\end{malayalam}}
\flushright{\begin{Arabic}
\quranayah[29][68]
\end{Arabic}}
\flushleft{\begin{malayalam}
അല്ലാഹുവിന്‍റെ പേരില്‍ കള്ളം കെട്ടിച്ചമയ്ക്കുകയോ, സത്യം വന്നുകിട്ടിയപ്പോള്‍ അത് നിഷേധിച്ച് തള്ളുകയോ ചെയ്തവനെക്കാള്‍ അക്രമിയായി ആരുണ്ട്‌.? നരകത്തില്‍ സത്യനിഷേധികള്‍ക്കു വാസസ്ഥലം ഇല്ലയോ?
\end{malayalam}}
\flushright{\begin{Arabic}
\quranayah[29][69]
\end{Arabic}}
\flushleft{\begin{malayalam}
നമ്മുടെ മാര്‍ഗത്തില്‍ സമരത്തില്‍ ഏര്‍പെട്ടവരാരോ, അവരെ നമ്മുടെ വഴികളിലേക്ക് നാം നയിക്കുക തന്നെ ചെയ്യുന്നതാണ്‌. തീര്‍ച്ചയായും അല്ലാഹു സദ്‌വൃത്തരോടൊപ്പമാകുന്നു.
\end{malayalam}}
\chapter{\textmalayalam{റൂം ( റോമാക്കാര്‍ )}}
\begin{Arabic}
\Huge{\centerline{\basmalah}}\end{Arabic}
\flushright{\begin{Arabic}
\quranayah[30][1]
\end{Arabic}}
\flushleft{\begin{malayalam}
അലിഫ്‌-ലാം-മീം
\end{malayalam}}
\flushright{\begin{Arabic}
\quranayah[30][2]
\end{Arabic}}
\flushleft{\begin{malayalam}
റോമക്കാര്‍ തോല്‍പിക്കപ്പെട്ടിരിക്കുന്നു.
\end{malayalam}}
\flushright{\begin{Arabic}
\quranayah[30][3]
\end{Arabic}}
\flushleft{\begin{malayalam}
അടുത്തനാട്ടില്‍ വെച്ച്‌. തങ്ങളുടെ പരാജയത്തിനു ശേഷം അവര്‍ വിജയം നേടുന്നതാണ്‌.
\end{malayalam}}
\flushright{\begin{Arabic}
\quranayah[30][4]
\end{Arabic}}
\flushleft{\begin{malayalam}
ഏതാനും വര്‍ഷങ്ങള്‍ക്കുള്ളില്‍ തന്നെ. മുമ്പും പിമ്പും അല്ലാഹുവിനാകുന്നു കാര്യങ്ങളുടെ നിയന്ത്രണം. അന്നേ ദിവസം സത്യവിശ്വാസികള്‍ സന്തുഷ്ടരാകുന്നതാണ്‌.
\end{malayalam}}
\flushright{\begin{Arabic}
\quranayah[30][5]
\end{Arabic}}
\flushleft{\begin{malayalam}
അല്ലാഹുവിന്‍റെ സഹായം കൊണ്ട്‌. താന്‍ ഉദ്ദേശിക്കുന്നവരെ അവന്‍ സഹായിക്കുന്നു. അവനത്രെ പ്രതാപിയും കരുണാനിധിയും.
\end{malayalam}}
\flushright{\begin{Arabic}
\quranayah[30][6]
\end{Arabic}}
\flushleft{\begin{malayalam}
അല്ലാഹുവിന്‍റെ വാഗ്ദാനമത്രെ ഇത്‌. അല്ലാഹു അവന്‍റെ വാഗ്ദാനം ലംഘിക്കുകയില്ല. പക്ഷെ മനുഷ്യരില്‍ അധികപേരും മനസ്സിലാക്കുന്നില്ല.
\end{malayalam}}
\flushright{\begin{Arabic}
\quranayah[30][7]
\end{Arabic}}
\flushleft{\begin{malayalam}
ഐഹികജീവിതത്തില്‍ നിന്ന് പ്രത്യക്ഷമായത് അവര്‍ മനസ്സിലാക്കുന്നു. പരലോകത്തെപ്പറ്റിയാകട്ടെ അവര്‍ അശ്രദ്ധയില്‍ തന്നെയാകുന്നു.
\end{malayalam}}
\flushright{\begin{Arabic}
\quranayah[30][8]
\end{Arabic}}
\flushleft{\begin{malayalam}
അവരുടെ സ്വന്തത്തെപ്പറ്റി അവര്‍ ചിന്തിച്ച് നോക്കിയിട്ടില്ലേ? ആകാശങ്ങളും ഭൂമിയും അവയ്ക്കിടയിലുള്ളതും ശരിയായ മുറപ്രകാരവും നിര്‍ണിതമായ അവധിയോട് കൂടിയുമല്ലാതെ അല്ലാഹു സൃഷ്ടിച്ചിട്ടില്ല. തീര്‍ച്ചയായും മനുഷ്യരില്‍ അധികപേരും തങ്ങളുടെ രക്ഷിതാവിനെ കണ്ടുമുട്ടുന്നതില്‍ വിശ്വാസമില്ലാത്തവരത്രെ.
\end{malayalam}}
\flushright{\begin{Arabic}
\quranayah[30][9]
\end{Arabic}}
\flushleft{\begin{malayalam}
അവര്‍ ഭൂമിയിലൂടെ സഞ്ചരിച്ചിട്ട് തങ്ങളുടെ മുമ്പുള്ളവരുടെ പര്യവസാനം എങ്ങനെയായിരുന്നു എന്ന് നോക്കുന്നില്ലേ? അവര്‍ ഇവരേക്കാള്‍ കൂടുതല്‍ ശക്തിയുള്ളവരായിരുന്നു. അവര്‍ ഭൂമി ഉഴുതുമറിക്കുകയും, ഇവര്‍ അധിവാസമുറപ്പിച്ചതിനെക്കാള്‍ കൂടുതല്‍ അതില്‍ അധിവാസമുറപ്പിക്കുകയും ചെയ്തു. നമ്മുടെ ദൂതന്‍മാര്‍ വ്യക്തമായ തെളിവുകളും കൊണ്ട് അവരുടെ അടുത്ത് ചെല്ലുകയുണ്ടായി. എന്നാല്‍ അല്ലാഹു അവരോട് അക്രമം ചെയ്യുകയുണ്ടായിട്ടില്ല. പക്ഷെ, അവര്‍ തങ്ങളോട് തന്നെ അക്രമം ചെയ്യുകയായിരുന്നു.
\end{malayalam}}
\flushright{\begin{Arabic}
\quranayah[30][10]
\end{Arabic}}
\flushleft{\begin{malayalam}
പിന്നീട്‌, ദുഷ്പ്രവൃത്തി ചെയ്തവരുടെ പര്യവസാനം ഏറ്റവും മോശമായിത്തീര്‍ന്നു. അല്ലാഹുവിന്‍റെ ദൃഷ്ടാന്തങ്ങള്‍ അവര്‍ നിഷേധിച്ച് തള്ളുകയും അവയെപ്പറ്റി അവര്‍ പരിഹസിച്ചു കൊണ്ടിരിക്കുകയും ചെയ്തതിന്‍റെ ഫലമത്രെ അത്‌.
\end{malayalam}}
\flushright{\begin{Arabic}
\quranayah[30][11]
\end{Arabic}}
\flushleft{\begin{malayalam}
അല്ലാഹു സൃഷ്ടി ആരംഭിക്കുകയും പിന്നീട് അത് ആവര്‍ത്തിക്കുകയും ചെയ്യുന്നു. പിന്നീട് അവങ്കലേക്ക് നിങ്ങള്‍ മടക്കപ്പെടുകയും ചെയ്യുന്നു.
\end{malayalam}}
\flushright{\begin{Arabic}
\quranayah[30][12]
\end{Arabic}}
\flushleft{\begin{malayalam}
അന്ത്യസമയം നിലവില്‍ വരുന്ന ദിവസം കുറ്റവാളികള്‍ ആശയറ്റവരാകും.
\end{malayalam}}
\flushright{\begin{Arabic}
\quranayah[30][13]
\end{Arabic}}
\flushleft{\begin{malayalam}
അവര്‍ പങ്കാളികളാക്കിയവരുടെ കൂട്ടത്തില്‍ അവര്‍ക്ക് ശുപാര്‍ശക്കാര്‍ ആരുമുണ്ടായിരിക്കുകയില്ല. അവരുടെ ആ പങ്കാളികളെത്തന്നെ അവര്‍ നിഷേധിക്കുന്നവരാവുകയും ചെയ്യും.
\end{malayalam}}
\flushright{\begin{Arabic}
\quranayah[30][14]
\end{Arabic}}
\flushleft{\begin{malayalam}
അന്ത്യസമയം നിലവില്‍ വരുന്ന ദിവസം - അന്നാണ് അവര്‍ വേര്‍പിരിയുന്നത്‌.
\end{malayalam}}
\flushright{\begin{Arabic}
\quranayah[30][15]
\end{Arabic}}
\flushleft{\begin{malayalam}
എന്നാല്‍ വിശ്വസിക്കുകയും സല്‍കര്‍മ്മങ്ങള്‍ പ്രവര്‍ത്തിക്കുകയും ചെയ്തവരാരോ അവര്‍ ഒരു പൂന്തോട്ടത്തില്‍ ആനന്ദം അനുഭവിക്കുന്നവരായിരിക്കും.
\end{malayalam}}
\flushright{\begin{Arabic}
\quranayah[30][16]
\end{Arabic}}
\flushleft{\begin{malayalam}
എന്നാല്‍ അവിശ്വസിക്കുകയും നമ്മുടെ ദൃഷ്ടാന്തങ്ങളെയും പരലോകത്തെ കണ്ടുമുട്ടുന്നതിനെയും നിഷേധിച്ചു കളയുകയും ചെയ്തവരാരോ അവര്‍ ശിക്ഷയ്ക്കായി ഹാജരാക്കപ്പെടുന്നവരാകുന്നു.
\end{malayalam}}
\flushright{\begin{Arabic}
\quranayah[30][17]
\end{Arabic}}
\flushleft{\begin{malayalam}
ആകയാല്‍ നിങ്ങള്‍ സന്ധ്യാവേളയിലാകുമ്പോഴും പ്രഭാതവേളയിലാകുമ്പോഴും അല്ലാഹുവിന്‍റെ പരിശുദ്ധിയെ പ്രകീര്‍ത്തിക്കുക.
\end{malayalam}}
\flushright{\begin{Arabic}
\quranayah[30][18]
\end{Arabic}}
\flushleft{\begin{malayalam}
ആകാശങ്ങളിലും ഭൂമിയിലും അവനുതന്നെയാകുന്നു സ്തുതി. വൈകുന്നേരവും ഉച്ചതിരിയുമ്പോഴും (അവനെ നിങ്ങള്‍ പ്രകീര്‍ത്തിക്കുക.)
\end{malayalam}}
\flushright{\begin{Arabic}
\quranayah[30][19]
\end{Arabic}}
\flushleft{\begin{malayalam}
നിര്‍ജീവമായതില്‍ നിന്ന് ജീവനുള്ളതിനെ അവന്‍ പുറത്ത് കൊണ്ട് വരുന്നു. ജീവനുള്ളതില്‍ നിന്ന് നിര്‍ജീവമായതിനെയും അവന്‍ പുറത്ത് കൊണ്ട് വരുന്നു. ഭൂമിയുടെ നിര്‍ജീവാവസ്ഥയ്ക്ക് ശേഷം അതിന്നവന്‍ ജീവന്‍ നല്‍കുകയും ചെയ്യുന്നു. അതുപോലെത്തന്നെ നിങ്ങളും പുറത്ത് കൊണ്ട് വരപ്പെടും.
\end{malayalam}}
\flushright{\begin{Arabic}
\quranayah[30][20]
\end{Arabic}}
\flushleft{\begin{malayalam}
നിങ്ങളെ അവന്‍ മണ്ണില്‍ നിന്ന് സൃഷ്ടിച്ചു. എന്നിട്ട് നിങ്ങളതാ (ലോകമാകെ) വ്യാപിക്കുന്ന മനുഷ്യവര്‍ഗമായിരിക്കുന്നു. ഇത് അവന്‍റെ ദൃഷ്ടാന്തങ്ങളില്‍ പെട്ടതത്രെ.
\end{malayalam}}
\flushright{\begin{Arabic}
\quranayah[30][21]
\end{Arabic}}
\flushleft{\begin{malayalam}
നിങ്ങള്‍ക്ക് സമാധാനപൂര്‍വ്വം ഒത്തുചേരേണ്ടതിനായി നിങ്ങളില്‍ നിന്ന് തന്നെ നിങ്ങള്‍ക്ക് ഇണകളെ സൃഷ്ടിക്കുകയും, നിങ്ങള്‍ക്കിടയില്‍ സ്നേഹവും കാരുണ്യവും ഉണ്ടാക്കുകയും ചെയ്തതും അവന്‍റെ ദൃഷ്ടാന്തങ്ങളില്‍ പെട്ടതത്രെ. തീര്‍ച്ചയായും അതില്‍ ചിന്തിക്കുന്ന ജനങ്ങള്‍ക്ക് ദൃഷ്ടാന്തങ്ങളുണ്ട്‌.
\end{malayalam}}
\flushright{\begin{Arabic}
\quranayah[30][22]
\end{Arabic}}
\flushleft{\begin{malayalam}
ആകാശഭൂമികളുടെ സൃഷ്ടിയും, നിങ്ങളുടെ ഭാഷകളിലും വര്‍ണങ്ങളിലുമുള്ള വ്യത്യാസവും അവന്‍റെ ദൃഷ്ടാന്തങ്ങളില്‍ പെട്ടതത്രെ. തീര്‍ച്ചയായും അതില്‍ അറിവുള്ളവര്‍ക്ക് ദൃഷ്ടാന്തങ്ങളുണ്ട്‌.
\end{malayalam}}
\flushright{\begin{Arabic}
\quranayah[30][23]
\end{Arabic}}
\flushleft{\begin{malayalam}
രാത്രിയും പകലും നിങ്ങള്‍ ഉറങ്ങുന്നതും, അവന്‍റെ അനുഗ്രഹത്തില്‍ നിന്ന് നിങ്ങള്‍ ഉപജീവനം തേടുന്നതും അവന്‍റെ ദൃഷ്ടാന്തങ്ങളില്‍ പെട്ടതത്രെ. തീര്‍ച്ചയായും അതില്‍ കേട്ടുമനസ്സിലാക്കുന്ന ജനങ്ങള്‍ക്ക് ദൃഷ്ടാന്തങ്ങളുണ്ട്‌.
\end{malayalam}}
\flushright{\begin{Arabic}
\quranayah[30][24]
\end{Arabic}}
\flushleft{\begin{malayalam}
ഭയവും ആശയും ഉളവാക്കിക്കൊണ്ട് നിങ്ങള്‍ക്ക് മിന്നല്‍ കാണിച്ചുതരുന്നതും ആകാശത്ത് നിന്ന് വെള്ളം ചൊരിയുകയും അത് മൂലം ഭൂമിക്ക് അതിന്‍റെ നിര്‍ജീവാവസ്ഥയ്ക്ക് ശേഷം ജീവന്‍ നല്‍കുകയും ചെയ്യുന്നതും അവന്‍റെ ദൃഷ്ടാന്തങ്ങളില്‍ പെട്ടതത്രെ. തീര്‍ച്ചയായും അതില്‍ ചിന്തിച്ച് മനസ്സിലാക്കുന്ന ജനങ്ങള്‍ക്ക് ദൃഷ്ടാന്തങ്ങളുണ്ട്‌.
\end{malayalam}}
\flushright{\begin{Arabic}
\quranayah[30][25]
\end{Arabic}}
\flushleft{\begin{malayalam}
അവന്‍റെ കല്‍പനപ്രകാരം ആകാശവും ഭൂമിയും നിലനിന്ന് വരുന്നതും അവന്‍റെ ദൃഷ്ടാന്തങ്ങളില്‍ പെട്ടതത്രെ. പിന്നെ, ഭൂമിയില്‍ നിന്ന് നിങ്ങളെ അവന്‍ ഒരു വിളി വിളിച്ചാല്‍ നിങ്ങളതാ പുറത്ത് വരുന്നു.
\end{malayalam}}
\flushright{\begin{Arabic}
\quranayah[30][26]
\end{Arabic}}
\flushleft{\begin{malayalam}
ആകാശങ്ങളിലും ഭൂമിയിലും ഉള്ളവരെല്ലാം അവന്‍റെ അധീനത്തിലത്രെ. എല്ലാവരും അവന്ന് കീഴടങ്ങുന്നവരാകുന്നു.
\end{malayalam}}
\flushright{\begin{Arabic}
\quranayah[30][27]
\end{Arabic}}
\flushleft{\begin{malayalam}
അവനാകുന്നു സൃഷ്ടി ആരംഭിക്കുന്നവന്‍. പിന്നെ അവന്‍ അത് ആവര്‍ത്തിക്കുന്നു. അത് അവനെ സംബന്ധിച്ചിടത്തോളം കൂടുതല്‍ എളുപ്പമുള്ളതാകുന്നു. ആകാശങ്ങളിലും ഭൂമിയിലും ഏറ്റവും ഉന്നതമായ അവസ്ഥയുള്ളത് അവന്നാകുന്നു. അവന്‍ പ്രതാപിയും യുക്തിമാനുമത്രെ.
\end{malayalam}}
\flushright{\begin{Arabic}
\quranayah[30][28]
\end{Arabic}}
\flushleft{\begin{malayalam}
നിങ്ങളുടെ കാര്യത്തില്‍ നിന്നു തന്നെ അല്ലാഹു നിങ്ങള്‍ക്കിതാ ഒരു ഉപമ വിവരിച്ചുതന്നിരിക്കുന്നു. നിങ്ങളുടെ വലതുകൈകള്‍ ഉടമപ്പെടുത്തിയ അടിമകളില്‍ ആരെങ്കിലും നിങ്ങള്‍ക്ക് നാം നല്‍കിയ കാര്യങ്ങളില്‍ നിങ്ങളുടെ പങ്കുകാരാകുന്നുണ്ടോ? എന്നിട്ട് നിങ്ങള്‍ അന്യോന്യം ഭയപ്പെടുന്നത് പോലെ അവരെ (അടിമകളെ) യും നിങ്ങള്‍ ഭയപ്പെടുമാറ് നിങ്ങളിരുകൂട്ടരും അതില്‍ സമാവകാശികളാവുകയും ചെയ്യുന്നുണ്ടോ? ചിന്തിച്ച് മനസ്സിലാക്കുന്ന ജനങ്ങള്‍ക്കു വേണ്ടി അപ്രകാരം നാം തെളിവുകള്‍ വിശദീകരിക്കുന്നു.
\end{malayalam}}
\flushright{\begin{Arabic}
\quranayah[30][29]
\end{Arabic}}
\flushleft{\begin{malayalam}
പക്ഷെ, അക്രമം പ്രവര്‍ത്തിച്ചവര്‍ യാതൊരു അറിവുമില്ലാതെ തങ്ങളുടെ തന്നിഷ്ടങ്ങളെ പിന്‍പറ്റിയിരിക്കുകയാണ്‌. അപ്പോള്‍ അല്ലാഹു വഴിതെറ്റിച്ചവരെ ആരാണ് സന്‍മാര്‍ഗത്തിലാക്കുക? അവര്‍ക്ക് സഹായികളായി ആരുമില്ല.
\end{malayalam}}
\flushright{\begin{Arabic}
\quranayah[30][30]
\end{Arabic}}
\flushleft{\begin{malayalam}
ആകയാല്‍ (സത്യത്തില്‍) നേരെ നിലകൊള്ളുന്നവനായിട്ട് നിന്‍റെ മുഖത്തെ നീ മതത്തിലേക്ക് തിരിച്ച് നിര്‍ത്തുക. അല്ലാഹു മനുഷ്യരെ ഏതൊരു പ്രകൃതിയില്‍ സൃഷ്ടിച്ചിരിക്കുന്നുവോ ആ പ്രകൃതിയത്രെ അത്‌. അല്ലാഹുവിന്‍റെ സൃഷ്ടി വ്യവസ്ഥയ്ക്ക് യാതൊരു മാറ്റവുമില്ല. അതത്രെ വക്രതയില്ലാത്ത മതം. പക്ഷെ മനുഷ്യരില്‍ അധിക പേരും മനസ്സിലാക്കുന്നില്ല.
\end{malayalam}}
\flushright{\begin{Arabic}
\quranayah[30][31]
\end{Arabic}}
\flushleft{\begin{malayalam}
(നിങ്ങള്‍) അവങ്കലേക്ക് തിരിഞ്ഞവരായിരിക്കുകയും, അവനെ സൂക്ഷിക്കുകയും, നമസ്കാരം മുറപോലെ നിര്‍വഹിക്കുകയും ചെയ്യുക. നിങ്ങള്‍ ബഹുദൈവാരാധകരുടെ കൂട്ടത്തിലായിപ്പോകരുത്‌.
\end{malayalam}}
\flushright{\begin{Arabic}
\quranayah[30][32]
\end{Arabic}}
\flushleft{\begin{malayalam}
അതായത്‌, തങ്ങളുടെ മതത്തെ ഛിന്നഭിന്നമാക്കുകയും, പലകക്ഷികളായി തിരിയുകയും ചെയ്തവരുടെ കൂട്ടത്തില്‍. ഓരോ കക്ഷിയും തങ്ങളുടെ പക്കലുള്ളതില്‍ സന്തോഷമടയുന്നവരത്രെ.
\end{malayalam}}
\flushright{\begin{Arabic}
\quranayah[30][33]
\end{Arabic}}
\flushleft{\begin{malayalam}
ജനങ്ങള്‍ക്ക് വല്ല ദുരിതവും ബാധിച്ചാല്‍ തങ്ങളുടെ രക്ഷിതാവിങ്കലേക്ക് തിരിഞ്ഞും കൊണ്ട് അവനോട് അവര്‍ പ്രാര്‍ത്ഥിക്കുന്നതാണ്‌. പിന്നെ തന്‍റെ പക്കല്‍ നിന്നുള്ള കാരുണ്യം അവര്‍ക്കവന്‍ അനുഭവിപ്പിച്ചാല്‍ അവരില്‍ ഒരു വിഭാഗമതാ തങ്ങളുടെ രക്ഷിതാവിനോട് പങ്കുചേര്‍ക്കുന്നു.
\end{malayalam}}
\flushright{\begin{Arabic}
\quranayah[30][34]
\end{Arabic}}
\flushleft{\begin{malayalam}
അങ്ങനെ നാം അവര്‍ക്ക് നല്‍കിയതിനു നന്ദികേട് കാണിക്കുകയത്രെ അവര്‍ ചെയ്യുന്നത്‌. ആകയാല്‍ നിങ്ങള്‍ സുഖം അനുഭവിച്ച് കൊള്ളുക. വഴിയെ നിങ്ങള്‍ മനസ്സിലാക്കികൊള്ളും.
\end{malayalam}}
\flushright{\begin{Arabic}
\quranayah[30][35]
\end{Arabic}}
\flushleft{\begin{malayalam}
അതല്ല, അവര്‍ (അല്ലാഹുവോട്‌) പങ്കുചേര്‍ത്തിരുന്നതിനനൂകൂലമായി അവരോട് സംസാരിക്കുന്ന വല്ല പ്രമാണവും നാം അവര്‍ക്ക് ഇറക്കികൊടുത്തിട്ടുണ്ടോ?
\end{malayalam}}
\flushright{\begin{Arabic}
\quranayah[30][36]
\end{Arabic}}
\flushleft{\begin{malayalam}
മനുഷ്യര്‍ക്ക് നാം കാരുണ്യം അനുഭവിപ്പിക്കുന്ന പക്ഷം അവര്‍ അതില്‍ ആഹ്ലാദം കൊള്ളുന്നു. തങ്ങളുടെ കൈകള്‍ മുന്‍കൂട്ടിചെയ്തതിന്‍റെ ഫലമായി അവര്‍ക്ക് വല്ല ദോഷവും ബാധിക്കുകയാണെങ്കിലോ അവരതാ ആശയറ്റവരാകുന്നു.
\end{malayalam}}
\flushright{\begin{Arabic}
\quranayah[30][37]
\end{Arabic}}
\flushleft{\begin{malayalam}
താന്‍ ഉദ്ദേശിക്കുന്നവര്‍ക്ക് അല്ലാഹു ഉപജീവനം വിശാലമാക്കുകയും (താന്‍ ഉദ്ദേശിക്കുന്നവര്‍ക്ക്‌) ഇടുങ്ങിയതാക്കുകയും ചെയ്യുന്നു എന്ന് അവര്‍ കണ്ടില്ലേ? വിശ്വസിക്കുന്ന ജനങ്ങള്‍ക്ക് അതില്‍ ദൃഷ്ടാന്തങ്ങളുണ്ട്‌; തീര്‍ച്ച.
\end{malayalam}}
\flushright{\begin{Arabic}
\quranayah[30][38]
\end{Arabic}}
\flushleft{\begin{malayalam}
ആകയാല്‍ കുടുംബബന്ധമുള്ളവന് നീ അവന്‍റെ അവകാശം കൊടുക്കുക. അഗതിക്കും വഴിപോക്കനും (അവരുടെ അവകാശവും നല്‍കുക). അല്ലാഹുവിന്‍റെ പ്രീതി ലക്ഷ്യമാക്കുന്നവര്‍ക്ക് അതാണുത്തമം. അവര്‍ തന്നെയാണ് വിജയികളും.
\end{malayalam}}
\flushright{\begin{Arabic}
\quranayah[30][39]
\end{Arabic}}
\flushleft{\begin{malayalam}
ജനങ്ങളുടെ സ്വത്തുക്കളിലൂടെ വളര്‍ച്ച നേടുവാനായി നിങ്ങള്‍ വല്ലതും പലിശയ്ക്ക് കൊടുക്കുന്ന പക്ഷം അല്ലാഹുവിങ്കല്‍ അത് വളരുകയില്ല. അല്ലാഹുവിന്‍റെ പ്രീതി ലക്ഷ്യമാക്കിക്കൊണ്ട് നിങ്ങള്‍ വല്ലതും സകാത്തായി നല്‍കുന്ന പക്ഷം അങ്ങനെ ചെയ്യുന്നവരത്രെ ഇരട്ടി സമ്പാദിക്കുന്നവര്‍.
\end{malayalam}}
\flushright{\begin{Arabic}
\quranayah[30][40]
\end{Arabic}}
\flushleft{\begin{malayalam}
അല്ലാഹുവാണ് നിങ്ങളെ സൃഷ്ടിച്ചത്‌. എന്നിട്ടവന്‍ നിങ്ങള്‍ക്ക് ഉപജീവനം നല്‍കി. പിന്നെ നിങ്ങളെ അവന്‍ മരിപ്പിക്കുന്നു. പിന്നീട് അവന്‍ നിങ്ങളെ ജീവിപ്പിക്കുകയും ചെയ്യും. അതില്‍ പെട്ട ഏതെങ്കിലും ഒരു കാര്യം ചെയ്യുന്ന വല്ലവരും നിങ്ങള്‍ പങ്കാളികളാക്കിയവരുടെ കൂട്ടത്തിലുണ്ടോ? അവന്‍ എത്രയോ പരിശുദ്ധന്‍. അവര്‍ പങ്കുചേര്‍ക്കുന്നതിനെല്ലാം അവന്‍ അതീതനായിരിക്കുന്നു.
\end{malayalam}}
\flushright{\begin{Arabic}
\quranayah[30][41]
\end{Arabic}}
\flushleft{\begin{malayalam}
മനുഷ്യരുടെ കൈകള്‍ പ്രവര്‍ത്തിച്ചത് നിമിത്തം കരയിലും കടലിലും കുഴപ്പം വെളിപ്പെട്ടിരിക്കുന്നു. അവര്‍ പ്രവര്‍ത്തിച്ചതില്‍ ചിലതിന്‍റെ ഫലം അവര്‍ക്ക് ആസ്വദിപ്പിക്കുവാന്‍ വേണ്ടിയത്രെ അത്‌. അവര്‍ ഒരു വേള മടങ്ങിയേക്കാം.
\end{malayalam}}
\flushright{\begin{Arabic}
\quranayah[30][42]
\end{Arabic}}
\flushleft{\begin{malayalam}
(നബിയേ,) പറയുക: നിങ്ങള്‍ ഭൂമിയിലൂടെ സഞ്ചരിച്ചിട്ട് മുമ്പുണ്ടായിരുന്നവരുടെ പര്യവസാനം എങ്ങനെയായിരുന്നു എന്ന് നോക്കുക. അവരില്‍ അധികപേരും ബഹുദൈവാരാധകരായിരുന്നു.
\end{malayalam}}
\flushright{\begin{Arabic}
\quranayah[30][43]
\end{Arabic}}
\flushleft{\begin{malayalam}
ആകയാല്‍ അല്ലാഹുവില്‍ നിന്ന് ആര്‍ക്കും തടുക്കാനാവാത്ത ഒരു ദിവസം വരുന്നതിന് മുമ്പായി നീ നിന്‍റെ മുഖം വക്രതയില്ലാത്ത മതത്തിലേക്ക് തിരിച്ച് നിര്‍ത്തുക. അന്നേദിവസം ജനങ്ങള്‍ (രണ്ടുവിഭാഗമായി) പിരിയുന്നതാണ്‌.
\end{malayalam}}
\flushright{\begin{Arabic}
\quranayah[30][44]
\end{Arabic}}
\flushleft{\begin{malayalam}
വല്ലവനും നന്ദികേട് കാണിച്ചാല്‍ അവന്‍റെ നന്ദികേടിന്‍റെ ദോഷം അവന്നുതന്നെയായിരിക്കും. വല്ലവനും സല്‍കര്‍മ്മം ചെയ്യുന്ന പക്ഷം തങ്ങള്‍ക്ക് വേണ്ടി തന്നെ സൌകര്യമൊരുക്കുകയാണ് അവര്‍ ചെയ്യുന്നത്‌.
\end{malayalam}}
\flushright{\begin{Arabic}
\quranayah[30][45]
\end{Arabic}}
\flushleft{\begin{malayalam}
വിശ്വസിക്കുകയും സല്‍കര്‍മ്മങ്ങള്‍ പ്രവര്‍ത്തിക്കുകയും ചെയ്തവര്‍ക്ക് തന്‍റെ അനുഗ്രഹത്താല്‍ അല്ലാഹു പ്രതിഫലം നല്‍കുന്നതിന് വേണ്ടിയത്രെ അത്‌. സത്യനിഷേധികളെ അവന്‍ ഇഷ്ടപ്പെടുകയില്ല; തീര്‍ച്ച.
\end{malayalam}}
\flushright{\begin{Arabic}
\quranayah[30][46]
\end{Arabic}}
\flushleft{\begin{malayalam}
(മഴയെപ്പറ്റി) സന്തോഷസൂചകമായിക്കൊണ്ടും, തന്‍റെ കാരുണ്യത്തില്‍ നിന്ന് നിങ്ങള്‍ക്ക് അനുഭവിപ്പിക്കാന്‍ വേണ്ടിയും, തന്‍റെ കല്‍പനപ്രകാരം കപ്പല്‍ സഞ്ചരിക്കുവാന്‍ വേണ്ടിയും, തന്‍റെ അനുഗ്രഹത്തില്‍ നിന്ന് നിങ്ങള്‍ ഉപജീവനം തേടുവാന്‍ വേണ്ടിയും, നിങ്ങള്‍ നന്ദികാണിക്കുവാന്‍ വേണ്ടിയും അവന്‍ കാറ്റുകളെ അയക്കുന്നത് അവന്‍റെ ദൃഷ്ടാന്തങ്ങളില്‍ പെട്ടതത്രെ.
\end{malayalam}}
\flushright{\begin{Arabic}
\quranayah[30][47]
\end{Arabic}}
\flushleft{\begin{malayalam}
നിനക്ക് മുമ്പ് പല ദൂതന്‍മാരെയും അവരുടെ ജനതയിലേക്ക് നാം നിയോഗിച്ചിട്ടുണ്ട്‌. എന്നിട്ട് വ്യക്തമായ തെളിവുകളും കൊണ്ട് അവര്‍ (ദൂതന്‍മാര്‍) അവരുടെയടുത്ത് ചെന്നു. അപ്പോള്‍ കുറ്റകരമായ നിലപാട് സ്വീകരിച്ചവരുടെ കാര്യത്തില്‍ നാം ശിക്ഷാനടപടി സ്വീകരിച്ചു. വിശ്വാസികളെ സഹായിക്കുക എന്നത് നമ്മുടെ ബാധ്യതയായിരിക്കുന്നു.
\end{malayalam}}
\flushright{\begin{Arabic}
\quranayah[30][48]
\end{Arabic}}
\flushleft{\begin{malayalam}
അല്ലാഹുവാകുന്നു കാറ്റുകളെ അയക്കുന്നവന്‍. എന്നിട്ട് അവ (കാറ്റുകള്‍) മേഘത്തെ ഇളക്കിവിടുന്നു. എന്നിട്ട് അവന്‍ ഉദ്ദേശിക്കുന്ന പ്രകാരം അതിനെ ആകാശത്ത് പരത്തുന്നു. അതിനെ പല കഷ്ണങ്ങളാക്കുകയും ചെയ്യുന്നു. അപ്പോള്‍ അതിന്നിടയില്‍ നിന്ന് മഴപുറത്ത് വരുന്നതായി നിനക്ക് കാണാം. എന്നിട്ട് തന്‍റെ ദാസന്‍മാരില്‍ നിന്ന് താന്‍ ഉദ്ദേശിക്കുന്നവര്‍ക്ക് അവന്‍ ആ മഴ എത്തിച്ചുകൊടുത്താല്‍ അവരതാ സന്തുഷ്ടരാകുന്നു.
\end{malayalam}}
\flushright{\begin{Arabic}
\quranayah[30][49]
\end{Arabic}}
\flushleft{\begin{malayalam}
ഇതിന് മുമ്പ് -ആ മഴ അവരുടെ മേല്‍ വര്‍ഷിപ്പിക്കപ്പെടുന്നതിന് മുമ്പ് -തീര്‍ച്ചയായും അവര്‍ ആശയറ്റവര്‍ തന്നെയായിരുന്നു.
\end{malayalam}}
\flushright{\begin{Arabic}
\quranayah[30][50]
\end{Arabic}}
\flushleft{\begin{malayalam}
അപ്പോള്‍ അല്ലാഹുവിന്‍റെ കാരുണ്യത്തിന്‍റെ ഫലങ്ങള്‍ നോക്കൂ. ഭൂമി നിര്‍ജീവമായിരുന്നതിന് ശേഷം എങ്ങനെയാണ് അവന്‍ അതിന് ജീവന്‍ നല്‍കുന്നത്‌? തീര്‍ച്ചയായും അത് ചെയ്യുന്നവന്‍ മരിച്ചവരെ ജീവിപ്പിക്കുക തന്നെ ചെയ്യും. അവന്‍ ഏത് കാര്യത്തിനും കഴിവുള്ളവനത്രെ.
\end{malayalam}}
\flushright{\begin{Arabic}
\quranayah[30][51]
\end{Arabic}}
\flushleft{\begin{malayalam}
ഇനി നാം മറ്റൊരു കാറ്റ് അയച്ചിട്ട് അത് (കൃഷി) മഞ്ഞനിറം ബാധിച്ചതായി അവര്‍ കണ്ടാല്‍ അതിന് ശേഷവും അവര്‍ നന്ദികേട് കാണിക്കുന്നവരായിക്കൊണേ്ടയിരിക്കുന്നതാണ്‌.
\end{malayalam}}
\flushright{\begin{Arabic}
\quranayah[30][52]
\end{Arabic}}
\flushleft{\begin{malayalam}
എന്നാല്‍ മരിച്ചവരെ നിനക്ക് കേള്‍പിക്കാനാവില്ല; തീര്‍ച്ച. ബധിരന്‍മാര്‍ പിന്നോക്കം തിരിഞ്ഞ് പോയാല്‍ അവരെ വിളികേള്‍പിക്കാനും നിനക്കാവില്ല.
\end{malayalam}}
\flushright{\begin{Arabic}
\quranayah[30][53]
\end{Arabic}}
\flushleft{\begin{malayalam}
അന്ധത ബാധിച്ചവരെ അവരുടെ വഴികേടില്‍ നിന്ന് നേര്‍വഴിയിലേക്ക് നയിക്കാനും നിനക്കാവില്ല. നമ്മുടെ ദൃഷ്ടാന്തങ്ങളില്‍ വിശ്വസിക്കുന്നവരും, എന്നിട്ട് കീഴ്പെട്ട് ജീവിക്കുന്നവരുമായിട്ടുള്ളവരെയല്ലാതെ നിനക്ക് കേള്‍പിക്കാനാവില്ല.
\end{malayalam}}
\flushright{\begin{Arabic}
\quranayah[30][54]
\end{Arabic}}
\flushleft{\begin{malayalam}
നിങ്ങളെ ബലഹീനമായ അവസ്ഥയില്‍ നിന്നു സൃഷ്ടിച്ചുണ്ടാക്കിയവനാകുന്നു അല്ലാഹു. പിന്നെ ബലഹീനതയ്ക്കു ശേഷം അവന്‍ ശക്തിയുണ്ടാക്കി. പിന്നെ അവന്‍ ശക്തിക്ക് ശേഷം ബലഹീനതയും നരയും ഉണ്ടാക്കി. അവന്‍ ഉദ്ദേശിക്കുന്നത് അവന്‍ സൃഷ്ടിക്കുന്നു. അവനത്രെ സര്‍വ്വജ്ഞനും സര്‍വ്വശക്തനും.
\end{malayalam}}
\flushright{\begin{Arabic}
\quranayah[30][55]
\end{Arabic}}
\flushleft{\begin{malayalam}
അന്ത്യസമയം നിലവില്‍ വരുന്ന ദിവസം കുറ്റവാളികള്‍ സത്യം ചെയ്ത് പറയും; തങ്ങള്‍ (ഇഹലോകത്ത്‌) ഒരു നാഴിക നേരമല്ലാതെ കഴിച്ചുകൂട്ടിയിട്ടില്ലെന്ന് .അപ്രകാരം തന്നെയായിരുന്നു അവര്‍ (സത്യത്തില്‍ നിന്ന്‌) തെറ്റിക്കപ്പെട്ടിരുന്നത്‌.
\end{malayalam}}
\flushright{\begin{Arabic}
\quranayah[30][56]
\end{Arabic}}
\flushleft{\begin{malayalam}
വിജ്ഞാനവും വിശ്വാസവും നല്‍കപ്പെട്ടവര്‍ ഇപ്രകാരം പറയുന്നതാണ്‌: അല്ലാഹുവിന്‍റെ രേഖയിലുള്ള പ്രകാരം ഉയിര്‍ത്തെഴുന്നേല്‍പിന്‍റെ നാളുവരെ നിങ്ങള്‍ കഴിച്ചുകൂട്ടിയിട്ടുണ്ട്‌. എന്നാല്‍ ഇതാ ഉയിര്‍ത്തെഴുന്നേല്‍പിന്‍റെ നാള്‍. പക്ഷെ നിങ്ങള്‍ (അതിനെപ്പറ്റി) മനസ്സിലാക്കിയിരുന്നില്ല.
\end{malayalam}}
\flushright{\begin{Arabic}
\quranayah[30][57]
\end{Arabic}}
\flushleft{\begin{malayalam}
എന്നാല്‍ അക്രമം പ്രവര്‍ത്തിച്ചവര്‍ക്ക് അന്നത്തെ ദിവസം അവരുടെ ഒഴികഴിവ് പ്രയോജനപ്പെടുകയില്ല. അവര്‍ പശ്ചാത്തപിക്കാന്‍ അനുശാസിക്കുപ്പെടുന്നതുമല്ല.
\end{malayalam}}
\flushright{\begin{Arabic}
\quranayah[30][58]
\end{Arabic}}
\flushleft{\begin{malayalam}
മനുഷ്യര്‍ക്ക് വേണ്ടി ഈ ഖുര്‍ആനില്‍ എല്ലാവിധ ഉപമയും നാം വിവരിച്ചിട്ടുണ്ട്‌. നീ വല്ല ദൃഷ്ടാന്തവും കൊണ്ട് അവരുടെ അടുത്ത് ചെന്നാല്‍ അവിശ്വാസികള്‍ പറയും: നിങ്ങള്‍ അസത്യവാദികള്‍ മാത്രമാണെന്ന് .
\end{malayalam}}
\flushright{\begin{Arabic}
\quranayah[30][59]
\end{Arabic}}
\flushleft{\begin{malayalam}
(കാര്യം) മനസ്സിലാക്കാത്തവരുടെ ഹൃദയങ്ങളില്‍ അപ്രകാരം അല്ലാഹു മുദ്രവെക്കുന്നു.
\end{malayalam}}
\flushright{\begin{Arabic}
\quranayah[30][60]
\end{Arabic}}
\flushleft{\begin{malayalam}
ആകയാല്‍ നീ ക്ഷമിക്കുക. തീര്‍ച്ചയായും അല്ലാഹുവിന്‍റെ വാഗ്ദാനം സത്യമാകുന്നു. ദൃഢവിശ്വാസമില്ലാത്ത ആളുകള്‍ നിനക്ക് ചാഞ്ചല്യം വരുത്താതിരിക്കുകയും ചെയ്യട്ടെ.
\end{malayalam}}
\chapter{\textmalayalam{ലുഖ്മാന്‍}}
\begin{Arabic}
\Huge{\centerline{\basmalah}}\end{Arabic}
\flushright{\begin{Arabic}
\quranayah[31][1]
\end{Arabic}}
\flushleft{\begin{malayalam}
അലിഫ്‌-ലാം-മീം
\end{malayalam}}
\flushright{\begin{Arabic}
\quranayah[31][2]
\end{Arabic}}
\flushleft{\begin{malayalam}
തത്വസമ്പൂര്‍ണ്ണമായ വേദഗ്രന്ഥത്തിലെ വചനങ്ങളത്രെ അവ.
\end{malayalam}}
\flushright{\begin{Arabic}
\quranayah[31][3]
\end{Arabic}}
\flushleft{\begin{malayalam}
സദ്‌വൃത്തര്‍ക്ക് മാര്‍ഗദര്‍ശനവും കാരുണ്യവുമത്രെ അത്‌.
\end{malayalam}}
\flushright{\begin{Arabic}
\quranayah[31][4]
\end{Arabic}}
\flushleft{\begin{malayalam}
നമസ്കാരം മുറപ്രകാരം നിര്‍വഹിക്കുകയും സകാത്ത് നല്‍കുകയും, പരലോകത്തില്‍ ദൃഢവിശ്വാസമുള്ളവരായിരിക്കുകയും ചെയ്യുന്നവര്‍ക്ക്‌.
\end{malayalam}}
\flushright{\begin{Arabic}
\quranayah[31][5]
\end{Arabic}}
\flushleft{\begin{malayalam}
തങ്ങളുടെ രക്ഷിതാവിങ്കല്‍ നിന്നുള്ള മാര്‍ഗദര്‍ശനമനുസരിച്ച് നിലകൊള്ളുന്നവരത്രെ അവര്‍. അവര്‍ തന്നെയാണ് വിജയികള്‍.
\end{malayalam}}
\flushright{\begin{Arabic}
\quranayah[31][6]
\end{Arabic}}
\flushleft{\begin{malayalam}
യാതൊരു അറിവുമില്ലാതെ ദൈവമാര്‍ഗത്തില്‍ നിന്ന് ജനങ്ങളെ തെറ്റിച്ചുകളയുവാനും, അതിനെ പരിഹാസ്യമാക്കിത്തീര്‍ക്കുവാനും വേണ്ടി വിനോദവാര്‍ത്തകള്‍ വിലയ്ക്കു വാങ്ങുന്ന ചിലര്‍ മനുഷ്യരുടെ കൂട്ടത്തിലുണ്ട്‌. അത്തരക്കാര്‍ക്കാണ് അപമാനകരമായ ശിക്ഷയുള്ളത്‌.
\end{malayalam}}
\flushright{\begin{Arabic}
\quranayah[31][7]
\end{Arabic}}
\flushleft{\begin{malayalam}
അത്തരം ഒരാള്‍ക്ക് നമ്മുടെ വചനങ്ങള്‍ ഓതികേള്‍പിക്കപ്പെടുകയാണെങ്കില്‍ അവന്‍ അഹങ്കരിച്ച് കൊണ്ട് തിരിഞ്ഞുകളയുന്നതാണ്‌. അവനത് കേട്ടിട്ടില്ലാത്തപോലെ. അവന്‍റെ ഇരുകാതുകളിലും അടപ്പുള്ളതുപോലെ. ആകയാല്‍ നീ അവന്ന് വേദനയേറിയ ശിക്ഷയെപ്പറ്റി സന്തോഷവാര്‍ത്ത യറിയിക്കുക.
\end{malayalam}}
\flushright{\begin{Arabic}
\quranayah[31][8]
\end{Arabic}}
\flushleft{\begin{malayalam}
തീര്‍ച്ചയായും; വിശ്വസിക്കുകയും സല്‍കര്‍മ്മങ്ങള്‍ പ്രവര്‍ത്തിക്കുകയും ചെയ്തവരാരോ അവര്‍ക്കുള്ളതാണ് സുഖാനുഭൂതിയുടെ സ്വര്‍ഗത്തോപ്പുകള്‍.
\end{malayalam}}
\flushright{\begin{Arabic}
\quranayah[31][9]
\end{Arabic}}
\flushleft{\begin{malayalam}
അവര്‍ അതില്‍ നിത്യവാസികളായിരിക്കും. അല്ലാഹുവിന്‍റെ സത്യവാഗ്ദാനമത്രെ അത്‌. അവനാകുന്നു പ്രതാപിയും യുക്തിമാനും.
\end{malayalam}}
\flushright{\begin{Arabic}
\quranayah[31][10]
\end{Arabic}}
\flushleft{\begin{malayalam}
നിങ്ങള്‍ക്ക് കാണാവുന്ന തൂണുകളൊന്നും കൂടാതെ ആകാശങ്ങളെ അവന്‍ സൃഷ്ടിച്ചിരിക്കുന്നു. ഭൂമി നിങ്ങളെയും കൊണ്ട് ഇളകാതിരിക്കുവാനായി അതില്‍ അവന്‍ ഉറച്ച പര്‍വ്വതങ്ങള്‍ സ്ഥാപിക്കുകയും ചെയ്തിരിക്കുന്നു. എല്ലാതരം ജന്തുക്കളെയും അവന്‍ അതില്‍ പരത്തുകയും ചെയ്തിരിക്കുന്നു. ആകാശത്ത് നിന്ന് നാം വെള്ളം ചൊരിയുകയും, എന്നിട്ട് വിശിഷ്ടമായ എല്ലാ (സസ്യ) ജോടികളെയും നാം അതില്‍ മുളപ്പിക്കുകയും ചെയ്തു.
\end{malayalam}}
\flushright{\begin{Arabic}
\quranayah[31][11]
\end{Arabic}}
\flushleft{\begin{malayalam}
ഇതൊക്കെ അല്ലാഹുവിന്‍റെ സൃഷ്ടിയാകുന്നു. എന്നാല്‍ അവന്നു പുറമെയുള്ളവര്‍ സൃഷ്ടിച്ചിട്ടുള്ളത് എന്താണെന്ന് നിങ്ങള്‍ എനിക്ക് കാണിച്ചുതരൂ. അല്ല, അക്രമകാരികള്‍ വ്യക്തമായ വഴികേടിലാകുന്നു.
\end{malayalam}}
\flushright{\begin{Arabic}
\quranayah[31][12]
\end{Arabic}}
\flushleft{\begin{malayalam}
ലുഖ്മാന് നാം തത്വജ്ഞാനം നല്‍കുകയുണ്ടായി, നീ അല്ലാഹുവോട് നന്ദികാണിക്കുക. ആര്‍ നന്ദികാണിച്ചാലും തന്‍റെ ഗുണത്തിനായി തന്നെയാണ് അവന്‍ നന്ദികാണിക്കുന്നത്‌. വല്ലവനും നന്ദികേട് കാണിക്കുകയാണെങ്കില്‍ തീര്‍ച്ചയായും അല്ലാഹു പരാശ്രയമുക്തനും സ്തുത്യര്‍ഹനുമാകുന്നു. (എന്ന് അദ്ദേഹത്തോട് നാം അനുശാസിച്ചു.)
\end{malayalam}}
\flushright{\begin{Arabic}
\quranayah[31][13]
\end{Arabic}}
\flushleft{\begin{malayalam}
ലുഖ്മാന്‍ തന്‍റെ മകന് സദുപദേശം നല്‍കികൊണ്ടിരിക്കെ അവനോട് ഇപ്രകാരം പറഞ്ഞ സന്ദര്‍ഭം (ശ്രദ്ധേയമാകുന്നു:) എന്‍റെ കുഞ്ഞുമകനേ, നീ അല്ലാഹുവോട് പങ്കുചേര്‍ക്കരുത്‌. തീര്‍ച്ചയായും അങ്ങനെ പങ്കുചേര്‍ക്കുന്നത് വലിയ അക്രമം തന്നെയാകുന്നു.
\end{malayalam}}
\flushright{\begin{Arabic}
\quranayah[31][14]
\end{Arabic}}
\flushleft{\begin{malayalam}
മനുഷ്യന് തന്‍റെ മാതാപിതാക്കളുടെ കാര്യത്തില്‍ നാം അനുശാസനം നല്‍കിയിരിക്കുന്നു- ക്ഷീണത്തിനുമേല്‍ ക്ഷീണവുമായിട്ടാണ് മാതാവ് അവനെ ഗര്‍ഭം ചുമന്ന് നടന്നത്‌. അവന്‍റെ മുലകുടി നിര്‍ത്തുന്നതാകട്ടെ രണ്ടുവര്‍ഷം കൊണ്ടുമാണ്‌- എന്നോടും നിന്‍റെ മാതാപിതാക്കളോടും നീ നന്ദികാണിക്കൂ. എന്‍റെ അടുത്തേക്കാണ് (നിന്‍റെ) മടക്കം.
\end{malayalam}}
\flushright{\begin{Arabic}
\quranayah[31][15]
\end{Arabic}}
\flushleft{\begin{malayalam}
നിനക്ക് യാതൊരു അറിവുമില്ലാത്ത വല്ലതിനെയും എന്നോട് നീ പങ്കുചേര്‍ക്കുന്ന കാര്യത്തില്‍ അവര്‍ ഇരുവരും നിന്‍റെ മേല്‍ നിര്‍ബന്ധം ചെലുത്തുന്ന പക്ഷം അവരെ നീ അനുസരിക്കരുത്‌. ഇഹലോകത്ത് നീ അവരോട് നല്ലനിലയില്‍ സഹവസിക്കുകയും, എന്നിലേക്ക് മടങ്ങിയവരുടെ മാര്‍ഗം നീ പിന്തുടരുകയും ചെയ്യുക. പിന്നെ എന്‍റെ അടുത്തേക്കാകുന്നു നിങ്ങളുടെ മടക്കം. അപ്പോള്‍ നിങ്ങള്‍ പ്രവര്‍ത്തിച്ചിരുന്നതിനെപ്പറ്റി ഞാന്‍ നിങ്ങളെ വിവരമറിയിക്കുന്നതാണ്‌.
\end{malayalam}}
\flushright{\begin{Arabic}
\quranayah[31][16]
\end{Arabic}}
\flushleft{\begin{malayalam}
എന്‍റെ കുഞ്ഞുമകനേ, തീര്‍ച്ചയായും അത് (കാര്യം) ഒരു കടുക് മണിയുടെ തൂക്കമുള്ളതായിരുന്നാലും, എന്നിട്ടത് ഒരു പാറക്കല്ലിനുള്ളിലോ ആകാശങ്ങളിലോ ഭൂമിയിലോ എവിടെ തന്നെ ആയാലും അല്ലാഹു അത് കൊണ്ടുവരുന്നതാണ്‌. തീര്‍ച്ചയായും അല്ലാഹു നയജ്ഞനും സൂക്ഷ്മജ്ഞനുമാകുന്നു.
\end{malayalam}}
\flushright{\begin{Arabic}
\quranayah[31][17]
\end{Arabic}}
\flushleft{\begin{malayalam}
എന്‍റെ കുഞ്ഞുമകനേ, നീ നമസ്കാരം മുറപോലെ നിര്‍വഹിക്കുകയും സദാചാരം കല്‍പിക്കുകയും ദുരാചാരത്തില്‍ നിന്ന് വിലക്കുകയും, നിനക്ക് ബാധിച്ച വിഷമങ്ങളില്‍ ക്ഷമിക്കുകയും ചെയ്യുക. തീര്‍ച്ചയായും ഖണ്ഡിതമായി നിര്‍ദേശിക്കപ്പെട്ട കാര്യങ്ങളില്‍ പെട്ടതത്രെ അത്‌.
\end{malayalam}}
\flushright{\begin{Arabic}
\quranayah[31][18]
\end{Arabic}}
\flushleft{\begin{malayalam}
നീ (അഹങ്കാരത്തോടെ) മനുഷ്യരുടെ നേര്‍ക്ക് നിന്‍റെ കവിള്‍ തിരിച്ചുകളയരുത്‌. ഭൂമിയിലൂടെ നീ പൊങ്ങച്ചം കാട്ടി നടക്കുകയും അരുത്‌. ദുരഭിമാനിയും പൊങ്ങച്ചക്കാരനുമായ യാതൊരാളെയും അല്ലാഹു ഇഷ്ടപ്പെടുകയില്ല.
\end{malayalam}}
\flushright{\begin{Arabic}
\quranayah[31][19]
\end{Arabic}}
\flushleft{\begin{malayalam}
നിന്‍റെ നടത്തത്തില്‍ നീ മിതത്വം പാലിക്കുക. നിന്‍റെ ശബ്ദം നീ ഒതുക്കുകയും ചെയ്യുക. തീര്‍ച്ചയായും ശബ്ദങ്ങളുടെ കൂട്ടത്തില്‍ ഏറ്റവും വെറുപ്പുളവാക്കുന്നത് കഴുതയുടെ ശബ്ദമത്രെ.
\end{malayalam}}
\flushright{\begin{Arabic}
\quranayah[31][20]
\end{Arabic}}
\flushleft{\begin{malayalam}
ആകാശങ്ങളിലുള്ളതും ഭൂമിയിലുള്ളതും അല്ലാഹു നിങ്ങള്‍ക്ക് അധീനപ്പെടുത്തിത്തന്നിരിക്കുന്നു എന്ന് നിങ്ങള്‍ കണ്ടില്ലേ? പ്രത്യക്ഷവും പരോക്ഷവുമായ അവന്‍റെ അനുഗ്രഹങ്ങള്‍ അവന്‍ നിങ്ങള്‍ക്ക് നിറവേറ്റിത്തരികയും ചെയ്തിരിക്കുന്നു. വല്ല അറിവോ മാര്‍ഗദര്‍ശനമോ വെളിച്ചം നല്‍കുന്ന വേദഗ്രന്ഥമോ ഇല്ലാതെ അല്ലാഹുവിന്‍റെ കാര്യത്തില്‍ തര്‍ക്കിച്ച് കൊണ്ടിരിക്കുന്ന ചിലര്‍ മനുഷ്യരിലുണ്ട്‌.
\end{malayalam}}
\flushright{\begin{Arabic}
\quranayah[31][21]
\end{Arabic}}
\flushleft{\begin{malayalam}
അല്ലാഹു അവതരിപ്പിച്ചതിനെ നിങ്ങള്‍ പിന്തുടരൂ എന്ന് അവരോട് പറയപ്പെട്ടാല്‍, അല്ല, ഞങ്ങളുടെ പിതാക്കള്‍ എന്തൊന്നില്‍ നിലകൊള്ളുന്നതായി ഞങ്ങള്‍ കണ്ടുവോ അതിനെയാണ് ഞങ്ങള്‍ പിന്തുടരുക എന്നായിരിക്കും അവര്‍ പറയുക. പിശാച് ജ്വലിക്കുന്ന നരകശിക്ഷയിലേക്കാണ് അവരെ ക്ഷണിക്കുന്നതെങ്കില്‍ പോലും (അവരതിനെ പിന്തുടരുകയോ?)
\end{malayalam}}
\flushright{\begin{Arabic}
\quranayah[31][22]
\end{Arabic}}
\flushleft{\begin{malayalam}
വല്ലവനും സദ്‌വൃത്തനായിക്കൊണ്ട് തന്‍റെ മുഖത്തെ അല്ലാഹുവിന് സമര്‍പ്പിക്കുന്ന പക്ഷം ഏറ്റവും ഉറപ്പുള്ള പിടികയറില്‍ തന്നെയാണ് അവന്‍ പിടിച്ചിരിക്കുന്നത്‌. അല്ലാഹുവിങ്കലേക്കാകുന്നു കാര്യങ്ങളുടെ പരിണതി.
\end{malayalam}}
\flushright{\begin{Arabic}
\quranayah[31][23]
\end{Arabic}}
\flushleft{\begin{malayalam}
വല്ലവനും അവിശ്വസിച്ചുവെങ്കില്‍ അവന്‍റെ അവിശ്വാസം നിന്നെ ദുഃഖിപ്പിക്കാതിരിക്കട്ടെ. നമ്മുടെ അടുത്തേക്കാണ് അവരുടെ മടക്കം. അപ്പോള്‍ അവര്‍ പ്രവര്‍ത്തിച്ചതിനെപ്പറ്റി നാം അവരെ വിവരമറിയിക്കുന്നതാണ്‌. തീര്‍ച്ചയായും അല്ലാഹു ഹൃദയങ്ങളിലുള്ളതിനെപ്പറ്റി അറിവുള്ളവനാകുന്നു.
\end{malayalam}}
\flushright{\begin{Arabic}
\quranayah[31][24]
\end{Arabic}}
\flushleft{\begin{malayalam}
നാം അവര്‍ക്ക് അല്‍പം സുഖം അനുഭവിപ്പിക്കുന്നു. പിന്നെ കഠിനമായ ശിക്ഷയിലേക്ക് നാം അവരെ തള്ളിവിടുന്നതാണ്‌.
\end{malayalam}}
\flushright{\begin{Arabic}
\quranayah[31][25]
\end{Arabic}}
\flushleft{\begin{malayalam}
ആകാശങ്ങളും ഭൂമിയും സൃഷ്ടിച്ചത് ആരെന്ന് നീ അവരോട് ചോദിച്ചാല്‍ തീര്‍ച്ചയായും അവര്‍ പറയും: അല്ലാഹുവാണെന്ന്‌. പറയുക: അല്ലാഹുവിന് സ്തുതി. പക്ഷെ, അവരില്‍ അധികപേരും മനസ്സിലാക്കുന്നില്ല.
\end{malayalam}}
\flushright{\begin{Arabic}
\quranayah[31][26]
\end{Arabic}}
\flushleft{\begin{malayalam}
അല്ലാഹുവിന്‍റെതാകുന്നു ആകാശങ്ങളിലും ഭൂമിയിലുമുള്ളത്‌. തീര്‍ച്ചയായും അല്ലാഹു പരാശ്രയമുക്തനും, സ്തുത്യര്‍ഹനുമാകുന്നു.
\end{malayalam}}
\flushright{\begin{Arabic}
\quranayah[31][27]
\end{Arabic}}
\flushleft{\begin{malayalam}
ഭൂമിയിലുള്ള വൃക്ഷമെല്ലാം പേനയായിരിക്കുകയും സമുദ്രം മഷിയാകുകയും അതിനു പുറമെ ഏഴു സമുദ്രങ്ങള്‍ അതിനെ പോഷിപ്പിക്കുകയും ചെയ്താലും അല്ലാഹുവിന്‍റെ വചനങ്ങള്‍ എഴുതിത്തീരുകയില്ല. തീര്‍ച്ചയായും അല്ലാഹു പ്രതാപിയും യുക്തിമാനുമാകുന്നു.
\end{malayalam}}
\flushright{\begin{Arabic}
\quranayah[31][28]
\end{Arabic}}
\flushleft{\begin{malayalam}
നിങ്ങളെ സൃഷ്ടിക്കുന്നതും പുനരുജ്ജീവിപ്പിക്കുന്നതും ഒരൊറ്റ വ്യക്തിയെ (സൃഷ്ടിക്കുകയും പുനരുജ്ജീവിപ്പിക്കുകയും ചെയ്യുന്നത്‌) പോലെ മാത്രമാകുന്നു തീര്‍ച്ചയായും അല്ലാഹു എല്ലാം കേള്‍ക്കുകയും കാണുകയും ചെയ്യുന്നവനത്രെ.
\end{malayalam}}
\flushright{\begin{Arabic}
\quranayah[31][29]
\end{Arabic}}
\flushleft{\begin{malayalam}
അല്ലാഹു രാത്രിയെ പകലില്‍ പ്രവേശിപ്പിക്കുകയും, പകലിനെ രാത്രിയില്‍ പ്രവേശിപ്പിക്കുകയും ചെയ്യുന്നു എന്ന് നീ ചിന്തിച്ചു നോക്കിയിട്ടില്ലേ? അവന്‍ സൂര്യനെയും ചന്ദ്രനെയും അധീനപ്പെടുത്തുകയും ചെയ്തിരിക്കുന്നു. എല്ലാം നിര്‍ണിതമായ ഒരു അവധിവരെ സഞ്ചരിച്ചു കൊണ്ടിരിക്കുന്നു. നിങ്ങള്‍ പ്രവര്‍ത്തിക്കുന്നതിനെപ്പറ്റിയെല്ലാം അല്ലാഹു സൂക്ഷ്മമായി അറിയുന്നവനാണെന്നും (നീ ആലോചിച്ചിട്ടില്ലേ?)
\end{malayalam}}
\flushright{\begin{Arabic}
\quranayah[31][30]
\end{Arabic}}
\flushleft{\begin{malayalam}
അതെന്തുകൊണ്ടെന്നാല്‍ അല്ലാഹുവാണ് സത്യമായിട്ടുള്ളവന്‍. അവന്നു പുറമെ അവര്‍ വിളിച്ച് പ്രാര്‍ത്ഥിക്കുന്നതെല്ലാം വ്യര്‍ത്ഥമാകുന്നു. അല്ലാഹു തന്നെയാകുന്നു ഉന്നതനും വലിയവനും.
\end{malayalam}}
\flushright{\begin{Arabic}
\quranayah[31][31]
\end{Arabic}}
\flushleft{\begin{malayalam}
കടലിലൂടെ കപ്പലുകള്‍ സഞ്ചരിക്കുന്നത് അല്ലാഹുവിന്‍റെ അനുഗ്രഹം നിമിത്തമാണെന്ന് നീ കണ്ടില്ലേ? അവന്‍റെ ദൃഷ്ടാന്തങ്ങളില്‍ ചിലത് നിങ്ങള്‍ക്ക് കാണിച്ചുതരാന്‍ വേണ്ടിയത്രെ അത്‌. ക്ഷമാശീലരും നന്ദിയുള്ളവരുമായ ഏവര്‍ക്കും തീര്‍ച്ചയായും അതില്‍ ദൃഷ്ടാന്തങ്ങളുണ്ട്‌.
\end{malayalam}}
\flushright{\begin{Arabic}
\quranayah[31][32]
\end{Arabic}}
\flushleft{\begin{malayalam}
പര്‍വ്വതങ്ങള്‍ പോലുള്ള തിരമാല അവരെ മൂടിക്കളഞ്ഞാല്‍ കീഴ്‌വണക്കം അല്ലാഹുവിന് മാത്രമാക്കിക്കൊണ്ട് അവനോട് അവര്‍ പ്രാര്‍ത്ഥിക്കുന്നതാണ്‌. എന്നാല്‍ അവരെ അവന്‍ കരയിലേക്ക് രക്ഷപ്പെടുത്തുമ്പോളോ അവരില്‍ ചിലര്‍ മാത്രം മര്യാദ പാലിക്കുന്നവരായിരിക്കും. പരമവഞ്ചകന്‍മാരും നന്ദികെട്ടവരും ആരെല്ലാമോ, അവര്‍ മാത്രമേ നമ്മുടെ ദൃഷ്ടാന്തങ്ങള്‍ നിഷേധിക്കുകയുള്ളൂ.
\end{malayalam}}
\flushright{\begin{Arabic}
\quranayah[31][33]
\end{Arabic}}
\flushleft{\begin{malayalam}
മനുഷ്യരേ, നിങ്ങള്‍ നിങ്ങളുടെ രക്ഷിതാവിനെ സൂക്ഷിക്കുക. ഒരു പിതാവും തന്‍റെ സന്താനത്തിന് പ്രയോജനം ചെയ്യാത്ത, ഒരു സന്തതിയും പിതാവിന് ഒട്ടും പ്രയോജനകാരിയാവാത്ത ഒരു ദിവസത്തെ നിങ്ങള്‍ ഭയപ്പെടുകയും ചെയ്യുക. തീര്‍ച്ചയായും അല്ലാഹുവിന്‍റെ വാഗ്ദാനം സത്യമാകുന്നു. അതിനാല്‍ ഐഹികജീവിതം നിങ്ങളെ വഞ്ചിച്ചു കളയാതിരിക്കട്ടെ. പരമവഞ്ചകനായ പിശാചും അല്ലാഹുവിന്‍റെ കാര്യത്തില്‍ നിങ്ങളെ വഞ്ചിച്ചു കളയാതിരിക്കട്ടെ.
\end{malayalam}}
\flushright{\begin{Arabic}
\quranayah[31][34]
\end{Arabic}}
\flushleft{\begin{malayalam}
തീര്‍ച്ചയായും അല്ലാഹുവിന്‍റെ പക്കലാണ് അന്ത്യസമയത്തെപ്പറ്റിയുള്ള അറിവ്‌. അവന്‍ മഴപെയ്യിക്കുന്നു. ഗര്‍ഭാശയത്തിലുള്ളത് അവന്‍ അറിയുകയും ചെയ്യുന്നു. നാളെ താന്‍ എന്താണ് പ്രവര്‍ത്തിക്കുക എന്ന് ഒരാളും അറിയുകയില്ല. താന്‍ ഏത് നാട്ടില്‍ വെച്ചാണ് മരിക്കുക എന്നും ഒരാളും അറിയുകയില്ല. തീര്‍ച്ചയായും അല്ലാഹു സര്‍വ്വജ്ഞനും സൂക്ഷ്മജ്ഞാനിയുമാകുന്നു.
\end{malayalam}}
\chapter{\textmalayalam{സജദ ( സാഷ്ടാംഗം )}}
\begin{Arabic}
\Huge{\centerline{\basmalah}}\end{Arabic}
\flushright{\begin{Arabic}
\quranayah[32][1]
\end{Arabic}}
\flushleft{\begin{malayalam}
അലിഫ്‌-ലാം-മീം
\end{malayalam}}
\flushright{\begin{Arabic}
\quranayah[32][2]
\end{Arabic}}
\flushleft{\begin{malayalam}
ഈ ഗ്രന്ഥത്തിന്‍റെ അവതരണം സര്‍വ്വലോകരക്ഷിതാവിങ്കല്‍ നിന്നാകുന്നു. ഇതില്‍ യാതൊരു സംശയവുമില്ല.
\end{malayalam}}
\flushright{\begin{Arabic}
\quranayah[32][3]
\end{Arabic}}
\flushleft{\begin{malayalam}
അതല്ല, ഇത് അദ്ദേഹം കെട്ടിച്ചമച്ചു എന്നാണോ അവര്‍ പറയുന്നത്‌? അല്ല, അത് നിന്‍റെ രക്ഷിതാവിങ്കല്‍ നിന്നുള്ള സത്യമാകുന്നു. നിനക്ക് മുമ്പ് ഒരു താക്കീതുകാരനും വന്നിട്ടില്ലാത്ത ഒരു ജനതക്ക് താക്കീത് നല്‍കുവാന്‍ വേണ്ടിയത്രെ അത്‌. അവര്‍ സന്‍മാര്‍ഗം പ്രാപിച്ചേക്കാം.
\end{malayalam}}
\flushright{\begin{Arabic}
\quranayah[32][4]
\end{Arabic}}
\flushleft{\begin{malayalam}
ആകാശങ്ങളും ഭൂമിയും അവയ്ക്കിടയിലുള്ളതും ആറു ദിവസങ്ങളില്‍ (ഘട്ടങ്ങളില്‍) സൃഷ്ടിച്ചവനാകുന്നു അല്ലാഹു. പിന്നീട് അവന്‍ സിംഹാസനസ്ഥനായി. അവന്നു പുറമെ നിങ്ങള്‍ക്ക് യാതൊരു രക്ഷാധികാരിയും ശുപാര്‍ശകനുമില്ല. എന്നിരിക്കെ നിങ്ങള്‍ ആലോചിച്ച് ഗ്രഹിക്കുന്നില്ലേ?
\end{malayalam}}
\flushright{\begin{Arabic}
\quranayah[32][5]
\end{Arabic}}
\flushleft{\begin{malayalam}
അവന്‍ ആകാശത്ത് നിന്ന് ഭൂമിയിലേക്ക് കാര്യങ്ങള്‍ നിയന്ത്രിച്ചയക്കുന്നു. പിന്നീട് ഒരു ദിവസം കാര്യം അവങ്കലേക്ക് ഉയര്‍ന്ന് പോകുന്നു. നിങ്ങള്‍ കണക്കാക്കുന്ന തരത്തിലുള്ള ആയിരം വര്‍ഷമാകുന്നു ആ ദിവസത്തിന്‍റെ അളവ്‌.
\end{malayalam}}
\flushright{\begin{Arabic}
\quranayah[32][6]
\end{Arabic}}
\flushleft{\begin{malayalam}
അദൃശ്യവും ദൃശ്യവും അറിയുന്നവനും പ്രതാപിയും കരുണാനിധിയുമാകുന്നു അവന്‍.
\end{malayalam}}
\flushright{\begin{Arabic}
\quranayah[32][7]
\end{Arabic}}
\flushleft{\begin{malayalam}
താന്‍ സൃഷ്ടിച്ച എല്ലാ വസ്തുക്കളെയും വിശിഷ്ടമാക്കിയവനത്രെ അവന്‍. മനുഷ്യന്‍റെ സൃഷ്ടി കളിമണ്ണില്‍ നിന്ന് അവന്‍ ആരംഭിച്ചു.
\end{malayalam}}
\flushright{\begin{Arabic}
\quranayah[32][8]
\end{Arabic}}
\flushleft{\begin{malayalam}
പിന്നെ അവന്‍റെ സന്തതിയെ നിസ്സാരമായ ഒരു വെള്ളത്തിന്‍റെ സത്തില്‍ നിന്ന് അവന്‍ ഉണ്ടാക്കി.
\end{malayalam}}
\flushright{\begin{Arabic}
\quranayah[32][9]
\end{Arabic}}
\flushleft{\begin{malayalam}
പിന്നെ അവനെ ശരിയായ രൂപത്തിലാക്കുകയും, തന്‍റെ വകയായുള്ള ആത്മാവ് അവനില്‍ ഊതുകയും ചെയ്തു. നിങ്ങള്‍ക്കവന്‍ കേള്‍വിയും കാഴ്ചകളും ഹൃദയങ്ങളും ഉണ്ടാക്കിത്തരികയും ചെയ്തു. കുറച്ച് മാത്രമേ നിങ്ങള്‍ നന്ദികാണിക്കുന്നുള്ളൂ.
\end{malayalam}}
\flushright{\begin{Arabic}
\quranayah[32][10]
\end{Arabic}}
\flushleft{\begin{malayalam}
അവര്‍ (അവിശ്വാസികള്‍) പറഞ്ഞു: ഞങ്ങള്‍ ഭൂമിയില്‍ ലയിച്ച് അപ്രത്യക്ഷരായാല്‍ പോലും ഞങ്ങള്‍ പുതുതായി സൃഷ്ടിക്കപ്പെടുമെന്നോ? അല്ല, അവര്‍ തങ്ങളുടെ രക്ഷിതാവിനെ കണ്ടുമുട്ടുന്നതിനെ നിഷേധിക്കുന്നവരാകുന്നു.
\end{malayalam}}
\flushright{\begin{Arabic}
\quranayah[32][11]
\end{Arabic}}
\flushleft{\begin{malayalam}
(നബിയേ,) പറയുക: നിങ്ങളുടെ കാര്യത്തില്‍ ഏല്‍പിക്കപ്പെട്ട മരണത്തിന്‍റെ മലക്ക് നിങ്ങളെ മരിപ്പിക്കുന്നതാണ്‌. പിന്നീട് നിങ്ങളുടെ രക്ഷിതാവിങ്കലേക്ക് മടക്കപ്പെടുന്നതുമാണ്‌.
\end{malayalam}}
\flushright{\begin{Arabic}
\quranayah[32][12]
\end{Arabic}}
\flushleft{\begin{malayalam}
കുറ്റവാളികള്‍ തങ്ങളുടെ രക്ഷിതാവിന്‍റെ അടുക്കല്‍ തല താഴ്ത്തിക്കൊണ്ട,് ഞങ്ങളുടെ രക്ഷിതാവേ, ഞങ്ങളിതാ (നേരില്‍) കാണുകയും കേള്‍ക്കുകയും ചെയ്തിരിക്കുന്നു. അതിനാല്‍ ഞങ്ങളെ നീ തിരിച്ചയച്ചുതരേണമേ. എങ്കില്‍ ഞങ്ങള്‍ നല്ലത് പ്രവര്‍ത്തിച്ച് കൊള്ളാം. തീര്‍ച്ചയായും ഞങ്ങളിപ്പോള്‍ ദൃഢവിശ്വാസമുള്ളവരാകുന്നു. എന്ന് പറയുന്ന സന്ദര്‍ഭം നീ കാണുകയാണെങ്കില്‍ (അതെന്തൊരു കാഴ്ചയായിരിക്കും!)
\end{malayalam}}
\flushright{\begin{Arabic}
\quranayah[32][13]
\end{Arabic}}
\flushleft{\begin{malayalam}
നാം ഉദ്ദേശിച്ചിരുന്നെങ്കില്‍ ഓരോ ആള്‍ക്കും തന്‍റെ സന്‍മാര്‍ഗം നാം നല്‍കുമായിരുന്നു. എന്നാല്‍ ജിന്നുകള്‍, മനുഷ്യര്‍ എന്നീ രണ്ടുവിഭാഗത്തെയും കൊണ്ട് ഞാന്‍ നരകം നിറക്കുക തന്നെചെയ്യും. എന്ന എന്‍റെ പക്കല്‍ നിന്നുള്ള വാക്ക് സ്ഥിരപ്പെട്ട് കഴിഞ്ഞിരിക്കുന്നു.
\end{malayalam}}
\flushright{\begin{Arabic}
\quranayah[32][14]
\end{Arabic}}
\flushleft{\begin{malayalam}
ആകയാല്‍ നിങ്ങളുടെ ഈ ദിവസത്തെ കണ്ടുമുട്ടുന്ന കാര്യം നിങ്ങള്‍ മറന്നുകളഞ്ഞതിന്‍റെ ഫലമായി നിങ്ങള്‍ ശിക്ഷ ആസ്വദിച്ച് കൊള്ളുക. തീര്‍ച്ചയായും നിങ്ങളെ നാം മറന്നുകളഞ്ഞിരിക്കുന്നു. നിങ്ങള്‍ പ്രവര്‍ത്തിച്ച് ക്കൊണ്ടിരുന്നതിന്‍റെ ഫലമായി ശാശ്വതമായ ശിക്ഷ നിങ്ങള്‍ ആസ്വദിച്ച് കൊള്ളുക.
\end{malayalam}}
\flushright{\begin{Arabic}
\quranayah[32][15]
\end{Arabic}}
\flushleft{\begin{malayalam}
നമ്മുടെ ദൃഷ്ടാന്തങ്ങള്‍ മുഖേന ഉല്‍ബോധനം നല്‍കപ്പെട്ടാല്‍ സാഷ്ടാംഗം പ്രണമിക്കുന്നവരായി വീഴുകയും, തങ്ങളുടെ രക്ഷിതാവിനെ സ്തുതിച്ചു കൊണ്ട് പ്രകീര്‍ത്തിക്കുകയും ചെയ്യുന്നവര്‍ മാത്രമേ നമ്മുടെ ദൃഷ്ടാന്തങ്ങളില്‍ വിശ്വസിക്കുകയുള്ളൂ. അവര്‍ അഹംഭാവം നടിക്കുകയുമില്ല.
\end{malayalam}}
\flushright{\begin{Arabic}
\quranayah[32][16]
\end{Arabic}}
\flushleft{\begin{malayalam}
ഭയത്തോടും പ്രത്യാശയോടും കൂടി തങ്ങളുടെ രക്ഷിതാവിനോട് പ്രാര്‍ത്ഥിക്കുവാനായി, കിടന്നുറങ്ങുന്ന സ്ഥലങ്ങള്‍ വിട്ട് അവരുടെ പാര്‍ശ്വങ്ങള്‍ അകലുന്നതാണ്‌. അവര്‍ക്ക് നാം നല്‍കിയതില്‍ നിന്ന് അവര്‍ ചെലവഴിക്കുകയും ചെയ്യും.
\end{malayalam}}
\flushright{\begin{Arabic}
\quranayah[32][17]
\end{Arabic}}
\flushleft{\begin{malayalam}
എന്നാല്‍ അവര്‍ പ്രവര്‍ത്തിച്ചിരുന്നതിനുള്ള പ്രതിഫലമായിക്കൊണ്ട് കണ്‍കുളിര്‍പ്പിക്കുന്ന എന്തെല്ലാം കാര്യങ്ങളാണ് അവര്‍ക്ക് വേണ്ടി രഹസ്യമാക്കിവെക്കപ്പെട്ടിട്ടുള്ളത് എന്ന് ഒരാള്‍ക്കും അറിയാവുന്നതല്ല.
\end{malayalam}}
\flushright{\begin{Arabic}
\quranayah[32][18]
\end{Arabic}}
\flushleft{\begin{malayalam}
അപ്പോള്‍ വിശ്വാസിയായിക്കഴിഞ്ഞവന്‍ ധിക്കാരിയായിക്കഴിഞ്ഞവനെപ്പോലെയാണോ? അവര്‍ തുല്യരാകുകയില്ല.
\end{malayalam}}
\flushright{\begin{Arabic}
\quranayah[32][19]
\end{Arabic}}
\flushleft{\begin{malayalam}
എന്നാല്‍ വിശ്വസിക്കുകയും സല്‍കര്‍മ്മങ്ങള്‍ പ്രവര്‍ത്തിക്കുകയും ചെയ്തവരാരോ അവര്‍ക്കാണ് -തങ്ങള്‍ പ്രവര്‍ത്തിച്ചിരുന്നതിന്‍റെ പേരില്‍ ആതിഥ്യമായിക്കൊണ്ട്‌- താമസിക്കുവാന്‍ സ്വര്‍ഗത്തോപ്പുകളുള്ളത്‌.
\end{malayalam}}
\flushright{\begin{Arabic}
\quranayah[32][20]
\end{Arabic}}
\flushleft{\begin{malayalam}
എന്നാല്‍ ധിക്കാരം കാണിച്ചവരാരോ അവരുടെ വാസസ്ഥലം നരകമാകുന്നു. അവര്‍ അതില്‍ നിന്ന് പുറത്ത് കടക്കാന്‍ ഉദ്ദേശിക്കുമ്പോഴൊക്കെ അതിലേക്ക് തന്നെ അവര്‍ തിരിച്ചയക്കപ്പെടുന്നതാണ്‌. നിങ്ങള്‍ നിഷേധിച്ച് തള്ളിക്കളഞ്ഞിരുന്ന ആ നരകത്തിലെ ശിക്ഷ നിങ്ങള്‍ ആസ്വദിച്ച് കൊള്ളുക.എന്ന് അവരോട് പറയപ്പെടുകയും ചെയ്യും.
\end{malayalam}}
\flushright{\begin{Arabic}
\quranayah[32][21]
\end{Arabic}}
\flushleft{\begin{malayalam}
ഏറ്റവും വലിയ ആ ശിക്ഷ കൂടാതെ (ഐഹികമായ) ചില ചെറിയതരം ശിക്ഷകളും നാം അവരെ ആസ്വദിപ്പിക്കുന്നതാണ്‌. അവര്‍ ഒരു വേള മടങ്ങിയേക്കാമല്ലോ.
\end{malayalam}}
\flushright{\begin{Arabic}
\quranayah[32][22]
\end{Arabic}}
\flushleft{\begin{malayalam}
തന്‍റെ രക്ഷിതാവിന്‍റെ ദൃഷ്ടാന്തങ്ങളെപ്പറ്റി ഉല്‍ബോധനം നല്‍കപ്പെട്ടിട്ട് അവയില്‍ നിന്ന് തിരിഞ്ഞുകളഞ്ഞവനെക്കാള്‍ അക്രമിയായി ആരുണ്ട്‌? തീര്‍ച്ചയായും അത്തരം കുറ്റവാളികളുടെ പേരില്‍ നാം ശിക്ഷാനടപടിയെടുക്കുന്നതാണ്‌.
\end{malayalam}}
\flushright{\begin{Arabic}
\quranayah[32][23]
\end{Arabic}}
\flushleft{\begin{malayalam}
തീര്‍ച്ചയായും മൂസായ്ക്ക് നാം വേദഗ്രന്ഥം നല്‍കിയിട്ടുണ്ട്‌. അതിനാല്‍ അത് കണ്ടെത്തുന്നതിനെ പറ്റി നീ സംശയത്തിലാകരുത്‌. ഇസ്രായീല്‍ സന്തതികള്‍ക്ക് നാം അതിനെ മാര്‍ഗദര്‍ശകമാക്കുകയും ചെയ്തു.
\end{malayalam}}
\flushright{\begin{Arabic}
\quranayah[32][24]
\end{Arabic}}
\flushleft{\begin{malayalam}
അവര്‍ ക്ഷമ കൈക്കൊള്ളുകയും നമ്മുടെ ദൃഷ്ടാന്തങ്ങളില്‍ ദൃഢമായി വിശ്വസിക്കുന്നവരാകുകയും ചെയ്തപ്പോള്‍ അവരില്‍ നിന്ന് നമ്മുടെ കല്‍പന അനുസരിച്ച് മാര്‍ഗദര്‍ശനം നല്‍കുന്ന നേതാക്കളെ നാം ഉണ്ടാക്കുകയും ചെയ്തു.
\end{malayalam}}
\flushright{\begin{Arabic}
\quranayah[32][25]
\end{Arabic}}
\flushleft{\begin{malayalam}
അവര്‍ ഭിന്നത പുലര്‍ത്തിയിരുന്ന വിഷയങ്ങളില്‍ നിന്‍റെ രക്ഷിതാവ് തന്നെ ഉയിര്‍ത്തെഴുന്നേല്‍പിന്‍റെ നാളില്‍ അവര്‍ക്കിടയില്‍ തീര്‍പ്പുകല്‍പിക്കുന്നതാണ്‌; തീര്‍ച്ച.
\end{malayalam}}
\flushright{\begin{Arabic}
\quranayah[32][26]
\end{Arabic}}
\flushleft{\begin{malayalam}
ഇവര്‍ക്ക് മുമ്പ് നാം പല തലമുറകളെയും നശിപ്പിച്ചിട്ടുണ്ട്‌. എന്ന വസ്തുത ഇവര്‍ക്ക് നേര്‍വഴി കാണിച്ചില്ലേ? അവരുടെ വാസസ്ഥലങ്ങളിലൂടെ ഇവര്‍ സഞ്ചരിച്ച് കൊണ്ടിരിക്കുന്നല്ലോ. തീര്‍ച്ചയായും അതില്‍ ദൃഷ്ടാന്തങ്ങളുണ്ട്‌. എന്നിട്ടും ഇവര്‍ കേട്ട് മനസ്സിലാക്കുന്നില്ലേ?
\end{malayalam}}
\flushright{\begin{Arabic}
\quranayah[32][27]
\end{Arabic}}
\flushleft{\begin{malayalam}
വരണ്ട ഭൂമിയിലേക്ക് നാം വെള്ളം കൊണ്ടുചെല്ലുകയും, അത് മൂലം ഇവരുടെ കന്നുകാലികള്‍ക്കും ഇവര്‍ക്കുതന്നെയും തിന്നാനുള്ള കൃഷി നാം ഉല്‍പാദിപ്പിക്കുകയും ചെയ്യുന്നു എന്ന് ഇവര്‍ കണ്ടില്ലേ? എന്നിട്ടും ഇവര്‍ കണ്ടറിയുന്നില്ലേ?
\end{malayalam}}
\flushright{\begin{Arabic}
\quranayah[32][28]
\end{Arabic}}
\flushleft{\begin{malayalam}
അവര്‍ പറയുന്നു: എപ്പോഴാണ് ഈ തീരുമാനം? (പറയൂ) നിങ്ങള്‍ സത്യവാന്‍മാരാണെങ്കില്‍.
\end{malayalam}}
\flushright{\begin{Arabic}
\quranayah[32][29]
\end{Arabic}}
\flushleft{\begin{malayalam}
(നബിയേ,) പറയുക: അവിശ്വസിച്ചിരുന്ന ആളുകള്‍ക്ക് ആ തീരുമാനത്തിന്‍റെ ദിവസം തങ്ങള്‍ വിശ്വസിക്കുന്നത് കൊണ്ട് പ്രയോജനം ഉണ്ടാവുകയില്ല. അവര്‍ക്ക് അവധി നല്‍കപ്പെടുകയുമില്ല.
\end{malayalam}}
\flushright{\begin{Arabic}
\quranayah[32][30]
\end{Arabic}}
\flushleft{\begin{malayalam}
അതിനാല്‍ നീ അവരില്‍ നിന്ന് തിരിഞ്ഞുകളയുകയും കാത്തിരിക്കുകയും ചെയ്യുകഠീര്‍ച്ചയായും അവര്‍ കാത്തിരിക്കുന്നവരാണല്ലോ.
\end{malayalam}}
\chapter{\textmalayalam{അഹ്സാബ് (സംഘടിത കക്ഷികള്‍ )}}
\begin{Arabic}
\Huge{\centerline{\basmalah}}\end{Arabic}
\flushright{\begin{Arabic}
\quranayah[33][1]
\end{Arabic}}
\flushleft{\begin{malayalam}
(നബിയേ,) നീ അല്ലാഹുവെ സൂക്ഷിക്കുക. സത്യനിഷേധികളെയും കപടവിശ്വാസികളെയും അനുസരിക്കാതിരിക്കുകയും ചെയ്യുക. തീര്‍ച്ചയായും അല്ലാഹു സര്‍വ്വജ്ഞനും യുക്തിമാനുമാകുന്നു.
\end{malayalam}}
\flushright{\begin{Arabic}
\quranayah[33][2]
\end{Arabic}}
\flushleft{\begin{malayalam}
നിനക്ക് നിന്‍റെ രക്ഷിതാവിങ്കല്‍ നിന്ന് ബോധനം നല്‍കപ്പെടുന്നതിനെ പിന്തുടരുകയും ചെയ്യുക. തീര്‍ച്ചയായും അല്ലാഹു നിങ്ങള്‍ പ്രവര്‍ത്തിക്കുന്നതിനെപ്പറ്റി സൂക്ഷ്മമായി അറിയുവന്നവനാകുന്നു.
\end{malayalam}}
\flushright{\begin{Arabic}
\quranayah[33][3]
\end{Arabic}}
\flushleft{\begin{malayalam}
അല്ലാഹുവെ നീ ഭരമേല്‍പിക്കുകയും ചെയ്യുക. കൈകാര്യകര്‍ത്താവായി അല്ലാഹു തന്നെ മതി.
\end{malayalam}}
\flushright{\begin{Arabic}
\quranayah[33][4]
\end{Arabic}}
\flushleft{\begin{malayalam}
യാതൊരു മനുഷ്യന്നും അവന്‍റെ ഉള്ളില്‍ അല്ലാഹു രണ്ടു ഹൃദങ്ങളുണ്ടാക്കിയിട്ടില്ല. നിങ്ങള്‍ നിങ്ങളുടെ മാതാക്കളെപ്പോലെയായി പ്രഖ്യാപിക്കുന്ന നിങ്ങളുടെ ഭാര്യമാരെ അവന്‍ നിങ്ങളുടെ മാതാക്കളാക്കിയിട്ടുമില്ല. നിങ്ങളിലേക്ക് ചേര്‍ത്തുവിളിക്കപ്പെടുന്ന നിങ്ങളുടെ ദത്തുപുത്രന്‍മാരെ അവന്‍ നിങ്ങളുടെ പുത്രന്‍മാരാക്കിയിട്ടുമില്ല. അതൊക്കെ നിങ്ങളുടെ വായ്കൊണ്ടു നിങ്ങള്‍ പറയുന്ന വാക്ക് മാത്രമാകുന്നു. അല്ലാഹു സത്യം പറയുന്നു. അവന്‍ നേര്‍വഴി കാണിച്ചുതരികയും ചെയ്യുന്നു.
\end{malayalam}}
\flushright{\begin{Arabic}
\quranayah[33][5]
\end{Arabic}}
\flushleft{\begin{malayalam}
നിങ്ങള്‍ അവരെ (ദത്തുപുത്രന്‍മാരെ) അവരുടെ പിതാക്കളിലേക്ക് ചേര്‍ത്ത് വിളിക്കുക. അതാണ് അല്ലാഹുവിന്‍റെ അടുക്കല്‍ ഏറ്റവും നീതിപൂര്‍വ്വകമായിട്ടുള്ളത്‌. ഇനി അവരുടെ പിതാക്കളെ നിങ്ങള്‍ അറിയില്ലെങ്കില്‍ അവര്‍ മതത്തില്‍ നിങ്ങളുടെ സഹോദരങ്ങളും മിത്രങ്ങളുമാകുന്നു. അബദ്ധവശാല്‍ നിങ്ങള്‍ ചെയ്തു പോയതില്‍ നിങ്ങള്‍ക്ക് കുറ്റമില്ല. പക്ഷെ നിങ്ങളുടെ ഹൃദയങ്ങള്‍ അറിഞ്ഞ്കൊണ്ടു ചെയ്തത് (കുറ്റകരമാകുന്നു.) അല്ലാഹു ഏറെ പൊറുക്കുന്നവനും കരുണാനിധിയുമാകുന്നു.
\end{malayalam}}
\flushright{\begin{Arabic}
\quranayah[33][6]
\end{Arabic}}
\flushleft{\begin{malayalam}
പ്രവാചകന്‍ സത്യവിശ്വാസികള്‍ക്ക് സ്വദേഹങ്ങളെക്കാളും അടുത്ത ആളാകുന്നു. അദ്ദേഹത്തിന്‍റെ ഭാര്യമാര്‍ അവരുടെ മാതാക്കളുമാകുന്നു. രക്തബന്ധമുള്ളവര്‍ അന്യോന്യം അല്ലാഹുവിന്‍റെ നിയമത്തില്‍ മറ്റു വിശ്വാസികളെക്കാളും മുഹാജിറുകളെക്കാളും കൂടുതല്‍ അടുപ്പമുള്ളവരാകുന്നു. നിങ്ങള്‍ നിങ്ങളുടെ മിത്രങ്ങള്‍ക്ക് വല്ല ഉപകാരവും ചെയ്യുന്നുവെങ്കില്‍ അത് ഇതില്‍ നിന്ന് ഒഴിവാകുന്നു. അത് വേദഗ്രന്ഥത്തില്‍ രേഖപ്പെടുത്തപ്പെട്ടതാകുന്നു.
\end{malayalam}}
\flushright{\begin{Arabic}
\quranayah[33][7]
\end{Arabic}}
\flushleft{\begin{malayalam}
പ്രവാചകന്‍മാരില്‍ നിന്ന് തങ്ങളുടെ കരാര്‍ നാം വാങ്ങിയ സന്ദര്‍ഭം (ശ്രദ്ധേയമാണ്‌.) നിന്‍റെ പക്കല്‍ നിന്നും നൂഹ്‌, ഇബ്രാഹീം, മൂസാ, മര്‍യമിന്‍റെ മകന്‍ ഈസാ എന്നിവരില്‍ നിന്നും (നാം കരാര്‍ വാങ്ങിയ സന്ദര്‍ഭം.) ഗൌരവമുള്ള ഒരു കരാറാണ് അവരില്‍ നിന്നെല്ലാം നാം വാങ്ങിയത്‌.
\end{malayalam}}
\flushright{\begin{Arabic}
\quranayah[33][8]
\end{Arabic}}
\flushleft{\begin{malayalam}
അവന് സത്യവാന്‍മാരോട് അവരുടെ സത്യസന്ധതയെപ്പറ്റി ചോദിക്കുവാന്‍ വേണ്ടിയത്രെ അത്‌. സത്യനിഷേധികള്‍ക്ക് അവന്‍ വേദനയേറിയ ശിക്ഷ ഒരുക്കിവെക്കുകയും ചെയ്തിരിക്കുന്നു.
\end{malayalam}}
\flushright{\begin{Arabic}
\quranayah[33][9]
\end{Arabic}}
\flushleft{\begin{malayalam}
സത്യവിശ്വാസികളേ, നിങ്ങളുടെ അടുത്ത് കുറെ സൈന്യങ്ങള്‍ വരികയും, അപ്പോള്‍ അവരുടെ നേരെ ഒരു കാറ്റും, നിങ്ങള്‍ കാണാത്ത സൈന്യങ്ങളേയും അയക്കുകയും ചെയ്ത സന്ദര്‍ഭത്തില്‍ അല്ലാഹു നിങ്ങള്‍ക്ക് ചെയ്തു തന്ന അനുഗ്രഹം നിങ്ങള്‍ ഓര്‍മിക്കുക. അല്ലാഹു നിങ്ങള്‍ പ്രവര്‍ത്തിക്കുന്നത് കണ്ടറിയുന്നവനാകുന്നു.
\end{malayalam}}
\flushright{\begin{Arabic}
\quranayah[33][10]
\end{Arabic}}
\flushleft{\begin{malayalam}
നിങ്ങളുടെ മുകള്‍ ഭാഗത്തു കൂടിയും നിങ്ങളുടെ താഴ്ഭാഗത്തു കൂടിയും അവര്‍ നിങ്ങളുടെ അടുക്കല്‍ വന്ന സന്ദര്‍ഭം. ദൃഷ്ടികള്‍ തെന്നിപ്പോകുകയും, ഹൃദയങ്ങള്‍ തൊണ്ടയിലെത്തുകയും, നിങ്ങള്‍ അല്ലാഹുവെപ്പറ്റി പല ധാരണകളും ധരിച്ച് പോകുകയും ചെയ്തിരുന്ന സന്ദര്‍ഭം.
\end{malayalam}}
\flushright{\begin{Arabic}
\quranayah[33][11]
\end{Arabic}}
\flushleft{\begin{malayalam}
അവിടെ വെച്ച് വിശ്വാസികള്‍ പരീക്ഷിക്കപ്പെടുകയും അവര്‍ കിടുകിടെ വിറപ്പിക്കപ്പെടുകയും ചെയ്തു.
\end{malayalam}}
\flushright{\begin{Arabic}
\quranayah[33][12]
\end{Arabic}}
\flushleft{\begin{malayalam}
നമ്മോട് അല്ലാഹുവും അവന്‍റെ ദൂതനും വാഗ്ദാനം ചെയ്തത് വഞ്ചനമാത്രമാണെന്ന് കപടവിശ്വാസികളും ഹൃദയങ്ങളില്‍ രോഗമുള്ളവരും പറയുകയും ചെയ്തിരുന്ന സന്ദര്‍ഭം.
\end{malayalam}}
\flushright{\begin{Arabic}
\quranayah[33][13]
\end{Arabic}}
\flushleft{\begin{malayalam}
യഥ്‌രിബുകാരേ! നിങ്ങള്‍ക്കു നില്‍ക്കക്കള്ളിയില്ല. അതിനാല്‍ നിങ്ങള്‍ മടങ്ങിക്കളയൂ. എന്ന് അവരില്‍ ഒരു വിഭാഗം പറയുകയും ചെയ്ത സന്ദര്‍ഭം. ഞങ്ങളുടെ വീടുകള്‍ ഭദ്രതയില്ലാത്തതാകുന്നു എന്ന് പറഞ്ഞു കൊണ്ട് അവരില്‍ ഒരു വിഭാഗം (യുദ്ധരംഗം വിട്ടുപോകാന്‍) നബിയോട് അനുവാദം തേടുകയും ചെയ്യുന്നു. യഥാര്‍ത്ഥത്തില്‍ അവ ഭദ്രതയില്ലാത്തതല്ല. അവര്‍ ഓടിക്കളയാന്‍ ഉദ്ദേശിക്കുന്നുവെന്ന് മാത്രം.
\end{malayalam}}
\flushright{\begin{Arabic}
\quranayah[33][14]
\end{Arabic}}
\flushleft{\begin{malayalam}
അതിന്‍റെ (മദീനയുടെ) വിവിധ ഭാഗങ്ങളിലൂടെ (ശത്രുക്കള്‍) അവരുടെ അടുത്ത് കടന്നു ചെല്ലുകയും, എന്നിട്ട് (മുസ്ലിംകള്‍ക്കെതിരില്‍) കുഴപ്പമുണ്ടാക്കാന്‍ അവരോട് ആവശ്യപ്പെടുകയുമാണെങ്കില്‍ അവരത് ചെയ്തു കൊടുക്കുന്നതാണ്‌. അവരതിന് താമസം വരുത്തുകയുമില്ല. കുറച്ച് മാത്രമല്ലാതെ.
\end{malayalam}}
\flushright{\begin{Arabic}
\quranayah[33][15]
\end{Arabic}}
\flushleft{\begin{malayalam}
തങ്ങള്‍ പിന്തിരിഞ്ഞ് പോകുകയില്ലെന്ന് മുമ്പ് അവര്‍ അല്ലാഹുവോട് ഉടമ്പടി ചെയ്തിട്ടുണ്ടായിരുന്നു. അല്ലാഹുവിന്‍റെ ഉടമ്പടി ചോദ്യം ചെയ്യപ്പെടുന്നതാണ്‌.
\end{malayalam}}
\flushright{\begin{Arabic}
\quranayah[33][16]
\end{Arabic}}
\flushleft{\begin{malayalam}
(നബിയേ,) പറയുക: മരണത്തില്‍ നിന്നോ കൊലയില്‍ നിന്നോ നിങ്ങള്‍ ഓടിക്കളയുകയാണെങ്കില്‍ ആ ഓട്ടം നിങ്ങള്‍ക്ക് പ്രയോജനപ്പെടുകയില്ല. അങ്ങനെ (ഓടിരക്ഷപ്പെട്ടാലും) അല്‍പമല്ലാതെ നിങ്ങള്‍ക്ക് ജീവിതസുഖം നല്‍കപ്പെടുകയില്ല.
\end{malayalam}}
\flushright{\begin{Arabic}
\quranayah[33][17]
\end{Arabic}}
\flushleft{\begin{malayalam}
പറയുക: അല്ലാഹു നിങ്ങള്‍ക്ക് വല്ല ദോഷവും വരുത്താന്‍ ഉദ്ദേശിച്ചിട്ടുണ്ടെങ്കില്‍ - അഥവാ അവന്‍ നിങ്ങള്‍ക്ക് വല്ല കാരുണ്യവും നല്‍കാന്‍ ഉദ്ദേശിച്ചിട്ടുണ്ടെങ്കില്‍ - അല്ലാഹുവില്‍ നിന്ന് നിങ്ങളെ കാത്തുരക്ഷിക്കാന്‍ ആരാണുള്ളത്‌? തങ്ങള്‍ക്ക് അല്ലാഹുവിനു പുറമെ യാതൊരു രക്ഷാധികാരിയേയും സഹായിയേയും അവര്‍ കണ്ടെത്തുകയില്ല.
\end{malayalam}}
\flushright{\begin{Arabic}
\quranayah[33][18]
\end{Arabic}}
\flushleft{\begin{malayalam}
നിങ്ങളുടെ കൂട്ടത്തിലുള്ള മുടക്കികളെയും തങ്ങളുടെ സഹോദരങ്ങളോട് ഞങ്ങളുടെ അടുത്തേക്ക് വരൂ എന്ന് പറയുന്നവരെയും അല്ലാഹു അറിയുന്നുണ്ട്‌. ചുരുക്കത്തിലല്ലാതെ അവര്‍ യുദ്ധത്തിന് ചെല്ലുകയില്ല.
\end{malayalam}}
\flushright{\begin{Arabic}
\quranayah[33][19]
\end{Arabic}}
\flushleft{\begin{malayalam}
നിങ്ങള്‍ക്കെതിരില്‍ പിശുക്ക് കാണിക്കുന്നവരായിരിക്കും അവര്‍. അങ്ങനെ (യുദ്ധ) ഭയം വന്നാല്‍ അവര്‍ നിന്നെ ഉറ്റുനോക്കുന്നതായി നിനക്ക് കാണാം. മരണവെപ്രാളം കാണിക്കുന്ന ഒരാളെപ്പോലെ അവരുടെ കണ്ണുകള്‍ കറങ്ങിക്കൊണ്ടിരിക്കും. എന്നാല്‍ (യുദ്ധ) ഭയം നീങ്ങിപ്പോയാലോ, ധനത്തില്‍ ദുര്‍മോഹം പൂണ്ടവരായിക്കൊണ്ട് മൂര്‍ച്ചയേറിയ നാവുകള്‍ കൊണ്ട് അവര്‍ നിങ്ങളെ കുത്തിപ്പറയുകയും ചെയ്യും. അത്തരക്കാര്‍ വിശ്വസിച്ചിട്ടില്ല. അതിനാല്‍ അല്ലാഹു അവരുടെ കര്‍മ്മങ്ങള്‍ നിഷ്ഫലമാക്കിയിരിക്കുന്നു. അത് അല്ലാഹുവെ സംബന്ധിച്ചേടത്തോളം നിസ്സാരമായ ഒരു കാര്യമാകുന്നു.
\end{malayalam}}
\flushright{\begin{Arabic}
\quranayah[33][20]
\end{Arabic}}
\flushleft{\begin{malayalam}
സംഘടിതകക്ഷികള്‍ പോയിക്കഴിഞ്ഞിട്ടില്ലെന്നാണ് അവര്‍ (കപടന്‍മാര്‍) വിചാരിക്കുന്നത്‌. സംഘടിതകക്ഷികള്‍ (ഇനിയും) വരികയാണെങ്കിലോ, (യുദ്ധത്തില്‍ പങ്കെടുക്കാതെ) നിങ്ങളുടെ വിവരങ്ങള്‍ അന്വേഷിച്ചറിഞ്ഞു കൊണ്ട് ഗ്രാമീണ അറബികളുടെ കൂടെ മരുഭൂവാസികളായി കഴിഞ്ഞിരുന്നെങ്കില്‍ എന്നായിരിക്കും അവര്‍ (കപടന്‍മാര്‍) കൊതിക്കുന്നത്‌. അവര്‍ നിങ്ങളുടെ കൂട്ടത്തിലായിരുന്നാലും ചുരുക്കത്തിലല്ലാതെ അവര്‍ യുദ്ധം ചെയ്യുകയില്ല.
\end{malayalam}}
\flushright{\begin{Arabic}
\quranayah[33][21]
\end{Arabic}}
\flushleft{\begin{malayalam}
തീര്‍ച്ചയായും നിങ്ങള്‍ക്ക് അല്ലാഹുവിന്‍റെ ദൂതനില്‍ ഉത്തമമായ മാതൃകയുണ്ട്‌. അതായത് അല്ലാഹുവെയും അന്ത്യദിനത്തെയും പ്രതീക്ഷിച്ചു കൊണ്ടിരിക്കുകയും, അല്ലാഹുവെ ധാരാളമായി ഓര്‍മിക്കുകയും ചെയ്തു വരുന്നവര്‍ക്ക്‌.
\end{malayalam}}
\flushright{\begin{Arabic}
\quranayah[33][22]
\end{Arabic}}
\flushleft{\begin{malayalam}
സത്യവിശ്വാസികള്‍ സംഘടിതകക്ഷികളെ കണ്ടപ്പോള്‍ ഇപ്രകാരം പറഞ്ഞു: ഇത് അല്ലാഹുവും അവന്‍റെ ദൂതനും ഞങ്ങളോട് വാഗ്ദാനം ചെയ്തിട്ടുള്ളതാകുന്നു. അല്ലാഹുവും അവന്‍റെ ദൂതനും സത്യമാണ് പറഞ്ഞിട്ടുള്ളത്‌. അതവര്‍ക്ക് വിശ്വാസവും അര്‍പ്പണവും വര്‍ദ്ധിപ്പിക്കുക മാത്രമേ ചെയ്തുള്ളൂ.
\end{malayalam}}
\flushright{\begin{Arabic}
\quranayah[33][23]
\end{Arabic}}
\flushleft{\begin{malayalam}
സത്യവിശ്വാസികളുടെ കൂട്ടത്തില്‍ ചില പുരുഷന്‍മാരുണ്ട്‌. ഏതൊരു കാര്യത്തില്‍ അല്ലാഹുവോട് അവര്‍ ഉടമ്പടി ചെയ്തുവോ, അതില്‍ അവര്‍ സത്യസന്ധത പുലര്‍ത്തി. അങ്ങനെ അവരില്‍ ചിലര്‍ (രക്ത സാക്ഷിത്വത്തിലൂടെ) തങ്ങളുടെ പ്രതിജ്ഞ നിറവേറ്റി. അവരില്‍ ചിലര്‍ (അത്‌) കാത്തിരിക്കുന്നു. അവര്‍ (ഉടമ്പടിക്ക്‌) യാതൊരു വിധ മാറ്റവും വരുത്തിയിട്ടില്ല.
\end{malayalam}}
\flushright{\begin{Arabic}
\quranayah[33][24]
\end{Arabic}}
\flushleft{\begin{malayalam}
സത്യവാന്‍മാര്‍ക്ക് തങ്ങളുടെ സത്യസന്ധതയ്ക്കുള്ള പ്രതിഫലം അല്ലാഹു നല്‍കുവാന്‍ വേണ്ടി. അവന്‍ ഉദ്ദേശിക്കുന്ന പക്ഷം കപടവിശ്വാസികളെ ശിക്ഷിക്കുകയോ, അല്ലെങ്കില്‍ അവരുടെ പശ്ചാത്താപം സ്വീകരിക്കുകയോ ചെയ്യാന്‍ വേണ്ടിയും. തീര്‍ച്ചയായും അല്ലാഹു ഏറെ പൊറുക്കുന്നവനും കരുണാനിധിയുമാകുന്നു.
\end{malayalam}}
\flushright{\begin{Arabic}
\quranayah[33][25]
\end{Arabic}}
\flushleft{\begin{malayalam}
സത്യനിഷേധികളെ അവരുടെ ഈര്‍ഷ്യയോടെത്തന്നെ അല്ലാഹു തിരിച്ചയക്കുകയും ചെയ്തു. യാതൊരു ഗുണവും അവര്‍ നേടിയില്ല. സത്യവിശ്വാസികള്‍ക്ക് അല്ലാഹു യുദ്ധത്തിന്‍റെ ആവശ്യം ഇല്ലാതാക്കി. അല്ലാഹു ശക്തനും പ്രതാപിയുമാകുന്നു.
\end{malayalam}}
\flushright{\begin{Arabic}
\quranayah[33][26]
\end{Arabic}}
\flushleft{\begin{malayalam}
വേദക്കാരില്‍ നിന്ന് അവര്‍ക്ക് (സത്യനിഷേധികള്‍ക്ക്‌) പിന്തുണ നല്‍കിയവരെ അവരുടെ കോട്ടകളില്‍ നിന്ന് അവന്‍ ഇറക്കിവിടുകയും അവരുടെ ഹൃദയങ്ങളില്‍ അവന്‍ ഭയം ഇട്ടുകൊടുക്കുകയും ചെയ്തു. അവരില്‍ ഒരു വിഭാഗത്തെ നിങ്ങളതാ കൊല്ലുന്നു. ഒരു വിഭാഗത്തെ നിങ്ങള്‍ തടവിലാക്കുകയും ചെയ്യുന്നു.
\end{malayalam}}
\flushright{\begin{Arabic}
\quranayah[33][27]
\end{Arabic}}
\flushleft{\begin{malayalam}
അവരുടെ ഭൂമിയും വീടുകളും സ്വത്തുക്കളും നിങ്ങള്‍ (മുമ്പ്‌) കാലെടുത്ത് വെച്ചിട്ടില്ലാത്ത ഒരു ഭൂപ്രദേശവും നിങ്ങള്‍ക്കവന്‍ അവകാശപ്പെടുത്തി തരികയും ചെയ്തു. അല്ലാഹു എല്ലാ കാര്യത്തിനും കഴിവുള്ളവനാകുന്നു.
\end{malayalam}}
\flushright{\begin{Arabic}
\quranayah[33][28]
\end{Arabic}}
\flushleft{\begin{malayalam}
നബിയേ, നിന്‍റെ ഭാര്യമാരോട് നീ പറയുക: ഐഹികജീവിതവും അതിന്‍റെ അലങ്കാരവുമാണ് നിങ്ങള്‍ ഉദ്ദേശിക്കുന്നതെങ്കില്‍ നിങ്ങള്‍ വരൂ! നിങ്ങള്‍ക്ക് ഞാന്‍ ജീവിതവിഭവം നല്‍കുകയും, ഭംഗിയായ നിലയില്‍ ഞാന്‍ നിങ്ങളെ മോചിപ്പിച്ച് അയച്ചുതരികയും ചെയ്യാം
\end{malayalam}}
\flushright{\begin{Arabic}
\quranayah[33][29]
\end{Arabic}}
\flushleft{\begin{malayalam}
അല്ലാഹുവെയും അവന്‍റെ ദൂതനെയും പരലോകഭവനത്തെയുമാണ് നിങ്ങള്‍ ഉദ്ദേശിക്കുന്നതെങ്കില്‍ നിങ്ങളുടെ കൂട്ടത്തില്‍ സദ്‌വൃത്തകളായിട്ടുള്ളവര്‍ക്ക് അല്ലാഹു മഹത്തായ പ്രതിഫലം ഒരുക്കിവെച്ചിട്ടുണ്ട്‌.
\end{malayalam}}
\flushright{\begin{Arabic}
\quranayah[33][30]
\end{Arabic}}
\flushleft{\begin{malayalam}
പ്രവാചക പത്നിമാരേ, നിങ്ങളില്‍ ആരെങ്കിലും വ്യക്തമായ നീചവൃത്തി ചെയ്യുന്ന പക്ഷം അവള്‍ക്ക് ശിക്ഷ രണ്ടിരട്ടിയായി വര്‍ദ്ധിപ്പിക്കപ്പെടുന്നതാണ്‌. അത് അല്ലാഹുവെ സംബന്ധിച്ചിടത്തോളം എളുപ്പമായിട്ടുള്ളതാകുന്നു.
\end{malayalam}}
\flushright{\begin{Arabic}
\quranayah[33][31]
\end{Arabic}}
\flushleft{\begin{malayalam}
നിങ്ങളില്‍ ആരെങ്കിലും അല്ലാഹുവോടും അവന്‍റെ ദൂതനോടും താഴ്മകാണിക്കുകയും സല്‍കര്‍മ്മം പ്രവര്‍ത്തിക്കുകയും ചെയ്യുന്ന പക്ഷം അവള്‍ക്ക് അവളുടെ പ്രതിഫലം രണ്ടുമടങ്ങായി നാം നല്‍കുന്നതാണ്‌. അവള്‍ക്ക് വേണ്ടി നാം മാന്യമായ ഉപജീവനം ഒരുക്കിവെക്കുകയും ചെയ്തിരിക്കുന്നു.
\end{malayalam}}
\flushright{\begin{Arabic}
\quranayah[33][32]
\end{Arabic}}
\flushleft{\begin{malayalam}
പ്രവാചക പത്നിമാരേ, സ്ത്രീകളില്‍ മറ്റു ആരെപ്പോലെയുമല്ല നിങ്ങള്‍. നിങ്ങള്‍ ധര്‍മ്മനിഷ്ഠ പാലിക്കുന്നുവെങ്കില്‍ നിങ്ങള്‍ (അന്യരോട്‌) അനുനയ സ്വരത്തില്‍ സംസാരിക്കരുത്‌. അപ്പോള്‍ ഹൃദയത്തില്‍ രോഗമുള്ളവന് മോഹം തോന്നിയേക്കും. ന്യായമായ വാക്ക് നിങ്ങള്‍ പറഞ്ഞു കൊള്ളുക.
\end{malayalam}}
\flushright{\begin{Arabic}
\quranayah[33][33]
\end{Arabic}}
\flushleft{\begin{malayalam}
നിങ്ങള്‍ നിങ്ങളുടെ വീടുകളില്‍ അടങ്ങിക്കഴിയുകയും ചെയ്യുക. പഴയ അജ്ഞാനകാലത്തെ സൌന്ദര്യപ്രകടനം പോലുള്ള സൌന്ദര്യപ്രകടനം നിങ്ങള്‍ നടത്തരുത്‌. നിങ്ങള്‍ നമസ്കാരം മുറപോലെ നിര്‍വഹിക്കുകയും, സകാത്ത് നല്‍കുകയും അല്ലാഹുവെയും അവന്‍റെ ദൂതനെയും അനുസരിക്കുകയും ചെയ്യുക. (പ്രവാചകന്‍റെ) വീട്ടുകാരേ! നിങ്ങളില്‍ നിന്ന് മാലിന്യം നീക്കികളയുവാനും, നിങ്ങളെ ശരിയായി ശുദ്ധീകരിക്കുവാനും മാത്രമാണ് അല്ലാഹു ഉദ്ദേശിക്കുന്നത്‌.
\end{malayalam}}
\flushright{\begin{Arabic}
\quranayah[33][34]
\end{Arabic}}
\flushleft{\begin{malayalam}
നിങ്ങളുടെ വീടുകളില്‍ വെച്ച് ഓതികേള്‍പിക്കപ്പെടുന്ന അല്ലാഹുവിന്‍റെ വചനങ്ങളും തത്വജ്ഞാനവും നിങ്ങള്‍ ഓര്‍മിക്കുകയും ചെയ്യുക. തീര്‍ച്ചയായും അല്ലാഹു നയജ്ഞനും സൂക്ഷ്മജ്ഞാനിയുമാകുന്നു.
\end{malayalam}}
\flushright{\begin{Arabic}
\quranayah[33][35]
\end{Arabic}}
\flushleft{\begin{malayalam}
(അല്ലാഹുവിന്‌) കീഴ്പെടുന്നവരായ പുരുഷന്‍മാര്‍, സ്ത്രീകള്‍, വിശ്വാസികളായ പുരുഷന്‍മാര്‍, സ്ത്രീകള്‍, ഭക്തിയുള്ളവരായ പുരുഷന്‍മാര്‍, സ്ത്രീകള്‍, സത്യസന്ധരായ പുരുഷന്‍മാര്‍, സ്ത്രീകള്‍, ക്ഷമാശീലരായ പുരുഷന്‍മാര്‍, സ്ത്രീകള്‍ വിനീതരായ പുരുഷന്‍മാര്‍, സ്ത്രീകള്‍, ദാനം ചെയ്യുന്നവരായ പുരുഷന്‍മാര്‍, സ്ത്രീകള്‍, വ്രതമനുഷ്ഠിക്കുന്നവരായ പുരുഷന്‍മാര്‍, സ്ത്രീകള്‍, തങ്ങളുടെ ഗുഹ്യാവയവങ്ങള്‍ കാത്തുസൂക്ഷിക്കുന്നവരായ പുരുഷന്‍മാര്‍, സ്ത്രീകള്‍, ധാരാളമായി അല്ലാഹുവെ ഓര്‍മിക്കുന്നവരായ പുരുഷന്‍മാര്‍, സ്ത്രീകള്‍ - ഇവര്‍ക്ക് തീര്‍ച്ചയായും അല്ലാഹു പാപമോചനവും മഹത്തായ പ്രതിഫലവും ഒരുക്കിവെച്ചിരിക്കുന്നു.
\end{malayalam}}
\flushright{\begin{Arabic}
\quranayah[33][36]
\end{Arabic}}
\flushleft{\begin{malayalam}
അല്ലാഹുവും അവന്‍റെ റസൂലും ഒരു കാര്യത്തില്‍ തീരുമാനമെടുത്ത് കഴിഞ്ഞാല്‍ സത്യവിശ്വാസിയായ ഒരു പുരുഷന്നാകട്ടെ, സ്ത്രീക്കാകട്ടെ തങ്ങളുടെ കാര്യത്തെ സംബന്ധിച്ച് സ്വതന്ത്രമായ അഭിപ്രായം ഉണ്ടായിരിക്കാവുന്നതല്ല. വല്ലവനും അല്ലാഹുവെയും അവന്‍റെ ദൂതനെയും ധിക്കരിക്കുന്ന പക്ഷം അവന്‍ വ്യക്തമായ നിലയില്‍ വഴിപിഴച്ചു പോയിരിക്കുന്നു.
\end{malayalam}}
\flushright{\begin{Arabic}
\quranayah[33][37]
\end{Arabic}}
\flushleft{\begin{malayalam}
നിന്‍റെ ഭാര്യയെ നീ നിന്‍റെ അടുത്ത് തന്നെ നിര്‍ത്തിപ്പോരുകയും, അല്ലാഹുവെ നീ സൂക്ഷിക്കുകയും ചെയ്യുക എന്ന്‌, അല്ലാഹു അനുഗ്രഹം ചെയ്തുകൊടുത്തിട്ടുള്ളവനും നീ അനുഗ്രഹം ചെയ്തുകൊടുത്തിട്ടുള്ളവനുമായ ഒരാളോട് നീ പറഞ്ഞിരുന്ന സന്ദര്‍ഭം (ഓര്‍ക്കുക.) അല്ലാഹു വെളിപ്പെടുത്താന്‍ പോകുന്ന ഒരു കാര്യം നിന്‍റെ മനസ്സില്‍ നീ മറച്ചു വെക്കുകയും ജനങ്ങളെ നീ പേടിക്കുകയും ചെയ്തിരുന്നു. എന്നാല്‍ നീ പേടിക്കുവാന്‍ ഏറ്റവും അര്‍ഹതയുള്ളവന്‍ അല്ലാഹുവാകുന്നു. അങ്ങനെ സൈദ് അവളില്‍ നിന്ന് ആവശ്യം നിറവേറ്റികഴിഞ്ഞപ്പോള്‍ അവളെ നാം നിനക്ക് ഭാര്യയാക്കിത്തന്നു. തങ്ങളുടെ ദത്തുപുത്രന്‍മാര്‍ അവരുടെ ഭാര്യമാരില്‍ നിന്ന് ആവശ്യം നിറവേറ്റിക്കഴിഞ്ഞിട്ട് അവരെ വിവാഹം കഴിക്കുന്ന കാര്യത്തില്‍ സത്യവിശ്വാസികള്‍ക്ക് യാതൊരു വിഷമവും ഉണ്ടാകാതിരിക്കാന്‍ വേണ്ടിയത്രെ അത്‌. അല്ലാഹുവിന്‍റെ കല്‍പന പ്രാവര്‍ത്തികമാക്കപ്പെടുന്നതാകുന്നു.
\end{malayalam}}
\flushright{\begin{Arabic}
\quranayah[33][38]
\end{Arabic}}
\flushleft{\begin{malayalam}
തനിക്ക് അല്ലാഹു നിശ്ചയിച്ചു തന്ന കാര്യത്തില്‍ പ്രവാചകന് യാതൊരു വിഷമവും തോന്നേണ്ടതില്ല. മുമ്പ് കഴിഞ്ഞുപോയിട്ടുള്ളവരില്‍ അല്ലാഹു നടപ്പാക്കിയിരുന്ന നടപടിക്രമം തന്നെ. അല്ലാഹുവിന്‍റെ കല്‍പന ഖണ്ഡിതമായ ഒരു വിധിയാകുന്നു.
\end{malayalam}}
\flushright{\begin{Arabic}
\quranayah[33][39]
\end{Arabic}}
\flushleft{\begin{malayalam}
അതായത് അല്ലാഹുവിന്‍റെ സന്ദേശങ്ങള്‍ എത്തിച്ചുകൊടുക്കുകയും, അവനെ പേടിക്കുകയും അല്ലാഹുവല്ലാത്ത ഒരാളെയും പേടിക്കാതിരിക്കുകയും ചെയ്തിരുന്നവരുടെ കാര്യത്തിലുള്ള (അല്ലാഹുവിന്‍റെ നടപടി.) കണക്ക് നോക്കുന്നവനായി അല്ലാഹു തന്നെ മതി.
\end{malayalam}}
\flushright{\begin{Arabic}
\quranayah[33][40]
\end{Arabic}}
\flushleft{\begin{malayalam}
മുഹമ്മദ് നിങ്ങളുടെ പുരുഷന്‍മാരില്‍ ഒരാളുടെയും പിതാവായിട്ടില്ല. പക്ഷെ, അദ്ദേഹം അല്ലാഹുവിന്‍റെ ദൂതനും പ്രവാചകന്‍മാരില്‍ അവസാനത്തെ ആളുമാകുന്നു. അല്ലാഹു ഏത് കാര്യത്തെപ്പറ്റിയും അറിവുള്ളവനാകുന്നു.
\end{malayalam}}
\flushright{\begin{Arabic}
\quranayah[33][41]
\end{Arabic}}
\flushleft{\begin{malayalam}
സത്യവിശ്വാസികളേ, നിങ്ങള്‍ അല്ലാഹുവെ ധാരാളമായി അനുസ്മരിക്കുകയും,
\end{malayalam}}
\flushright{\begin{Arabic}
\quranayah[33][42]
\end{Arabic}}
\flushleft{\begin{malayalam}
കാലത്തും വൈകുന്നേരവും അവനെ പ്രകീര്‍ത്തിക്കുകയും ചെയ്യുവിന്‍.
\end{malayalam}}
\flushright{\begin{Arabic}
\quranayah[33][43]
\end{Arabic}}
\flushleft{\begin{malayalam}
അവന്‍ നിങ്ങളുടെ മേല്‍ കരുണ ചൊരിയുന്നവനാകുന്നു. അവന്‍റെ മലക്കുകളും (കരുണ കാണിക്കുന്നു.) അന്ധകാരങ്ങളില്‍ നിന്ന് നിങ്ങളെ വെളിച്ചത്തിലേക്ക് ആനയിക്കുന്നതിന് വേണ്ടിയത്രെ അത്‌. അവന്‍ സത്യവിശ്വാസികളോട് അത്യന്തം കരുണയുള്ളവനാകുന്നു.
\end{malayalam}}
\flushright{\begin{Arabic}
\quranayah[33][44]
\end{Arabic}}
\flushleft{\begin{malayalam}
അവര്‍ അവനെ കണ്ടുമുട്ടുന്ന ദിവസം അവര്‍ക്കുള്ള അഭിവാദ്യം സലാം ആയിരിക്കും.അവര്‍ക്കവന്‍ മാന്യമായ പ്രതിഫലം ഒരുക്കിവെച്ചിട്ടുണ്ട്‌.
\end{malayalam}}
\flushright{\begin{Arabic}
\quranayah[33][45]
\end{Arabic}}
\flushleft{\begin{malayalam}
നബിയേ, തീര്‍ച്ചയായും നിന്നെ നാം ഒരു സാക്ഷിയും സന്തോഷവാര്‍ത്ത അറിയിക്കുന്നവനും, താക്കീതുകാരനും ആയിക്കൊണ്ട് നിയോഗിച്ചിരിക്കുന്നു.
\end{malayalam}}
\flushright{\begin{Arabic}
\quranayah[33][46]
\end{Arabic}}
\flushleft{\begin{malayalam}
അല്ലാഹുവിന്‍റെ ഉത്തരവനുസരിച്ച് അവങ്കലേക്ക് ക്ഷണിക്കുന്നവനും, പ്രകാശം നല്‍കുന്ന ഒരു വിളക്കും ആയിക്കൊണ്ട്‌.
\end{malayalam}}
\flushright{\begin{Arabic}
\quranayah[33][47]
\end{Arabic}}
\flushleft{\begin{malayalam}
സത്യവിശ്വാസികള്‍ക്ക് അല്ലാഹുവിങ്കല്‍ നിന്ന് വലിയ ഔദാര്യം ലഭിക്കാനുണ്ട് എന്ന് നീ അവരെ സന്തോഷവാര്‍ത്ത അറിയിക്കുക.
\end{malayalam}}
\flushright{\begin{Arabic}
\quranayah[33][48]
\end{Arabic}}
\flushleft{\begin{malayalam}
സത്യനിഷേധികളെയും കപടവിശ്വാസികളെയും നീ അനുസരിച്ചു പോകരുത്‌. അവരുടെ ദ്രോഹം നീ അവഗണിക്കുകയും, അല്ലാഹുവെ ഭരമേല്‍പിക്കുകയും ചെയ്യുക. കൈകാര്യകര്‍ത്താവായി അല്ലാഹു തന്നെ മതി.
\end{malayalam}}
\flushright{\begin{Arabic}
\quranayah[33][49]
\end{Arabic}}
\flushleft{\begin{malayalam}
സത്യവിശ്വാസികളേ, നിങ്ങള്‍ സത്യവിശ്വാസിനികളെ വിവാഹം കഴിക്കുകയും, എന്നിട്ട് നിങ്ങളവരെ സ്പര്‍ശിക്കുന്നതിന് മുമ്പായി അവരെ വിവാഹമോചനം നടത്തുകയും ചെയ്താല്‍ നിങ്ങള്‍ എണ്ണികണക്കാക്കുന്ന ഇദ്ദഃ ആചരിക്കേണ്ട ബാധ്യത അവര്‍ക്കു നിങ്ങളോടില്ല. എന്നാല്‍ നിങ്ങള്‍ അവര്‍ക്ക് മതാഅ് നല്‍കുകയും, അവരെ ഭംഗിയായി പിരിച്ചയക്കുകയും ചെയ്യുക.
\end{malayalam}}
\flushright{\begin{Arabic}
\quranayah[33][50]
\end{Arabic}}
\flushleft{\begin{malayalam}
നബിയേ, നീ വിവാഹമൂല്യം കൊടുത്തിട്ടുള്ളവരായ നിന്‍റെ ഭാര്യമാരെ നിനക്ക് നാം അനുവദിച്ചു തന്നിരിക്കുന്നു. അല്ലാഹു നിനക്ക് (യുദ്ധത്തില്‍) അധീനപ്പെടുത്തിത്തന്ന കൂട്ടത്തില്‍ നിന്‍റെ വലതുകൈ ഉടമപ്പെടുത്തിയ (അടിമ) സ്ത്രീകളെയും നിന്നോടൊപ്പം സ്വദേശം വിട്ടുപോന്നവരായ നിന്‍റെ പിതൃവ്യന്‍റെ പുത്രിമാര്‍, നിന്‍റെ പിതൃസഹോദരിമാരുടെ പുത്രിമാര്‍, നിന്‍റെ അമ്മാവന്‍റെ പുത്രിമാര്‍, നിന്‍റെ മാതൃസഹോദരിമാരുടെ പുത്രിമാര്‍ എന്നിവരെയും (വിവാഹം ചെയ്യാന്‍ അനുവദിച്ചിരിക്കുന്നു.) സത്യവിശ്വാസിനിയായ ഒരു സ്ത്രീ സ്വദേഹം നബിക്ക് ദാനം ചെയ്യുന്ന പക്ഷം നബി അവളെ വിവാഹം കഴിക്കാന്‍ ഉദ്ദേശിക്കുന്നെങ്കില്‍ അതും (അനുവദിച്ചിരിക്കുന്നു.) ഇത് സത്യവിശ്വാസികളെ കൂടാതെ നിനക്ക് മാത്രമുള്ളതാകുന്നു. അവരുടെ ഭാര്യമാരുടെയും അവരുടെ വലതുകൈകള്‍ ഉടമപ്പെടുത്തിയവരുടേയും കാര്യത്തില്‍ നാം നിയമമായി നിശ്ചയിച്ചിട്ടുള്ളത് നമുക്കറിയാം. നിനക്ക് യാതൊരു വിഷമവും ഉണ്ടാവാതിരിക്കാന്‍ വേണ്ടിയത്രെ ഇത്‌. അല്ലാഹു ഏറെ പൊറുക്കുന്നവനും കരുണാനിധിയുമാകുന്നു.
\end{malayalam}}
\flushright{\begin{Arabic}
\quranayah[33][51]
\end{Arabic}}
\flushleft{\begin{malayalam}
അവരില്‍ നിന്ന് നീ ഉദ്ദേശിക്കുന്നവരെ നിനക്ക് മാറ്റി നിര്‍ത്താം. നീ ഉദ്ദേശിക്കുന്നവരെ നിന്‍റെ അടുക്കലേക്ക് അടുപ്പിക്കുകയും ചെയ്യാം. നീ മാറ്റി നിര്‍ത്തിയവരില്‍ നിന്ന് വല്ലവരെയും നീ ആഗ്രഹിക്കുന്നുവെങ്കില്‍ നിനക്ക് കുറ്റമില്ല. അവരുടെ കണ്ണുകള്‍ കുളിര്‍ക്കുവാനും, അവര്‍ ദുഃഖിക്കാതിരിക്കുവാനും, നീ അവര്‍ക്ക് നല്‍കിയതില്‍ അവരെല്ലാം സംതൃപ്തി അടയുവാനും ഏറ്റവും അനുയോജ്യമായ മാര്‍ഗമാകുന്നു അത്‌. നിങ്ങളുടെ ഹൃദയങ്ങളിലുള്ളത് അല്ലാഹു അറിയുന്നു. അല്ലാഹു സര്‍വ്വജ്ഞനും സഹനശീലനുമാകുന്നു.
\end{malayalam}}
\flushright{\begin{Arabic}
\quranayah[33][52]
\end{Arabic}}
\flushleft{\begin{malayalam}
ഇനിമേല്‍ നിനക്ക് (വേറെ) സ്ത്രീകളെ വിവാഹം കഴിക്കാന്‍ അനുവാദമില്ല. ഇവര്‍ക്ക് പകരം വേറെ ഭാര്യമാരെ സ്വീകരിക്കുവാനും (അനുവാദമില്ല) അവരുടെ സൌന്ദര്യം നിനക്ക് കൌതുകം തോന്നിച്ചാലും ശരി. നിന്‍റെ വലതുകൈ ഉടമപ്പെടുത്തിയവര്‍ (അടിമസ്ത്രീകള്‍) ഒഴികെ. അല്ലാഹു എല്ലാകാര്യവും നിരീക്ഷിച്ചു കൊണ്ടിരിക്കുന്നവനാകുന്നു.
\end{malayalam}}
\flushright{\begin{Arabic}
\quranayah[33][53]
\end{Arabic}}
\flushleft{\begin{malayalam}
സത്യവിശ്വാസികളേ, ഭക്ഷണത്തിന് (നിങ്ങളെ ക്ഷണിക്കുകയും) നിങ്ങള്‍ക്ക് സമ്മതം കിട്ടുകയും ചെയ്താലല്ലാതെ നബിയുടെ വീടുകളില്‍ നിങ്ങള്‍ കടന്നു ചെല്ലരുത്‌. അത് (ഭക്ഷണം) പാകമാകുന്നത് നിങ്ങള്‍ നോക്കിയിരിക്കുന്നവരാകരുത്‌. പക്ഷെ നിങ്ങള്‍ ക്ഷണിക്കപ്പെട്ടാല്‍ നിങ്ങള്‍ കടന്ന് ചെല്ലുക. നിങ്ങള്‍ ഭക്ഷണം കഴിച്ചാല്‍ പിരിഞ്ഞു പോകുകയും ചെയ്യുക. നിങ്ങള്‍ വര്‍ത്തമാനം പറഞ്ഞ് രസിച്ചിരിക്കുന്നവരാവുകയും അരുത്‌. തീര്‍ച്ചയായും അതൊക്കെ നബിയെ ശല്യപ്പെടുത്തുന്നതാകുന്നു. എന്നാല്‍ നിങ്ങളോട് (അത് പറയാന്‍) അദ്ദേഹത്തിന് ലജ്ജ തോന്നുന്നു. സത്യത്തിന്‍റെ കാര്യത്തില്‍ അല്ലാഹുവിന് ലജ്ജ തോന്നുകയില്ല. നിങ്ങള്‍ അവരോട് (നബിയുടെ ഭാര്യമാരോട്‌) വല്ല സാധനവും ചോദിക്കുകയാണെങ്കില്‍ നിങ്ങളവരോട് മറയുടെ പിന്നില്‍ നിന്ന് ചോദിച്ചുകൊള്ളുക. അതാണ് നിങ്ങളുടെ ഹൃദയങ്ങള്‍ക്കും അവരുടെ ഹൃദയങ്ങള്‍ക്കും കൂടുതല്‍ സംശുദ്ധമായിട്ടുള്ളത്‌. അല്ലാഹുവിന്‍റെ ദൂതന് ശല്യമുണ്ടാക്കാന്‍ നിങ്ങള്‍ക്ക് പാടില്ല. അദ്ദേഹത്തിന് ശേഷം ഒരിക്കലും അദ്ദേഹത്തിന്‍റെ ഭാര്യമാരെ നിങ്ങള്‍ വിവാഹം കഴിക്കാനും പാടില്ല. തീര്‍ച്ചയായും അതൊക്കെ അല്ലാഹുവിങ്കല്‍ ഗൌരവമുള്ള കാര്യമാകുന്നു.
\end{malayalam}}
\flushright{\begin{Arabic}
\quranayah[33][54]
\end{Arabic}}
\flushleft{\begin{malayalam}
നിങ്ങള്‍ എന്തെങ്കിലും വെളിപ്പെടുത്തുകയാണെങ്കിലും അത് മറച്ചു വെക്കുകയാണെങ്കിലും തീര്‍ച്ചയായും അല്ലാഹു ഏത് കാര്യത്തെപ്പറ്റിയും അറിവുള്ളവനാകുന്നു.
\end{malayalam}}
\flushright{\begin{Arabic}
\quranayah[33][55]
\end{Arabic}}
\flushleft{\begin{malayalam}
ആ സ്ത്രീകള്‍ക്ക് തങ്ങളുടെ പിതാക്കളുമായോ, പുത്രന്‍മാരുമായോ, സഹോദരന്‍മാരുമായോ, സഹോദരപുത്രന്‍മാരുമായോ, സഹോദരീ പുത്രന്‍മാരുമായോ, തങ്ങളുടെ കൂട്ടത്തില്‍പെട്ട സ്ത്രീകളുമായോ, തങ്ങളുടെ വലതുകൈകള്‍ ഉടമപ്പെടുത്തിയവരുമായോ ഇടപഴകുന്നതിന് വിരോധമില്ല. നിങ്ങള്‍ അല്ലാഹുവെ സൂക്ഷിക്കുക. തീര്‍ച്ചയായും അല്ലാഹു എല്ലാകാര്യത്തിനും സാക്ഷിയാകുന്നു.
\end{malayalam}}
\flushright{\begin{Arabic}
\quranayah[33][56]
\end{Arabic}}
\flushleft{\begin{malayalam}
തീര്‍ച്ചയായും അല്ലാഹുവും അവന്‍റെ മലക്കുകളും നബിയോട് കാരുണ്യം കാണിക്കുന്നു. സത്യവിശ്വാസികളേ, നിങ്ങള്‍ അദ്ദേഹത്തിന്‍റെ മേല്‍ (അല്ലാഹുവിന്‍റെ) കാരുണ്യവും ശാന്തിയുമുണ്ടാകാന്‍ പ്രാര്‍ത്ഥിക്കുക.
\end{malayalam}}
\flushright{\begin{Arabic}
\quranayah[33][57]
\end{Arabic}}
\flushleft{\begin{malayalam}
അല്ലാഹുവെയും അവന്‍റെ റസൂലിനെയും ദ്രോഹിക്കുന്നവരാരോ അവരെ ഇഹത്തിലും പരത്തിലും അല്ലാഹു ശപിച്ചിരിക്കുന്നു. അവര്‍ക്കുവേണ്ടി അപമാനകരമായ ശിക്ഷ അവന്‍ ഒരുക്കിവെച്ചിട്ടുമുണ്ട്‌.
\end{malayalam}}
\flushright{\begin{Arabic}
\quranayah[33][58]
\end{Arabic}}
\flushleft{\begin{malayalam}
സത്യവിശ്വാസികളായ പുരുഷന്‍മാരെയും സ്ത്രീകളെയും അവര്‍ (തെറ്റായ) യാതൊന്നും ചെയ്യാതിരിക്കെ ശല്യപ്പെടുത്തുന്നവരാരോ അവര്‍ അപവാദവും പ്രത്യക്ഷമായ പാപവും പേറിയിരിക്കയാണ്‌.
\end{malayalam}}
\flushright{\begin{Arabic}
\quranayah[33][59]
\end{Arabic}}
\flushleft{\begin{malayalam}
നബിയേ, നിന്‍റെ പത്നിമാരോടും പുത്രിമാരോടും സത്യവിശ്വാസികളുടെ സ്ത്രീകളോടും അവര്‍ തങ്ങളുടെ മൂടുപടങ്ങള്‍ തങ്ങളുടെമേല്‍ താഴ്ത്തിയിടാന്‍ പറയുക: അവര്‍ തിരിച്ചറിയപ്പെടുവാനും, അങ്ങനെ അവര്‍ ശല്യം ചെയ്യപ്പെടാതിരിക്കുവാനും അതാണ് ഏറ്റവും അനുയോജ്യമായത്‌. അല്ലാഹു ഏറെ പൊറുക്കുന്നവനും കരുണാനിധിയുമാകുന്നു.
\end{malayalam}}
\flushright{\begin{Arabic}
\quranayah[33][60]
\end{Arabic}}
\flushleft{\begin{malayalam}
കപടവിശ്വാസികളും, തങ്ങളുടെ ഹൃദയങ്ങളില്‍ രോഗമുള്ളവരും, നുണ പ്രചരിപ്പിച്ച് മദീനയില്‍ കുഴപ്പം ഇളക്കിവിടുന്നവരും (അതില്‍ നിന്ന്‌) വിരമിക്കാത്ത പക്ഷം അവര്‍ക്കു നേരെ നിന്നെ നാം തിരിച്ചുവിടുക തന്നെ ചെയ്യും. പിന്നെ അവര്‍ക്ക് നിന്‍റെ അയല്‍വാസികളായി അല്‍പം മാത്രമേ അവിടെ കഴിക്കാനൊക്കൂ.
\end{malayalam}}
\flushright{\begin{Arabic}
\quranayah[33][61]
\end{Arabic}}
\flushleft{\begin{malayalam}
അവര്‍ ശാപം ബാധിച്ച നിലയിലായിരിക്കും. എവിടെ വെച്ച് കണ്ടുമുട്ടിയാലും അവര്‍ പിടിക്കപ്പെടുകയും, കൊന്നൊടുക്കപ്പെടുകയും ചെയ്യും.
\end{malayalam}}
\flushright{\begin{Arabic}
\quranayah[33][62]
\end{Arabic}}
\flushleft{\begin{malayalam}
മുമ്പ് കഴിഞ്ഞുപോയവരുടെ കാര്യത്തില്‍ അല്ലാഹു സ്വീകരിച്ച അതേ നടപടിക്രമം തന്നെ. അല്ലാഹുവിന്‍റെ നടപടിക്രമത്തിന് യാതൊരു മാറ്റവും നീ കണ്ടെത്തുകയില്ല.
\end{malayalam}}
\flushright{\begin{Arabic}
\quranayah[33][63]
\end{Arabic}}
\flushleft{\begin{malayalam}
ജനങ്ങള്‍ അന്ത്യസമയത്തെപ്പറ്റി നിന്നോട് ചോദിക്കുന്നു. പറയുക: അതിനെപ്പറ്റിയുള്ള അറിവ് അല്ലാഹുവിങ്കല്‍ മാത്രമാകുന്നു. നിനക്ക് (അതിനെപ്പറ്റി) അറിവുനല്‍കുന്ന എന്തൊന്നാണുള്ളത്‌? അന്ത്യസമയം ഒരു വേള സമീപസ്ഥമായിരിക്കാം.
\end{malayalam}}
\flushright{\begin{Arabic}
\quranayah[33][64]
\end{Arabic}}
\flushleft{\begin{malayalam}
തീര്‍ച്ചയായും അല്ലാഹു സത്യനിഷേധികളെ ശപിക്കുകയും അവര്‍ക്കുവേണ്ടി ജ്വലിക്കുന്ന നരകാഗ്നി ഒരുക്കിവെക്കുകയും ചെയ്തിരിക്കുന്നു.
\end{malayalam}}
\flushright{\begin{Arabic}
\quranayah[33][65]
\end{Arabic}}
\flushleft{\begin{malayalam}
എന്നെന്നും അവരതില്‍ ശാശ്വതവാസികളായിരിക്കും. യാതൊരു രക്ഷാധികാരിയെയും സഹായിയെയും അവര്‍ കണ്ടെത്തുകയില്ല.
\end{malayalam}}
\flushright{\begin{Arabic}
\quranayah[33][66]
\end{Arabic}}
\flushleft{\begin{malayalam}
അവരുടെ മുഖങ്ങള്‍ നരകത്തില്‍ കീഴ്മേല്‍ മറിക്കപ്പെടുന്ന ദിവസം. അവര്‍ പറയും: ഞങ്ങള്‍ അല്ലാഹുവെയും റസൂലിനെയും അനുസരിച്ചിരുന്നെങ്കില്‍ എത്ര നന്നായിരുന്നേനെ!
\end{malayalam}}
\flushright{\begin{Arabic}
\quranayah[33][67]
\end{Arabic}}
\flushleft{\begin{malayalam}
അവര്‍ പറയും: ഞങ്ങളുടെ രക്ഷിതാവേ, ഞങ്ങള്‍ ഞങ്ങളുടെ നേതാക്കന്‍മാരെയും പ്രമുഖന്‍മാരെയും അനുസരിക്കുകയും, അങ്ങനെ അവര്‍ ഞങ്ങളെ വഴിതെറ്റിക്കുകയുമാണുണ്ടായത്‌.
\end{malayalam}}
\flushright{\begin{Arabic}
\quranayah[33][68]
\end{Arabic}}
\flushleft{\begin{malayalam}
ഞങ്ങളുടെ രക്ഷിതാവേ, അവര്‍ക്ക് നീ രണ്ടിരട്ടി ശിക്ഷ നല്‍കുകയും അവര്‍ക്ക് നീ വന്‍ ശാപം ഏല്‍പിക്കുകയും ചെയ്യണമേ (എന്നും അവര്‍ പറയും.)
\end{malayalam}}
\flushright{\begin{Arabic}
\quranayah[33][69]
\end{Arabic}}
\flushleft{\begin{malayalam}
സത്യവിശ്വാസികളേ, നിങ്ങള്‍ മൂസാ നബിയെ ശല്യപ്പെടുത്തിയവരെപ്പോലെയാകരുത്‌. എന്നിട്ട് അല്ലാഹു അവര്‍ പറഞ്ഞതില്‍ നിന്ന് അദ്ദേഹത്തെ മുക്തനാക്കുകയും ചെയ്തു അദ്ദേഹം അല്ലാഹുവിന്‍റെ അടുക്കല്‍ ഉല്‍കൃഷ്ടനായിരിക്കുന്നു.
\end{malayalam}}
\flushright{\begin{Arabic}
\quranayah[33][70]
\end{Arabic}}
\flushleft{\begin{malayalam}
സത്യവിശ്വാസികളേ, നിങ്ങള്‍ അല്ലാഹുവെ സൂക്ഷിക്കുകയും, ശരിയായ വാക്ക് പറയുകയും ചെയ്യുക.
\end{malayalam}}
\flushright{\begin{Arabic}
\quranayah[33][71]
\end{Arabic}}
\flushleft{\begin{malayalam}
എങ്കില്‍ അവന്‍ നിങ്ങള്‍ക്ക് നിങ്ങളുടെ കര്‍മ്മങ്ങള്‍ നന്നാക്കിത്തരികയും, നിങ്ങളുടെ പാപങ്ങള്‍ അവന്‍ പൊറുത്തുതരികയും ചെയ്യും. അല്ലാഹുവെയും അവന്‍റെ ദൂതനെയും ആര്‍ അനുസരിക്കുന്നുവോ അവന്‍ മഹത്തായ വിജയം നേടിയിരിക്കുന്നു.
\end{malayalam}}
\flushright{\begin{Arabic}
\quranayah[33][72]
\end{Arabic}}
\flushleft{\begin{malayalam}
തീര്‍ച്ചയായും നാം ആ വിശ്വസ്ത ദൌത്യം (ഉത്തരവാദിത്തം) ആകാശങ്ങളുടെയും ഭൂമിയുടെയും പര്‍വ്വതങ്ങളുടെയും മുമ്പാകെ എടുത്തുകാട്ടുകയുണ്ടായി. എന്നാല്‍ അത് ഏറ്റെടുക്കുന്നതിന് അവ വിസമ്മതിക്കുകയും അതിനെപ്പറ്റി അവയ്ക്ക് പേടി തോന്നുകയും ചെയ്തു. മനുഷ്യന്‍ അത് ഏറ്റെടുത്തു. തീര്‍ച്ചയായും അവന്‍ കടുത്ത അക്രമിയും അവിവേകിയുമായിരിക്കുന്നു.
\end{malayalam}}
\flushright{\begin{Arabic}
\quranayah[33][73]
\end{Arabic}}
\flushleft{\begin{malayalam}
കപടവിശ്വാസികളായ പുരുഷന്‍മാരെയും സ്ത്രീകളേയും, ബഹുദൈവവിശ്വാസികളായ പുരുഷന്‍മാരെയും സ്ത്രീകളെയും അല്ലാഹു ശിക്ഷിക്കുവാനും, സത്യവിശ്വാസികളായ പുരുഷന്‍മാരുടെയും, സ്ത്രീകളുടെയും പശ്ചാത്താപം അല്ലാഹു സ്വീകരിക്കുവാനും. അല്ലാഹു ഏറെ പൊറുക്കുന്നവനും കരുണാനിധിയുമാകുന്നു.
\end{malayalam}}
\chapter{\textmalayalam{സബഅ്}}
\begin{Arabic}
\Huge{\centerline{\basmalah}}\end{Arabic}
\flushright{\begin{Arabic}
\quranayah[34][1]
\end{Arabic}}
\flushleft{\begin{malayalam}
ആകാശങ്ങളിലുള്ളതും ഭൂമിയിലുള്ളതും ആരുടേതാണോ ആ അല്ലാഹുവിന് സ്തുതി. അവന്‍ യുക്തിമാനും സൂക്ഷ്മജ്ഞനുമത്രെ.
\end{malayalam}}
\flushright{\begin{Arabic}
\quranayah[34][2]
\end{Arabic}}
\flushleft{\begin{malayalam}
ഭൂമിയില്‍ പ്രവേശിക്കുന്നതും, അതില്‍ നിന്ന് പുറത്ത് വരുന്നതും, ആകാശത്ത് നിന്ന് ഇറങ്ങുന്നതും അതില്‍ കയറുന്നതുമായ വസ്തുക്കളെ പറ്റി അവന്‍ അറിയുന്നു. അവന്‍ കരുണാനിധിയും ഏറെ പൊറുക്കുന്നവനുമത്രെ.
\end{malayalam}}
\flushright{\begin{Arabic}
\quranayah[34][3]
\end{Arabic}}
\flushleft{\begin{malayalam}
ആ അന്ത്യസമയം ഞങ്ങള്‍ക്ക് വന്നെത്തുകയില്ലെന്ന് സത്യനിഷേധികള്‍ പറഞ്ഞുണീ പറയുക: അല്ല, എന്‍റെ രക്ഷിതാവിനെ തന്നെയാണ, അത് നിങ്ങള്‍ക്ക് വന്നെത്തുക തന്നെ ചെയ്യും. അദൃശ്യകാര്യങ്ങള്‍ അറിയുന്നവനായ (രക്ഷിതാവ്‌). ആകാശങ്ങളിലാകട്ടെ ഭൂമിയിലാകട്ടെ ഒരു അണുവിന്‍റെ തൂക്കമുള്ളതോ അതിനെക്കാള്‍ ചെറുതോ വലുതോ ആയ യാതൊന്നും അവനില്‍ നിന്ന് മറഞ്ഞ് പോകുകയില്ല. സ്പഷ്ടമായ ഒരു രേഖയില്‍ ഉള്‍പെടാത്തതായി യാതൊന്നുമില്ല.
\end{malayalam}}
\flushright{\begin{Arabic}
\quranayah[34][4]
\end{Arabic}}
\flushleft{\begin{malayalam}
വിശ്വസിക്കുകയും സല്‍കര്‍മ്മങ്ങള്‍ പ്രവര്‍ത്തിക്കുകയും ചെയ്തവര്‍ക്ക് അവന്‍ പ്രതിഫലം നല്‍കുന്നതിന് വേണ്ടിയത്രെ അത്‌. അങ്ങനെയുള്ളവര്‍ക്കാകുന്നു പാപമോചനവും മാന്യമായ ഉപജീവനവും ഉള്ളത്‌.
\end{malayalam}}
\flushright{\begin{Arabic}
\quranayah[34][5]
\end{Arabic}}
\flushleft{\begin{malayalam}
(നമ്മെ) തോല്‍പിച്ച് കളയുവാനായി നമ്മുടെ ദൃഷ്ടാന്തങ്ങളെ എതിര്‍ക്കുന്നതിന് ശ്രമിച്ചവരാരോ അവര്‍ക്കത്രെ വേദനാജനകമായ കഠിനശിക്ഷയുള്ളത്‌.
\end{malayalam}}
\flushright{\begin{Arabic}
\quranayah[34][6]
\end{Arabic}}
\flushleft{\begin{malayalam}
നിനക്ക് നിന്‍റെ രക്ഷിതാവിങ്കല്‍ നിന്ന് അവതരിപ്പിക്കപ്പെട്ടതു തന്നെയാണ് സത്യമെന്നും, പ്രതാപിയും സ്തുത്യര്‍ഹനുമായ അല്ലാഹുവിന്‍റെ മാര്‍ഗത്തിലേക്കാണ് അത് നയിക്കുന്നതെന്നും ജ്ഞാനം നല്‍കപ്പെട്ടവര്‍ കാണുന്നുണ്ട്‌.
\end{malayalam}}
\flushright{\begin{Arabic}
\quranayah[34][7]
\end{Arabic}}
\flushleft{\begin{malayalam}
സത്യനിഷേധികള്‍ (പരിഹാസസ്വരത്തില്‍) പറഞ്ഞു: നിങ്ങള്‍ സര്‍വ്വത്ര ഛിന്നഭിന്നമാക്കപ്പെട്ടു കഴിഞ്ഞാലും നിങ്ങള്‍ പുതുതായി സൃഷ്ടിക്കപ്പെടുക തന്നെ ചെയ്യുമെന്ന് നിങ്ങള്‍ക്ക് വിവരം തരുന്ന ഒരാളെപ്പറ്റി ഞങ്ങള്‍ നിങ്ങള്‍ക്കു അറിയിച്ചു തരട്ടെയോ?
\end{malayalam}}
\flushright{\begin{Arabic}
\quranayah[34][8]
\end{Arabic}}
\flushleft{\begin{malayalam}
അല്ലാഹുവിന്‍റെ പേരില്‍ അയാള്‍ കള്ളം കെട്ടിച്ചമച്ചതാണോ അതല്ല അയാള്‍ക്കു ഭ്രാന്തുണ്ടോ? അല്ല, പരലോകത്തില്‍ വിശ്വസിക്കാത്തവര്‍ ശിക്ഷയിലും വിദൂരമായ വഴികേടിലുമാകുന്നു.
\end{malayalam}}
\flushright{\begin{Arabic}
\quranayah[34][9]
\end{Arabic}}
\flushleft{\begin{malayalam}
അവരുടെ മുമ്പിലും അവരുടെ പിന്നിലുമുള്ള ആകാശത്തേക്കും ഭൂമിയിലേക്കും അവര്‍ നോക്കിയിട്ടില്ലേ? നാം ഉദ്ദേശിക്കുകയാണെങ്കില്‍ അവരെ നാം ഭൂമിയില്‍ ആഴ്ത്തിക്കളയുകയോ അവരുടെ മേല്‍ ആകാശത്ത് നിന്ന് കഷ്ണങ്ങള്‍ വീഴ്ത്തുകയോ ചെയ്യുന്നതാണ്‌. അല്ലാഹുവിലേക്ക് (വിനയാന്വിതനായി) മടങ്ങുന്ന ഏതൊരു ദാസനും തീര്‍ച്ചയായും അതില്‍ ദൃഷ്ടാന്തമുണ്ട്‌.
\end{malayalam}}
\flushright{\begin{Arabic}
\quranayah[34][10]
\end{Arabic}}
\flushleft{\begin{malayalam}
തീര്‍ച്ചയായും ദാവൂദിന് നാം നമ്മുടെ പക്കല്‍ നിന്ന് അനുഗ്രഹം നല്‍കുകയുണ്ടായി.(നാം നിര്‍ദേശിച്ചു:) പര്‍വ്വതങ്ങളേ, നിങ്ങള്‍ അദ്ദേഹത്തോടൊപ്പം (കീര്‍ത്തനങ്ങള്‍) ഏറ്റുചൊല്ലുക. പക്ഷികളേ, നിങ്ങളും നാം അദ്ദേഹത്തിന് ഇരുമ്പ് മയപ്പെടുത്തികൊടുക്കുകയും ചെയ്തു.
\end{malayalam}}
\flushright{\begin{Arabic}
\quranayah[34][11]
\end{Arabic}}
\flushleft{\begin{malayalam}
പൂര്‍ണ്ണവലുപ്പമുള്ള കവചങ്ങള്‍ നിര്‍മിക്കുകയും, അതിന്‍റെ കണ്ണികള്‍ ശരിയായ അളവിലാക്കുകയും, നിങ്ങളെല്ലാവരും സല്‍കര്‍മ്മം പ്രവര്‍ത്തിക്കുകയും ചെയ്യുക എന്ന് (നാം അദ്ദേഹത്തിന് നിര്‍ദേശം നല്‍കി.) തീര്‍ച്ചയായും ഞാന്‍ നിങ്ങള്‍ പ്രവര്‍ത്തിക്കുന്നതെല്ലാം കാണുന്നവനാകുന്നു.
\end{malayalam}}
\flushright{\begin{Arabic}
\quranayah[34][12]
\end{Arabic}}
\flushleft{\begin{malayalam}
സുലൈമാന്ന് കാറ്റിനെയും (നാം അധീനപ്പെടുത്തികൊടുത്തു.) അതിന്‍റെ പ്രഭാത സഞ്ചാരം ഒരു മാസത്തെ ദൂരവും അതിന്‍റെ സായാഹ്ന സഞ്ചാരം ഒരു മാസത്തെ ദൂരവുമാകുന്നു. അദ്ദേഹത്തിന് നാം ചെമ്പിന്‍റെ ഒരു ഉറവ് ഒഴുക്കികൊടുക്കുകയും ചെയ്തു. അദ്ദേഹത്തിന്‍റെ രക്ഷിതാവിന്‍റെ കല്‍പനപ്രകാരം അദ്ദേഹത്തിന്‍റെ മുമ്പാകെ ജിന്നുകളില്‍ ചിലര്‍ ജോലി ചെയ്യുന്നുമുണ്ടായിരുന്നു. അവരില്‍ ആരെങ്കിലും നമ്മുടെ കല്‍പനക്ക് എതിരുപ്രവര്‍ത്തിക്കുന്ന പക്ഷം നാം അവന്ന് ജ്വലിക്കുന്ന നരകശിക്ഷ ആസ്വദിപ്പിക്കുന്നതാണ്‌.
\end{malayalam}}
\flushright{\begin{Arabic}
\quranayah[34][13]
\end{Arabic}}
\flushleft{\begin{malayalam}
അദ്ദേഹത്തിന് വേണ്ടി ഉന്നത സൌധങ്ങള്‍, ശില്‍പങ്ങള്‍, വലിയ ജലസംഭരണിപോലെയുള്ള തളികകള്‍, നിലത്ത് ഉറപ്പിച്ച് നിര്‍ത്തിയിട്ടുള്ള പാചക പാത്രങ്ങള്‍ എന്നിങ്ങനെ അദ്ദേഹം ഉദ്ദേശിക്കുന്നതെന്തും അവര്‍ (ജിന്നുകള്‍) നിര്‍മിച്ചിരുന്നു. ദാവൂദ് കുടുംബമേ, നിങ്ങള്‍ നന്ദിപൂര്‍വ്വം പ്രവര്‍ത്തിക്കുക. തികഞ്ഞ നന്ദിയുള്ളവര്‍ എന്‍റെ ദാസന്‍മാരില്‍ അപൂര്‍വ്വമത്രെ.
\end{malayalam}}
\flushright{\begin{Arabic}
\quranayah[34][14]
\end{Arabic}}
\flushleft{\begin{malayalam}
നാം അദ്ദേഹത്തിന്‍റെ മേല്‍ മരണം വിധിച്ചപ്പോള്‍ അദ്ദേഹത്തിന്‍റെ ഊന്നുവടി തിന്നുകൊണ്ടിരുന്ന ചിതല്‍ മാത്രമാണ് അദ്ദേഹത്തിന്‍റെ മരണത്തെപ്പറ്റി അവര്‍ക്ക് (ജിന്നുകള്‍ക്ക്‌) അറിവ് നല്‍കിയത്‌. അങ്ങനെ അദ്ദേഹം വീണപ്പോള്‍, തങ്ങള്‍ക്ക് അദൃശ്യകാര്യം അറിയാമായിരുന്നെങ്കില്‍ അപമാനകരമായ ശിക്ഷയില്‍ തങ്ങള്‍ കഴിച്ചുകൂട്ടേണ്ടിവരില്ലായിരുന്നു എന്ന് ജിന്നുകള്‍ക്ക് ബോധ്യമായി.
\end{malayalam}}
\flushright{\begin{Arabic}
\quranayah[34][15]
\end{Arabic}}
\flushleft{\begin{malayalam}
തീര്‍ച്ചയായും സബഅ് ദേശക്കാര്‍ക്ക് തങ്ങളുടെ അധിവാസ കേന്ദ്രത്തില്‍ തന്നെ ദൃഷ്ടാന്തമുണ്ടായിരുന്നു. അതയാത്‌, വലതുഭാഗത്തും ഇടതുഭാഗത്തുമായി രണ്ടു തോട്ടങ്ങള്‍. (അവരോട് പറയപ്പെട്ടു:) നിങ്ങളുടെ രക്ഷിതാവ് തന്ന ഉപജീവനത്തില്‍ നിന്ന് നിങ്ങള്‍ ഭക്ഷിക്കുകയും, അവനോട് നിങ്ങള്‍ നന്ദികാണിക്കുകയും ചെയ്യുക. നല്ലൊരു രാജ്യവും ഏറെ പൊറുക്കുന്ന രക്ഷിതാവും.
\end{malayalam}}
\flushright{\begin{Arabic}
\quranayah[34][16]
\end{Arabic}}
\flushleft{\begin{malayalam}
എന്നാല്‍ അവര്‍ പിന്തിരിഞ്ഞ് കളഞ്ഞു. അപ്പോള്‍ അണക്കെട്ടില്‍ നിന്നുള്ള ജലപ്രവാഹത്തെ അവരുടെ നേരെ നാം അയച്ചു. അവരുടെ ആ രണ്ട് തോട്ടങ്ങള്‍ക്ക് പകരം കയ്പുള്ള കായ്കനികളും കാറ്റാടി മരവും, അല്‍പം ചില വാകമരങ്ങളും ഉള്ള രണ്ട് തോട്ടങ്ങള്‍ നാം അവര്‍ക്ക് നല്‍കുകയും ചെയ്തു.
\end{malayalam}}
\flushright{\begin{Arabic}
\quranayah[34][17]
\end{Arabic}}
\flushleft{\begin{malayalam}
അവര്‍ നന്ദികേട് കാണിച്ചതിന് നാം അവര്‍ക്ക് പ്രതിഫലമായി നല്‍കിയതാണത്‌. കടുത്ത നന്ദികേട് കാണിക്കുന്നവന്‍റെ നേരെയല്ലാതെ നാം ശിക്ഷാനടപടി എടുക്കുമോ?
\end{malayalam}}
\flushright{\begin{Arabic}
\quranayah[34][18]
\end{Arabic}}
\flushleft{\begin{malayalam}
അവര്‍ക്കും (സബഅ് ദേശക്കാര്‍ക്കും) നാം അനുഗ്രഹം നല്‍കിയ (സിറിയന്‍) ഗ്രാമങ്ങള്‍ക്കുമിടയില്‍ തെളിഞ്ഞ് കാണാവുന്ന പല ഗ്രാമങ്ങളും നാം ഉണ്ടാക്കി. അവിടെ നാം യാത്രയ്ക്ക് താവളങ്ങള്‍ നിര്‍ണയിക്കുകയും ചെയ്തു. രാപകലുകളില്‍ നിര്‍ഭയരായിക്കൊണ്ട് നിങ്ങള്‍ അതിലൂടെ സഞ്ചരിച്ച് കൊള്ളുക. (എന്ന് നാം നിര്‍ദേശിക്കുകയും ചെയ്തു.)
\end{malayalam}}
\flushright{\begin{Arabic}
\quranayah[34][19]
\end{Arabic}}
\flushleft{\begin{malayalam}
അപ്പോള്‍ അവര്‍ പറഞ്ഞു: ഞങ്ങളുടെ രക്ഷിതാവേ, ഞങ്ങളുടെ യാത്രാതാവളങ്ങള്‍ക്കിടയില്‍ നീ അകലമുണ്ടാക്കണമേ. അങ്ങനെ തങ്ങള്‍ക്കു തന്നെ അവര്‍ ദ്രോഹം വരുത്തി വെച്ചു. അപ്പോള്‍ നാം അവരെ കഥാവശേഷരാക്കികളഞ്ഞു. അവരെ നാം സര്‍വ്വത്ര ഛിന്നഭിന്നമാക്കി ക്ഷമാശീലനും നന്ദിയുള്ളവനുമായ ഏതൊരാള്‍ക്കും തീര്‍ച്ചയായും അതില്‍ ദൃഷ്ടാന്തങ്ങളുണ്ട്‌.
\end{malayalam}}
\flushright{\begin{Arabic}
\quranayah[34][20]
\end{Arabic}}
\flushleft{\begin{malayalam}
തീര്‍ച്ചയായും തന്‍റെ ധാരണ ശരിയാണെന്ന് ഇബ്ലീസ് അവരില്‍ തെളിയിച്ചു. അങ്ങനെ അവര്‍ അവനെ പിന്തുടര്‍ന്നു. ഒരു സംഘം സത്യവിശ്വാസികളൊഴികെ.
\end{malayalam}}
\flushright{\begin{Arabic}
\quranayah[34][21]
\end{Arabic}}
\flushleft{\begin{malayalam}
അവന്ന് (ഇബ്ലീസിന്‌) അവരുടെ മേല്‍ യാതൊരധികാരവും ഉണ്ടായിരുന്നില്ല. പരലോകത്തില്‍ വിശ്വസിക്കുന്നവരെ അതിനെ പറ്റി സംശയത്തില്‍ കഴിയുന്നവരുടെ കൂട്ടത്തില്‍ നിന്ന് നാം തിരിച്ചറിയുവാന്‍ വേണ്ടി മാത്രമാണിത്‌. നിന്‍റെ രക്ഷിതാവ് ഏതു കാര്യവും നിരീക്ഷിച്ച് കൊണ്ടിരിക്കുന്നവനാകുന്നു.
\end{malayalam}}
\flushright{\begin{Arabic}
\quranayah[34][22]
\end{Arabic}}
\flushleft{\begin{malayalam}
പറയുക: അല്ലാഹുവിന് പുറമെ നിങ്ങള്‍ ജല്‍പിച്ചുകൊണ്ടിരിക്കുന്നവരോടെല്ലാം നിങ്ങള്‍ പ്രാര്‍ത്ഥിച്ച് നോക്കുക. ആകാശത്തിലാകട്ടെ ഭൂമിയിലാകട്ടെ ഒരു അണുവിന്‍റെ തൂക്കം പോലും അവര്‍ ഉടമപ്പെടുത്തുന്നില്ല. അവ രണ്ടിലും അവര്‍ക്ക് യാതൊരു പങ്കുമില്ല. അവരുടെ കൂട്ടത്തില്‍ അവന്ന് സഹായിയായി ആരുമില്ല.
\end{malayalam}}
\flushright{\begin{Arabic}
\quranayah[34][23]
\end{Arabic}}
\flushleft{\begin{malayalam}
ആര്‍ക്കു വേണ്ടി അവന്‍ അനുമതി നല്‍കിയോ അവര്‍ക്കല്ലാതെ അവന്‍റെ അടുക്കല്‍ ശുപാര്‍ശ പ്രയോജനപ്പെടുകയുമില്ല. അങ്ങനെ അവരുടെ ഹൃദയങ്ങളില്‍ നിന്ന് പരിഭ്രമം നീങ്ങികഴിയുമ്പോള്‍ അവര്‍ ചോദിക്കും; നിങ്ങളുടെ രക്ഷിതാവ് എന്താണു പറഞ്ഞതെന്ന് അവര്‍ മറുപടി പറയും: സത്യമാണ് (അവന്‍ പറഞ്ഞത്‌) അവന്‍ ഉന്നതനും മഹാനുമാകുന്നു.
\end{malayalam}}
\flushright{\begin{Arabic}
\quranayah[34][24]
\end{Arabic}}
\flushleft{\begin{malayalam}
ചോദിക്കുക: ആകാശങ്ങളില്‍ നിന്നും ഭൂമിയില്‍ നിന്നും നിങ്ങള്‍ക്ക് ഉപജീവനം നല്‍കുന്നവന്‍ ആരാകുന്നു? നീ പറയുക: അല്ലാഹുവാകുന്നു. തീര്‍ച്ചയായും ഒന്നുകില്‍ ഞങ്ങള്‍ അല്ലെങ്കില്‍ നിങ്ങള്‍ സന്‍മാര്‍ഗത്തിലാകുന്നു. അല്ലെങ്കില്‍ വ്യക്തമായ ദുര്‍മാര്‍ഗത്തില്‍.
\end{malayalam}}
\flushright{\begin{Arabic}
\quranayah[34][25]
\end{Arabic}}
\flushleft{\begin{malayalam}
പറയുക: ഞങ്ങള്‍ കുറ്റം ചെയ്തതിനെപ്പറ്റി നിങ്ങള്‍ ചോദിക്കപ്പെടുകയില്ല. നിങ്ങള്‍ പ്രവര്‍ത്തിക്കുന്നതിനെ പറ്റി ഞങ്ങളും ചോദിക്കപ്പെടുകയില്ല.
\end{malayalam}}
\flushright{\begin{Arabic}
\quranayah[34][26]
\end{Arabic}}
\flushleft{\begin{malayalam}
പറയുക: നമ്മുടെ രക്ഷിതാവ് നമ്മെ തമ്മില്‍ ഒരുമിച്ചുകൂട്ടുകയും, അനന്തരം നമുക്കിടയില്‍ അവന്‍ സത്യപ്രകാരം തീര്‍പ്പുകല്‍പിക്കുകയും ചെയ്യുന്നതാണ്‌. അവന്‍ സര്‍വ്വജ്ഞനായ തീര്‍പ്പുകാരനത്രെ.
\end{malayalam}}
\flushright{\begin{Arabic}
\quranayah[34][27]
\end{Arabic}}
\flushleft{\begin{malayalam}
പറയുക: പങ്കുകാരെന്ന നിലയില്‍ അവനോട് (അല്ലാഹുവോട്‌) നിങ്ങള്‍ കൂട്ടിചേര്‍ത്തിട്ടുള്ളവരെ എനിക്ക് നിങ്ങളൊന്ന് കാണിച്ചുതരൂ. ഇല്ല, (അങ്ങനെ ഒരു പങ്കാളിയുമില്ല.) എന്നാല്‍ അവന്‍ പ്രതാപിയും യുക്തിമാനുമായ അല്ലാഹുവത്രെ.
\end{malayalam}}
\flushright{\begin{Arabic}
\quranayah[34][28]
\end{Arabic}}
\flushleft{\begin{malayalam}
നിന്നെ നാം മനുഷ്യര്‍ക്കാകമാനം സന്തോഷവാര്‍ത്ത അറിയിക്കുവാനും താക്കീത് നല്‍കുവാനും ആയികൊണ്ട് തന്നെയാണ് അയച്ചിട്ടുള്ളത്‌. പക്ഷെ, മനുഷ്യരില്‍ അധികപേരും അറിയുന്നില്ല.
\end{malayalam}}
\flushright{\begin{Arabic}
\quranayah[34][29]
\end{Arabic}}
\flushleft{\begin{malayalam}
അവര്‍ ചോദിക്കുന്നു; നിങ്ങള്‍ സത്യവാദികളാണെങ്കില്‍, ഈ താക്കീത് എപ്പോഴാണ് (പുലരുക) എന്ന്‌.
\end{malayalam}}
\flushright{\begin{Arabic}
\quranayah[34][30]
\end{Arabic}}
\flushleft{\begin{malayalam}
പറയുക: നിങ്ങള്‍ക്കൊരു നിശ്ചിത ദിവസമുണ്ട്‌. അത് വിട്ട് ഒരു നിമിഷം പോലും നിങ്ങള്‍ പിന്നോട്ട് പോകുകയോ, മുന്നോട്ട് പോകുകയോ ഇല്ല.
\end{malayalam}}
\flushright{\begin{Arabic}
\quranayah[34][31]
\end{Arabic}}
\flushleft{\begin{malayalam}
ഈ ഖുര്‍ആനിലാകട്ടെ, ഇതിന് മുമ്പുള്ള വേദത്തിലാകട്ടെ ഞങ്ങള്‍ വിശ്വസിക്കുന്നതേ അല്ല എന്ന് സത്യനിഷേധികള്‍ പറഞ്ഞു. (നബിയേ,) ഈ അക്രമികള്‍ തങ്ങളുടെ രക്ഷിതാവിന്‍റെ അടുക്കല്‍ നിര്‍ത്തപ്പെടുന്ന സന്ദര്‍ഭം നീ കണ്ടിരുന്നെങ്കില്‍! അവരില്‍ ഓരോ വിഭാഗവും മറുവിഭാഗത്തിന്‍റെ മേല്‍ കുറ്റം ആരോപിച്ച് കൊണ്ടിരിക്കും. ബലഹീനരായി ഗണിക്കപ്പെട്ടവര്‍ വലുപ്പം നടിച്ചിരുന്നവരോട് പറയും: നിങ്ങളില്ലായിരുന്നെങ്കില്‍ ഞങ്ങള്‍ വിശ്വാസികളായിരുന്നേനെ.
\end{malayalam}}
\flushright{\begin{Arabic}
\quranayah[34][32]
\end{Arabic}}
\flushleft{\begin{malayalam}
വലുപ്പം നടിച്ചവര്‍ ബലഹീനരായി ഗണിക്കപ്പെട്ടവരോട് പറയും: മാര്‍ഗദര്‍ശനം നിങ്ങള്‍ക്ക് വന്നെത്തിയതിന് ശേഷം അതില്‍ നിന്ന് നിങ്ങളെ തടഞ്ഞത് ഞങ്ങളാണോ? അല്ല, നിങ്ങള്‍ കുറ്റവാളികള്‍ തന്നെയായിരുന്നു.
\end{malayalam}}
\flushright{\begin{Arabic}
\quranayah[34][33]
\end{Arabic}}
\flushleft{\begin{malayalam}
ബലഹീനരായി ഗണിക്കപ്പെട്ടവര്‍ വലുപ്പം നടിച്ചവരോട് പറയും: അല്ല, ഞങ്ങള്‍ അല്ലാഹുവില്‍ അവിശ്വസിക്കാനും, അവന്ന് സമന്‍മാരെ സ്ഥാപിക്കുവാനും നിങ്ങള്‍ ഞങ്ങളോട് കല്‍പിച്ചു കൊണ്ടിരുന്ന സന്ദര്‍ഭത്തില്‍ (നിങ്ങള്‍) രാവും പകലും നടത്തിയ കുതന്ത്രത്തിന്‍റെ ഫലമാണത്‌. ശിക്ഷ കാണുമ്പോള്‍ അവര്‍ ഖേദം മനസ്സില്‍ ഒളിപ്പിക്കും. സത്യനിഷേധികളുടെ കഴുത്തുകളില്‍ നാം ചങ്ങലകള്‍ വെക്കുകയും ചെയ്യും. തങ്ങള്‍ പ്രവര്‍ത്തിച്ചിരുന്നതിന്‍റെ ഫലമല്ലാതെ അവര്‍ക്ക് നല്‍കപ്പെടുമോ
\end{malayalam}}
\flushright{\begin{Arabic}
\quranayah[34][34]
\end{Arabic}}
\flushleft{\begin{malayalam}
ഏതൊരു നാട്ടില്‍ നാം താക്കീതുകാരനെ അയച്ചപ്പോഴും, നിങ്ങള്‍ എന്തൊന്നുമായി നിയോഗിക്കപ്പെട്ടിരിക്കുന്നുവോ അതില്‍ ഞങ്ങള്‍ അവിശ്വസിക്കുന്നവരാകുന്നു എന്ന് അവിടത്തെ സുഖലോലുപര്‍ പറയാതിരുന്നിട്ടില്ല.
\end{malayalam}}
\flushright{\begin{Arabic}
\quranayah[34][35]
\end{Arabic}}
\flushleft{\begin{malayalam}
അവര്‍ പറഞ്ഞു: ഞങ്ങള്‍ കൂടുതല്‍ സ്വത്തുക്കളും സന്താനങ്ങളുമുള്ളവരാകുന്നു. ഞങ്ങള്‍ ശിക്ഷിക്കപ്പെടുന്നവരല്ല.
\end{malayalam}}
\flushright{\begin{Arabic}
\quranayah[34][36]
\end{Arabic}}
\flushleft{\begin{malayalam}
നീ പറയുക: തീര്‍ച്ചയായും എന്‍റെ രക്ഷിതാവ് താന്‍ ഉദ്ദേശിക്കുന്നവര്‍ക്ക് ഉപജീവനം വിശാലമാക്കുകയും (താന്‍ ഉദ്ദേശിക്കുന്നവര്‍ക്ക്‌) അത് ഇടുങ്ങിയതാക്കുകയും ചെയ്യുന്നു. പക്ഷെ ജനങ്ങളില്‍ അധികപേരും അറിയുന്നില്ല.
\end{malayalam}}
\flushright{\begin{Arabic}
\quranayah[34][37]
\end{Arabic}}
\flushleft{\begin{malayalam}
നിങ്ങളുടെ സമ്പത്തുക്കളും നിങ്ങളുടെ സന്താനങ്ങളുമൊന്നും നമ്മുടെ അടുക്കല്‍ നിങ്ങള്‍ക്ക് സാമീപ്യമുണ്ടാക്കിത്തരുന്നവയല്ല. വിശ്വസിക്കുകയും നല്ലത് പ്രവര്‍ത്തിക്കുകയും ചെയ്തവര്‍ക്കൊഴികെ. അത്തരക്കാര്‍ക്ക് തങ്ങള്‍ പ്രവര്‍ത്തിച്ചതിന്‍റെ ഫലമായി ഇരട്ടി പ്രതിഫലമുണ്ട്‌. അവര്‍ ഉന്നത സൌധങ്ങളില്‍ നിര്‍ഭയരായി കഴിയുന്നതുമാണ്‌.
\end{malayalam}}
\flushright{\begin{Arabic}
\quranayah[34][38]
\end{Arabic}}
\flushleft{\begin{malayalam}
(നമ്മെ) തോല്‍പിക്കുവാനായി നമ്മുടെ ദൃഷ്ടാന്തങ്ങളെ എതിര്‍ക്കുവാന്‍ ശ്രമിക്കുന്നവരാരോ അവര്‍ ശിക്ഷയില്‍ ഹാജരാക്കപ്പെടുന്നവരാകുന്നു.
\end{malayalam}}
\flushright{\begin{Arabic}
\quranayah[34][39]
\end{Arabic}}
\flushleft{\begin{malayalam}
നീ പറയുക: തീര്‍ച്ചയായും എന്‍റെ രക്ഷിതാവ് തന്‍റെ ദാസന്‍മാരില്‍ നിന്ന് താന്‍ ഉദ്ദേശിക്കുന്നവര്‍ക്ക് ഉപജീവനം വിശാലമാക്കുകയും, താന്‍ ഉദ്ദേശിക്കുന്നവര്‍ക്ക് ഇടുങ്ങിയതാക്കുകയും ചെയ്യുന്നതാണ്‌. നിങ്ങള്‍ എന്തൊന്ന് ചെലവഴിച്ചാലും അവന്‍ അതിന് പകരം നല്‍കുന്നതാണ്‌. അവന്‍ ഉപജീവനം നല്‍കുന്നവരില്‍ ഏറ്റവും ഉത്തമനത്രെ.
\end{malayalam}}
\flushright{\begin{Arabic}
\quranayah[34][40]
\end{Arabic}}
\flushleft{\begin{malayalam}
അവരെ മുഴുവന്‍ അവന്‍ ഒരുമിച്ചുകൂട്ടുന്ന ദിവസം (ശ്രദ്ധേയമാകുന്നു.) എന്നിട്ട് അവന്‍ മലക്കുകളോട് ചോദിക്കും: നിങ്ങളെയാണോ ഇക്കൂട്ടര്‍ ആരാധിച്ചിരുന്നത് ?
\end{malayalam}}
\flushright{\begin{Arabic}
\quranayah[34][41]
\end{Arabic}}
\flushleft{\begin{malayalam}
അവര്‍ പറയും: നീ എത്ര പരിശുദ്ധന്‍! നീയാണ് ഞങ്ങളുടെ രക്ഷാധികാരി. അവരല്ല. എന്നാല്‍ അവര്‍ ജിന്നുകളെയായിരുന്നു ആരാധിച്ചിരുന്നത് അവരില്‍ അധികപേരും അവരില്‍ (ജിന്നുകളില്‍) വിശ്വസിക്കുന്നവരത്രെ.
\end{malayalam}}
\flushright{\begin{Arabic}
\quranayah[34][42]
\end{Arabic}}
\flushleft{\begin{malayalam}
ആകയാല്‍ അന്ന് നിങ്ങള്‍ക്ക് അന്യോന്യം ഉപകാരമോ ഉപദ്രവമോ ചെയ്യാന്‍ കഴിവുണ്ടായിരിക്കുന്നതല്ല. അക്രമം ചെയ്തവരോട് നിങ്ങള്‍ നിഷേധിച്ച് തള്ളിക്കൊണ്ടിരുന്ന ആ നരക ശിക്ഷ നിങ്ങള്‍ ആസ്വദിച്ച് കൊള്ളുക. എന്ന് നാം പറയുകയും ചെയ്യും.
\end{malayalam}}
\flushright{\begin{Arabic}
\quranayah[34][43]
\end{Arabic}}
\flushleft{\begin{malayalam}
നമ്മുടെ ദൃഷ്ടാന്തങ്ങള്‍ സ്പഷ്ടമായ നിലയില്‍ അവര്‍ക്ക് വായിച്ചുകേള്‍പിക്കപ്പെട്ടാല്‍ അവര്‍ (ജനങ്ങളോട്‌) പറയും: നിങ്ങളുടെ പിതാക്കന്‍മാര്‍ ആരാധിച്ച് വന്നിരുന്നതില്‍ നിന്ന് നിങ്ങളെ തടയുവാന്‍ ആഗ്രഹിക്കുന്ന ഒരാള്‍ മാത്രമാണിത്‌. ഇത് കെട്ടിച്ചമച്ചുണ്ടാക്കിയ കള്ളം മാത്രമാണ് എന്നും അവര്‍ പറയും. തങ്ങള്‍ക്ക് സത്യം വന്നുകിട്ടിയപ്പോള്‍ അതിനെ പറ്റി അവിശ്വാസികള്‍ പറഞ്ഞു: ഇത് സ്പഷ്ടമായ ജാലവിദ്യ മാത്രമാകുന്നു.
\end{malayalam}}
\flushright{\begin{Arabic}
\quranayah[34][44]
\end{Arabic}}
\flushleft{\begin{malayalam}
അവര്‍ക്ക് പഠിക്കാനുള്ള വേദഗ്രന്ഥങ്ങളൊന്നും നാം അവര്‍ക്ക് നല്‍കിയിരുന്നില്ല. നിനക്ക് മുമ്പ് അവരിലേക്ക് ഒരു താക്കീതുകാരനെയും നാം നിയോഗിച്ചിരുന്നുമില്ല.
\end{malayalam}}
\flushright{\begin{Arabic}
\quranayah[34][45]
\end{Arabic}}
\flushleft{\begin{malayalam}
ഇവര്‍ക്ക് മുമ്പുള്ളവരും നിഷേധിച്ച് തള്ളിയിട്ടുണ്ട്‌. അവര്‍ക്ക് നാം കൊടുത്തിരുന്നതിന്‍റെ പത്തിലൊന്നുപോലും ഇവര്‍ നേടിയിട്ടില്ല. അങ്ങനെ നമ്മുടെ ദൂതന്‍മാരെ അവര്‍ നിഷേധിച്ചു തള്ളി. അപ്പോള്‍ എന്‍റെ രോഷം എങ്ങനെയുള്ളതായിരുന്നു!
\end{malayalam}}
\flushright{\begin{Arabic}
\quranayah[34][46]
\end{Arabic}}
\flushleft{\begin{malayalam}
നീ പറയുക: ഞാന്‍ നിങ്ങളോട് ഒരു കാര്യം മാത്രമേ ഉപദേശിക്കുന്നുള്ളൂ. അല്ലാഹുവിന് വേണ്ടി നിങ്ങള്‍ ഈരണ്ടു പേരായോ ഒറ്റയായോ നില്‍ക്കുകയും എന്നിട്ട് നിങ്ങള്‍ ചിന്തിക്കുകയും ചെയ്യണമെന്ന് നിങ്ങളുടെ കൂട്ടുകാരന്ന് (മുഹമ്മദ് നബി (സ)ക്ക്‌) യാതൊരു ഭ്രാന്തുമില്ല. ഭയങ്കരമായ ശിക്ഷയുടെ മുമ്പില്‍ നിങ്ങള്‍ക്കു താക്കീത് നല്‍കുന്ന ആള്‍ മാത്രമാകുന്നു അദ്ദേഹം.
\end{malayalam}}
\flushright{\begin{Arabic}
\quranayah[34][47]
\end{Arabic}}
\flushleft{\begin{malayalam}
നീ പറയുക: നിങ്ങളോട് ഞാന്‍ വല്ല പ്രതിഫലവും ആവശ്യപ്പെട്ടിട്ടുണ്ടെങ്കില്‍ അത് നിങ്ങള്‍ക്ക് വേണ്ടിതന്നെയാകുന്നു. എനിക്കുള്ള പ്രതിഫലം അല്ലാഹുവിങ്കല്‍ നിന്ന് മാത്രമാകുന്നു. അവന്‍ എല്ലാകാര്യത്തിനും സാക്ഷിയാകുന്നു.
\end{malayalam}}
\flushright{\begin{Arabic}
\quranayah[34][48]
\end{Arabic}}
\flushleft{\begin{malayalam}
തീര്‍ച്ചയായും എന്‍റെ രക്ഷിതാവ് സത്യത്തെ ഇട്ടുതരുന്നു. (അവന്‍) അദൃശ്യകാര്യങ്ങള്‍ നല്ലവണ്ണം അറിയുന്നവനാകുന്നു.
\end{malayalam}}
\flushright{\begin{Arabic}
\quranayah[34][49]
\end{Arabic}}
\flushleft{\begin{malayalam}
നീ പറയുക: സത്യം വന്നു കഴിഞ്ഞു. അസത്യം (യാതൊന്നിനും) തുടക്കം കുറിക്കുകയില്ല. (യാതൊന്നും) പുനസ്ഥാപിക്കുകയുമില്ല.
\end{malayalam}}
\flushright{\begin{Arabic}
\quranayah[34][50]
\end{Arabic}}
\flushleft{\begin{malayalam}
നീ പറയുക: ഞാന്‍ പിഴച്ച് പോയിട്ടുണ്ടെങ്കില്‍ ഞാന്‍ പിഴക്കുന്നതിന്‍റെ ദോഷം എനിക്കു തന്നെയാണ്‌. ഞാന്‍ നേര്‍മാര്‍ഗം പ്രാപിച്ചുവെങ്കിലോ അത് എനിക്ക് എന്‍റെ രക്ഷിതാവ് ബോധനം നല്‍കുന്നതിന്‍റെ ഫലമായിട്ടാണ്‌. തീര്‍ച്ചയായും അവന്‍ കേള്‍ക്കുന്നവനും സമീപസ്ഥനുമാകുന്നു.
\end{malayalam}}
\flushright{\begin{Arabic}
\quranayah[34][51]
\end{Arabic}}
\flushleft{\begin{malayalam}
അവര്‍ (സത്യനിഷേധികള്‍) പരിഭ്രാന്തരായിപോയ സന്ദര്‍ഭം നീ കണ്ടിരുന്നെങ്കില്‍ എന്നാല്‍ അവര്‍ (പിടിയില്‍ നിന്ന്‌) ഒഴിവാകുകയില്ല. അടുത്ത സ്ഥലത്ത് നിന്ന് തന്നെ അവര്‍ പിടിക്കപ്പെടും.
\end{malayalam}}
\flushright{\begin{Arabic}
\quranayah[34][52]
\end{Arabic}}
\flushleft{\begin{malayalam}
ഇതില്‍ ഞങ്ങള്‍ വിശ്വസിച്ചിരിക്കുന്നു എന്നവര്‍ പറയുകയും ചെയ്യും. വിദൂരമായ ഒരു സ്ഥലത്ത് നിന്ന് അവര്‍ക്ക് എങ്ങനെയാണ് (ആ വിശ്വാസം) നേടിയെടുക്കാന്‍ കഴിയുക.
\end{malayalam}}
\flushright{\begin{Arabic}
\quranayah[34][53]
\end{Arabic}}
\flushleft{\begin{malayalam}
മുമ്പ് അവര്‍ അതില്‍ അവിശ്വസിച്ചതായിരുന്നു. വിദൂരസ്ഥലത്ത് നിന്ന് നേരിട്ടറിയാതെ അവര്‍ ആരോപണം ഉന്നയിക്കുകയും ചെയ്തിരുന്നു.
\end{malayalam}}
\flushright{\begin{Arabic}
\quranayah[34][54]
\end{Arabic}}
\flushleft{\begin{malayalam}
അങ്ങനെ മുമ്പ് അവരുടെ പക്ഷക്കാരെക്കൊണ്ട് ചെയ്തത് പോലെത്തന്നെ അവര്‍ക്കും അവര്‍ ആഗ്രഹിക്കുന്ന കാര്യത്തിനുമിടയില്‍ തടസ്സം സൃഷ്ടിക്കപ്പെട്ടു. തീര്‍ച്ചയായും അവര്‍ അവിശ്വാസജനകമായ സംശയത്തിലായിരുന്നു.
\end{malayalam}}
\chapter{\textmalayalam{ഫാത്വിര്‍ ( സ്രഷ്ടാവ് )}}
\begin{Arabic}
\Huge{\centerline{\basmalah}}\end{Arabic}
\flushright{\begin{Arabic}
\quranayah[35][1]
\end{Arabic}}
\flushleft{\begin{malayalam}
ആകാശങ്ങളും ഭൂമിയും സൃഷ്ടിച്ചുണ്ടാക്കിയവനും രണ്ടും മൂന്നും നാലും ചിറകുകളുള്ള മലക്കുകളെ ദൂതന്‍മാരായി നിയോഗിച്ചവനുമായ അല്ലാഹുവിന് സ്തുതി. സൃഷ്ടിയില്‍ താന്‍ ഉദ്ദേശിക്കുന്നത് അവന്‍ അധികമാക്കുന്നു. തീര്‍ച്ചയായും അല്ലാഹു ഏത് കാര്യത്തിനും കഴിവുള്ളവനാകുന്നു.
\end{malayalam}}
\flushright{\begin{Arabic}
\quranayah[35][2]
\end{Arabic}}
\flushleft{\begin{malayalam}
അല്ലാഹു മനുഷ്യര്‍ക്ക് വല്ല കാരുണ്യവും തുറന്നുകൊടുക്കുന്ന പക്ഷം അത് പിടിച്ച് വെക്കാനാരുമില്ല. അവന്‍ വല്ലതും പിടിച്ച് വെക്കുന്ന പക്ഷം അതിന് ശേഷം അത് വിട്ടുകൊടുക്കാനും ആരുമില്ല. അവനത്രെ പ്രതാപിയും യുക്തിമാനും.
\end{malayalam}}
\flushright{\begin{Arabic}
\quranayah[35][3]
\end{Arabic}}
\flushleft{\begin{malayalam}
മനുഷ്യരേ, അല്ലാഹു നിങ്ങള്‍ക്ക് ചെയ്ത അനുഗ്രഹം നിങ്ങള്‍ ഓര്‍മിക്കുക. ആകാശത്ത് നിന്നും ഭൂമിയില്‍ നിന്നും നിങ്ങള്‍ക്ക് ഉപജീവനം നല്‍കാന്‍ അല്ലാഹുവല്ലാത്ത വല്ല സ്രഷ്ടാവുമുണ്ടോ? അവനല്ലാതെ യാതൊരു ദൈവവുമില്ല. അപ്പോള്‍ നിങ്ങള്‍ എങ്ങനെയാണ് തെറ്റിക്കപ്പെടുന്നത്‌?
\end{malayalam}}
\flushright{\begin{Arabic}
\quranayah[35][4]
\end{Arabic}}
\flushleft{\begin{malayalam}
അവര്‍ നിന്നെ നിഷേധിച്ചു തള്ളുകയാണെങ്കില്‍ നിനക്ക് മുമ്പും ദൂതന്‍മാര്‍ നിഷേധിച്ചു തള്ളപ്പെട്ടിട്ടുണ്ട്‌. അല്ലാഹുവിങ്കലേക്കാണ് കാര്യങ്ങള്‍ മടക്കപ്പെടുന്നത്‌.
\end{malayalam}}
\flushright{\begin{Arabic}
\quranayah[35][5]
\end{Arabic}}
\flushleft{\begin{malayalam}
മനുഷ്യരേ, തീര്‍ച്ചയായും അല്ലാഹുവിന്‍റെ വാഗ്ദാനം സത്യമാകുന്നു. ഐഹികജീവിതം നിങ്ങളെ വഞ്ചിച്ച് കളയാതിരിക്കട്ടെ. പരമവഞ്ചകനായ പിശാചും അല്ലാഹുവിന്‍റെ കാര്യത്തില്‍ നിങ്ങളെ വഞ്ചിക്കാതിരിക്കട്ടെ.
\end{malayalam}}
\flushright{\begin{Arabic}
\quranayah[35][6]
\end{Arabic}}
\flushleft{\begin{malayalam}
തീര്‍ച്ചയായും പിശാച് നിങ്ങളുടെ ശത്രുവാകുന്നു. അതിനാല്‍ അവനെ നിങ്ങള്‍ ശത്രുവായിത്തന്നെ ഗണിക്കുക. അവന്‍ തന്‍റെ പക്ഷക്കാരെ ക്ഷണിക്കുന്നത് അവര്‍ നരകാവകാശികളുടെ കൂട്ടത്തിലായിരുക്കുവാന്‍ വേണ്ടി മാത്രമാണ്‌.
\end{malayalam}}
\flushright{\begin{Arabic}
\quranayah[35][7]
\end{Arabic}}
\flushleft{\begin{malayalam}
അവിശ്വസിച്ചവരാരോ അവര്‍ക്കു കഠിനശിക്ഷയുണ്ട്‌. വിശ്വസിക്കുകയും സല്‍കര്‍മ്മങ്ങള്‍ പ്രവര്‍ത്തിക്കുകയും ചെയ്തവരാരോ അവര്‍ക്ക് പാപമോചനവും വലിയ പ്രതിഫലവുമുണ്ട്‌.
\end{malayalam}}
\flushright{\begin{Arabic}
\quranayah[35][8]
\end{Arabic}}
\flushleft{\begin{malayalam}
എന്നാല്‍ തന്‍റെ ദുഷ്പ്രവൃത്തികള്‍ അലംകൃതമായി തോന്നിക്കപ്പെടുകയും, അങ്ങനെ അത് നല്ലതായി കാണുകയും ചെയ്തവന്‍റെ കാര്യമോ? അല്ലാഹു താന്‍ ഉദ്ദേശിക്കുന്നവരെ വഴിപിഴപ്പിക്കുകയും താന്‍ ഉദ്ദേശിക്കുന്നവരെ നേര്‍വഴിയിലാക്കുകയും ചെയ്യുന്നതാണ്‌. അതിനാല്‍ അവരെപ്പറ്റിയുള്ള കൊടുംഖേദം നിമിത്തം നിന്‍റെ പ്രാണന്‍ പോകാതിരിക്കട്ടെ. തീര്‍ച്ചയായും അല്ലാഹു അവര്‍ പ്രവര്‍ത്തിക്കുന്നതിനെപ്പറ്റി അറിവുള്ളവനാകുന്നു.
\end{malayalam}}
\flushright{\begin{Arabic}
\quranayah[35][9]
\end{Arabic}}
\flushleft{\begin{malayalam}
അല്ലാഹുവാണ് കാറ്റുകളെ അയച്ചവന്‍. അങ്ങനെ അവ മേഘത്തെ ഇളക്കിവിടുന്നു. എന്നിട്ട് ആ മേഘത്തെ നിര്‍ജീവമായ നാട്ടിലേക്ക് നാം തെളിച്ചുകൊണ്ട് പോകുകയും, അതുമുഖേന ഭൂമിയെ അതിന്‍റെ നിര്‍ജീവാവസ്ഥയ്ക്ക് ശേഷം നാം സജീവമാക്കുകയും ചെയ്യുന്നു. അതുപോലെ തന്നെയാകുന്നു ഉയിര്‍ത്തെഴുന്നേല്‍പ്‌.
\end{malayalam}}
\flushright{\begin{Arabic}
\quranayah[35][10]
\end{Arabic}}
\flushleft{\begin{malayalam}
ആരെങ്കിലും പ്രതാപം ആഗ്രഹിക്കുന്നുവെങ്കില്‍ പ്രതാപമെല്ലാം അല്ലാഹുവിന്‍റെ അധീനത്തിലാകുന്നു. അവങ്കലേക്കാണ് ഉത്തമ വചനങ്ങള്‍ കയറിപോകുന്നത്‌. നല്ല പ്രവര്‍ത്തനത്തെ അവന്‍ ഉയര്‍ത്തുകയും ചെയ്യുന്നു. ദുഷിച്ച തന്ത്രങ്ങള്‍ പ്രയോഗിക്കുന്നതാരോ അവര്‍ക്ക് കഠിനശിക്ഷയുണ്ട്‌. അത്തരക്കാരുടെ തന്ത്രം നാശമടയുക തന്നെ ചെയ്യും.
\end{malayalam}}
\flushright{\begin{Arabic}
\quranayah[35][11]
\end{Arabic}}
\flushleft{\begin{malayalam}
അല്ലാഹു നിങ്ങളെ മണ്ണില്‍ നിന്നും പിന്നീട് ബീജകണത്തില്‍ നിന്നും സൃഷ്ടിച്ചു. പിന്നെ അവന്‍ നിങ്ങളെ ഇണകളാക്കി. അവന്‍റെ അറിവനുസരിച്ചല്ലാതെ ഒരു സ്ത്രീയും ഗര്‍ഭം ധരിക്കുന്നില്ല, പ്രസവിക്കുന്നുമില്ല. ഒരു ദീര്‍ഘായുസ്സ് നല്‍കപ്പെട്ട ആള്‍ക്കും ആയുസ്സ് നീട്ടികൊടുക്കപ്പെടുന്നതോ അയാളുടെ ആയുസ്സില്‍ കുറവ് വരുത്തപ്പെടുന്നതോ ഒരു രേഖയില്‍ ഉള്ളത് അനുസരിച്ചല്ലാതെ നടക്കുന്നില്ല. തീര്‍ച്ചയായും അത് അല്ലാഹുവിന് എളുപ്പമുള്ളതാകുന്നു.
\end{malayalam}}
\flushright{\begin{Arabic}
\quranayah[35][12]
\end{Arabic}}
\flushleft{\begin{malayalam}
രണ്ടു ജലാശയങ്ങള്‍ സമമാവുകയില്ല. ഒന്ന് കുടിക്കാന്‍ സുഖമുള്ള ഹൃദ്യമായ ശുദ്ധജലം, മറ്റൊന്ന് കയ്പുറ്റ ഉപ്പു വെള്ളവും. രണ്ടില്‍ നിന്നും നിങ്ങള്‍ പുത്തന്‍മാംസം എടുത്ത് തിന്നുന്നു. നിങ്ങള്‍ക്ക് ധരിക്കുവാനുള്ള ആഭരണം (അതില്‍ നിന്ന്‌) പുറത്തെടുക്കുകയും ചെയ്യുന്നു. അതിലൂടെ കപ്പലുകള്‍ കീറിക്കടന്നു പോകുന്നതും നിനക്ക് കാണാം. അല്ലാഹുവിന്‍റെ അനുഗ്രഹത്തില്‍ നിന്നും നിങ്ങള്‍ തേടിപ്പിടിക്കുവാന്‍ വേണ്ടിയും നിങ്ങള്‍ നന്ദികാണിക്കുവാന്‍ വേണ്ടിയുമത്രെ അത്‌.
\end{malayalam}}
\flushright{\begin{Arabic}
\quranayah[35][13]
\end{Arabic}}
\flushleft{\begin{malayalam}
രാവിനെ അവന്‍ പകലില്‍ പ്രവേശിപ്പിക്കുന്നു. പകലിനെ രാവിലും പ്രവേശിപ്പിക്കുന്നു. സൂര്യനെയും ചന്ദ്രനെയും അവന്‍ (തന്‍റെ നിയമത്തിന്‌) വിധേയമാക്കുകയും ചെയ്തിരിക്കുന്നു. അവയോരോന്നും നിശ്ചിതമായ ഒരു പരിധി വരെ സഞ്ചരിക്കുന്നു. അങ്ങനെയുള്ളവനാകുന്നു നിങ്ങളുടെ രക്ഷിതാവായ അല്ലാഹു. അവന്നാകുന്നു ആധിപത്യം. അവനു പുറമെ ആരോട് നിങ്ങള്‍ പ്രാര്‍ത്ഥിക്കുന്നുവോ അവര്‍ ഒരു ഈന്തപ്പഴക്കുരുവിന്‍റെ പാടപോലും ഉടമപ്പെടുത്തുന്നില്ല.
\end{malayalam}}
\flushright{\begin{Arabic}
\quranayah[35][14]
\end{Arabic}}
\flushleft{\begin{malayalam}
നിങ്ങള്‍ അവരോട് പ്രാര്‍ത്ഥിക്കുന്ന പക്ഷം അവര്‍ നിങ്ങളുടെ പ്രാര്‍ത്ഥന കേള്‍ക്കുകയില്ല. അവര്‍ കേട്ടാലും നിങ്ങള്‍ക്കവര്‍ ഉത്തരം നല്‍കുന്നതല്ല. ഉയിര്‍ത്തെഴുന്നേല്‍പിന്‍റെ നാളിലാകട്ടെ നിങ്ങള്‍ അവരെ പങ്കാളികളാക്കിയതിനെ അവര്‍ നിഷേധിക്കുന്നതുമാണ്‌. സൂക്ഷ്മജ്ഞാനമുള്ളവനെ (അല്ലാഹുവെ) പ്പോലെ നിനക്ക് വിവരം തരാന്‍ ആരുമില്ല.
\end{malayalam}}
\flushright{\begin{Arabic}
\quranayah[35][15]
\end{Arabic}}
\flushleft{\begin{malayalam}
മനുഷ്യരേ, നിങ്ങള്‍ അല്ലാഹുവിന്‍റെ ആശ്രിതന്‍മാരാകുന്നു. അല്ലാഹുവാകട്ടെ സ്വയം പര്യാപ്തനും സ്തുത്യര്‍ഹനുമാകുന്നു.
\end{malayalam}}
\flushright{\begin{Arabic}
\quranayah[35][16]
\end{Arabic}}
\flushleft{\begin{malayalam}
അവന്‍ ഉദ്ദേശിക്കുന്ന പക്ഷം നിങ്ങളെ അവന്‍ നീക്കം ചെയ്യുകയും, പുതിയൊരു സൃഷ്ടിയെ അവന്‍ കൊണ്ടുവരുകയും ചെയ്യുന്നതാണ്‌.
\end{malayalam}}
\flushright{\begin{Arabic}
\quranayah[35][17]
\end{Arabic}}
\flushleft{\begin{malayalam}
അത് അല്ലാഹുവിന് പ്രയാസമുള്ള കാര്യമല്ല.
\end{malayalam}}
\flushright{\begin{Arabic}
\quranayah[35][18]
\end{Arabic}}
\flushleft{\begin{malayalam}
പാപഭാരം വഹിക്കുന്ന യാതൊരാളും മറ്റൊരാളുടെ പാപഭാരം ഏറ്റെടുക്കുകയില്ല. ഭാരം കൊണ്ട് ഞെരുങ്ങുന്ന ഒരാള്‍ തന്‍റെ ചുമട് താങ്ങുവാന്‍ (ആരെയെങ്കിലും) വിളിക്കുന്ന പക്ഷം അതില്‍ നിന്ന് ഒട്ടും തന്നെ ഏറ്റെടുക്കപ്പെടുകയുമില്ല. (വിളിക്കുന്നത്‌) അടുത്ത ബന്ധുവിനെയാണെങ്കില്‍ പോലും. തങ്ങളുടെ രക്ഷിതാവിനെ അദൃശ്യമായ വിധത്തില്‍ തന്നെ ഭയപ്പെടുകയും നമസ്കാരം മുറപോലെ നിര്‍വഹിക്കുകയും ചെയ്യുന്നവര്‍ക്ക് മാത്രമേ നിന്‍റെ താക്കീത് ഫലപ്പെടുകയുള്ളൂ. വല്ലവനും വിശുദ്ധി പാലിക്കുന്ന പക്ഷം തന്‍റെ സ്വന്തം നന്‍മക്കായി തന്നെയാണ് അവന്‍ വിശുദ്ധി പാലിക്കുന്നത.് അല്ലാഹുവിങ്കലേക്കാണ് മടക്കം.പ
\end{malayalam}}
\flushright{\begin{Arabic}
\quranayah[35][19]
\end{Arabic}}
\flushleft{\begin{malayalam}
അന്ധനും കാഴ്ചയുള്ളവനും സമമാവുകയില്ല.
\end{malayalam}}
\flushright{\begin{Arabic}
\quranayah[35][20]
\end{Arabic}}
\flushleft{\begin{malayalam}
ഇരുളുകളും വെളിച്ചവും (സമമാവുകയില്ല.)
\end{malayalam}}
\flushright{\begin{Arabic}
\quranayah[35][21]
\end{Arabic}}
\flushleft{\begin{malayalam}
തണലും ചൂടുള്ള വെയിലും (സമമാവുകയില്ല.)
\end{malayalam}}
\flushright{\begin{Arabic}
\quranayah[35][22]
\end{Arabic}}
\flushleft{\begin{malayalam}
ജീവിച്ചിരിക്കുന്നവരും മരിച്ചവരും സമമാകുകയില്ല. തീര്‍ച്ചയായും അല്ലാഹു അവന്‍ ഉദ്ദേശിക്കുന്നവരെ കേള്‍പിക്കുന്നു. നിനക്ക് ഖബ്‌റുകളിലുള്ളവരെ കേള്‍പിക്കാനാവില്ല.
\end{malayalam}}
\flushright{\begin{Arabic}
\quranayah[35][23]
\end{Arabic}}
\flushleft{\begin{malayalam}
നീ ഒരു താക്കീതുകാരന്‍ മാത്രമാകുന്നു.
\end{malayalam}}
\flushright{\begin{Arabic}
\quranayah[35][24]
\end{Arabic}}
\flushleft{\begin{malayalam}
തീര്‍ച്ചയായും നിന്നെ നാം അയച്ചിരിക്കുന്നത് സത്യവും കൊണ്ടാണ്‌. ഒരു സന്തോഷവാര്‍ത്ത അറിയിക്കുന്നവനും താക്കീതുകാരനുമായിട്ട്‌. ഒരു താക്കീതുകാരന്‍ കഴിഞ്ഞുപോകാത്ത ഒരു സമുദായവുമില്ല.
\end{malayalam}}
\flushright{\begin{Arabic}
\quranayah[35][25]
\end{Arabic}}
\flushleft{\begin{malayalam}
അവര്‍ നിന്നെ നിഷേധിച്ചു തള്ളുന്നുവെങ്കില്‍ അവര്‍ക്ക് മുമ്പുള്ളവരും നിഷേധിച്ചു തള്ളിയിട്ടുണ്ട്‌. അവരിലേക്കുള്ള ദൂതന്‍മാര്‍ പ്രത്യക്ഷലക്ഷ്യങ്ങളും ന്യായപ്രമാണങ്ങളും വെളിച്ചം നല്‍കുന്ന ഗ്രന്ഥവും കൊണ്ട് അവരുടെ അടുത്ത് ചെല്ലുകയുണ്ടായി.
\end{malayalam}}
\flushright{\begin{Arabic}
\quranayah[35][26]
\end{Arabic}}
\flushleft{\begin{malayalam}
പിന്നീട് നിഷേധിച്ചവരെ ഞാന്‍ പിടികൂടി. അപ്പോള്‍ എന്‍റെ രോഷം എങ്ങനെയുള്ളതായിരുന്നു!
\end{malayalam}}
\flushright{\begin{Arabic}
\quranayah[35][27]
\end{Arabic}}
\flushleft{\begin{malayalam}
നീ കണ്ടില്ലേ; അല്ലാഹു ആകാശത്ത് നിന്നും വെള്ളം ചൊരിഞ്ഞു. എന്നിട്ട് അത് മുഖേന വ്യത്യസ്ത വര്‍ണങ്ങളുള്ള പഴങ്ങള്‍ നാം ഉല്‍പാദിപ്പിച്ചു. പര്‍വ്വതങ്ങളിലുമുണ്ട് വെളുത്തതും ചുവന്നതുമായ നിറഭേദങ്ങളുള്ള പാതകള്‍. കറുത്തിരുണ്ടവയുമുണ്ട്‌.
\end{malayalam}}
\flushright{\begin{Arabic}
\quranayah[35][28]
\end{Arabic}}
\flushleft{\begin{malayalam}
മനുഷ്യരിലും മൃഗങ്ങളിലും കന്നുകാലികളിലും അതുപോലെ വിഭിന്ന വര്‍ണങ്ങളുള്ളവയുണ്ട്‌. അല്ലാഹുവെ ഭയപ്പെടുന്നത് അവന്‍റെ ദാസന്‍മാരില്‍ നിന്ന് അറിവുള്ളവര്‍ മാത്രമാകുന്നു. തീര്‍ച്ചയായും അല്ലാഹു പ്രതാപിയും ഏറെ പൊറുക്കുന്നവനുമാകുന്നു.
\end{malayalam}}
\flushright{\begin{Arabic}
\quranayah[35][29]
\end{Arabic}}
\flushleft{\begin{malayalam}
തീര്‍ച്ചയായും അല്ലാഹുവിന്‍റെ ഗ്രന്ഥം പാരായണം ചെയ്യുകയും, നമസ്കാരം മുറപോലെ നിര്‍വഹിക്കുകയും, നാം കൊടുത്തിട്ടുള്ളതില്‍ നിന്ന് രഹസ്യമായും പരസ്യമായും ചെലവഴിക്കുകയും ചെയ്യുന്നവരാരോ അവര്‍ ആശിക്കുന്നത് ഒരിക്കലും നഷ്ടം സംഭവിക്കാത്ത ഒരു കച്ചവടമാകുന്നു.
\end{malayalam}}
\flushright{\begin{Arabic}
\quranayah[35][30]
\end{Arabic}}
\flushleft{\begin{malayalam}
അവര്‍ക്ക് അവരുടെ പ്രതിഫലങ്ങള്‍ അവന്‍ പൂര്‍ത്തിയാക്കി കൊടുക്കുവാനും അവന്‍റെ അനുഗ്രഹത്തില്‍ നിന്ന് അവന്‍ അവര്‍ക്ക് കൂടുതലായി നല്‍കുവാനും വേണ്ടി. തീര്‍ച്ചയായും അവന്‍ ഏറെ പൊറുക്കുന്നവനും നന്ദിയുള്ളവനുമാകുന്നു.
\end{malayalam}}
\flushright{\begin{Arabic}
\quranayah[35][31]
\end{Arabic}}
\flushleft{\begin{malayalam}
നിനക്ക് നാം ബോധനം നല്‍കിയ ഗ്രന്ഥം തന്നെയാകുന്നു സത്യം. അതിന്‍റെ മുമ്പുള്ളതിനെ (വേദങ്ങളെ) സത്യപ്പെടുത്തുന്നതായിട്ട്‌. തീര്‍ച്ചയായും അല്ലാഹു തന്‍റെ ദാസന്‍മാരെപ്പറ്റി സൂക്ഷ്മമായി അറിയുന്നവനും കാണുന്നവനുമാകുന്നു.
\end{malayalam}}
\flushright{\begin{Arabic}
\quranayah[35][32]
\end{Arabic}}
\flushleft{\begin{malayalam}
പിന്നീട് നമ്മുടെ ദാസന്‍മാരില്‍ നിന്ന് നാം തെരഞ്ഞെടുത്തവര്‍ക്ക് നാം വേദഗ്രന്ഥം അവകാശപ്പെടുത്തികൊടുത്തു. അവരുടെ കൂട്ടത്തില്‍ സ്വന്തത്തോട് അന്യായം ചെയ്തവരുണ്ട്‌. മദ്ധ്യനിലപാടുകാരും അവരിലുണ്ട്‌. അല്ലാഹുവിന്‍റെ അനുമതിയോടെ നന്‍മകളില്‍ മുങ്കടന്നവരും അവരിലുണ്ട്‌. അതു തന്നെയാണ് മഹത്തായ അനുഗ്രഹം.
\end{malayalam}}
\flushright{\begin{Arabic}
\quranayah[35][33]
\end{Arabic}}
\flushleft{\begin{malayalam}
സ്ഥിരവാസത്തിനുള്ള സ്വര്‍ഗത്തോപ്പുകളില്‍ അവര്‍ പ്രവേശിക്കുന്നതാണ്‌. സ്വര്‍ണം കൊണ്ടുള്ള ചില വളകളും മുത്തും അവര്‍ക്ക് അവിടെ അണിയിക്കപ്പെടും. അവിടെ അവരുടെ വസ്ത്രം പട്ടായിരിക്കും.
\end{malayalam}}
\flushright{\begin{Arabic}
\quranayah[35][34]
\end{Arabic}}
\flushleft{\begin{malayalam}
അവര്‍ പറയും: ഞങ്ങളില്‍ നിന്നും ദുഃഖം നീക്കം ചെയ്ത അല്ലാഹുവിന് സ്തുതി. തീര്‍ച്ചയായും ഞങ്ങളുടെ രക്ഷിതാവ് ഏറെ പൊറുക്കുന്നവനും നന്ദിയുള്ളവനുമത്രെ.
\end{malayalam}}
\flushright{\begin{Arabic}
\quranayah[35][35]
\end{Arabic}}
\flushleft{\begin{malayalam}
തന്‍റെ അനുഗ്രഹത്താല്‍ സ്ഥിരവാസത്തിനുള്ള ഈ ഭവനത്തില്‍ ഞങ്ങളെ കുടിയിരുത്തിയവനാകുന്നു അവന്‍. യാതൊരു ബുദ്ധിമുട്ടും ഇവിടെ ഞങ്ങളെ ബാധിക്കുകയില്ല. യാതൊരു ക്ഷീണവും ഇവിടെ ഞങ്ങളെ സ്പര്‍ശിക്കുകയില്ല.
\end{malayalam}}
\flushright{\begin{Arabic}
\quranayah[35][36]
\end{Arabic}}
\flushleft{\begin{malayalam}
അവിശ്വസിച്ചവരാരോ അവര്‍ക്കാണ് നരകാഗ്നി. അവരുടെ മേല്‍ (മരണം) വിധിക്കപ്പെടുന്നതല്ല. എങ്കില്‍ അവര്‍ക്ക് മരിക്കാമായിരുന്നു. അതിലെ ശിക്ഷയില്‍ നിന്ന് ഒട്ടും അവര്‍ക്ക് ഇളവുചെയ്യപ്പെടുകയുമില്ല. അപ്രകാരം എല്ലാ നന്ദികെട്ടവര്‍ക്കും നാം പ്രതിഫലം നല്‍കുന്നു.
\end{malayalam}}
\flushright{\begin{Arabic}
\quranayah[35][37]
\end{Arabic}}
\flushleft{\begin{malayalam}
അവര്‍ അവിടെ വെച്ച് മുറവിളികൂട്ടും: ഞങ്ങളുടെ രക്ഷിതാവേ, ഞങ്ങളെ നീ പുറത്തയക്കണമേ. (മുമ്പ്‌) ചെയ്തിരുന്നതില്‍ നിന്ന് വ്യത്യസ്തമായി ഞങ്ങള്‍ സല്‍കര്‍മ്മം ചെയ്തുകൊള്ളാം. (അപ്പോള്‍ നാം പറയും:) ആലോചിക്കുന്നവന് ആലോചിക്കാന്‍ മാത്രം നിങ്ങള്‍ക്ക് നാം ആയുസ്സ് തന്നില്ലേ? താക്കീതുകാരന്‍ നിങ്ങളുടെ അടുത്ത് വരികയും ചെയ്തു. അതിനാല്‍ നിങ്ങള്‍ അനുഭവിച്ചു കൊള്ളുക. അക്രമികള്‍ക്ക് യാതൊരു സഹായിയുമില്ല.
\end{malayalam}}
\flushright{\begin{Arabic}
\quranayah[35][38]
\end{Arabic}}
\flushleft{\begin{malayalam}
തീര്‍ച്ചയായും അല്ലാഹു ആകാശങ്ങളിലെയും ഭൂമിയിലെയും അദൃശ്യകാര്യങ്ങള്‍ അറിയുന്നവനാകുന്നു. തീര്‍ച്ചയായും അവന്‍ ഹൃദയങ്ങളിലുള്ളത് അറിയുന്നവനാകുന്നു.
\end{malayalam}}
\flushright{\begin{Arabic}
\quranayah[35][39]
\end{Arabic}}
\flushleft{\begin{malayalam}
അവനാണ് നിങ്ങളെ ഭൂമിയില്‍ പ്രതിനിധികളാക്കിയവന്‍. ആകയാല്‍ വല്ലവനും അവിശ്വസിക്കുന്ന പക്ഷം അവന്‍റെ അവിശ്വാസത്തിന്‍റെ ദോഷം അവന്ന് തന്നെ. അവിശ്വാസികള്‍ക്ക് അവരുടെ അവിശ്വാസം അവരുടെ രക്ഷിതാവിങ്കല്‍ കോപമല്ലാതൊന്നും വര്‍ദ്ധിപ്പിക്കുകയില്ല. അവിശ്വാസികള്‍ക്ക് അവരുടെ അവിശ്വാസം നഷ്ടമല്ലാതൊന്നും വര്‍ദ്ധിപ്പിക്കുകയില്ല.
\end{malayalam}}
\flushright{\begin{Arabic}
\quranayah[35][40]
\end{Arabic}}
\flushleft{\begin{malayalam}
നീ പറയുക: അല്ലാഹുവിന് പുറമെ നിങ്ങള്‍ വിളിച്ചു പ്രാര്‍ത്ഥിക്കുന്ന നിങ്ങളുടെ പങ്കാളികളെ പറ്റി നിങ്ങള്‍ ചിന്തിച്ചു നോക്കിയിട്ടുണ്ടോ? ഭൂമിയില്‍ എന്തൊന്നാണവര്‍ സൃഷ്ടിച്ചിട്ടുള്ളതെന്ന് നിങ്ങള്‍ എനിക്ക് കാണിച്ചുതരിക. അതല്ല, ആകാശങ്ങളില്‍ അവര്‍ക്ക് വല്ല പങ്കുമുണ്ടോ? അതല്ല, നാം അവര്‍ക്ക് വല്ല ഗ്രന്ഥവും നല്‍കിയിട്ട് അതില്‍ നിന്നുള്ള തെളിവനുസരിച്ചാണോ അവര്‍ നിലകൊള്ളുന്നത്‌? അല്ല അക്രമകാരികള്‍ അന്യോന്യം വാഗ്ദാനം ചെയ്യുന്നത് വഞ്ചന മാത്രമാകുന്നു.
\end{malayalam}}
\flushright{\begin{Arabic}
\quranayah[35][41]
\end{Arabic}}
\flushleft{\begin{malayalam}
തീര്‍ച്ചയായും അല്ലാഹു ആകാശങ്ങളെയും ഭൂമിയെയും (യഥാര്‍ത്ഥ സ്ഥാനങ്ങളില്‍ നിന്ന്‌) നീങ്ങാതെ പിടിച്ചു നിര്‍ത്തുന്നു. അവ നീങ്ങിപ്പോകുകയാണെങ്കില്‍ അവനു പുറമെ യാതൊരാള്‍ക്കും അവയെ പിടിച്ചു നിര്‍ത്താനാവില്ല. തീര്‍ച്ചയായും അവന്‍ സഹനശീലനും ഏറെ പൊറുക്കുന്നവനുമാകുന്നു.
\end{malayalam}}
\flushright{\begin{Arabic}
\quranayah[35][42]
\end{Arabic}}
\flushleft{\begin{malayalam}
തങ്ങളുടെ അടുത്ത് ഒരു താക്കീതുകാരന്‍ വരുന്ന പക്ഷം തങ്ങള്‍ ഏതൊരു സമുദായത്തെക്കാളും സന്‍മാര്‍ഗം സ്വീകരിക്കുന്നവരാകാമെന്ന് അവരെക്കൊണ്ട് സത്യം ചെയ്യാന്‍ കഴിയുന്നതിന്‍റെ പരമാവധി അവര്‍ അല്ലാഹുവിന്‍റെ പേരില്‍ സത്യം ചെയ്ത് പറഞ്ഞു. എന്നാല്‍ ഒരു താക്കീതുകാരന്‍ അവരുടെ അടുത്ത് വന്നപ്പോള്‍ അത് അവര്‍ക്ക് അകല്‍ച്ച മാത്രമേ വര്‍ദ്ധിപ്പിച്ചുള്ളൂ.
\end{malayalam}}
\flushright{\begin{Arabic}
\quranayah[35][43]
\end{Arabic}}
\flushleft{\begin{malayalam}
ഭൂമിയില്‍ അവര്‍ അഹങ്കരിച്ചു നടക്കുകയും, ദുഷിച്ച തന്ത്രം കൈക്കൊള്ളുകയും ചെയ്യുന്നതിനാലാണ് അത്‌. ദുഷിച്ച തന്ത്രം (അതിന്‍റെ ഫലം) അത് പ്രയോഗിച്ചവരില്‍ തന്നെയാണ് വന്നുഭവിക്കുക. അപ്പോള്‍ പൂര്‍വ്വികന്‍മാരുടെ കാര്യത്തില്‍ ഉണ്ടായ നടപടിക്രമമല്ലാതെ എന്താണവര്‍ കാത്തിരിക്കുന്നത്‌? അല്ലാഹുവിന്‍റെ നടപടിക്രമത്തിന് യാതൊരു ഭേദഗതിയും നീ കണ്ടെത്തുകയില്ല. അല്ലാഹുവിന്‍റെ നടപടിക്രമത്തിന് യാതൊരു മാറ്റവും നീ കണ്ടെത്തുകയില്ല.
\end{malayalam}}
\flushright{\begin{Arabic}
\quranayah[35][44]
\end{Arabic}}
\flushleft{\begin{malayalam}
അവര്‍ ഭൂമിയിലൂടെ സഞ്ചരിച്ചിട്ട് തങ്ങളുടെ മുന്‍ഗാമികളുടെ പര്യവസാനം എങ്ങനെയായിരുന്നു എന്ന് നോക്കിയില്ലേ? അവര്‍ ഇവരെക്കാള്‍ മികച്ച ശക്തിയുള്ളവരായിരുന്നു. ആകാശങ്ങളിലും ഭൂമിയിലുമുള്ള യാതൊന്നിനും അല്ലാഹുവെ തോല്‍പിക്കാനാവില്ല. തീര്‍ച്ചയായും അവന്‍ സര്‍വ്വജ്ഞനും സര്‍വ്വശക്തനുമാകുന്നു.
\end{malayalam}}
\flushright{\begin{Arabic}
\quranayah[35][45]
\end{Arabic}}
\flushleft{\begin{malayalam}
അല്ലാഹു മനുഷ്യരെ അവര്‍ പ്രവര്‍ത്തിച്ചതിന്‍റെ പേരില്‍ (ഉടനെതന്നെ) പിടിച്ച് ശിക്ഷിക്കുകയായിരുന്നുവെങ്കില്‍ ഭൂമുഖത്ത് ഒരു ജന്തുവെയും അവന്‍ വിട്ടേക്കുകയില്ലായിരുന്നു. എന്നാല്‍ ഒരു നിശ്ചിത അവധിവരെ അവരെ അവന്‍ നീട്ടിയിടുന്നു. അങ്ങനെ അവരുടെ അവധി വന്നെത്തിയാല്‍ (അവര്‍ക്ക് രക്ഷപ്പെടാനാവില്ല.) കാരണം, തീര്‍ച്ചയായും അല്ലാഹു തന്‍റെ ദാസന്‍മാരെപ്പറ്റി കണ്ടറിയുന്നവനാകുന്നു.
\end{malayalam}}
\chapter{\textmalayalam{യാസീന്‍}}
\begin{Arabic}
\Huge{\centerline{\basmalah}}\end{Arabic}
\flushright{\begin{Arabic}
\quranayah[36][1]
\end{Arabic}}
\flushleft{\begin{malayalam}
യാസീന്‍ ‍.
\end{malayalam}}
\flushright{\begin{Arabic}
\quranayah[36][2]
\end{Arabic}}
\flushleft{\begin{malayalam}
തത്വസമ്പൂര്‍ണമായ ഖുര്‍ആന്‍ തന്നെയാണ സത്യം;
\end{malayalam}}
\flushright{\begin{Arabic}
\quranayah[36][3]
\end{Arabic}}
\flushleft{\begin{malayalam}
നീ ദൈവദൂതന്‍മാരില്‍ പെട്ടവന്‍ തന്നെയാകുന്നു.
\end{malayalam}}
\flushright{\begin{Arabic}
\quranayah[36][4]
\end{Arabic}}
\flushleft{\begin{malayalam}
നേരായ പാതയിലാകുന്നു (നീ.)
\end{malayalam}}
\flushright{\begin{Arabic}
\quranayah[36][5]
\end{Arabic}}
\flushleft{\begin{malayalam}
പ്രതാപിയും കരുണാനിധിയുമായിട്ടുള്ളവന്‍ അവതരിപ്പിച്ചതത്രെ ഇത്‌. (ഖുര്‍ആന്‍).
\end{malayalam}}
\flushright{\begin{Arabic}
\quranayah[36][6]
\end{Arabic}}
\flushleft{\begin{malayalam}
ഒരു ജനതയ്ക്ക് നീ താക്കീത് നല്‍കുവാന്‍ വേണ്ടി. അവരുടെ പിതാക്കന്‍മാര്‍ക്ക് താക്കീത് നല്‍കപ്പെട്ടിട്ടില്ല. അതിനാല്‍ അവര്‍ അശ്രദ്ധയില്‍ കഴിയുന്നവരാകുന്നു.
\end{malayalam}}
\flushright{\begin{Arabic}
\quranayah[36][7]
\end{Arabic}}
\flushleft{\begin{malayalam}
അവരില്‍ മിക്കവരുടെ കാര്യത്തിലും (ശിക്ഷയെ സംബന്ധിച്ച) വചനം സത്യമായി പുലര്‍ന്നിരിക്കുന്നു. അതിനാല്‍ അവര്‍ വിശ്വസിക്കുകയില്ല.
\end{malayalam}}
\flushright{\begin{Arabic}
\quranayah[36][8]
\end{Arabic}}
\flushleft{\begin{malayalam}
അവരുടെ കഴുത്തുകളില്‍ നാം ചങ്ങലകള്‍ വെച്ചിരിക്കുന്നു. അത് (അവരുടെ) താടിയെല്ലുകള്‍ വരെ എത്തുന്നു. തന്‍മൂലം അവര്‍ തലകുത്തനെ പിടിച്ചവരായിരിക്കും.
\end{malayalam}}
\flushright{\begin{Arabic}
\quranayah[36][9]
\end{Arabic}}
\flushleft{\begin{malayalam}
അവരുടെ മുമ്പില്‍ ഒരു തടസ്സവും അവരുടെ പിന്നില്‍ ഒരു തടസ്സവും നാം വെച്ചിരിക്കുന്നു. അങ്ങനെ നാം അവരെ മൂടിക്കളഞ്ഞു; അതിനാല്‍ അവര്‍ക്ക് കാണാന്‍ കഴിയില്ല.
\end{malayalam}}
\flushright{\begin{Arabic}
\quranayah[36][10]
\end{Arabic}}
\flushleft{\begin{malayalam}
നീ അവര്‍ക്ക് താക്കീത് നല്‍കിയോ അതല്ല താക്കീത് നല്‍കിയില്ലേ എന്നത് അവരെ സംബന്ധിച്ചിടത്തോളം സമമാകുന്നു. അവര്‍ വിശ്വസിക്കുകയില്ല.
\end{malayalam}}
\flushright{\begin{Arabic}
\quranayah[36][11]
\end{Arabic}}
\flushleft{\begin{malayalam}
ബോധനം പിന്‍പറ്റുകയും, അദൃശ്യാവസ്ഥയില്‍ പരമകാരുണികനെ ഭയപ്പെടുകയും ചെയ്തവനു മാത്രമേ നിന്‍റെ താക്കീത് ഫലപ്പെടുകയുള്ളൂ. ആകയാല്‍ പാപമോചനത്തെയും ഉദാരമായ പ്രതിഫലത്തെയും പറ്റി അവന്ന് സന്തോഷവാര്‍ത്ത അറിയിക്കുക.
\end{malayalam}}
\flushright{\begin{Arabic}
\quranayah[36][12]
\end{Arabic}}
\flushleft{\begin{malayalam}
തീര്‍ച്ചയായും നാം തന്നെയാണ് മരിച്ചവരെ ജീവിപ്പിക്കുന്നത്‌. അവര്‍ ചെയ്തു വെച്ചതും അവരുടെ (പ്രവര്‍ത്തനങ്ങളുടെ) അനന്തരഫലങ്ങളും നാം എഴുതിവെക്കുകയും ചെയ്യുന്നു. എല്ലാകാര്യങ്ങളും വ്യക്തമായ ഒരു രേഖയില്‍ നാം നിജപ്പെടുത്തി വെച്ചിരിക്കുന്നു.
\end{malayalam}}
\flushright{\begin{Arabic}
\quranayah[36][13]
\end{Arabic}}
\flushleft{\begin{malayalam}
ആ രാജ്യക്കാരെ ഒരു ഉദാഹരണമെന്ന നിലയ്ക്ക് നീ അവര്‍ക്ക് പറഞ്ഞുകൊടുക്കുക. ദൈവദൂതന്‍മാര്‍ അവിടെ ചെന്ന സന്ദര്‍ഭം.
\end{malayalam}}
\flushright{\begin{Arabic}
\quranayah[36][14]
\end{Arabic}}
\flushleft{\begin{malayalam}
അവരിലേക്ക് രണ്ടുപേരെ നാം ദൂതന്‍മാരായി അയച്ചപ്പോള്‍ അവരെ അവര്‍ നിഷേധിച്ചുതള്ളി. അപ്പോള്‍ ഒരു മൂന്നാമനെക്കൊണ്ട് നാം അവര്‍ക്ക് പിന്‍ബലം നല്‍കി. എന്നിട്ടവര്‍ പറഞ്ഞു: തീര്‍ച്ചയായും ഞങ്ങള്‍ നിങ്ങളുടെ അടുത്തേക്ക് നിയോഗിക്കപ്പെട്ടവരാകുന്നു.
\end{malayalam}}
\flushright{\begin{Arabic}
\quranayah[36][15]
\end{Arabic}}
\flushleft{\begin{malayalam}
അവര്‍ (ജനങ്ങള്‍) പറഞ്ഞു. നിങ്ങള്‍ ഞങ്ങളെ പോലെയുള്ള മനുഷ്യര്‍ മാത്രമാകുന്നു. പരമകാരുണികന്‍ യാതൊന്നും അവതരിപ്പിച്ചിട്ടില്ല. നിങ്ങള്‍ കളവ് പറയുക തന്നെയാണ്‌.
\end{malayalam}}
\flushright{\begin{Arabic}
\quranayah[36][16]
\end{Arabic}}
\flushleft{\begin{malayalam}
അവര്‍ (ദൂതന്‍മാര്‍) പറഞ്ഞു: ഞങ്ങളുടെ രക്ഷിതാവിനറിയാം; തീര്‍ച്ചയായും ഞങ്ങള്‍ നിങ്ങളുടെ അടുക്കലേക്ക് നിയോഗിക്കപ്പെട്ടവര്‍ തന്നെയാണെന്ന്‌.
\end{malayalam}}
\flushright{\begin{Arabic}
\quranayah[36][17]
\end{Arabic}}
\flushleft{\begin{malayalam}
വ്യക്തമായ പ്രബോധനമല്ലാതെ ഞങ്ങള്‍ക്ക് യാതൊരു ബാധ്യതയുമില്ല.
\end{malayalam}}
\flushright{\begin{Arabic}
\quranayah[36][18]
\end{Arabic}}
\flushleft{\begin{malayalam}
അവര്‍ (ജനങ്ങള്‍) പറഞ്ഞു: തീര്‍ച്ചയായും ഞങ്ങള്‍ നിങ്ങളെ ഒരു ദുശ്ശകുനമായി കരുതുന്നു. നിങ്ങള്‍ (ഇതില്‍ നിന്ന്‌) വിരമിക്കാത്ത പക്ഷം നിങ്ങളെ ഞങ്ങള്‍ എറിഞ്ഞോടിക്കുക തന്നെ ചെയ്യും. ഞങ്ങളില്‍ നിന്ന് വേദനിപ്പിക്കുന്ന ശിക്ഷ നിങ്ങളെ സ്പര്‍ശിക്കുക തന്നെ ചെയ്യും.
\end{malayalam}}
\flushright{\begin{Arabic}
\quranayah[36][19]
\end{Arabic}}
\flushleft{\begin{malayalam}
അവര്‍ (ദൂതന്‍മാര്‍) പറഞ്ഞു: നിങ്ങളുടെ ശകുനപ്പിഴ നിങ്ങളുടെ കൂടെയുള്ളത് തന്നെയാകുന്നു. നിങ്ങള്‍ക്ക് ഉല്‍ബോധനം നല്‍കപ്പെട്ടാല്‍ ഇതാണോ (നിങ്ങളുടെ നിലപാട്‌?) എന്നാല്‍ നിങ്ങള്‍ ഒരു അതിരുകവിഞ്ഞ ജനത തന്നെയാകുന്നു.
\end{malayalam}}
\flushright{\begin{Arabic}
\quranayah[36][20]
\end{Arabic}}
\flushleft{\begin{malayalam}
പട്ടണത്തിന്‍റെ അങ്ങേ അറ്റത്ത് നിന്ന് ഒരാള്‍ ഓടിവന്ന് പറഞ്ഞു: എന്‍റെ ജനങ്ങളേ, നിങ്ങള്‍ ദൂതന്‍മാരെ പിന്തുടരുവിന്‍.
\end{malayalam}}
\flushright{\begin{Arabic}
\quranayah[36][21]
\end{Arabic}}
\flushleft{\begin{malayalam}
നിങ്ങളോട് യാതൊരു പ്രതിഫലവും ചോദിക്കാത്തവരും സന്‍മാര്‍ഗം പ്രാപിച്ചവരും ആയിട്ടുള്ളവരെ നിങ്ങള്‍ പിന്തുടരുക.
\end{malayalam}}
\flushright{\begin{Arabic}
\quranayah[36][22]
\end{Arabic}}
\flushleft{\begin{malayalam}
ഏതൊരുവന്‍ എന്നെ സൃഷ്ടിച്ചുവോ, ഏതൊരുവന്‍റെ അടുത്തേക്ക് നിങ്ങള്‍ മടക്കപ്പെടുന്നുവോ അവനെ ഞാന്‍ ആരാധിക്കാതിരിക്കാന്‍ എനിക്കെന്തുന്യായം?
\end{malayalam}}
\flushright{\begin{Arabic}
\quranayah[36][23]
\end{Arabic}}
\flushleft{\begin{malayalam}
അവനു പുറമെ വല്ല ദൈവങ്ങളേയും ഞാന്‍ സ്വീകരിക്കുകയോ? പരമകാരുണികന്‍ എനിക്ക് വല്ല ദോഷവും വരുത്താന്‍ ഉദ്ദേശിക്കുന്ന പക്ഷം അവരുടെ ശുപാര്‍ശ എനിക്ക് യാതൊരു പ്രയോജനവും ചെയ്യുകയില്ല. അവര്‍ എന്നെ രക്ഷപ്പെടുത്തുകയുമില്ല.
\end{malayalam}}
\flushright{\begin{Arabic}
\quranayah[36][24]
\end{Arabic}}
\flushleft{\begin{malayalam}
അങ്ങനെ ചെയ്യുന്ന പക്ഷം തീര്‍ച്ചയായും ഞാന്‍ വ്യക്തമായ ദുര്‍മാര്‍ഗത്തിലായിരിക്കും.
\end{malayalam}}
\flushright{\begin{Arabic}
\quranayah[36][25]
\end{Arabic}}
\flushleft{\begin{malayalam}
തീര്‍ച്ചയായും ഞാന്‍ നിങ്ങളുടെ രക്ഷിതാവില്‍ വിശ്വസിച്ചിരിക്കുന്നു. അത് കൊണ്ട് നിങ്ങള്‍ എന്‍റെ വാക്ക് കേള്‍ക്കുക.
\end{malayalam}}
\flushright{\begin{Arabic}
\quranayah[36][26]
\end{Arabic}}
\flushleft{\begin{malayalam}
സ്വര്‍ഗത്തില്‍ പ്രവേശിച്ച് കൊള്ളുക. എന്ന് പറയപ്പെട്ടു. അദ്ദേഹം പറഞ്ഞു: എന്‍റെ ജനത അറിഞ്ഞിരുന്നെങ്കില്‍ എത്ര നന്നായിരുന്നു!
\end{malayalam}}
\flushright{\begin{Arabic}
\quranayah[36][27]
\end{Arabic}}
\flushleft{\begin{malayalam}
എന്‍റെ രക്ഷിതാവ് എനിക്ക് പൊറുത്തുതരികയും ആദരിക്കപ്പെട്ടവരുടെ കൂട്ടത്തില്‍ എന്നെ ഉള്‍പെടുത്തുകയും ചെയ്തതിനെ പറ്റി.
\end{malayalam}}
\flushright{\begin{Arabic}
\quranayah[36][28]
\end{Arabic}}
\flushleft{\begin{malayalam}
അദ്ദേഹത്തിനു ശേഷം അദ്ദേഹത്തിന്‍റെ ജനതയുടെ നേരെ ആകാശത്ത് നിന്ന് സൈനിക സംഘത്തെയൊന്നും നാം ഇറക്കിയിട്ടില്ല. നാം അങ്ങനെ ഇറക്കാറുണ്ടായിരുന്നുമില്ല.
\end{malayalam}}
\flushright{\begin{Arabic}
\quranayah[36][29]
\end{Arabic}}
\flushleft{\begin{malayalam}
അത് ഒരൊറ്റ ശബ്ദം മാത്രമായിരുന്നു. അപ്പോഴേക്കും അവരതാ കെട്ടടങ്ങിക്കഴിഞ്ഞു.
\end{malayalam}}
\flushright{\begin{Arabic}
\quranayah[36][30]
\end{Arabic}}
\flushleft{\begin{malayalam}
ആ ദാസന്‍മാരുടെ കാര്യം എത്ര പരിതാപകരം. ഏതൊരു ദൂതന്‍ അവരുടെ അടുത്ത് ചെല്ലുമ്പോഴും അവര്‍ അദ്ദേഹത്തെ പരിഹാസിക്കാതിരുന്നിട്ടില്ല.
\end{malayalam}}
\flushright{\begin{Arabic}
\quranayah[36][31]
\end{Arabic}}
\flushleft{\begin{malayalam}
അവര്‍ക്കു മുമ്പ് എത്രയെത്ര തലമുറകളെ നാം നശിപ്പിച്ചു! അവരാരും ഇവരുടെ അടുത്തേക്ക് തിരിച്ചുവരുന്നില്ല എന്ന് അവര്‍ കണ്ടില്ലേ?
\end{malayalam}}
\flushright{\begin{Arabic}
\quranayah[36][32]
\end{Arabic}}
\flushleft{\begin{malayalam}
തീര്‍ച്ചയായും അവരെല്ലാവരും ഒന്നൊഴിയാതെ നമ്മുടെ മുമ്പില്‍ ഹാജരാക്കപ്പെടുന്നവരാകുന്നു.
\end{malayalam}}
\flushright{\begin{Arabic}
\quranayah[36][33]
\end{Arabic}}
\flushleft{\begin{malayalam}
അവര്‍ക്കൊരു ദൃഷ്ടാന്തമുണ്ട്‌; നിര്‍ജീവമായ ഭൂമി. അതിന് നാം ജീവന്‍ നല്‍കുകയും, അതില്‍ നിന്ന് നാം ധാന്യം ഉല്‍പാദിപ്പിക്കുകയും ചെയ്തു. എന്നിട്ട് അതില്‍ നിന്നാണ് അവര്‍ ഭക്ഷിക്കുന്നത്‌.
\end{malayalam}}
\flushright{\begin{Arabic}
\quranayah[36][34]
\end{Arabic}}
\flushleft{\begin{malayalam}
ഈന്തപ്പനയുടെയും മുന്തിരിയുടെയും തോട്ടങ്ങള്‍ അതില്‍ നാം ഉണ്ടാക്കുകയും, അതില്‍ നാം ഉറവിടങ്ങള്‍ ഒഴുക്കുകയും ചെയ്തു.
\end{malayalam}}
\flushright{\begin{Arabic}
\quranayah[36][35]
\end{Arabic}}
\flushleft{\begin{malayalam}
അതിന്‍റെ ഫലങ്ങളില്‍ നിന്നും അവരുടെ കൈകള്‍ അദ്ധ്വാനിച്ചുണ്ടാക്കിയതില്‍ നിന്നും അവര്‍ ഭക്ഷിക്കുവാന്‍ വേണ്ടി. എന്നിരിക്കെ അവര്‍ നന്ദികാണിക്കുന്നില്ലേ?
\end{malayalam}}
\flushright{\begin{Arabic}
\quranayah[36][36]
\end{Arabic}}
\flushleft{\begin{malayalam}
ഭൂമി മുളപ്പിക്കുന്ന സസ്യങ്ങളിലും, അവരുടെ സ്വന്തം വര്‍ഗങ്ങളിലും, അവര്‍ക്കറിയാത്ത വസ്തുക്കളിലും പെട്ട എല്ലാ ഇണകളെയും സൃഷ്ടിച്ചവന്‍ എത്ര പരിശുദ്ധന്‍!
\end{malayalam}}
\flushright{\begin{Arabic}
\quranayah[36][37]
\end{Arabic}}
\flushleft{\begin{malayalam}
രാത്രിയും അവര്‍ക്കൊരു ദൃഷ്ടാന്തമത്രെ . അതില്‍ നിന്ന് പകലിനെ നാം ഊരിയെടുക്കുന്നു. അപ്പോള്‍ അവരതാ ഇരുട്ടില്‍ അകപ്പെടുന്നു.
\end{malayalam}}
\flushright{\begin{Arabic}
\quranayah[36][38]
\end{Arabic}}
\flushleft{\begin{malayalam}
സൂര്യന്‍ അതിന് സ്ഥിരമായുള്ള ഒരു സ്ഥാനത്തേക്ക് സഞ്ചരിക്കുന്നു. പ്രതാപിയും സര്‍വ്വജ്ഞനുമായ അല്ലാഹു കണക്കാക്കിയതാണത്‌.
\end{malayalam}}
\flushright{\begin{Arabic}
\quranayah[36][39]
\end{Arabic}}
\flushleft{\begin{malayalam}
ചന്ദ്രന് നാം ചില ഘട്ടങ്ങള്‍ നിശ്ചയിച്ചിരിക്കുന്നു. അങ്ങനെ അത് പഴയ ഈന്തപ്പഴക്കുലയുടെ വളഞ്ഞ തണ്ടുപോലെ ആയിത്തീരുന്നു.
\end{malayalam}}
\flushright{\begin{Arabic}
\quranayah[36][40]
\end{Arabic}}
\flushleft{\begin{malayalam}
സൂര്യന് ചന്ദ്രനെ പ്രാപിക്കാനൊക്കുകയില്ല. രാവ് പകലിനെ മറികടക്കുന്നതുമല്ല. ഓരോന്നും ഓരോ (നിശ്ചിത) ഭ്രമണപഥത്തില്‍ നീന്തികൊണ്ടിരിക്കുന്നു.
\end{malayalam}}
\flushright{\begin{Arabic}
\quranayah[36][41]
\end{Arabic}}
\flushleft{\begin{malayalam}
അവരുടെ സന്തതികളെ ഭാരം നിറച്ച കപ്പലില്‍ നാം കയറ്റികൊണ്ട് പോയതും അവര്‍ക്കൊരു ദൃഷ്ടാന്തമാകുന്നു.
\end{malayalam}}
\flushright{\begin{Arabic}
\quranayah[36][42]
\end{Arabic}}
\flushleft{\begin{malayalam}
അതുപോലെ അവര്‍ക്ക് വാഹനമായി ഉപയോഗിക്കാവുന്ന മറ്റു വസ്തുക്കളും അവര്‍ക്ക് വേണ്ടി നാം സൃഷ്ടിച്ചിട്ടുണ്ട്‌.
\end{malayalam}}
\flushright{\begin{Arabic}
\quranayah[36][43]
\end{Arabic}}
\flushleft{\begin{malayalam}
നാം ഉദ്ദേശിക്കുന്ന പക്ഷം നാം അവരെ മുക്കിക്കളയുന്നതാണ്‌.അപ്പോള്‍ അവര്‍ക്കൊരു സഹായിയും ഉണ്ടായിരിക്കുന്നതല്ല. അവര്‍ രക്ഷിക്കപ്പെടുന്നതുമല്ല.
\end{malayalam}}
\flushright{\begin{Arabic}
\quranayah[36][44]
\end{Arabic}}
\flushleft{\begin{malayalam}
നമ്മുടെ പക്കല്‍ നിന്നുള്ള കാരുണ്യവും, ഒരു നിശ്ചിത കാലം വരെയുള്ള സുഖാനുഭവവും ആയിക്കൊണ്ട് (നാം അവര്‍ക്ക് നല്‍കുന്നത്‌.) അല്ലാതെ.
\end{malayalam}}
\flushright{\begin{Arabic}
\quranayah[36][45]
\end{Arabic}}
\flushleft{\begin{malayalam}
നിങ്ങളുടെ മുമ്പില്‍ വരാനിരിക്കുന്നതും, നിങ്ങളുടെ പിന്നില്‍ കഴിഞ്ഞതുമായ ശിക്ഷയെ നിങ്ങള്‍ സൂക്ഷിക്കുക. നിങ്ങള്‍ക്ക് കാരുണ്യം ലഭിച്ചേക്കാം എന്ന് അവരോട് പറയപ്പെട്ടാല്‍ (അവരത് അവഗണിക്കുന്നു.)
\end{malayalam}}
\flushright{\begin{Arabic}
\quranayah[36][46]
\end{Arabic}}
\flushleft{\begin{malayalam}
അവരുടെ രക്ഷിതാവിന്‍റെ ദൃഷ്ടാന്തങ്ങളില്‍ പെട്ട ഏതൊരു ദൃഷ്ടാന്തം അവര്‍ക്ക് വന്നെത്തിയാലും അവര്‍ അതില്‍ നിന്ന് തിരിഞ്ഞുകളയാതിരിക്കുന്നില്ല.
\end{malayalam}}
\flushright{\begin{Arabic}
\quranayah[36][47]
\end{Arabic}}
\flushleft{\begin{malayalam}
നിങ്ങള്‍ക്ക് അല്ലാഹു നല്‍കിയതില്‍ നിന്ന് നിങ്ങള്‍ ചെലവഴിക്കൂ എന്ന് അവരോട് പറയപ്പെട്ടാല്‍ അവിശ്വാസികള്‍ വിശ്വാസികളോട് പറയും: അല്ലാഹു ഉദ്ദേശിച്ചിരുന്നുവെങ്കില്‍ അവന്‍ തന്നെ ഭക്ഷണം നല്‍കുമായിരുന്ന ആളുകള്‍ക്ക് ഞങ്ങള്‍ ഭക്ഷണം നല്‍കുകയോ? നിങ്ങള്‍ വ്യക്തമായ വഴികേടില്‍ തന്നെയാകുന്നു.
\end{malayalam}}
\flushright{\begin{Arabic}
\quranayah[36][48]
\end{Arabic}}
\flushleft{\begin{malayalam}
അവര്‍ ചോദിക്കുന്നു. നിങ്ങള്‍ സത്യവാന്‍മാരാണെങ്കില്‍ ഈ വാഗ്ദത്തം എപ്പോഴാണ് പുലരുക?
\end{malayalam}}
\flushright{\begin{Arabic}
\quranayah[36][49]
\end{Arabic}}
\flushleft{\begin{malayalam}
ഒരൊറ്റ ഘോരശബ്ദം മാത്രമാണ് അവര്‍ കാത്തിരിക്കുന്നത്‌. അവര്‍ അന്യോന്യം തര്‍ക്കിച്ച് കൊണ്ടിരിക്കെ അതവരെ പിടികൂടും.
\end{malayalam}}
\flushright{\begin{Arabic}
\quranayah[36][50]
\end{Arabic}}
\flushleft{\begin{malayalam}
അപ്പോള്‍ യാതൊരു വസ്വിയ്യത്തും നല്‍കാന്‍ അവര്‍ക്ക് സാധിക്കുകയില്ല. അവര്‍ക്ക് അവരുടെ കുടുംബത്തിലേക്ക് മടങ്ങാനും ആകുകയില്ല.
\end{malayalam}}
\flushright{\begin{Arabic}
\quranayah[36][51]
\end{Arabic}}
\flushleft{\begin{malayalam}
കാഹളത്തില്‍ ഊതപ്പെടും. അപ്പോള്‍ അവര്‍ ഖബ്‌റുകളില്‍ നിന്ന് അവരുടെ രക്ഷിതാവിങ്കലേക്ക് കുതിച്ച് ചെല്ലും.
\end{malayalam}}
\flushright{\begin{Arabic}
\quranayah[36][52]
\end{Arabic}}
\flushleft{\begin{malayalam}
അവര്‍ പറയും: നമ്മുടെ നാശമേ! നമ്മുടെ ഉറക്കത്തില്‍ നിന്ന് നമ്മെ എഴുന്നേല്‍പിച്ചതാരാണ്‌? ഇത് പരമകാരുണികന്‍ വാഗ്ദാനം ചെയ്തതാണല്ലോ. ദൈവദൂതന്‍മാര്‍ സത്യം തന്നെയാണ് പറഞ്ഞത്‌.
\end{malayalam}}
\flushright{\begin{Arabic}
\quranayah[36][53]
\end{Arabic}}
\flushleft{\begin{malayalam}
അത് ഒരൊറ്റ ഘോരശബ്ദം മാത്രമായിരിക്കും. അപ്പോഴതാ അവര്‍ ഒന്നടങ്കം നമ്മുടെ അടുക്കല്‍ ഹാജരാക്കപ്പെടുന്നു.
\end{malayalam}}
\flushright{\begin{Arabic}
\quranayah[36][54]
\end{Arabic}}
\flushleft{\begin{malayalam}
അന്നേ ദിവസം യാതൊരാളോടും അനീതി ചെയ്യപ്പെടുകയില്ല. നിങ്ങള്‍ പ്രവര്‍ത്തിച്ചു കൊണ്ടിരുന്നതിനല്ലാതെ നിങ്ങള്‍ക്ക് പ്രതിഫലം നല്‍കപ്പെടുകയുമില്ല.
\end{malayalam}}
\flushright{\begin{Arabic}
\quranayah[36][55]
\end{Arabic}}
\flushleft{\begin{malayalam}
തീര്‍ച്ചയായും സ്വര്‍ഗവാസികള്‍ അന്ന് ഓരോ ജോലിയിലായിക്കൊണ്ട് സുഖമനുഭവിക്കുന്നവരായിരിക്കും.
\end{malayalam}}
\flushright{\begin{Arabic}
\quranayah[36][56]
\end{Arabic}}
\flushleft{\begin{malayalam}
അവരും അവരുടെ ഇണകളും തണലുകളില്‍ അലംകൃതമായ കട്ടിലുകളില്‍ ചാരിയിരിക്കുന്നവരായിരിക്കും.
\end{malayalam}}
\flushright{\begin{Arabic}
\quranayah[36][57]
\end{Arabic}}
\flushleft{\begin{malayalam}
അവര്‍ക്കവിടെ പഴവര്‍ഗങ്ങളുണ്ട്‌, അവര്‍ക്ക് തങ്ങള്‍ ആവശ്യപ്പെടുന്നതല്ലാമുണ്ട്‌.
\end{malayalam}}
\flushright{\begin{Arabic}
\quranayah[36][58]
\end{Arabic}}
\flushleft{\begin{malayalam}
സമാധാനം! അതായിരിക്കും കരുണാനിധിയായ രക്ഷിതാവിങ്കല്‍ നിന്ന് അവര്‍ക്കുള്ള അഭിവാദ്യം.
\end{malayalam}}
\flushright{\begin{Arabic}
\quranayah[36][59]
\end{Arabic}}
\flushleft{\begin{malayalam}
കുറ്റവാളികളേ, ഇന്ന് നിങ്ങള്‍ വേറിട്ട് നില്‍ക്കുക (എന്ന് അവിടെ വെച്ച് പ്രഖ്യാപിക്കപ്പെടും.)
\end{malayalam}}
\flushright{\begin{Arabic}
\quranayah[36][60]
\end{Arabic}}
\flushleft{\begin{malayalam}
ആദം സന്തതികളേ, ഞാന്‍ നിങ്ങളോട് അനുശാസിച്ചിട്ടില്ലേ നിങ്ങള്‍ പിശാചിനെ ആരാധിക്കരുത്‌. തീര്‍ച്ചയായും അവന്‍ നിങ്ങള്‍ക്ക് പ്രത്യക്ഷശത്രുവാകുന്നു.
\end{malayalam}}
\flushright{\begin{Arabic}
\quranayah[36][61]
\end{Arabic}}
\flushleft{\begin{malayalam}
നിങ്ങള്‍ എന്നെ ആരാധിക്കുവിന്‍. ഇതാണ് നേരായ മാര്‍ഗം എന്ന്‌.
\end{malayalam}}
\flushright{\begin{Arabic}
\quranayah[36][62]
\end{Arabic}}
\flushleft{\begin{malayalam}
തീര്‍ച്ചയായും നിങ്ങളുടെ കൂട്ടത്തില്‍ നിന്ന് അനേകം സംഘങ്ങളെ അവന്‍ (പിശാച്‌) പിഴപ്പിച്ചിട്ടുണ്ട്‌. എന്നിട്ടും നിങ്ങള്‍ ചിന്തിച്ച് മനസ്സിലാക്കുന്നവരായില്ലേ?
\end{malayalam}}
\flushright{\begin{Arabic}
\quranayah[36][63]
\end{Arabic}}
\flushleft{\begin{malayalam}
ഇതാ, നിങ്ങള്‍ക്ക് മുന്നറിയിപ്പ് നല്‍കപ്പെട്ടിരുന്ന നരകം!
\end{malayalam}}
\flushright{\begin{Arabic}
\quranayah[36][64]
\end{Arabic}}
\flushleft{\begin{malayalam}
നിങ്ങള്‍ അവിശ്വസിച്ചിരുന്നതിന്‍റെ ഫലമായി അതില്‍ കടന്നു എരിഞ്ഞ് കൊള്ളുക.
\end{malayalam}}
\flushright{\begin{Arabic}
\quranayah[36][65]
\end{Arabic}}
\flushleft{\begin{malayalam}
അന്ന് നാം അവരുടെ വായകള്‍ക്കു മുദ്രവെക്കുന്നതും, അവരുടെ കൈകള്‍ നമ്മോട് സംസാരിക്കുന്നതും , അവര്‍ പ്രവര്‍ത്തിച്ചിരുന്നതിനെപ്പറ്റി അവരുടെ കാലുകള്‍ സാക്ഷ്യം വഹിക്കുന്നതുമാണ്‌.
\end{malayalam}}
\flushright{\begin{Arabic}
\quranayah[36][66]
\end{Arabic}}
\flushleft{\begin{malayalam}
നാം ഉദ്ദേശിച്ചിരുന്നെങ്കില്‍ അവരുടെ കണ്ണുകളെ നാം തുടച്ചുനീക്കുമായിരുന്നു. എന്നിട്ടും പാതയിലൂടെ മുന്നോട്ട് നീങ്ങാന്‍ അവര്‍ ശ്രമിച്ചേനെ. എന്നാല്‍ അവര്‍ക്കെങ്ങനെ കാണാന്‍ കഴിയും?
\end{malayalam}}
\flushright{\begin{Arabic}
\quranayah[36][67]
\end{Arabic}}
\flushleft{\begin{malayalam}
നാം ഉദ്ദേശിച്ചിരുന്നെങ്കില്‍ അവര്‍ നില്‍ക്കുന്നേടത്ത് വെച്ച് തന്നെ അവര്‍ക്ക് നാം രൂപഭേദം വരുത്തുമായിരുന്നു. അപ്പോള്‍ അവര്‍ക്ക് മുന്നോട്ട് നീങ്ങാന്‍ സാധിക്കുകയില്ല. അവര്‍ക്ക് തിരിച്ചുപോവാനുമാവില്ല.
\end{malayalam}}
\flushright{\begin{Arabic}
\quranayah[36][68]
\end{Arabic}}
\flushleft{\begin{malayalam}
വല്ലവന്നും നാം ദീര്‍ഘായുസ്സ് നല്‍കുന്നുവെങ്കില്‍ അവന്‍റെ പ്രകൃതി നാം തലതിരിച്ചു കൊണ്ടുവരുന്നു. എന്നിരിക്കെ അവര്‍ ചിന്തിക്കുന്നില്ലേ?
\end{malayalam}}
\flushright{\begin{Arabic}
\quranayah[36][69]
\end{Arabic}}
\flushleft{\begin{malayalam}
അദ്ദേഹത്തിന് (നബിക്ക്‌) നാം കവിത പഠിപ്പിച്ചിട്ടില്ല. അത് അദ്ദേഹത്തിന് അനുയോജ്യമാകുകയുമില്ല. ഇത് ഒരു ഉല്‍ബോധനവും കാര്യങ്ങള്‍ സ്പഷ്ടമാക്കുന്ന ഖുര്‍ആനും മാത്രമാകുന്നു.
\end{malayalam}}
\flushright{\begin{Arabic}
\quranayah[36][70]
\end{Arabic}}
\flushleft{\begin{malayalam}
ജീവനുള്ളവര്‍ക്ക് താക്കീത് നല്‍കുന്നതിന് വേണ്ടിയത്രെ ഇത്‌. സത്യനിഷേധികളുടെ കാര്യത്തില്‍ (ശിക്ഷയുടെ) വചനം സത്യമായിപുലരുവാന്‍ വേണ്ടിയും.
\end{malayalam}}
\flushright{\begin{Arabic}
\quranayah[36][71]
\end{Arabic}}
\flushleft{\begin{malayalam}
നമ്മുടെ കൈകള്‍ നിര്‍മിച്ചതില്‍പ്പെട്ട കാലികളെ അവര്‍ക്ക് വേണ്ടിയാണ് നാം സൃഷ്ടിച്ചിരിക്കുന്നത് എന്ന് അവര്‍ കണ്ടില്ലേ? അങ്ങനെ അവര്‍ അവയുടെ ഉടമസ്ഥരായിരിക്കുന്നു.
\end{malayalam}}
\flushright{\begin{Arabic}
\quranayah[36][72]
\end{Arabic}}
\flushleft{\begin{malayalam}
അവയെ അവര്‍ക്ക് വേണ്ടി നാം കീഴ്പെടുത്തികൊടുക്കുകയും ചെയ്തിരിക്കുന്നു. അങ്ങനെ അവയില്‍ നിന്നാകുന്നു അവര്‍ക്കുള്ള വാഹനം. അവയില്‍ നിന്ന് അവര്‍ (മാംസം) ഭക്ഷിക്കുകയും ചെയ്യുന്നു.
\end{malayalam}}
\flushright{\begin{Arabic}
\quranayah[36][73]
\end{Arabic}}
\flushleft{\begin{malayalam}
അവര്‍ക്ക് അവയില്‍ പല പ്രയോജനങ്ങളുമുണ്ട്‌. (പുറമെ) പാനീയങ്ങളും. എന്നിരിക്കെ അവര്‍ നന്ദികാണിക്കുന്നില്ലേ?
\end{malayalam}}
\flushright{\begin{Arabic}
\quranayah[36][74]
\end{Arabic}}
\flushleft{\begin{malayalam}
തങ്ങള്‍ക്ക് സഹായം ലഭിക്കുവാന്‍ വേണ്ടി അല്ലാഹുവിന് പുറമെ പല ദൈവങ്ങളേയും അവര്‍ സ്വീകരിച്ചിരിക്കുന്നു.
\end{malayalam}}
\flushright{\begin{Arabic}
\quranayah[36][75]
\end{Arabic}}
\flushleft{\begin{malayalam}
അവരെ സഹായിക്കാന്‍ അവര്‍ക്ക് (ദൈവങ്ങള്‍ക്ക്‌) സാധിക്കുകയില്ല. അവര്‍ അവര്‍ക്ക് (ദൈവങ്ങള്‍ക്ക്‌) വേണ്ടി സജ്ജീകരിക്കപ്പെട്ട പട്ടാളമാകുന്നു.
\end{malayalam}}
\flushright{\begin{Arabic}
\quranayah[36][76]
\end{Arabic}}
\flushleft{\begin{malayalam}
അതിനാല്‍ അവരുടെ വാക്ക് നിന്നെ ദുഃഖിപ്പിക്കാതിരിക്കട്ടെ. തീര്‍ച്ചയായും അവര്‍ രഹസ്യമാക്കുന്നതും പരസ്യമാക്കുന്നതും നാം അറിയുന്നു.
\end{malayalam}}
\flushright{\begin{Arabic}
\quranayah[36][77]
\end{Arabic}}
\flushleft{\begin{malayalam}
മനുഷ്യന്‍ കണ്ടില്ലേ; അവനെ നാം ഒരു ബീജകണത്തില്‍ നിന്നാണ് സൃഷ്ടിച്ചിരിക്കുന്നതെന്ന്‌? എന്നിട്ട് അവനതാ ഒരു പ്രത്യക്ഷമായ എതിര്‍പ്പുകാരനായിരിക്കുന്നു.
\end{malayalam}}
\flushright{\begin{Arabic}
\quranayah[36][78]
\end{Arabic}}
\flushleft{\begin{malayalam}
അവന്‍ നമുക്ക് ഒരു ഉപമ എടുത്തുകാണിക്കുകയും ചെയ്തിരിക്കുന്നു. തന്നെ സൃഷ്ടിച്ചത് അവന്‍ മറന്നുകളയുകയും ചെയ്തു. അവന്‍ പറഞ്ഞു: എല്ലുകള്‍ ദ്രവിച്ച് പോയിരിക്കെ ആരാണ് അവയ്ക്ക് ജീവന്‍ നല്‍കുന്നത്‌?
\end{malayalam}}
\flushright{\begin{Arabic}
\quranayah[36][79]
\end{Arabic}}
\flushleft{\begin{malayalam}
പറയുക: ആദ്യതവണ അവയെ ഉണ്ടാക്കിയവനാരോ അവന്‍ തന്നെ അവയ്ക്ക് ജീവന്‍ നല്‍കുന്നതാണ്‌. അവന്‍ എല്ലാതരം സൃഷ്ടിപ്പിനെപ്പറ്റിയും അറിവുള്ളവനത്രെ.
\end{malayalam}}
\flushright{\begin{Arabic}
\quranayah[36][80]
\end{Arabic}}
\flushleft{\begin{malayalam}
പച്ചമരത്തില്‍ നിന്ന് നിങ്ങള്‍ക്ക് തീ ഉണ്ടാക്കിത്തന്നവനത്രെ അവന്‍ അങ്ങനെ നിങ്ങളതാ അതില്‍ നിന്ന് കത്തിച്ചെടുക്കുന്നു.
\end{malayalam}}
\flushright{\begin{Arabic}
\quranayah[36][81]
\end{Arabic}}
\flushleft{\begin{malayalam}
ആകാശങ്ങളും ഭൂമിയും സൃഷ്ടിച്ചവന്‍ അവരെപ്പോലുള്ളവരെ സൃഷ്ടിക്കാന്‍ കഴിവുള്ളവനല്ലേ? അതെ, അവനത്രെ സര്‍വ്വവും സൃഷ്ടിക്കുന്നവനും എല്ലാം അറിയുന്നവനും.
\end{malayalam}}
\flushright{\begin{Arabic}
\quranayah[36][82]
\end{Arabic}}
\flushleft{\begin{malayalam}
താന്‍ ഒരു കാര്യം ഉദ്ദേശിച്ചാല്‍ അതിനോട് ഉണ്ടാകൂ എന്ന് പറയുക മാത്രമാകുന്നു അവന്‍റെ കാര്യം. അപ്പോഴതാ അതുണ്ടാകുന്നു.
\end{malayalam}}
\flushright{\begin{Arabic}
\quranayah[36][83]
\end{Arabic}}
\flushleft{\begin{malayalam}
മുഴുവന്‍ കാര്യങ്ങളുടെയും ആധിപത്യം ആരുടെ കയ്യിലാണോ, നിങ്ങള്‍ മടക്കപ്പെടുന്നത് ആരുടെ അടുത്തേക്കാണോ അവന്‍ എത്ര പരിശുദ്ധന്‍!
\end{malayalam}}
\chapter{\textmalayalam{സ്വാഫ്ഫാത്ത് ( അണിനിരന്നവ‍ )}}
\begin{Arabic}
\Huge{\centerline{\basmalah}}\end{Arabic}
\flushright{\begin{Arabic}
\quranayah[37][1]
\end{Arabic}}
\flushleft{\begin{malayalam}
ശരിക്ക് അണിനിരന്നു നില്‍ക്കുന്നവരും,
\end{malayalam}}
\flushright{\begin{Arabic}
\quranayah[37][2]
\end{Arabic}}
\flushleft{\begin{malayalam}
എന്നിട്ട് ശക്തിയായി തടയുന്നവരും,
\end{malayalam}}
\flushright{\begin{Arabic}
\quranayah[37][3]
\end{Arabic}}
\flushleft{\begin{malayalam}
എന്നിട്ട് കീര്‍ത്തനം ചൊല്ലുന്നവരുമായ മലക്കുകളെ തന്നെയാണ സത്യം;
\end{malayalam}}
\flushright{\begin{Arabic}
\quranayah[37][4]
\end{Arabic}}
\flushleft{\begin{malayalam}
തീര്‍ച്ചയായും നിങ്ങളുടെ ദൈവം ഏകന്‍ തന്നെയാകുന്നു.
\end{malayalam}}
\flushright{\begin{Arabic}
\quranayah[37][5]
\end{Arabic}}
\flushleft{\begin{malayalam}
അതെ, ആകാശങ്ങളുടെയും ഭൂമിയുടെയും അവയ്ക്കിടയിലുള്ളതിന്‍റെയും രക്ഷിതാവും, ഉദയസ്ഥാനങ്ങളുടെ രക്ഷിതാവും ആയിട്ടുള്ളവന്‍.
\end{malayalam}}
\flushright{\begin{Arabic}
\quranayah[37][6]
\end{Arabic}}
\flushleft{\begin{malayalam}
തീര്‍ച്ചയായും അടുത്തുള്ള ആകാശത്തെ നാം നക്ഷത്രാലങ്കാരത്താല്‍ മോടിപിടിപ്പിച്ചിരിക്കുന്നു.
\end{malayalam}}
\flushright{\begin{Arabic}
\quranayah[37][7]
\end{Arabic}}
\flushleft{\begin{malayalam}
ധിക്കാരിയായ ഏതു പിശാചില്‍ നിന്നും (അതിനെ) സുരക്ഷിതമാക്കുകയും ചെയ്തിരിക്കുന്നു.
\end{malayalam}}
\flushright{\begin{Arabic}
\quranayah[37][8]
\end{Arabic}}
\flushleft{\begin{malayalam}
അത്യുന്നതമായ സമൂഹത്തിന്‍റെ നേരെ അവര്‍ക്ക് (പിശാചുക്കള്‍ക്ക്‌) ചെവികൊടുത്തു കേള്‍ക്കാനാവില്ല. എല്ലാവശത്തു നിന്നും അവര്‍ എറിഞ്ഞ് ഓടിക്കപ്പെടുകയും ചെയ്യും;
\end{malayalam}}
\flushright{\begin{Arabic}
\quranayah[37][9]
\end{Arabic}}
\flushleft{\begin{malayalam}
ബഹിഷ്കൃതരായിക്കൊണ്ട് അവര്‍ക്ക് ശാശ്വതമായ ശിക്ഷയുമുണ്ട്‌.
\end{malayalam}}
\flushright{\begin{Arabic}
\quranayah[37][10]
\end{Arabic}}
\flushleft{\begin{malayalam}
പക്ഷെ, ആരെങ്കിലും പെട്ടെന്ന് വല്ലതും റാഞ്ചിഎടുക്കുകയാണെങ്കില്‍ തുളച്ച് കടക്കുന്ന ഒരു തീജ്വാല അവനെ പിന്തുടരുന്നതാണ്‌.
\end{malayalam}}
\flushright{\begin{Arabic}
\quranayah[37][11]
\end{Arabic}}
\flushleft{\begin{malayalam}
ആകയാല്‍ (നബിയേ,) നീ അവരോട് (ആ നിഷേധികളോട്‌) അഭിപ്രായം ആരായുക: സൃഷ്ടിക്കാന്‍ ഏറ്റവും പ്രയാസമുള്ളത് അവരെയാണോ, അതല്ല, നാം സൃഷ്ടിച്ചിട്ടുള്ള മറ്റു സൃഷ്ടികളെയാണോ? തീര്‍ച്ചയായും നാം അവരെ സൃഷ്ടിച്ചിരിക്കുന്നത് പശിമയുള്ള കളിമണ്ണില്‍ നിന്നാകുന്നു.
\end{malayalam}}
\flushright{\begin{Arabic}
\quranayah[37][12]
\end{Arabic}}
\flushleft{\begin{malayalam}
പക്ഷെ, നിനക്ക് അത്ഭുതം തോന്നി. അവരാകട്ടെ പരിഹസിക്കുകയും ചെയ്യുന്നു.
\end{malayalam}}
\flushright{\begin{Arabic}
\quranayah[37][13]
\end{Arabic}}
\flushleft{\begin{malayalam}
അവര്‍ക്ക് ഉപദേശം നല്‍കപ്പെട്ടാല്‍ അവര്‍ ആലോചിക്കുന്നില്ല.
\end{malayalam}}
\flushright{\begin{Arabic}
\quranayah[37][14]
\end{Arabic}}
\flushleft{\begin{malayalam}
അവര്‍ ഏതൊരു ദൃഷ്ടാന്തം കണ്ടാലും തമാശയാക്കിക്കളയുന്നു.
\end{malayalam}}
\flushright{\begin{Arabic}
\quranayah[37][15]
\end{Arabic}}
\flushleft{\begin{malayalam}
അവര്‍ പറയും: ഇത് പ്രത്യക്ഷമായ ഒരു ജാലവിദ്യ മാത്രമാകുന്നു എന്ന്‌.
\end{malayalam}}
\flushright{\begin{Arabic}
\quranayah[37][16]
\end{Arabic}}
\flushleft{\begin{malayalam}
(അവര്‍ പറയും:) മരിച്ച് മണ്ണും അസ്ഥിശകലങ്ങളുമായിക്കഴിഞ്ഞാല്‍ ഞങ്ങള്‍ ഉയിര്‍ത്തെഴുന്നേല്‍പിക്കപ്പെടുക തന്നെ ചെയ്യുമോ?
\end{malayalam}}
\flushright{\begin{Arabic}
\quranayah[37][17]
\end{Arabic}}
\flushleft{\begin{malayalam}
ഞങ്ങളുടെ പൂര്‍വ്വപിതാക്കളും (ഉയിര്‍ത്തെഴുന്നേല്‍പിക്കപ്പെടുമോ?)
\end{malayalam}}
\flushright{\begin{Arabic}
\quranayah[37][18]
\end{Arabic}}
\flushleft{\begin{malayalam}
പറയുക: അതെ. (അന്ന്‌) നിങ്ങള്‍ അപമാനിതരാകുകയും ചെയ്യും.
\end{malayalam}}
\flushright{\begin{Arabic}
\quranayah[37][19]
\end{Arabic}}
\flushleft{\begin{malayalam}
എന്നാല്‍ അത് ഒരു ഘോരശബ്ദം മാത്രമായിരിക്കും. അപ്പോഴതാ അവര്‍ (എഴുന്നേറ്റ് നിന്ന്‌) നോക്കുന്നു.
\end{malayalam}}
\flushright{\begin{Arabic}
\quranayah[37][20]
\end{Arabic}}
\flushleft{\begin{malayalam}
അവര്‍ പറയും: അഹോ! ഞങ്ങള്‍ക്ക് കഷ്ടം! ഇത് പ്രതിഫലത്തിന്‍റെ ദിനമാണല്ലോ!
\end{malayalam}}
\flushright{\begin{Arabic}
\quranayah[37][21]
\end{Arabic}}
\flushleft{\begin{malayalam}
(അവര്‍ക്ക് മറുപടി നല്‍കപ്പെടും:) അതെ; നിങ്ങള്‍ നിഷേധിച്ച് തള്ളിക്കളഞ്ഞിരുന്ന നിര്‍ണായകമായ തീരുമാനത്തിന്‍റെ ദിവസമത്രെ ഇത്‌.
\end{malayalam}}
\flushright{\begin{Arabic}
\quranayah[37][22]
\end{Arabic}}
\flushleft{\begin{malayalam}
(അപ്പോള്‍ അല്ലാഹുവിന്‍റെ കല്‍പനയുണ്ടാകും;) അക്രമം ചെയ്തവരെയും അവരുടെ ഇണകളെയും അവര്‍ ആരാധിച്ചിരുന്നവയെയും നിങ്ങള്‍ ഒരുമിച്ചുകൂട്ടുക.
\end{malayalam}}
\flushright{\begin{Arabic}
\quranayah[37][23]
\end{Arabic}}
\flushleft{\begin{malayalam}
അല്ലാഹുവിനു പുറമെ. എന്നിട്ട് അവരെ നിങ്ങള്‍ നരകത്തിന്‍റെ വഴിയിലേക്ക് നയിക്കുക.
\end{malayalam}}
\flushright{\begin{Arabic}
\quranayah[37][24]
\end{Arabic}}
\flushleft{\begin{malayalam}
അവരെ നിങ്ങളൊന്നു നിര്‍ത്തുക. അവരോട് ചോദ്യം ചെയ്യേണ്ടതാകുന്നു.
\end{malayalam}}
\flushright{\begin{Arabic}
\quranayah[37][25]
\end{Arabic}}
\flushleft{\begin{malayalam}
നിങ്ങള്‍ക്ക് എന്തുപറ്റി? നിങ്ങള്‍ പരസ്പരം സഹായിക്കുന്നില്ലല്ലോ എന്ന്‌
\end{malayalam}}
\flushright{\begin{Arabic}
\quranayah[37][26]
\end{Arabic}}
\flushleft{\begin{malayalam}
അല്ല, അവര്‍ ആ ദിവസത്തില്‍ കീഴടങ്ങിയവരായിരിക്കും.
\end{malayalam}}
\flushright{\begin{Arabic}
\quranayah[37][27]
\end{Arabic}}
\flushleft{\begin{malayalam}
അവരില്‍ ചിലര്‍ ചിലരുടെ നേരെ തിരിഞ്ഞ് പരസ്പരം ചോദ്യം ചെയ്യും.
\end{malayalam}}
\flushright{\begin{Arabic}
\quranayah[37][28]
\end{Arabic}}
\flushleft{\begin{malayalam}
അവര്‍ പറയും: തീര്‍ച്ചയായും നിങ്ങള്‍ ഞങ്ങളുടെ അടുത്ത് കൈയ്യൂക്കുമായി വന്ന് (ഞങ്ങളെ സത്യത്തില്‍ നിന്ന് പിന്തിരിപ്പിക്കുകയായിരുന്നു.)
\end{malayalam}}
\flushright{\begin{Arabic}
\quranayah[37][29]
\end{Arabic}}
\flushleft{\begin{malayalam}
അവര്‍ മറുപടി പറയും: അല്ല, നിങ്ങള്‍ തന്നെ വിശ്വാസികളാവാതിരിക്കുകയാണുണ്ടായത്‌.
\end{malayalam}}
\flushright{\begin{Arabic}
\quranayah[37][30]
\end{Arabic}}
\flushleft{\begin{malayalam}
ഞങ്ങള്‍ക്കാകട്ടെ നിങ്ങളുടെ മേല്‍ ഒരധികാരവും ഉണ്ടായിരുന്നതുമില്ല. പ്രത്യുത, നിങ്ങള്‍ അതിക്രമകാരികളായ ഒരു ജനവിഭാഗമായിരുന്നു.
\end{malayalam}}
\flushright{\begin{Arabic}
\quranayah[37][31]
\end{Arabic}}
\flushleft{\begin{malayalam}
അങ്ങനെ നമ്മുടെ മേല്‍ നമ്മുടെ രക്ഷിതാവിന്‍റെ വചനം യാഥാര്‍ത്ഥ്യമായിതീര്‍ന്നു. തീര്‍ച്ചയായും നാം (ശിക്ഷ) അനുഭവിക്കാന്‍ പോകുകയാണ്‌.
\end{malayalam}}
\flushright{\begin{Arabic}
\quranayah[37][32]
\end{Arabic}}
\flushleft{\begin{malayalam}
അപ്പോള്‍ ഞങ്ങള്‍ നിങ്ങളെ വഴികേടിലെത്തിച്ചിരിക്കുന്നു.(കാരണം) തീര്‍ച്ചയായും ഞങ്ങള്‍ വഴിതെറ്റിയവരായിരുന്നു.
\end{malayalam}}
\flushright{\begin{Arabic}
\quranayah[37][33]
\end{Arabic}}
\flushleft{\begin{malayalam}
അപ്പോള്‍ അന്നേ ദിവസം തീര്‍ച്ചയായും അവര്‍ (ഇരുവിഭാഗവും) ശിക്ഷയില്‍ പങ്കാളികളായിരിക്കും.
\end{malayalam}}
\flushright{\begin{Arabic}
\quranayah[37][34]
\end{Arabic}}
\flushleft{\begin{malayalam}
തീര്‍ച്ചയായും നാം കുറ്റവാളികളെക്കൊണ്ട് ചെയ്യുന്നത് അപ്രകാരമാകുന്നു.
\end{malayalam}}
\flushright{\begin{Arabic}
\quranayah[37][35]
\end{Arabic}}
\flushleft{\begin{malayalam}
അല്ലാഹു അല്ലാതെ ഒരു ദൈവവുമില്ല എന്ന് അവരോട് പറയപ്പെട്ടാല്‍ അവര്‍ അഹങ്കാരം നടിക്കുമായിരുന്നു.
\end{malayalam}}
\flushright{\begin{Arabic}
\quranayah[37][36]
\end{Arabic}}
\flushleft{\begin{malayalam}
ഭ്രാന്തനായ ഒരു കവിക്ക് വേണ്ടി ഞങ്ങള്‍ ഞങ്ങളുടെ ദൈവങ്ങളെ ഉപേക്ഷിച്ച് കളയണമോ എന്ന് ചോദിക്കുകയും ചെയ്യുമായിരുന്നു.
\end{malayalam}}
\flushright{\begin{Arabic}
\quranayah[37][37]
\end{Arabic}}
\flushleft{\begin{malayalam}
അല്ല, സത്യവും കൊണ്ടാണ് അദ്ദേഹം വന്നത്‌. (മുമ്പ് വന്ന) ദൈവദൂതന്‍മാരെ അദ്ദേഹം സത്യപ്പെടുത്തുകയും ചെയ്തിരിക്കുന്നു.
\end{malayalam}}
\flushright{\begin{Arabic}
\quranayah[37][38]
\end{Arabic}}
\flushleft{\begin{malayalam}
തീര്‍ച്ചയായും നിങ്ങള്‍ വേദനയേറിയ ശിക്ഷ ആസ്വദിക്കുക തന്നെ ചെയ്യേണ്ടവരാകുന്നു.
\end{malayalam}}
\flushright{\begin{Arabic}
\quranayah[37][39]
\end{Arabic}}
\flushleft{\begin{malayalam}
നിങ്ങള്‍ പ്രവര്‍ത്തിച്ചിരുന്നതിനു മാത്രമേ നിങ്ങള്‍ക്ക് പ്രതിഫലം നല്‍കപ്പെടുകയുള്ളു.
\end{malayalam}}
\flushright{\begin{Arabic}
\quranayah[37][40]
\end{Arabic}}
\flushleft{\begin{malayalam}
അല്ലാഹുവിന്‍റെ നിഷ്കളങ്കരായ ദാസന്‍മാര്‍ ഇതില്‍ നിന്ന് ഒഴിവാകുന്നു.
\end{malayalam}}
\flushright{\begin{Arabic}
\quranayah[37][41]
\end{Arabic}}
\flushleft{\begin{malayalam}
അങ്ങനെയുള്ളവര്‍ക്കാകുന്നു അറിയപ്പെട്ട ഉപജീവനം.
\end{malayalam}}
\flushright{\begin{Arabic}
\quranayah[37][42]
\end{Arabic}}
\flushleft{\begin{malayalam}
വിവിധ തരം പഴവര്‍ഗങ്ങള്‍. അവര്‍ ആദരിക്കപ്പെടുന്നവരായിരിക്കും.
\end{malayalam}}
\flushright{\begin{Arabic}
\quranayah[37][43]
\end{Arabic}}
\flushleft{\begin{malayalam}
സൌഭാഗ്യത്തിന്‍റെ സ്വര്‍ഗത്തോപ്പുകളില്‍.
\end{malayalam}}
\flushright{\begin{Arabic}
\quranayah[37][44]
\end{Arabic}}
\flushleft{\begin{malayalam}
അവര്‍ ചില കട്ടിലുകളില്‍ പരസ്പരം അഭിമുഖമായി ഇരിക്കുന്നവരായിരിക്കും.
\end{malayalam}}
\flushright{\begin{Arabic}
\quranayah[37][45]
\end{Arabic}}
\flushleft{\begin{malayalam}
ഒരു തരം ഉറവു ജലം നിറച്ച കോപ്പകള്‍ അവരുടെ ചുറ്റും കൊണ്ടു നടക്കപ്പെടും.
\end{malayalam}}
\flushright{\begin{Arabic}
\quranayah[37][46]
\end{Arabic}}
\flushleft{\begin{malayalam}
വെളുത്തതും കുടിക്കുന്നവര്‍ക്ക് ഹൃദ്യവുമായ പാനീയം.
\end{malayalam}}
\flushright{\begin{Arabic}
\quranayah[37][47]
\end{Arabic}}
\flushleft{\begin{malayalam}
അതില്‍ യാതൊരു ദോഷവുമില്ല. അത് നിമിത്തം അവര്‍ക്ക് ലഹരി ബാധിക്കുകയുമില്ല.
\end{malayalam}}
\flushright{\begin{Arabic}
\quranayah[37][48]
\end{Arabic}}
\flushleft{\begin{malayalam}
ദൃഷ്ടി നിയന്ത്രിക്കുന്നവരും വിശാലമായ കണ്ണുകളുള്ളവരുമായ സ്ത്രീകള്‍ അവരുടെ അടുത്ത് ഉണ്ടായിരിക്കും.
\end{malayalam}}
\flushright{\begin{Arabic}
\quranayah[37][49]
\end{Arabic}}
\flushleft{\begin{malayalam}
സൂക്ഷിച്ചു വെക്കപ്പെട്ട മുട്ടകള്‍ പോലെയിരിക്കും അവര്‍.
\end{malayalam}}
\flushright{\begin{Arabic}
\quranayah[37][50]
\end{Arabic}}
\flushleft{\begin{malayalam}
ആ സ്വര്‍ഗവാസികളില്‍ ചിലര്‍ ചിലരുടെ നേരെ തിരിഞ്ഞു കൊണ്ട് പരസ്പരം (പല ചോദ്യങ്ങളും) ചോദിക്കും
\end{malayalam}}
\flushright{\begin{Arabic}
\quranayah[37][51]
\end{Arabic}}
\flushleft{\begin{malayalam}
അവരില്‍ നിന്ന് ഒരു വക്താവ് പറയും: തീര്‍ച്ചയായും എനിക്ക് ഒരു കൂട്ടുകാരനുണ്ടായിരുന്നു.
\end{malayalam}}
\flushright{\begin{Arabic}
\quranayah[37][52]
\end{Arabic}}
\flushleft{\begin{malayalam}
അവന്‍ പറയുമായിരുന്നു: തീര്‍ച്ചയായും നീ (പരലോകത്തില്‍) വിശ്വസിക്കുന്നവരുടെ കൂട്ടത്തില്‍ തന്നെയാണോ?
\end{malayalam}}
\flushright{\begin{Arabic}
\quranayah[37][53]
\end{Arabic}}
\flushleft{\begin{malayalam}
നാം മരിച്ചിട്ട് മണ്ണും അസ്ഥിശകലങ്ങളുമായി കഴിഞ്ഞാലും നമുക്ക് നമ്മുടെ കര്‍മ്മഫലങ്ങള്‍ നല്‍കപ്പെടുന്നതാണോ?
\end{malayalam}}
\flushright{\begin{Arabic}
\quranayah[37][54]
\end{Arabic}}
\flushleft{\begin{malayalam}
തുടര്‍ന്ന് ആ വക്താവ് (കൂടെയുള്ളവരോട്‌) പറയും: നിങ്ങള്‍ (ആ കൂട്ടുകാരനെ) എത്തിനോക്കാന്‍ ഉദ്ദേശിക്കുന്നുണ്ടോ?
\end{malayalam}}
\flushright{\begin{Arabic}
\quranayah[37][55]
\end{Arabic}}
\flushleft{\begin{malayalam}
എന്നിട്ട് അദ്ദേഹം എത്തിനോക്കും. അപ്പോള്‍ അദ്ദേഹം അവനെ നരകത്തിന്‍റെ മദ്ധ്യത്തില്‍ കാണും.
\end{malayalam}}
\flushright{\begin{Arabic}
\quranayah[37][56]
\end{Arabic}}
\flushleft{\begin{malayalam}
അദ്ദേഹം (അവനോട്‌) പറയും: അല്ലാഹുവെ തന്നെയാണ! നീ എന്നെ നാശത്തില്‍ അകപ്പെടുത്തുക തന്നെ ചെയ്തേക്കുമായിരുന്നു.
\end{malayalam}}
\flushright{\begin{Arabic}
\quranayah[37][57]
\end{Arabic}}
\flushleft{\begin{malayalam}
എന്‍റെ രക്ഷിതാവിന്‍റെ അനുഗ്രഹം ഇല്ലായിരുന്നുവെങ്കില്‍ (ആ നരകത്തില്‍) ഹാജരാക്കപ്പെടുന്നവരില്‍ ഞാനും ഉള്‍പെടുമായിരുന്നു.
\end{malayalam}}
\flushright{\begin{Arabic}
\quranayah[37][58]
\end{Arabic}}
\flushleft{\begin{malayalam}
(സ്വര്‍ഗവാസികള്‍ പറയും:) ഇനി നാം മരണപ്പെടുന്നവരല്ലല്ലോ
\end{malayalam}}
\flushright{\begin{Arabic}
\quranayah[37][59]
\end{Arabic}}
\flushleft{\begin{malayalam}
നമ്മുടെ ആദ്യത്തെ മരണമല്ലാതെ. നാം ശിക്ഷിക്കപ്പെടുന്നവരുമല്ല.
\end{malayalam}}
\flushright{\begin{Arabic}
\quranayah[37][60]
\end{Arabic}}
\flushleft{\begin{malayalam}
തീര്‍ച്ചയായും ഇതു തന്നെയാണ് മഹത്തായ ഭാഗ്യം.
\end{malayalam}}
\flushright{\begin{Arabic}
\quranayah[37][61]
\end{Arabic}}
\flushleft{\begin{malayalam}
ഇതുപോലെയുള്ളതിന് വേണ്ടിയാകട്ടെ പ്രവര്‍ത്തകന്‍മാര്‍ പ്രവര്‍ത്തിക്കുന്നത്‌.
\end{malayalam}}
\flushright{\begin{Arabic}
\quranayah[37][62]
\end{Arabic}}
\flushleft{\begin{malayalam}
അതാണോ വിശിഷ്ടമായ സല്‍ക്കാരം? അതല്ല സഖ്ഖൂം വൃക്ഷമാണോ?
\end{malayalam}}
\flushright{\begin{Arabic}
\quranayah[37][63]
\end{Arabic}}
\flushleft{\begin{malayalam}
തീര്‍ച്ചയായും അതിനെ നാം അക്രമകാരികള്‍ക്ക് ഒരു പരീക്ഷണമാക്കിയിരിക്കുന്നു.
\end{malayalam}}
\flushright{\begin{Arabic}
\quranayah[37][64]
\end{Arabic}}
\flushleft{\begin{malayalam}
നരകത്തിന്‍റെ അടിയില്‍ മുളച്ചു പൊങ്ങുന്ന ഒരു വൃക്ഷമത്രെ അത്‌.
\end{malayalam}}
\flushright{\begin{Arabic}
\quranayah[37][65]
\end{Arabic}}
\flushleft{\begin{malayalam}
അതിന്‍റെ കുല പിശാചുക്കളുടെ തലകള്‍ പോലെയിരിക്കും.
\end{malayalam}}
\flushright{\begin{Arabic}
\quranayah[37][66]
\end{Arabic}}
\flushleft{\begin{malayalam}
തീര്‍ച്ചയായും അവര്‍ അതില്‍ നിന്ന് തിന്ന് വയറ് നിറക്കുന്നവരായിരിക്കും.
\end{malayalam}}
\flushright{\begin{Arabic}
\quranayah[37][67]
\end{Arabic}}
\flushleft{\begin{malayalam}
പിന്നീട് അവര്‍ക്ക് അതിനു മീതെ ചുട്ടുതിളക്കുന്ന വെള്ളത്തിന്‍റെ ഒരു ചേരുവയുണ്ട്‌.
\end{malayalam}}
\flushright{\begin{Arabic}
\quranayah[37][68]
\end{Arabic}}
\flushleft{\begin{malayalam}
പിന്നീട് തീര്‍ച്ചയായും അവരുടെ മടക്കം നരകത്തിലേക്ക് തന്നെയാകുന്നു.
\end{malayalam}}
\flushright{\begin{Arabic}
\quranayah[37][69]
\end{Arabic}}
\flushleft{\begin{malayalam}
തീര്‍ച്ചയായും അവര്‍ തങ്ങളുടെ പിതാക്കളെ കണ്ടെത്തിയത് വഴിപിഴച്ചവരായിട്ടാണ്‌.
\end{malayalam}}
\flushright{\begin{Arabic}
\quranayah[37][70]
\end{Arabic}}
\flushleft{\begin{malayalam}
അങ്ങനെ ഇവര്‍ അവരുടെ (പിതാക്കളുടെ) കാല്‍പാടുകളിലൂടെ കുതിച്ചു പായുന്നു.
\end{malayalam}}
\flushright{\begin{Arabic}
\quranayah[37][71]
\end{Arabic}}
\flushleft{\begin{malayalam}
ഇവര്‍ക്ക് മുമ്പ് പൂര്‍വ്വികരില്‍ അധികപേരും വഴിപിഴച്ചു പോകുക തന്നെയാണുണ്ടായത്‌.
\end{malayalam}}
\flushright{\begin{Arabic}
\quranayah[37][72]
\end{Arabic}}
\flushleft{\begin{malayalam}
അവരില്‍ നാം താക്കീതുകാരെ നിയോഗിക്കുകയുമുണ്ടായിട്ടുണ്ട്‌.
\end{malayalam}}
\flushright{\begin{Arabic}
\quranayah[37][73]
\end{Arabic}}
\flushleft{\begin{malayalam}
എന്നിട്ട് നോക്കൂ; ആ താക്കീത് നല്‍കപ്പെട്ടവരുടെ പര്യവസാനം എങ്ങനെയായിരുന്നു വെന്ന്‌.
\end{malayalam}}
\flushright{\begin{Arabic}
\quranayah[37][74]
\end{Arabic}}
\flushleft{\begin{malayalam}
അല്ലാഹുവിന്‍റെ നിഷ്കളങ്കരായ ദാസന്‍മാര്‍ ഒഴികെ.
\end{malayalam}}
\flushright{\begin{Arabic}
\quranayah[37][75]
\end{Arabic}}
\flushleft{\begin{malayalam}
നൂഹ് നമ്മെ വിളിക്കുകയുണ്ടായി. അപ്പോള്‍ ഉത്തരം നല്‍കിയവന്‍ എത്ര നല്ലവന്‍!
\end{malayalam}}
\flushright{\begin{Arabic}
\quranayah[37][76]
\end{Arabic}}
\flushleft{\begin{malayalam}
അദ്ദേഹത്തെയും അദ്ദേഹത്തിന്‍റെ ആളുകളെയും നാം വമ്പിച്ച ദുരന്തത്തില്‍ നിന്ന് രക്ഷപ്പെടുത്തി.
\end{malayalam}}
\flushright{\begin{Arabic}
\quranayah[37][77]
\end{Arabic}}
\flushleft{\begin{malayalam}
അദ്ദേഹത്തിന്‍റെ സന്തതികളെ നാം (ഭൂമിയില്‍) നിലനില്‍ക്കുന്നവരാക്കുകയും.
\end{malayalam}}
\flushright{\begin{Arabic}
\quranayah[37][78]
\end{Arabic}}
\flushleft{\begin{malayalam}
പില്‍ക്കാലത്ത് വന്നവരില്‍ അദ്ദേഹത്തെപറ്റിയുള്ള സല്‍കീര്‍ത്തി നാം അവശേഷിപ്പിക്കുകയും ചെയ്തു.
\end{malayalam}}
\flushright{\begin{Arabic}
\quranayah[37][79]
\end{Arabic}}
\flushleft{\begin{malayalam}
ലോകരില്‍ നൂഹിന് സമാധാനം!
\end{malayalam}}
\flushright{\begin{Arabic}
\quranayah[37][80]
\end{Arabic}}
\flushleft{\begin{malayalam}
തീര്‍ച്ചയായും അപ്രകാരമാണ് സദ്‌വൃത്തന്‍മാര്‍ക്ക് നാം പ്രതിഫലം നല്‍കുന്നത്‌.
\end{malayalam}}
\flushright{\begin{Arabic}
\quranayah[37][81]
\end{Arabic}}
\flushleft{\begin{malayalam}
തീര്‍ച്ചയായും അദ്ദേഹം നമ്മുടെ സത്യവിശ്വാസികളായ ദാസന്‍മാരുടെ കൂട്ടത്തിലാകുന്നു.
\end{malayalam}}
\flushright{\begin{Arabic}
\quranayah[37][82]
\end{Arabic}}
\flushleft{\begin{malayalam}
പിന്നീട് നാം മറ്റുള്ളവരെ മുക്കിനശിപ്പിച്ചു.
\end{malayalam}}
\flushright{\begin{Arabic}
\quranayah[37][83]
\end{Arabic}}
\flushleft{\begin{malayalam}
തീര്‍ച്ചയായും അദ്ദേഹത്തിന്‍റെ കക്ഷികളില്‍ പെട്ട ആള്‍ തന്നെയാകുന്നു ഇബ്രാഹീം.
\end{malayalam}}
\flushright{\begin{Arabic}
\quranayah[37][84]
\end{Arabic}}
\flushleft{\begin{malayalam}
നിഷ്കളങ്കമായ ഹൃദയത്തോടു കൂടി അദ്ദേഹം തന്‍റെ രക്ഷിതാവിങ്കല്‍ വന്ന സന്ദര്‍ഭം (ശ്രദ്ധേയമാകുന്നു.)
\end{malayalam}}
\flushright{\begin{Arabic}
\quranayah[37][85]
\end{Arabic}}
\flushleft{\begin{malayalam}
തന്‍റെ പിതാവിനോടും ജനതയോടും അദ്ദേഹം ഇപ്രകാരം പറഞ്ഞ സന്ദര്‍ഭം: എന്തൊന്നിനെയാണ് നിങ്ങള്‍ ആരാധിക്കുന്നത്‌?
\end{malayalam}}
\flushright{\begin{Arabic}
\quranayah[37][86]
\end{Arabic}}
\flushleft{\begin{malayalam}
അല്ലാഹുവിന്നു പുറമെ വ്യാജമായി നിങ്ങള്‍ മറ്റു ദൈവങ്ങളെ ആഗ്രഹിക്കുകയാണോ?
\end{malayalam}}
\flushright{\begin{Arabic}
\quranayah[37][87]
\end{Arabic}}
\flushleft{\begin{malayalam}
അപ്പോള്‍ ലോകരക്ഷിതാവിനെപ്പറ്റി നിങ്ങളുടെ വിചാരമെന്താണ്‌?
\end{malayalam}}
\flushright{\begin{Arabic}
\quranayah[37][88]
\end{Arabic}}
\flushleft{\begin{malayalam}
എന്നിട്ട് അദ്ദേഹം നക്ഷത്രങ്ങളുടെ നേരെ ഒരു നോട്ടം നോക്കി.
\end{malayalam}}
\flushright{\begin{Arabic}
\quranayah[37][89]
\end{Arabic}}
\flushleft{\begin{malayalam}
തുടര്‍ന്ന് അദ്ദേഹം പറഞ്ഞു: തീര്‍ച്ചയായും എനിക്ക് അസുഖമാകുന്നു.
\end{malayalam}}
\flushright{\begin{Arabic}
\quranayah[37][90]
\end{Arabic}}
\flushleft{\begin{malayalam}
അപ്പോള്‍ അവര്‍ അദ്ദേഹത്തെ വിട്ട് പിന്തിരിഞ്ഞു പോയി.
\end{malayalam}}
\flushright{\begin{Arabic}
\quranayah[37][91]
\end{Arabic}}
\flushleft{\begin{malayalam}
എന്നിട്ട് അദ്ദേഹം അവരുടെ ദൈവങ്ങളുടെ നേര്‍ക്ക് തിരിഞ്ഞിട്ടു പറഞ്ഞു: നിങ്ങള്‍ തിന്നുന്നില്ലേ?
\end{malayalam}}
\flushright{\begin{Arabic}
\quranayah[37][92]
\end{Arabic}}
\flushleft{\begin{malayalam}
നിങ്ങള്‍ക്കെന്തുപറ്റി? നിങ്ങള്‍ മിണ്ടുന്നില്ലല്ലോ?
\end{malayalam}}
\flushright{\begin{Arabic}
\quranayah[37][93]
\end{Arabic}}
\flushleft{\begin{malayalam}
തുടര്‍ന്ന് അദ്ദേഹം അവയുടെ നേരെ തിരിഞ്ഞു വലതുകൈ കൊണ്ട് ഊക്കോടെ അവയെ വെട്ടിക്കളഞ്ഞു.
\end{malayalam}}
\flushright{\begin{Arabic}
\quranayah[37][94]
\end{Arabic}}
\flushleft{\begin{malayalam}
എന്നിട്ട് അവര്‍ അദ്ദേഹത്തിന്‍റെ അടുത്തേക്ക് കുതിച്ച് ചെന്നു.
\end{malayalam}}
\flushright{\begin{Arabic}
\quranayah[37][95]
\end{Arabic}}
\flushleft{\begin{malayalam}
അദ്ദേഹം പറഞ്ഞു: നിങ്ങള്‍ തന്നെ കൊത്തിയുണ്ടാക്കുന്നവയെയാണോ നിങ്ങള്‍ ആരാധിക്കുന്നത്‌?
\end{malayalam}}
\flushright{\begin{Arabic}
\quranayah[37][96]
\end{Arabic}}
\flushleft{\begin{malayalam}
അല്ലാഹുവാണല്ലോ നിങ്ങളെയും നിങ്ങള്‍ നിര്‍മിക്കുന്നവയെയും സൃഷ്ടിച്ചത്‌.
\end{malayalam}}
\flushright{\begin{Arabic}
\quranayah[37][97]
\end{Arabic}}
\flushleft{\begin{malayalam}
അവര്‍ (അന്യോന്യം) പറഞ്ഞു: നിങ്ങള്‍ അവന്ന് (ഇബ്രാഹീമിന്‌) വേണ്ടി ഒരു ചൂള പണിയുക. എന്നിട്ടവനെ ജ്വലിക്കുന്ന അഗ്നിയില്‍ ഇട്ടേക്കുക.
\end{malayalam}}
\flushright{\begin{Arabic}
\quranayah[37][98]
\end{Arabic}}
\flushleft{\begin{malayalam}
അങ്ങനെ അദ്ദേഹത്തിന്‍റെ കാര്യത്തില്‍ അവര്‍ ഒരു തന്ത്രം ഉദ്ദേശിച്ചു. എന്നാല്‍ നാം അവരെ ഏറ്റവും അധമന്‍മാരാക്കുകയാണ് ചെയ്തത്‌.
\end{malayalam}}
\flushright{\begin{Arabic}
\quranayah[37][99]
\end{Arabic}}
\flushleft{\begin{malayalam}
അദ്ദേഹം പറഞ്ഞു: തീര്‍ച്ചയായും ഞാന്‍ എന്‍റെ രക്ഷിതാവിങ്കലേക്ക് പോകുകയാണ്‌. അവന്‍ എനിക്ക് വഴി കാണിക്കുന്നതാണ്‌.
\end{malayalam}}
\flushright{\begin{Arabic}
\quranayah[37][100]
\end{Arabic}}
\flushleft{\begin{malayalam}
എന്‍റെ രക്ഷിതാവേ, സദ്‌വൃത്തരില്‍ ഒരാളെ നീ എനിക്ക് (പുത്രനായി) പ്രദാനം ചെയ്യേണമേ.
\end{malayalam}}
\flushright{\begin{Arabic}
\quranayah[37][101]
\end{Arabic}}
\flushleft{\begin{malayalam}
അപ്പോള്‍ സഹനശീലനായ ഒരു ബാലനെപ്പറ്റി നാം അദ്ദേഹത്തിന് സന്തോഷവാര്‍ത്ത അറിയിച്ചു.
\end{malayalam}}
\flushright{\begin{Arabic}
\quranayah[37][102]
\end{Arabic}}
\flushleft{\begin{malayalam}
എന്നിട്ട് ആ ബാലന്‍ അദ്ദേഹത്തോടൊപ്പം പ്രയത്നിക്കാനുള്ള പ്രായമെത്തിയപ്പോള്‍ അദ്ദേഹം പറഞ്ഞു: എന്‍റെ കുഞ്ഞുമകനേ! ഞാന്‍ നിന്നെ അറുക്കണമെന്ന് ഞാന്‍ സ്വപ്നത്തില്‍ കാണുന്നു. അതുകൊണ്ട് നോക്കൂ: നീ എന്താണ് അഭിപ്രായപ്പെടുന്നത്‌? അവന്‍ പറഞ്ഞു: എന്‍റെ പിതാവേ, കല്‍പിക്കപ്പെടുന്നതെന്തോ അത് താങ്കള്‍ ചെയ്തുകൊള്ളുക. അല്ലാഹു ഉദ്ദേശിക്കുന്ന പക്ഷം ക്ഷമാശീലരുടെ കൂട്ടത്തില്‍ താങ്കള്‍ എന്നെ കണ്ടെത്തുന്നതാണ്‌.
\end{malayalam}}
\flushright{\begin{Arabic}
\quranayah[37][103]
\end{Arabic}}
\flushleft{\begin{malayalam}
അങ്ങനെ അവര്‍ ഇരുവരും (കല്‍പനക്ക്‌) കീഴ്പെടുകയും, അവനെ നെറ്റി (ചെന്നി) മേല്‍ ചെരിച്ചു കിടത്തുകയും ചെയ്ത സന്ദര്‍ഭം!
\end{malayalam}}
\flushright{\begin{Arabic}
\quranayah[37][104]
\end{Arabic}}
\flushleft{\begin{malayalam}
നാം അദ്ദേഹത്തെ വിളിച്ചുപറഞ്ഞു: ഹേ! ഇബ്രാഹീം,
\end{malayalam}}
\flushright{\begin{Arabic}
\quranayah[37][105]
\end{Arabic}}
\flushleft{\begin{malayalam}
തീര്‍ച്ചയായും നീ സ്വപ്നം സാക്ഷാത്കരിച്ചിരിക്കുന്നു. തീര്‍ച്ചയായും അപ്രകാരമാണ് നാം സദ്‌വൃത്തര്‍ക്ക് പ്രതിഫലം നല്‍കുന്നത്‌.
\end{malayalam}}
\flushright{\begin{Arabic}
\quranayah[37][106]
\end{Arabic}}
\flushleft{\begin{malayalam}
തീര്‍ച്ചയായും ഇത് സ്പഷ്ടമായ പരീക്ഷണം തന്നെയാണ്‌.
\end{malayalam}}
\flushright{\begin{Arabic}
\quranayah[37][107]
\end{Arabic}}
\flushleft{\begin{malayalam}
അവന്ന് പകരം ബലിയര്‍പ്പിക്കാനായി മഹത്തായ ഒരു ബലിമൃഗത്തെ നാം നല്‍കുകയും ചെയ്തു.
\end{malayalam}}
\flushright{\begin{Arabic}
\quranayah[37][108]
\end{Arabic}}
\flushleft{\begin{malayalam}
പില്‍ക്കാലക്കാരില്‍ അദ്ദേഹത്തിന്‍റെ (ഇബ്രാഹീമിന്‍റെ) സല്‍കീര്‍ത്തി നാം അവശേഷിപ്പിക്കുകയും ചെയ്തു.
\end{malayalam}}
\flushright{\begin{Arabic}
\quranayah[37][109]
\end{Arabic}}
\flushleft{\begin{malayalam}
ഇബ്രാഹീമിന് സമാധാനം!
\end{malayalam}}
\flushright{\begin{Arabic}
\quranayah[37][110]
\end{Arabic}}
\flushleft{\begin{malayalam}
അപ്രകാരമാണ് നാം സദ്‌വൃത്തര്‍ക്ക് പ്രതിഫലം നല്‍കുന്നത്‌.
\end{malayalam}}
\flushright{\begin{Arabic}
\quranayah[37][111]
\end{Arabic}}
\flushleft{\begin{malayalam}
തീര്‍ച്ചയയും അദ്ദേഹം നമ്മുടെ സത്യവിശ്വാസികളായ ദാസന്‍മാരില്‍ പെട്ടവനാകുന്നു.
\end{malayalam}}
\flushright{\begin{Arabic}
\quranayah[37][112]
\end{Arabic}}
\flushleft{\begin{malayalam}
ഇഷാഖ് എന്ന മകന്‍റെ ജനനത്തെപ്പറ്റിയും അദ്ദേഹത്തിന് നാം സന്തോഷവാര്‍ത്ത അറിയിച്ചു. സദ്‌വൃത്തരില്‍ പെട്ട ഒരു പ്രവാചകന്‍ എന്ന നിലയില്‍.
\end{malayalam}}
\flushright{\begin{Arabic}
\quranayah[37][113]
\end{Arabic}}
\flushleft{\begin{malayalam}
അദ്ദേഹത്തിനും ഇഷാഖിനും നാം അനുഗ്രഹം നല്‍കുകയും ചെയ്തു. അവര്‍ ഇരുവരുടെയും സന്തതികളില്‍ സദ്‌വൃത്തരുണ്ട്‌. സ്വന്തത്തോട് തന്നെ സ്പഷ്ടമായ അന്യായം ചെയ്യുന്നവരുമുണ്ട്‌.
\end{malayalam}}
\flushright{\begin{Arabic}
\quranayah[37][114]
\end{Arabic}}
\flushleft{\begin{malayalam}
തീര്‍ച്ചയായും മൂസായോടും ഹാറൂനോടും നാം ഔദാര്യം കാണിച്ചു.
\end{malayalam}}
\flushright{\begin{Arabic}
\quranayah[37][115]
\end{Arabic}}
\flushleft{\begin{malayalam}
അവര്‍ ഇരുവരെയും അവരുടെ ജനതയെയും മഹാദുരിതത്തില്‍ നിന്ന് നാം രക്ഷപ്പെടുത്തുകയും ചെയ്തു.
\end{malayalam}}
\flushright{\begin{Arabic}
\quranayah[37][116]
\end{Arabic}}
\flushleft{\begin{malayalam}
അവരെ നാം സഹായിക്കുകയും അങ്ങനെ വിജയികള്‍ അവര്‍ തന്നെ ആകുകയും ചെയ്തു.
\end{malayalam}}
\flushright{\begin{Arabic}
\quranayah[37][117]
\end{Arabic}}
\flushleft{\begin{malayalam}
അവര്‍ക്ക് രണ്ടുപേര്‍ക്കും നാം (കാര്യങ്ങള്‍) വ്യക്തമാക്കുന്ന ഗ്രന്ഥം നല്‍കുകയും,
\end{malayalam}}
\flushright{\begin{Arabic}
\quranayah[37][118]
\end{Arabic}}
\flushleft{\begin{malayalam}
അവരെ നേരായ പാതയിലേക്ക് നയിക്കുകയും ചെയ്തു.
\end{malayalam}}
\flushright{\begin{Arabic}
\quranayah[37][119]
\end{Arabic}}
\flushleft{\begin{malayalam}
പില്‍ക്കാലക്കാരില്‍ അവരുടെ സല്‍കീര്‍ത്തി നാം അവശേഷിപ്പിക്കുകയും ചെയ്തു.
\end{malayalam}}
\flushright{\begin{Arabic}
\quranayah[37][120]
\end{Arabic}}
\flushleft{\begin{malayalam}
മൂസായ്ക്കും ഹാറൂന്നും സമാധാനം!
\end{malayalam}}
\flushright{\begin{Arabic}
\quranayah[37][121]
\end{Arabic}}
\flushleft{\begin{malayalam}
തീര്‍ച്ചയായും അപ്രകാരമാകുന്നു സദ്‌വൃത്തര്‍ക്ക് നാം പ്രതിഫലം നല്‍കുന്നത്‌.
\end{malayalam}}
\flushright{\begin{Arabic}
\quranayah[37][122]
\end{Arabic}}
\flushleft{\begin{malayalam}
തീര്‍ച്ചയായും അവര്‍ ഇരുവരും നമ്മുടെ സത്യവിശ്വാസികളായ ദാസന്‍മാരുടെ കൂട്ടത്തിലാകുന്നു.
\end{malayalam}}
\flushright{\begin{Arabic}
\quranayah[37][123]
\end{Arabic}}
\flushleft{\begin{malayalam}
ഇല്‍യാസും ദൂതന്‍മാരിലൊരാള്‍ തന്നെ.
\end{malayalam}}
\flushright{\begin{Arabic}
\quranayah[37][124]
\end{Arabic}}
\flushleft{\begin{malayalam}
അദ്ദേഹം തന്‍റെ ജനതയോട് ഇപ്രകാരം പറഞ്ഞ സന്ദര്‍ഭം: നിങ്ങള്‍ സൂക്ഷ്മത പാലിക്കുന്നില്ലേ?
\end{malayalam}}
\flushright{\begin{Arabic}
\quranayah[37][125]
\end{Arabic}}
\flushleft{\begin{malayalam}
നിങ്ങള്‍ ബഅ്ല‍ൈന്‍ വിളിച്ച് പ്രാര്‍ത്ഥിക്കുകയും, ഏറ്റവും നല്ല സൃഷ്ടികര്‍ത്താവിനെ വിട്ടുകളയുകയുമാണോ?
\end{malayalam}}
\flushright{\begin{Arabic}
\quranayah[37][126]
\end{Arabic}}
\flushleft{\begin{malayalam}
അഥവാ നിങ്ങളുടെയും നിങ്ങളുടെ പൂര്‍വ്വപിതാക്കളുടെയും രക്ഷിതാവായ അല്ലാഹുവെ.
\end{malayalam}}
\flushright{\begin{Arabic}
\quranayah[37][127]
\end{Arabic}}
\flushleft{\begin{malayalam}
അപ്പോള്‍ അവര്‍ അദ്ദേഹത്തെ നിഷേധിച്ചു കളഞ്ഞു. അതിനാല്‍ അവര്‍ (ശിക്ഷയ്ക്ക്‌) ഹാജരാക്കപ്പെടുക തന്നെ ചെയ്യും.
\end{malayalam}}
\flushright{\begin{Arabic}
\quranayah[37][128]
\end{Arabic}}
\flushleft{\begin{malayalam}
അല്ലാഹുവിന്‍റെ നിഷ്കളങ്കരായ ദാസന്‍മാര്‍ ഒഴികെ.
\end{malayalam}}
\flushright{\begin{Arabic}
\quranayah[37][129]
\end{Arabic}}
\flushleft{\begin{malayalam}
പില്‍ക്കാലക്കാരില്‍ അദ്ദേഹത്തിന്‍റെ സല്‍കീര്‍ത്തി നാം അവശേഷിപ്പിക്കുകയും ചെയ്തു.
\end{malayalam}}
\flushright{\begin{Arabic}
\quranayah[37][130]
\end{Arabic}}
\flushleft{\begin{malayalam}
ഇല്‍യാസിന് സമാധാനം!
\end{malayalam}}
\flushright{\begin{Arabic}
\quranayah[37][131]
\end{Arabic}}
\flushleft{\begin{malayalam}
തീര്‍ച്ചയായും അപ്രകാരമാകുന്നു സദ്‌വൃത്തര്‍ക്ക് നാം പ്രതിഫലം നല്‍കുന്നത്‌.
\end{malayalam}}
\flushright{\begin{Arabic}
\quranayah[37][132]
\end{Arabic}}
\flushleft{\begin{malayalam}
തീര്‍ച്ചയായും അദ്ദേഹം നമ്മുടെ സത്യവിശ്വാസികളായ ദാസന്‍മാരുടെ കൂട്ടത്തിലാകുന്നു.
\end{malayalam}}
\flushright{\begin{Arabic}
\quranayah[37][133]
\end{Arabic}}
\flushleft{\begin{malayalam}
ലൂത്വും ദൂതന്‍മാരിലൊരാള്‍ തന്നെ.
\end{malayalam}}
\flushright{\begin{Arabic}
\quranayah[37][134]
\end{Arabic}}
\flushleft{\begin{malayalam}
അദ്ദേഹത്തെയും അദ്ദേഹത്തിന്‍റെ ആളുകളേയും മുഴുവന്‍ നാം രക്ഷപ്പെടുത്തിയ സന്ദര്‍ഭം (ശ്രദ്ധേയമത്രെ).
\end{malayalam}}
\flushright{\begin{Arabic}
\quranayah[37][135]
\end{Arabic}}
\flushleft{\begin{malayalam}
പിന്‍മാറി നിന്നവരില്‍പ്പെട്ട ഒരു കിഴവിയൊഴികെ.
\end{malayalam}}
\flushright{\begin{Arabic}
\quranayah[37][136]
\end{Arabic}}
\flushleft{\begin{malayalam}
പിന്നെ മറ്റുള്ളവരെ നാം തകര്‍ത്തു കളഞ്ഞു.
\end{malayalam}}
\flushright{\begin{Arabic}
\quranayah[37][137]
\end{Arabic}}
\flushleft{\begin{malayalam}
തീര്‍ച്ചയായും നിങ്ങള്‍ രാവിലെ അവരുടെ അടുത്തു കൂടി കടന്നു പോവാറുണ്ട്‌.
\end{malayalam}}
\flushright{\begin{Arabic}
\quranayah[37][138]
\end{Arabic}}
\flushleft{\begin{malayalam}
രാത്രിയിലും. എന്നിട്ടും നിങ്ങള്‍ ചിന്തിച്ച് ഗ്രഹിക്കുന്നില്ലേ?
\end{malayalam}}
\flushright{\begin{Arabic}
\quranayah[37][139]
\end{Arabic}}
\flushleft{\begin{malayalam}
യൂനുസും ദൂതന്‍മാരിലൊരാള്‍ തന്നെ.
\end{malayalam}}
\flushright{\begin{Arabic}
\quranayah[37][140]
\end{Arabic}}
\flushleft{\begin{malayalam}
അദ്ദേഹം ഭാരം നിറച്ച കപ്പലിലേക്ക് ഒളിച്ചോടിയ സന്ദര്‍ഭം (ശ്രദ്ധേയമത്രെ).
\end{malayalam}}
\flushright{\begin{Arabic}
\quranayah[37][141]
\end{Arabic}}
\flushleft{\begin{malayalam}
എന്നിട്ട് അദ്ദേഹം (കപ്പല്‍ യാത്രക്കാരോടൊപ്പം) നറുക്കെടുപ്പില്‍ പങ്കെടുത്തു. അപ്പോള്‍ അദ്ദേഹം പരാജിതരുടെ കൂട്ടത്തിലായിപോയി.
\end{malayalam}}
\flushright{\begin{Arabic}
\quranayah[37][142]
\end{Arabic}}
\flushleft{\begin{malayalam}
അങ്ങനെ അദ്ദേഹം ആക്ഷേപത്തിന് അര്‍ഹനായിരിക്കെ ആ വന്‍മത്സ്യം അദ്ദേഹത്തെ വിഴുങ്ങി.
\end{malayalam}}
\flushright{\begin{Arabic}
\quranayah[37][143]
\end{Arabic}}
\flushleft{\begin{malayalam}
എന്നാല്‍ അദ്ദേഹം അല്ലാഹുവിന്‍റെ പരിശുദ്ധിയെ പ്രകീര്‍ത്തിക്കുന്നവരുടെ കൂട്ടത്തിലായിരുന്നില്ലെങ്കില്‍
\end{malayalam}}
\flushright{\begin{Arabic}
\quranayah[37][144]
\end{Arabic}}
\flushleft{\begin{malayalam}
ജനങ്ങള്‍ ഉയിര്‍ത്തെഴുന്നേല്‍പിക്കപ്പെടുന്ന ദിവസം വരെ അതിന്‍റെ വയറ്റില്‍ തന്നെ അദ്ദേഹത്തിന് കഴിഞ്ഞ് കൂടേണ്ടി വരുമായിരുന്നു.
\end{malayalam}}
\flushright{\begin{Arabic}
\quranayah[37][145]
\end{Arabic}}
\flushleft{\begin{malayalam}
എന്നിട്ട് അദ്ദേഹത്തെ അനാരോഗ്യവാനായ നിലയില്‍ തുറന്ന സ്ഥലത്തേക്ക് നാം തള്ളി
\end{malayalam}}
\flushright{\begin{Arabic}
\quranayah[37][146]
\end{Arabic}}
\flushleft{\begin{malayalam}
അദ്ദേഹത്തിന്‍റെ മേല്‍ നാം യഖ്ത്വീന്‍ വൃക്ഷം മുളപ്പിക്കുകയും ചെയ്തു.
\end{malayalam}}
\flushright{\begin{Arabic}
\quranayah[37][147]
\end{Arabic}}
\flushleft{\begin{malayalam}
അദ്ദേഹത്തെ നാം ഒരു ലക്ഷമോ അതിലധികമോ വരുന്ന ജനവിഭാഗത്തിലേക്ക് നിയോഗിച്ചു.
\end{malayalam}}
\flushright{\begin{Arabic}
\quranayah[37][148]
\end{Arabic}}
\flushleft{\begin{malayalam}
അങ്ങനെ അവര്‍ വിശ്വസിക്കുകയും തല്‍ഫലമായി കുറെ കാലത്തേക്ക് അവര്‍ക്ക് നാം സുഖജീവിതം നല്‍കുകയും ചെയ്തു.
\end{malayalam}}
\flushright{\begin{Arabic}
\quranayah[37][149]
\end{Arabic}}
\flushleft{\begin{malayalam}
എന്നാല്‍ (നബിയേ,) നീ അവരോട് (ബഹുദൈവവിശ്വാസികളോട്‌) അഭിപ്രായം ആരായുക; നിന്‍റെ രക്ഷിതാവിന് പെണ്‍മക്കളും അവര്‍ക്ക് ആണ്‍മക്കളുമാണോ എന്ന്‌.
\end{malayalam}}
\flushright{\begin{Arabic}
\quranayah[37][150]
\end{Arabic}}
\flushleft{\begin{malayalam}
അതല്ല നാം മലക്കുകളെ സ്ത്രീകളായി സൃഷ്ടിച്ചതിന് അവര്‍ ദൃക്സാക്ഷികളായിരുന്നോ?
\end{malayalam}}
\flushright{\begin{Arabic}
\quranayah[37][151]
\end{Arabic}}
\flushleft{\begin{malayalam}
അറിഞ്ഞേക്കുക: അവര്‍ പറയുന്നത് തീര്‍ച്ചയായും അവരുടെ വ്യാജനിര്‍മിതിയില്‍ പെട്ടതാകുന്നു.
\end{malayalam}}
\flushright{\begin{Arabic}
\quranayah[37][152]
\end{Arabic}}
\flushleft{\begin{malayalam}
അല്ലാഹു സന്തതികള്‍ക്കു ജന്‍മം നല്‍കിയിട്ടുണ്ടെന്ന്‌. തീര്‍ച്ചയായും അവര്‍ കള്ളം പറയുന്നവര്‍ തന്നെയാകുന്നു.
\end{malayalam}}
\flushright{\begin{Arabic}
\quranayah[37][153]
\end{Arabic}}
\flushleft{\begin{malayalam}
ആണ്‍മക്കളെക്കാളുപരിയായി അവന്‍ പെണ്‍മക്കളെ തെരഞ്ഞെടുത്തുവെന്നോ?
\end{malayalam}}
\flushright{\begin{Arabic}
\quranayah[37][154]
\end{Arabic}}
\flushleft{\begin{malayalam}
നിങ്ങള്‍ക്കെന്തുപറ്റി? എപ്രകാരമാണ് നിങ്ങള്‍ വിധികല്‍പിക്കുന്നത്‌?
\end{malayalam}}
\flushright{\begin{Arabic}
\quranayah[37][155]
\end{Arabic}}
\flushleft{\begin{malayalam}
നിങ്ങള്‍ ആലോചിച്ച് നോക്കുന്നില്ലേ?
\end{malayalam}}
\flushright{\begin{Arabic}
\quranayah[37][156]
\end{Arabic}}
\flushleft{\begin{malayalam}
അതല്ല, വ്യക്തമായ വല്ല പ്രമാണവും നിങ്ങള്‍ക്കു കിട്ടിയിട്ടുണ്ടോ?
\end{malayalam}}
\flushright{\begin{Arabic}
\quranayah[37][157]
\end{Arabic}}
\flushleft{\begin{malayalam}
എന്നാല്‍ നിങ്ങള്‍ നിങ്ങളുടെ രേഖ കൊണ്ടുവരുവിന്‍; നിങ്ങള്‍ സത്യവാന്‍മാരാണെങ്കില്‍.
\end{malayalam}}
\flushright{\begin{Arabic}
\quranayah[37][158]
\end{Arabic}}
\flushleft{\begin{malayalam}
അല്ലാഹുവിനും ജിന്നുകള്‍ക്കുമിടയില്‍ അവര്‍ കുടുംബബന്ധം സ്ഥാപിക്കുകയും ചെയ്തിരിക്കുന്നു. എന്നാല്‍ തീര്‍ച്ചയായും തങ്ങള്‍ ശിക്ഷയ്ക്ക് ഹാജരാക്കപ്പെടുക തന്നെ ചെയ്യുമെന്ന് ജിന്നുകള്‍ മനസ്സിലാക്കിയിട്ടുണ്ട്‌.
\end{malayalam}}
\flushright{\begin{Arabic}
\quranayah[37][159]
\end{Arabic}}
\flushleft{\begin{malayalam}
അവര്‍ ചമച്ചു പറയുന്നതില്‍ നിന്നെല്ലാം അല്ലാഹു എത്രയോ പരിശുദ്ധന്‍!
\end{malayalam}}
\flushright{\begin{Arabic}
\quranayah[37][160]
\end{Arabic}}
\flushleft{\begin{malayalam}
എന്നാല്‍ അല്ലാഹുവിന്‍റെ നിഷ്കളങ്കരായ ദാസന്‍മാര്‍ (ഇതില്‍ നിന്നെല്ലാം) ഒഴിവാകുന്നു.
\end{malayalam}}
\flushright{\begin{Arabic}
\quranayah[37][161]
\end{Arabic}}
\flushleft{\begin{malayalam}
എന്നാല്‍ നിങ്ങള്‍ക്കും നിങ്ങള്‍ എന്തിനെ ആരാധിക്കുന്നുവോ അവയ്ക്കും
\end{malayalam}}
\flushright{\begin{Arabic}
\quranayah[37][162]
\end{Arabic}}
\flushleft{\begin{malayalam}
അല്ലാഹുവിന്നെതിരായി (ആരെയും) കുഴപ്പത്തിലാക്കാനാവില്ല; തീര്‍ച്ച.
\end{malayalam}}
\flushright{\begin{Arabic}
\quranayah[37][163]
\end{Arabic}}
\flushleft{\begin{malayalam}
നരകത്തില്‍ വെന്തെരിയാന്‍ പോകുന്നവനാരോ അവനെയല്ലാതെ.
\end{malayalam}}
\flushright{\begin{Arabic}
\quranayah[37][164]
\end{Arabic}}
\flushleft{\begin{malayalam}
(മലക്കുകള്‍ ഇപ്രകാരം പറയും:) നിശ്ചിതമായ ഓരോ സ്ഥാനമുള്ളവരായിട്ടല്ലാതെ ഞങ്ങളില്‍ ആരും തന്നെയില്ല.
\end{malayalam}}
\flushright{\begin{Arabic}
\quranayah[37][165]
\end{Arabic}}
\flushleft{\begin{malayalam}
തീര്‍ച്ചയായും ഞങ്ങള്‍ തന്നെയാണ് അണിനിരന്ന് നില്‍ക്കുന്നവര്‍.
\end{malayalam}}
\flushright{\begin{Arabic}
\quranayah[37][166]
\end{Arabic}}
\flushleft{\begin{malayalam}
തീര്‍ച്ചയായും ഞങ്ങള്‍ തന്നെയാണ് (അല്ലാഹുവിന്‍റെ) പരിശുദ്ധിയെ പ്രകീര്‍ത്തിക്കുന്നവര്‍.
\end{malayalam}}
\flushright{\begin{Arabic}
\quranayah[37][167]
\end{Arabic}}
\flushleft{\begin{malayalam}
തീര്‍ച്ചയായും അവര്‍ (സത്യനിഷേധികള്‍) ഇപ്രകാരം പറയാറുണ്ടായിരുന്നു:
\end{malayalam}}
\flushright{\begin{Arabic}
\quranayah[37][168]
\end{Arabic}}
\flushleft{\begin{malayalam}
പൂര്‍വ്വികന്‍മാരില്‍ നിന്ന് ലഭിച്ച വല്ല ഉല്‍ബോധനവും ഞങ്ങളുടെ പക്കല്‍ ഉണ്ടായിരുന്നെങ്കില്‍
\end{malayalam}}
\flushright{\begin{Arabic}
\quranayah[37][169]
\end{Arabic}}
\flushleft{\begin{malayalam}
ഞങ്ങള്‍ അല്ലാഹുവിന്‍റെ നിഷ്കളങ്കരായ ദാസന്‍മാരാവുക തന്നെ ചെയ്യുമായിരുന്നു.
\end{malayalam}}
\flushright{\begin{Arabic}
\quranayah[37][170]
\end{Arabic}}
\flushleft{\begin{malayalam}
എന്നിട്ട് അവര്‍ ഇതില്‍ (ഈ വേദഗ്രന്ഥത്തില്‍) അവിശ്വസിക്കുകയാണ് ചെയ്തത്‌. അതിനാല്‍ അവര്‍ പിന്നീട് (കാര്യം) മനസ്സിലാക്കിക്കൊള്ളും.
\end{malayalam}}
\flushright{\begin{Arabic}
\quranayah[37][171]
\end{Arabic}}
\flushleft{\begin{malayalam}
ദൂതന്‍മാരായി നിയോഗിക്കപ്പെട്ട നമ്മുടെ ദാസന്‍മാരോട് നമ്മുടെ വചനം മുമ്പേ ഉണ്ടായിട്ടുണ്ട്‌.
\end{malayalam}}
\flushright{\begin{Arabic}
\quranayah[37][172]
\end{Arabic}}
\flushleft{\begin{malayalam}
തീര്‍ച്ചയായും അവര്‍ തന്നെയായിരിക്കും സഹായം നല്‍കപ്പെടുന്നവരെന്നും,
\end{malayalam}}
\flushright{\begin{Arabic}
\quranayah[37][173]
\end{Arabic}}
\flushleft{\begin{malayalam}
തീര്‍ച്ചയായും നമ്മുടെ സൈന്യം തന്നെയാണ് ജേതാക്കളായിരിക്കുക എന്നും.
\end{malayalam}}
\flushright{\begin{Arabic}
\quranayah[37][174]
\end{Arabic}}
\flushleft{\begin{malayalam}
അതിനാല്‍ ഒരു അവധി വരെ നീ അവരില്‍ നിന്ന് തിരിഞ്ഞുകളയുക.
\end{malayalam}}
\flushright{\begin{Arabic}
\quranayah[37][175]
\end{Arabic}}
\flushleft{\begin{malayalam}
നീ അവരെ വീക്ഷിക്കുകയും ചെയ്യുക. അവര്‍ പിന്നീട് കണ്ടറിഞ്ഞു കൊള്ളും.
\end{malayalam}}
\flushright{\begin{Arabic}
\quranayah[37][176]
\end{Arabic}}
\flushleft{\begin{malayalam}
അപ്പോള്‍ നമ്മുടെ ശിക്ഷയുടെ കാര്യത്തിലാണോ അവര്‍ തിടുക്കം കൂട്ടികൊണ്ടിരിക്കുന്നത്‌?
\end{malayalam}}
\flushright{\begin{Arabic}
\quranayah[37][177]
\end{Arabic}}
\flushleft{\begin{malayalam}
എന്നാല്‍ അത് അവരുടെ മുറ്റത്ത് വന്ന് ഇറങ്ങിയാല്‍ ആ താക്കീത് നല്‍കപ്പെട്ടവരുടെ പ്രഭാതം എത്ര മോശമായിരിക്കും!
\end{malayalam}}
\flushright{\begin{Arabic}
\quranayah[37][178]
\end{Arabic}}
\flushleft{\begin{malayalam}
(അതിനാല്‍) ഒരു അവധി വരെ നീ അവരില്‍ നിന്ന് തിരിഞ്ഞുകളയുക.
\end{malayalam}}
\flushright{\begin{Arabic}
\quranayah[37][179]
\end{Arabic}}
\flushleft{\begin{malayalam}
നീ വീക്ഷിച്ചു കൊണ്ടിരിക്കുകയും ചെയ്യുക. അവര്‍ പിന്നീട് കണ്ടറിഞ്ഞു കൊള്ളും.
\end{malayalam}}
\flushright{\begin{Arabic}
\quranayah[37][180]
\end{Arabic}}
\flushleft{\begin{malayalam}
പ്രതാപത്തിന്‍റെ നാഥനായ നിന്‍റെ രക്ഷിതാവ് അവര്‍ ചമച്ചു പറയുന്നതില്‍ നിന്നെല്ലാം എത്ര പരിശുദ്ധന്‍!
\end{malayalam}}
\flushright{\begin{Arabic}
\quranayah[37][181]
\end{Arabic}}
\flushleft{\begin{malayalam}
ദൂതന്‍മാര്‍ക്കു സമാധാനം!
\end{malayalam}}
\flushright{\begin{Arabic}
\quranayah[37][182]
\end{Arabic}}
\flushleft{\begin{malayalam}
ലോകരക്ഷിതാവായ അല്ലാഹുവിന് സ്തുതി!
\end{malayalam}}
\chapter{\textmalayalam{സ്വാദ്}}
\begin{Arabic}
\Huge{\centerline{\basmalah}}\end{Arabic}
\flushright{\begin{Arabic}
\quranayah[38][1]
\end{Arabic}}
\flushleft{\begin{malayalam}
സ്വാദ്‌- ഉല്‍ബോധനം ഉള്‍കൊള്ളുന്ന ഖുര്‍ആന്‍ തന്നെ സത്യം.
\end{malayalam}}
\flushright{\begin{Arabic}
\quranayah[38][2]
\end{Arabic}}
\flushleft{\begin{malayalam}
എന്നാല്‍ സത്യനിഷേധികള്‍ ദുരഭിമാനത്തിലും കക്ഷി മാത്സര്യത്തിലുമാകുന്നു.
\end{malayalam}}
\flushright{\begin{Arabic}
\quranayah[38][3]
\end{Arabic}}
\flushleft{\begin{malayalam}
അവര്‍ക്ക് മുമ്പ് എത്രയെത്ര തലമുറകളെ നാം നശിപ്പിച്ചു. അപ്പോള്‍ അവര്‍ മുറവിളികൂട്ടി. എന്നാല്‍ അത് രക്ഷപ്രാപിക്കാനുള്ള സമയമല്ല.
\end{malayalam}}
\flushright{\begin{Arabic}
\quranayah[38][4]
\end{Arabic}}
\flushleft{\begin{malayalam}
അവരില്‍ നിന്നുതന്നെയുള്ള ഒരു താക്കീതുകാരന്‍ അവരുടെ അടുത്തു വന്നതില്‍ അവര്‍ക്ക് ആശ്ചര്യം തോന്നിയിരിക്കുന്നു. സത്യനിഷേധികള്‍ പറഞ്ഞു: ഇവന്‍ കള്ളവാദിയായ ഒരു ജാലവിദ്യക്കാരനാകുന്നു
\end{malayalam}}
\flushright{\begin{Arabic}
\quranayah[38][5]
\end{Arabic}}
\flushleft{\begin{malayalam}
ഇവന്‍ പല ദൈവങ്ങളെ ഒരൊറ്റ ദൈവമാക്കിയിരിക്കുകയാണോ? തീര്‍ച്ചയായും ഇത് ഒരു അത്ഭുതകരമായ കാര്യം തന്നെ
\end{malayalam}}
\flushright{\begin{Arabic}
\quranayah[38][6]
\end{Arabic}}
\flushleft{\begin{malayalam}
അവരിലെ പ്രധാനികള്‍ (ഇപ്രകാരം പറഞ്ഞു കൊണ്ട്‌) പോയി: നിങ്ങള്‍ മുന്നോട്ട് പോയിക്കൊള്ളുക. നിങ്ങളുടെ ദൈവങ്ങളുടെ കാര്യത്തില്‍ നിങ്ങള്‍ ക്ഷമാപൂര്‍വ്വം ഉറച്ചുനില്‍ക്കുകയും ചെയ്യുക. തീര്‍ച്ചയായും ഇത് ഉദ്ദേശപൂര്‍വ്വം ചെയ്യപ്പെടുന്ന ഒരു കാര്യം തന്നെയാകുന്നു.
\end{malayalam}}
\flushright{\begin{Arabic}
\quranayah[38][7]
\end{Arabic}}
\flushleft{\begin{malayalam}
അവസാനത്തെ മതത്തില്‍ ഇതിനെ പറ്റി ഞങ്ങള്‍ കേള്‍ക്കുകയുണ്ടായിട്ടില്ല. ഇത് ഒരു കൃത്രിമ സൃഷ്ടി മാത്രമാകുന്നു.
\end{malayalam}}
\flushright{\begin{Arabic}
\quranayah[38][8]
\end{Arabic}}
\flushleft{\begin{malayalam}
ഞങ്ങളുടെ ഇടയില്‍ നിന്ന് ഉല്‍ബോധനം ഇറക്കപ്പെട്ടത് ഇവന്‍റെ മേലാണോ? അങ്ങനെയൊന്നുമല്ല. അവര്‍ എന്‍റെ ഉല്‍ബോധനത്തെപ്പറ്റി തന്നെ സംശയത്തിലാകുന്നു. അല്ല, അവര്‍ എന്‍റെ ശിക്ഷ ഇതുവരെ ആസ്വദിക്കുകയുണ്ടായിട്ടില്ല.
\end{malayalam}}
\flushright{\begin{Arabic}
\quranayah[38][9]
\end{Arabic}}
\flushleft{\begin{malayalam}
അതല്ല, പ്രതാപിയും അത്യുദാരനുമായ നിന്‍റെ രക്ഷിതാവിന്‍റെ കാരുണ്യത്തിന്‍റെ ഖജനാവുകള്‍ അവരുടെ പക്കലാണോ?
\end{malayalam}}
\flushright{\begin{Arabic}
\quranayah[38][10]
\end{Arabic}}
\flushleft{\begin{malayalam}
അതല്ല, ആകാശങ്ങളുടെയും ഭൂമിയുടെയും അവയ്ക്കിടയിലുള്ളതിന്‍റെയും ആധിപത്യം അവര്‍ക്കാണോ? എങ്കില്‍ ആ മാര്‍ഗങ്ങളിലൂടെ അവര്‍ കയറിനോക്കട്ടെ.
\end{malayalam}}
\flushright{\begin{Arabic}
\quranayah[38][11]
\end{Arabic}}
\flushleft{\begin{malayalam}
പല കക്ഷികളില്‍ പെട്ട പരാജയപ്പെടാന്‍ പോകുന്ന ഒരു സൈനികവ്യൂഹമത്രെ അവിടെയുള്ളത്‌.
\end{malayalam}}
\flushright{\begin{Arabic}
\quranayah[38][12]
\end{Arabic}}
\flushleft{\begin{malayalam}
അവര്‍ക്ക് മുമ്പ് നൂഹിന്‍റെ ജനതയും, ആദ് സമുദായവും, ആണികളുറപ്പിച്ചിരുന്ന ഫിര്‍ഔനും നിഷേധിച്ചു തള്ളിയിട്ടുണ്ട്‌,
\end{malayalam}}
\flushright{\begin{Arabic}
\quranayah[38][13]
\end{Arabic}}
\flushleft{\begin{malayalam}
ഥമൂദ് സമുദായവും, ലൂത്വിന്‍റെ ജനതയും, മരക്കൂട്ടങ്ങളില്‍ വസിച്ചിരുന്നവരും (സത്യത്തെ നിഷേധിച്ചു തള്ളിയിട്ടുണ്ട്‌.) അക്കൂട്ടരത്രെ (സത്യത്തിനെതിരില്‍ അണിനിരന്ന) കക്ഷികള്‍.
\end{malayalam}}
\flushright{\begin{Arabic}
\quranayah[38][14]
\end{Arabic}}
\flushleft{\begin{malayalam}
ഇവരാരും തന്നെ ദൂതന്‍മാരെ നിഷേധിച്ചു കളയാതിരുന്നിട്ടില്ല. അങ്ങനെ എന്‍റെ ശിക്ഷ (അവരില്‍) അനിവാര്യമായിത്തീര്‍ന്നു.
\end{malayalam}}
\flushright{\begin{Arabic}
\quranayah[38][15]
\end{Arabic}}
\flushleft{\begin{malayalam}
ഒരൊറ്റ ഘോരശബ്ദമല്ലാതെ മറ്റൊന്നും ഇക്കൂട്ടര്‍ നോക്കിയിരിക്കുന്നില്ല. (അതു സംഭവിച്ചു കഴിഞ്ഞാല്‍) ഒട്ടും സാവകാശമുണ്ടായിരിക്കുകയില്ല.
\end{malayalam}}
\flushright{\begin{Arabic}
\quranayah[38][16]
\end{Arabic}}
\flushleft{\begin{malayalam}
അവര്‍ പറയുന്നു: ഞങ്ങളുടെ രക്ഷിതാവേ, വിചാരണയുടെ ദിവസത്തിനു മുമ്പ് തന്നെ ഞങ്ങള്‍ക്കുള്ള (ശിക്ഷയുടെ) വിഹിതം ഞങ്ങള്‍ക്കൊന്നു വേഗത്തിലാക്കിതന്നേക്കണേ എന്ന്‌.
\end{malayalam}}
\flushright{\begin{Arabic}
\quranayah[38][17]
\end{Arabic}}
\flushleft{\begin{malayalam}
(നബിയേ,) അവര്‍ പറയുന്നതിനെപ്പറ്റി നീ ക്ഷമിച്ചു കൊള്ളുക. നമ്മുടെ കൈയ്യൂക്കുള്ള ദാസനായ ദാവൂദിനെ നീ അനുസ്മരിക്കുകയും ചെയ്യുക. തീര്‍ച്ചയായും അദ്ദേഹം (ദൈവത്തിങ്കലേക്ക്‌) ഏറ്റവും അധികം ഖേദിച്ചുമടങ്ങിയവനാകുന്നു.
\end{malayalam}}
\flushright{\begin{Arabic}
\quranayah[38][18]
\end{Arabic}}
\flushleft{\begin{malayalam}
സന്ധ്യാസമയത്തും, സൂര്യോദയ സമയത്തും സ്തോത്രകീര്‍ത്തനം നടത്തുന്ന നിലയില്‍ നാം പര്‍വ്വതങ്ങളെ അദ്ദേഹത്തോടൊപ്പം കീഴ്പെടുത്തുക തന്നെ ചെയ്തു.
\end{malayalam}}
\flushright{\begin{Arabic}
\quranayah[38][19]
\end{Arabic}}
\flushleft{\begin{malayalam}
ശേഖരിക്കപ്പെട്ട നിലയില്‍ പറവകളെയും (നാം കീഴ്പെടുത്തി.) എല്ലാം തന്നെ അദ്ദേഹത്തിങ്കലേക്ക് ഏറ്റവും അധികം വിനയത്തോടെ തിരിഞ്ഞവയായിരുന്നു.
\end{malayalam}}
\flushright{\begin{Arabic}
\quranayah[38][20]
\end{Arabic}}
\flushleft{\begin{malayalam}
അദ്ദേഹത്തിന്‍റെ ആധിപത്യം നാം സുശക്തമാക്കുകയും, അദ്ദേഹത്തിന് നാം തത്വജ്ഞാനവും തീര്‍പ്പുകല്‍പിക്കുവാന്‍ വേണ്ട സംസാരവൈഭവവും നല്‍കുകയും ചെയ്തു.
\end{malayalam}}
\flushright{\begin{Arabic}
\quranayah[38][21]
\end{Arabic}}
\flushleft{\begin{malayalam}
വഴക്ക് കൂടുന്ന കക്ഷികള്‍ പ്രാര്‍ത്ഥനാമണ്ഡപത്തിന്‍റെ മതില്‍ കയറിച്ചെന്ന സമയത്തെ വര്‍ത്തമാനം നിനക്ക് ലഭിച്ചിട്ടുണ്ടോ?
\end{malayalam}}
\flushright{\begin{Arabic}
\quranayah[38][22]
\end{Arabic}}
\flushleft{\begin{malayalam}
അവര്‍ ദാവൂദിന്‍റെ അടുത്ത് കടന്നു ചെല്ലുകയും, അദ്ദേഹം അവരെപ്പറ്റി പരിഭ്രാന്തനാകുകയും ചെയ്ത സന്ദര്‍ഭം! അവര്‍ പറഞ്ഞു. താങ്കള്‍ ഭയപ്പെടേണ്ട. ഞങ്ങള്‍ രണ്ട് എതിര്‍ കക്ഷികളാകുന്നു. ഞങ്ങളില്‍ ഒരു കക്ഷി മറുകക്ഷിയോട് അന്യായം ചെയ്തിരിക്കുന്നു. അതിനാല്‍ ഞങ്ങള്‍ക്കിടയില്‍ താങ്കള്‍ ന്യായപ്രകാരം വിധി കല്‍പിക്കണം. താങ്കള്‍ നീതികേട് കാണിക്കരുത്‌. ഞങ്ങള്‍ക്ക് നേരായ പാതയിലേക്ക് വഴി കാണിക്കണം.
\end{malayalam}}
\flushright{\begin{Arabic}
\quranayah[38][23]
\end{Arabic}}
\flushleft{\begin{malayalam}
ഇതാ, ഇവന്‍ എന്‍റെ സഹോദരനാകുന്നു. അവന്ന് തൊണ്ണൂറ്റി ഒമ്പതു പെണ്ണാടുകളുണ്ട്‌. എനിക്ക് ഒരു പെണ്ണാടും. എന്നിട്ട് അവന്‍ പറഞ്ഞു; അതിനെയും കൂടി എനിക്ക് ഏല്‍പിച്ചു തരണമെന്ന്‌. സംഭാഷണത്തില്‍ അവന്‍ എന്നെ തോല്‍പിച്ച് കളയുകയും ചെയ്തു.
\end{malayalam}}
\flushright{\begin{Arabic}
\quranayah[38][24]
\end{Arabic}}
\flushleft{\begin{malayalam}
അദ്ദേഹം (ദാവൂദ്‌) പറഞ്ഞു: തന്‍റെ പെണ്ണാടുകളുടെ കൂട്ടത്തിലേക്ക് നിന്‍റെ പെണ്ണാടിനെ കൂടി ആവശ്യപ്പെട്ടതു മുഖേന അവന്‍ നിന്നോട് അനീതി കാണിക്കുക തന്നെ ചെയ്തിരിക്കുന്നു. തീര്‍ച്ചയായും പങ്കാളികളില്‍ (കൂട്ടുകാരില്‍) പലരും പരസ്പരം അതിക്രമം കാണിക്കുകയാണ് ചെയ്യുന്നത്‌. വിശ്വസിക്കുകയും സല്‍കര്‍മ്മങ്ങള്‍ പ്രവര്‍ത്തിക്കുകയും ചെയ്തവരൊഴികെ. വളരെ കുറച്ച് പേരേയുള്ളു അത്തരക്കാര്‍. ദാവൂദ് വിചാരിച്ചു; നാം അദ്ദേഹത്തെ പരീക്ഷിക്കുക തന്നെയാണ് ചെയ്തതെന്ന്‌. തുടര്‍ന്ന് അദ്ദേഹം തന്‍റെ രക്ഷിതാവിനോട് പാപമോചനം തേടുകയും അദ്ദേഹം കുമ്പിട്ടു കൊണ്ട് വീഴുകയും ഖേദിച്ചുമടങ്ങുകയും ചെയ്തു.
\end{malayalam}}
\flushright{\begin{Arabic}
\quranayah[38][25]
\end{Arabic}}
\flushleft{\begin{malayalam}
അപ്പോള്‍ അദ്ദേഹത്തിന് നാം അത് പൊറുത്തുകൊടുത്തു. തീര്‍ച്ചയായും അദ്ദേഹത്തിന് നമ്മുടെ അടുക്കല്‍ സാമീപ്യവും മടങ്ങിവരാന്‍ ഉത്തമമായ സ്ഥാനവുമുണ്ട്‌
\end{malayalam}}
\flushright{\begin{Arabic}
\quranayah[38][26]
\end{Arabic}}
\flushleft{\begin{malayalam}
(അല്ലാഹു പറഞ്ഞു:) ഹേ; ദാവൂദ്‌, തീര്‍ച്ചയായും നിന്നെ നാം ഭൂമിയില്‍ ഒരു പ്രതിനിധിയാക്കിയിരിക്കുന്നു. ആകയാല്‍ ജനങ്ങള്‍ക്കിടയില്‍ ന്യായപ്രകാരം നീ വിധികല്‍പിക്കുക. തന്നിഷ്ടത്തെ നീ പിന്തുടര്‍ന്നു പോകരുത്‌. കാരണം അത് അല്ലാഹുവിന്‍റെ മാര്‍ഗത്തില്‍ നിന്ന് നിന്നെ തെറ്റിച്ചുകളയും. അല്ലാഹുവിന്‍റെ മാര്‍ഗത്തില്‍ നിന്ന് തെറ്റിപ്പോകുന്നവരാരോ അവര്‍ക്ക് തന്നെയാകുന്നു കഠിനമായ ശിക്ഷയുള്ളത്‌. കണക്കുനോക്കുന്ന ദിവസത്തെ അവര്‍ മറന്നുകളഞ്ഞതിന്‍റെ ഫലമത്രെ അത്‌.
\end{malayalam}}
\flushright{\begin{Arabic}
\quranayah[38][27]
\end{Arabic}}
\flushleft{\begin{malayalam}
ആകാശവും ഭൂമിയും അവയ്ക്കിടയിലുള്ളതും നാം നിരര്‍ത്ഥകമായി സൃഷ്ടിച്ചതല്ല. സത്യനിഷേധികളുടെ ധാരണയത്രെ അത്‌. ആകയാല്‍ സത്യനിഷേധികള്‍ക്ക് നരകശിക്ഷയാല്‍ മഹാനാശം!
\end{malayalam}}
\flushright{\begin{Arabic}
\quranayah[38][28]
\end{Arabic}}
\flushleft{\begin{malayalam}
അതല്ല, വിശ്വസിക്കുകയും സല്‍കര്‍മ്മങ്ങള്‍ പ്രവര്‍ത്തിക്കുകയും ചെയ്തവരെ ഭൂമിയില്‍ കുഴപ്പമുണ്ടാക്കുന്നവരെ പോലെ നാം ആക്കുമോ? അതല്ല, ധര്‍മ്മനിഷ്ഠ പാലിക്കുന്നവരെ ദുഷ്ടന്‍മാരെ പോലെ നാം ആക്കുമോ?
\end{malayalam}}
\flushright{\begin{Arabic}
\quranayah[38][29]
\end{Arabic}}
\flushleft{\begin{malayalam}
നിനക്ക് നാം അവതരിപ്പിച്ചുതന്ന അനുഗൃഹീത ഗ്രന്ഥമത്രെ ഇത്‌. ഇതിലെ ദൃഷ്ടാന്തങ്ങളെപ്പറ്റി അവര്‍ ചിന്തിച്ചു നോക്കുന്നതിനും ബുദ്ധിമാന്‍മാര്‍ ഉല്‍ബുദ്ധരാകേണ്ടതിനും വേണ്ടി.
\end{malayalam}}
\flushright{\begin{Arabic}
\quranayah[38][30]
\end{Arabic}}
\flushleft{\begin{malayalam}
ദാവൂദിന് നാം സുലൈമാനെ (പുത്രന്‍) പ്രദാനം ചെയ്തു. വളരെ നല്ല ദാസന്‍! തീര്‍ച്ചയായും അദ്ദേഹം (അല്ലാഹുവിങ്കലേക്ക്‌) ഏറ്റവും അധികം ഖേദിച്ചുമടങ്ങുന്നവനാകുന്നു.
\end{malayalam}}
\flushright{\begin{Arabic}
\quranayah[38][31]
\end{Arabic}}
\flushleft{\begin{malayalam}
കുതിച്ചോടാന്‍ തയ്യാറായി നില്‍ക്കുന്ന വിശിഷ്ടമായ കുതിരകള്‍ വൈകുന്നേരം അദ്ദേഹത്തിന്‍റെ മുമ്പില്‍ പ്രദര്‍ശിപ്പിക്കപ്പെട്ട സന്ദര്‍ഭം.
\end{malayalam}}
\flushright{\begin{Arabic}
\quranayah[38][32]
\end{Arabic}}
\flushleft{\begin{malayalam}
അപ്പോള്‍ അദ്ദേഹം പറഞ്ഞു: എന്‍റെ രക്ഷിതാവിന്‍റെ സ്മരണയുടെ അടിസ്ഥാനത്തിലാണ് ഐശ്വര്യത്തെ ഞാന്‍ സ്നേഹിച്ചിട്ടുള്ളത്‌. അങ്ങനെ അവ (കുതിരകള്‍) മറവില്‍ പോയി മറഞ്ഞു.
\end{malayalam}}
\flushright{\begin{Arabic}
\quranayah[38][33]
\end{Arabic}}
\flushleft{\begin{malayalam}
(അപ്പോള്‍ അദ്ദേഹം പറഞ്ഞു:) നിങ്ങള്‍ അവയെ എന്‍റെ അടുത്തേക്ക് തിരിച്ചു കൊണ്ട് വരൂ. എന്നിട്ട് അദ്ദേഹം (അവയുടെ) കണങ്കാലുകളിലും കഴുത്തുകളിലും തടവാന്‍ തുടങ്ങി.
\end{malayalam}}
\flushright{\begin{Arabic}
\quranayah[38][34]
\end{Arabic}}
\flushleft{\begin{malayalam}
സുലൈമാനെ നാം പരീക്ഷിക്കുകയുണ്ടായി. അദ്ദേഹത്തിന്‍റെ സിംഹാസനത്തിന്‍മേല്‍ നാം ഒരു ജഡത്തെ ഇടുകയും ചെയ്തു. പിന്നീട് അദ്ദേഹം താഴ്മയുള്ളവനായി മടങ്ങി.
\end{malayalam}}
\flushright{\begin{Arabic}
\quranayah[38][35]
\end{Arabic}}
\flushleft{\begin{malayalam}
അദ്ദേഹം പറഞ്ഞു. എന്‍റെ രക്ഷിതാവേ, നീ എനിക്ക് പൊറുത്തുതരികയും എനിക്ക് ശേഷം ഒരാള്‍ക്കും തരപ്പെടാത്ത ഒരു രാജവാഴ്ച നീ എനിക്ക് പ്രദാനം ചെയ്യുകയും ചെയ്യേണമേ. തീര്‍ച്ചയായും നീ തന്നെയാണ് ഏറ്റവും വലിയ ദാനശീലന്‍.
\end{malayalam}}
\flushright{\begin{Arabic}
\quranayah[38][36]
\end{Arabic}}
\flushleft{\begin{malayalam}
അപ്പോള്‍ അദ്ദേഹത്തിന് കാറ്റിനെ നാം കീഴ്പെടുത്തികൊടുത്തു. അദ്ദേഹത്തിന്‍റെ കല്‍പന പ്രകാരം അദ്ദേഹം ലക്ഷ്യമാക്കിയേടത്തേക്ക് സൌമ്യമായ നിലയില്‍ അത് സഞ്ചരിക്കുന്നു.
\end{malayalam}}
\flushright{\begin{Arabic}
\quranayah[38][37]
\end{Arabic}}
\flushleft{\begin{malayalam}
എല്ലാ കെട്ടിടനിര്‍മാണ വിദഗ്ദ്ധരും മുങ്ങല്‍ വിദഗ്ദ്ധരുമായ പിശാചുക്കളെയും (അദ്ദേഹത്തിന്നു കീഴ്പെടുത്തികൊടുത്തു.)
\end{malayalam}}
\flushright{\begin{Arabic}
\quranayah[38][38]
\end{Arabic}}
\flushleft{\begin{malayalam}
ചങ്ങലകളില്‍ ബന്ധിക്കപ്പെട്ട മറ്റു ചിലരെ (പിശാചുക്കളെ)യും (അധീനപ്പെടുത്തികൊടുത്തു.)
\end{malayalam}}
\flushright{\begin{Arabic}
\quranayah[38][39]
\end{Arabic}}
\flushleft{\begin{malayalam}
ഇത് നമ്മുടെ ദാനമാകുന്നു. ആകയാല്‍ നീ ഔദാര്യം ചെയ്യുകയോ കൈവശം വെച്ച് കൊള്ളുകയോ ചെയ്യുക. കണക്കു ചോദിക്കല്‍ ഉണ്ടാവില്ല. (എന്ന് നാം സുലൈമാനോട് പറയുകയും ചെയ്തു.)
\end{malayalam}}
\flushright{\begin{Arabic}
\quranayah[38][40]
\end{Arabic}}
\flushleft{\begin{malayalam}
തീര്‍ച്ചയായും അദ്ദേഹത്തിന് നമ്മുടെ അടുക്കല്‍ സാമീപ്യമുണ്ട്‌. മടങ്ങിയെത്താന്‍ ഉത്തമമായ സ്ഥാനവും.
\end{malayalam}}
\flushright{\begin{Arabic}
\quranayah[38][41]
\end{Arabic}}
\flushleft{\begin{malayalam}
നമ്മുടെ ദാസനായ അയ്യൂബിനെ ഓര്‍മിക്കുക. പിശാച് എനിക്ക് അവശതയും പീഡനവും ഏല്‍പിച്ചിരിക്കുന്നു എന്ന് തന്‍റെ രക്ഷിതാവിനെ വിളിച്ച് അദ്ദേഹം പറഞ്ഞ സന്ദര്‍ഭം.
\end{malayalam}}
\flushright{\begin{Arabic}
\quranayah[38][42]
\end{Arabic}}
\flushleft{\begin{malayalam}
(നാം നിര്‍ദേശിച്ചു:) നിന്‍റെ കാലുകൊണ്ട് നീ ചവിട്ടുക, ഇതാ! തണുത്ത സ്നാനജലവും കുടിനീരും
\end{malayalam}}
\flushright{\begin{Arabic}
\quranayah[38][43]
\end{Arabic}}
\flushleft{\begin{malayalam}
അദ്ദേഹത്തിന് അദ്ദേഹത്തിന്‍റെ സ്വന്തക്കാരെയും അവരോടൊപ്പം അവരുടെ അത്ര ആളുകളെയും നാം പ്രദാനം ചെയ്യുകയും ചെയ്തു. നമ്മുടെ പക്കല്‍ നിന്നുള്ള കാരുണ്യവും ബുദ്ധിമാന്‍മാര്‍ക്ക് ഒരു ഉല്‍ബോധനവുമെന്ന നിലയില്‍.
\end{malayalam}}
\flushright{\begin{Arabic}
\quranayah[38][44]
\end{Arabic}}
\flushleft{\begin{malayalam}
നീ ഒരു പിടി പുല്ല് നിന്‍റെ കൈയില്‍ എടുക്കുക. എന്നിട്ട് അതു കൊണ്ട് നീ അടിക്കുകയും ശപഥം ലംഘിക്കാതിരിക്കുകയും ചെയ്യുക. തീര്‍ച്ചയായും അദ്ദേഹത്തെ നാം ക്ഷമാശീലനായി കണ്ടു. വളരെ നല്ല ദാസന്‍! തീര്‍ച്ചയായും അദ്ദേഹം ഏറെ ഖേദിച്ചുമടങ്ങുന്നവനാകുന്നു.
\end{malayalam}}
\flushright{\begin{Arabic}
\quranayah[38][45]
\end{Arabic}}
\flushleft{\begin{malayalam}
കൈക്കരുത്തും കാഴ്ചപ്പാടുകളും ഉള്ളവരായിരുന്ന നമ്മുടെ ദാസന്‍മാരായ ഇബ്രാഹീം, ഇഷാഖ്‌, യഅ്ഖൂബ് എന്നിവരെയും ഓര്‍ക്കുക.
\end{malayalam}}
\flushright{\begin{Arabic}
\quranayah[38][46]
\end{Arabic}}
\flushleft{\begin{malayalam}
നിഷ്കളങ്കമായ ഒരു വിചാരം കൊണ്ട് നാം അവരെ ഉല്‍കൃഷ്ടരാക്കിയിരിക്കുന്നു. പരലോക സ്മരണയത്രെ അത്‌.
\end{malayalam}}
\flushright{\begin{Arabic}
\quranayah[38][47]
\end{Arabic}}
\flushleft{\begin{malayalam}
തീര്‍ച്ചയായും അവര്‍ നമ്മുടെ അടുക്കല്‍ തെരഞ്ഞെടുക്കപ്പെട്ട ഉത്തമന്‍മാരില്‍ പെട്ടവരാകുന്നു.
\end{malayalam}}
\flushright{\begin{Arabic}
\quranayah[38][48]
\end{Arabic}}
\flushleft{\begin{malayalam}
ഇസ്മാഈല്‍, അല്‍യസഅ്‌, ദുല്‍കിഫ്ല് എന്നിവരെയും ഓര്‍ക്കുക. അവരെല്ലാവരും ഉത്തമന്‍മാരില്‍ പെട്ടവരാകുന്നു.
\end{malayalam}}
\flushright{\begin{Arabic}
\quranayah[38][49]
\end{Arabic}}
\flushleft{\begin{malayalam}
ഇതൊരു ഉല്‍ബോധനമത്രെ. തീര്‍ച്ചയായും സൂക്ഷ്മതയുള്ളവര്‍ക്ക് മടങ്ങിച്ചെല്ലാന്‍ ഉത്തമമായ സ്ഥാനമുണ്ട്‌.
\end{malayalam}}
\flushright{\begin{Arabic}
\quranayah[38][50]
\end{Arabic}}
\flushleft{\begin{malayalam}
അവര്‍ക്ക് വേണ്ടി കവാടങ്ങള്‍ തുറന്നുവെച്ചിട്ടുള്ള സ്ഥിരവാസത്തിന്‍റെ സ്വര്‍ഗത്തോപ്പുകള്‍.
\end{malayalam}}
\flushright{\begin{Arabic}
\quranayah[38][51]
\end{Arabic}}
\flushleft{\begin{malayalam}
അവര്‍ അവിടെ ചാരി ഇരുന്നു വിശ്രമിച്ചു കൊണ്ട് സമൃദ്ധമായുള്ള ഫലവര്‍ഗങ്ങള്‍ക്കും പാനീയത്തിനും ആവശ്യപ്പെട്ട് കൊണ്ടിരിക്കും.
\end{malayalam}}
\flushright{\begin{Arabic}
\quranayah[38][52]
\end{Arabic}}
\flushleft{\begin{malayalam}
അവരുടെ അടുത്ത് ദൃഷ്ടി നിയന്ത്രിക്കുന്ന സമവയസ്ക്കരായ സ്ത്രീകളുണ്ടായിരിക്കും.
\end{malayalam}}
\flushright{\begin{Arabic}
\quranayah[38][53]
\end{Arabic}}
\flushleft{\begin{malayalam}
ഇതത്രെ വിചാരണയുടെ ദിവസത്തേക്ക് നിങ്ങള്‍ക്ക് വാഗ്ദാനം നല്‍കപ്പെടുന്നത്‌.
\end{malayalam}}
\flushright{\begin{Arabic}
\quranayah[38][54]
\end{Arabic}}
\flushleft{\begin{malayalam}
തീര്‍ച്ചയായും ഇത് നാം നല്‍കുന്ന ഉപജീവനമാകുന്നു. അത് തീര്‍ന്നു പോകുന്നതല്ല.
\end{malayalam}}
\flushright{\begin{Arabic}
\quranayah[38][55]
\end{Arabic}}
\flushleft{\begin{malayalam}
ഇതത്രെ (അവരുടെ അവസ്ഥ). തീര്‍ച്ചയായും ധിക്കാരികള്‍ക്ക് മടങ്ങിച്ചെല്ലാന്‍ മോശപ്പെട്ട സ്ഥാനമാണുള്ളത്‌.
\end{malayalam}}
\flushright{\begin{Arabic}
\quranayah[38][56]
\end{Arabic}}
\flushleft{\begin{malayalam}
നരകമത്രെ അത്‌. അവര്‍ അതില്‍ കത്തിഎരിയും. അതെത്ര മോശമായ വിശ്രമസ്ഥാനം!
\end{malayalam}}
\flushright{\begin{Arabic}
\quranayah[38][57]
\end{Arabic}}
\flushleft{\begin{malayalam}
ഇതാണവര്‍ക്കുള്ളത്‌. ആകയാല്‍ അവര്‍ അത് ആസ്വദിച്ചു കൊള്ളട്ടെ. കൊടും ചൂടുള്ള വെള്ളവും കൊടും തണുപ്പുള്ള വെള്ളവും.
\end{malayalam}}
\flushright{\begin{Arabic}
\quranayah[38][58]
\end{Arabic}}
\flushleft{\begin{malayalam}
ഇത്തരത്തില്‍പ്പെട്ട മറ്റു പല ഇനം ശിക്ഷകളും.
\end{malayalam}}
\flushright{\begin{Arabic}
\quranayah[38][59]
\end{Arabic}}
\flushleft{\begin{malayalam}
(നരകത്തില്‍ ആദ്യമെത്തിയവരോട് അല്ലാഹു പറയും:) ഇതാ, ഒരുകൂട്ടം നിങ്ങളോടൊപ്പം തള്ളിക്കയറി വരുന്നു. (അപ്പോള്‍ അവര്‍ പറയും:) അവര്‍ക്ക് സ്വാഗതമില്ല. തീര്‍ച്ചയായും അവര്‍ നരകത്തില്‍ കത്തിഎരിയുന്നവരത്രെ.
\end{malayalam}}
\flushright{\begin{Arabic}
\quranayah[38][60]
\end{Arabic}}
\flushleft{\begin{malayalam}
അവര്‍ (ആ കടന്ന് വരുന്നവര്‍) പറയും; അല്ല, നിങ്ങള്‍ക്ക് തന്നെയാണ് സ്വാഗതമില്ലാത്തത്‌. നിങ്ങളാണ് ഞങ്ങള്‍ക്കിത് വരുത്തിവെച്ചത്‌. അപ്പോള്‍ വാസസ്ഥലം ചീത്ത തന്നെ.
\end{malayalam}}
\flushright{\begin{Arabic}
\quranayah[38][61]
\end{Arabic}}
\flushleft{\begin{malayalam}
അവര്‍ പറയും: ഞങ്ങളുടെ രക്ഷിതാവേ, ഞങ്ങള്‍ക്ക് ഇത് (ശിക്ഷ) വരുത്തിവെച്ചതാരോ അവന്ന് നീ നരകത്തില്‍ ഇരട്ടി ശിക്ഷ വര്‍ദ്ധിപ്പിച്ചു കൊടുക്കേണമേ.
\end{malayalam}}
\flushright{\begin{Arabic}
\quranayah[38][62]
\end{Arabic}}
\flushleft{\begin{malayalam}
അവര്‍ പറയും: നമുക്കെന്തു പറ്റി! ദുര്‍ജനങ്ങളില്‍ പെട്ടവരായി നാം ഗണിച്ചിരുന്ന പല ആളുകളെയും നാം കാണുന്നില്ലല്ലോ.
\end{malayalam}}
\flushright{\begin{Arabic}
\quranayah[38][63]
\end{Arabic}}
\flushleft{\begin{malayalam}
നാം അവരെ (അബദ്ധത്തില്‍) പരിഹാസപാത്രമാക്കിയതാണോ? അതല്ല, അവരെയും വിട്ട് കണ്ണുകള്‍ തെന്നിപ്പോയതാണോ?
\end{malayalam}}
\flushright{\begin{Arabic}
\quranayah[38][64]
\end{Arabic}}
\flushleft{\begin{malayalam}
നരകവാസികള്‍ തമ്മിലുള്ള വഴക്ക്‌- തീര്‍ച്ചയായും അതൊരു യാഥാര്‍ത്ഥ്യം തന്നെയാണ്‌.
\end{malayalam}}
\flushright{\begin{Arabic}
\quranayah[38][65]
\end{Arabic}}
\flushleft{\begin{malayalam}
(നബിയേ,) പറയുക: ഞാനൊരു മുന്നറിയിപ്പുകാരന്‍ മാത്രമാണ്‌. ഏകനും സര്‍വ്വാധിപതിയുമായ അല്ലാഹുവല്ലാതെ യാതൊരു ദൈവവുമില്ല.
\end{malayalam}}
\flushright{\begin{Arabic}
\quranayah[38][66]
\end{Arabic}}
\flushleft{\begin{malayalam}
ആകാശങ്ങളുടെയും ഭൂമിയുടെയും അവയ്ക്കിടയിലുള്ളതിന്‍റെയും രക്ഷിതാവും പ്രതാപശാലിയും ഏറെ പൊറുക്കുന്നവനുമത്രെ അവന്‍.
\end{malayalam}}
\flushright{\begin{Arabic}
\quranayah[38][67]
\end{Arabic}}
\flushleft{\begin{malayalam}
പറയുക: അത് ഒരു ഗൌരവമുള്ള വര്‍ത്തമാനമാകുന്നു.
\end{malayalam}}
\flushright{\begin{Arabic}
\quranayah[38][68]
\end{Arabic}}
\flushleft{\begin{malayalam}
നിങ്ങള്‍ അത് അവഗണിച്ചു കളയുന്നവരാകുന്നു.
\end{malayalam}}
\flushright{\begin{Arabic}
\quranayah[38][69]
\end{Arabic}}
\flushleft{\begin{malayalam}
അത്യുന്നത സമൂഹം വിവാദം നടത്തിയിരുന്ന സന്ദര്‍ഭത്തെപ്പറ്റി എനിക്ക് യാതൊരു അറിവും ഉണ്ടായിരുന്നില്ല.
\end{malayalam}}
\flushright{\begin{Arabic}
\quranayah[38][70]
\end{Arabic}}
\flushleft{\begin{malayalam}
ഞാനൊരു സ്പഷ്ടമായ താക്കീതുകാരനായിരിക്കേണ്ടതിന് മാത്രമാണ് എനിക്ക് സന്ദേശം നല്‍കപ്പെടുന്നത്‌.
\end{malayalam}}
\flushright{\begin{Arabic}
\quranayah[38][71]
\end{Arabic}}
\flushleft{\begin{malayalam}
നിന്‍റെ രക്ഷിതാവ് മലക്കുകളോട് പറഞ്ഞ സന്ദര്‍ഭം: തീര്‍ച്ചയായും ഞാന്‍ കളിമണ്ണില്‍ നിന്നും ഒരു മനുഷ്യനെ സൃഷ്ടിക്കാന്‍ പോകുകയാണ്‌.
\end{malayalam}}
\flushright{\begin{Arabic}
\quranayah[38][72]
\end{Arabic}}
\flushleft{\begin{malayalam}
അങ്ങനെ ഞാന്‍ അവനെ സംവിധാനിക്കുകയും, അവനില്‍ എന്‍റെ ആത്മാവില്‍ നിന്ന് ഞാന്‍ ഊതുകയും ചെയ്താല്‍ നിങ്ങള്‍ അവന്ന് പ്രണാമം ചെയ്യുന്നവരായി വീഴണം.
\end{malayalam}}
\flushright{\begin{Arabic}
\quranayah[38][73]
\end{Arabic}}
\flushleft{\begin{malayalam}
അപ്പോള്‍ മലക്കുകള്‍ എല്ലാവരും ഒന്നടങ്കം പ്രണാമം ചെയ്തു;
\end{malayalam}}
\flushright{\begin{Arabic}
\quranayah[38][74]
\end{Arabic}}
\flushleft{\begin{malayalam}
ഇബ്ലീസ് ഒഴികെ. അവന്‍ അഹങ്കരിക്കുകയും സത്യനിഷേധികളുടെ കൂട്ടത്തിലാകുകയും ചെയ്തു.
\end{malayalam}}
\flushright{\begin{Arabic}
\quranayah[38][75]
\end{Arabic}}
\flushleft{\begin{malayalam}
അവന്‍ (അല്ലാഹു) പറഞ്ഞു: ഇബ്ലീസേ, എന്‍റെ കൈകൊണ്ട് ഞാന്‍ സൃഷ്ടിച്ചുണ്ടാക്കിയതിനെ നീ പ്രണമിക്കുന്നതിന് നിനക്കെന്ത് തടസ്സമാണുണ്ടായത്‌? നീ അഹങ്കരിച്ചിരിക്കുകയാണോ, അതല്ല നീ പൊങ്ങച്ചക്കാരുടെ കൂട്ടത്തില്‍ പെട്ടിരിക്കുകയാണോ?
\end{malayalam}}
\flushright{\begin{Arabic}
\quranayah[38][76]
\end{Arabic}}
\flushleft{\begin{malayalam}
അവന്‍ (ഇബ്ലീസ്‌) പറഞ്ഞു: ഞാന്‍ അവനെ (മനുഷ്യനെ)ക്കാള്‍ ഉത്തമനാകുന്നു. എന്നെ നീ അഗ്നിയില്‍ നിന്ന് സൃഷ്ടിച്ചു. അവനെ നീ കളിമണ്ണില്‍ നിന്നും സൃഷ്ടിച്ചു.
\end{malayalam}}
\flushright{\begin{Arabic}
\quranayah[38][77]
\end{Arabic}}
\flushleft{\begin{malayalam}
അവന്‍ (അല്ലാഹു) പറഞ്ഞു: എന്നാല്‍ നീ ഇവിടെ നിന്ന് പുറത്ത് പോകണം. തീര്‍ച്ചയായും നീ ആട്ടിയകറ്റപ്പെട്ടവനാകുന്നു.
\end{malayalam}}
\flushright{\begin{Arabic}
\quranayah[38][78]
\end{Arabic}}
\flushleft{\begin{malayalam}
തീര്‍ച്ചയായും ന്യായവിധിയുടെ നാള്‍ വരെയും നിന്‍റെ മേല്‍ എന്‍റെ ശാപം ഉണ്ടായിരിക്കുന്നതാണ്‌.
\end{malayalam}}
\flushright{\begin{Arabic}
\quranayah[38][79]
\end{Arabic}}
\flushleft{\begin{malayalam}
അവന്‍ (ഇബ്ലീസ്‌) പറഞ്ഞു: എന്‍റെ രക്ഷിതാവേ, എന്നാല്‍ അവര്‍ ഉയിര്‍ത്തെഴുന്നേല്‍പിക്കപ്പെടുന്ന ദിവസം വരെ നീ എനിക്ക് അവധി അനുവദിച്ചു തരേണമേ.
\end{malayalam}}
\flushright{\begin{Arabic}
\quranayah[38][80]
\end{Arabic}}
\flushleft{\begin{malayalam}
(അല്ലാഹു) പറഞ്ഞു: എന്നാല്‍ നീ അവധി അനുവദിക്കപ്പെട്ടവരുടെ കൂട്ടത്തില്‍ തന്നെയാകുന്നു.
\end{malayalam}}
\flushright{\begin{Arabic}
\quranayah[38][81]
\end{Arabic}}
\flushleft{\begin{malayalam}
നിശ്ചിതമായ ആ സമയം സമാഗതമാകുന്ന ദിവസം വരെ.
\end{malayalam}}
\flushright{\begin{Arabic}
\quranayah[38][82]
\end{Arabic}}
\flushleft{\begin{malayalam}
അവന്‍ (ഇബ്ലീസ്‌) പറഞ്ഞു: നിന്‍റെ പ്രതാപമാണ സത്യം; അവരെ മുഴുവന്‍ ഞാന്‍ വഴിതെറ്റിക്കുക തന്നെ ചെയ്യും.
\end{malayalam}}
\flushright{\begin{Arabic}
\quranayah[38][83]
\end{Arabic}}
\flushleft{\begin{malayalam}
അവരില്‍ നിന്‍റെ നിഷ്കളങ്കരായ ദാസന്‍മാരൊഴികെ
\end{malayalam}}
\flushright{\begin{Arabic}
\quranayah[38][84]
\end{Arabic}}
\flushleft{\begin{malayalam}
അവന്‍ (അല്ലാഹു) പറഞ്ഞു: അപ്പോള്‍ സത്യം ഇതത്രെ- സത്യമേ ഞാന്‍ പറയുകയുള്ളൂ-
\end{malayalam}}
\flushright{\begin{Arabic}
\quranayah[38][85]
\end{Arabic}}
\flushleft{\begin{malayalam}
നിന്നെയും അവരില്‍ നിന്ന് നിന്നെ പിന്തുടര്‍ന്ന മുഴുവന്‍ പേരെയും കൊണ്ട് ഞാന്‍ നരകം നിറക്കുക തന്നെ ചെയ്യും.
\end{malayalam}}
\flushright{\begin{Arabic}
\quranayah[38][86]
\end{Arabic}}
\flushleft{\begin{malayalam}
(നബിയേ,) പറയുക: ഇതിന്‍റെ പേരില്‍ നിങ്ങളോട് ഞാന്‍ യാതൊരു പ്രതിഫലവും ചോദിക്കുന്നില്ല. ഞാന്‍ കൃത്രിമം കെട്ടിച്ചമയ്ക്കുന്നവരുടെ കൂട്ടത്തിലുമല്ല.
\end{malayalam}}
\flushright{\begin{Arabic}
\quranayah[38][87]
\end{Arabic}}
\flushleft{\begin{malayalam}
ഇത് ലോകര്‍ക്കുള്ള ഒരു ഉല്‍ബോധനം മാത്രമാകുന്നു.
\end{malayalam}}
\flushright{\begin{Arabic}
\quranayah[38][88]
\end{Arabic}}
\flushleft{\begin{malayalam}
ഒരു കാലയളവിനു ശേഷം ഇതിലെ വൃത്താന്തം നിങ്ങള്‍ക്ക് മനസ്സിലാവുക തന്നെ ചെയ്യും.
\end{malayalam}}
\chapter{\textmalayalam{സുമര്‍ ( കൂട്ടങ്ങള്‍ )}}
\begin{Arabic}
\Huge{\centerline{\basmalah}}\end{Arabic}
\flushright{\begin{Arabic}
\quranayah[39][1]
\end{Arabic}}
\flushleft{\begin{malayalam}
ഈ ഗ്രന്ഥത്തിന്‍റെ അവതരണം പ്രതാപിയും യുക്തിമാനുമായ അല്ലാഹുവിങ്കല്‍ നിന്നാകുന്നു.
\end{malayalam}}
\flushright{\begin{Arabic}
\quranayah[39][2]
\end{Arabic}}
\flushleft{\begin{malayalam}
തീര്‍ച്ചയായും നിനക്ക് നാം ഈ ഗ്രന്ഥം അവതരിപ്പിച്ചു തന്നത് സത്യപ്രകാരമാകുന്നു. അതിനാല്‍ കീഴ്‌വണക്കം അല്ലാഹുവിന് നിഷ്കളങ്കമാക്കികൊണ്ട് അവനെ നീ ആരാധിക്കുക.
\end{malayalam}}
\flushright{\begin{Arabic}
\quranayah[39][3]
\end{Arabic}}
\flushleft{\begin{malayalam}
അറിയുക: അല്ലാഹുവിന് മാത്രം അവകാശപ്പെട്ടതാകുന്നു നിഷ്കളങ്കമായ കീഴ്‌വണക്കം. അവന്നു പുറമെ രക്ഷാധികാരികളെ സ്വീകരിച്ചവര്‍ (പറയുന്നു:) അല്ലാഹുവിങ്കലേക്ക് ഞങ്ങള്‍ക്ക് കൂടുതല്‍ അടുപ്പമുണ്ടാക്കിത്തരാന്‍ വേണ്ടിമാത്രമാകുന്നു ഞങ്ങള്‍ അവരെ ആരാധിക്കുന്നത്‌. അവര്‍ ഏതൊരു കാര്യത്തില്‍ ഭിന്നത പുലര്‍ത്തുന്നുവോ അതില്‍ അല്ലാഹു അവര്‍ക്കിടയില്‍ വിധികല്‍പിക്കുക തന്നെ ചെയ്യും. നുണയനും നന്ദികെട്ടവനുമായിട്ടുള്ളവനാരോ അവനെ അല്ലാഹു നേര്‍വഴിയിലാക്കുകയില്ല; തീര്‍ച്ച.
\end{malayalam}}
\flushright{\begin{Arabic}
\quranayah[39][4]
\end{Arabic}}
\flushleft{\begin{malayalam}
ഒരു സന്താനത്തെ സ്വീകരിക്കണമെന്ന് അല്ലാഹു ഉദ്ദേശിച്ചിരുന്നെങ്കില്‍ അവന്‍ സൃഷ്ടിക്കുന്നതില്‍ നിന്ന് അവന്‍ ഇഷ്ടപ്പെടുന്നത് അവന്‍ തെരഞ്ഞെടുക്കുമായിരുന്നു. അവന്‍ എത്ര പരിശുദ്ധന്‍! ഏകനും സര്‍വ്വാധിപതിയുമായ അല്ലാഹുവത്രെ അവന്‍.
\end{malayalam}}
\flushright{\begin{Arabic}
\quranayah[39][5]
\end{Arabic}}
\flushleft{\begin{malayalam}
ആകാശങ്ങളും ഭൂമിയും അവന്‍ യാഥാര്‍ത്ഥ്യപൂര്‍വ്വം സൃഷ്ടിച്ചിരിക്കുന്നു. രാത്രിയെ ക്കൊണ്ട് അവന്‍ പകലിന്‍മേല്‍ ചുറ്റിപ്പൊതിയുന്നു. പകലിനെക്കൊണ്ട് അവന്‍ രാത്രിമേലും ചുറ്റിപ്പൊതിയുന്നു. സൂര്യനെയും ചന്ദ്രനെയും അവന്‍ നിയന്ത്രണവിധേയമാക്കുകയും ചെയ്തിരിക്കുന്നു. എല്ലാം നിശ്ചിതമായ പരിധിവരെ സഞ്ചരിക്കുന്നു. അറിയുക: അവനത്രെ പ്രതാപിയും ഏറെ പൊറുക്കുന്നവനും.
\end{malayalam}}
\flushright{\begin{Arabic}
\quranayah[39][6]
\end{Arabic}}
\flushleft{\begin{malayalam}
ഒരൊറ്റ അസ്തിത്വത്തില്‍ നിന്ന് അവന്‍ നിങ്ങളെ സൃഷ്ടിച്ചു. പിന്നീട് അതില്‍ നിന്ന് അതിന്‍റെ ഇണയെയും അവന്‍ ഉണ്ടാക്കി. കന്നുകാലികളില്‍ നിന്ന് എട്ടു ജോഡികളെയും അവന്‍ നിങ്ങള്‍ക്ക് ഇറക്കിതന്നു. നിങ്ങളുടെ മാതാക്കളുടെ വയറുകളില്‍ നിങ്ങളെ അവന്‍ സൃഷ്ടിക്കുന്നു. മൂന്ന് തരം അന്ധകാരങ്ങള്‍ക്കുള്ളില്‍ സൃഷ്ടിയുടെ ഒരു ഘട്ടത്തിന് ശേഷം മറ്റൊരു ഘട്ടമായിക്കൊണ്ട്‌. അങ്ങനെയുള്ളവനാകുന്നു നിങ്ങളുടെ രക്ഷിതാവായ അല്ലാഹു. അവന്നാണ് ആധിപത്യം. അവനല്ലാതെ യാതൊരു ദൈവവുമില്ല. എന്നിരിക്കെ നിങ്ങള്‍ എങ്ങനെയാണ് (സത്യത്തില്‍ നിന്ന്‌) തെറ്റിക്കപ്പെടുന്നത്‌?
\end{malayalam}}
\flushright{\begin{Arabic}
\quranayah[39][7]
\end{Arabic}}
\flushleft{\begin{malayalam}
നിങ്ങള്‍ നന്ദികേട് കാണിക്കുകയാണെങ്കില്‍ തീര്‍ച്ചയായും അല്ലാഹു നിങ്ങളുടെ ആശ്രയത്തില്‍ നിന്ന് മുക്തനാകുന്നു. തന്‍റെ ദാസന്‍മാര്‍ നന്ദികേട് കാണിക്കുന്നത് അവന്‍ തൃപ്തിപ്പെടുകയില്ല. നിങ്ങള്‍ നന്ദികാണിക്കുന്ന പക്ഷം നിങ്ങളോട് അത് വഴി അവന്‍ സംതൃപ്തനായിരിക്കുന്നതാണ്‌. പാപഭാരം വഹിക്കുന്ന യാതൊരാളും മറ്റൊരാളുടെ ഭാരം വഹിക്കുകയില്ല. പിന്നീട് നിങ്ങളുടെ രക്ഷിതാവിങ്കലേക്കാകുന്നു നിങ്ങളുടെ മടക്കം. നിങ്ങള്‍ പ്രവര്‍ത്തിച്ചിരുന്നതിനെ പറ്റി അപ്പോള്‍ അവന്‍ നിങ്ങളെ വിവരം അറിയിക്കുന്നതാണ്‌. തീര്‍ച്ചയായും അവന്‍ ഹൃദയങ്ങളിലുള്ളതിനെ പറ്റി അറിവുള്ളവനാകുന്നു.
\end{malayalam}}
\flushright{\begin{Arabic}
\quranayah[39][8]
\end{Arabic}}
\flushleft{\begin{malayalam}
മനുഷ്യന് വല്ല വിഷമവും ബാധിച്ചാല്‍ അവന്‍ തന്‍റെ രക്ഷിതാവിങ്കലേക്ക് താഴ്മയോടെ മടങ്ങിക്കൊണ്ട് പ്രാര്‍ത്ഥിക്കും. എന്നിട്ട് തന്‍റെ പക്കല്‍ നിന്നുള്ള വല്ല അനുഗ്രഹവും അല്ലാഹു അവന്ന് പ്രദാനം ചെയ്താല്‍ ഏതൊന്നിനായി അവന്‍ മുമ്പ് പ്രാര്‍ത്ഥിച്ചിരുന്നുവോ അതവന്‍ മറന്നുപോകുന്നു. അല്ലാഹുവിന്‍റെ മാര്‍ഗത്തില്‍ നിന്ന് വഴിതെറ്റിച്ച് കളയുവാന്‍ വേണ്ടി അവന്ന് സമന്‍മാരെ സ്ഥാപിക്കുകയും ചെയ്യുന്നു. (നബിയേ,) പറയുക: നീ അല്‍പകാലം നിന്‍റെ ഈ സത്യനിഷേധവും കൊണ്ട് സുഖിച്ചു കൊള്ളുക. തീര്‍ച്ചയായും നീ നരകാവകാശികളുടെ കൂട്ടത്തിലാകുന്നു.
\end{malayalam}}
\flushright{\begin{Arabic}
\quranayah[39][9]
\end{Arabic}}
\flushleft{\begin{malayalam}
അതല്ല, പരലോകത്തെ പറ്റി ജാഗ്രത പുലര്‍ത്തുകയും, തന്‍റെ രക്ഷിതാവിന്‍റെ കാരുണ്യം ആശിക്കുകയും ചെയ്തു കൊണ്ട് സാഷ്ടാംഗം ചെയ്തും, നിന്നു പ്രാര്‍ത്ഥിച്ചും രാത്രി സമയങ്ങളില്‍ കീഴ്‌വണക്കം ചെയ്യുന്നവനോ (അതല്ല സത്യനിഷേധിയോ ഉത്തമന്‍?) പറയുക: അറിവുള്ളവരും അറിവില്ലാത്തവരും സമമാകുമോ? ബുദ്ധിമാന്‍മാര്‍ മാത്രമേ ആലോചിച്ചു മനസ്സിലാക്കുകയുള്ളൂ.
\end{malayalam}}
\flushright{\begin{Arabic}
\quranayah[39][10]
\end{Arabic}}
\flushleft{\begin{malayalam}
പറയുക: വിശ്വസിച്ചവരായ എന്‍റെ ദാസന്‍മാരേ, നിങ്ങള്‍ നിങ്ങളുടെ രക്ഷിതാവിനെ സൂക്ഷിക്കുക. ഈ ഐഹികജീവിതത്തില്‍ നന്‍മ പ്രവര്‍ത്തിച്ചവര്‍ക്കാണ് സല്‍ഫലമുള്ളത്‌. അല്ലാഹുവിന്‍റെ ഭൂമിയാകട്ടെ വിശാലമാകുന്നു. ക്ഷമാശീലര്‍ക്കു തന്നെയാകുന്നു തങ്ങളുടെ പ്രതിഫലം കണക്കുനോക്കാതെ നിറവേറ്റികൊടുക്കപ്പെടുന്നത്‌.
\end{malayalam}}
\flushright{\begin{Arabic}
\quranayah[39][11]
\end{Arabic}}
\flushleft{\begin{malayalam}
പറയുക: കീഴ്‌വണക്കം അല്ലാഹുവിന് നിഷ്കളങ്കമാക്കിക്കൊണ്ട് അവനെ ആരാധിക്കുവാനാണ് ഞാന്‍ കല്‍പിക്കപ്പെട്ടിട്ടുള്ളത്‌
\end{malayalam}}
\flushright{\begin{Arabic}
\quranayah[39][12]
\end{Arabic}}
\flushleft{\begin{malayalam}
ഞാന്‍ കീഴ്പെടുന്നവരില്‍ ഒന്നാമനായിരിക്കണമെന്നും എനിക്ക് കല്‍പന നല്‍കപ്പെട്ടിരിക്കുന്നു.
\end{malayalam}}
\flushright{\begin{Arabic}
\quranayah[39][13]
\end{Arabic}}
\flushleft{\begin{malayalam}
പറയുക: ഞാന്‍ എന്‍റെ രക്ഷിതാവിനെ ധിക്കരിക്കുന്ന പക്ഷം ഭയങ്കരമായ ഒരു ദിവസത്തെ ശിക്ഷ തീര്‍ച്ചയായും ഞാന്‍ പേടിക്കുന്നു.
\end{malayalam}}
\flushright{\begin{Arabic}
\quranayah[39][14]
\end{Arabic}}
\flushleft{\begin{malayalam}
പറയുക: അല്ലാഹുവെയാണ് ഞാന്‍ ആരാധിക്കുന്നത്‌. ; എന്‍റെ കീഴ്‌വണക്കം അവന്ന് നിഷ്കളങ്കമാക്കിക്കൊണ്ട്‌.
\end{malayalam}}
\flushright{\begin{Arabic}
\quranayah[39][15]
\end{Arabic}}
\flushleft{\begin{malayalam}
എന്നാല്‍ നിങ്ങള്‍ അവന്നു പുറമെ നിങ്ങള്‍ ഉദ്ദേശിച്ചതിന് ആരാധന ചെയ്തുകൊള്ളുക. പറയുക: ഉയിര്‍ത്തെഴുന്നേല്‍പിന്‍റെ നാളില്‍ സ്വദേഹങ്ങള്‍ക്കും തങ്ങളുടെ ആളുകള്‍ക്കും നഷ്ടം വരുത്തിവെച്ചതാരോ അവരത്രെ തീര്‍ച്ചയായും നഷ്ടക്കാര്‍. അതു തന്നെയാണ് വ്യക്തമായ നഷ്ടം
\end{malayalam}}
\flushright{\begin{Arabic}
\quranayah[39][16]
\end{Arabic}}
\flushleft{\begin{malayalam}
അവര്‍ക്ക് അവരുടെ മുകള്‍ ഭാഗത്ത് തിയ്യിന്‍റെ തട്ടുകളുണ്ട്‌. അവരുടെ കീഴ്ഭാഗത്തുമുണ്ട് തട്ടുകള്‍. അതിനെ പറ്റിയാകുന്നു അല്ലാഹു തന്‍റെ ദാസന്‍മാരെ ഭയപ്പെടുത്തുന്നത്‌. ആകയാല്‍ എന്‍റെ ദാസന്‍മാരേ, നിങ്ങള്‍ എന്നെ സൂക്ഷിക്കുവിന്‍.
\end{malayalam}}
\flushright{\begin{Arabic}
\quranayah[39][17]
\end{Arabic}}
\flushleft{\begin{malayalam}
ദുര്‍മൂര്‍ത്തിയെ -അതിനെ ആരാധിക്കുന്നത്‌- വര്‍ജ്ജിക്കുകയും, അല്ലാഹുവിലേക്ക് വിനയത്തോടെ മടങ്ങുകയും ചെയ്തവരാരോ അവര്‍ക്കാണ് സന്തോഷവാര്‍ത്ത. അതിനാല്‍ എന്‍റെ ദാസന്‍മാര്‍ക്ക് നീ സന്തോഷവാര്‍ത്ത അറിയിക്കുക.
\end{malayalam}}
\flushright{\begin{Arabic}
\quranayah[39][18]
\end{Arabic}}
\flushleft{\begin{malayalam}
അതായത് വാക്ക് ശ്രദ്ധിച്ചു കേള്‍ക്കുകയും അതില്‍ ഏറ്റവും നല്ലത് പിന്‍പറ്റുകയും ചെയ്യുന്നവര്‍ക്ക് .അക്കൂട്ടര്‍ക്കാകുന്നു അല്ലാഹു മാര്‍ഗദര്‍ശനം നല്‍കിയിട്ടുള്ളത്‌. അവര്‍ തന്നെയാകുന്നു ബുദ്ധിമാന്‍മാര്‍.
\end{malayalam}}
\flushright{\begin{Arabic}
\quranayah[39][19]
\end{Arabic}}
\flushleft{\begin{malayalam}
അപ്പോള്‍ വല്ലവന്‍റെ കാര്യത്തിലും ശിക്ഷയുടെ വചനം സ്ഥിരപ്പെട്ടു കഴിഞ്ഞിട്ടുണ്ടെങ്കിലും (അവനെ നിനക്ക് സഹായിക്കാനാകുമോ?) അപ്പോള്‍ നരകത്തിലുള്ളവനെ നിനക്ക് രക്ഷപ്പെടുത്താനാകുമോ?
\end{malayalam}}
\flushright{\begin{Arabic}
\quranayah[39][20]
\end{Arabic}}
\flushleft{\begin{malayalam}
പക്ഷെ, തങ്ങളുടെ രക്ഷിതാവിനെ സൂക്ഷിച്ച് ജീവിച്ചവരാരോ അവര്‍ക്കാണ് മേല്‍ക്കുമേല്‍ തട്ടുകളായി നിര്‍മിക്കപ്പെട്ടിട്ടുള്ള മണിമേടകളുള്ളത്‌. അവയുടെ താഴ്ഭാഗത്തു കൂടി അരുവികള്‍ ഒഴുകികൊണ്ടിരിക്കുന്നു. അല്ലാഹുവിന്‍റെ വാഗ്ദാനമത്രെ അത്‌. അല്ലാഹു വാഗ്ദാനം ലംഘിക്കുകയില്ല.
\end{malayalam}}
\flushright{\begin{Arabic}
\quranayah[39][21]
\end{Arabic}}
\flushleft{\begin{malayalam}
നീ കണ്ടില്ലേ, അല്ലാഹു ആകാശത്തു നിന്ന് വെള്ളം ചൊരിഞ്ഞു. എന്നിട്ട് ഭൂമിയിലെ ഉറവിടങ്ങളില്‍ അതവന്‍ പ്രവേശിപ്പിച്ചു. അനന്തരം അത് മുഖേന വ്യത്യസ്ത വര്‍ണങ്ങളിലുള്ള വിള അവന്‍ ഉല്‍പാദിപ്പിക്കുന്നു. പിന്നെ അത് ഉണങ്ങിപോകുന്നു. അപ്പോള്‍ അത് മഞ്ഞനിറം പൂണ്ടതായി നിനക്ക് കാണാം. പിന്നീട് അവന്‍ അതിനെ വൈക്കോല്‍ തുരുമ്പാക്കുന്നു. തീര്‍ച്ചയായും അതില്‍ ബുദ്ധിമാന്‍മാര്‍ക്ക് ഒരു ഗുണപാഠമുണ്ട്‌.
\end{malayalam}}
\flushright{\begin{Arabic}
\quranayah[39][22]
\end{Arabic}}
\flushleft{\begin{malayalam}
അപ്പോള്‍ ഏതൊരാളുടെ ഹൃദയത്തിന് ഇസ്ലാം സ്വീകരിക്കാന്‍ അല്ലാഹു വിശാലത നല്‍കുകയും അങ്ങനെ അവന്‍ തന്‍റെ രക്ഷിതാവിങ്കല്‍ നിന്നുള്ള പ്രകാശത്തിലായിരിക്കുകയും ചെയ്തുവോ (അവന്‍ ഹൃദയം കടുത്തുപോയവനെപ്പേലെയാണോ?) എന്നാല്‍ അല്ലാഹുവിന്‍റെ സ്മരണയില്‍ നിന്ന് അകന്ന് ഹൃദയങ്ങള്‍ കടുത്തുപോയവര്‍ക്കാകുന്നു നാശം. അത്തരക്കാര്‍ വ്യക്തമായ ദുര്‍മാര്‍ഗത്തിലത്രെ.
\end{malayalam}}
\flushright{\begin{Arabic}
\quranayah[39][23]
\end{Arabic}}
\flushleft{\begin{malayalam}
അല്ലാഹുവാണ് ഏറ്റവും ഉത്തമമായ വര്‍ത്തമാനം അവതരിപ്പിച്ചിരിക്കുന്നത്‌. അഥവാ വചനങ്ങള്‍ക്ക് പരസ്പരം സാമ്യമുള്ളതും ആവര്‍ത്തിക്കപ്പെടുന്ന വചനങ്ങളുള്ളതുമായ ഒരു ഗ്രന്ഥം. തങ്ങളുടെ രക്ഷിതാവിനെ ഭയപ്പെടുന്നവരുടെ ചര്‍മ്മങ്ങള്‍ അതു നിമിത്തം രോമാഞ്ചമണിയുന്നു. പിന്നീട് അവരുടെ ചര്‍മ്മങ്ങളും ഹൃദയങ്ങളും അല്ലാഹുവെ സ്മരിക്കുന്നതിനായി വിനീതമാവുകയും ചെയ്യുന്നു. അതത്രെ അല്ലാഹുവിന്‍റെ മാര്‍ഗദര്‍ശനം. അതുമുഖേന താന്‍ ഉദ്ദേശിക്കുന്നവരെ അവന്‍ നേര്‍വഴിയിലാക്കുന്നു. വല്ലവനെയും അവന്‍ പിഴവിലാക്കുന്ന പക്ഷം അവന് വഴി കാട്ടാന്‍ ആരും തന്നെയില്ല.
\end{malayalam}}
\flushright{\begin{Arabic}
\quranayah[39][24]
\end{Arabic}}
\flushleft{\begin{malayalam}
എന്നാല്‍ ഉയിര്‍ത്തെഴുന്നേല്‍പിന്‍റെ നാളില്‍ കടുത്ത ശിക്ഷയെ സ്വന്തം മുഖം കൊണ്ട് തടുക്കേണ്ടിവരുന്ന ഒരാള്‍ (അന്ന് നിര്‍ഭയനായിരിക്കുന്നവനെ പോലെ ആകുമോ?) നിങ്ങള്‍ സമ്പാദിച്ചു കൊണ്ടിരിക്കുന്നത്‌, നിങ്ങള്‍ ആസ്വദിച്ചു കൊള്ളുക. എന്ന് അക്രമികളോട് പറയപ്പെടുകയും ചെയ്യും.
\end{malayalam}}
\flushright{\begin{Arabic}
\quranayah[39][25]
\end{Arabic}}
\flushleft{\begin{malayalam}
അവര്‍ക്ക് മുമ്പുള്ളവരും സത്യത്തെ നിഷേധിച്ചു കളഞ്ഞു. അപ്പോള്‍ അവര്‍ അറിയാത്ത ഭാഗത്ത്കൂടി അവര്‍ക്ക് ശിക്ഷ വന്നെത്തി.
\end{malayalam}}
\flushright{\begin{Arabic}
\quranayah[39][26]
\end{Arabic}}
\flushleft{\begin{malayalam}
അങ്ങനെ ഐഹികജീവിതത്തില്‍ അല്ലാഹു അവര്‍ക്ക് അപമാനം ആസ്വദിപ്പിച്ചു. പരലോകശിക്ഷ തന്നെയാകുന്നു ഏറ്റവും ഗുരുതരമായത്‌. അവര്‍ അത് മനസ്സിലാക്കിയിരുന്നെങ്കില്‍!
\end{malayalam}}
\flushright{\begin{Arabic}
\quranayah[39][27]
\end{Arabic}}
\flushleft{\begin{malayalam}
തീര്‍ച്ചയായും ഈ ഖുര്‍ആനില്‍ ജനങ്ങള്‍ക്ക് വേണ്ടി നാം എല്ലാവിധത്തിലുമുള്ള ഉപമകള്‍ വിവരിച്ചിട്ടുണ്ട്‌; അവര്‍ ആലോചിച്ച് മനസ്സിലാക്കുവാന്‍ വേണ്ടി.
\end{malayalam}}
\flushright{\begin{Arabic}
\quranayah[39][28]
\end{Arabic}}
\flushleft{\begin{malayalam}
അതെ, ഒട്ടും വക്രതയുള്ളതല്ലാത്ത, അറബിഭാഷയിലുള്ള ഒരു ഖുര്‍ആന്‍. അവര്‍ സൂക്ഷ്മത പാലിക്കുവാന്‍ വേണ്ടി.
\end{malayalam}}
\flushright{\begin{Arabic}
\quranayah[39][29]
\end{Arabic}}
\flushleft{\begin{malayalam}
അല്ലാഹു ഇതാ ഒരു മനുഷ്യനെ ഉപമയായി എടുത്തുകാണിച്ചിരിക്കുന്നു. പരസ്പരം വഴക്കടിക്കുന്ന ഏതാനും പങ്കുകാരാണ് അവന്‍റെ യജമാനന്‍മാര്‍. ഒരു യജമാനന് മാത്രം കീഴ്പെടേണ്ടവനായ മറ്റൊരാളെയും (ഉപമയായി എടുത്തുകാണിച്ചിരിക്കുന്നു.) ഉപമയില്‍ ഇവര്‍ രണ്ടുപേരും ഒരുപോലെയാകുമോ? അല്ലാഹുവിന് സ്തുതി. പക്ഷെ അവരില്‍ അധികപേരും അറിയുന്നില്ല.
\end{malayalam}}
\flushright{\begin{Arabic}
\quranayah[39][30]
\end{Arabic}}
\flushleft{\begin{malayalam}
തീര്‍ച്ചയായും നീ മരിക്കുന്നവനാകുന്നു. അവരും മരിക്കുന്നവരാകുന്നു.
\end{malayalam}}
\flushright{\begin{Arabic}
\quranayah[39][31]
\end{Arabic}}
\flushleft{\begin{malayalam}
പിന്നീട് നിങ്ങള്‍ ഉയിര്‍ത്തെഴുന്നേല്‍പിന്‍റെ നാളില്‍ നിങ്ങളുടെ രക്ഷിതാവിന്‍റെ അടുക്കല്‍ വെച്ച് വഴക്ക് കൂടുന്നതാണ്‌.
\end{malayalam}}
\flushright{\begin{Arabic}
\quranayah[39][32]
\end{Arabic}}
\flushleft{\begin{malayalam}
അപ്പോള്‍ അല്ലാഹുവിന്‍റെ പേരില്‍ കള്ളം പറയുകയും, സത്യം തനിക്ക് വന്നെത്തിയപ്പോള്‍ അതിനെ നിഷേധിച്ചു തള്ളുകയും ചെയ്തവനെക്കാള്‍ കടുത്ത അക്രമി ആരുണ്ട്‌? നരകത്തിലല്ലയോ സത്യനിഷേധികള്‍ക്കുള്ള പാര്‍പ്പിടം?
\end{malayalam}}
\flushright{\begin{Arabic}
\quranayah[39][33]
\end{Arabic}}
\flushleft{\begin{malayalam}
സത്യവും കൊണ്ട് വരുകയും അതില്‍ വിശ്വസിക്കുകയും ചെയ്തതാരോ അത്തരക്കാര്‍ തന്നെയാകുന്നു സൂക്ഷ്മത പാലിച്ചവര്‍.
\end{malayalam}}
\flushright{\begin{Arabic}
\quranayah[39][34]
\end{Arabic}}
\flushleft{\begin{malayalam}
അവര്‍ക്ക് തങ്ങളുടെ രക്ഷിതാവിങ്കല്‍ അവര്‍ ഉദ്ദേശിക്കുന്നതെന്തോ അതുണ്ടായിരിക്കും. അതത്രെ സദ്‌വൃത്തര്‍ക്കുള്ള പ്രതിഫലം.
\end{malayalam}}
\flushright{\begin{Arabic}
\quranayah[39][35]
\end{Arabic}}
\flushleft{\begin{malayalam}
അവര്‍ പ്രവര്‍ത്തിച്ചതില്‍ നിന്ന് ഏറ്റവും ചീത്തയായതു പോലും അല്ലാഹു അവരില്‍ നിന്ന് മായ്ച്ചുകളയും. അവര്‍ പ്രവര്‍ത്തിച്ചതില്‍ ഏറ്റവും ഉത്തമമായതനുസരിച്ച് അവര്‍ക്കവന്‍ പ്രതിഫലം നല്‍കുകയും ചെയ്യും.
\end{malayalam}}
\flushright{\begin{Arabic}
\quranayah[39][36]
\end{Arabic}}
\flushleft{\begin{malayalam}
തന്‍റെ ദാസന്ന് അല്ലാഹു മതിയായവനല്ലയോ? അവന്ന് പുറമെയുള്ളവരെ പറ്റി അവര്‍ നിന്നെ പേടിപ്പിക്കുന്നു. വല്ലവനെയും അല്ലാഹു പിഴവിലാക്കുന്ന പക്ഷം അവന്ന് വഴി കാട്ടാന്‍ ആരുമില്ല.
\end{malayalam}}
\flushright{\begin{Arabic}
\quranayah[39][37]
\end{Arabic}}
\flushleft{\begin{malayalam}
വല്ലവനെയും അല്ലാഹു നേര്‍വഴിയിലാക്കുന്ന പക്ഷം അവനെ വഴിപിഴപ്പിക്കുവാനും ആരുമില്ല. അല്ലാഹു പ്രതാപിയും ശിക്ഷാനടപടി എടുക്കുന്നവനും അല്ലയോ?
\end{malayalam}}
\flushright{\begin{Arabic}
\quranayah[39][38]
\end{Arabic}}
\flushleft{\begin{malayalam}
ആകാശങ്ങളും ഭൂമിയും സൃഷ്ടിച്ചത് ആരെന്ന് നീ അവരോട് ചോദിക്കുന്ന പക്ഷം തീര്‍ച്ചയായും അവര്‍ പറയും: അല്ലാഹു എന്ന്‌. നീ പറയുക: എങ്കില്‍ അല്ലാഹുവിന് പുറമെ നിങ്ങള്‍ വിളിച്ച് പ്രാര്‍ത്ഥിക്കുന്നവയെപ്പറ്റി നിങ്ങള്‍ ചിന്തിച്ച് നോക്കിയിട്ടുണ്ടോ? എനിക്ക് വല്ല ഉപദ്രവവും വരുത്താന്‍ അല്ലാഹു ഉദ്ദേശിച്ചിട്ടുണ്ടെങ്കില്‍ അവയ്ക്ക് അവന്‍റെ ഉപദ്രവം നീക്കം ചെയ്യാനാവുമോ? അല്ലെങ്കില്‍ അവന്‍ എനിക്ക് വല്ല അനുഗ്രഹവും ചെയ്യുവാന്‍ ഉദ്ദേശിച്ചാല്‍ അവയ്ക്ക് അവന്‍റെ അനുഗ്രഹം പിടിച്ചു വെക്കാനാകുമോ? പറയുക: എനിക്ക് അല്ലാഹു മതി. അവന്‍റെ മേലാകുന്നു ഭരമേല്‍പിക്കുന്നവര്‍ ഭരമേല്‍പിക്കുന്നത്‌.
\end{malayalam}}
\flushright{\begin{Arabic}
\quranayah[39][39]
\end{Arabic}}
\flushleft{\begin{malayalam}
പറയുക: എന്‍റെ ജനങ്ങളേ, നിങ്ങളുടെ നിലപാടനുസരിച്ച് നിങ്ങള്‍ പ്രവര്‍ത്തിച്ചുകൊള്ളുക. ഞാനും പ്രവര്‍ത്തിച്ചുകൊണ്ടിരിക്കുക തന്നെയാകുന്നു. എന്നാല്‍ വഴിയെ നിങ്ങള്‍ക്ക് അറിയുമാറാകും;
\end{malayalam}}
\flushright{\begin{Arabic}
\quranayah[39][40]
\end{Arabic}}
\flushleft{\begin{malayalam}
അപമാനകരമായ ശിക്ഷ വന്നെത്തുന്നതും, ശാശ്വതമായ ശിക്ഷ വന്നിറങ്ങുന്നതും ആര്‍ക്കാണെന്ന്‌.
\end{malayalam}}
\flushright{\begin{Arabic}
\quranayah[39][41]
\end{Arabic}}
\flushleft{\begin{malayalam}
തീര്‍ച്ചയായും നാം മനുഷ്യര്‍ക്ക് വേണ്ടി സത്യപ്രകാരമുള്ള വേദഗ്രന്ഥം നിന്‍റെ മേല്‍ ഇറക്കിത്തന്നിരിക്കുന്നു. ആകയാല്‍ വല്ലവനും സന്‍മാര്‍ഗം സ്വീകരിച്ചാല്‍ അത് അവന്‍റെ ഗുണത്തിന് തന്നെയാണ്‌. വല്ലവനും വഴിപിഴച്ചു പോയാല്‍ അവന്‍ വഴിപിഴച്ചു പോകുന്നതിന്‍റെ ദോഷവും അവന് തന്നെ. നീ അവരുടെ മേല്‍ കൈകാര്യകര്‍ത്താവൊന്നുമല്ല.
\end{malayalam}}
\flushright{\begin{Arabic}
\quranayah[39][42]
\end{Arabic}}
\flushleft{\begin{malayalam}
ആത്മാവുകളെ അവയുടെ മരണവേളയില്‍ അല്ലാഹു പൂര്‍ണ്ണമായി ഏറ്റെടുക്കുന്നു. മരണപ്പെടാത്തവയെ അവയുടെ ഉറക്കത്തിലും. എന്നിട്ട് ഏതൊക്കെ ആത്മാവിന് അവന്‍ മരണം വിധിച്ചിരിക്കുന്നുവോ അവയെ അവന്‍ പിടിച്ചു വെയ്ക്കുന്നു. മറ്റുള്ളവയെ നിശ്ചിതമായ ഒരു അവധിവരെ അവന്‍ വിട്ടയക്കുകയും ചെയ്യുന്നു. തീര്‍ച്ചയായും അതില്‍ ചിന്തിക്കുന്ന ജനങ്ങള്‍ക്ക് ദൃഷ്ടാന്തങ്ങളുണ്ട്‌.
\end{malayalam}}
\flushright{\begin{Arabic}
\quranayah[39][43]
\end{Arabic}}
\flushleft{\begin{malayalam}
അതല്ല, അല്ലാഹുവിനു പുറമെ അവര്‍ ശുപാര്‍ശക്കാരെ സ്വീകരിച്ചിരിക്കുകയാണോ? പറയുക: അവര്‍ (ശുപാര്‍ശക്കാര്‍) യാതൊന്നും അധീനപ്പെടുത്തുകയോ ചിന്തിച്ചു മനസ്സിലാക്കുകയോ ചെയ്യുന്നില്ലെങ്കില്‍ പോലും (അവരെ ശുപാര്‍ശക്കാരാക്കുകയോ?)
\end{malayalam}}
\flushright{\begin{Arabic}
\quranayah[39][44]
\end{Arabic}}
\flushleft{\begin{malayalam}
പറയുക: അല്ലാഹുവിനാകുന്നു ശുപാര്‍ശ മുഴുവന്‍. അവന്നാകുന്നു ആകാശങ്ങളുടെയും, ഭൂമിയുടെയും ആധിപത്യം. പിന്നീട് അവങ്കലേക്ക് തന്നെയാകുന്നു നിങ്ങള്‍ മടക്കപ്പെടുന്നത്‌.
\end{malayalam}}
\flushright{\begin{Arabic}
\quranayah[39][45]
\end{Arabic}}
\flushleft{\begin{malayalam}
അല്ലാഹുവെപ്പറ്റി മാത്രം പ്രസ്താവിക്കപ്പെട്ടാല്‍ പരലോകത്തില്‍ വിശ്വാസമില്ലാത്തവരുടെ ഹൃദയങ്ങള്‍ക്ക് അസഹ്യത അനുഭവപ്പെടുന്നതാണ്‌. അല്ലാഹുവിന് പുറമെയുള്ളവരെപ്പറ്റി പ്രസ്താവിക്കപ്പെട്ടാലോ അപ്പോഴതാ അവര്‍ സന്തുഷ്ടചിത്തരാകുന്നു.
\end{malayalam}}
\flushright{\begin{Arabic}
\quranayah[39][46]
\end{Arabic}}
\flushleft{\begin{malayalam}
പറയുക: ആകാശങ്ങളുടെയും ഭൂമിയുടെയും സ്രഷ്ടാവും, അദൃശ്യവും ദൃശ്യവും അറിയുന്നവനുമായ അല്ലാഹുവേ, നിന്‍റെ ദാസന്‍മാര്‍ക്കിടയില്‍ അവര്‍ ഭിന്നിച്ചു കൊണ്ടിരിക്കുന്ന വിഷയത്തില്‍ നീ തന്നെയാണ് വിധികല്‍പിക്കുന്നത്‌.
\end{malayalam}}
\flushright{\begin{Arabic}
\quranayah[39][47]
\end{Arabic}}
\flushleft{\begin{malayalam}
ഭൂമിയിലുള്ളത് മുഴുവനും അതോടൊപ്പം അത്രയും കൂടിയും അക്രമം പ്രവര്‍ത്തിച്ചവരുടെ അധീനത്തില്‍ ഉണ്ടായിരുന്നാല്‍ പോലും ഉയിര്‍ത്തെഴുന്നേല്‍പിന്‍റെ നാളിലെ കടുത്ത ശിക്ഷയില്‍ നിന്ന് രക്ഷപ്പെടാന്‍ അതവര്‍ പ്രായശ്ചിത്തമായി നല്‍കിയേക്കും. അവര്‍ കണക്ക് കൂട്ടിയിട്ടില്ലായിരുന്ന പലതും അല്ലാഹുവിങ്കല്‍ നിന്ന് അവര്‍ക്ക് വെളിപ്പെടുകയും ചെയ്യും.
\end{malayalam}}
\flushright{\begin{Arabic}
\quranayah[39][48]
\end{Arabic}}
\flushleft{\begin{malayalam}
അവര്‍ സമ്പാദിച്ചതിന്‍റെ ദൂഷ്യങ്ങള്‍ അവര്‍ക്ക് വെളിപ്പെടുകയും ചെയ്യും. എന്തൊന്നിനെപറ്റി അവര്‍ പരിഹസിച്ചു കൊണ്ടിരിക്കുന്നുവോ അത് അവരെ വലയം ചെയ്യുകയും ചെയ്യും.
\end{malayalam}}
\flushright{\begin{Arabic}
\quranayah[39][49]
\end{Arabic}}
\flushleft{\begin{malayalam}
എന്നാല്‍ മനുഷ്യന് വല്ല ദോഷവും ബാധിച്ചാല്‍ നമ്മോടവന്‍ പ്രാര്‍ത്ഥിക്കുന്നു. പിന്നീട് നാം അവന്ന് നമ്മുടെ പക്കല്‍ നിന്നുള്ള വല്ല അനുഗ്രഹവും പ്രദാനം ചെയ്താല്‍ അവന്‍ പറയും; അറിവിന്‍റെ അടിസ്ഥാനത്തില്‍ തന്നെ യാണ് തനിക്ക് അത് നല്‍കപ്പെട്ടിട്ടുള്ളത് എന്ന്‌. പക്ഷെ, അത് ഒരു പരീക്ഷണമാകുന്നു. എന്നാല്‍ അവരില്‍ അധികപേരും അത് മനസ്സിലാക്കുന്നില്ല.
\end{malayalam}}
\flushright{\begin{Arabic}
\quranayah[39][50]
\end{Arabic}}
\flushleft{\begin{malayalam}
ഇവരുടെ മുമ്പുള്ളവരും ഇപ്രകാരം പറയുകയുണ്ടായിട്ടുണ്ട്‌. എന്നാല്‍ അവര്‍ സമ്പാദിച്ചിരുന്നത് അവര്‍ക്ക് പ്രയോജനപ്പെടുകയുണ്ടായില്ല.
\end{malayalam}}
\flushright{\begin{Arabic}
\quranayah[39][51]
\end{Arabic}}
\flushleft{\begin{malayalam}
അങ്ങനെ അവര്‍ സമ്പാദിച്ചിരുന്നതിന്‍റെ ദൂഷ്യങ്ങള്‍ അവര്‍ക്ക് ബാധിച്ചു. ഇക്കൂട്ടരില്‍ നിന്ന് അക്രമം ചെയ്തിട്ടുള്ളവര്‍ക്കും തങ്ങള്‍ സമ്പാദിച്ചതിന്‍റെ ദൂഷ്യങ്ങള്‍ ബാധിക്കാന്‍ പോകുകയാണ്‌. അവര്‍ക്ക് (നമ്മെ) തോല്‍പിച്ചു കളയാനാവില്ല.
\end{malayalam}}
\flushright{\begin{Arabic}
\quranayah[39][52]
\end{Arabic}}
\flushleft{\begin{malayalam}
അല്ലാഹു താന്‍ ഉദ്ദേശിക്കുന്നവര്‍ക്ക് ഉപജീവനം വിശാലമാക്കികൊടുക്കുകയും താന്‍ ഉദ്ദേശിക്കുന്നവര്‍ക്ക് ഇടുങ്ങിയതാക്കുകയും ചെയ്യുന്നു എന്ന് അവര്‍ മനസ്സിലാക്കിയിട്ടില്ലേ? വിശ്വസിക്കുന്ന ജനങ്ങള്‍ക്ക് തീര്‍ച്ചയായും അതില്‍ ദൃഷ്ടാന്തങ്ങളുണ്ട്‌.
\end{malayalam}}
\flushright{\begin{Arabic}
\quranayah[39][53]
\end{Arabic}}
\flushleft{\begin{malayalam}
പറയുക: സ്വന്തം ആത്മാക്കളോട് അതിക്രമം പ്രവര്‍ത്തിച്ച് പോയ എന്‍റെ ദാസന്‍മാരേ, അല്ലാഹുവിന്‍റെ കാരുണ്യത്തെപ്പറ്റി നിങ്ങള്‍ നിരാശപ്പെടരുത്‌. തീര്‍ച്ചയായും അല്ലാഹു പാപങ്ങളെല്ലാം പൊറുക്കുന്നതാണ്‌. തീര്‍ച്ചയായും അവന്‍ തന്നെയാകുന്നു ഏറെ പൊറുക്കുന്നവനും കരുണാനിധിയും.
\end{malayalam}}
\flushright{\begin{Arabic}
\quranayah[39][54]
\end{Arabic}}
\flushleft{\begin{malayalam}
നിങ്ങള്‍ക്ക് ശിക്ഷ വന്നെത്തുന്നതിനു മുമ്പായി നിങ്ങള്‍ നിങ്ങളുടെ രക്ഷിതാവിങ്കലേക്ക് താഴ്മയോടെ മടങ്ങുകയും, അവന്നു കീഴ്പെടുകയും ചെയ്യുവിന്‍. പിന്നെ (അത് വന്നതിന് ശേഷം)നിങ്ങള്‍ സഹായിക്കപ്പെടുന്നതല്ല.
\end{malayalam}}
\flushright{\begin{Arabic}
\quranayah[39][55]
\end{Arabic}}
\flushleft{\begin{malayalam}
നിങ്ങള്‍ ഓര്‍ക്കാതിരിക്കെ പെട്ടെന്ന് നിങ്ങള്‍ക്ക് ശിക്ഷ വന്നെത്തുന്നതിന് മുമ്പായി നിങ്ങളുടെ രക്ഷിതാവിങ്കല്‍ നിന്ന് നിങ്ങള്‍ക്ക് അവതരിപ്പിക്കപ്പെട്ടതില്‍ നിന്ന് ഏറ്റവും ഉത്തമമായത് നിങ്ങള്‍ പിന്‍പറ്റുകയും ചെയ്യുക.
\end{malayalam}}
\flushright{\begin{Arabic}
\quranayah[39][56]
\end{Arabic}}
\flushleft{\begin{malayalam}
എന്‍റെ കഷ്ടമേ, അല്ലാഹുവിന്‍റെ ഭാഗത്തേക്ക് ഞാന്‍ ചെയ്യേണ്ടതില്‍ ഞാന്‍ വീഴ്ചവരുത്തിയല്ലോ. തീര്‍ച്ചയായും ഞാന്‍ കളിയാക്കുന്നവരുടെ കൂട്ടത്തില്‍ തന്നെ ആയിപ്പോയല്ലോ എന്ന് വല്ല വ്യക്തിയും പറഞ്ഞേക്കും എന്നതിനാലാണിത്‌.
\end{malayalam}}
\flushright{\begin{Arabic}
\quranayah[39][57]
\end{Arabic}}
\flushleft{\begin{malayalam}
അല്ലെങ്കില്‍ അല്ലാഹു എന്നെ നേര്‍വഴിയിലാക്കിയിരുന്നെങ്കില്‍ ഞാന്‍ സൂക്ഷ്മത പാലിക്കുന്നവരുടെ കൂട്ടത്തില്‍ ആകുമായിരുന്നു. എന്ന് പറഞ്ഞേക്കുമെന്നതിനാല്‍.
\end{malayalam}}
\flushright{\begin{Arabic}
\quranayah[39][58]
\end{Arabic}}
\flushleft{\begin{malayalam}
അല്ലെങ്കില്‍ ശിക്ഷ നേരില്‍ കാണുന്ന സന്ദര്‍ഭത്തില്‍ എനിക്കൊന്ന് മടങ്ങിപ്പോകാന്‍ കഴിഞ്ഞിരുന്നെങ്കില്‍ ഞാന്‍ സദ്‌വൃത്തരുടെ കൂട്ടത്തില്‍ ആകുമായിരുന്നു എന്ന് പറഞ്ഞേക്കുമെന്നതിനാല്‍.
\end{malayalam}}
\flushright{\begin{Arabic}
\quranayah[39][59]
\end{Arabic}}
\flushleft{\begin{malayalam}
അതെ, തീര്‍ച്ചയായും എന്‍റെ ദൃഷ്ടാന്തങ്ങള്‍ നിനക്ക് വന്നെത്തുകയുണ്ടായി. അപ്പോള്‍ നീ അവയെ നിഷേധിച്ച് തള്ളുകയും അഹങ്കരിക്കുകയും സത്യനിഷേധികളുടെ കൂട്ടത്തിലാകുകയും ചെയ്തു.
\end{malayalam}}
\flushright{\begin{Arabic}
\quranayah[39][60]
\end{Arabic}}
\flushleft{\begin{malayalam}
ഉയിര്‍ത്തെഴുന്നേല്‍പിന്‍റെ നാളില്‍, അല്ലാഹുവിന്‍റെ പേരില്‍ കള്ളം പറഞ്ഞവരുടെ മുഖങ്ങള്‍ കറുത്തിരുണ്ടതായി നിനക്ക് കാണാം. നരകത്തിലല്ലയോ അഹങ്കാരികള്‍ക്കുള്ള വാസസ്ഥലം!
\end{malayalam}}
\flushright{\begin{Arabic}
\quranayah[39][61]
\end{Arabic}}
\flushleft{\begin{malayalam}
സൂക്ഷ്മത പുലര്‍ത്തിയവരെ രക്ഷപ്പെടുത്തി അവര്‍ക്കുള്ളതായ സുരക്ഷിതസ്ഥാനത്ത് അല്ലാഹു എത്തിക്കുകയും ചെയ്യും. ശിക്ഷ അവരെ സ്പര്‍ശിക്കുകയില്ല. അവര്‍ ദുഃഖിക്കേണ്ടി വരികയുമില്ല.
\end{malayalam}}
\flushright{\begin{Arabic}
\quranayah[39][62]
\end{Arabic}}
\flushleft{\begin{malayalam}
അല്ലാഹു എല്ലാ വസ്തുക്കളുടെയും സ്രഷ്ടാവാകുന്നു. അവന്‍ എല്ലാ വസ്തുക്കളുടെ മേലും കൈകാര്യകര്‍ത്താവുമാകുന്നു.
\end{malayalam}}
\flushright{\begin{Arabic}
\quranayah[39][63]
\end{Arabic}}
\flushleft{\begin{malayalam}
ആകാശങ്ങളുടെയും ഭൂമിയുടെയും താക്കോലുകള്‍ അവന്‍റെ അധീനത്തിലാകുന്നു. അല്ലാഹുവിന്‍റെ ദൃഷ്ടാന്തങ്ങളെ നിഷേധിച്ചവരാരോ അവര്‍ തന്നെയാകുന്നു നഷ്ടക്കാര്‍.
\end{malayalam}}
\flushright{\begin{Arabic}
\quranayah[39][64]
\end{Arabic}}
\flushleft{\begin{malayalam}
(നബിയേ,) പറയുക: ഹേ; വിവരംകെട്ടവരേ, അപ്പോള്‍ അല്ലാഹുവല്ലാത്തവരെ ഞാന്‍ ആരാധിക്കണമെന്നാണോ നിങ്ങള്‍ എന്നോട് കല്‍പിക്കുന്നത്‌?
\end{malayalam}}
\flushright{\begin{Arabic}
\quranayah[39][65]
\end{Arabic}}
\flushleft{\begin{malayalam}
തീര്‍ച്ചയായും നിനക്കും നിന്‍റെ മുമ്പുള്ളവര്‍ക്കും സന്ദേശം നല്‍കപ്പെട്ടിട്ടുള്ളത് ഇതത്രെ: (അല്ലാഹുവിന്‌) നീ പങ്കാളിയെ ചേര്‍ക്കുന്ന പക്ഷം തീര്‍ച്ചയായും നിന്‍റെ കര്‍മ്മം നിഷ്ഫലമായിപ്പോകുകയും തീര്‍ച്ചയായും നീ നഷ്ടക്കാരുടെ കൂട്ടത്തില്‍ ആകുകയും ചെയ്യും.
\end{malayalam}}
\flushright{\begin{Arabic}
\quranayah[39][66]
\end{Arabic}}
\flushleft{\begin{malayalam}
അല്ല, അല്ലാഹുവെ തന്നെ നീ ആരാധിക്കുകയും നീ നന്ദിയുള്ളവരുടെ കൂട്ടത്തിലായിരിക്കുകയും ചെയ്യുക.
\end{malayalam}}
\flushright{\begin{Arabic}
\quranayah[39][67]
\end{Arabic}}
\flushleft{\begin{malayalam}
അല്ലാഹുവെ കണക്കാക്കേണ്ട നിലയില്‍ അവര്‍ കണക്കാക്കിയിട്ടില്ല. ഉയിര്‍ത്തെഴുന്നേല്‍പിന്‍റെ നാളില്‍ ഭൂമി മുഴുവന്‍ അവന്‍റെ ഒരു കൈപിടിയില്‍ ഒതുങ്ങുന്നതായിരിക്കും. ആകാശങ്ങള്‍ അവന്‍റെ വലതുകൈയ്യില്‍ ചുരുട്ടിപിടിക്കപ്പെട്ടവയുമായിരിക്കും. അവനെത്ര പരിശുദ്ധന്‍! അവര്‍ പങ്കുചേര്‍ക്കുന്നതിനെല്ലാം അവന്‍ അതീതനായിരിക്കുന്നു.
\end{malayalam}}
\flushright{\begin{Arabic}
\quranayah[39][68]
\end{Arabic}}
\flushleft{\begin{malayalam}
കാഹളത്തില്‍ ഊതപ്പെടും. അപ്പോള്‍ ആകാശങ്ങളിലുള്ളവരും ഭൂമിയിലുള്ളവരും ചലനമറ്റവരായിത്തീരും; അല്ലാഹു ഉദ്ദേശിച്ചവരൊഴികെ. പിന്നീട് അതില്‍ (കാഹളത്തില്‍) മറ്റൊരിക്കല്‍ ഊതപ്പെടും. അപ്പോഴതാ അവര്‍ എഴുന്നേറ്റ് നോക്കുന്നു.
\end{malayalam}}
\flushright{\begin{Arabic}
\quranayah[39][69]
\end{Arabic}}
\flushleft{\begin{malayalam}
ഭൂമി അതിന്‍റെ രക്ഷിതാവിന്‍റെ പ്രഭകൊണ്ട് പ്രകാശിക്കുകയും ചെയ്യും (കര്‍മ്മങ്ങളുടെ) രേഖവെക്കപ്പെടുകയും പ്രവാചകന്‍മാരും സാക്ഷികളും കൊണ്ട് വരപ്പെടുകയും ജനങ്ങള്‍ക്കിടയില്‍ സത്യപ്രകാരം വിധിക്കപ്പെടുകയും ചെയ്യും. അവരോട് അനീതി കാണിക്കപ്പെടുകയില്ല.
\end{malayalam}}
\flushright{\begin{Arabic}
\quranayah[39][70]
\end{Arabic}}
\flushleft{\begin{malayalam}
ഓരോ വ്യക്തിക്കും താന്‍ പ്രവര്‍ത്തിച്ചത് നിറവേറ്റികൊടുക്കപ്പെടുകയും ചെയ്യും. അവര്‍ പ്രവര്‍ത്തിക്കുന്നതിനെ പറ്റി അവന്‍ നല്ലവണ്ണം അറിയുന്നവനത്രെ.
\end{malayalam}}
\flushright{\begin{Arabic}
\quranayah[39][71]
\end{Arabic}}
\flushleft{\begin{malayalam}
സത്യനിഷേധികള്‍ കൂട്ടം കൂട്ടമായി നരകത്തിലേക്ക് നയിക്കപ്പെടുകയും ചെയ്യും. അങ്ങനെ അവര്‍ അതിന്നടുത്തു വന്നാല്‍ അതിന്‍റെ വാതിലുകള്‍ തുറക്കപ്പെടും. നിങ്ങള്‍ക്ക് നിങ്ങളുടെ രക്ഷിതാവിന്‍റെ ദൃഷ്ടാന്തങ്ങള്‍ ഓതികേള്‍പിക്കുകയും, നിങ്ങള്‍ക്കുള്ളതായ ഈ ദിവസത്തെ കണ്ടുമുട്ടുന്നതിനെ പറ്റി നിങ്ങള്‍ക്ക് താക്കീത് നല്‍കുകയും ചെയ്യുന്ന നിങ്ങളുടെ കൂട്ടത്തില്‍ നിന്നുതന്നെയുള്ള ദൂതന്‍മാര്‍ നിങ്ങളുടെ അടുക്കല്‍ വന്നിട്ടില്ലേ. എന്ന് അതിന്‍റെ (നരകത്തിന്‍റെ) കാവല്‍ക്കാര്‍ അവരോട് ചോദിക്കുകയും ചെയ്യും. അവര്‍ പറയും: അതെ. പക്ഷെ സത്യനിഷേധികളുടെ മേല്‍ ശിക്ഷയുടെ വചനം സ്ഥിരപ്പെട്ടു പോയി.
\end{malayalam}}
\flushright{\begin{Arabic}
\quranayah[39][72]
\end{Arabic}}
\flushleft{\begin{malayalam}
(അവരോട്‌) പറയപ്പെടും: നിങ്ങള്‍ നരകത്തിന്‍റെ വാതിലുകളിലൂടെ പ്രവേശിക്കുക. നിങ്ങളതില്‍ നിത്യവാസികളായിരിക്കും. എന്നാല്‍ അഹങ്കാരികളുടെ പാര്‍പ്പിടം എത്ര ചീത്ത!
\end{malayalam}}
\flushright{\begin{Arabic}
\quranayah[39][73]
\end{Arabic}}
\flushleft{\begin{malayalam}
തങ്ങളുടെ രക്ഷിതാവിനെ സൂക്ഷിച്ചു ജീവിച്ചവര്‍ സ്വര്‍ഗത്തിലേക്ക് കൂട്ടംകൂട്ടമായി നയിക്കപ്പെടും. അങ്ങനെ അതിന്‍റെ കവാടങ്ങള്‍ തൂറന്ന് വെക്കപ്പെട്ട നിലയില്‍ അവര്‍ അതിന്നടുത്ത് വരുമ്പോള്‍ അവരോട് അതിന്‍റെ കാവല്‍ക്കാര്‍ പറയും: നിങ്ങള്‍ക്ക് സമാധാനം. നിങ്ങള്‍ സംശുദ്ധരായിരിക്കുന്നു. അതിനാല്‍ നിത്യവാസികളെന്ന നിലയില്‍ നിങ്ങള്‍ അതില്‍ പ്രവേശിച്ചു കൊള്ളുക.
\end{malayalam}}
\flushright{\begin{Arabic}
\quranayah[39][74]
\end{Arabic}}
\flushleft{\begin{malayalam}
അവര്‍ പറയും: നമ്മളോടുള്ള തന്‍റെ വാഗ്ദാനം സത്യമായി പാലിക്കുകയും സ്വര്‍ഗത്തില്‍ നിന്ന് നാം ഉദ്ദേശിക്കുന്ന സ്ഥലത്ത് നമുക്ക് താമസിക്കാവുന്ന വിധം ഈ (സ്വര്‍ഗ) ഭൂമി നമുക്ക് അവകാശപ്പെടുത്തിത്തരികയും ചെയ്ത അല്ലാഹുവിന് സ്തുതി. അപ്പോള്‍ പ്രവര്‍ത്തിച്ചവര്‍ക്കുള്ള പ്രതിഫലം എത്ര വിശിഷ്ടം!
\end{malayalam}}
\flushright{\begin{Arabic}
\quranayah[39][75]
\end{Arabic}}
\flushleft{\begin{malayalam}
മലക്കുകള്‍ തങ്ങളുടെ രക്ഷിതാവിനെ സ്തുതിക്കുന്നതോടൊപ്പം കീര്‍ത്തനം ചെയ്തുകൊണ്ട് സിംഹാസനത്തിന്‍റെ ചുറ്റും വലയം ചെയ്യുന്നതായി നിനക്ക് കാണാം. അവര്‍ക്കിടയില്‍ സത്യപ്രകാരം വിധികല്‍പിക്കപ്പെടും. ലോകരക്ഷിതാവായ അല്ലാഹുവിന് സ്തുതി എന്ന് പറയപ്പെടുകയും ചെയ്യും.
\end{malayalam}}
\chapter{\textmalayalam{മുഅ്മിന്‍‍ ( വിശ്വാസി )}}
\begin{Arabic}
\Huge{\centerline{\basmalah}}\end{Arabic}
\flushright{\begin{Arabic}
\quranayah[40][1]
\end{Arabic}}
\flushleft{\begin{malayalam}
ഹാ-മീം.
\end{malayalam}}
\flushright{\begin{Arabic}
\quranayah[40][2]
\end{Arabic}}
\flushleft{\begin{malayalam}
ഈ ഗ്രന്ഥത്തിന്‍റെ അവതരണം പ്രതാപിയും സര്‍വ്വജ്ഞനുമായ അല്ലാഹുവിങ്കല്‍ നിന്നാകുന്നു.
\end{malayalam}}
\flushright{\begin{Arabic}
\quranayah[40][3]
\end{Arabic}}
\flushleft{\begin{malayalam}
പാപം പൊറുക്കുന്നവനും പശ്ചാത്താപം സ്വീകരിക്കുന്നവനും കഠിനമായി ശിക്ഷിക്കുന്നവനും വിപുലമായ കഴിവുള്ളവനുമത്രെ അവന്‍. അവനല്ലാതെ യാതൊരു ദൈവവുമില്ല. അവങ്കലേക്ക് തന്നെയാകുന്നു മടക്കം.
\end{malayalam}}
\flushright{\begin{Arabic}
\quranayah[40][4]
\end{Arabic}}
\flushleft{\begin{malayalam}
സത്യനിഷേധികളല്ലാത്തവര്‍ അല്ലാഹുവിന്‍റെ ദൃഷ്ടാന്തങ്ങളെ പറ്റി തര്‍ക്കിക്കുകയില്ല. അതിനാല്‍ നാടുകളില്‍ അവരുടെ സ്വൈരവിഹാരം നിന്നെ വഞ്ചിതനാക്കാതിരിക്കട്ടെ.
\end{malayalam}}
\flushright{\begin{Arabic}
\quranayah[40][5]
\end{Arabic}}
\flushleft{\begin{malayalam}
അവര്‍ക്ക് മുമ്പ് നൂഹിന്‍റെ ജനതയും അവരുടെ ശേഷമുള്ള കക്ഷികളും (സത്യത്തെ) നിഷേധിച്ചു തള്ളിക്കളയുകയുണ്ടായി. ഓരോ സമുദായവും തങ്ങളുടെ റസൂലിനെ പിടികൂടാന്‍ ഉദ്യമിക്കുകയും, അസത്യത്തെകൊണ്ട് സത്യത്തെ തകര്‍ക്കുവാന്‍ വേണ്ടി അവര്‍ തര്‍ക്കം നടത്തുകയും ചെയ്തു. തന്നിമിത്തം ഞാന്‍ അവരെ പിടികൂടി. അപ്പോള്‍ എന്‍റെ ശിക്ഷ എങ്ങനെയുണ്ടായിരുന്നു!
\end{malayalam}}
\flushright{\begin{Arabic}
\quranayah[40][6]
\end{Arabic}}
\flushleft{\begin{malayalam}
സത്യനിഷേധികളുടെ മേല് ‍, അവര്‍ നരകാവകാശികളാണ് എന്നുള്ള നിന്‍റെ രക്ഷിതാവിന്‍റെ വചനം അപ്രകാരം സ്ഥിരപ്പെട്ട് കഴിഞ്ഞു.
\end{malayalam}}
\flushright{\begin{Arabic}
\quranayah[40][7]
\end{Arabic}}
\flushleft{\begin{malayalam}
സിംഹാസനം വഹിക്കുന്നവരും അതിന്‍റെ ചുറ്റിലുള്ളവരും (മലക്കുകള്‍) തങ്ങളുടെ രക്ഷിതാവിനെ സ്തുതിക്കുന്നതോടൊപ്പം കീര്‍ത്തനം നടത്തുകയും അവനില്‍ വിശ്വസിക്കുകയും, വിശ്വസിച്ചവര്‍ക്ക് വേണ്ടി (ഇപ്രകാരം) പാപമോചനം തേടുകയും ചെയ്യുന്നു: ഞങ്ങളുടെ രക്ഷിതാവേ! നിന്‍റെ കാരുണ്യവും അറിവും സകല വസ്തുക്കളെയും ഉള്‍കൊള്ളുന്നതായിരിക്കുന്നു. ആകയാല്‍ പശ്ചാത്തപിക്കുകയും നിന്‍റെ മാര്‍ഗം പിന്തുടരുകയും ചെയ്യുന്നവര്‍ക്ക് നീ പൊറുത്തുകൊടുക്കേണമേ. അവരെ നീ നരകശിക്ഷയില്‍ നിന്ന് കാക്കുകയും ചെയ്യേണമേ.
\end{malayalam}}
\flushright{\begin{Arabic}
\quranayah[40][8]
\end{Arabic}}
\flushleft{\begin{malayalam}
ഞങ്ങളുടെ രക്ഷിതാവേ, അവര്‍ക്ക് നീ വാഗ്ദാനം ചെയ്തിട്ടുള്ള സ്ഥിരവാസത്തിനുള്ള സ്വര്‍ഗങ്ങളില്‍ അവരെയും അവരുടെ മാതാപിതാക്കളെയും, ഭാര്യമാര്‍, സന്തതികള്‍ എന്നിവരില്‍ നിന്നു സദ്‌വൃത്തരായിട്ടുള്ളവരെയും നീ പ്രവേശിപ്പിക്കേണമേ. തീര്‍ച്ചയായും നീ തന്നെയാകുന്നു പ്രതാപിയും യുക്തിമാനും.
\end{malayalam}}
\flushright{\begin{Arabic}
\quranayah[40][9]
\end{Arabic}}
\flushleft{\begin{malayalam}
അവരെ നീ തിന്‍മകളില്‍ നിന്ന് കാക്കുകയും ചെയ്യേണമേ. അന്നേ ദിവസം നീ ഏതൊരാളെ തിന്‍മകളില്‍ നിന്ന് കാക്കുന്നുവോ, അവനോട് തീര്‍ച്ചയായും നീ കരുണ കാണിച്ചിരിക്കുന്നു. അതു തന്നെയാകുന്നു മഹാഭാഗ്യം.
\end{malayalam}}
\flushright{\begin{Arabic}
\quranayah[40][10]
\end{Arabic}}
\flushleft{\begin{malayalam}
തീര്‍ച്ചയായും സത്യനിഷേധികളോട് ഇപ്രകാരം വിളിച്ചുപറയപ്പെടും: നിങ്ങള്‍ വിശ്വാസത്തിലേക്ക് ക്ഷണിക്കപ്പെടുകയും, എന്നിട്ട് നിങ്ങള്‍ അവിശ്വസിക്കുകയും ചെയ്തിരുന്ന സന്ദര്‍ഭത്തില്‍ അല്ലാഹുവിന് (നിങ്ങളോടുള്ള) അമര്‍ഷം നിങ്ങള്‍ തമ്മിലുള്ള അമര്‍ഷത്തെക്കാള്‍ വലുതാകുന്നു.
\end{malayalam}}
\flushright{\begin{Arabic}
\quranayah[40][11]
\end{Arabic}}
\flushleft{\begin{malayalam}
അവര്‍ പറയും: ഞങ്ങളുടെ നാഥാ! രണ്ടുപ്രാവശ്യം നീ ഞങ്ങളെ നിര്‍ജീവാവസ്ഥയിലാക്കുകയും രണ്ടുപ്രാവശ്യം നീ ഞങ്ങള്‍ക്ക് ജീവന്‍ നല്‍കുകയും ചെയ്തു. എന്നാല്‍ ഞങ്ങളിതാ ഞങ്ങളുടെ കുറ്റങ്ങള്‍ സമ്മതിച്ചിരിക്കുന്നു. ആകയാല്‍ ഒന്നു പുറത്ത്പോകേണ്ടതിലേക്ക് വല്ല മാര്‍ഗവുമുണ്ടോ?
\end{malayalam}}
\flushright{\begin{Arabic}
\quranayah[40][12]
\end{Arabic}}
\flushleft{\begin{malayalam}
അല്ലാഹുവോട് മാത്രം പ്രാര്‍ത്ഥിക്കപ്പെട്ടാല്‍ നിങ്ങള്‍ അവിശ്വസിക്കുകയും, അവനോട് പങ്കാളികള്‍ കൂട്ടിചേര്‍ക്കപ്പെട്ടാല്‍ നിങ്ങള്‍ വിശ്വസിക്കുകയും ചെയ്തിരുന്നത് നിമിത്തമത്രെ അത്‌. എന്നാല്‍ (ഇന്ന്‌) വിധികല്‍പിക്കാനുള്ള അധികാരം ഉന്നതനും മഹാനുമായ അല്ലാഹുവിനാകുന്നു.
\end{malayalam}}
\flushright{\begin{Arabic}
\quranayah[40][13]
\end{Arabic}}
\flushleft{\begin{malayalam}
അവനാണ് നിങ്ങള്‍ക്ക് തന്‍റെ ദൃഷ്ടാന്തങ്ങള്‍ കാണിച്ചുതരുന്നത്‌. ആകാശത്ത് നിന്ന് അവന്‍ നിങ്ങള്‍ക്ക് ഉപജീവനം ഇറക്കിത്തരികയും ചെയ്യുന്നു. (അവങ്കലേക്ക്‌) മടങ്ങുന്നവര്‍ മാത്രമേ ആലോചിച്ച് ഗ്രഹിക്കുകയുള്ളൂ.
\end{malayalam}}
\flushright{\begin{Arabic}
\quranayah[40][14]
\end{Arabic}}
\flushleft{\begin{malayalam}
അതിനാല്‍ കീഴ്‌വണക്കം അല്ലാഹുവിന് നിഷ്കളങ്കമാക്കികൊണ്ട് അവനോട് നിങ്ങള്‍ പ്രാര്‍ത്ഥിക്കുക. അവിശ്വാസികള്‍ക്ക് അനിഷ്ടകരമായാലും ശരി.
\end{malayalam}}
\flushright{\begin{Arabic}
\quranayah[40][15]
\end{Arabic}}
\flushleft{\begin{malayalam}
അവന്‍ പദവികള്‍ ഉയര്‍ന്നവനും സിംഹാസനത്തിന്‍റെ അധിപനുമാകുന്നു. തന്‍റെ ദാസന്‍മാരില്‍ നിന്ന് താന്‍ ഉദ്ദേശിക്കുന്നവര്‍ക്ക് തന്‍റെ സന്ദേശമാകുന്ന ചൈതന്യം അവന്‍ നല്‍കുന്നു. (മനുഷ്യര്‍) പരസ്പരം കണ്ടുമുട്ടുന്ന ദിവസത്തെപ്പറ്റി താക്കീത് നല്‍കുന്നതിന് വേണ്ടിയത്രെ അത്‌.
\end{malayalam}}
\flushright{\begin{Arabic}
\quranayah[40][16]
\end{Arabic}}
\flushleft{\begin{malayalam}
അവര്‍ വെളിക്കു വരുന്ന ദിവസമത്രെ അത്‌. അവരെ സംബന്ധിച്ച് യാതൊരു കാര്യവും അല്ലാഹുവിന്ന് ഗോപ്യമായിരിക്കുകയില്ല. ഈ ദിവസം ആര്‍ക്കാണ് രാജാധികാരം? ഏകനും സര്‍വ്വാധിപതിയുമായ അല്ലാഹുവിന്‌.
\end{malayalam}}
\flushright{\begin{Arabic}
\quranayah[40][17]
\end{Arabic}}
\flushleft{\begin{malayalam}
ഈ ദിവസം ഓരോ വ്യക്തിക്കും താന്‍ സമ്പാദിച്ചതിനുള്ള പ്രതിഫലം നല്‍കപ്പെടും. ഈ ദിവസം അനീതിയില്ല. തീര്‍ച്ചയായും അല്ലാഹു അതിവേഗം കണക്കു നോക്കുന്നവനാകുന്നു.
\end{malayalam}}
\flushright{\begin{Arabic}
\quranayah[40][18]
\end{Arabic}}
\flushleft{\begin{malayalam}
ആസന്നമായ ആ സംഭവത്തിന്‍റെ ദിവസത്തെപ്പറ്റി നീ അവര്‍ക്ക് മുന്നറിയിപ്പു നല്‍കുക. അതായത് ഹൃദയങ്ങള്‍ തൊണ്ടക്കുഴികളുടെ അടുത്തെത്തുന്ന, അവര്‍ ശ്വാസമടക്കിപ്പിടിച്ചവരായിരിക്കുന്ന സന്ദര്‍ഭം. അക്രമകാരികള്‍ക്ക് ഉറ്റബന്ധുവായോ സ്വീകാര്യനായ ശുപാര്‍ശകനായോ ആരും തന്നെയില്ല.
\end{malayalam}}
\flushright{\begin{Arabic}
\quranayah[40][19]
\end{Arabic}}
\flushleft{\begin{malayalam}
കണ്ണുകളുടെ കള്ളനോട്ടവും, ഹൃദയങ്ങള്‍ മറച്ച് വെക്കുന്നതും അവന്‍ (അല്ലാഹു) അറിയുന്നു.
\end{malayalam}}
\flushright{\begin{Arabic}
\quranayah[40][20]
\end{Arabic}}
\flushleft{\begin{malayalam}
അല്ലാഹു സത്യപ്രകാരം തീര്‍പ്പുകല്‍പിക്കുന്നു. അവന്ന് പുറമെ അവര്‍ വിളിച്ച് പ്രാര്‍ത്ഥിക്കുന്നവരാകട്ടെ യാതൊന്നിലും തീര്‍പ്പുകല്‍പിക്കുകയില്ല. തീര്‍ച്ചയായും അല്ലാഹു തന്നെയാകുന്നു എല്ലാം കേള്‍ക്കുന്നവനും കണ്ടറിയുന്നവനും.
\end{malayalam}}
\flushright{\begin{Arabic}
\quranayah[40][21]
\end{Arabic}}
\flushleft{\begin{malayalam}
ഇവര്‍ ഭൂമിയിലൂടെ സഞ്ചരിച്ചിട്ടില്ലേ? അപ്പോള്‍ ഇവര്‍ക്ക് മുമ്പുണ്ടായിരുന്നവരുടെ പര്യവസാനം എങ്ങനെയായിരുന്നുവെന്ന് ഇവര്‍ക്ക് നോക്കാമല്ലോ. അവര്‍ ശക്തികൊണ്ടും ഭൂമിയില്‍ (അവശേഷിപ്പിച്ച) സ്മാരകങ്ങള്‍കൊണ്ടും ഇവരെക്കാള്‍ കരുത്തരായിരുന്നു. എന്നിട്ട് അവരുടെ പാപങ്ങള്‍ നിമിത്തം അല്ലാഹു അവരെ പിടികൂടി. അല്ലാഹുവിന്‍റെ ശിക്ഷയില്‍ നിന്ന് അവര്‍ക്ക് കാവല്‍ നല്‍കാന്‍ ആരുമുണ്ടായില്ല.
\end{malayalam}}
\flushright{\begin{Arabic}
\quranayah[40][22]
\end{Arabic}}
\flushleft{\begin{malayalam}
അതെന്തുകൊണ്ടെന്നാല്‍ അവരിലേക്കുള്ള ദൈവദൂതന്‍മാര്‍ വ്യക്തമായ തെളിവുകളും കൊണ്ട് അവരുടെ അടുക്കല്‍ ചെല്ലാറുണ്ടായിരുന്നു. എന്നിട്ട് അവര്‍ അവിശ്വസിച്ചു കളഞ്ഞു. അപ്പോള്‍ അല്ലാഹു അവരെ പിടികൂടി. തീര്‍ച്ചയായും അവന്‍ ശക്തനും കഠിനമായി ശിക്ഷിക്കുന്നവനുമത്രെ.
\end{malayalam}}
\flushright{\begin{Arabic}
\quranayah[40][23]
\end{Arabic}}
\flushleft{\begin{malayalam}
തീര്‍ച്ചയായും നാം നമ്മുടെ ദൃഷ്ടാന്തങ്ങളും വ്യക്തമായ പ്രമാണവും കൊണ്ട് മൂസായെ അയക്കുകയുണ്ടായി
\end{malayalam}}
\flushright{\begin{Arabic}
\quranayah[40][24]
\end{Arabic}}
\flushleft{\begin{malayalam}
ഫിര്‍ഔന്‍റെയും ഹാമാന്‍റെയും ഖാറൂന്‍റെയും അടുക്കലേക്ക് . അപ്പോള്‍ അവര്‍ പറഞ്ഞു: വ്യാജവാദിയായ ഒരു ജാലവിദ്യക്കാരന്‍ എന്ന്‌.
\end{malayalam}}
\flushright{\begin{Arabic}
\quranayah[40][25]
\end{Arabic}}
\flushleft{\begin{malayalam}
അങ്ങനെ നമ്മുടെ പക്കല്‍ നിന്നുള്ള സത്യവും കൊണ്ട് അദ്ദേഹം അവരുടെ അടുക്കല്‍ ചെന്നപ്പോള്‍ അവര്‍ പറഞ്ഞു: ഇവനോടൊപ്പം വിശ്വസിച്ചവരുടെ ആണ്‍മക്കളെ നിങ്ങള്‍ കൊന്നുകളയുകയും അവരുടെ സ്ത്രീകളെ ജീവിക്കാന്‍ അനുവദിക്കുകയും ചെയ്യുക. (പക്ഷെ) സത്യനിഷേധികളുടെ കുതന്ത്രം വഴികേടില്‍ മാത്രമേ കലാശിക്കൂ.
\end{malayalam}}
\flushright{\begin{Arabic}
\quranayah[40][26]
\end{Arabic}}
\flushleft{\begin{malayalam}
ഫിര്‍ഔന്‍ പറഞ്ഞു: നിങ്ങള്‍ എന്നെ വിടൂ; മൂസായെ ഞാന്‍ കൊല്ലും. അവന്‍ അവന്‍റെ രക്ഷിതാവിനെ വിളിച്ചു പ്രാര്‍ത്ഥിച്ചു കൊള്ളട്ടെ. അവന്‍ നിങ്ങളുടെ മതം മാറ്റി മറിക്കുകയോ ഭൂമിയില്‍ കുഴപ്പം കുത്തിപ്പൊക്കുകയോ ചെയ്യുമെന്ന് തീര്‍ച്ചയായും ഞാന്‍ ഭയപ്പെടുന്നു.
\end{malayalam}}
\flushright{\begin{Arabic}
\quranayah[40][27]
\end{Arabic}}
\flushleft{\begin{malayalam}
മൂസാ പറഞ്ഞു: എന്‍റെ രക്ഷിതാവും നിങ്ങളുടെ രക്ഷിതാവുമായിട്ടുള്ളവനോട്‌, വിചാരണയുടെ ദിവസത്തില്‍ വിശ്വസിക്കാത്ത എല്ലാ അഹങ്കാരികളില്‍ നിന്നും ഞാന്‍ ശരണം തേടുന്നു.
\end{malayalam}}
\flushright{\begin{Arabic}
\quranayah[40][28]
\end{Arabic}}
\flushleft{\begin{malayalam}
ഫിര്‍ഔന്‍റെ ആള്‍ക്കാരില്‍പ്പെട്ട - തന്‍റെ വിശ്വാസം മറച്ചു വെച്ചുകൊണ്ടിരുന്ന - ഒരു വിശ്വാസിയായ മനുഷ്യന്‍ പറഞ്ഞു: എന്‍റെ രക്ഷിതാവ് അല്ലാഹുവാണ് എന്ന് പറയുന്നതിനാല്‍ നിങ്ങള്‍ ഒരു മനുഷ്യനെ കൊല്ലുകയോ? അദ്ദേഹം നിങ്ങള്‍ക്ക് നിങ്ങളുടെ രക്ഷിതാവിങ്കല്‍ നിന്നുള്ള വ്യക്തമായ തെളിവുകള്‍ കൊണ്ടു വന്നിട്ടുണ്ട്‌. അദ്ദേഹം കള്ളം പറയുന്നവനാണെങ്കില്‍ കള്ളം പറയുന്നതിന്‍റെ ദോഷം അദ്ദേഹത്തിന് തന്നെയാണ്‌. അദ്ദേഹം സത്യം പറയുന്നവനാണെങ്കിലോ അദ്ദേഹം നിങ്ങള്‍ക്ക് താക്കീത് നല്‍കുന്ന ചില കാര്യങ്ങള്‍ (ശിക്ഷകള്‍) നിങ്ങളെ ബാധിക്കുകയും ചെയ്യും. അതിക്രമകാരിയും വ്യാജവാദിയുമായിട്ടുള്ള ഒരാളെയും അല്ലാഹു നേര്‍വഴിയിലാക്കുകയില്ല; തീര്‍ച്ച.
\end{malayalam}}
\flushright{\begin{Arabic}
\quranayah[40][29]
\end{Arabic}}
\flushleft{\begin{malayalam}
എന്‍റെ ജനങ്ങളേ, ഭൂമിയില്‍ മികച്ചുനില്‍ക്കുന്നവര്‍ എന്ന നിലയില്‍ ഇന്ന് ആധിപത്യം നിങ്ങള്‍ക്ക് തന്നെ. എന്നാല്‍ അല്ലാഹുവിന്‍റെ ശിക്ഷ നമുക്ക് വന്നാല്‍ അതില്‍ നിന്ന് നമ്മെ രക്ഷിച്ചു സഹായിക്കാന്‍ ആരുണ്ട്‌? ഫിര്‍ഔന്‍ പറഞ്ഞു: ഞാന്‍ (ശരിയായി) കാണുന്ന മാര്‍ഗം മാത്രമാണ് ഞാന്‍ നിങ്ങള്‍ക്ക് കാണിച്ചുതരുന്നത്‌. ശരിയായ മാര്‍ഗത്തിലേക്കല്ലാതെ ഞാന്‍ നിങ്ങളെ നയിക്കുകയില്ല.
\end{malayalam}}
\flushright{\begin{Arabic}
\quranayah[40][30]
\end{Arabic}}
\flushleft{\begin{malayalam}
ആ വിശ്വസിച്ച ആള്‍ പറഞ്ഞു: എന്‍റെ ജനങ്ങളേ, ആ കക്ഷികളുടെ ദിവസം പോലെയുള്ള ഒന്ന് തീര്‍ച്ചയായും നിങ്ങളുടെ കാര്യത്തിലും ഞാന്‍ ഭയപ്പെടുന്നു.
\end{malayalam}}
\flushright{\begin{Arabic}
\quranayah[40][31]
\end{Arabic}}
\flushleft{\begin{malayalam}
അതായത് നൂഹിന്‍റെ ജനതയുടെയും ആദിന്‍റെയും ഥമൂദിന്‍റെയും അവര്‍ക്ക് ശേഷമുള്ളവരുടെയും അനുഭവത്തിന് തുല്യമായത്‌. ദാസന്‍മാരോട് യാതൊരു അക്രമവും ചെയ്യാന്‍ അല്ലാഹു ഉദ്ദേശിക്കുന്നില്ല.
\end{malayalam}}
\flushright{\begin{Arabic}
\quranayah[40][32]
\end{Arabic}}
\flushleft{\begin{malayalam}
എന്‍റെ ജനങ്ങളേ, (നിങ്ങള്‍) പരസ്പരം വിളിച്ചുകേഴുന്ന ദിവസത്തെ നിങ്ങളുടെ കാര്യത്തില്‍ തീര്‍ച്ചയായും ഞാന്‍ ഭയപ്പെടുന്നു.
\end{malayalam}}
\flushright{\begin{Arabic}
\quranayah[40][33]
\end{Arabic}}
\flushleft{\begin{malayalam}
അതായത് നിങ്ങള്‍ പിന്നോക്കം തിരിഞ്ഞോടുന്ന ദിവസം. അല്ലാഹുവിന്‍റെ ശിക്ഷയില്‍ നിന്നും രക്ഷനല്‍കുന്ന ഒരാളും നിങ്ങള്‍ക്കില്ല. ഏതൊരാളെ അല്ലാഹു വഴിതെറ്റിക്കുന്നുവോ, അവന് നേര്‍വഴി കാണിക്കാന്‍ ആരുമില്ല.
\end{malayalam}}
\flushright{\begin{Arabic}
\quranayah[40][34]
\end{Arabic}}
\flushleft{\begin{malayalam}
വ്യക്തമായ തെളിവുകളും കൊണ്ട് മുമ്പ് യൂസുഫ് നിങ്ങളുടെ അടുത്ത് വരികയുണ്ടായിട്ടുണ്ട്‌. അപ്പോള്‍ അദ്ദേഹം നിങ്ങള്‍ക്ക് കൊണ്ടുവന്നതിനെ പറ്റി നിങ്ങള്‍ സംശയത്തിലായിക്കൊണേ്ടയിരുന്നു. എന്നിട്ട് അദ്ദേഹം മരണപ്പെട്ടപ്പോള്‍ ഇദ്ദേഹത്തിനു ശേഷം അല്ലാഹു ഇനി ഒരു ദൂതനെയും നിയോഗിക്കുകയേ ഇല്ല എന്ന് നിങ്ങള്‍ പറഞ്ഞു. അപ്രകാരം അതിക്രമകാരിയും സംശയാലുവുമായിട്ടുള്ളതാരോ അവരെ അല്ലാഹു വഴിതെറ്റിക്കുന്നു.
\end{malayalam}}
\flushright{\begin{Arabic}
\quranayah[40][35]
\end{Arabic}}
\flushleft{\begin{malayalam}
അതായത് തങ്ങള്‍ക്ക് യാതൊരു ആധികാരിക പ്രമാണവും വന്നുകിട്ടാതെ അല്ലാഹുവിന്‍റെ ദൃഷ്ടാന്തങ്ങളില്‍ തര്‍ക്കം നടത്തുന്നവരെ. അത് അല്ലാഹുവിന്‍റെ അടുക്കലും സത്യവിശ്വാസികളുടെ അടുക്കലും വലിയ കോപഹേതുവായിരിക്കുന്നു. അപ്രകാരം അഹങ്കാരികളും ഗര്‍വ്വിഷ്ഠരും ആയിട്ടുള്ളവരുടെ ഹൃദയങ്ങളിലെല്ലാം അല്ലാഹു മുദ്രവെക്കുന്നു.
\end{malayalam}}
\flushright{\begin{Arabic}
\quranayah[40][36]
\end{Arabic}}
\flushleft{\begin{malayalam}
ഫിര്‍ഔന്‍ പറഞ്ഞു. ഹാമാനേ, എനിക്ക് ആ മാര്‍ഗങ്ങളില്‍ എത്താവുന്ന വിധം എനിക്കു വേണ്ടി നീ ഒരു ഉന്നത സൌധം പണിതു തരൂ!
\end{malayalam}}
\flushright{\begin{Arabic}
\quranayah[40][37]
\end{Arabic}}
\flushleft{\begin{malayalam}
അഥവാ ആകാശമാര്‍ഗങ്ങളില്‍. എന്നിട്ടു മൂസായുടെ ദൈവത്തിന്‍റെ അടുത്തേക്ക് എത്തിനോക്കുവാന്‍. തീര്‍ച്ചയായും അവന്‍ (മൂസാ) കളവു പറയുകയാണെന്നാണ് ഞാന്‍ വിചാരിക്കുന്നത്‌. അപ്രകാരം ഫിര്‍ഔന് തന്‍റെ ദുഷ്പ്രവൃത്തി അലംകൃതമായി തോന്നിക്കപ്പെട്ടു. നേരായ മാര്‍ഗത്തില്‍ നിന്ന് അവന്‍ തടയപ്പെടുകയും ചെയ്തു. ഫറോവയുടെ തന്ത്രം നഷ്ടത്തില്‍ തന്നെയായിരുന്നു.
\end{malayalam}}
\flushright{\begin{Arabic}
\quranayah[40][38]
\end{Arabic}}
\flushleft{\begin{malayalam}
ആ വിശ്വസിച്ച വ്യക്തി പറഞ്ഞു: എന്‍റെ ജനങ്ങളേ, നിങ്ങള്‍ എന്നെ പിന്തുടരൂ. ഞാന്‍ നിങ്ങള്‍ക്ക് വിവേകത്തിന്‍റെ മാര്‍ഗം കാട്ടിത്തരാം.
\end{malayalam}}
\flushright{\begin{Arabic}
\quranayah[40][39]
\end{Arabic}}
\flushleft{\begin{malayalam}
എന്‍റെ ജനങ്ങളേ, ഈ ഐഹികജീവിതം ഒരു താല്‍ക്കാലിക വിഭവം മാത്രമാണ്‌. തീര്‍ച്ചയായും പരലോകം തന്നെയാണ് സ്ഥിരവാസത്തിനുള്ള ഭവനം.
\end{malayalam}}
\flushright{\begin{Arabic}
\quranayah[40][40]
\end{Arabic}}
\flushleft{\begin{malayalam}
ആരെങ്കിലും ഒരു തിന്‍മപ്രവര്‍ത്തിച്ചാല്‍ തത്തുല്യമായ പ്രതിഫലമേ അവന്നു നല്‍കപ്പെടുകയുള്ളൂ സത്യവിശ്വാസിയായികൊണ്ട് സല്‍കര്‍മ്മം പ്രവര്‍ത്തിക്കുന്നതാരോ -പുരുഷനോ സ്ത്രീയോ ആകട്ടെ- അവര്‍ സ്വര്‍ഗത്തില്‍ പ്രവേശിക്കുന്നതാണ്‌. കണക്കുനോക്കാതെ അവര്‍ക്ക് അവിടെ ഉപജീവനം നല്‍കപ്പെട്ടുകൊണ്ടിരിക്കും.
\end{malayalam}}
\flushright{\begin{Arabic}
\quranayah[40][41]
\end{Arabic}}
\flushleft{\begin{malayalam}
എന്‍റെ ജനങ്ങളേ, എനിക്കെന്തൊരനുഭവം! ഞാന്‍ നിങ്ങളെ രക്ഷയിലേക്ക് ക്ഷണിക്കുന്നു. നിങ്ങളാകട്ടെ എന്നെ നരകത്തിലേക്കും ക്ഷണിക്കുന്നു.
\end{malayalam}}
\flushright{\begin{Arabic}
\quranayah[40][42]
\end{Arabic}}
\flushleft{\begin{malayalam}
ഞാന്‍ അല്ലാഹുവില്‍ അവിശ്വസിക്കുവാനും എനിക്ക് യാതൊരു അറിവുമില്ലാത്തത് അവനോട് ഞാന്‍ പങ്കുചേര്‍ക്കുവാനും നിങ്ങളെന്നെ ക്ഷണിക്കുന്നു. ഞാനാകട്ടെ, പ്രതാപശാലിയും ഏറെ പൊറുക്കുന്നവനുമായ അല്ലാഹുവിലേക്ക് നിങ്ങളെ ക്ഷണിക്കുന്നു.
\end{malayalam}}
\flushright{\begin{Arabic}
\quranayah[40][43]
\end{Arabic}}
\flushleft{\begin{malayalam}
നിങ്ങള്‍ എന്നെ ഏതൊന്നിലേക്ക് ക്ഷണിച്ചു കൊണ്ടിരിക്കുന്നുവോ അതിന് ഇഹലോകത്താകട്ടെ പരലോകത്താകട്ടെ യാതൊരു പ്രാര്‍ത്ഥനയും ഉണ്ടാകാവുന്നതല്ല എന്നതും, നമ്മുടെ മടക്കം അല്ലാഹുവിങ്കലേക്കാണ് എന്നതും, അതിക്രമകാരികള്‍ തന്നെയാണ് നരകാവകാശികള്‍ എന്നതും ഉറപ്പായ കാര്യമാകുന്നു.
\end{malayalam}}
\flushright{\begin{Arabic}
\quranayah[40][44]
\end{Arabic}}
\flushleft{\begin{malayalam}
എന്നാല്‍ ഞാന്‍ നിങ്ങളോട് പറയുന്നത് വഴിയെ നിങ്ങള്‍ ഓര്‍ക്കും. എന്‍റെ കാര്യം ഞാന്‍ അല്ലാഹുവിങ്കലേക്ക് ഏല്‍പിച്ച് വിടുന്നു. തീര്‍ച്ചയായും അല്ലാഹു ദാസന്‍മാരെപ്പറ്റി കണ്ടറിയുന്നവനാകുന്നു.
\end{malayalam}}
\flushright{\begin{Arabic}
\quranayah[40][45]
\end{Arabic}}
\flushleft{\begin{malayalam}
അപ്പോള്‍ അവര്‍ നടത്തിയ കുതന്ത്രങ്ങളുടെ ദുഷ്ഫലങ്ങളില്‍ നിന്ന് അല്ലാഹു അദ്ദേഹത്തെ കാത്തു. ഫിര്‍ഔന്‍റെ ആളുകളെ കടുത്ത ശിക്ഷ വലയം ചെയ്യുകയുമുണ്ടായി.
\end{malayalam}}
\flushright{\begin{Arabic}
\quranayah[40][46]
\end{Arabic}}
\flushleft{\begin{malayalam}
നരകം! രാവിലെയും വൈകുന്നേരവും അവര്‍ അതിനുമുമ്പില്‍ പ്രദര്‍ശിപ്പിക്കപ്പെടും. ആ അന്ത്യസമയം നിലവില്‍ വരുന്ന ദിവസം ഫിര്‍ഔന്‍റെ ആളുകളെ ഏറ്റവും കഠിനമായ ശിക്ഷയില്‍ നിങ്ങള്‍ പ്രവേശിപ്പിക്കുക. (എന്ന് കല്‍പിക്കപ്പെടും)
\end{malayalam}}
\flushright{\begin{Arabic}
\quranayah[40][47]
\end{Arabic}}
\flushleft{\begin{malayalam}
നരകത്തില്‍ അവര്‍ അന്യോന്യം ന്യായവാദം നടത്തുന്ന സന്ദര്‍ഭം (ശ്രദ്ധേയമാകുന്നു.) അപ്പോള്‍ ദുര്‍ബലര്‍ അഹംഭാവം നടിച്ചവരോട് പറയും: തീര്‍ച്ചയായും ഞങ്ങള്‍ നിങ്ങളെ പിന്തുടര്‍ന്ന് ജീവിക്കുകയായിരുന്നു. അതിനാല്‍ നരകശിക്ഷയില്‍ നിന്നുള്ള വല്ല വിഹിതവും ഞങ്ങളില്‍ നിന്ന് ഒഴിവാക്കിത്തരാന്‍ നിങ്ങള്‍ക്ക് കഴിയുമോ?
\end{malayalam}}
\flushright{\begin{Arabic}
\quranayah[40][48]
\end{Arabic}}
\flushleft{\begin{malayalam}
അഹംഭാവം നടിച്ചവര്‍ പറയും: തീര്‍ച്ചയായും നമ്മളെല്ലാം ഇതില്‍ തന്നെയാകുന്നു. തീര്‍ച്ചയായും അല്ലാഹു ദാസന്‍മാര്‍ക്കിടയില്‍ വിധി കല്‍പിച്ചു കഴിഞ്ഞു.
\end{malayalam}}
\flushright{\begin{Arabic}
\quranayah[40][49]
\end{Arabic}}
\flushleft{\begin{malayalam}
നരകത്തിലുള്ളവര്‍ നരകത്തിന്‍റെ കാവല്‍ക്കാരോട് പറയും: നിങ്ങള്‍ നിങ്ങളുടെ രക്ഷിതാവിനോടൊന്ന് പ്രാര്‍ത്ഥിക്കുക. ഞങ്ങള്‍ക്ക് ഒരു ദിവസത്തെ ശിക്ഷയെങ്കിലും അവന്‍ ലഘൂകരിച്ചു തരട്ടെ.
\end{malayalam}}
\flushright{\begin{Arabic}
\quranayah[40][50]
\end{Arabic}}
\flushleft{\begin{malayalam}
അവര്‍ (കാവല്‍ക്കാര്‍) പറയും: നിങ്ങളിലേക്കുള്ള ദൂതന്‍മാര്‍ വ്യക്തമായ തെളിവുകളും കൊണ്ട് നിങ്ങളുടെ അടുത്ത് വന്നിട്ടുണ്ടായിരുന്നില്ലേ? അവര്‍ പറയും: അതെ. അവര്‍ (കാവല്‍ക്കാര്‍) പറയും: എന്നാല്‍ നിങ്ങള്‍ തന്നെ പ്രാര്‍ത്ഥിച്ചു കൊള്ളുക. സത്യനിഷേധികളുടെ പ്രാര്‍ത്ഥന വൃഥാവിലായിപ്പോകുകയേയുള്ളൂ.
\end{malayalam}}
\flushright{\begin{Arabic}
\quranayah[40][51]
\end{Arabic}}
\flushleft{\begin{malayalam}
തീര്‍ച്ചയായും നാം നമ്മുടെ ദൂതന്‍മാരെയും വിശ്വസിച്ചവരെയും ഐഹികജീവിതത്തിലും സാക്ഷികള്‍ രംഗത്തു വരുന്ന ദിവസത്തിലും സഹായിക്കുക തന്നെ ചെയ്യും.
\end{malayalam}}
\flushright{\begin{Arabic}
\quranayah[40][52]
\end{Arabic}}
\flushleft{\begin{malayalam}
അതായത് അക്രമികള്‍ക്ക് അവരുടെ ഒഴികഴിവ് പ്രയോജനപ്പെടാത്ത ദിവസം. അവര്‍ക്കാകുന്നു ശാപം. അവര്‍ക്കാകുന്നു ചീത്തഭവനം.
\end{malayalam}}
\flushright{\begin{Arabic}
\quranayah[40][53]
\end{Arabic}}
\flushleft{\begin{malayalam}
മൂസായ്ക്ക് നാം മാര്‍ഗദര്‍ശനം നല്‍കുകയും, ഇസ്രായീല്യരെ നാം വേദഗ്രന്ഥത്തിന്‍റെ അവകാശികളാക്കിത്തീര്‍ക്കുകയും ചെയ്തു.
\end{malayalam}}
\flushright{\begin{Arabic}
\quranayah[40][54]
\end{Arabic}}
\flushleft{\begin{malayalam}
ബുദ്ധിയുള്ളവര്‍ക്ക് മാര്‍ഗദര്‍ശനവും ഉല്‍ബോധനവുമായിരുന്നു അത്‌.
\end{malayalam}}
\flushright{\begin{Arabic}
\quranayah[40][55]
\end{Arabic}}
\flushleft{\begin{malayalam}
അതിനാല്‍ നീ ക്ഷമിക്കുക. തീര്‍ച്ചയായും അല്ലാഹുവിന്‍റെ വാഗ്ദാനം സത്യമാകുന്നു. നിന്‍റെ പാപത്തിന് നീ മാപ്പുതേടുകയും വൈകുന്നേരവും രാവിലെയും നിന്‍റെ രക്ഷിതാവിനെ സ്തുതിക്കുന്നതോടൊപ്പം പ്രകീര്‍ത്തിക്കുകയും ചെയ്യുക.
\end{malayalam}}
\flushright{\begin{Arabic}
\quranayah[40][56]
\end{Arabic}}
\flushleft{\begin{malayalam}
തങ്ങള്‍ക്ക് യാതൊരു പ്രമാണവും വന്നുകിട്ടാതെ അല്ലാഹുവിന്‍റെ ദൃഷ്ടാന്തങ്ങളെപ്പറ്റി തര്‍ക്കിക്കുന്നതാരോ അവരുടെ ഹൃദയങ്ങളില്‍ തീര്‍ച്ചയായും അഹങ്കാരം മാത്രമേയുള്ളൂ. അവര്‍ അവിടെ എത്തുന്നതേ അല്ല. അതുകൊണ്ട് നീ അല്ലാഹുവോട് ശരണം തേടുക. തീര്‍ച്ചയായും അവനാണ് എല്ലാം കേള്‍ക്കുന്നവനും കാണുന്നവനും.
\end{malayalam}}
\flushright{\begin{Arabic}
\quranayah[40][57]
\end{Arabic}}
\flushleft{\begin{malayalam}
തീര്‍ച്ചയായും ആകാശങ്ങളും ഭൂമിയും സൃഷ്ടിക്കുക എന്നതാണ് മനുഷ്യനെ സൃഷ്ടിക്കുന്നതിനെക്കാള്‍ വലിയ കാര്യം. പക്ഷെ, അവരില്‍ അധികപേരും മനസ്സിലാക്കുന്നില്ല.
\end{malayalam}}
\flushright{\begin{Arabic}
\quranayah[40][58]
\end{Arabic}}
\flushleft{\begin{malayalam}
അന്ധനും കാഴ്ചയുള്ളവനും സമമാകുകയില്ല. വിശ്വസിച്ച് സല്‍കര്‍മ്മങ്ങള്‍ ചെയ്തവരും ദുഷ്കൃത്യം ചെയ്തവരും സമമാകുകയില്ല. ചുരുക്കത്തില്‍ മാത്രമേ നിങ്ങള്‍ ആലോചിച്ചു മനസ്സിലാക്കുന്നുള്ളൂ.
\end{malayalam}}
\flushright{\begin{Arabic}
\quranayah[40][59]
\end{Arabic}}
\flushleft{\begin{malayalam}
ആ അന്ത്യസമയം വരാനുള്ളത് തന്നെയാണ്‌. അതില്‍ സംശയമേ ഇല്ല. പക്ഷെ മനുഷ്യരില്‍ അധികപേരും വിശ്വസിക്കുന്നില്ല.
\end{malayalam}}
\flushright{\begin{Arabic}
\quranayah[40][60]
\end{Arabic}}
\flushleft{\begin{malayalam}
നിങ്ങളുടെ രക്ഷിതാവ് പറഞ്ഞിരിക്കുന്നു: നിങ്ങള്‍ എന്നോട് പ്രാര്‍ത്ഥിക്കൂ. ഞാന്‍ നിങ്ങള്‍ക്ക് ഉത്തരം നല്‍കാം. എന്നെ ആരാധിക്കാതെ അഹങ്കാരം നടിക്കുന്നവരാരോ അവര്‍ വഴിയെ നിന്ദ്യരായിക്കൊണ്ട് നരകത്തില്‍ പ്രവേശിക്കുന്നതാണ്‌; തീര്‍ച്ച.
\end{malayalam}}
\flushright{\begin{Arabic}
\quranayah[40][61]
\end{Arabic}}
\flushleft{\begin{malayalam}
അല്ലാഹുവാകുന്നു നിങ്ങള്‍ക്കു വേണ്ടി രാത്രിയെ നിങ്ങള്‍ക്കു ശാന്തമായി വസിക്കാന്‍ തക്കവണ്ണവും, പകലിനെ വെളിച്ചമുള്ളതും ആക്കിയവന്‍. തീര്‍ച്ചയായും അല്ലാഹു ജനങ്ങളോട് ഔദാര്യമുള്ളവനാകുന്നു. പക്ഷെ മനുഷ്യരില്‍ അധികപേരും നന്ദികാണിക്കുന്നില്ല.
\end{malayalam}}
\flushright{\begin{Arabic}
\quranayah[40][62]
\end{Arabic}}
\flushleft{\begin{malayalam}
അവനാകുന്നു നിങ്ങളുടെ രക്ഷിതാവും എല്ലാ വസ്തുക്കളുടെയും സൃഷ്ടികര്‍ത്താവുമായ അല്ലാഹു. അവനല്ലാതെ യാതൊരു ദൈവവുമില്ല. എന്നിരിക്കെ നിങ്ങള്‍ എങ്ങനെയാണ് (സന്‍മാര്‍ഗത്തില്‍ നിന്ന്‌) തെറ്റിക്കപ്പെടുന്നത്‌?
\end{malayalam}}
\flushright{\begin{Arabic}
\quranayah[40][63]
\end{Arabic}}
\flushleft{\begin{malayalam}
അപ്രകാരം തന്നെയാണ് അല്ലാഹുവിന്‍റെ ദൃഷ്ടാന്തങ്ങളെ നിഷേധിച്ചിരുന്നവര്‍ (സന്‍മാര്‍ഗത്തില്‍ നിന്ന്‌) തെറ്റിക്കപ്പെടുന്നത്‌.
\end{malayalam}}
\flushright{\begin{Arabic}
\quranayah[40][64]
\end{Arabic}}
\flushleft{\begin{malayalam}
അല്ലാഹുവാകുന്നു നിങ്ങള്‍ക്ക് വേണ്ടി ഭൂമിയെ വാസസ്ഥലവും ആകാശത്തെ മേല്‍പുരയും ആക്കിയവന്‍. അവന്‍ നിങ്ങളെ രൂപപ്പെടുത്തുകയും ചെയ്തു. അങ്ങനെ അവന്‍ നിങ്ങളുടെ രൂപങ്ങള്‍ മികച്ചതാക്കി. വിശിഷ്ട വസ്തുക്കളില്‍ നിന്ന് അവന്‍ നിങ്ങള്‍ക്ക് ഉപജീവനം നല്‍കുകയും ചെയ്തു. അവനാകുന്നു നിങ്ങളുടെ രക്ഷിതാവായ അല്ലാഹു. അപ്പോള്‍ ലോകങ്ങളുടെ രക്ഷിതാവായ അല്ലാഹു അനുഗ്രഹപൂര്‍ണ്ണനായിരിക്കുന്നു.
\end{malayalam}}
\flushright{\begin{Arabic}
\quranayah[40][65]
\end{Arabic}}
\flushleft{\begin{malayalam}
അവനാകുന്നു ജീവിച്ചിരിക്കുന്നവന്‍. അവനല്ലാതെ യാതൊരു ദൈവവുമില്ല. അതിനാല്‍ കീഴ്‌വണക്കം അവന് നിഷ്കളങ്കമാക്കിക്കൊണ്ട് നിങ്ങള്‍ അവനോട് പ്രാര്‍ത്ഥിക്കുക. ലോകങ്ങളുടെ രക്ഷിതാവായ അല്ലാഹുവിന്ന് സ്തുതി.
\end{malayalam}}
\flushright{\begin{Arabic}
\quranayah[40][66]
\end{Arabic}}
\flushleft{\begin{malayalam}
(നബിയേ,) പറയുക: എന്‍റെ രക്ഷിതാവിങ്കല്‍ നിന്ന് എനിക്ക് തെളിവുകള്‍ വന്നുകിട്ടിയിരിക്കെ അല്ലാഹുവിന് പുറമെ നിങ്ങള്‍ വിളിച്ചു പ്രാര്‍ത്ഥിക്കുന്നവരെ ആരാധിക്കുന്നതില്‍ നിന്ന് തീര്‍ച്ചയായും ഞാന്‍ വിലക്കപ്പെട്ടിരിക്കുന്നു. ലോകങ്ങളുടെ രക്ഷിതാവിന് ഞാന്‍ കീഴ്പെടണമെന്ന് കല്‍പിക്കപ്പെടുകയും ചെയ്തിരിക്കുന്നു.
\end{malayalam}}
\flushright{\begin{Arabic}
\quranayah[40][67]
\end{Arabic}}
\flushleft{\begin{malayalam}
മണ്ണില്‍ നിന്നും, പിന്നെ ബീജകണത്തില്‍ നിന്നും, പിന്നെ ഭ്രൂണത്തില്‍ നിന്നും നിങ്ങളെ സൃഷ്ടിച്ചത് അവനാകുന്നു. പിന്നീട് ഒരു ശിശുവായി നിങ്ങളെ അവന്‍ പുറത്തു കൊണ്ട് വരുന്നു. പിന്നീട് നിങ്ങള്‍ നിങ്ങളുടെ പൂര്‍ണ്ണശക്തി പ്രാപിക്കുവാനും പിന്നീട് നിങ്ങള്‍ വൃദ്ധരായിത്തീരുവാനും വേണ്ടി. നിങ്ങളില്‍ ചിലര്‍ മുമ്പേതന്നെ മരണമടയുന്നു. നിര്‍ണിതമായ ഒരു അവധിയില്‍ നിങ്ങള്‍ എത്തിച്ചേരുവാനും നിങ്ങള്‍ ഒരു വേള ചിന്തിക്കുന്നതിനും വേണ്ടി.
\end{malayalam}}
\flushright{\begin{Arabic}
\quranayah[40][68]
\end{Arabic}}
\flushleft{\begin{malayalam}
അവനാണ് ജീവിപ്പിക്കുകയും മരിപ്പിക്കുകയും ചെയ്യുന്നവന്‍. ഒരു കാര്യം അവന്‍ തീരുമാനിച്ചു കഴിഞ്ഞാല്‍ ഉണ്ടാകൂ എന്ന് അതിനോട് അവന്‍ പറയുക മാത്രം ചെയ്യുന്നു. അപ്പോള്‍ അത് ഉണ്ടാകുന്നു.
\end{malayalam}}
\flushright{\begin{Arabic}
\quranayah[40][69]
\end{Arabic}}
\flushleft{\begin{malayalam}
അല്ലാഹുവിന്‍റെ ദൃഷ്ടാന്തങ്ങളെപ്പറ്റി തര്‍ക്കിക്കുന്നവരുടെ നേര്‍ക്ക് നീ നോക്കിയില്ലേ? എങ്ങനെയാണ് അവര്‍ വ്യതിചലിപ്പിക്കപ്പെടുന്നത് എന്ന്‌.
\end{malayalam}}
\flushright{\begin{Arabic}
\quranayah[40][70]
\end{Arabic}}
\flushleft{\begin{malayalam}
വേദഗ്രന്ഥത്തെയും, നാം നമ്മുടെ ദൂതന്‍മാരെ അയച്ചത് എന്തൊരു ദൌത്യം കൊണ്ടാണോ അതിനെയും നിഷേധിച്ചു കളഞ്ഞവരത്രെ അവര്‍. എന്നാല്‍ വഴിയെ അവര്‍ അറിഞ്ഞു കൊള്ളും.
\end{malayalam}}
\flushright{\begin{Arabic}
\quranayah[40][71]
\end{Arabic}}
\flushleft{\begin{malayalam}
അതെ; അവരുടെ കഴുത്തുകളില്‍ കുരുക്കുകളും ചങ്ങലകളുമായി അവര്‍ വലിച്ചിഴക്കപ്പെടുന്ന സന്ദര്‍ഭം.
\end{malayalam}}
\flushright{\begin{Arabic}
\quranayah[40][72]
\end{Arabic}}
\flushleft{\begin{malayalam}
ചുട്ടുതിളക്കുന്ന വെള്ളത്തിലൂടെ. പിന്നീട് അവര്‍ നരകാഗ്നിയില്‍ എരിക്കപ്പെടുകയും ചെയ്യും.
\end{malayalam}}
\flushright{\begin{Arabic}
\quranayah[40][73]
\end{Arabic}}
\flushleft{\begin{malayalam}
പിന്നീട് അവരോട് പറയപ്പെടും: നിങ്ങള്‍ പങ്കാളികളായി ചേര്‍ത്തിരുന്നവര്‍ എവിടെയാകുന്നു?
\end{malayalam}}
\flushright{\begin{Arabic}
\quranayah[40][74]
\end{Arabic}}
\flushleft{\begin{malayalam}
അല്ലാഹുവിന് പുറമെ. അവര്‍ പറയും: അവര്‍ ഞങ്ങളെ വിട്ട് അപ്രത്യക്ഷരായിരിക്കുന്നു. അല്ല, ഞങ്ങള്‍ മുമ്പ് പ്രാര്‍ത്ഥിച്ചിരുന്നത് യാതൊന്നിനോടുമായിരുന്നില്ല. അപ്രകാരം അല്ലാഹു സത്യനിഷേധികളെ പിഴവിലാക്കുന്നു.
\end{malayalam}}
\flushright{\begin{Arabic}
\quranayah[40][75]
\end{Arabic}}
\flushleft{\begin{malayalam}
ന്യായമില്ലാതെ നിങ്ങള്‍ ഭൂമിയില്‍ ആഹ്ലാദം കൊണ്ടിരുന്നതിന്‍റെയും, ഗര്‍വ്വ് നടിച്ചിരുന്നതിന്‍റെയും ഫലമത്രെ അത്‌.
\end{malayalam}}
\flushright{\begin{Arabic}
\quranayah[40][76]
\end{Arabic}}
\flushleft{\begin{malayalam}
നരകത്തിന്‍റെ കവാടങ്ങളിലൂടെ അതില്‍ നിത്യവാസികളെന്ന നിലയില്‍ നിങ്ങള്‍ കടന്നു കൊള്ളുക. അഹങ്കാരികളുടെ പാര്‍പ്പിടം ചീത്ത തന്നെ. (എന്ന് അവരോട് പറയപ്പെടും.)
\end{malayalam}}
\flushright{\begin{Arabic}
\quranayah[40][77]
\end{Arabic}}
\flushleft{\begin{malayalam}
അതിനാല്‍ നീ ക്ഷമിക്കുക. തീര്‍ച്ചയായും അല്ലാഹുവിന്‍റെ വാഗ്ദാനം സത്യമാകുന്നു. എന്നാല്‍ നാം അവര്‍ക്ക് താക്കീത് നല്‍കുന്ന ശിക്ഷയില്‍ ചിലത് നിനക്ക് നാം കാണിച്ചുതരുന്നതായാലും (അതിന്നിടക്കു തന്നെ) നിന്നെ നാം മരിപ്പിക്കുന്നതായാലും നമ്മുടെ അടുത്തേക്ക് തന്നെയാണ് അവര്‍ മടക്കപ്പെടുന്നത്‌.
\end{malayalam}}
\flushright{\begin{Arabic}
\quranayah[40][78]
\end{Arabic}}
\flushleft{\begin{malayalam}
നിനക്ക് മുമ്പ് നാം പല ദൂതന്‍മാരെയും അയച്ചിട്ടുണ്ട്‌. അവരില്‍ ചിലരെപ്പറ്റി നാം നിനക്ക് വിവരിച്ചുതന്നിട്ടുണ്ട്‌. അവരില്‍ ചിലരെപ്പറ്റി നിനക്ക് നാം വിവരിച്ചുതന്നിട്ടില്ല. യാതൊരു ദൂതന്നും അല്ലാഹുവിന്‍റെ അനുമതിയോട് കൂടിയല്ലാതെ ഒരു ദൃഷ്ടാന്തം കൊണ്ടു വരാനാവില്ല. എന്നാല്‍ അല്ലാഹുവിന്‍റെ കല്‍പന വന്നാല്‍ ന്യായപ്രകാരം വിധിക്കപ്പെടുന്നതാണ്‌. അസത്യവാദികള്‍ അവിടെ നഷ്ടത്തിലാവുകയും ചെയ്യും.
\end{malayalam}}
\flushright{\begin{Arabic}
\quranayah[40][79]
\end{Arabic}}
\flushleft{\begin{malayalam}
അല്ലാഹുവാകുന്നു നിങ്ങള്‍ക്ക് വേണ്ടി കന്നുകാലികളെ സൃഷ്ടിച്ചു തന്നവന്‍. അവയില്‍ ചിലതിനെ നിങ്ങള്‍ വാഹനമായി ഉപയോഗിക്കുന്നതിന് വേണ്ടി. അവയില്‍ ചിലതിനെ നിങ്ങള്‍ ഭക്ഷിക്കുകയും ചെയ്യുന്നു.
\end{malayalam}}
\flushright{\begin{Arabic}
\quranayah[40][80]
\end{Arabic}}
\flushleft{\begin{malayalam}
നിങ്ങള്‍ക്ക് അവയില്‍ പല പ്രയോജനങ്ങളുമുണ്ട്‌. അവ മുഖേന നിങ്ങളുടെ ഹൃദയങ്ങളിലുള്ള വല്ല ആവശ്യത്തിലും നിങ്ങള്‍ എത്തിച്ചേരുകയും ചെയ്യുന്നു. അവയുടെ പുറത്തും കപ്പലുകളിലുമായി നിങ്ങള്‍ വഹിക്കപ്പെടുകയും ചെയ്യുന്നു.
\end{malayalam}}
\flushright{\begin{Arabic}
\quranayah[40][81]
\end{Arabic}}
\flushleft{\begin{malayalam}
അവന്‍റെ ദൃഷ്ടാന്തങ്ങള്‍ അവന്‍ നിങ്ങള്‍ക്ക് കാണിച്ചുതരികയും ചെയ്യുന്നു. അപ്പോള്‍ അല്ലാഹുവിന്‍റെ ദൃഷ്ടാന്തങ്ങളില്‍ എതൊന്നിനെയാണ് നിങ്ങള്‍ നിഷേധിക്കുന്നത്‌?
\end{malayalam}}
\flushright{\begin{Arabic}
\quranayah[40][82]
\end{Arabic}}
\flushleft{\begin{malayalam}
എന്നാല്‍ അവര്‍ക്ക് മുമ്പുണ്ടായിരുന്നവരുടെ പര്യവസാനം എങ്ങനെയായിരുന്നു എന്ന് കാണാന്‍ അവര്‍ ഭൂമിയില്‍ സഞ്ചരിച്ചു നോക്കിയിട്ടില്ലേ? അവര്‍ ഇവരെക്കാള്‍ എണ്ണം കൂടിയവരും, ശക്തികൊണ്ടും ഭൂമിയില്‍ വിട്ടേച്ചുപോയ അവശിഷ്ടങ്ങള്‍ കൊണ്ടും ഏറ്റവും പ്രബലന്‍മാരുമായിരുന്നു. എന്നിട്ടും അവര്‍ നേടിയെടുത്തിരുന്നതൊന്നും അവര്‍ക്ക് പ്രയോജനപ്പെട്ടില്ല.
\end{malayalam}}
\flushright{\begin{Arabic}
\quranayah[40][83]
\end{Arabic}}
\flushleft{\begin{malayalam}
അങ്ങനെ അവരിലേക്കുള്ള ദൂതന്‍മാര്‍ വ്യക്തമായ തെളിവുകളും കൊണ്ട് അവരുടെ അടുത്ത് ചെന്നപ്പോള്‍ അവരുടെ പക്കലുള്ള അറിവുകൊണ്ട് അവര്‍ തൃപ്തിയടയുകയാണ് ചെയ്തത്‌. എന്തൊന്നിനെപ്പറ്റി അവര്‍ പരിഹസിച്ചിരുന്നുവോ അത് (ശിക്ഷ) അവരെ വലയം ചെയ്യുകയുമുണ്ടായി.
\end{malayalam}}
\flushright{\begin{Arabic}
\quranayah[40][84]
\end{Arabic}}
\flushleft{\begin{malayalam}
എന്നിട്ട് നമ്മുടെ ശിക്ഷ കണ്ടപ്പോള്‍ അവര്‍ പറഞ്ഞു: ഞങ്ങള്‍ അല്ലാഹുവില്‍ മാത്രം വിശ്വസിക്കുകയും അവനോട് ഞങ്ങള്‍ പങ്കുചേര്‍ത്തിരുന്നതില്‍ (ദൈവങ്ങളില്‍) ഞങ്ങള്‍ അവിശ്വസിക്കുകയും ചെയ്തിരിക്കുന്നു.
\end{malayalam}}
\flushright{\begin{Arabic}
\quranayah[40][85]
\end{Arabic}}
\flushleft{\begin{malayalam}
എന്നാല്‍ അവര്‍ നമ്മുടെ ശിക്ഷ കണ്ടപ്പോഴത്തെ അവരുടെ വിശ്വാസം അവര്‍ക്ക് പ്രയോജനപ്പെടുകയുണ്ടായില്ല. അല്ലാഹു തന്‍റെ ദാസന്‍മാരുടെ കാര്യത്തില്‍ മുമ്പേ നടപ്പിലാക്കി കഴിഞ്ഞിട്ടുള്ള നടപടിക്രമമത്രെ അത്‌. അവിടെ സത്യനിഷേധികള്‍ നഷ്ടത്തിലാവുകയും ചെയ്തു.
\end{malayalam}}
\chapter{\textmalayalam{ഫുസ്സിലത്ത്}}
\begin{Arabic}
\Huge{\centerline{\basmalah}}\end{Arabic}
\flushright{\begin{Arabic}
\quranayah[41][1]
\end{Arabic}}
\flushleft{\begin{malayalam}
ഹാമീം.
\end{malayalam}}
\flushright{\begin{Arabic}
\quranayah[41][2]
\end{Arabic}}
\flushleft{\begin{malayalam}
പരമകാരുണികനും കരുണാനിധിയുമായിട്ടുള്ളവന്‍റെ പക്കല്‍ നിന്ന് അവതരിപ്പിക്കപ്പെട്ടതത്രെ ഇത്‌.
\end{malayalam}}
\flushright{\begin{Arabic}
\quranayah[41][3]
\end{Arabic}}
\flushleft{\begin{malayalam}
വചനങ്ങള്‍ വിശദീകരിക്കപ്പെട്ട ഒരു വേദഗ്രന്ഥം. മനസ്സിലാക്കുന്ന ആളുകള്‍ക്ക് വേണ്ടി അറബിഭാഷയില്‍ പാരായണം ചെയ്യപ്പെടുന്ന (ഒരു ഗ്രന്ഥം.)
\end{malayalam}}
\flushright{\begin{Arabic}
\quranayah[41][4]
\end{Arabic}}
\flushleft{\begin{malayalam}
സന്തോഷവാര്‍ത്ത അറിയിക്കുന്നതും താക്കീത് നല്‍കുന്നതുമായിട്ടുള്ള (ഗ്രന്ഥം) എന്നാല്‍ അവരില്‍ അധികപേരും തിരിഞ്ഞുകളഞ്ഞു. അവര്‍ കേട്ട് മനസ്സിലാക്കുന്നില്ല.
\end{malayalam}}
\flushright{\begin{Arabic}
\quranayah[41][5]
\end{Arabic}}
\flushleft{\begin{malayalam}
അവര്‍ പറഞ്ഞു: നീ ഞങ്ങളെ എന്തൊന്നിലേക്ക് വിളിക്കുന്നുവോ അത് മനസ്സിലാക്കാനാവാത്ത വിധം ഞങ്ങളുടെ ഹൃദയങ്ങള്‍ മൂടികള്‍ക്കുള്ളിലാകുന്നു. ഞങ്ങളുടെ കാതുകള്‍ക്ക് ബധിരതയുമാകുന്നു. ഞങ്ങള്‍ക്കും നിനക്കുമിടയില്‍ ഒരു മറയുണ്ട്‌. അതിനാല്‍ നീ പ്രവര്‍ത്തിച്ച് കൊള്ളുക. തീര്‍ച്ചയായും ഞങ്ങളും പ്രവര്‍ത്തിക്കുന്നവരാകുന്നു.
\end{malayalam}}
\flushright{\begin{Arabic}
\quranayah[41][6]
\end{Arabic}}
\flushleft{\begin{malayalam}
നീ പറയുക: ഞാന്‍ നിങ്ങളെപ്പോലെ ഒരു മനുഷ്യന്‍ മാത്രമാകുന്നു. നിങ്ങളുടെ ദൈവം ഏകദൈവമാകുന്നു എന്ന് എനിക്ക് ബോധനം നല്‍കപ്പെടുന്നു. ആകയാല്‍ അവങ്കലേക്കുള്ള മാര്‍ഗത്തില്‍ നിങ്ങള്‍ നേരെ നിലകൊള്ളുകയും അവനോട് നിങ്ങള്‍ പാപമോചനം തേടുകയും ചെയ്യുവിന്‍. ബഹുദൈവാരാധകര്‍ക്കാകുന്നു നാശം.
\end{malayalam}}
\flushright{\begin{Arabic}
\quranayah[41][7]
\end{Arabic}}
\flushleft{\begin{malayalam}
സകാത്ത് നല്‍കാത്തവരും പരലോകത്തില്‍ വിശ്വാസമില്ലാത്തവരുമായ.
\end{malayalam}}
\flushright{\begin{Arabic}
\quranayah[41][8]
\end{Arabic}}
\flushleft{\begin{malayalam}
തീര്‍ച്ചയായും വിശ്വസിക്കുകയും സല്‍കര്‍മ്മങ്ങള്‍ പ്രവര്‍ത്തിക്കുകയും ചെയ്തവരാരോ അവര്‍ക്കാണ് മുറിഞ്ഞ് പോവാത്ത പ്രതിഫലമുള്ളത്‌.
\end{malayalam}}
\flushright{\begin{Arabic}
\quranayah[41][9]
\end{Arabic}}
\flushleft{\begin{malayalam}
നീ പറയുക: രണ്ടുദിവസ(ഘട്ട)ങ്ങളിലായി ഭൂമിയെ സൃഷ്ടിച്ചവനില്‍ നിങ്ങള്‍ അവിശ്വസിക്കുകയും അവന്ന് നിങ്ങള്‍ സമന്‍മാരെ സ്ഥാപിക്കുകയും തന്നെയാണോ ചെയ്യുന്നത്‌? അവനാകുന്നു ലോകങ്ങളുടെ രക്ഷിതാവ്‌.
\end{malayalam}}
\flushright{\begin{Arabic}
\quranayah[41][10]
\end{Arabic}}
\flushleft{\begin{malayalam}
അതില്‍ (ഭൂമിയില്‍) - അതിന്‍റെ ഉപരിഭാഗത്ത് - ഉറച്ചുനില്‍ക്കുന്ന പര്‍വ്വതങ്ങള്‍ അവന്‍ സ്ഥാപിക്കുകയും അതില്‍ അഭിവൃദ്ധിയുണ്ടാക്കുകയും, അതിലെ ആഹാരങ്ങള്‍ അവിടെ വ്യവസ്ഥപ്പെടുത്തി വെക്കുകയും ചെയ്തിരിക്കുന്നു. നാലു ദിവസ(ഘട്ട)ങ്ങളിലായിട്ടാണ് (അവനത് ചെയ്തത്‌.) ആവശ്യപ്പെടുന്നവര്‍ക്ക് വേണ്ടി ശരിയായ അനുപാതത്തില്‍
\end{malayalam}}
\flushright{\begin{Arabic}
\quranayah[41][11]
\end{Arabic}}
\flushleft{\begin{malayalam}
അതിനു പുറമെ അവന്‍ ആകാശത്തിന്‍റെ നേര്‍ക്ക് തിരിഞ്ഞു. അത് ഒരു പുകയായിരുന്നു.എന്നിട്ട് അതിനോടും ഭൂമിയോടും അവന്‍ പറഞ്ഞു: നിങ്ങള്‍ അനുസരണപൂര്‍വ്വമോ നിര്‍ബന്ധിതമായോ വരിക. അവ രണ്ടും പറഞ്ഞു: ഞങ്ങളിതാ അനുസരണമുള്ളവരായി വന്നിരിക്കുന്നു.
\end{malayalam}}
\flushright{\begin{Arabic}
\quranayah[41][12]
\end{Arabic}}
\flushleft{\begin{malayalam}
അങ്ങനെ രണ്ടുദിവസ(ഘട്ട)ങ്ങളിലായി അവയെ അവന്‍ ഏഴുആകാശങ്ങളാക്കിത്തീര്‍ത്തു. ഓരോ ആകാശത്തിലും അതാതിന്‍റെ കാര്യം അവന്‍ നിര്‍ദേശിക്കുകയും ചെയ്തു. സമീപത്തുള്ള ആകാശത്തെ നാം ചില വിളക്കുകള്‍ കൊണ്ട് അലങ്കരിക്കുകയും സംരക്ഷണം ഏര്‍പെടുത്തുകയും ചെയ്തു. പ്രതാപശാലിയും സര്‍വ്വജ്ഞനുമായ അല്ലാഹു വ്യവസ്ഥപ്പെടുത്തിയതത്രെ അത്‌.
\end{malayalam}}
\flushright{\begin{Arabic}
\quranayah[41][13]
\end{Arabic}}
\flushleft{\begin{malayalam}
എന്നിട്ട് അവര്‍ തിരിഞ്ഞുകളയുന്ന പക്ഷം നീ പറഞ്ഞേക്കുക: ആദ്‌, ഥമൂദ് എന്നീ സമുദായങ്ങള്‍ക്ക് നേരിട്ട ഭയങ്കരശിക്ഷ പോലെയുള്ള ഒരു ശിക്ഷയെപ്പറ്റി ഞാനിതാ നിങ്ങള്‍ക്ക് താക്കീത് നല്‍കുന്നു.
\end{malayalam}}
\flushright{\begin{Arabic}
\quranayah[41][14]
\end{Arabic}}
\flushleft{\begin{malayalam}
അവരുടെ മുന്നിലൂടെയും, പിന്നിലൂടെയും ചെന്ന്‌, അല്ലാഹുവെയല്ലാതെ നിങ്ങള്‍ ആരാധിക്കരുത് എന്ന് പറഞ്ഞുകൊണ്ട് അവരുടെ അടുത്ത് ദൈവദൂതന്‍മാര്‍ ചെന്ന സമയത്ത് അവര്‍ പറഞ്ഞു: ഞങ്ങളുടെ രക്ഷിതാവ് ഉദ്ദേശിച്ചിരുന്നെങ്കില്‍ അവന്‍ മലക്കുകളെ ഇറക്കുമായിരുന്നു. അതിനാല്‍ നിങ്ങള്‍ ഏതൊന്നുമായി അയക്കപ്പെട്ടിരിക്കുന്നുവോ, അതില്‍ തീര്‍ച്ചയായും ഞങ്ങള്‍ വിശ്വാസമില്ലാത്തവരാകുന്നു.
\end{malayalam}}
\flushright{\begin{Arabic}
\quranayah[41][15]
\end{Arabic}}
\flushleft{\begin{malayalam}
എന്നാല്‍ ആദ് സമുദായം ന്യായം കൂടാതെ ഭൂമിയില്‍ അഹംഭാവം നടിക്കുകയും ഞങ്ങളെക്കാള്‍ ശക്തിയില്‍ മികച്ചവര്‍ ആരുണ്ട് എന്ന് പറയുകയുമാണ് ചെയ്തത്‌. അവര്‍ക്ക് കണ്ടുകൂടെ; അവരെ സൃഷ്ടിച്ച അല്ലാഹു തന്നെയാണ് അവരെക്കാള്‍ ശക്തിയില്‍ മികച്ചവനെന്ന്‌? നമ്മുടെ ദൃഷ്ടാന്തങ്ങളെ അവര്‍ നിഷേധിച്ച് കളയുകയായിരുന്നു.
\end{malayalam}}
\flushright{\begin{Arabic}
\quranayah[41][16]
\end{Arabic}}
\flushleft{\begin{malayalam}
അങ്ങനെ ദുരിതം പിടിച്ച ഏതാനും ദിവസങ്ങളില്‍ അവരുടെ നേര്‍ക്ക് ഉഗ്രമായ ഒരു ശീതക്കാറ്റ് നാം അയച്ചു. ഐഹികജീവിതത്തില്‍ അവര്‍ക്ക് അപമാനകരമായ ശിക്ഷ നാം ആസ്വദിപ്പിക്കാന്‍ വേണ്ടിയത്രെ അത്‌. എന്നാല്‍ പരലോകത്തിലെ ശിക്ഷയാണ് കൂടുതല്‍ അപമാനകരം. അവര്‍ക്ക് സഹായമൊന്നും നല്‍കപ്പെടുകയുമില്ല.
\end{malayalam}}
\flushright{\begin{Arabic}
\quranayah[41][17]
\end{Arabic}}
\flushleft{\begin{malayalam}
എന്നാല്‍ ഥമൂദ് ഗോത്രമോ, അവര്‍ക്ക് നാം നേര്‍വഴി കാണിച്ചുകൊടുത്തു. അപ്പോള്‍ സന്‍മാര്‍ഗത്തേക്കാളുപരി അന്ധതയെ അവര്‍ പ്രിയങ്കരമായി കരുതുകയാണ് ചെയ്തത്‌. അങ്ങനെ അവര്‍ ചെയ്തുകൊണ്ടിരുന്നതിന്‍റെ ഫലമായി അപമാനകരമായ ഒരു ഭയങ്കര ശിക്ഷ അവരെ പിടികൂടി.
\end{malayalam}}
\flushright{\begin{Arabic}
\quranayah[41][18]
\end{Arabic}}
\flushleft{\begin{malayalam}
വിശ്വസിക്കുകയും ധര്‍മ്മനിഷ്ഠ പുലര്‍ത്തിക്കൊണ്ടിരിക്കുകയും ചെയ്തവരെ നാം രക്ഷപ്പെടുത്തുകയും ചെയ്തു.
\end{malayalam}}
\flushright{\begin{Arabic}
\quranayah[41][19]
\end{Arabic}}
\flushleft{\begin{malayalam}
അല്ലാഹുവിന്‍റെ ശത്രുക്കളെ നരകത്തിലേക്ക് പോകാനായി വിളിച്ചുകൂട്ടുകയും, എന്നിട്ടവരെ തെളിച്ചുകൂട്ടികൊണ്ടുപോകുകയും ചെയ്യുന്ന ദിവസം (ശ്രദ്ധേയമാകുന്നു.)
\end{malayalam}}
\flushright{\begin{Arabic}
\quranayah[41][20]
\end{Arabic}}
\flushleft{\begin{malayalam}
അങ്ങനെ അവര്‍ അവിടെ (നരകത്തില്‍) ചെന്നാല്‍ അവരുടെ കാതും അവരുടെ കണ്ണുകളും അവരുടെ തൊലികളും അവര്‍ക്ക് എതിരായി അവര്‍ പ്രവര്‍ത്തിച്ചിരുന്നതിനെപ്പറ്റി സാക്ഷ്യം വഹിക്കുന്നതാണ്‌.
\end{malayalam}}
\flushright{\begin{Arabic}
\quranayah[41][21]
\end{Arabic}}
\flushleft{\begin{malayalam}
തങ്ങളുടെ തൊലികളോട് അവര്‍ പറയും: നിങ്ങളെന്തിനാണ് ഞങ്ങള്‍ക്കെതിരായി സാക്ഷ്യം വഹിച്ചത്‌? അവ (തൊലികള്‍) പറയും: എല്ലാ വസ്തുക്കളെയും സംസാരിപ്പിച്ച അല്ലാഹു ഞങ്ങളെ സംസാരിപ്പിച്ചതാകുന്നു. ആദ്യതവണ നിങ്ങളെ സൃഷ്ടിച്ചത് അവനാണല്ലോ. അവങ്കലേക്കുതന്നെ നിങ്ങള്‍ മടക്കപ്പെടുകയും ചെയ്യുന്നു.
\end{malayalam}}
\flushright{\begin{Arabic}
\quranayah[41][22]
\end{Arabic}}
\flushleft{\begin{malayalam}
നിങ്ങളുടെ കാതോ നിങ്ങളുടെ കണ്ണുകളോ നിങ്ങളുടെ തൊലികളോ നിങ്ങള്‍ക്ക് എതിരില്‍ സാക്ഷ്യം വഹിക്കുമെന്ന് കരുതി നിങ്ങള്‍ (അവയില്‍ നിന്നും) ഒളിഞ്ഞിരിക്കാറുണ്ടായിരുന്നില്ലല്ലോ. എന്നാല്‍ നിങ്ങള്‍ വിചാരിച്ചത് നിങ്ങള്‍ പ്രവര്‍ത്തിക്കുന്നതില്‍ മിക്കതും അല്ലാഹു അറിയില്ലെന്നാണ്‌.
\end{malayalam}}
\flushright{\begin{Arabic}
\quranayah[41][23]
\end{Arabic}}
\flushleft{\begin{malayalam}
അതത്രെ നിങ്ങളുടെ രക്ഷിതാവിനെപ്പറ്റി നിങ്ങള്‍ ധരിച്ചുവെച്ച ധാരണ: അത് നിങ്ങള്‍ക്ക് നാശം വരുത്തി. അങ്ങനെ നിങ്ങള്‍ നഷ്ടക്കാരില്‍പ്പെട്ടവരായിത്തീര്‍ന്നു.
\end{malayalam}}
\flushright{\begin{Arabic}
\quranayah[41][24]
\end{Arabic}}
\flushleft{\begin{malayalam}
ഇനി അവര്‍ സഹിച്ചു കഴിയുകയാണെങ്കില്‍ ആ നരകം തന്നെയാകുന്നു അവര്‍ക്കുള്ള പാര്‍പ്പിടം. അവര്‍ വിട്ടുവീഴ്ച തേടുകയാണെങ്കിലോ വിട്ടുവീഴ്ച നല്‍കപ്പെടുന്നവരുടെ കൂട്ടത്തില്‍ അവര്‍ പെടുകയുമില്ല.
\end{malayalam}}
\flushright{\begin{Arabic}
\quranayah[41][25]
\end{Arabic}}
\flushleft{\begin{malayalam}
അവര്‍ക്ക് നാം ചില കൂട്ടുകാരെ ഏര്‍പെടുത്തി കൊടുത്തു. എന്നിട്ട് ആ കൂട്ടാളികള്‍ അവര്‍ക്ക് തങ്ങളുടെ മുമ്പിലുള്ളതും പിന്നിലുള്ളതും അലംകൃതമായി തോന്നിച്ചു. ജിന്നുകളില്‍ നിന്നും മനുഷ്യരില്‍ നിന്നും അവര്‍ക്ക് മുമ്പ് കഴിഞ്ഞുപോയിട്ടുള്ള സമുദായങ്ങളുടെ കൂട്ടത്തില്‍ ഇവരുടെ മേലും (ശിക്ഷയെപറ്റിയുള്ള) പ്രഖ്യാപനം സ്ഥിരപ്പെടുകയുണ്ടായി. തീര്‍ച്ചയായും അവര്‍ നഷ്ടം പറ്റിയവരായിരുന്നു.
\end{malayalam}}
\flushright{\begin{Arabic}
\quranayah[41][26]
\end{Arabic}}
\flushleft{\begin{malayalam}
സത്യനിഷേധികള്‍ പറഞ്ഞു: നിങ്ങള്‍ ഈ ഖുര്‍ആന്‍ ശ്രദ്ധിച്ചു കേള്‍ക്കരുത്‌. അത് പാരായണം ചെയ്യുമ്പോള്‍ നിങ്ങള്‍ ബഹളമുണ്ടാക്കുക. നിങ്ങള്‍ക്ക് അതിനെ അതിജയിക്കാന്‍ കഴിഞ്ഞേക്കാം.
\end{malayalam}}
\flushright{\begin{Arabic}
\quranayah[41][27]
\end{Arabic}}
\flushleft{\begin{malayalam}
എന്നാല്‍ ആ സത്യനിഷേധികള്‍ക്ക് നാം കഠിനമായ ശിക്ഷ ആസ്വദിപ്പിക്കുക തന്നെചെയ്യും. അവര്‍ പ്രവര്‍ത്തിച്ച് കൊണ്ടിരുന്നതില്‍ അതിനീചമായതിന്നുള്ള പ്രതിഫലം നാം അവര്‍ക്ക് നല്‍കുക തന്നെചെയ്യും.
\end{malayalam}}
\flushright{\begin{Arabic}
\quranayah[41][28]
\end{Arabic}}
\flushleft{\begin{malayalam}
അതത്രെ അല്ലാഹുവിന്‍റെ ശത്രുക്കള്‍ക്കുള്ള പ്രതിഫലമായ നരകം. അവര്‍ക്ക് അവിടെയാണ് സ്ഥിരവാസത്തിന്നുള്ള വസതി. നമ്മുടെ ദൃഷ്ടാന്തങ്ങളെ അവര്‍ നിഷേധിച്ച് കളഞ്ഞിരുന്നതിനുള്ള പ്രതിഫലമത്രെ അത്‌.
\end{malayalam}}
\flushright{\begin{Arabic}
\quranayah[41][29]
\end{Arabic}}
\flushleft{\begin{malayalam}
സത്യനിഷേധികള്‍ പറയും: ഞങ്ങളുടെ രക്ഷിതാവേ, ഞങ്ങളെ പിഴപ്പിച്ചവരായ ജിന്നുകളില്‍ നിന്നും മനുഷ്യരില്‍ നിന്നുമുള്ള രണ്ടുവിഭാഗത്തെ നീ ഞങ്ങള്‍ക്ക് കാണിച്ചുതരേണമേ. അവര്‍ അധമന്‍മാരുടെ കൂട്ടത്തിലാകത്തക്കവണ്ണം ഞങ്ങള്‍ അവരെ ഞങ്ങളുടെ പാദങ്ങള്‍ക്ക് ചുവട്ടിലിട്ട് ചവിട്ടട്ടെ.
\end{malayalam}}
\flushright{\begin{Arabic}
\quranayah[41][30]
\end{Arabic}}
\flushleft{\begin{malayalam}
ഞങ്ങളുടെ രക്ഷിതാവ് അല്ലാഹുവാണെന്ന് പറയുകയും, പിന്നീട് നേരാംവണ്ണം നിലകൊള്ളുകയും ചെയ്തിട്ടുള്ളവരാരോ അവരുടെ അടുക്കല്‍ മലക്കുകള്‍ ഇറങ്ങിവന്നുകൊണ്ട് ഇപ്രകാരം പറയുന്നതാണ്‌: നിങ്ങള്‍ ഭയപ്പെടുകയോ ദുഃഖിക്കുകയോ വേണ്ട നിങ്ങള്‍ക്ക് വാഗ്ദാനം നല്‍കപ്പെട്ടിരുന്ന സ്വര്‍ഗത്തെപ്പറ്റി നിങ്ങള്‍ സന്തോഷമടഞ്ഞ് കൊള്ളുക.
\end{malayalam}}
\flushright{\begin{Arabic}
\quranayah[41][31]
\end{Arabic}}
\flushleft{\begin{malayalam}
ഐഹികജീവിതത്തിലും പരലോകത്തിലും ഞങ്ങള്‍ നിങ്ങളുടെ മിത്രങ്ങളാകുന്നു. നിങ്ങള്‍ക്കവിടെ (പരലോകത്ത്‌) നിങ്ങളുടെ മനസ്സുകള്‍ കൊതിക്കുന്നതെല്ലാമുണ്ടായിരിക്കും. നിങ്ങള്‍ക്കവിടെ നിങ്ങള്‍ ആവശ്യപ്പെടുന്നതെല്ലാമുണ്ടായിരിക്കും.
\end{malayalam}}
\flushright{\begin{Arabic}
\quranayah[41][32]
\end{Arabic}}
\flushleft{\begin{malayalam}
ഏറെ പൊറുക്കുന്നവനും കരുണാനിധിയുമായ അല്ലാഹുവിങ്കല്‍ നിന്നുള്ള സല്‍ക്കാരമത്രെ അത്‌.
\end{malayalam}}
\flushright{\begin{Arabic}
\quranayah[41][33]
\end{Arabic}}
\flushleft{\begin{malayalam}
അല്ലാഹുവിങ്കലേക്ക് ക്ഷണിക്കുകയും സല്‍കര്‍മ്മം പ്രവര്‍ത്തിക്കുകയും തീര്‍ച്ചയായും ഞാന്‍ മുസ്ലിംകളുടെ കൂട്ടത്തിലാകുന്നു എന്ന് പറയുകയും ചെയ്തവനെക്കാള്‍ വിശിഷ്ടമായ വാക്ക് പറയുന്ന മറ്റാരുണ്ട്‌?
\end{malayalam}}
\flushright{\begin{Arabic}
\quranayah[41][34]
\end{Arabic}}
\flushleft{\begin{malayalam}
നല്ലതും ചീത്തയും സമമാവുകയില്ല. ഏറ്റവും നല്ലത് ഏതോ അത് കൊണ്ട് നീ (തിന്‍മയെ) പ്രതിരോധിക്കുക. അപ്പോള്‍ ഏതൊരുവനും നീയും തമ്മില്‍ ശത്രുതയുണ്ടോ അവനതാ (നിന്‍റെ) ഉറ്റബന്ധു എന്നോണം ആയിത്തീരുന്നു.
\end{malayalam}}
\flushright{\begin{Arabic}
\quranayah[41][35]
\end{Arabic}}
\flushleft{\begin{malayalam}
ക്ഷമ കൈക്കൊണ്ടവര്‍ക്കല്ലാതെ അതിനുള്ള അനുഗ്രഹം നല്‍കപ്പെടുകയില്ല. വമ്പിച്ച ഭാഗ്യമുള്ളവന്നല്ലാതെ അതിനുള്ള അനുഗ്രഹം നല്‍കപ്പെടുകയില്ല.
\end{malayalam}}
\flushright{\begin{Arabic}
\quranayah[41][36]
\end{Arabic}}
\flushleft{\begin{malayalam}
പിശാചില്‍ നിന്നുള്ള വല്ല ദുഷ്പ്രേരണയും നിന്നെ വ്യതിചലിപ്പിച്ചുകളയുന്ന പക്ഷം അല്ലാഹുവോട് നീ ശരണം തേടിക്കൊള്ളുക. തീര്‍ച്ചയായും അവന്‍ തന്നെയാകുന്നു എല്ലാം കേള്‍ക്കുന്നവനും അറിയുന്നവനും.
\end{malayalam}}
\flushright{\begin{Arabic}
\quranayah[41][37]
\end{Arabic}}
\flushleft{\begin{malayalam}
അവന്‍റെ ദൃഷ്ടാന്തങ്ങളില്‍ പെട്ടതത്രെ രാവും പകലും സൂര്യനും ചന്ദ്രനും. സൂര്യന്നോ, ചന്ദ്രന്നോ നിങ്ങള്‍ പ്രണാമം ചെയ്യരുത്‌. അവയെ സൃഷ്ടിച്ചവനായ അല്ലാഹുവിന്ന് നിങ്ങള്‍ പ്രണാമം ചെയ്യുക; നിങ്ങള്‍ അവനെയാണ് ആരാധിക്കുന്നതെങ്കില്‍.
\end{malayalam}}
\flushright{\begin{Arabic}
\quranayah[41][38]
\end{Arabic}}
\flushleft{\begin{malayalam}
ഇനി അവര്‍ അഹംഭാവം നടിക്കുകയാണെങ്കില്‍ നിന്‍റെ രക്ഷിതാവിന്‍റെ അടുക്കലുള്ളവര്‍ (മലക്കുകള്‍) രാവും പകലും അവനെ പ്രകീര്‍ത്തിക്കുന്നുണ്ട്‌. അവര്‍ക്ക് മടുപ്പ് തോന്നുകയില്ല.
\end{malayalam}}
\flushright{\begin{Arabic}
\quranayah[41][39]
\end{Arabic}}
\flushleft{\begin{malayalam}
നീ ഭൂമിയെ വരണ്ടുണങ്ങിയതായി കാണുന്നു. എന്നിട്ട് അതില്‍ നാം വെള്ളം വര്‍ഷിച്ചാല്‍ അതിന് ചലനമുണ്ടാവുകയും അത് വളരുകയും ചെയ്യുന്നു. ഇതും അവന്‍റെ ദൃഷ്ടാന്തങ്ങളില്‍ പെട്ടതത്രെ. അതിന് ജീവന്‍ നല്‍കിയവന്‍ തീര്‍ച്ചയായും മരിച്ചവര്‍ക്കും ജീവന്‍ നല്‍കുന്നവനാകുന്നു. തീര്‍ച്ചയായും അവന്‍ ഏതുകാര്യത്തിനും കഴിവുള്ളവനാകുന്നു.
\end{malayalam}}
\flushright{\begin{Arabic}
\quranayah[41][40]
\end{Arabic}}
\flushleft{\begin{malayalam}
നമ്മുടെ ദൃഷ്ടാന്തങ്ങളുടെ നേരെ വക്രത കാണിക്കുന്നവരാരോ അവര്‍ നമ്മുടെ ദൃഷ്ടിയില്‍ നിന്ന് മറഞ്ഞു പോകുകയില്ല; തീര്‍ച്ച. അപ്പോള്‍ നരകത്തിലെറിയപ്പെടുന്നവനാണോ ഉത്തമന്‍ അതല്ല ഉയിര്‍ത്തെഴുന്നേല്‍പിന്‍റെ നാളില്‍ നിര്‍ഭയനായിട്ട് വരുന്നവനോ? നിങ്ങള്‍ ഉദ്ദേശിച്ചത് നിങ്ങള്‍ ചെയ്തുകൊള്ളുക. തീര്‍ച്ചയായും അവന്‍ നിങ്ങള്‍ പ്രവര്‍ത്തിക്കുന്നത് കണ്ടറിയുന്നവനാകുന്നു.
\end{malayalam}}
\flushright{\begin{Arabic}
\quranayah[41][41]
\end{Arabic}}
\flushleft{\begin{malayalam}
തീര്‍ച്ചയായും ഈ ഉല്‍ബോധനം തങ്ങള്‍ക്കു വന്നുകിട്ടിയപ്പോള്‍ അതില്‍ അവിശ്വസിച്ചവര്‍ (നഷ്ടം പറ്റിയവര്‍ തന്നെ) തീര്‍ച്ചയായും അത് പ്രതാപമുള്ള ഒരു ഗ്രന്ഥം തന്നെയാകുന്നു.
\end{malayalam}}
\flushright{\begin{Arabic}
\quranayah[41][42]
\end{Arabic}}
\flushleft{\begin{malayalam}
അതിന്‍റെ മുന്നിലൂടെയോ, പിന്നിലൂടെയോ അതില്‍ അസത്യം വന്നെത്തുകയില്ല. യുക്തിമാനും സ്തുത്യര്‍ഹനുമായിട്ടുള്ളവന്‍റെ പക്കല്‍ നിന്ന് അവതരിപ്പിക്കപ്പെട്ടതത്രെ അത്‌.
\end{malayalam}}
\flushright{\begin{Arabic}
\quranayah[41][43]
\end{Arabic}}
\flushleft{\begin{malayalam}
(നബിയേ,) നിനക്ക് മുമ്പുണ്ടായിരുന്ന ദൂതന്‍മാരോട് പറയപ്പെട്ടതല്ലാത്ത ഒന്നും നിന്നോട് പറയപ്പെടുന്നില്ല. തീര്‍ച്ചയായും നിന്‍റെ രക്ഷിതാവ് പാപമോചനം നല്‍കുന്നവനും വേദനയേറിയ ശിക്ഷ നല്‍കുന്നവനുമാകുന്നു.
\end{malayalam}}
\flushright{\begin{Arabic}
\quranayah[41][44]
\end{Arabic}}
\flushleft{\begin{malayalam}
നാം ഇതിനെ ഒരു അനറബി ഖുര്‍ആന്‍ ആക്കിയിരുന്നെങ്കില്‍ അവര്‍ പറഞ്ഞേക്കും: എന്തുകൊണ്ട് ഇതിലെ വചനങ്ങള്‍ വിശദമാക്കപ്പെട്ടവയായില്ല. (ഗ്രന്ഥം) അനറബിയും (പ്രവാചകന്‍) അറബിയും ആവുകയോ? നീ പറയുക: അത് (ഖുര്‍ആന്‍) സത്യവിശ്വാസികള്‍ക്ക് മാര്‍ഗദര്‍ശനവും ശമനൌഷധവുമാകുന്നു. വിശ്വസിക്കാത്തവര്‍ക്കാകട്ടെ അവരുടെ കാതുകളില്‍ ഒരു തരം ബധിരതയുണ്ട്‌. അത് (ഖുര്‍ആന്‍) അവരുടെ മേല്‍ ഒരു അന്ധതയായിരിക്കുന്നു. ആ കൂട്ടര്‍ വിദൂരമായ ഏതോ സ്ഥലത്ത് നിന്ന് വിളിക്കപ്പെടുന്നു (എന്ന പോലെയാകുന്നു അവരുടെ പ്രതികരണം).
\end{malayalam}}
\flushright{\begin{Arabic}
\quranayah[41][45]
\end{Arabic}}
\flushleft{\begin{malayalam}
മൂസായ്ക്ക് നാം വേദഗ്രന്ഥം നല്‍കുകയുണ്ടായി. എന്നിട്ട് അതിന്‍റെ കാര്യത്തിലും അഭിപ്രായവ്യത്യാസമുണ്ടായി. ഒരു വചനം മുമ്പ് തന്നെ നിന്‍റെ രക്ഷിതാവിന്‍റെ പക്കല്‍ നിന്ന് ഉണ്ടായിട്ടില്ലായിരുന്നുവെങ്കില്‍ അവര്‍ക്കിടയില്‍ (ഇപ്പോള്‍ തന്നെ) തീര്‍പ്പുകല്‍പിക്കപ്പെടുമായിരുന്നു. തീര്‍ച്ചയായും അവര്‍ ഇതിനെ (ഖുര്‍ആനിനെ) പറ്റി അവിശ്വാസജനകമായ സംശയത്തിലാകുന്നു.
\end{malayalam}}
\flushright{\begin{Arabic}
\quranayah[41][46]
\end{Arabic}}
\flushleft{\begin{malayalam}
വല്ലവനും നല്ലത് പ്രവര്‍ത്തിച്ചാല്‍ അതിന്‍റെ ഗുണം അവന് തന്നെയാകുന്നു. വല്ലവനും തിന്‍മചെയ്താല്‍ അതിന്‍റെ ദോഷവും അവന് തന്നെ. നിന്‍റെ രക്ഷിതാവ് (തന്‍റെ) അടിമകളോട് അനീതി കാണിക്കുന്നവനേ അല്ല.
\end{malayalam}}
\flushright{\begin{Arabic}
\quranayah[41][47]
\end{Arabic}}
\flushleft{\begin{malayalam}
ആ അന്ത്യസമയത്തെപ്പറ്റിയുള്ള അറിവ് അവങ്കലേക്കാണ് മടക്കപ്പെടുന്നത്‌. പഴങ്ങളൊന്നും അവയുടെ പോളകളില്‍ നിന്ന് പുറത്ത് വരുന്നില്ല; ഒരു സ്ത്രീയും ഗര്‍ഭം ധരിക്കുകയോ, പ്രസവിക്കുകയോ ചെയ്യുന്നുമില്ല; അവന്‍റെ അറിവോട് കൂടിയല്ലാതെ. എന്‍റെ പങ്കാളികളെവിടെ എന്ന് അവന്‍ അവരോട് വിളിച്ചുചോദിക്കുന്ന ദിവസം അവര്‍ പറയും: ഞങ്ങളിതാ നിന്നെ അറിയിക്കുന്നു. ഞങ്ങളില്‍ (അതിന്ന്‌) സാക്ഷികളായി ആരുമില്ല
\end{malayalam}}
\flushright{\begin{Arabic}
\quranayah[41][48]
\end{Arabic}}
\flushleft{\begin{malayalam}
മുമ്പ് അവര്‍ വിളിച്ച് പ്രാര്‍ത്ഥിച്ചിരുന്നതെല്ലാം അവരെ വിട്ട് മറഞ്ഞു പോകുകയും തങ്ങള്‍ക്ക് യാതൊരു രക്ഷാസങ്കേതവുമില്ല എന്ന് അവര്‍ക്ക് ബോധ്യം വരികയും ചെയ്യും.
\end{malayalam}}
\flushright{\begin{Arabic}
\quranayah[41][49]
\end{Arabic}}
\flushleft{\begin{malayalam}
നന്‍മയ്ക്ക് വേണ്ടി പ്രാര്‍ത്ഥിക്കുന്നതില്‍ മനുഷ്യന് മടുപ്പ് തോന്നുന്നില്ല. തിന്‍മ അവനെ ബാധിച്ചാലോ അവന്‍ മനം മടുത്തവനും നിരാശനുമായിത്തീരുന്നു.
\end{malayalam}}
\flushright{\begin{Arabic}
\quranayah[41][50]
\end{Arabic}}
\flushleft{\begin{malayalam}
അവന്ന് കഷ്ടത ബാധിച്ചതിനു ശേഷം നമ്മുടെ പക്കല്‍ നിന്നുള്ള കാരുണ്യം നാം അവന്ന് അനുഭവിപ്പിച്ചാല്‍ തീര്‍ച്ചയായും അവന്‍ പറയും: ഇത് എനിക്ക് അവകാശപ്പെട്ടതാകുന്നു. അന്ത്യസമയം നിലവില്‍ വരുമെന്ന് ഞാന്‍ വിചാരിക്കുന്നില്ല. ഇനി എന്‍റെ രക്ഷിതാവിങ്കലേക്ക് ഞാന്‍ തിരിച്ചയക്കപ്പെടുകയാണെങ്കിലോ എനിക്ക് അവന്‍റെ അടുക്കല്‍ തീര്‍ച്ചയായും ഏറ്റവും മെച്ചപ്പെട്ട നില തന്നെയാണുണ്ടായിരിക്കുക. എന്നാല്‍ സത്യനിഷേധികള്‍ക്ക് അവര്‍ പ്രവര്‍ത്തിച്ചതിനെപ്പറ്റി നാം വിവരം നല്‍കുകയും കഠിനമായ ശിക്ഷയില്‍ നിന്ന് നാം അവര്‍ക്ക് അനുഭവിപ്പിക്കുകയും ചെയ്യും.
\end{malayalam}}
\flushright{\begin{Arabic}
\quranayah[41][51]
\end{Arabic}}
\flushleft{\begin{malayalam}
നാം മനുഷ്യന് അനുഗ്രഹം ചെയ്താല്‍ അവനതാ പിന്തിരിഞ്ഞ് കളയുകയും, അവന്‍റെ പാട്ടിന് മാറിക്കളയുകയും ചെയ്യുന്നു. അവന്ന് തിന്‍മ ബാധിച്ചാലോ അവനതാ നീണ്ട പ്രാര്‍ത്ഥനക്കാരനായിത്തീരുന്നു.
\end{malayalam}}
\flushright{\begin{Arabic}
\quranayah[41][52]
\end{Arabic}}
\flushleft{\begin{malayalam}
നീ പറയുക: നിങ്ങള്‍ ആലോചിച്ചു നോക്കിയിട്ടുണ്ടോ? ഇത് (ഖുര്‍ആന്‍) അല്ലാഹുവിങ്കല്‍ നിന്നുള്ളതായിരിക്കുകയും എന്നിട്ട് നിങ്ങളതില്‍ അവിശ്വസിച്ചിരിക്കുകയുമാണെങ്കില്‍ കടുത്ത മാത്സര്യത്തില്‍ കഴിയുന്നവനെക്കാളും കൂടുതല്‍ പിഴച്ച് പോയവന്‍ ആരുണ്ട്‌.?
\end{malayalam}}
\flushright{\begin{Arabic}
\quranayah[41][53]
\end{Arabic}}
\flushleft{\begin{malayalam}
ഇത് (ഖുര്‍ആന്‍) സത്യമാണെന്ന് അവര്‍ക്ക് വ്യക്തമാകത്തക്കവണ്ണം വിവിധ ദിക്കുകളിലും അവരില്‍ തന്നെയും നമ്മുടെ ദൃഷ്ടാന്തങ്ങള്‍ വഴിയെ നാം അവര്‍ക്ക് കാണിച്ചുകൊടുക്കുന്നതാണ്‌. നിന്‍റെ രക്ഷിതാവ് ഏത് കാര്യത്തിനും സാക്ഷിയാണ് എന്നതു തന്നെ മതിയായതല്ലേ?
\end{malayalam}}
\flushright{\begin{Arabic}
\quranayah[41][54]
\end{Arabic}}
\flushleft{\begin{malayalam}
ഓര്‍ക്കുക, തീര്‍ച്ചയായും അവര്‍ തങ്ങളുടെ രക്ഷിതാവിനെ കണ്ടുമുട്ടുന്ന കാര്യത്തെപ്പറ്റി സംശയത്തിലാകുന്നു. ഓര്‍ക്കുക, തീര്‍ച്ചയായും അവന്‍ ഏതൊരു വസ്തുവിനെയും വലയം ചെയ്തിട്ടുള്ളവനാകുന്നു.
\end{malayalam}}
\chapter{\textmalayalam{ശൂറാ ( കൂടിയാലോചന )}}
\begin{Arabic}
\Huge{\centerline{\basmalah}}\end{Arabic}
\flushright{\begin{Arabic}
\quranayah[42][1]
\end{Arabic}}
\flushleft{\begin{malayalam}
ഹാമീം.
\end{malayalam}}
\flushright{\begin{Arabic}
\quranayah[42][2]
\end{Arabic}}
\flushleft{\begin{malayalam}
ഐന്‍ സീന്‍ ഖാഫ്‌.
\end{malayalam}}
\flushright{\begin{Arabic}
\quranayah[42][3]
\end{Arabic}}
\flushleft{\begin{malayalam}
അപ്രകാരം നിനക്കും നിന്‍റെ മുമ്പുള്ളവര്‍ക്കും പ്രതാപിയും യുക്തിമാനുമായ അല്ലാഹു ബോധനം നല്‍കുന്നു.
\end{malayalam}}
\flushright{\begin{Arabic}
\quranayah[42][4]
\end{Arabic}}
\flushleft{\begin{malayalam}
അവന്നാകുന്നു ആകാശങ്ങളിലുള്ളതും ഭൂമിയിലുള്ളതും. അവനാകുന്നു ഉന്നതനും മഹാനുമായിട്ടുള്ളവന്‍.
\end{malayalam}}
\flushright{\begin{Arabic}
\quranayah[42][5]
\end{Arabic}}
\flushleft{\begin{malayalam}
ആകാശങ്ങള്‍ അവയുടെ ഉപരിഭാഗത്ത് നിന്ന് പൊട്ടിപ്പിളരുമാറാകുന്നു. മലക്കുകള്‍ തങ്ങളുടെ രക്ഷിതാവിനെ സ്തുതിക്കുന്നതോടൊപ്പം പ്രകീര്‍ത്തിച്ചു കൊണ്ടിരിക്കുന്നു.ഭൂമിയിലുള്ളവര്‍ക്ക് വേണ്ടി അവര്‍ പാപമോചനം തേടുകയും ചെയ്യുന്നു. അറിയുക! തീര്‍ച്ചയായും അല്ലാഹു തന്നെയാകുന്നു ഏറെ പൊറുക്കുന്നവനും കരുണാനിധിയും.
\end{malayalam}}
\flushright{\begin{Arabic}
\quranayah[42][6]
\end{Arabic}}
\flushleft{\begin{malayalam}
അവനു പുറമെ രക്ഷാധികാരികളെ സ്വീകരിച്ചവരാരോ, അവരെ അല്ലാഹു സൂക്ഷ്മനിരീക്ഷണം ചെയ്തുകൊണ്ടിരിക്കുകയാകുന്നു. നീ അവരുടെ കാര്യത്തില്‍ ചുമതല ഏല്‍പിക്കപ്പെട്ടവനേ അല്ല.
\end{malayalam}}
\flushright{\begin{Arabic}
\quranayah[42][7]
\end{Arabic}}
\flushleft{\begin{malayalam}
അപ്രകാരം നിനക്ക് നാം അറബിഭാഷയിലുള്ള ഖുര്‍ആന്‍ ബോധനം നല്‍കിയിരിക്കുന്നു. ഉമ്മുല്‍ഖുറാ (മക്ക) യിലുള്ളവര്‍ക്കും അതിനു ചുറ്റുമുള്ളവര്‍ക്കും നീ താക്കീത് നല്‍കുവാന്‍ വേണ്ടിയും, സംശയരഹിതമായ സമ്മേളന ദിവസത്തെപ്പറ്റി നീ താക്കീത് നല്‍കുവാന്‍ വേണ്ടിയും. അന്ന് ഒരു വിഭാഗക്കാര്‍ സ്വര്‍ഗത്തിലായിരിക്കും. മറ്റൊരു വിഭാഗക്കാര്‍ കത്തിജ്വലിക്കുന്ന നരകത്തിലും.
\end{malayalam}}
\flushright{\begin{Arabic}
\quranayah[42][8]
\end{Arabic}}
\flushleft{\begin{malayalam}
അല്ലാഹു ഉദ്ദേശിച്ചിരുന്നെങ്കില്‍ അവരെ (മനുഷ്യരെ) യെല്ലാം അവന്‍ ഒരേ സമുദായമാക്കുമായിരുന്നു. പക്ഷെ, താന്‍ ഉദ്ദേശിക്കുന്നവരെ തന്‍റെ കാരുണ്യത്തില്‍ അവന്‍ പ്രവേശിപ്പിക്കുന്നു. അക്രമികളാരോ അവര്‍ക്ക് യാതൊരു രക്ഷാധികാരിയും സഹായിയുമില്ല.
\end{malayalam}}
\flushright{\begin{Arabic}
\quranayah[42][9]
\end{Arabic}}
\flushleft{\begin{malayalam}
അതല്ല, അവര്‍ അവന്നുപുറമെ രക്ഷാധികാരികളെ സ്വീകരിച്ചിരിക്കുകയാണോ? എന്നാല്‍ അല്ലാഹു തന്നെയാകുന്നു രക്ഷാധികാരി. അവന്‍ മരിച്ചവരെ ജീവിപ്പിക്കുന്നു. അവന്‍ ഏത് കാര്യത്തിനും കഴിവുള്ളവനത്രെ.
\end{malayalam}}
\flushright{\begin{Arabic}
\quranayah[42][10]
\end{Arabic}}
\flushleft{\begin{malayalam}
നിങ്ങള്‍ അഭിപ്രായവ്യത്യാസക്കാരായിട്ടുള്ളത് ഏത് കാര്യത്തിലാവട്ടെ അതില്‍ തീര്‍പ്പുകല്‍പിക്കാനുള്ള അവകാശം അല്ലാഹുവിന്നാകുന്നു. അവനാണ് എന്‍റെ രക്ഷിതാവായ അല്ലാഹു. അവന്‍റെ മേല്‍ ഞാന്‍ ഭരമേല്‍പിച്ചിരിക്കുന്നു. അവങ്കലേക്ക് ഞാന്‍ താഴ്മയോടെ മടങ്ങുകയും ചെയ്യുന്നു.
\end{malayalam}}
\flushright{\begin{Arabic}
\quranayah[42][11]
\end{Arabic}}
\flushleft{\begin{malayalam}
ആകാശങ്ങളുടെയും ഭൂമിയുടെയും സ്രഷ്ടാവാകുന്നു (അവന്‍.) നിങ്ങള്‍ക്ക് വേണ്ടി നിങ്ങളുടെ വര്‍ഗത്തില്‍ നിന്നു തന്നെ അവന്‍ ഇണകളെ (ഉണ്ടാക്കിത്തന്നിരിക്കുന്നു.) അതിലൂടെ നിങ്ങളെ അവന്‍ സൃഷ്ടിച്ച് വര്‍ധിപ്പിക്കുന്നു. അവന് തുല്യമായി യാതൊന്നുമില്ല. അവന്‍ എല്ലാം കാണുന്നവനും എല്ലാം കേള്‍ക്കുന്നവനുമാകുന്നു.
\end{malayalam}}
\flushright{\begin{Arabic}
\quranayah[42][12]
\end{Arabic}}
\flushleft{\begin{malayalam}
ആകാശങ്ങളുടെയും ഭൂമിയുടെയും താക്കോലുകള്‍ അവന്‍റെ അധീനത്തിലാകുന്നു. അവന്‍ ഉദ്ദേശിക്കുന്നവര്‍ക്ക് ഉപജീവനം അവന്‍ വിശാലമാക്കുന്നു. (മറ്റുള്ളവര്‍ക്ക്‌) അവന്‍ അത് ഇടുങ്ങിയതാക്കുകയും ചെയ്യുന്നു. തീര്‍ച്ചയായും അവന്‍ ഏത് കാര്യത്തെപ്പറ്റിയും അറിവുള്ളവനാകുന്നു.
\end{malayalam}}
\flushright{\begin{Arabic}
\quranayah[42][13]
\end{Arabic}}
\flushleft{\begin{malayalam}
നൂഹിനോട് കല്‍പിച്ചതും നിനക്ക് നാം ബോധനം നല്‍കിയതും ഇബ്രാഹീം, മൂസാ, ഈസാ എന്നിവരോട് നാം കല്‍പിച്ചതുമായ കാര്യം - നിങ്ങള്‍ മതത്തെ നേരാംവണ്ണം നിലനിര്‍ത്തുക, അതില്‍ നിങ്ങള്‍ ഭിന്നിക്കാതിരിക്കുക. എന്നകാര്യം - അവന്‍ നിങ്ങള്‍ക്ക് മതനിയമമായി നിശ്ചയിച്ചിരിക്കുന്നു. ആ ബഹുദൈവവിശ്വാസികളെ നിങ്ങള്‍ ഏതൊരു കാര്യത്തിലേക്ക് ക്ഷണിക്കുന്നുവോ അത് അവര്‍ക്ക് വലിയ ഭാരമായി തോന്നിയിരിക്കുന്നു. താന്‍ ഉദ്ദേശിക്കുന്നവരെ അല്ലാഹു തന്‍റെ അടുക്കലേക്ക് തെരഞ്ഞെടുക്കുന്നു. താഴ്മയോടെ മടങ്ങുന്നവരെ അവങ്കലേക്കുള്ള മാര്‍ഗത്തില്‍ നയിക്കുകയും ചെയ്യുന്നു.
\end{malayalam}}
\flushright{\begin{Arabic}
\quranayah[42][14]
\end{Arabic}}
\flushleft{\begin{malayalam}
പൂര്‍വ്വവേദക്കാര്‍ ഭിന്നിച്ചത് അവര്‍ക്ക് അറിവ് വന്നുകിട്ടിയതിന് ശേഷം തന്നെയാണ്‌. അവര്‍ തമ്മിലുള്ള വിരോധം നിമിത്തമാണത്‌. നിര്‍ണിതമായ ഒരു അവധിവരേക്ക് ബാധകമായ ഒരു വചനം നിങ്ങളുടെ രക്ഷിതാവിങ്കല്‍ നിന്ന് മുമ്പ് തന്നെ ഉണ്ടായിട്ടില്ലായിരുന്നുവെങ്കില്‍ അവര്‍ക്കിടയില്‍ (ഉടനെ) തീര്‍പ്പുകല്‍പിക്കപ്പെടുമായിരുന്നു. അവര്‍ക്ക് ശേഷം വേദഗ്രന്ഥത്തിന്‍റെ അനന്തരാവകാശം നല്‍കപ്പെട്ടവര്‍ തീര്‍ച്ചയായും അതിനെപ്പറ്റി അവിശ്വാസജനകമായ സംശയത്തിലാകുന്നു.
\end{malayalam}}
\flushright{\begin{Arabic}
\quranayah[42][15]
\end{Arabic}}
\flushleft{\begin{malayalam}
അതിനാല്‍ നീ പ്രബോധനം ചെയ്തുകൊള്ളുക. നീ കല്‍പിക്കപ്പെട്ടത് പോലെ നേരെ നിലകൊള്ളുകയും ചെയ്യുക. അവരുടെ തന്നിഷ്ടങ്ങളെ നീ പിന്തുടര്‍ന്ന് പോകരുത്‌. നീ പറയുക: അല്ലാഹു അവതരിപ്പിച്ച ഏത് ഗ്രന്ഥത്തിലും ഞാന്‍ വിശ്വസിച്ചിരിക്കുന്നു. നിങ്ങള്‍ക്കിടയില്‍ നീതിപുലര്‍ത്തുവാന്‍ ഞാന്‍ കല്‍പിക്കപ്പെടുകയും ചെയ്തിരിക്കുന്നു. അല്ലാഹുവാകുന്നു ഞങ്ങളുടെ രക്ഷിതാവും നിങ്ങളുടെ രക്ഷിതാവും. ഞങ്ങള്‍ക്കുള്ളത് ഞങ്ങളുടെ കര്‍മ്മങ്ങളും നിങ്ങള്‍ക്കുള്ളത് നിങ്ങളുടെ കര്‍മ്മങ്ങളും. ഞങ്ങള്‍ക്കും നിങ്ങള്‍ക്കുമിടയില്‍ യാതൊരു തര്‍ക്കപ്രശ്നവുമില്ല. അല്ലാഹു നമ്മെ തമ്മില്‍ ഒരുമിച്ചുകൂട്ടും. അവങ്കലേക്കാകുന്നു ചെന്നെത്താനുള്ളത്‌.
\end{malayalam}}
\flushright{\begin{Arabic}
\quranayah[42][16]
\end{Arabic}}
\flushleft{\begin{malayalam}
അല്ലാഹുവിന്‍റെ ആഹ്വാനത്തിന് സ്വീകാര്യത ലഭിച്ചതിന് ശേഷം അവന്‍റെ കാര്യത്തില്‍ തര്‍ക്കിക്കുന്നവരാരോ, അവരുടെ തര്‍ക്കം അവരുടെ രക്ഷിതാവിങ്കല്‍ നിഷ്ഫലമാകുന്നു. അവരുടെ മേല്‍ കോപമുണ്ടായിരിക്കും.അവര്‍ക്കാണ് കഠിനമായ ശിക്ഷ.
\end{malayalam}}
\flushright{\begin{Arabic}
\quranayah[42][17]
\end{Arabic}}
\flushleft{\begin{malayalam}
അല്ലാഹുവാകുന്നു സത്യപ്രകാരം വേദഗ്രന്ഥവും (തെറ്റും ശരിയും തൂക്കിനോക്കാനുള്ള) തുലാസും ഇറക്കിത്തന്നവന്‍. നിനക്ക് എന്തറിയാം. ആ അന്ത്യസമയം അടുത്ത് തന്നെ ആയിരിക്കാം.
\end{malayalam}}
\flushright{\begin{Arabic}
\quranayah[42][18]
\end{Arabic}}
\flushleft{\begin{malayalam}
അതില്‍ (അന്ത്യസമയത്തില്‍) വിശ്വസിക്കാത്തവര്‍ അതിന്‍റെ കാര്യത്തില്‍ ധൃതികൂട്ടിക്കൊണ്ടിരിക്കുന്നു.വിശ്വസിച്ചവരാകട്ടെ അതിനെപ്പറ്റി ഭയവിഹ്വലരാകുന്നു. അവര്‍ക്കറിയാം അത് സത്യമാണെന്ന്‌. ശ്രദ്ധിക്കുക: തീര്‍ച്ചയായും അന്ത്യസമയത്തിന്‍റെ കാര്യത്തില്‍ തര്‍ക്കം നടത്തുന്നവര്‍ വിദൂരമായ പിഴവില്‍ തന്നെയാകുന്നു.
\end{malayalam}}
\flushright{\begin{Arabic}
\quranayah[42][19]
\end{Arabic}}
\flushleft{\begin{malayalam}
അല്ലാഹു തന്‍റെ ദാസന്‍മാരോട് കനിവുള്ളവനാകുന്നു. അവന്‍ ഉദ്ദേശിക്കുന്നവര്‍ക്ക് അവന്‍ ഉപജീവനം നല്‍കുന്നു. അവനാകുന്നു ശക്തനും പ്രതാപശാലിയുമായിട്ടുള്ളവന്‍.
\end{malayalam}}
\flushright{\begin{Arabic}
\quranayah[42][20]
\end{Arabic}}
\flushleft{\begin{malayalam}
വല്ലവനും പരലോകത്തെ കൃഷിയാണ് ഉദ്ദേശിക്കുന്നതെങ്കില്‍ അവന്‍റെ കൃഷിയില്‍ നാം അവന് വര്‍ദ്ധന നല്‍കുന്നതാണ്‌. വല്ലവനും ഇഹലോകത്തെ കൃഷിയാണ് ഉദ്ദേശിക്കുന്നതെങ്കില്‍ നാം അവന് അതില്‍ നിന്ന് നല്‍കുന്നതാണ്‌.അവന് പരലോകത്ത് യാതൊരു വിഹിതവും ഉണ്ടായിരിക്കുന്നതല്ല.
\end{malayalam}}
\flushright{\begin{Arabic}
\quranayah[42][21]
\end{Arabic}}
\flushleft{\begin{malayalam}
അതല്ല, അല്ലാഹു അനുവദിച്ചിട്ടില്ലാത്ത കാര്യം മതമായി അവര്‍ക്ക് നിശ്ചയിച്ചു കൊടുത്ത വല്ല പങ്കാളികളും അവര്‍ക്കുണ്ടോ? നിര്‍ണായക വിധിയെ പറ്റിയുള്ള കല്‍പന നിലവിലില്ലായിരുന്നെങ്കില്‍ അവര്‍ക്കിടയില്‍ ഉടനെ വിധികല്‍പിക്കപ്പെടുമായിരുന്നു. അക്രമികളാരോ അവര്‍ക്ക് തീര്‍ച്ചയായും വേദനയേറിയ ശിക്ഷയുണ്ട്‌.
\end{malayalam}}
\flushright{\begin{Arabic}
\quranayah[42][22]
\end{Arabic}}
\flushleft{\begin{malayalam}
(പരലോകത്ത് വെച്ച്‌) ആ അക്രമകാരികളെ തങ്ങള്‍ സമ്പാദിച്ചു വെച്ചതിനെപ്പറ്റി ഭയചകിതരായ നിലയില്‍ നിനക്ക് കാണാം. അത് (സമ്പാദിച്ചു വെച്ചതിനുള്ള ശിക്ഷ) അവരില്‍ വന്നുഭവിക്കുക തന്നെചെയ്യും. വിശ്വസിക്കുകയും സല്‍കര്‍മ്മം പ്രവര്‍ത്തിക്കുകയും ചെയ്തവര്‍ സ്വര്‍ഗത്തോപ്പുകളിലായിരിക്കും. അവര്‍ ഉദ്ദേശിക്കുന്നതെന്തോ അത് അവരുടെ രക്ഷിതാവിങ്കല്‍ അവര്‍ക്കുണ്ടായിരിക്കും. അതത്രെ മഹത്തായ അനുഗ്രഹം.
\end{malayalam}}
\flushright{\begin{Arabic}
\quranayah[42][23]
\end{Arabic}}
\flushleft{\begin{malayalam}
വിശ്വസിക്കുകയും സല്‍കര്‍മ്മങ്ങള്‍ പ്രവര്‍ത്തിക്കുകയും ചെയ്ത തന്‍റെ ദാസന്‍മാര്‍ക്ക് അല്ലാഹു സന്തോഷവാര്‍ത്ത അറിയിക്കുന്നതത്രെ അത്‌. നീ പറയുക: അതിന്‍റെ പേരില്‍ നിങ്ങളോട് ഞാന്‍ യാതൊരു പ്രതിഫലവും ആവശ്യപ്പെടുന്നില്ല. അടുത്ത ബന്ധത്തിന്‍റെ പേരിലുള്ള സ്നേഹമല്ലാതെ. വല്ലവനും ഒരു നന്‍മ പ്രവര്‍ത്തിക്കുന്ന പക്ഷം അതിലൂടെ അവന്ന് നാം ഗുണം വര്‍ദ്ധിപ്പിച്ചു കൊടുക്കുന്നതാണ്‌. തീര്‍ച്ചയായും അല്ലാഹു ഏറെ പൊറുക്കുന്നവനും ഏറ്റവും നന്ദിയുള്ളവനുമാകുന്നു.
\end{malayalam}}
\flushright{\begin{Arabic}
\quranayah[42][24]
\end{Arabic}}
\flushleft{\begin{malayalam}
അതല്ല, അദ്ദേഹം (പ്രവാചകന്‍) അല്ലാഹുവിന്‍റെ പേരില്‍ കള്ളം കെട്ടിച്ചമച്ചു എന്നാണോ അവര്‍ പറയുന്നത്‌? എന്നാല്‍ അല്ലാഹു ഉദ്ദേശിച്ചിരുന്നെങ്കില്‍ നിന്‍റെ ഹൃദയത്തിനുമേല്‍ അവന്‍ മുദ്രവെക്കുമായിരുന്നു. അല്ലാഹു അസത്യത്തെ മായ്ച്ചുകളയുകയും തന്‍റെ വചനങ്ങള്‍ കൊണ്ട് സത്യത്തെ സ്ഥിരീകരിക്കുകയും ചെയ്യുന്നു. തീര്‍ച്ചയായും അവന്‍ ഹൃദങ്ങളിലുള്ളതിനെപ്പറ്റി അറിവുള്ളവനാകുന്നു.
\end{malayalam}}
\flushright{\begin{Arabic}
\quranayah[42][25]
\end{Arabic}}
\flushleft{\begin{malayalam}
അവനാകുന്നു തന്‍റെ ദാസന്‍മാരില്‍ നിന്ന് പശ്ചാത്താപം സ്വീകരിക്കുന്നവന്‍. അവന്‍ ദുഷ്കൃത്യങ്ങള്‍ക്ക് മാപ്പുനല്‍കുകയും ചെയ്യുന്നു. നിങ്ങള്‍ പ്രവര്‍ത്തിക്കുന്നതെന്തോ അത് അവന്‍ അറിയുകയും ചെയ്യുന്നു.
\end{malayalam}}
\flushright{\begin{Arabic}
\quranayah[42][26]
\end{Arabic}}
\flushleft{\begin{malayalam}
വിശ്വസിക്കുകയും സല്‍കര്‍മ്മങ്ങള്‍ പ്രവര്‍ത്തിക്കുകയും ചെയ്തവര്‍ക്ക് അവന്‍ (പ്രാര്‍ത്ഥനയ്ക്ക്‌) ഉത്തരം നല്‍കുകയും, തന്‍റെ അനുഗ്രഹത്തില്‍ നിന്ന് അവര്‍ക്ക് കൂടുതല്‍ നല്‍കുകയും ചെയ്യും. സത്യനിഷേധികളാവട്ടെ കഠിനമായ ശിക്ഷയാണവര്‍ക്കുള്ളത്‌.
\end{malayalam}}
\flushright{\begin{Arabic}
\quranayah[42][27]
\end{Arabic}}
\flushleft{\begin{malayalam}
അല്ലാഹു തന്‍റെ ദാസന്‍മാര്‍ക്ക് ഉപജീവനം വിശാലമാക്കികൊടുത്തിരുന്നെങ്കില്‍ ഭൂമിയില്‍ അവര്‍ അതിക്രമം പ്രവര്‍ത്തിക്കുമായിരുന്നു. പക്ഷെ, അവന്‍ ഒരു കണക്കനുസരിച്ച് താന്‍ ഉദ്ദേശിക്കുന്നത് ഇറക്കി കൊടുക്കുന്നുഠീര്‍ച്ചയായും അവന്‍ തന്‍റെ ദാസന്‍മാരെപ്പറ്റി സൂക്ഷ്മജ്ഞാനമുള്ളവനും കണ്ടറിയുന്നവനുമാകുന്നു.
\end{malayalam}}
\flushright{\begin{Arabic}
\quranayah[42][28]
\end{Arabic}}
\flushleft{\begin{malayalam}
അവന്‍ തന്നെയാകുന്നു മനുഷ്യര്‍ നിരാശപ്പെട്ടുകഴിഞ്ഞതിനു ശേഷം മഴ വര്‍ഷിപ്പിക്കുകയും, തന്‍റെ കാരുണ്യം വ്യാപിപ്പിക്കുകയും ചെയ്യുന്നവന്‍. അവന്‍ തന്നെയാകുന്നു സ്തുത്യര്‍ഹനായ രക്ഷാധികാരി.
\end{malayalam}}
\flushright{\begin{Arabic}
\quranayah[42][29]
\end{Arabic}}
\flushleft{\begin{malayalam}
ആകാശങ്ങളും ഭൂമിയും സൃഷ്ടിച്ചതും അവ രണ്ടിലും ജീവജാലങ്ങളെ വ്യാപിപ്പിച്ചതും അവന്‍റെ ദൃഷ്ടാന്തങ്ങളില്‍ പെട്ടതത്രെ . അവന്‍ ഉദ്ദേശിക്കുമ്പോള്‍ അവരെ ഒരുമിച്ചുകൂട്ടുവാന്‍ കഴിവുള്ളവനാണ് അവന്‍.
\end{malayalam}}
\flushright{\begin{Arabic}
\quranayah[42][30]
\end{Arabic}}
\flushleft{\begin{malayalam}
നിങ്ങള്‍ക്ക് ഏതൊരു ആപത്ത് ബാധിച്ചിട്ടുണ്ടെങ്കിലും അത് നിങ്ങളുടെ കൈകള്‍ പ്രവര്‍ത്തിച്ചതിന്‍റെ ഫലമായിട്ടുതന്നെയാണ്‌. മിക്കതും അവന്‍ മാപ്പാക്കുകയും ചെയ്യുന്നു.
\end{malayalam}}
\flushright{\begin{Arabic}
\quranayah[42][31]
\end{Arabic}}
\flushleft{\begin{malayalam}
നിങ്ങള്‍ക്ക് ഭൂമിയില്‍ വെച്ച് (അല്ലാഹുവിനെ) തോല്‍പിച്ച് കളയാനാവില്ല.അല്ലാഹുവിന് പുറമെ നിങ്ങള്‍ക്ക് യാതൊരു രക്ഷാധികാരിയും സഹായിയുമില്ലതാനും.
\end{malayalam}}
\flushright{\begin{Arabic}
\quranayah[42][32]
\end{Arabic}}
\flushleft{\begin{malayalam}
കടലിലൂടെ മലകളെന്നോണം സഞ്ചരിക്കുന്ന കപ്പലുകളും അവന്‍റെ ദൃഷ്ടാന്തങ്ങളില്‍ പെട്ടതത്രെ.
\end{malayalam}}
\flushright{\begin{Arabic}
\quranayah[42][33]
\end{Arabic}}
\flushleft{\begin{malayalam}
അവന്‍ ഉദ്ദേശിക്കുന്ന പക്ഷം അവന്‍ കാറ്റിനെ അടക്കി നിര്‍ത്തും. അപ്പോള്‍ അവ കടല്‍ പരപ്പില്‍ നിശ്ചലമായി നിന്നുപോകും. തീര്‍ച്ചയായും അതില്‍ ക്ഷമാശീലരും നന്ദിയുള്ളവരുമായ ഏവര്‍ക്കും ദൃഷ്ടാന്തങ്ങളുണ്ട്‌.
\end{malayalam}}
\flushright{\begin{Arabic}
\quranayah[42][34]
\end{Arabic}}
\flushleft{\begin{malayalam}
അല്ലെങ്കില്‍ അവര്‍ പ്രവര്‍ത്തിച്ചതിന്‍റെ ഫലമായി അവയെ (കപ്പലുകളെ) അവന്‍ തകര്‍ത്തുകളയും. മിക്കതും അവന്‍ മാപ്പാക്കുകയും ചെയ്യും.
\end{malayalam}}
\flushright{\begin{Arabic}
\quranayah[42][35]
\end{Arabic}}
\flushleft{\begin{malayalam}
നമ്മുടെ ദൃഷ്ടാന്തങ്ങളുടെ കാര്യത്തില്‍ തര്‍ക്കം നടത്തുന്നവര്‍ തങ്ങള്‍ക്ക് രക്ഷപ്രാപിക്കുവാന്‍ ഒരു സ്ഥാനവുമില്ലെന്ന് മനസ്സിലാക്കേണ്ടതിനുമാണത്‌.
\end{malayalam}}
\flushright{\begin{Arabic}
\quranayah[42][36]
\end{Arabic}}
\flushleft{\begin{malayalam}
നിങ്ങള്‍ക്ക് വല്ലതും നല്‍കപ്പെട്ടിട്ടുണ്ടെങ്കില്‍ അത് ഐഹികജീവിതത്തിലെ (താല്‍ക്കാലിക) വിഭവം മാത്രമാകുന്നു. അല്ലാഹുവിന്‍റെ പക്കലുള്ളത് കൂടുതല്‍ ഉത്തമവും കൂടുതല്‍ നീണ്ടുനില്‍ക്കുന്നതുമാകുന്നു. വിശ്വസിക്കുകയും തങ്ങളുടെ രക്ഷിതാവിന്‍റെ മേല്‍ ഭരമേല്‍പിക്കുകയും ചെയ്തവര്‍ക്കുള്ളതത്രെ അത്‌.
\end{malayalam}}
\flushright{\begin{Arabic}
\quranayah[42][37]
\end{Arabic}}
\flushleft{\begin{malayalam}
മഹാപാപങ്ങളും നീചവൃത്തികളും വര്‍ജ്ജിക്കുന്നവരും, കോപം വന്നാലും പൊറുക്കുന്നവരുമായിട്ടുള്ളവര്‍ക്ക്‌.
\end{malayalam}}
\flushright{\begin{Arabic}
\quranayah[42][38]
\end{Arabic}}
\flushleft{\begin{malayalam}
തങ്ങളുടെ രക്ഷിതാവിന്‍റെ ആഹ്വാനം സ്വീകരിക്കുകയും നമസ്കാരം മുറപോലെ നിര്‍വഹിക്കുകയും, തങ്ങളുടെ കാര്യം തീരുമാനിക്കുന്നത് അന്യോന്യമുള്ള കൂടിയാലോചനയിലൂടെ ആയിരിക്കുകയും, നാം നല്‍കിയിട്ടുള്ളതില്‍ നിന്ന് ചെലവഴിക്കുകയും ചെയ്തവരാരോ, അവര്‍ക്കും.
\end{malayalam}}
\flushright{\begin{Arabic}
\quranayah[42][39]
\end{Arabic}}
\flushleft{\begin{malayalam}
തങ്ങള്‍ക്ക് വല്ല മര്‍ദ്ദനവും നേരിട്ടാല്‍ രക്ഷാനടപടി സ്വീകരിക്കുന്നവര്‍ക്കും.
\end{malayalam}}
\flushright{\begin{Arabic}
\quranayah[42][40]
\end{Arabic}}
\flushleft{\begin{malayalam}
ഒരു തിന്‍മയ്ക്കുള്ള പ്രതിഫലം അതുപോലുള്ള ഒരു തിന്‍മതന്നെയാകുന്നു. എന്നാല്‍ ആരെങ്കിലും മാപ്പുനല്‍കുകയും രഞ്ജിപ്പുണ്ടാക്കുകയും ആണെങ്കില്‍ അവന്നുള്ള പ്രതിഫലം അല്ലാഹുവിന്‍റെ ബാധ്യതയിലാകുന്നു. തീര്‍ച്ചയായും അവന്‍ അക്രമം പ്രവര്‍ത്തിക്കുന്നവരെ ഇഷ്ടപ്പെടുകയില്ല.
\end{malayalam}}
\flushright{\begin{Arabic}
\quranayah[42][41]
\end{Arabic}}
\flushleft{\begin{malayalam}
താന്‍ മര്‍ദ്ദിക്കപ്പെട്ടതിന് ശേഷം വല്ലവനും രക്ഷാനടപടി സ്വീകരിക്കുന്ന പക്ഷം അത്തരക്കാര്‍ക്കെതിരില്‍ (കുറ്റം ചുമത്താന്‍) യാതൊരു മാര്‍ഗവുമില്ല.
\end{malayalam}}
\flushright{\begin{Arabic}
\quranayah[42][42]
\end{Arabic}}
\flushleft{\begin{malayalam}
ജനങ്ങളോട് അനീതി കാണിക്കുകയും ന്യായമില്ലാതെ ഭൂമിയില്‍ അതിക്രമം പ്രവര്‍ത്തിക്കുകയും ചെയ്യുന്നവര്‍ക്കെതിരില്‍ മാത്രമേ (കുറ്റം ചുമത്താന്‍) മാര്‍ഗമുള്ളൂ. അത്തരക്കാര്‍ക്ക് തന്നെയാകുന്നു വേദനയേറിയ ശിക്ഷയുള്ളതും.
\end{malayalam}}
\flushright{\begin{Arabic}
\quranayah[42][43]
\end{Arabic}}
\flushleft{\begin{malayalam}
വല്ലവനും ക്ഷമിക്കുകയും പൊറുക്കുകയും ചെയ്യുന്ന പക്ഷം തീര്‍ച്ചയായും അത് ദൃഢനിശ്ചയം ചെയ്യേണ്ട കാര്യങ്ങളില്‍ പെട്ടതാകുന്നു.
\end{malayalam}}
\flushright{\begin{Arabic}
\quranayah[42][44]
\end{Arabic}}
\flushleft{\begin{malayalam}
അല്ലാഹു ഏതൊരുവനെ വഴിപിഴവിലാക്കിയോ അവന്ന് അതിന് ശേഷം യാതൊരു രക്ഷാധികാരിയുമില്ല. ശിക്ഷ നേരില്‍ കാണുമ്പോള്‍ ഒരു തിരിച്ചുപോക്കിന് വല്ല മാര്‍ഗവുമുണ്ടോ എന്ന് അക്രമകാരികള്‍ പറയുന്നതായി നിനക്ക് കാണാം.
\end{malayalam}}
\flushright{\begin{Arabic}
\quranayah[42][45]
\end{Arabic}}
\flushleft{\begin{malayalam}
നിന്ദ്യതയാല്‍ കീഴൊതുങ്ങിയവരായിക്കൊണ്ട് അവര്‍ അതിന് (നരകത്തിന്‌) മുമ്പില്‍ പ്രദര്‍ശിപ്പിക്കപ്പെടുന്നത് നിനക്ക് കാണാം. ഒളികണ്ണിട്ടായിരിക്കും അവര്‍ നോക്കുന്നത്‌. വിശ്വസിച്ചവര്‍ പറയും: ഉയിര്‍ത്തെഴുന്നേല്‍പിന്‍റെ നാളില്‍ സ്വദേഹങ്ങളും തങ്ങളുടെ സ്വന്തക്കാരും നഷ്ടപ്പെട്ടവരാരോ, അവര്‍ തന്നെയാകുന്നു തീര്‍ച്ചയായും നഷ്ടക്കാര്‍. ശ്രദ്ധിക്കുക; തീര്‍ച്ചയായും അക്രമികള്‍ നിരന്തരമായ ശിക്ഷയിലാകുന്നു.
\end{malayalam}}
\flushright{\begin{Arabic}
\quranayah[42][46]
\end{Arabic}}
\flushleft{\begin{malayalam}
അല്ലാഹുവിന് പുറമെ തങ്ങളെ സഹായിക്കുന്ന രക്ഷാധികാരികളാരും അവര്‍ക്ക് ഉണ്ടായിരിക്കുകയുമില്ല. ഏതൊരുവനെ അല്ലാഹു വഴിപിഴവിലാക്കിയോ അവന്ന് (ലക്ഷ്യപ്രാപ്തിക്ക്‌) യാതൊരു മാര്‍ഗവുമില്ല.
\end{malayalam}}
\flushright{\begin{Arabic}
\quranayah[42][47]
\end{Arabic}}
\flushleft{\begin{malayalam}
ഒരു ദിവസം വന്നെത്തുന്നതിന് മുമ്പായി നിങ്ങളുടെ രക്ഷിതാവിന്‍റെ ആഹ്വാനം നിങ്ങള്‍ സ്വീകരിക്കുക. അല്ലാഹുവിങ്കല്‍ നിന്നുള്ള ആ ദിവസത്തെ തടുക്കുക സാധ്യമല്ല. അന്ന് നിങ്ങള്‍ക്ക് യാതൊരു അഭയസ്ഥാനവുമുണ്ടാവില്ല. നിങ്ങള്‍ക്ക് (കുറ്റങ്ങള്‍) നിഷേധിക്കാനുമാവില്ല.
\end{malayalam}}
\flushright{\begin{Arabic}
\quranayah[42][48]
\end{Arabic}}
\flushleft{\begin{malayalam}
ഇനി അവര്‍ തിരിഞ്ഞുകളയുകയാണെങ്കില്‍ (നബിയേ,) നിന്നെ നാം അവരുടെ മേല്‍ കാവല്‍ക്കാരനായി അയച്ചിട്ടില്ല. നിന്‍റെ മേല്‍ പ്രബോധന ബാധ്യത മാത്രമേയുള്ളു. തീര്‍ച്ചയായും നാം മനുഷ്യന് നമ്മുടെ പക്കല്‍ നിന്നുള്ള ഒരു അനുഗ്രഹം ആസ്വദിപ്പിച്ചാല്‍ അതിന്‍റെ പേരില്‍ അവന്‍ ആഹ്ലാദം കൊള്ളുന്നു. അവരുടെ കൈകള്‍ മുമ്പ് ചെയ്തു വെച്ചതിന്‍റെ ഫലമായി അവര്‍ക്ക് വല്ല തിന്‍മയും ബാധിക്കുന്നുവെങ്കിലോ അപ്പോഴതാ മനുഷ്യന്‍ നന്ദികെട്ടവന്‍ തന്നെയാകുന്നു.
\end{malayalam}}
\flushright{\begin{Arabic}
\quranayah[42][49]
\end{Arabic}}
\flushleft{\begin{malayalam}
അല്ലാഹുവിന്നാകുന്നു ആകാശങ്ങളുടെയും ഭൂമിയുടെയും ആധിപത്യം.അവന്‍ ഉദ്ദേശിക്കുന്നത് അവന്‍ സൃഷ്ടിക്കുന്നു. അവന്‍ ഉദ്ദേശിക്കുന്നവര്‍ക്ക് അവന്‍ പെണ്‍മക്കളെ പ്രദാനം ചെയ്യുന്നു. അവന്‍ ഉദ്ദേശിക്കുന്നവര്‍ക്ക് ആണ്‍മക്കളെയും പ്രദാനം ചെയ്യുന്നു.
\end{malayalam}}
\flushright{\begin{Arabic}
\quranayah[42][50]
\end{Arabic}}
\flushleft{\begin{malayalam}
അല്ലെങ്കില്‍ അവര്‍ക്ക് അവന്‍ ആണ്‍മക്കളെയും പെണ്‍മക്കളെയും ഇടകലര്‍ത്തികൊടുക്കുന്നു. അവന്‍ ഉദ്ദേശിക്കുന്നവരെ അവന്‍ വന്ധ്യരാക്കുകയും ചെയ്യുന്നു. തീര്‍ച്ചയായും അവന്‍ സര്‍വ്വജ്ഞനും സര്‍വ്വശക്തനുമാകുന്നു.
\end{malayalam}}
\flushright{\begin{Arabic}
\quranayah[42][51]
\end{Arabic}}
\flushleft{\begin{malayalam}
(നേരിട്ടുള്ള) ഒരു ബോധനം എന്ന നിലയിലോ ഒരു മറയുടെ പിന്നില്‍ നിന്നായിക്കൊണ്ടോ, ഒരു ദൂതനെ അയച്ച് അല്ലാഹുവിന്‍റെ അനുവാദപ്രകാരം അവന്‍ ഉദ്ദേശിക്കുന്നത് അദ്ദേഹം (ദൂതന്‍) ബോധനം നല്‍കുക എന്ന നിലയിലോ അല്ലാതെ അല്ലാഹു തന്നോട് സംസാരിക്കുക എന്ന കാര്യം യാതൊരു മനുഷ്യന്നും ഉണ്ടാവുകയില്ല. തീര്‍ച്ചയായും അവന്‍ ഉന്നതനും യുക്തിമാനുമാകുന്നു.
\end{malayalam}}
\flushright{\begin{Arabic}
\quranayah[42][52]
\end{Arabic}}
\flushleft{\begin{malayalam}
അപ്രകാരം തന്നെ നിനക്ക് നാം നമ്മുടെ കല്‍പനയാല്‍ ഒരു ചൈതന്യവത്തായ സന്ദേശം ബോധനം ചെയ്തിരിക്കുന്നു. വേദഗ്രന്ഥമോ സത്യവിശ്വാസമോ എന്തെന്ന് നിനക്കറിയുമായിരുന്നില്ല. പക്ഷെ, നാം അതിനെ ഒരു പ്രകാശമാക്കിയിരിക്കുന്നു. അതുമുഖേന നമ്മുടെ ദാസന്‍മാരില്‍ നിന്ന് നാം ഉദ്ദേശിക്കുന്നവര്‍ക്ക് നാം വഴി കാണിക്കുന്നു. തീര്‍ച്ചയായും നീ നേരായ പാതയിലേക്കാകുന്നു മാര്‍ഗദര്‍ശനം നല്‍കുന്നത്‌.
\end{malayalam}}
\flushright{\begin{Arabic}
\quranayah[42][53]
\end{Arabic}}
\flushleft{\begin{malayalam}
ആകാശങ്ങളിലുള്ളതും ഭൂമിയിലുള്ളതും ഏതൊരുവന്നുള്ളതാണോ ആ അല്ലാഹുവിന്‍റെ പാതയിലേക്ക്‌. ശ്രദ്ധിക്കുക; അല്ലാഹുവിലേക്കാകുന്നു കാര്യങ്ങള്‍ ചെന്നെത്തുന്നത്‌.
\end{malayalam}}
\chapter{\textmalayalam{സുഖ്റുഫ് ( സുവര്‍ണ്ണാലങ്കാരം )}}
\begin{Arabic}
\Huge{\centerline{\basmalah}}\end{Arabic}
\flushright{\begin{Arabic}
\quranayah[43][1]
\end{Arabic}}
\flushleft{\begin{malayalam}
ഹാമീം.
\end{malayalam}}
\flushright{\begin{Arabic}
\quranayah[43][2]
\end{Arabic}}
\flushleft{\begin{malayalam}
(കാര്യങ്ങള്‍) വിശദമാക്കുന്ന വേദഗ്രന്ഥം തന്നെയാണ,
\end{malayalam}}
\flushright{\begin{Arabic}
\quranayah[43][3]
\end{Arabic}}
\flushleft{\begin{malayalam}
തീര്‍ച്ചയായും നാം ഇതിനെ അറബി ഭാഷയിലുള്ള ഒരു ഖുര്‍ആന്‍ ആക്കിയിരിക്കുന്നത് നിങ്ങള്‍ ചിന്തിച്ചു മനസ്സിലാക്കുവാന്‍ വേണ്ടിയാകുന്നു.
\end{malayalam}}
\flushright{\begin{Arabic}
\quranayah[43][4]
\end{Arabic}}
\flushleft{\begin{malayalam}
തീര്‍ച്ചയായും അത് മൂലഗ്രന്ഥത്തില്‍ നമ്മുടെ അടുക്കല്‍ (സൂക്ഷിക്കപ്പെട്ടതത്രെ.) അത് ഉന്നതവും വിജ്ഞാനസമ്പന്നവും തന്നെയാകുന്നു.
\end{malayalam}}
\flushright{\begin{Arabic}
\quranayah[43][5]
\end{Arabic}}
\flushleft{\begin{malayalam}
അപ്പോള്‍ നിങ്ങള്‍ അതിക്രമകാരികളായ ജനങ്ങളായതിനാല്‍ (നിങ്ങളെ) ഒഴിവാക്കി വിട്ടുകൊണ്ട് ഈ ഉല്‍ബോധനം നിങ്ങളില്‍ നിന്ന് മേറ്റീവ്ക്കുകയോ?
\end{malayalam}}
\flushright{\begin{Arabic}
\quranayah[43][6]
\end{Arabic}}
\flushleft{\begin{malayalam}
പൂര്‍വ്വസമുദായങ്ങളില്‍ എത്രയോ പ്രവാചകന്‍മാരെ നാം നിയോഗിച്ചിട്ടുണ്ട്‌.
\end{malayalam}}
\flushright{\begin{Arabic}
\quranayah[43][7]
\end{Arabic}}
\flushleft{\begin{malayalam}
ഏതൊരു പ്രവാചകന്‍ അവരുടെ അടുത്ത് ചെല്ലുകയാണെങ്കിലും അവര്‍ അദ്ദേഹത്തെ പരിഹസിക്കുന്നവരാകാതിരുന്നിട്ടില്ല.
\end{malayalam}}
\flushright{\begin{Arabic}
\quranayah[43][8]
\end{Arabic}}
\flushleft{\begin{malayalam}
അങ്ങനെ ഇവരെക്കാള്‍ കനത്ത കൈയ്യൂക്കുണ്ടായിരുന്നവരെ നാം നശിപ്പിച്ചു കളഞ്ഞു. പൂര്‍വ്വികന്‍മാരുടെ ഉദാഹരണങ്ങള്‍ മുമ്പ് കഴിഞ്ഞുപോയിട്ടുമുണ്ട്‌.
\end{malayalam}}
\flushright{\begin{Arabic}
\quranayah[43][9]
\end{Arabic}}
\flushleft{\begin{malayalam}
ആകാശങ്ങളും ഭൂമിയും സൃഷ്ടിച്ചതാരാണെന്ന് നീ അവരോട് ചോദിച്ചാല്‍ തീര്‍ച്ചയായും അവര്‍ പറയും; പ്രതാപിയും സര്‍വ്വജ്ഞനുമായിട്ടുള്ളവനാണ് അവ സൃഷ്ടിച്ചത് എന്ന്‌.
\end{malayalam}}
\flushright{\begin{Arabic}
\quranayah[43][10]
\end{Arabic}}
\flushleft{\begin{malayalam}
അതെ, നിങ്ങള്‍ക്ക് വേണ്ടി ഭൂമിയെ ഒരു തൊട്ടിലാക്കുകയും നിങ്ങള്‍ നേരായ മാര്‍ഗം കണ്ടെത്താന്‍ വേണ്ടി നിങ്ങള്‍ക്കവിടെ പാതകളുണ്ടക്കിത്തരികയും ചെയ്തവന്‍.
\end{malayalam}}
\flushright{\begin{Arabic}
\quranayah[43][11]
\end{Arabic}}
\flushleft{\begin{malayalam}
ആകാശത്ത് നിന്ന് ഒരു തോത് അനുസരിച്ച് വെള്ളം വര്‍ഷിച്ചു തരികയും ചെയ്തവന്‍. എന്നിട്ട് അത് മൂലം നാം നിര്‍ജീവമായ രാജ്യത്തെ പുനരുജ്ജീവിപ്പിച്ചു. അത് പോലെ തന്നെ നിങ്ങളും (മരണാനന്തരം) പുറത്തു കൊണ്ടു വരപ്പെടുന്നതാണ്‌.
\end{malayalam}}
\flushright{\begin{Arabic}
\quranayah[43][12]
\end{Arabic}}
\flushleft{\begin{malayalam}
എല്ലാ ഇണകളെയും സൃഷ്ടിക്കുകയും നിങ്ങള്‍ക്ക് സവാരി ചെയ്യാനുള്ള കപ്പലുകളും കാലികളെയും നിങ്ങള്‍ക്ക് ഏര്‍പെടുത്തിത്തരികയും ചെയ്തവന്‍.
\end{malayalam}}
\flushright{\begin{Arabic}
\quranayah[43][13]
\end{Arabic}}
\flushleft{\begin{malayalam}
അവയുടെ പുറത്ത് നിങ്ങള്‍ ഇരിപ്പുറപ്പിക്കാനും എന്നിട്ട് നിങ്ങള്‍ അവിടെ ഇരിപ്പുറപ്പിച്ചു കഴിയുമ്പോള്‍ നിങ്ങളുടെ രക്ഷിതാവിന്‍റെ അനുഗ്രഹം നിങ്ങള്‍ ഓര്‍മിക്കുവാനും, നിങ്ങള്‍ ഇപ്രകാരം പറയുവാനും വേണ്ടി: ഞങ്ങള്‍ക്ക് വേണ്ടി ഇതിനെ വിധേയമാക്കിത്തന്നവന്‍ എത്ര പരിശുദ്ധന്‍! ഞങ്ങള്‍ക്കതിനെ ഇണക്കുവാന്‍ കഴിയുമായിരുന്നില്ല.
\end{malayalam}}
\flushright{\begin{Arabic}
\quranayah[43][14]
\end{Arabic}}
\flushleft{\begin{malayalam}
തീര്‍ച്ചയായും ഞങ്ങള്‍ ഞങ്ങളുടെ രക്ഷിതാവിങ്കലേക്ക് തിരിച്ചെത്തുന്നവര്‍ തന്നെയാകുന്നു.
\end{malayalam}}
\flushright{\begin{Arabic}
\quranayah[43][15]
\end{Arabic}}
\flushleft{\begin{malayalam}
അവന്‍റെ ദാസന്‍മാരില്‍ ഒരു വിഭാഗത്തെ അവരതാ അവന്‍റെ അംശം (അഥവാ മക്കള്‍) ആക്കിവെച്ചിരിക്കുന്നു. തീര്‍ച്ചയായും മനുഷ്യന്‍ പ്രത്യക്ഷമായിത്തന്നെ തികച്ചും നന്ദികെട്ടവനാകുന്നു.
\end{malayalam}}
\flushright{\begin{Arabic}
\quranayah[43][16]
\end{Arabic}}
\flushleft{\begin{malayalam}
അതല്ല, താന്‍ സൃഷ്ടിക്കുന്ന കൂട്ടത്തില്‍ നിന്ന് പെണ്‍മക്കളെ അവന്‍ (സ്വന്തമായി) സ്വീകരിക്കുകയും, ആണ്‍മക്കളെ നിങ്ങള്‍ക്ക് പ്രത്യേകമായി നല്‍കുകയും ചെയ്തിരിക്കുകയാണോ?
\end{malayalam}}
\flushright{\begin{Arabic}
\quranayah[43][17]
\end{Arabic}}
\flushleft{\begin{malayalam}
അവരില്‍ ഒരാള്‍ക്ക്‌, താന്‍ പരമകാരുണികന്ന് ഉപമയായി എടുത്തുകാണിക്കാറുള്ളതിനെ (പെണ്‍കുഞ്ഞിനെ) പ്പറ്റി സന്തോഷവാര്‍ത്ത അറിയിക്കപ്പെട്ടാല്‍ അവന്‍റെ മുഖം കരുവാളിച്ചതാകുകയും അവന്‍ കുണ്ഠിതനാവുകയും ചെയ്യുന്നു.
\end{malayalam}}
\flushright{\begin{Arabic}
\quranayah[43][18]
\end{Arabic}}
\flushleft{\begin{malayalam}
ആഭരണമണിയിച്ച് വളര്‍ത്തപ്പെടുന്ന, വാഗ്വാദത്തില്‍ (ന്യായം) തെളിയിക്കാന്‍ കഴിവില്ലാത്ത ഒരാളാണോ (അല്ലാഹുവിന് സന്താനമായി കല്‍പിക്കപ്പെടുന്നത്‌?)
\end{malayalam}}
\flushright{\begin{Arabic}
\quranayah[43][19]
\end{Arabic}}
\flushleft{\begin{malayalam}
പരമകാരുണികന്‍റെ ദാസന്‍മാരായ മലക്കുകളെ അവര്‍ പെണ്ണുങ്ങളാക്കിയിരിക്കുന്നു. അവരെ (മലക്കുകളെ) സൃഷ്ടിച്ചതിന് അവര്‍ സാക്ഷ്യം വഹിച്ചിരുന്നോ? അവരുടെ സാക്ഷ്യം രേഖപ്പെടുത്തുന്നതും അവര്‍ ചോദ്യം ചെയ്യപ്പെടുന്നതുമാണ്‌.
\end{malayalam}}
\flushright{\begin{Arabic}
\quranayah[43][20]
\end{Arabic}}
\flushleft{\begin{malayalam}
പരമകാരുണികന്‍ ഉദ്ദേശിച്ചിരുന്നെങ്കില്‍ ഞങ്ങള്‍ അവരെ (മലക്കുകളെ) ആരാധിക്കുമായിരുന്നില്ല. എന്ന് അവര്‍ പറയുകയും ചെയ്യും. അവര്‍ക്ക് അതിനെ പറ്റി യാതൊരു അറിവുമില്ല. അവര്‍ ഊഹിച്ച് പറയുക മാത്രമാകുന്നു.
\end{malayalam}}
\flushright{\begin{Arabic}
\quranayah[43][21]
\end{Arabic}}
\flushleft{\begin{malayalam}
അതല്ല, അവര്‍ക്ക് നാം ഇതിനു മുമ്പ് വല്ല ഗ്രന്ഥവും നല്‍കിയിട്ട് അവര്‍ അതില്‍ മുറുകെപിടിച്ച് നില്‍ക്കുകയാണോ?
\end{malayalam}}
\flushright{\begin{Arabic}
\quranayah[43][22]
\end{Arabic}}
\flushleft{\begin{malayalam}
അല്ല, ഞങ്ങളുടെ പിതാക്കള്‍ ഒരു മാര്‍ഗത്തില്‍ നിലകൊള്ളുന്നതായി ഞങ്ങള്‍ കണ്ടെത്തിയിരിക്കുന്നുഠീര്‍ച്ചയായും ഞങ്ങള്‍ അവരുടെ കാല്‍പാടുകളില്‍ നേര്‍മാര്‍ഗം കണ്ടെത്തിയിരിക്കയാണ്‌. എന്നാണ് അവര്‍ പറഞ്ഞത്‌.
\end{malayalam}}
\flushright{\begin{Arabic}
\quranayah[43][23]
\end{Arabic}}
\flushleft{\begin{malayalam}
അത് പോലെത്തന്നെ നിനക്ക് മുമ്പ് ഏതൊരു രാജ്യത്ത് നാം താക്കീതുകാരനെ അയച്ചപ്പോഴും ഞങ്ങളുടെ പിതാക്കളെ ഒരു മാര്‍ഗത്തില്‍ നിലകൊള്ളുന്നവരായി ഞങ്ങള്‍ കണ്ടെത്തിയിരിക്കുന്നു; തീര്‍ച്ചയായും ഞങ്ങള്‍ അവരുടെ കാല്‍പാടുകളെ അനുഗമിക്കുന്നവരാകുന്നു. എന്ന് അവിടെയുള്ള സുഖലോലുപന്‍മാര്‍ പറയാതിരുന്നിട്ടില്ല.
\end{malayalam}}
\flushright{\begin{Arabic}
\quranayah[43][24]
\end{Arabic}}
\flushleft{\begin{malayalam}
അദ്ദേഹം (താക്കീതുകാരന്‍) പറഞ്ഞു: നിങ്ങള്‍ നിങ്ങളുടെ പിതാക്കളെ ഏതൊരു മാര്‍ഗത്തില്‍ കണ്ടെത്തിയോ, അതിനെക്കാളും നല്ല മാര്‍ഗം കാണിച്ചുതരുന്ന ഒരു സന്ദേശവും കൊണ്ട് ഞാന്‍ നിങ്ങളുടെ അടുത്ത് വന്നാലും (നിങ്ങള്‍ പിതാക്കളെത്തന്നെ അനുകരിക്കുകയോ?) അവര്‍ പറഞ്ഞു; നിങ്ങള്‍ ഏതൊരു സന്ദേശവും കൊണ്ട് അയക്കപ്പെട്ടിരിക്കുന്നുവോ അതില്‍ തീര്‍ച്ചയായും ഞങ്ങള്‍ വിശ്വാസമില്ലാത്തവരാകുന്നു.
\end{malayalam}}
\flushright{\begin{Arabic}
\quranayah[43][25]
\end{Arabic}}
\flushleft{\begin{malayalam}
അതിനാല്‍ നാം അവര്‍ക്ക് ശിക്ഷ നല്‍കി. അപ്പോള്‍ ആ സത്യനിഷേധികളുടെ പര്യവസാനം എങ്ങനെയായിരുന്നു വെന്ന് നോക്കുക.
\end{malayalam}}
\flushright{\begin{Arabic}
\quranayah[43][26]
\end{Arabic}}
\flushleft{\begin{malayalam}
ഇബ്രാഹീം തന്‍റെ പിതാവിനോടും ജനങ്ങളോടും ഇപ്രകാരം പറഞ്ഞ സന്ദര്‍ഭം (ശ്രദ്ധേയമാകുന്നു:) തീര്‍ച്ചയായും ഞാന്‍ നിങ്ങള്‍ ആരാധിക്കുന്നതില്‍നിന്ന് ഒഴിഞ്ഞു നില്‍ക്കുന്നവനാകുന്നു.
\end{malayalam}}
\flushright{\begin{Arabic}
\quranayah[43][27]
\end{Arabic}}
\flushleft{\begin{malayalam}
എന്നെ സൃഷ്ടിച്ചവനൊഴികെ. കാരണം തീര്‍ച്ചയായും അവന്‍ എനിക്ക് മാര്‍ഗദര്‍ശനം നല്‍കുന്നതാണ്‌.
\end{malayalam}}
\flushright{\begin{Arabic}
\quranayah[43][28]
\end{Arabic}}
\flushleft{\begin{malayalam}
അദ്ദേഹത്തിന്‍റെ പിന്‍ഗാമികള്‍ (സത്യത്തിലേക്കു) മടങ്ങേണ്ടതിനായി അതിനെ (ആ പ്രഖ്യാപനത്തെ) അദ്ദേഹം അവര്‍ക്കിടയില്‍ നിലനില്‍ക്കുന്ന ഒരു വചനമാക്കുകയും ചെയ്തു.
\end{malayalam}}
\flushright{\begin{Arabic}
\quranayah[43][29]
\end{Arabic}}
\flushleft{\begin{malayalam}
അല്ല, ഇക്കൂട്ടര്‍ക്കും അവരുടെ പിതാക്കള്‍ക്കും ഞാന്‍ ജീവിതസുഖം നല്‍കി. സത്യസന്ദേശവും, വ്യക്തമായി വിവരിച്ചുകൊടുക്കുന്ന ഒരു ദൂതനും അവരുടെ അടുത്ത് വരുന്നത് വരെ.
\end{malayalam}}
\flushright{\begin{Arabic}
\quranayah[43][30]
\end{Arabic}}
\flushleft{\begin{malayalam}
അവര്‍ക്ക് സത്യം വന്നെത്തിയപ്പോഴാകട്ടെ അവര്‍ പറഞ്ഞു: ഇതൊരു മായാജാലമാണ്‌. തീര്‍ച്ചയായും ഞങ്ങള്‍ അതില്‍ വിശ്വാസമില്ലാത്തവരാകുന്നു.
\end{malayalam}}
\flushright{\begin{Arabic}
\quranayah[43][31]
\end{Arabic}}
\flushleft{\begin{malayalam}
ഈ രണ്ട് പട്ടണങ്ങളില്‍ നിന്നുള്ള ഏതെങ്കിലും ഒരു മഹാപുരുഷന്‍റെ മേല്‍ എന്തുകൊണ്ട് ഈ ഖുര്‍ആന്‍ അവതരിപ്പിക്കപ്പെട്ടില്ല എന്നും അവര്‍ പറഞ്ഞു.
\end{malayalam}}
\flushright{\begin{Arabic}
\quranayah[43][32]
\end{Arabic}}
\flushleft{\begin{malayalam}
അവരാണോ നിന്‍റെ രക്ഷിതാവിന്‍റെ അനുഗ്രഹം പങ്ക് വെച്ചു കൊടുക്കുന്നത്‌? നാമാണ് ഐഹികജീവിതത്തില്‍ അവര്‍ക്കിടയില്‍ അവരുടെ ജീവിതമാര്‍ഗം പങ്ക് വെച്ചുകൊടുത്തത്‌. അവരില്‍ ചിലര്‍ക്ക് ചിലരെ കീഴാളരാക്കി വെക്കത്തക്കവണ്ണം അവരില്‍ ചിലരെ മറ്റു ചിലരെക്കാള്‍ ഉപരി നാം പല പടികള്‍ ഉയര്‍ത്തുകയും ചെയ്തിരിക്കുന്നു. നിന്‍റെ രക്ഷിതാവിന്‍റെ കാരുണ്യമാകുന്നു അവര്‍ ശേഖരിച്ചു വെക്കുന്നതിനെക്കാള്‍ ഉത്തമം.
\end{malayalam}}
\flushright{\begin{Arabic}
\quranayah[43][33]
\end{Arabic}}
\flushleft{\begin{malayalam}
മനുഷ്യര്‍ ഒരേ തരത്തിലുള്ള (ദുഷിച്ച) സമുദായമായിപ്പോകുകയില്ലായിരുന്നെങ്കില്‍ പരമകാരുണികനില്‍ അവിശ്വസിക്കുന്നവര്‍ക്ക് അവരുടെ വീടുകള്‍ക്ക് വെള്ളി കൊണ്ടുള്ള മേല്‍പുരകളും, അവര്‍ക്ക് കയറിപോകാന്‍ (വെള്ളികൊണ്ടുള്ള) കോണികളും നാം ഉണ്ടാക്കികൊടുക്കുമായിരുന്നു.
\end{malayalam}}
\flushright{\begin{Arabic}
\quranayah[43][34]
\end{Arabic}}
\flushleft{\begin{malayalam}
അവരുടെ വീടുകള്‍ക്ക് (വെള്ളി) വാതിലുകളും അവര്‍ക്ക് ചാരിയിരിക്കാന്‍ (വെള്ളി) കട്ടിലുകളും
\end{malayalam}}
\flushright{\begin{Arabic}
\quranayah[43][35]
\end{Arabic}}
\flushleft{\begin{malayalam}
സ്വര്‍ണം കൊണ്ടുള്ള അലങ്കാരവും നാം നല്‍കുമായിരുന്നു. എന്നാല്‍ അതെല്ലാം ഐഹികജീവിതത്തിലെ സുഖഭോഗം മാത്രമാകുന്നു. പരലോകം തന്‍റെ രക്ഷിതാവിന്‍റെ അടുക്കല്‍ സൂക്ഷ്മത പാലിക്കുന്നവര്‍ക്കുള്ളതാകുന്നു.
\end{malayalam}}
\flushright{\begin{Arabic}
\quranayah[43][36]
\end{Arabic}}
\flushleft{\begin{malayalam}
പരമകാരുണികന്‍റെ ഉല്‍ബോധനത്തിന്‍റെ നേര്‍ക്ക് വല്ലവനും അന്ധത നടിക്കുന്ന പക്ഷം അവന്നു നാം ഒരു പിശാചിനെ ഏര്‍പെടുത്തികൊടുക്കും. എന്നിട്ട് അവന്‍ (പിശാച്‌) അവന്ന് കൂട്ടാളിയായിരിക്കും
\end{malayalam}}
\flushright{\begin{Arabic}
\quranayah[43][37]
\end{Arabic}}
\flushleft{\begin{malayalam}
തീര്‍ച്ചയായും അവര്‍ (പിശാചുക്കള്‍) അവരെ നേര്‍മാര്‍ഗത്തില്‍ നിന്ന് തടയും. തങ്ങള്‍ സന്‍മാര്‍ഗം പ്രാപിച്ചവരാണെന്ന് അവര്‍ വിചാരിക്കുകയും ചെയ്യും.
\end{malayalam}}
\flushright{\begin{Arabic}
\quranayah[43][38]
\end{Arabic}}
\flushleft{\begin{malayalam}
അങ്ങനെ നമ്മുടെ അടുത്ത് വന്നെത്തുമ്പോള്‍ (തന്‍റെ കൂട്ടാളിയായ പിശാചിനോട്‌) അവന്‍ പറയും: എനിക്കും നിനക്കുമിടയില്‍ ഉദയാസ്തമനസ്ഥാനങ്ങള്‍ തമ്മിലുള്ള അകലം ഉണ്ടായിരുന്നെങ്കില്‍ എത്ര നന്നായിരുന്നേനെ. അപ്പോള്‍ ആ കൂട്ടുകാരന്‍ എത്ര ചീത്ത!
\end{malayalam}}
\flushright{\begin{Arabic}
\quranayah[43][39]
\end{Arabic}}
\flushleft{\begin{malayalam}
നിങ്ങള്‍ അക്രമം പ്രവര്‍ത്തിച്ചിരിക്കെ നിങ്ങള്‍ ശിക്ഷയില്‍ പങ്കാളികളാകുന്നു എന്ന വസ്തുത ഇന്ന് നിങ്ങള്‍ക്ക് ഒട്ടും പ്രയോജനപ്പെടുകയില്ല.
\end{malayalam}}
\flushright{\begin{Arabic}
\quranayah[43][40]
\end{Arabic}}
\flushleft{\begin{malayalam}
എന്നാല്‍ (നബിയേ,) നിനക്ക് ബധിരന്‍മാരെ കേള്‍പിക്കാനും, അന്ധന്‍മാരെയും വ്യക്തമായ ദുര്‍മാര്‍ഗത്തിലായവരെയും വഴി കാണിക്കാനും കഴിയുമോ?
\end{malayalam}}
\flushright{\begin{Arabic}
\quranayah[43][41]
\end{Arabic}}
\flushleft{\begin{malayalam}
ഇനി നിന്നെ നാം കൊണ്ടു പോകുന്ന പക്ഷം അവരുടെ നേരെ നാം ശിക്ഷാനടപടി എടുക്കുക തന്നെ ചെയ്യുന്നതാണ്‌.
\end{malayalam}}
\flushright{\begin{Arabic}
\quranayah[43][42]
\end{Arabic}}
\flushleft{\begin{malayalam}
അഥവാ നാം അവര്‍ക്ക് താക്കീത് നല്‍കിയത് (ശിക്ഷ) നിനക്ക് നാം കാട്ടിത്തരികയാണെങ്കിലോ നാം അവരുടെ കാര്യത്തില്‍ കഴിവുള്ളവന്‍ തന്നെയാകുന്നു.
\end{malayalam}}
\flushright{\begin{Arabic}
\quranayah[43][43]
\end{Arabic}}
\flushleft{\begin{malayalam}
ആകയാല്‍ നിനക്ക് ബോധനം നല്‍കപ്പെട്ടത് നീ മുറുകെപിടിക്കുക. തീര്‍ച്ചയായും നീ നേരായ പാതയിലാകുന്നു.
\end{malayalam}}
\flushright{\begin{Arabic}
\quranayah[43][44]
\end{Arabic}}
\flushleft{\begin{malayalam}
തീര്‍ച്ചയായും അത് നിനക്കും നിന്‍റെ ജനതയ്ക്കും ഒരു ഉല്‍ബോധനം തന്നെയാകുന്നു. വഴിയെ നിങ്ങള്‍ ചോദ്യം ചെയ്യപ്പെടുന്നതുമാണ്‌.
\end{malayalam}}
\flushright{\begin{Arabic}
\quranayah[43][45]
\end{Arabic}}
\flushleft{\begin{malayalam}
നിനക്ക് മുമ്പ് നമ്മുടെ ദൂതന്‍മാരായി നാം അയച്ചവരോട് ചോദിച്ചു നോക്കുക. പരമകാരുണികന് പുറമെ ആരാധിക്കപ്പെടേണ്ട വല്ല ദൈവങ്ങളേയും നാം നിശ്ചയിച്ചിട്ടുണ്ടോ എന്ന്‌.
\end{malayalam}}
\flushright{\begin{Arabic}
\quranayah[43][46]
\end{Arabic}}
\flushleft{\begin{malayalam}
മൂസായെ നമ്മുടെ ദൃഷ്ടാന്തങ്ങളുമായി ഫിര്‍ഔന്‍റെയും അവന്‍റെ പൌരമുഖ്യന്‍മാരുടെയും അടുത്തേക്ക് നാം അയക്കുകയുണ്ടായി. എന്നിട്ട് അദ്ദേഹം പറഞ്ഞു: തീര്‍ച്ചയായും ഞാന്‍ ലോകരക്ഷിതാവിന്‍റെ ദൂതനാകുന്നു.
\end{malayalam}}
\flushright{\begin{Arabic}
\quranayah[43][47]
\end{Arabic}}
\flushleft{\begin{malayalam}
അങ്ങനെ അദ്ദേഹം നമ്മുടെ ദൃഷ്ടാന്തങ്ങളുമായി അവരുടെ അടുത്ത് ചെന്നപ്പോള്‍ അവരതാ അവയെ കളിയാക്കി ചിരിക്കുന്നു.
\end{malayalam}}
\flushright{\begin{Arabic}
\quranayah[43][48]
\end{Arabic}}
\flushleft{\begin{malayalam}
അവര്‍ക്ക് നാം ഓരോ ദൃഷ്ടാന്തവും കാണിച്ചുകൊടുത്തു കൊണ്ടിരുന്നത് അതിന്‍റെ ഇണയെക്കാള്‍ മഹത്തരമായിക്കൊണ്ട് തന്നെയായിരുന്നു. അവര്‍ (ഖേദിച്ചു) മടങ്ങുവാന്‍ വേണ്ടി നാം അവരെ ശിക്ഷകള്‍ മുഖേന പിടികൂടുകയും ചെയ്തു.
\end{malayalam}}
\flushright{\begin{Arabic}
\quranayah[43][49]
\end{Arabic}}
\flushleft{\begin{malayalam}
അവര്‍ പറഞ്ഞു: ഹേ, ജാലവിദ്യക്കാരാ! താങ്കളുമായി താങ്കളുടെ രക്ഷിതാവ് കരാര്‍ ചെയ്തിട്ടുള്ളതനുസരിച്ച് ഞങ്ങള്‍ക്ക് വേണ്ടി താങ്കള്‍ അവനോട് പ്രാര്‍ത്ഥിക്കുക. തീര്‍ച്ചയായും ഞങ്ങള്‍ നേര്‍മാര്‍ഗം പ്രാപിക്കുന്നവര്‍ തന്നെയാകുന്നു.
\end{malayalam}}
\flushright{\begin{Arabic}
\quranayah[43][50]
\end{Arabic}}
\flushleft{\begin{malayalam}
എന്നിട്ട് അവരില്‍ നിന്ന് നാം ശിക്ഷ എടുത്തുകളഞ്ഞപ്പോള്‍ അവരതാ വാക്കുമാറുന്നു.
\end{malayalam}}
\flushright{\begin{Arabic}
\quranayah[43][51]
\end{Arabic}}
\flushleft{\begin{malayalam}
ഫിര്‍ഔന്‍ തന്‍റെ ജനങ്ങള്‍ക്കിടയില്‍ ഒരു വിളംബരം നടത്തി. അവന്‍ പറഞ്ഞു: എന്‍റെ ജനങ്ങളേ, ഈജിപ്തിന്‍റെ ആധിപത്യം എനിക്കല്ലേ? ഈ നദികള്‍ ഒഴുകുന്നതാകട്ടെ എന്‍റെ കീഴിലൂടെയാണ്‌. എന്നിരിക്കെ നിങ്ങള്‍ (കാര്യങ്ങള്‍) കണ്ടറിയുന്നില്ലേ?
\end{malayalam}}
\flushright{\begin{Arabic}
\quranayah[43][52]
\end{Arabic}}
\flushleft{\begin{malayalam}
അല്ല, ഹീനനായിട്ടുള്ളവനും വ്യക്തമായി സംസാരിക്കാന്‍ കഴിയാത്തവനുമായ ഇവനെക്കാള്‍ ഉത്തമന്‍ ഞാന്‍ തന്നെയാകുന്നു.
\end{malayalam}}
\flushright{\begin{Arabic}
\quranayah[43][53]
\end{Arabic}}
\flushleft{\begin{malayalam}
അപ്പോള്‍ ഇവന്‍റെ മേല്‍ സ്വര്‍ണവളകള്‍ അണിയിക്കപ്പെടുകയോ, ഇവനോടൊപ്പം തുണയായികൊണ്ട് മലക്കുകള്‍ വരികയോ ചെയ്യാത്തതെന്താണ്‌?
\end{malayalam}}
\flushright{\begin{Arabic}
\quranayah[43][54]
\end{Arabic}}
\flushleft{\begin{malayalam}
അങ്ങനെ ഫിര്‍ഔന്‍ തന്‍റെ ജനങ്ങളെ വിഡ്ഢികളാക്കി. അവര്‍ അവനെ അനുസരിച്ചു. തീര്‍ച്ചയായും അവര്‍ അധര്‍മ്മകാരികളായ ഒരു ജനതയായിരുന്നു.
\end{malayalam}}
\flushright{\begin{Arabic}
\quranayah[43][55]
\end{Arabic}}
\flushleft{\begin{malayalam}
അങ്ങനെ അവര്‍ നമ്മെ പ്രകോപിപ്പിച്ചപ്പോള്‍ നാം അവരെ ശിക്ഷിച്ചു. അവരെ നാം മുക്കി നശിപ്പിച്ചു.
\end{malayalam}}
\flushright{\begin{Arabic}
\quranayah[43][56]
\end{Arabic}}
\flushleft{\begin{malayalam}
അങ്ങനെ അവരെ പൂര്‍വ്വമാതൃകയും പിന്നീട് വരുന്നവര്‍ക്ക് ഒരു ഉദാഹരണവും ആക്കിത്തീര്‍ത്തു.
\end{malayalam}}
\flushright{\begin{Arabic}
\quranayah[43][57]
\end{Arabic}}
\flushleft{\begin{malayalam}
മര്‍യമിന്‍റെ മകന്‍ ഒരു ഉദാഹരണമായി എടുത്തുകാണിക്കപ്പെട്ടപ്പോള്‍ നിന്‍റെ ജനതയതാ അതിന്‍റെ പേരില്‍ ആര്‍ത്തുവിളിക്കുന്നു.
\end{malayalam}}
\flushright{\begin{Arabic}
\quranayah[43][58]
\end{Arabic}}
\flushleft{\begin{malayalam}
ഞങ്ങളുടെ ദൈവങ്ങളാണോ ഉത്തമം, അതല്ല, അദ്ദേഹമാണോ എന്നവര്‍ പറയുകയും ചെയ്തു. അവര്‍ നിന്‍റെ മുമ്പില്‍ അതെടുത്തു കാണിച്ചത് ഒരു തര്‍ക്കത്തിനായി മാത്രമാണ്‌. എന്നു തന്നെയല്ല അവര്‍ പിടിവാശിക്കാരായ ഒരു ജനവിഭാഗമാകുന്നു.
\end{malayalam}}
\flushright{\begin{Arabic}
\quranayah[43][59]
\end{Arabic}}
\flushleft{\begin{malayalam}
അദ്ദേഹം നമ്മുടെ ഒരു ദാസന്‍ മാത്രമാകുന്നു. അദ്ദേഹത്തിന് നാം അനുഗ്രഹം നല്‍കുകയും അദ്ദേഹത്തെ ഇസ്രായീല്‍ സന്തതികള്‍ക്ക് നാം ഒരു മാതൃകയാക്കുകയും ചെയ്തു.
\end{malayalam}}
\flushright{\begin{Arabic}
\quranayah[43][60]
\end{Arabic}}
\flushleft{\begin{malayalam}
നാം ഉദ്ദേശിച്ചിരുന്നെങ്കില്‍ (നിങ്ങളുടെ) പിന്‍തലമുറയായിരിക്കത്തക്കവിധം നിങ്ങളില്‍ നിന്നു തന്നെ നാം മലക്കുകളെ ഭൂമിയില്‍ ഉണ്ടാക്കുമായിരുന്നു.
\end{malayalam}}
\flushright{\begin{Arabic}
\quranayah[43][61]
\end{Arabic}}
\flushleft{\begin{malayalam}
തീര്‍ച്ചയായും അദ്ദേഹം അന്ത്യസമയത്തിന്നുള്ള ഒരു അറിയിപ്പാകുന്നു. അതിനാല്‍ അതിനെ (അന്ത്യസമയത്തെ) പ്പറ്റി നിങ്ങള്‍ സംശയിച്ചു പോകരുത്‌. എന്നെ നിങ്ങള്‍ പിന്തുടരുക. ഇതാകുന്നു നേരായ പാത.
\end{malayalam}}
\flushright{\begin{Arabic}
\quranayah[43][62]
\end{Arabic}}
\flushleft{\begin{malayalam}
പിശാച് (അതില്‍ നിന്ന്‌) നിങ്ങളെ തടയാതിരിക്കട്ടെ. തീര്‍ച്ചയായും അവന്‍ നിങ്ങള്‍ക്ക് പ്രത്യക്ഷശത്രുവാകുന്നു.
\end{malayalam}}
\flushright{\begin{Arabic}
\quranayah[43][63]
\end{Arabic}}
\flushleft{\begin{malayalam}
വ്യക്തമായ തെളിവുകളും കൊണ്ട് ഈസാ വന്നിട്ട് ഇപ്രകാരം പറഞ്ഞു: തീര്‍ച്ചയായും വിജ്ഞാനവും കൊണ്ടാണ് ഞാന്‍ നിങ്ങളുടെ അടുത്ത് വന്നിരിക്കുന്നത്‌. നിങ്ങള്‍ അഭിപ്രായഭിന്നത പുലര്‍ത്തികൊണ്ടിരിക്കുന്ന കാര്യങ്ങളില്‍ ചിലത് ഞാന്‍ നിങ്ങള്‍ക്ക് വിവരിച്ചുതരാന്‍ വേണ്ടിയും. ആകയാല്‍ നിങ്ങള്‍ അല്ലാഹുവിനെ സൂക്ഷിക്കുകയും എന്നെ അനുസരിക്കുകയും ചെയ്യുവിന്‍.
\end{malayalam}}
\flushright{\begin{Arabic}
\quranayah[43][64]
\end{Arabic}}
\flushleft{\begin{malayalam}
തീര്‍ച്ചയായും അല്ലാഹു തന്നെയാകുന്നു എന്‍റെ രക്ഷിതാവും, നിങ്ങളുടെ രക്ഷിതാവും. അതിനാല്‍ അവനെ നിങ്ങള്‍ ആരാധിക്കുക. ഇതാകുന്നു നേരായ പാത.
\end{malayalam}}
\flushright{\begin{Arabic}
\quranayah[43][65]
\end{Arabic}}
\flushleft{\begin{malayalam}
എന്നിട്ട് അവര്‍ക്കിടയിലുള്ള കക്ഷികള്‍ ഭിന്നിച്ചു. അതിനാല്‍ അക്രമം പ്രവര്‍ത്തിച്ചവര്‍ക്ക് വേദനയേറിയ ഒരു ദിവസത്തെ ശിക്ഷ മൂലം നാശം!
\end{malayalam}}
\flushright{\begin{Arabic}
\quranayah[43][66]
\end{Arabic}}
\flushleft{\begin{malayalam}
അവര്‍ ഓര്‍ക്കാതിരിക്കെ പെട്ടെന്ന് ആ അന്ത്യസമയം അവര്‍ക്ക് വന്നെത്തുന്നതിനെയല്ലാതെ അവര്‍ നോക്കിയിരിക്കുന്നുണ്ടോ?
\end{malayalam}}
\flushright{\begin{Arabic}
\quranayah[43][67]
\end{Arabic}}
\flushleft{\begin{malayalam}
സുഹൃത്തുക്കള്‍ ആ ദിവസം അന്യോന്യം ശത്രുക്കളായിരിക്കും. സൂക്ഷ്മത പാലിക്കുന്നവരൊഴികെ.
\end{malayalam}}
\flushright{\begin{Arabic}
\quranayah[43][68]
\end{Arabic}}
\flushleft{\begin{malayalam}
എന്‍റെ ദാസന്‍മാരേ, ഇന്ന് നിങ്ങള്‍ക്ക് യാതൊരു ഭയവുമില്ല. നിങ്ങള്‍ ദുഃഖിക്കേണ്ടതുമില്ല.
\end{malayalam}}
\flushright{\begin{Arabic}
\quranayah[43][69]
\end{Arabic}}
\flushleft{\begin{malayalam}
നമ്മുടെ ദൃഷ്ടാന്തങ്ങളില്‍ വിശ്വസിക്കുകയും കീഴ്പെട്ടു ജീവിക്കുന്നവരായിരിക്കുകയും ചെയ്തവരത്രെ(നിങ്ങള്‍)
\end{malayalam}}
\flushright{\begin{Arabic}
\quranayah[43][70]
\end{Arabic}}
\flushleft{\begin{malayalam}
നിങ്ങളും നിങ്ങളുടെ ഇണകളും സന്തോഷഭരിതരായികൊണ്ട് സ്വര്‍ഗത്തില്‍ പ്രവേശിച്ചു കൊള്ളുക.
\end{malayalam}}
\flushright{\begin{Arabic}
\quranayah[43][71]
\end{Arabic}}
\flushleft{\begin{malayalam}
സ്വര്‍ണത്തിന്‍റെ തളികകളും പാനപാത്രങ്ങളും അവര്‍ക്ക് ചുറ്റും കൊണ്ടു നടക്കപ്പെടും. മനസ്സുകള്‍ കൊതിക്കുന്നതും കണ്ണുകള്‍ക്ക് ആനന്ദകരവുമായ കാര്യങ്ങള്‍ അവിടെ ഉണ്ടായിരിക്കും. നിങ്ങള്‍ അവിടെ നിത്യവാസികളായിരിക്കുകയും ചെയ്യും.
\end{malayalam}}
\flushright{\begin{Arabic}
\quranayah[43][72]
\end{Arabic}}
\flushleft{\begin{malayalam}
നിങ്ങള്‍ പ്രവര്‍ത്തിച്ചുകൊണ്ടിരുന്നതിന്‍റെ ഫലമായി നിങ്ങള്‍ക്ക് അവകാശപ്പെടുത്തിത്തന്നിട്ടുള്ള സ്വര്‍ഗമത്രെ അത്‌.
\end{malayalam}}
\flushright{\begin{Arabic}
\quranayah[43][73]
\end{Arabic}}
\flushleft{\begin{malayalam}
നിങ്ങള്‍ക്കതില്‍ പഴങ്ങള്‍ ധാരാളമായി ഉണ്ടാകും. അതില്‍ നിന്ന് നിങ്ങള്‍ക്ക് ഭക്ഷിക്കാം.
\end{malayalam}}
\flushright{\begin{Arabic}
\quranayah[43][74]
\end{Arabic}}
\flushleft{\begin{malayalam}
തീര്‍ച്ചയായും കുറ്റവാളികള്‍ നരകശിക്ഷയില്‍ നിത്യവാസികളായിരിക്കും.
\end{malayalam}}
\flushright{\begin{Arabic}
\quranayah[43][75]
\end{Arabic}}
\flushleft{\begin{malayalam}
അവര്‍ക്ക് അത് ലഘൂകരിച്ച് കൊടുക്കപ്പെടുകയില്ല. അവര്‍ അതില്‍ ആശയറ്റവരായിരിക്കും.
\end{malayalam}}
\flushright{\begin{Arabic}
\quranayah[43][76]
\end{Arabic}}
\flushleft{\begin{malayalam}
നാം അവരോട് അക്രമം ചെയ്തിട്ടില്ല. പക്ഷെ അവര്‍ തന്നെയായിരുന്നു അക്രമകാരികള്‍.
\end{malayalam}}
\flushright{\begin{Arabic}
\quranayah[43][77]
\end{Arabic}}
\flushleft{\begin{malayalam}
അവര്‍ വിളിച്ചുപറയും; ഹേ, മാലിക്‌! താങ്കളുടെ രക്ഷിതാവ് ഞങ്ങളുടെ മേല്‍ (മരണം) വിധിക്കട്ടെ. അദ്ദേഹം (മാലിക്‌) പറയും: നിങ്ങള്‍ (ഇവിടെ) താമസിക്കേണ്ടവര്‍ തന്നെയാകുന്നു.
\end{malayalam}}
\flushright{\begin{Arabic}
\quranayah[43][78]
\end{Arabic}}
\flushleft{\begin{malayalam}
(അല്ലാഹു പറയും:) തീര്‍ച്ചയായും നാം നിങ്ങള്‍ക്ക് സത്യം കൊണ്ടു വന്ന് തരികയുണ്ടായി. പക്ഷെ നിങ്ങളില്‍ അധിക പേരും സത്യത്തെ വെറുക്കുന്നവരാകുന്നു.
\end{malayalam}}
\flushright{\begin{Arabic}
\quranayah[43][79]
\end{Arabic}}
\flushleft{\begin{malayalam}
അതല്ല, അവര്‍ (നമുക്കെതിരില്‍) വല്ല കാര്യവും തീരുമാനിച്ചിരിക്കയാണോ? എന്നാല്‍ നാം തന്നെയാകുന്നു തീരുമാനമെടുക്കുന്നവന്‍.
\end{malayalam}}
\flushright{\begin{Arabic}
\quranayah[43][80]
\end{Arabic}}
\flushleft{\begin{malayalam}
അതല്ല, അവരുടെ രഹസ്യവും ഗൂഢാലോചനയും നാം കേള്‍ക്കുന്നില്ല എന്ന് അവര്‍ വിചാരിക്കുന്നുണ്ടോ? അതെ, നമ്മുടെ ദൂതന്‍മാര്‍ (മലക്കുകള്‍) അവരുടെ അടുക്കല്‍ എഴുതിയെടുക്കുന്നുണ്ട്‌.
\end{malayalam}}
\flushright{\begin{Arabic}
\quranayah[43][81]
\end{Arabic}}
\flushleft{\begin{malayalam}
നീ പറയുക: പരമകാരുണികന് സന്താനമുണ്ടായിരുന്നെങ്കില്‍ ഞാനായിരിക്കും അതിനെ ആരാധിക്കുന്നവരില്‍ ഒന്നാമന്‍.
\end{malayalam}}
\flushright{\begin{Arabic}
\quranayah[43][82]
\end{Arabic}}
\flushleft{\begin{malayalam}
എന്നാല്‍ ആകാശങ്ങളുടെയും ഭൂമിയുടെയും രക്ഷിതാവ്‌, സിംഹാസനത്തിന്‍റെ നാഥന്‍ അവര്‍ പറഞ്ഞുണ്ടാക്കുന്നതില്‍ നിന്ന് എത്രയോ പരിശുദ്ധനത്രെ.
\end{malayalam}}
\flushright{\begin{Arabic}
\quranayah[43][83]
\end{Arabic}}
\flushleft{\begin{malayalam}
അതിനാല്‍ നീ അവരെ വിട്ടേക്കുക. അവര്‍ക്കു താക്കീത് നല്‍കപ്പെടുന്ന അവരുടെ ആ ദിവസം അവര്‍ കണ്ടുമുട്ടുന്നതു വരെ അവര്‍ അസംബന്ധങ്ങള്‍ പറയുകയും കളിതമാശയില്‍ ഏര്‍പെടുകയും ചെയ്തുകൊള്ളട്ടെ.
\end{malayalam}}
\flushright{\begin{Arabic}
\quranayah[43][84]
\end{Arabic}}
\flushleft{\begin{malayalam}
അവനാകുന്നു ആകാശത്ത് ദൈവമായിട്ടുള്ളവനും, ഭൂമിയില്‍ ദൈവമായിട്ടുള്ളവനും.അവനാകുന്നു യുക്തിമാനും സര്‍വ്വജ്ഞനും.
\end{malayalam}}
\flushright{\begin{Arabic}
\quranayah[43][85]
\end{Arabic}}
\flushleft{\begin{malayalam}
ആകാശങ്ങളുടെയും, ഭൂമിയുടെയും, അവയ്ക്കിടയിലുള്ളതിന്‍റെയും ആധിപത്യം ഏതൊരുവന്നാണോ അവന്‍ അനുഗ്രഹപൂര്‍ണ്ണനാകുന്നു. അവന്‍റെ പക്കല്‍ തന്നെയാകുന്നു ആ (അന്ത്യ) സമയത്തെപറ്റിയുള്ള അറിവ്‌. അവന്‍റെ അടുത്തേക്ക് തന്നെ നിങ്ങള്‍ മടക്കപ്പെടുകയും ചെയ്യും.
\end{malayalam}}
\flushright{\begin{Arabic}
\quranayah[43][86]
\end{Arabic}}
\flushleft{\begin{malayalam}
അവന്നു പുറമെ ഇവര്‍ ആരെ വിളിച്ചു പ്രാര്‍ത്ഥിക്കുന്നുവോ അവര്‍ ശുപാര്‍ശ അധീനപ്പെടുത്തുന്നില്ല; അറിഞ്ഞു കൊണ്ടു തന്നെ സത്യത്തിന് സാക്ഷ്യം വഹിച്ചവരൊഴികെ.
\end{malayalam}}
\flushright{\begin{Arabic}
\quranayah[43][87]
\end{Arabic}}
\flushleft{\begin{malayalam}
ആരാണ് അവരെ സൃഷ്ടിച്ചതെന്ന് നീ അവരോട് ചോദിച്ചാല്‍ തീര്‍ച്ചയായും അവര്‍ പറയും: അല്ലാഹു എന്ന്‌. അപ്പോള്‍ എങ്ങനെയാണ് അവര്‍ വ്യതിചലിപ്പിക്കപ്പെടുന്നത്‌?
\end{malayalam}}
\flushright{\begin{Arabic}
\quranayah[43][88]
\end{Arabic}}
\flushleft{\begin{malayalam}
എന്‍റെ രക്ഷിതാവേ! തീര്‍ച്ചയായും ഇക്കൂട്ടര്‍ വിശ്വസിക്കാത്ത ഒരു ജനതയാകുന്നു എന്ന് അദ്ദേഹം (പ്രവാചകന്‍) പറയുന്നതും (അല്ലാഹു അറിയും.)
\end{malayalam}}
\flushright{\begin{Arabic}
\quranayah[43][89]
\end{Arabic}}
\flushleft{\begin{malayalam}
അതിനാല്‍ നീ അവരെ വിട്ടു തിരിഞ്ഞുകളയുക. സലാം! എന്ന് പറയുകയും ചെയ്യുക. അവര്‍ വഴിയെ അറിഞ്ഞു കൊള്ളും.
\end{malayalam}}
\chapter{\textmalayalam{ദുഖാന്‍ ( പുക )}}
\begin{Arabic}
\Huge{\centerline{\basmalah}}\end{Arabic}
\flushright{\begin{Arabic}
\quranayah[44][1]
\end{Arabic}}
\flushleft{\begin{malayalam}
ഹാമീം,
\end{malayalam}}
\flushright{\begin{Arabic}
\quranayah[44][2]
\end{Arabic}}
\flushleft{\begin{malayalam}
സ്പഷ്ടമായ വേദഗ്രന്ഥം തന്നെയാണ സത്യം;
\end{malayalam}}
\flushright{\begin{Arabic}
\quranayah[44][3]
\end{Arabic}}
\flushleft{\begin{malayalam}
തീര്‍ച്ചയായും നാം അതിനെ ഒരു അനുഗൃഹീത രാത്രിയില്‍ അവതരിപ്പിച്ചിരിക്കുന്നു. തീര്‍ച്ചയായും നാം മുന്നറിയിപ്പ് നല്‍കുന്നവനാകുന്നു.
\end{malayalam}}
\flushright{\begin{Arabic}
\quranayah[44][4]
\end{Arabic}}
\flushleft{\begin{malayalam}
ആ രാത്രിയില്‍ യുക്തിപൂര്‍ണ്ണമായ ഓരോ കാര്യവും വേര്‍തിരിച്ചു വിവരിക്കപ്പെടുന്നു.
\end{malayalam}}
\flushright{\begin{Arabic}
\quranayah[44][5]
\end{Arabic}}
\flushleft{\begin{malayalam}
അതെ, നമ്മുടെ പക്കല്‍ നിന്നുള്ള കല്‍പന. തീര്‍ച്ചയായും നാം (ദൂതന്‍മാരെ) നിയോഗിച്ചു കൊണ്ടിരിക്കുന്നവനാകുന്നു.
\end{malayalam}}
\flushright{\begin{Arabic}
\quranayah[44][6]
\end{Arabic}}
\flushleft{\begin{malayalam}
നിന്‍റെ രക്ഷിതാവിങ്കല്‍ നിന്നുള്ള ഒരു കാരുണ്യമത്രെ അത്‌. തീര്‍ച്ചയായും അവന്‍ തന്നെയാകുന്നു എല്ലാം കേള്‍ക്കുന്നനും അറിയുന്നവനും.
\end{malayalam}}
\flushright{\begin{Arabic}
\quranayah[44][7]
\end{Arabic}}
\flushleft{\begin{malayalam}
ആകാശങ്ങളുടെയും ഭൂമിയുടെയും അവയ്ക്കിടയിലുള്ളതിന്‍റെയും രക്ഷിതാവ്‌. നിങ്ങള്‍ ദൃഢവിശ്വാസമുള്ളവരാണെങ്കില്‍.
\end{malayalam}}
\flushright{\begin{Arabic}
\quranayah[44][8]
\end{Arabic}}
\flushleft{\begin{malayalam}
അവനല്ലാതെ യാതൊരു ദൈവവുമില്ല. അവന്‍ ജീവിപ്പിക്കുകയും മരിപ്പിക്കുകയും ചെയ്യുന്നു. നിങ്ങളുടെ രക്ഷിതാവും നിങ്ങളുടെ പൂര്‍വ്വപിതാക്കളുടെ രക്ഷിതാവും ആയിട്ടുള്ളവന്‍.
\end{malayalam}}
\flushright{\begin{Arabic}
\quranayah[44][9]
\end{Arabic}}
\flushleft{\begin{malayalam}
എങ്കിലും അവര്‍ സംശയത്തില്‍ കളിക്കുകയാകുന്നു.
\end{malayalam}}
\flushright{\begin{Arabic}
\quranayah[44][10]
\end{Arabic}}
\flushleft{\begin{malayalam}
അതിനാല്‍ ആകാശം, തെളിഞ്ഞു കാണാവുന്ന ഒരു പുകയും കൊണ്ട് വരുന്ന ദിവസം നീ പ്രതീക്ഷിച്ചിരിക്കുക.
\end{malayalam}}
\flushright{\begin{Arabic}
\quranayah[44][11]
\end{Arabic}}
\flushleft{\begin{malayalam}
മനുഷ്യരെ അത് പൊതിയുന്നതാണ്‌. ഇത് വേദനയേറിയ ഒരു ശിക്ഷയായിരിക്കും.
\end{malayalam}}
\flushright{\begin{Arabic}
\quranayah[44][12]
\end{Arabic}}
\flushleft{\begin{malayalam}
(അവര്‍ പറയും:) ഞങ്ങളുടെ രക്ഷിതാവേ, ഞങ്ങളില്‍ നിന്ന് നീ ഈ ശിക്ഷ ഒഴിവാക്കിത്തരേണമേ, തീര്‍ച്ചയായും ഞങ്ങള്‍ വിശ്വസിച്ചു കൊള്ളാം.
\end{malayalam}}
\flushright{\begin{Arabic}
\quranayah[44][13]
\end{Arabic}}
\flushleft{\begin{malayalam}
എങ്ങനെയാണ് അവര്‍ക്ക് ഉല്‍ബോധനം ഫലപ്പെടുക? (കാര്യം) വ്യക്തമാക്കുന്ന ഒരു ദൂതന്‍ അവരുടെ അടുക്കല്‍ ചെന്നിട്ടുണ്ട്‌.
\end{malayalam}}
\flushright{\begin{Arabic}
\quranayah[44][14]
\end{Arabic}}
\flushleft{\begin{malayalam}
എന്നിട്ട് അദ്ദേഹത്തെ വിട്ട് അവന്‍ പിന്തിരിഞ്ഞു കളയുകയാണ് ചെയ്തത്‌. ആരോ പഠിപ്പിച്ചുവിട്ടവന്‍, ഭ്രാന്തന്‍ എന്നൊക്കെ അവര്‍ പറയുകയും ചെയ്തു.
\end{malayalam}}
\flushright{\begin{Arabic}
\quranayah[44][15]
\end{Arabic}}
\flushleft{\begin{malayalam}
തീര്‍ച്ചയായും നാം ശിക്ഷ അല്‍പം ഒഴിവാക്കിത്തരാം. എന്നാല്‍ നിങ്ങള്‍ (പഴയ അവസ്ഥയിലേക്ക്‌) മടങ്ങുക തന്നെ ചെയ്യുമല്ലോ.
\end{malayalam}}
\flushright{\begin{Arabic}
\quranayah[44][16]
\end{Arabic}}
\flushleft{\begin{malayalam}
ഏറ്റവും വലിയ പിടുത്തം നാം പിടിക്കുന്ന ദിവസം തീര്‍ച്ചയായും നാം ശിക്ഷാനടപടി സ്വീകരിക്കുന്നതാണ്‌.
\end{malayalam}}
\flushright{\begin{Arabic}
\quranayah[44][17]
\end{Arabic}}
\flushleft{\begin{malayalam}
ഇവര്‍ക്ക് മുമ്പ് ഫിര്‍ഔന്‍റെ ജനതയെ നാം പരീക്ഷിച്ചിട്ടുണ്ട്‌. മാന്യനായ ഒരു ദൂതന്‍ അവരുടെ അടുത്ത് ചെന്നു.
\end{malayalam}}
\flushright{\begin{Arabic}
\quranayah[44][18]
\end{Arabic}}
\flushleft{\begin{malayalam}
അല്ലാഹുവിന്‍റെ ദാസന്‍മാരെ നിങ്ങള്‍ എനിക്ക് ഏല്‍പിച്ചു തരണം. തീര്‍ച്ചയായും ഞാന്‍ നിങ്ങളിലേക്കുള്ള വിശ്വസ്തനായ ദൂതനാകുന്നു. (എന്ന് അദ്ദേഹം പറഞ്ഞു.)
\end{malayalam}}
\flushright{\begin{Arabic}
\quranayah[44][19]
\end{Arabic}}
\flushleft{\begin{malayalam}
അല്ലാഹുവിനെതിരില്‍ നിങ്ങള്‍ പൊങ്ങച്ചം കാണിക്കുകയും ചെയ്യരുത്‌. തീര്‍ച്ചയായും ഞാന്‍ സ്പഷ്ടമായ തെളിവും കൊണ്ട് നിങ്ങളുടെ അടുത്ത് വരാം.
\end{malayalam}}
\flushright{\begin{Arabic}
\quranayah[44][20]
\end{Arabic}}
\flushleft{\begin{malayalam}
നിങ്ങളെന്നെ കല്ലെറിയാതിരിക്കാന്‍ എന്‍റെ രക്ഷിതാവും നിങ്ങളുടെ രക്ഷിതാവും ആയിട്ടുള്ളവനോട് തീര്‍ച്ചയായും ഞാന്‍ ശരണം തേടിയിരിക്കുന്നു.
\end{malayalam}}
\flushright{\begin{Arabic}
\quranayah[44][21]
\end{Arabic}}
\flushleft{\begin{malayalam}
നിങ്ങള്‍ക്കെന്നെ വിശ്വാസമായില്ലെങ്കില്‍ എന്നില്‍ നിന്ന് നിങ്ങള്‍ വിട്ടുമാറുക.
\end{malayalam}}
\flushright{\begin{Arabic}
\quranayah[44][22]
\end{Arabic}}
\flushleft{\begin{malayalam}
ഇക്കൂട്ടര്‍ കുറ്റവാളികളായ ഒരു ജനവിഭാഗമാണെന്നതിനാല്‍ അദ്ദേഹം തന്‍റെ രക്ഷിതാവിനെ വിളിച്ച് (സഹായത്തിനായി) പ്രാര്‍ത്ഥിച്ചു.
\end{malayalam}}
\flushright{\begin{Arabic}
\quranayah[44][23]
\end{Arabic}}
\flushleft{\begin{malayalam}
(അപ്പോള്‍ അല്ലാഹു നിര്‍ദേശിച്ചു:) എന്‍റെ ദാസന്‍മാരെയും കൊണ്ട് നീ രാത്രിയില്‍ പ്രയാണം ചെയ്തുകൊള്ളുക. തീര്‍ച്ചയായും നിങ്ങള്‍ (ശത്രുക്കളാല്‍) പിന്തുടരപ്പെടുന്നതാണ്‌.
\end{malayalam}}
\flushright{\begin{Arabic}
\quranayah[44][24]
\end{Arabic}}
\flushleft{\begin{malayalam}
സമുദ്രത്തെ ശാന്തമായ നിലയില്‍ നീ വിട്ടേക്കുകയും ചെയ്യുക തീര്‍ച്ചയായും അവര്‍ മുക്കിനശിപ്പിക്കപ്പെടാന്‍ പോകുന്ന ഒരു സൈന്യമാകുന്നു.
\end{malayalam}}
\flushright{\begin{Arabic}
\quranayah[44][25]
\end{Arabic}}
\flushleft{\begin{malayalam}
എത്രയെത്ര തോട്ടങ്ങളും അരുവികളുമാണ് അവര്‍ വിട്ടേച്ചു പോയത്‌.!
\end{malayalam}}
\flushright{\begin{Arabic}
\quranayah[44][26]
\end{Arabic}}
\flushleft{\begin{malayalam}
(എത്രയെത്ര) കൃഷികളും മാന്യമായ പാര്‍പ്പിടങ്ങളും!
\end{malayalam}}
\flushright{\begin{Arabic}
\quranayah[44][27]
\end{Arabic}}
\flushleft{\begin{malayalam}
അവര്‍ ആഹ്ലാദപൂര്‍വ്വം അനുഭവിച്ചിരുന്ന (എത്രയെത്ര) സൌഭാഗ്യങ്ങള്‍!
\end{malayalam}}
\flushright{\begin{Arabic}
\quranayah[44][28]
\end{Arabic}}
\flushleft{\begin{malayalam}
അങ്ങനെയാണത് (കലാശിച്ചത്‌.) അതെല്ലാം മറ്റൊരു ജനതയ്ക്ക് നാം അവകാശപ്പെടുത്തി കൊടുക്കുകയും ചെയ്തു.
\end{malayalam}}
\flushright{\begin{Arabic}
\quranayah[44][29]
\end{Arabic}}
\flushleft{\begin{malayalam}
അപ്പോള്‍ അവരുടെ പേരില്‍ ആകാശവും ഭൂമിയും കരഞ്ഞില്ല. അവര്‍ക്ക് ഇടകൊടുക്കപ്പെടുകയുമുണ്ടായില്ല.
\end{malayalam}}
\flushright{\begin{Arabic}
\quranayah[44][30]
\end{Arabic}}
\flushleft{\begin{malayalam}
ഇസ്രായീല്‍ സന്തതികളെ അപമാനകരമായ ശിക്ഷയില്‍ നിന്ന് നാം രക്ഷിക്കുക തന്നെ ചെയ്തു.
\end{malayalam}}
\flushright{\begin{Arabic}
\quranayah[44][31]
\end{Arabic}}
\flushleft{\begin{malayalam}
ഫിര്‍ഔനില്‍ നിന്ന്‌. തീര്‍ച്ചയായും അവന്‍ അഹങ്കാരിയായിരുന്നു. അതിക്രമകാരികളില്‍ പെട്ടവനുമായിരുന്നു.
\end{malayalam}}
\flushright{\begin{Arabic}
\quranayah[44][32]
\end{Arabic}}
\flushleft{\begin{malayalam}
അറിഞ്ഞു കൊണ്ട് തന്നെ തീര്‍ച്ചയായും അവരെ നാം ലോകരെക്കാള്‍ ഉല്‍കൃഷ്ടരായി തെരഞ്ഞെടുക്കുകയുണ്ടായി.
\end{malayalam}}
\flushright{\begin{Arabic}
\quranayah[44][33]
\end{Arabic}}
\flushleft{\begin{malayalam}
വ്യക്തമായ പരീക്ഷണം ഉള്‍കൊള്ളുന്ന ചില ദൃഷ്ടാന്തങ്ങള്‍ നാം അവര്‍ക്ക് നല്‍കുകയുമുണ്ടായി.
\end{malayalam}}
\flushright{\begin{Arabic}
\quranayah[44][34]
\end{Arabic}}
\flushleft{\begin{malayalam}
എന്നാല്‍ ഇക്കൂട്ടരിതാ പറയുന്നു;
\end{malayalam}}
\flushright{\begin{Arabic}
\quranayah[44][35]
\end{Arabic}}
\flushleft{\begin{malayalam}
നമ്മുടെ ഒന്നാമത്തെ മരണമല്ലാതെ മറ്റൊന്നുമില്ല. നാം ഉയിര്‍ത്തെഴുന്നേല്‍പിക്കപ്പെടുന്നവരുമല്ല.
\end{malayalam}}
\flushright{\begin{Arabic}
\quranayah[44][36]
\end{Arabic}}
\flushleft{\begin{malayalam}
അതിനാല്‍ നിങ്ങള്‍ സത്യവാന്‍മാരാണെങ്കില്‍ ഞങ്ങളുടെ പിതാക്കളെ നിങ്ങള്‍ (ജീവിപ്പിച്ചു) കൊണ്ട് വരിക എന്ന്‌.
\end{malayalam}}
\flushright{\begin{Arabic}
\quranayah[44][37]
\end{Arabic}}
\flushleft{\begin{malayalam}
ഇവരാണോ കൂടുതല്‍ മെച്ചപ്പെട്ടവര്‍, അതല്ല തുബ്ബഇന്‍റെ ജനതയും അവര്‍ക്കു മുമ്പുള്ളവരുമാണോ? അവരെയെല്ലാം നാം നശിപ്പിക്കുകയുണ്ടായി. കാരണം അവര്‍ കുറ്റവാളികളായിരുന്നത് തന്നെ.
\end{malayalam}}
\flushright{\begin{Arabic}
\quranayah[44][38]
\end{Arabic}}
\flushleft{\begin{malayalam}
ആകാശങ്ങളും ഭൂമിയും അവയ്ക്കിടയിലുള്ളതും നാം കളിയായിക്കൊണ്ട് സൃഷ്ടിച്ചതല്ല.
\end{malayalam}}
\flushright{\begin{Arabic}
\quranayah[44][39]
\end{Arabic}}
\flushleft{\begin{malayalam}
ശരിയായ ഉദ്ദേശത്തോടു കൂടി തന്നെയാണ് നാം അവയെ സൃഷ്ടിച്ചത്‌. പക്ഷെ അവരില്‍ അധികപേരും അറിയുന്നില്ല.
\end{malayalam}}
\flushright{\begin{Arabic}
\quranayah[44][40]
\end{Arabic}}
\flushleft{\begin{malayalam}
തീര്‍ച്ചയായും ആ നിര്‍ണായക തീരുമാനത്തിന്‍റെ ദിവസമാകുന്നു അവര്‍ക്കെല്ലാമുള്ള നിശ്ചിത സമയം.
\end{malayalam}}
\flushright{\begin{Arabic}
\quranayah[44][41]
\end{Arabic}}
\flushleft{\begin{malayalam}
അതെ, ഒരു ബന്ധു മറ്റൊരു ബന്ധുവിന് യാതൊരു പ്രയോജനവും ചെയ്യാത്ത, അവര്‍ക്ക് ഒരു സഹായവും ലഭിക്കാത്ത ഒരു ദിവസം.
\end{malayalam}}
\flushright{\begin{Arabic}
\quranayah[44][42]
\end{Arabic}}
\flushleft{\begin{malayalam}
അല്ലാഹു ആരോട് കരുണ കാണിച്ചുവോ അവര്‍ക്കൊഴികെ. തീര്‍ച്ചയായും അവന്‍ തന്നെയാകുന്നു പ്രതാപിയും കരുണാനിധിയും.
\end{malayalam}}
\flushright{\begin{Arabic}
\quranayah[44][43]
\end{Arabic}}
\flushleft{\begin{malayalam}
തീര്‍ച്ചയായും സഖ്ഖൂം വൃക്ഷമാകുന്നു.
\end{malayalam}}
\flushright{\begin{Arabic}
\quranayah[44][44]
\end{Arabic}}
\flushleft{\begin{malayalam}
(നരകത്തില്‍) പാപിയുടെ ആഹാരം.
\end{malayalam}}
\flushright{\begin{Arabic}
\quranayah[44][45]
\end{Arabic}}
\flushleft{\begin{malayalam}
ഉരുകിയ ലോഹം പോലിരിക്കും (അതിന്‍റെ കനി.) അത് വയറുകളില്‍ തിളയ്ക്കും.
\end{malayalam}}
\flushright{\begin{Arabic}
\quranayah[44][46]
\end{Arabic}}
\flushleft{\begin{malayalam}
ചുടുവെള്ളം തിളയ്ക്കുന്നത് പോലെ
\end{malayalam}}
\flushright{\begin{Arabic}
\quranayah[44][47]
\end{Arabic}}
\flushleft{\begin{malayalam}
നിങ്ങള്‍ അവനെ പിടിക്കൂ. എന്നിട്ട് നരകത്തിന്‍റെ മദ്ധ്യത്തിലേക്ക് വലിച്ചിഴക്കൂ.
\end{malayalam}}
\flushright{\begin{Arabic}
\quranayah[44][48]
\end{Arabic}}
\flushleft{\begin{malayalam}
അനന്തരം ചുടുവെള്ളം കൊണ്ടുള്ള ശിക്ഷ അവന്‍റെ തലയ്ക്കുമീതെ നിങ്ങള്‍ ചൊരിഞ്ഞേക്കൂ. (എന്ന് നിര്‍ദേശിക്കപ്പെടും.)
\end{malayalam}}
\flushright{\begin{Arabic}
\quranayah[44][49]
\end{Arabic}}
\flushleft{\begin{malayalam}
ഇത് ആസ്വദിച്ചോളൂ. തീര്‍ച്ചയായും നീ തന്നെയായിരുന്നല്ലോ പ്രതാപിയും മാന്യനും.
\end{malayalam}}
\flushright{\begin{Arabic}
\quranayah[44][50]
\end{Arabic}}
\flushleft{\begin{malayalam}
നിങ്ങള്‍ ഏതൊരു കാര്യത്തില്‍ സംശയാലുക്കളായിരുന്നുവോ ആ കാര്യമത്രെ ഇത്‌.
\end{malayalam}}
\flushright{\begin{Arabic}
\quranayah[44][51]
\end{Arabic}}
\flushleft{\begin{malayalam}
സൂക്ഷ്മത പാലിച്ചവര്‍ തീര്‍ച്ചയായും നിര്‍ഭയമായ വാസസ്ഥലത്താകുന്നു.
\end{malayalam}}
\flushright{\begin{Arabic}
\quranayah[44][52]
\end{Arabic}}
\flushleft{\begin{malayalam}
തോട്ടങ്ങള്‍ക്കും അരുവികള്‍ക്കുമിടയില്‍
\end{malayalam}}
\flushright{\begin{Arabic}
\quranayah[44][53]
\end{Arabic}}
\flushleft{\begin{malayalam}
നേര്‍ത്ത പട്ടുതുണിയും കട്ടിയുള്ള പട്ടുതുണിയും അവര്‍ ധരിക്കും. അവര്‍ അന്യോന്യം അഭിമുഖമായിട്ടായിരിക്കും ഇരിക്കുന്നത്‌.
\end{malayalam}}
\flushright{\begin{Arabic}
\quranayah[44][54]
\end{Arabic}}
\flushleft{\begin{malayalam}
അങ്ങനെയാകുന്നു (അവരുടെ അവസ്ഥ.) വിശാലമായ നേത്രങ്ങളുള്ള വെളുത്ത സ്ത്രീകളെ അവര്‍ക്ക് ഇണകളായി നല്‍കുകയും ചെയ്യും.
\end{malayalam}}
\flushright{\begin{Arabic}
\quranayah[44][55]
\end{Arabic}}
\flushleft{\begin{malayalam}
സുരക്ഷിതത്വ ബോധത്തോട് കൂടി എല്ലാവിധ പഴങ്ങളും അവര്‍ അവിടെ വെച്ച് ആവശ്യപ്പെട്ടുകൊണ്ടിരിക്കും.
\end{malayalam}}
\flushright{\begin{Arabic}
\quranayah[44][56]
\end{Arabic}}
\flushleft{\begin{malayalam}
ആദ്യത്തെ മരണമല്ലാതെ മറ്റൊരു മരണം അവര്‍ക്കവിടെ അനുഭവിക്കേണ്ടതില്ല. നരകശിക്ഷയില്‍ നിന്ന് അല്ലാഹു അവരെ കാത്തുരക്ഷിക്കുകയും ചെയ്തിരിക്കുന്നു.
\end{malayalam}}
\flushright{\begin{Arabic}
\quranayah[44][57]
\end{Arabic}}
\flushleft{\begin{malayalam}
നിന്‍റെ രക്ഷിതാവിങ്കല്‍ നിന്നുള്ള ഔദാര്യമത്രെ അത്‌. അത് തന്നെയാണ് മഹത്തായ ഭാഗ്യം.
\end{malayalam}}
\flushright{\begin{Arabic}
\quranayah[44][58]
\end{Arabic}}
\flushleft{\begin{malayalam}
നിനക്ക് നിന്‍റെ ഭാഷയില്‍ ഇതിനെ (ഖുര്‍ആനിനെ) നാം ലളിതമാക്കിത്തന്നിട്ടുള്ളത് അവര്‍ ആലോചിച്ചു മനസ്സിലാക്കാന്‍ വേണ്ടി മാത്രമാകുന്നു.
\end{malayalam}}
\flushright{\begin{Arabic}
\quranayah[44][59]
\end{Arabic}}
\flushleft{\begin{malayalam}
ആകയാല്‍ നീ കാത്തിരിക്കുക. അവരും കാത്തിരിക്കുന്നവര്‍ തന്നെയാകുന്നു.
\end{malayalam}}
\chapter{\textmalayalam{ജാഥിയ ( മുട്ടുകുത്തുന്നവര്‍ )}}
\begin{Arabic}
\Huge{\centerline{\basmalah}}\end{Arabic}
\flushright{\begin{Arabic}
\quranayah[45][1]
\end{Arabic}}
\flushleft{\begin{malayalam}
ഹാമീം.
\end{malayalam}}
\flushright{\begin{Arabic}
\quranayah[45][2]
\end{Arabic}}
\flushleft{\begin{malayalam}
ഈ വേദഗ്രന്ഥത്തിന്‍റെ അവതരണം പ്രതാപിയും യുക്തിമാനുമായ അല്ലാഹുവിങ്കല്‍ നിന്നാകുന്നു.
\end{malayalam}}
\flushright{\begin{Arabic}
\quranayah[45][3]
\end{Arabic}}
\flushleft{\begin{malayalam}
തീര്‍ച്ചയായും ആകാശങ്ങളിലും ഭൂമിയിലും വിശ്വാസികള്‍ക്ക് പല ദൃഷ്ടാന്തങ്ങളുമുണ്ട്‌.
\end{malayalam}}
\flushright{\begin{Arabic}
\quranayah[45][4]
\end{Arabic}}
\flushleft{\begin{malayalam}
നിങ്ങളുടെ സൃഷ്ടിപ്പിലും ജന്തുജാലങ്ങളെ അവന്‍ വിന്യസിക്കുന്നതിലുമുണ്ട് ദൃഢമായി വിശ്വസിക്കുന്ന ജനങ്ങള്‍ക്ക് പല ദൃഷ്ടാന്തങ്ങളും.
\end{malayalam}}
\flushright{\begin{Arabic}
\quranayah[45][5]
\end{Arabic}}
\flushleft{\begin{malayalam}
രാവും പകലും മാറിമാറി വരുന്നതിലും, അല്ലാഹു ആകാശത്തു നിന്ന് ഉപജീവനം ഇറക്കി അതുമുഖേന ഭൂമിക്ക് അതിന്‍റെ നിര്‍ജീവാവസ്ഥയ്ക്ക് ശേഷം ജീവന്‍ നല്‍കിയതിലും, കാറ്റുകളുടെ ഗതി നിയന്ത്രണത്തിലും ചിന്തിച്ചു മനസ്സിലാക്കുന്ന ആളുകള്‍ക്ക് പല ദൃഷ്ടാന്തങ്ങളുമുണ്ട്‌.
\end{malayalam}}
\flushright{\begin{Arabic}
\quranayah[45][6]
\end{Arabic}}
\flushleft{\begin{malayalam}
അല്ലാഹുവിന്‍റെ തെളിവുകളത്രെ അവ. സത്യപ്രകാരം നാം നിനക്ക് അവ ഓതികേള്‍പിക്കുന്നു. അല്ലാഹുവിനും അവന്‍റെ തെളിവുകള്‍ക്കും പുറമെ ഇനി ഏതൊരു വൃത്താന്തത്തിലാണ് അവര്‍ വിശ്വസിക്കുന്നത്‌?
\end{malayalam}}
\flushright{\begin{Arabic}
\quranayah[45][7]
\end{Arabic}}
\flushleft{\begin{malayalam}
വ്യാജവാദിയും അധര്‍മകാരിയുമായ ഏതൊരാള്‍ക്കും നാശം.
\end{malayalam}}
\flushright{\begin{Arabic}
\quranayah[45][8]
\end{Arabic}}
\flushleft{\begin{malayalam}
അല്ലാഹുവിന്‍റെ ദൃഷ്ടാന്തങ്ങള്‍ തനിക്ക് ഓതികേള്‍പിക്കപ്പെടുന്നത് അവന്‍ കേള്‍ക്കുകയും എന്നിട്ട് അത് കേട്ടിട്ടില്ലാത്തത് പോലെ അഹങ്കാരിയായിക്കൊണ്ട് ശഠിച്ചു നില്‍ക്കുകയും ചെയ്യുന്നു. ആകയാല്‍ അവന്ന് വേദനയേറിയ ശിക്ഷയെ പറ്റി സന്തോഷവാര്‍ത്ത അറിയിച്ചു കൊള്ളുക.
\end{malayalam}}
\flushright{\begin{Arabic}
\quranayah[45][9]
\end{Arabic}}
\flushleft{\begin{malayalam}
നമ്മുടെ തെളിവുകളില്‍ നിന്ന് വല്ലതും അവന്‍ അറിഞ്ഞാലോ അവനത് ഒരു പരിഹാസവിഷയമാക്കിക്കളയുകയും ചെയ്യും. അത്തരക്കാര്‍ക്കാകുന്നു അപമാനകരമായ ശിക്ഷ.
\end{malayalam}}
\flushright{\begin{Arabic}
\quranayah[45][10]
\end{Arabic}}
\flushleft{\begin{malayalam}
അവരുടെ പുറകെ നരകമുണ്ട്‌. അവര്‍ സമ്പാദിച്ചു വെച്ചിട്ടുള്ളതോ, അല്ലാഹുവിനു പുറമെ അവര്‍ സ്വീകരിച്ചിട്ടുള്ള രക്ഷാധികാരികളോ അവര്‍ക്ക് ഒട്ടും പ്രയോജനം ചെയ്യുകയില്ല. അവര്‍ക്കാണ് കനത്ത ശിക്ഷയുള്ളത്‌.
\end{malayalam}}
\flushright{\begin{Arabic}
\quranayah[45][11]
\end{Arabic}}
\flushleft{\begin{malayalam}
ഇത് ഒരു മാര്‍ഗദര്‍ശനമാകുന്നു. തങ്ങളുടെ രക്ഷിതാവിന്‍റെ ദൃഷ്ടാന്തങ്ങളില്‍ അവിശ്വസിച്ചവരാരോ അവര്‍ക്ക് കഠിനമായ തരത്തിലുള്ള വേദനയേറിയ ശിക്ഷയുണ്ട്‌.
\end{malayalam}}
\flushright{\begin{Arabic}
\quranayah[45][12]
\end{Arabic}}
\flushleft{\begin{malayalam}
അല്ലാഹുവാകുന്നു സമുദ്രത്തെ നിങ്ങള്‍ക്ക് അധീനമാക്കി തന്നവന്‍. അവന്‍റെ കല്‍പന പ്രകാരം അതിലൂടെ കപ്പലുകള്‍ സഞ്ചരിക്കുവാനും, അവന്‍റെ അനുഗ്രഹത്തില്‍നിന്ന് നിങ്ങള്‍ തേടുവാനും, നിങ്ങള്‍ നന്ദികാണിക്കുന്നവരായേക്കാനും വേണ്ടി.
\end{malayalam}}
\flushright{\begin{Arabic}
\quranayah[45][13]
\end{Arabic}}
\flushleft{\begin{malayalam}
ആകാശങ്ങളിലുള്ളതും ഭൂമിയിലുള്ളതുമെല്ലാം തന്‍റെ വകയായി അവന്‍ നിങ്ങള്‍ക്ക് അധീനപ്പെടുത്തിത്തരികയും ചെയ്തിരിക്കുന്നു. ചിന്തിക്കുന്ന ജനങ്ങള്‍ക്ക് തീര്‍ച്ചയായും അതില്‍ പല ദൃഷ്ടാന്തങ്ങളുമുണ്ട്‌.
\end{malayalam}}
\flushright{\begin{Arabic}
\quranayah[45][14]
\end{Arabic}}
\flushleft{\begin{malayalam}
(നബിയേ,) നീ സത്യവിശ്വാസികളോട് പറയുക: അല്ലാഹുവിന്‍റെ (ശിക്ഷയുടെ) നാളുകള്‍ പ്രതീക്ഷിക്കാത്ത സത്യനിഷേധികള്‍ക്ക് അവര്‍ മാപ്പുചെയ്ത് കൊടുക്കണമെന്ന്‌. ഓരോ ജനതയ്ക്കും അവര്‍ സമ്പാദിച്ച് കൊണ്ടിരിക്കുന്നതിന്‍റെ ഫലം അല്ലാഹു നല്‍കുവാന്‍ വേണ്ടിയത്രെ അത്‌.
\end{malayalam}}
\flushright{\begin{Arabic}
\quranayah[45][15]
\end{Arabic}}
\flushleft{\begin{malayalam}
വല്ലവനും നല്ലത് പ്രവര്‍ത്തിച്ചാല്‍ അത് അവന്‍റെ ഗുണത്തിന് തന്നെയാകുന്നു. വല്ലവനും തിന്‍മ പ്രവര്‍ത്തിച്ചാല്‍ അതിന്‍റെ ദോഷവും അവന് തന്നെ. പിന്നീട് നിങ്ങളുടെ രക്ഷിതാവിങ്കലേക്ക് നിങ്ങള്‍ മടക്കപ്പെടുന്നതാണ്‌.
\end{malayalam}}
\flushright{\begin{Arabic}
\quranayah[45][16]
\end{Arabic}}
\flushleft{\begin{malayalam}
ഇസ്രായീല്‍ സന്തതികള്‍ക്ക് വേദഗ്രന്ഥവും വിജ്ഞാനവും പ്രവാചകത്വവും നാം നല്‍കുകയുണ്ടായി. വിശിഷ്ട വസ്തുക്കളില്‍ നിന്ന് അവര്‍ക്ക് ആഹാരം നല്‍കുകയും ലോകരെക്കാള്‍ അവര്‍ക്ക് നാം ശ്രേഷ്ഠത നല്‍കുകയും ചെയ്തു.
\end{malayalam}}
\flushright{\begin{Arabic}
\quranayah[45][17]
\end{Arabic}}
\flushleft{\begin{malayalam}
അവര്‍ക്ക് നാം (മത) കാര്യത്തെപ്പറ്റിയുള്ള വ്യക്തമായ തെളിവുകള്‍ നല്‍കുകയും ചെയ്തു. എന്നാല്‍ അവര്‍ ഭിന്നിച്ചത് അവര്‍ക്കു അറിവ് വന്നുകിട്ടിയതിന് ശേഷം തന്നെയാണ്‌. അവര്‍ തമ്മിലുള്ള മാത്സര്യം നിമിത്തമാണത്‌. ഏതൊരു കാര്യത്തില്‍ അവര്‍ ഭിന്നിച്ച് കൊണ്ടിരിക്കുന്നുവോ അതില്‍ ഉയിര്‍ത്തെഴുന്നേല്‍പിന്‍റെ നാളില്‍ അവര്‍ക്കിടയില്‍ നിന്‍റെ രക്ഷിതാവ് വിധികല്‍പിക്കുക തന്നെ ചെയ്യും.
\end{malayalam}}
\flushright{\begin{Arabic}
\quranayah[45][18]
\end{Arabic}}
\flushleft{\begin{malayalam}
(നബിയേ,) പിന്നീട് നിന്നെ നാം (മത) കാര്യത്തില്‍ ഒരു തെളിഞ്ഞ മാര്‍ഗത്തിലാക്കിയിരിക്കുന്നു. ആകയാല്‍ നീ അതിനെ പിന്തുടരുക. അറിവില്ലാത്തവരുടെ തന്നിഷ്ടങ്ങളെ നീ പിന്‍പറ്റരുത്‌.
\end{malayalam}}
\flushright{\begin{Arabic}
\quranayah[45][19]
\end{Arabic}}
\flushleft{\begin{malayalam}
അല്ലാഹുവിങ്കല്‍ നിന്നുള്ള യാതൊരു കാര്യത്തിനും അവര്‍ നിനക്ക് ഒട്ടും പ്രയോജനപ്പെടുകയേയില്ല. തീര്‍ച്ചയായും അക്രമകാരികളില്‍ ചിലര്‍ ചിലര്‍ക്ക് രക്ഷാകര്‍ത്താക്കളാകുന്നു. അല്ലാഹു സൂക്ഷ്മത പാലിക്കുന്നവരുടെ രക്ഷാകര്‍ത്താവാകുന്നു.
\end{malayalam}}
\flushright{\begin{Arabic}
\quranayah[45][20]
\end{Arabic}}
\flushleft{\begin{malayalam}
ഇത് മനുഷ്യരുടെ കണ്ണ് തുറപ്പിക്കുന്ന തെളിവുകളും ദൃഢമായി വിശ്വസിക്കുന്ന ജനങ്ങള്‍ക്ക് മാര്‍ഗദര്‍ശനവും കാരുണ്യവുമാകുന്നു.
\end{malayalam}}
\flushright{\begin{Arabic}
\quranayah[45][21]
\end{Arabic}}
\flushleft{\begin{malayalam}
അതല്ല, തിന്‍മകള്‍ പ്രവര്‍ത്തിച്ചവര്‍ വിചാരിച്ചിരിക്കുകയാണോ; അവരെ നാം വിശ്വസിക്കുകയും സല്‍കര്‍മ്മങ്ങള്‍ പ്രവര്‍ത്തിക്കുകയും ചെയ്തവരെപ്പോലെ, അതായത് അവരുടെ (രണ്ട് കൂട്ടരുടെയും) ജീവിതവും മരണവും തുല്യമായ നിലയില്‍ ആക്കുമെന്ന്‌? അവര്‍ വിധികല്‍പിക്കുന്നത് വളരെ മോശം തന്നെ.
\end{malayalam}}
\flushright{\begin{Arabic}
\quranayah[45][22]
\end{Arabic}}
\flushleft{\begin{malayalam}
ആകാശങ്ങളും ഭൂമിയും അല്ലാഹു ശരിയായ ലക്ഷ്യത്തോടെ സൃഷ്ടിച്ചിരിക്കുന്നു. ഓരോ ആള്‍ക്കും താന്‍ പ്രവര്‍ത്തിച്ചതിനുള്ള പ്രതിഫലം നല്‍കപ്പെടാന്‍ വേണ്ടിയുമാണ് അത്‌. അവരോട് അനീതി കാണിക്കപ്പെടുന്നതല്ല.
\end{malayalam}}
\flushright{\begin{Arabic}
\quranayah[45][23]
\end{Arabic}}
\flushleft{\begin{malayalam}
എന്നാല്‍ തന്‍റെ ദൈവത്തെ തന്‍റെ തന്നിഷ്ടമാക്കിയവനെ നീ കണ്ടുവോ? അറിഞ്ഞ് കൊണ്ട് തന്നെ അല്ലാഹു അവനെ പിഴവിലാക്കുകയും, അവന്‍റെ കാതിനും ഹൃദയത്തിനും മുദ്രവെക്കുകയും, അവന്‍റെ കണ്ണിന് മേല്‍ ഒരു മൂടി ഉണ്ടാക്കുകയും ചെയ്തിരിക്കുന്നു. അപ്പോള്‍ അല്ലാഹുവിന് പുറമെ ആരാണ് അവനെ നേര്‍വഴിയിലാക്കുവാനുള്ളത്‌? എന്നിരിക്കെ നിങ്ങള്‍ ആലോചിച്ചു മനസ്സിലാക്കുന്നില്ലേ?
\end{malayalam}}
\flushright{\begin{Arabic}
\quranayah[45][24]
\end{Arabic}}
\flushleft{\begin{malayalam}
അവര്‍ പറഞ്ഞു: ജീവിതമെന്നാല്‍ നമ്മുടെ ഐഹികജീവിതം മാത്രമാകുന്നു. നാം മരിക്കുന്നു. നാം ജീവിക്കുന്നു. നമ്മെ നശിപ്പിക്കുന്നത് കാലം മാത്രമാകുന്നു. (വാസ്തവത്തില്‍) അവര്‍ക്ക് അതിനെപ്പറ്റി യാതൊരു അറിവുമില്ല. അവര്‍ ഊഹിക്കുക മാത്രമാകുന്നു.
\end{malayalam}}
\flushright{\begin{Arabic}
\quranayah[45][25]
\end{Arabic}}
\flushleft{\begin{malayalam}
നമ്മുടെ ദൃഷ്ടാന്തങ്ങള്‍ വ്യക്തമായി അവര്‍ക്ക് വായിച്ചുകേള്‍പിക്കപ്പെടുകയാണെങ്കില്‍ അവരുടെ ന്യായവാദം നിങ്ങള്‍ സത്യവാന്‍മാരാണെങ്കില്‍ ഞങ്ങളുടെ പിതാക്കളെ (ജീവിപ്പിച്ചു) കൊണ്ട് വരിക. എന്ന അവരുടെ വാക്ക് മാത്രമായിരിക്കും.
\end{malayalam}}
\flushright{\begin{Arabic}
\quranayah[45][26]
\end{Arabic}}
\flushleft{\begin{malayalam}
പറയുക: അല്ലാഹു നിങ്ങളെ ജീവിപ്പിക്കുന്നു. പിന്നീട് അവന്‍ നിങ്ങളെ മരിപ്പിക്കുകയും പിന്നീട് ഉയിര്‍ത്തെഴുന്നേല്‍പിന്‍റെ നാളിലേക്ക് നിങ്ങളെ അവന്‍ ഒരുമിച്ചുകൂട്ടുകയും ചെയ്യും. അതില്‍ യാതൊരു സംശയവുമില്ല. പക്ഷെ മനുഷ്യരില്‍ അധികപേരും അറിയുന്നില്ല.
\end{malayalam}}
\flushright{\begin{Arabic}
\quranayah[45][27]
\end{Arabic}}
\flushleft{\begin{malayalam}
അല്ലാഹുവിന്നാകുന്നു ആകാശങ്ങളുടെയും ഭൂമിയുടെയും ആധിപത്യം. ആ അന്ത്യസമയം നിലവില്‍ വരുന്ന ദിവസമുണ്ടല്ലോ അന്നായിരിക്കും അസത്യവാദികള്‍ക്കു നഷ്ടം നേരിടുന്ന ദിവസം.
\end{malayalam}}
\flushright{\begin{Arabic}
\quranayah[45][28]
\end{Arabic}}
\flushleft{\begin{malayalam}
(അന്ന്‌) എല്ലാ സമുദായങ്ങളെയും മുട്ടുകുത്തിയ നിലയില്‍ നീ കാണുന്നതാണ്‌. ഓരോ സമുദായവും അതിന്‍റെ രേഖയിലേക്ക് വിളിക്കപ്പെടും. നിങ്ങള്‍ പ്രവര്‍ത്തിച്ചു കൊണ്ടിരുന്നതിന് ഇന്ന് നിങ്ങള്‍ക്ക് പ്രതിഫലം നല്‍കപ്പെടുന്നതാണ്‌. (എന്ന് അവരോട് പറയപ്പെടുകയും ചെയ്യും.)
\end{malayalam}}
\flushright{\begin{Arabic}
\quranayah[45][29]
\end{Arabic}}
\flushleft{\begin{malayalam}
ഇതാ നമ്മുടെ രേഖ. നിങ്ങള്‍ക്കെതിരായി അത് സത്യം തുറന്നുപറയുന്നതാണ്‌. തീര്‍ച്ചയായും നിങ്ങള്‍ പ്രവര്‍ത്തിച്ചു കൊണ്ടിരിക്കുന്നതെല്ലാം നാം എഴുതിക്കുന്നുണ്ടായിരുന്നു.
\end{malayalam}}
\flushright{\begin{Arabic}
\quranayah[45][30]
\end{Arabic}}
\flushleft{\begin{malayalam}
എന്നാല്‍ വിശ്വസിക്കുകയും സല്‍കര്‍മ്മം പ്രവര്‍ത്തിക്കുകയും ചെയ്തവരാരോ അവരെ അവരുടെ രക്ഷിതാവ് തന്‍റെ കാരുണ്യത്തില്‍ പ്രവേശിപ്പിക്കുന്നതാണ്‌. അതു തന്നെയാകുന്നു വ്യക്തമായ ഭാഗ്യം.
\end{malayalam}}
\flushright{\begin{Arabic}
\quranayah[45][31]
\end{Arabic}}
\flushleft{\begin{malayalam}
എന്നാല്‍ അവിശ്വസിച്ചവരാരോ (അവരോട് പറയപ്പെടും:) എന്‍റെ ദൃഷ്ടാന്തങ്ങള്‍ നിങ്ങള്‍ക്ക് ഓതികേള്‍പിക്കപ്പെട്ടിരുന്നില്ലേ? എന്നിട്ട് നിങ്ങള്‍ അഹങ്കരിക്കുകയും കുറ്റവാളികളായ ഒരു ജനതയാകുകയും ചെയ്തു.
\end{malayalam}}
\flushright{\begin{Arabic}
\quranayah[45][32]
\end{Arabic}}
\flushleft{\begin{malayalam}
തീര്‍ച്ചയായും അല്ലാഹുവിന്‍റെ വാഗ്ദാനം സത്യമാണ്‌. ആ അന്ത്യസമയമാകട്ടെ അതിന്‍റെ കാര്യത്തില്‍ യാതൊരു സംശയവുമില്ല എന്ന് പറയപ്പെട്ടാല്‍ നിങ്ങള്‍ പറയും: എന്താണ് അന്ത്യസമയമെന്ന് ഞങ്ങള്‍ക്കറിഞ്ഞ് കൂടാ. ഞങ്ങള്‍ക്ക് ഒരു തരം ഊഹം മാത്രമാണുള്ളത്‌. ഞങ്ങള്‍ക്ക് ഒരു ഉറപ്പുമില്ല.
\end{malayalam}}
\flushright{\begin{Arabic}
\quranayah[45][33]
\end{Arabic}}
\flushleft{\begin{malayalam}
തങ്ങള്‍ പ്രവര്‍ത്തിച്ചതിന്‍റെ ദൂഷ്യങ്ങള്‍ അവര്‍ക്കു വെളിപ്പെടുന്നതാണ്‌. അവര്‍ എന്തിനെയാണോ പരിഹസിച്ചു കൊണ്ടിരുന്നത് അത് അവരെ വലയം ചെയ്യുന്നതുമാണ്‌.
\end{malayalam}}
\flushright{\begin{Arabic}
\quranayah[45][34]
\end{Arabic}}
\flushleft{\begin{malayalam}
(അവരോട്‌) പറയപ്പെടും: നിങ്ങളുടെ ഈ ദിവസത്തെ കണ്ടുമുട്ടുന്നത് നിങ്ങള്‍ മറന്നത് പോലെ ഇന്ന് നിങ്ങളെ നാം മറന്നുകളയുന്നു. നിങ്ങളുടെ വാസസ്ഥലം നരകമാകുന്നു. നിങ്ങള്‍ക്ക് സഹായികളാരും ഇല്ലതാനും.
\end{malayalam}}
\flushright{\begin{Arabic}
\quranayah[45][35]
\end{Arabic}}
\flushleft{\begin{malayalam}
അല്ലാഹുവിന്‍റെ ദൃഷ്ടാന്തങ്ങളെ നിങ്ങള്‍ പരിഹാസ്യമാക്കിത്തീര്‍ക്കുകയും ഐഹികജീവിതം നിങ്ങളെ വഞ്ചിക്കുകയും ചെയ്തത് കൊണ്ടാണ് അങ്ങനെ സംഭവിച്ചത്‌. ആകയാല്‍ ഇന്ന് അവര്‍ അവിടെ നിന്ന് പുറത്തയക്കപ്പെടുന്നതല്ല. അവരോട് പ്രായശ്ചിത്തം ആവശ്യപ്പെടുകയുമില്ല.
\end{malayalam}}
\flushright{\begin{Arabic}
\quranayah[45][36]
\end{Arabic}}
\flushleft{\begin{malayalam}
അപ്പോള്‍ ആകാശങ്ങളുടെ രക്ഷിതാവും ഭൂമിയുടെ രക്ഷിതാവും ലോകരുടെ രക്ഷിതാവുമായ അല്ലാഹുവിനാണ് സ്തുതി.
\end{malayalam}}
\flushright{\begin{Arabic}
\quranayah[45][37]
\end{Arabic}}
\flushleft{\begin{malayalam}
ആകാശങ്ങളിലും ഭൂമിയിലും അവന്നു തന്നെയാകുന്നു മഹത്വം. അവന്‍ തന്നെയാകുന്നു പ്രതാപിയും യുക്തിമാനും.
\end{malayalam}}
\chapter{\textmalayalam{അഹ്ഖാഫ്}}
\begin{Arabic}
\Huge{\centerline{\basmalah}}\end{Arabic}
\flushright{\begin{Arabic}
\quranayah[46][1]
\end{Arabic}}
\flushleft{\begin{malayalam}
ഹാമീം
\end{malayalam}}
\flushright{\begin{Arabic}
\quranayah[46][2]
\end{Arabic}}
\flushleft{\begin{malayalam}
ഈ വേദഗ്രന്ഥത്തിന്‍റെ അവതരണം പ്രതാപിയും യുക്തിമാനുമായ അല്ലാഹുവിങ്കല്‍ നിന്നാകുന്നു.
\end{malayalam}}
\flushright{\begin{Arabic}
\quranayah[46][3]
\end{Arabic}}
\flushleft{\begin{malayalam}
ആകാശങ്ങളും ഭൂമിയും അവയ്ക്കിടയിലുള്ളതും നാം സൃഷ്ടിച്ചത് ശരിയായ ഉദ്ദേശത്തോടു കൂടിയും നിര്‍ണിതമായ ഒരു അവധിവെച്ചുകൊണ്ടും മാത്രമാകുന്നു. സത്യനിഷേധികളാകട്ടെ തങ്ങള്‍ക്ക് താക്കീത് നല്‍കപ്പെട്ടതു ശ്രദ്ധിക്കാതെ തിരിഞ്ഞുകളയുന്നവരാകുന്നു.
\end{malayalam}}
\flushright{\begin{Arabic}
\quranayah[46][4]
\end{Arabic}}
\flushleft{\begin{malayalam}
(നബിയേ,) പറയുക: അല്ലാഹുവിന് പുറമെ നിങ്ങള്‍ വിളിച്ചു പ്രാര്‍ത്ഥിക്കുന്നതിനെ പറ്റി നിങ്ങള്‍ ചിന്തിച്ചു നോക്കിയിട്ടുണ്ടോ? ഭൂമിയില്‍ അവര്‍ എന്താണ് സൃഷ്ടിച്ചിട്ടുള്ളതെന്ന് നിങ്ങള്‍ എനിക്ക് കാണിച്ചുതരൂ. അതല്ല ആകാശങ്ങളുടെ സൃഷ്ടിയില്‍ വല്ല പങ്കും അവര്‍ക്കുണ്ടോ? നിങ്ങള്‍ സത്യവാന്‍മാരാണെങ്കില്‍ ഇതിന് മുമ്പുള്ള ഏതെങ്കിലും വേദഗ്രന്ഥമോ, അറിവിന്‍റെ വല്ല അംശമോ നിങ്ങള്‍ എനിക്ക് കൊണ്ടു വന്നു തരൂ.
\end{malayalam}}
\flushright{\begin{Arabic}
\quranayah[46][5]
\end{Arabic}}
\flushleft{\begin{malayalam}
അല്ലാഹുവിനു പുറമെ, ഉയിര്‍ത്തെഴുന്നേല്‍പിന്‍റെ നാളുവരെയും തനിക്ക് ഉത്തരം നല്‍കാത്തവരെ വിളിച്ചു പ്രാര്‍ത്ഥിക്കുന്നവനെക്കാള്‍ വഴിപിഴച്ചവന്‍ ആരുണ്ട്‌? അവരാകട്ടെ ഇവരുടെ പ്രാര്‍ത്ഥനയെപ്പറ്റി ബോധമില്ലാത്തവരാകുന്നു.
\end{malayalam}}
\flushright{\begin{Arabic}
\quranayah[46][6]
\end{Arabic}}
\flushleft{\begin{malayalam}
മനുഷ്യരെല്ലാം ഒരുമിച്ചുകൂട്ടപ്പെടുന്ന സന്ദര്‍ഭത്തില്‍ അവര്‍ ഇവരുടെ ശത്രുക്കളായിരിക്കുകയും ചെയ്യും. ഇവര്‍ അവരെ ആരാധിച്ചിരുന്നതിനെ അവര്‍ നിഷേധിക്കുന്നവരായിത്തീരുകയും ചെയ്യും.
\end{malayalam}}
\flushright{\begin{Arabic}
\quranayah[46][7]
\end{Arabic}}
\flushleft{\begin{malayalam}
സുവ്യക്തമായ നിലയില്‍ നമ്മുടെ ദൃഷ്ടാന്തങ്ങള്‍ അവര്‍ക്ക് ഓതികേള്‍പിക്കപ്പെടുകയാണെങ്കില്‍ സത്യം തങ്ങള്‍ക്ക് വന്നെത്തുമ്പോള്‍ അതിനെപ്പറ്റി ആ സത്യനിഷേധികള്‍ പറയും; ഇത് വ്യക്തമായ ഒരു മായാജാലമാണെന്ന്‌.
\end{malayalam}}
\flushright{\begin{Arabic}
\quranayah[46][8]
\end{Arabic}}
\flushleft{\begin{malayalam}
അതല്ല, അദ്ദേഹം (റസൂല്‍) അത് കെട്ടിച്ചമച്ചു എന്നാണോ അവര്‍ പറയുന്നത്‌? നീ പറയുക: ഞാനത് കെട്ടിച്ചമച്ചതാണെങ്കില്‍ എനിക്ക് അല്ലാഹുവിന്‍റെ ശിക്ഷയില്‍ നിന്ന് ഒട്ടും രക്ഷനല്‍കാന്‍ നിങ്ങള്‍ക്ക് കഴിയില്ല. അതിന്‍റെ (ഖുര്‍ആന്‍റെ) കാര്യത്തില്‍ നിങ്ങള്‍ കടന്നു സംസാരിക്കുന്നതിനെപ്പറ്റി അവന്‍ നല്ലവണ്ണം അറിയുന്നവനാകുന്നു. എനിക്കും നിങ്ങള്‍ക്കുമിടയില്‍ സാക്ഷിയായി അവന്‍ തന്നെ മതി. അവന്‍ ഏറെ പൊറുക്കുന്നവനും കരുണാനിധിയുമാണ്‌.
\end{malayalam}}
\flushright{\begin{Arabic}
\quranayah[46][9]
\end{Arabic}}
\flushleft{\begin{malayalam}
(നബിയേ,) പറയുക: ഞാന്‍ ദൈവദൂതന്‍മാരില്‍ ഒരു പുതുമക്കാരനൊന്നുമല്ല. എന്നെക്കൊണ്ടോ നിങ്ങളെക്കൊണ്ടോ എന്ത് ചെയ്യപ്പെടും എന്ന് എനിക്ക് അറിയുകയുമില്ല. എനിക്ക് ബോധനം നല്‍കപ്പെടുന്നതിനെ പിന്തുടരുക മാത്രമാകുന്നു ഞാന്‍ ചെയ്യുന്നത്‌. ഞാന്‍ വ്യക്തമായ താക്കീതുകാരന്‍ മാത്രമാകുന്നു.
\end{malayalam}}
\flushright{\begin{Arabic}
\quranayah[46][10]
\end{Arabic}}
\flushleft{\begin{malayalam}
(നബിയേ,) പറയുക: നിങ്ങള്‍ ചിന്തിച്ചു നോക്കിയിട്ടുണ്ടോ? ഇത് (ഖുര്‍ആന്‍) അല്ലാഹുവിന്‍റെ പക്കല്‍ നിന്നുള്ളതായിരിക്കുകയും, എന്നിട്ട് നിങ്ങള്‍ ഇതില്‍ അവിശ്വസിക്കുകയും, ഇതു പോലുള്ളതിന് ഇസ്രായീല്‍ സന്തതികളില്‍ നിന്നുള്ള ഒരു സാക്ഷി സാക്ഷ്യം വഹിക്കുകയും, അങ്ങനെ അയാള്‍ (ഇതില്‍) വിശ്വസിക്കുകയും, നിങ്ങള്‍ അഹംഭാവം നടിക്കുകയുമാണ് ഉണ്ടായിട്ടുള്ളതെങ്കില്‍ (നിങ്ങളുടെ നില എത്ര മോശമായിരിക്കും?) അക്രമകാരികളായ ജനങ്ങളെ അല്ലാഹു നേര്‍വഴിയിലാക്കുകയില്ല; തീര്‍ച്ച.
\end{malayalam}}
\flushright{\begin{Arabic}
\quranayah[46][11]
\end{Arabic}}
\flushleft{\begin{malayalam}
വിശ്വസിച്ചവരെപ്പറ്റി ആ സത്യനിഷേധികള്‍ പറഞ്ഞു: ഇതൊരു നല്ലകാര്യമായിരുന്നെങ്കില്‍ ഞങ്ങളെക്കാള്‍ മുമ്പ് ഇവര്‍ അതില്‍ എത്തിച്ചേരുകയില്ലായിരുന്നു. ഇതുമുഖേന അവര്‍ സന്‍മാര്‍ഗം പ്രാപിച്ചിട്ടില്ലാത്തതു കൊണ്ട് അവര്‍ പറഞ്ഞേക്കും; ഇതൊരു പഴക്കം ചെന്ന വ്യാജവാദമാണെന്ന്‌.
\end{malayalam}}
\flushright{\begin{Arabic}
\quranayah[46][12]
\end{Arabic}}
\flushleft{\begin{malayalam}
മാതൃകായോഗ്യമായിക്കൊണ്ടും കാരുണ്യമായിക്കൊണ്ടും ഇതിനു മുമ്പ് മൂസായുടെ ഗ്രന്ഥം വന്നിട്ടുണ്ട്‌. ഇത് (അതിനെ) സത്യപ്പെടുത്തുന്ന അറബിഭാഷയിലുള്ള ഒരു ഗ്രന്ഥമാകുന്നു. അക്രമം ചെയ്തവര്‍ക്ക് താക്കീത് നല്‍കുവാന്‍ വേണ്ടിയും, സദ്‌വൃത്തര്‍ക്ക് സന്തോഷവാര്‍ത്ത ആയിക്കൊണ്ടും.
\end{malayalam}}
\flushright{\begin{Arabic}
\quranayah[46][13]
\end{Arabic}}
\flushleft{\begin{malayalam}
ഞങ്ങളുടെ രക്ഷിതാവ് അല്ലാഹുവാണ് എന്ന് പറയുകയും പിന്നീട് ചൊവ്വെ നിലകൊള്ളുകയും ചെയ്തവരാരോ അവര്‍ക്ക് യാതൊന്നും ഭയപ്പെടാനില്ല. അവര്‍ ദുഃഖിക്കേണ്ടി വരികയുമില്ല.
\end{malayalam}}
\flushright{\begin{Arabic}
\quranayah[46][14]
\end{Arabic}}
\flushleft{\begin{malayalam}
അവരാകുന്നു സ്വര്‍ഗാവകാശികള്‍. അവരതില്‍ നിത്യവാസികളായിരിക്കും. അവര്‍ പ്രവര്‍ത്തിച്ചിരുന്നതിനുള്ള പ്രതിഫലമത്രെ അത്‌.
\end{malayalam}}
\flushright{\begin{Arabic}
\quranayah[46][15]
\end{Arabic}}
\flushleft{\begin{malayalam}
തന്‍റെ മാതാപിതാക്കളോട് നല്ലനിലയില്‍ വര്‍ത്തിക്കണമെന്ന് നാം മനുഷ്യനോട് അനുശാസിച്ചിരിക്കുന്നു. അവന്‍റെ മാതാവ് പ്രയാസപ്പെട്ടുകൊണ്ട് അവനെ ഗര്‍ഭം ധരിക്കുകയും, പ്രയാസപ്പെട്ടുകൊണ്ട് അവനെ പ്രസവിക്കുകയും ചെയ്തു. അവന്‍റെ ഗര്‍ഭകാലവും മുലകുടിനിര്‍ത്തലും കൂടി മുപ്പത് മാസക്കാലമാകുന്നു. അങ്ങനെ അവന്‍ തന്‍റെ പൂര്‍ണ്ണശക്തി പ്രാപിക്കുകയും നാല്‍പത് വയസ്സിലെത്തുകയും ചെയ്താല്‍ ഇപ്രകാരം പറയും: എന്‍റെ രക്ഷിതാവേ, എനിക്കും എന്‍റെ മാതാപിതാക്കള്‍ക്കും നീ ചെയ്തു തന്നിട്ടുള്ള അനുഗ്രഹത്തിന് നന്ദികാണിക്കുവാനും നീ തൃപ്തിപ്പെടുന്ന സല്‍കര്‍മ്മം പ്രവര്‍ത്തിക്കുവാനും നീ എനിക്ക് പ്രചോദനം നല്‍കേണമേ. എന്‍റെ സന്തതികളില്‍ നീ എനിക്ക് നന്‍മയുണ്ടാക്കിത്തരികയും ചെയ്യേണമേ. തീര്‍ച്ചയായും ഞാന്‍ നിന്നിലേക്ക് ഖേദിച്ചുമടങ്ങിയിരിക്കുന്നു. തീര്‍ച്ചയായും ഞാന്‍ കീഴ്പെടുന്നവരുടെ കൂട്ടത്തിലാകുന്നു.
\end{malayalam}}
\flushright{\begin{Arabic}
\quranayah[46][16]
\end{Arabic}}
\flushleft{\begin{malayalam}
അത്തരക്കാരില്‍ നിന്നാകുന്നു അവര്‍ പ്രവര്‍ത്തിച്ചതില്‍ ഏറ്റവും ഉത്തമമായത് നാം സ്വീകരിക്കുന്നത്‌. അവരുടെ ദുഷ്പ്രവൃത്തികളെ സംബന്ധിച്ചിടത്തോളം നാം വിട്ടുവീഴ്ച കാണിക്കുകയും ചെയ്യും. (അവര്‍) സ്വര്‍ഗാവകാശികളുടെ കൂട്ടത്തിലായിരിക്കും. അവര്‍ക്ക് നല്‍കപ്പെട്ടിരുന്ന സത്യവാഗ്ദാനമത്രെ അത്‌.
\end{malayalam}}
\flushright{\begin{Arabic}
\quranayah[46][17]
\end{Arabic}}
\flushleft{\begin{malayalam}
ഒരാള്‍- തന്‍റെ മാതാപിതാക്കളോട് അവന്‍ പറഞ്ഞു: ഛെ, നിങ്ങള്‍ക്ക് കഷ്ടം! ഞാന്‍ (മരണാനന്തരം) പുറത്ത് കൊണ്ടവരപ്പെടും എന്ന് നിങ്ങള്‍ രണ്ടുപേരും എന്നോട് വാഗ്ദാനം ചെയ്യുകയാണോ? എനിക്ക് മുമ്പ് തലമുറകള്‍ കഴിഞ്ഞുപോയിട്ടുണ്ട്‌. അവര്‍ (മാതാപിതാക്കള്‍) അല്ലാഹുവോട് സഹായം തേടിക്കൊണ്ട് പറയുന്നു: നിനക്ക് നാശം. തീര്‍ച്ചയായും അല്ലാഹുവിന്‍റെ വാഗ്ദാനം സത്യമാകുന്നു. അപ്പോള്‍ അവന്‍ പറയുന്നു. ഇതൊക്കെ പൂര്‍വ്വികന്‍മാരുടെ കെട്ടുകഥകള്‍ മാത്രമാകുന്നു.
\end{malayalam}}
\flushright{\begin{Arabic}
\quranayah[46][18]
\end{Arabic}}
\flushleft{\begin{malayalam}
അത്തരക്കാരുടെ കാര്യത്തിലാകുന്നു (ശിക്ഷയുടെ) വചനം സ്ഥിരപ്പെട്ട് കഴിഞ്ഞിരിക്കുന്നത്‌. ജിന്നുകളില്‍ നിന്നും മനുഷ്യരില്‍ നിന്നും അവരുടെ മുമ്പ് കഴിഞ്ഞുപോയിട്ടുള്ള പല സമുദായങ്ങളുടെ കൂട്ടത്തില്‍. തീര്‍ച്ചയായും അവര്‍ നഷ്ടം പറ്റിയവരാകുന്നു.
\end{malayalam}}
\flushright{\begin{Arabic}
\quranayah[46][19]
\end{Arabic}}
\flushleft{\begin{malayalam}
ഓരോരുത്തര്‍ക്കും അവരവര്‍ പ്രവര്‍ത്തിച്ചതനുസരിച്ചുള്ള പദവികളുണ്ട്‌. അവര്‍ക്ക് അവരുടെ കര്‍മ്മങ്ങള്‍ക്ക് ഫലം പൂര്‍ത്തിയാക്കികൊടുക്കാനുമാണത്‌. അവരോട് അനീതി കാണിക്കപ്പെടുകയില്ല.
\end{malayalam}}
\flushright{\begin{Arabic}
\quranayah[46][20]
\end{Arabic}}
\flushleft{\begin{malayalam}
സത്യനിഷേധികള്‍ നരകത്തിനുമുമ്പില്‍ പ്രദര്‍ശിപ്പിക്കപ്പെടുന്ന ദിവസം (അവരോട് പറയപ്പെടും:) ഐഹികജീവിതത്തില്‍ നിങ്ങളുടെ നല്ല വസ്തുക്കളെല്ലാം നിങ്ങള്‍ പാഴാക്കിക്കളയുകയും, നിങ്ങള്‍ അവകൊണ്ട് സുഖമനുഭവിക്കുകയും ചെയ്തു. അതിനാല്‍ ന്യായം കൂടാതെ നിങ്ങള്‍ ഭൂമിയില്‍ അഹംഭാവം നടിച്ചിരുന്നതിന്‍റെ ഫലമായും നിങ്ങള്‍ ധിക്കാരം കാണിച്ചിരുന്നതിന്‍റെ ഫലമായും ഇന്നു നിങ്ങള്‍ക്ക് അപമാനകരമായ ശിക്ഷ പ്രതിഫലമായി നല്‍കപ്പെടുന്നു.
\end{malayalam}}
\flushright{\begin{Arabic}
\quranayah[46][21]
\end{Arabic}}
\flushleft{\begin{malayalam}
ആദിന്‍റെ സഹോദരനെ (അഥവാ ഹൂദിനെ) പ്പറ്റി നീ ഓര്‍മിക്കുക. അഹ്ഖാഫിലുള്ള തന്‍റെ ജനതയ്ക്ക് അദ്ദേഹം താക്കീത് നല്‍കിയ സന്ദര്‍ഭം. അദ്ദേഹത്തിന്‍റെ മുമ്പും അദ്ദേഹത്തിന്‍റെ പിന്നിലും താക്കീതുകാര്‍ കഴിഞ്ഞുപോയിട്ടുമുണ്ട്‌. അല്ലാഹുവെയല്ലാതെ നിങ്ങള്‍ ആരാധിക്കരുത്‌. നിങ്ങളുടെ മേല്‍ ഭയങ്കരമായ ഒരു ദിവസത്തെ ശിക്ഷ ഞാന്‍ ഭയപ്പെടുന്നു. (എന്നാണ് അദ്ദേഹം താക്കീത് നല്‍കിയത്‌.)
\end{malayalam}}
\flushright{\begin{Arabic}
\quranayah[46][22]
\end{Arabic}}
\flushleft{\begin{malayalam}
അവര്‍ പറഞ്ഞു: ഞങ്ങളുടെ ദൈവങ്ങളില്‍ നിന്ന് ഞങ്ങളെ തിരിച്ചുവിടാന്‍ വേണ്ടിയാണോ നീ ഞങ്ങളുടെ അടുത്ത് വന്നിരിക്കുന്നത്‌. എന്നാല്‍ നീ സത്യവാന്‍മാരുടെ കൂട്ടത്തിലാണെങ്കില്‍ ഞങ്ങള്‍ക്കു നീ താക്കീത് നല്‍കുന്നത് (ശിക്ഷ) ഞങ്ങള്‍ക്കു കൊണ്ടു വന്നു തരൂ.
\end{malayalam}}
\flushright{\begin{Arabic}
\quranayah[46][23]
\end{Arabic}}
\flushleft{\begin{malayalam}
അദ്ദേഹം പറഞ്ഞു: (അതിനെപ്പറ്റിയുള്ള) അറിവ് അല്ലാഹുവിങ്കല്‍ മാത്രമാകുന്നു. ഞാന്‍ എന്തൊന്നുമായി നിയോഗിക്കപ്പെട്ടിരിക്കുന്നുവോ അതു ഞാന്‍ നിങ്ങള്‍ക്ക് എത്തിച്ചുതരുന്നു. എന്നാല്‍ നിങ്ങളെ ഞാന്‍ കാണുന്നത് അവിവേകം കാണിക്കുന്ന ഒരു ജനതയായിട്ടാണ്‌.
\end{malayalam}}
\flushright{\begin{Arabic}
\quranayah[46][24]
\end{Arabic}}
\flushleft{\begin{malayalam}
അങ്ങനെ അതിനെ (ശിക്ഷയെ) തങ്ങളുടെ താഴ്‌വരകള്‍ക്ക് അഭിമുഖമായിക്കൊണ്ട് വെളിപ്പെട്ട ഒരു മേഘമായി അവര്‍ കണ്ടപ്പോള്‍ അവര്‍ പറഞ്ഞു: ഇതാ നമുക്ക് മഴ നല്‍കുന്ന ഒരു മേഘം! അല്ല, നിങ്ങള്‍ എന്തൊന്നിന് ധൃതികൂട്ടിയോ അതു തന്നെയാണിത്‌. അതെ വേദനയേറിയ ശിക്ഷ ഉള്‍കൊള്ളുന്ന ഒരു കാറ്റ്‌.
\end{malayalam}}
\flushright{\begin{Arabic}
\quranayah[46][25]
\end{Arabic}}
\flushleft{\begin{malayalam}
അതിന്‍റെ രക്ഷിതാവിന്‍റെ കല്‍പന പ്രകാരം സകല വസ്തുക്കളെയും അത് നശിപ്പിച്ചുകളയുന്നു. അങ്ങനെ അവര്‍ താമസിച്ചിരുന്ന സ്ഥലങ്ങളല്ലാതെ മറ്റൊന്നും കാണപ്പെടാത്ത അവസ്ഥയില്‍ അവര്‍ ആയിത്തീര്‍ന്നു. അപ്രകാരമാണ് കുറ്റവാളികളായ ജനങ്ങള്‍ക്ക് നാം പ്രതിഫലം നല്‍കുന്നത്‌.
\end{malayalam}}
\flushright{\begin{Arabic}
\quranayah[46][26]
\end{Arabic}}
\flushleft{\begin{malayalam}
നിങ്ങള്‍ക്ക് നാം സ്വാധീനം നല്‍കിയിട്ടില്ലാത്ത മേഖലകളില്‍ അവര്‍ക്കു നാം സ്വാധീനം നല്‍കുകയുണ്ടായി. കേള്‍വിയും കാഴ്ചകളും ഹൃദയങ്ങളും അവര്‍ക്കു നല്‍കുകയും ചെയ്തു. എന്നാല്‍ അല്ലാഹുവിന്‍റെ ദൃഷ്ടാന്തങ്ങളെ അവര്‍ നിഷേധിച്ചു കൊണ്ടിരുന്നതിനാല്‍ അവരുടെ കേള്‍വിയും കാഴ്ചകളും ഹൃദയങ്ങളും അവര്‍ക്ക് യാതൊരു പ്രയോജനവും ചെയ്തില്ല. എന്തൊന്നിനെ അവര്‍ പരിഹസിച്ചിരുന്നുവോ അത് അവരില്‍ വന്നുഭവിക്കുകയും ചെയ്തു.
\end{malayalam}}
\flushright{\begin{Arabic}
\quranayah[46][27]
\end{Arabic}}
\flushleft{\begin{malayalam}
നിങ്ങളുടെ ചുറ്റുമുള്ള ചില രാജ്യങ്ങളെയും നാം നശിപ്പിക്കുകയുണ്ടായി. ആ രാജ്യക്കാര്‍ സത്യത്തിലേക്കു മടങ്ങുവാന്‍ വേണ്ടി നാം തെളിവുകള്‍ വിവിധ രൂപത്തില്‍ വിവരിച്ചുകൊടുക്കുകയും ചെയ്തു.
\end{malayalam}}
\flushright{\begin{Arabic}
\quranayah[46][28]
\end{Arabic}}
\flushleft{\begin{malayalam}
അല്ലാഹുവിന് പുറമെ (അവനിലേക്ക്‌) സാമീപ്യം കിട്ടുവാനായി അവര്‍ ദൈവങ്ങളായി സ്വീകരിച്ചവര്‍ അപ്പോള്‍ എന്തുകൊണ്ട് അവരെ സഹായിച്ചില്ല? അല്ല, അവരെ വിട്ട് അവര്‍ (ദൈവങ്ങള്‍) അപ്രത്യക്ഷരായി. അത് (ബഹുദൈവവാദം) അവരുടെ വകയായുള്ള വ്യാജവും, അവര്‍ കൃത്രിമമായി സൃഷ്ടിച്ചുണ്ടാക്കിയിരുന്നതുമത്രെ.
\end{malayalam}}
\flushright{\begin{Arabic}
\quranayah[46][29]
\end{Arabic}}
\flushleft{\begin{malayalam}
ജിന്നുകളില്‍ ഒരു സംഘത്തെ നാം നിന്‍റെ അടുത്തേക്ക് ഖുര്‍ആന്‍ ശ്രദ്ധിച്ചുകേള്‍ക്കുവാനായി തിരിച്ചുവിട്ട സന്ദര്‍ഭം (ശ്രദ്ധേയമാണ്‌.) അങ്ങനെ അവര്‍ അതിന് സന്നിഹിതരായപ്പോള്‍ അവര്‍ അന്യോന്യം പറഞ്ഞു: നിങ്ങള്‍ നിശ്ശബ്ദരായിരിക്കൂ. അങ്ങനെ അത് കഴിഞ്ഞപ്പോള്‍ അവര്‍ തങ്ങളുടെ സമുദായത്തിലേക്ക് താക്കീതുകാരായിക്കൊണ്ട് തിരിച്ചുപോയി.
\end{malayalam}}
\flushright{\begin{Arabic}
\quranayah[46][30]
\end{Arabic}}
\flushleft{\begin{malayalam}
അവര്‍ പറഞ്ഞു: ഞങ്ങളുടെ സമുദായമേ, തീര്‍ച്ചയായും മൂസായ്ക്ക് ശേഷം അവതരിപ്പിക്കപ്പെട്ടതും, അതിന് മുമ്പുള്ളതിനെ സത്യപ്പെടുത്തുന്നതുമായ ഒരു വേദഗ്രന്ഥം ഞങ്ങള്‍ കേട്ടിരിക്കുന്നു. സത്യത്തിലേക്കും നേരായ പാതയിലേക്കും അത് വഴി കാട്ടുന്നു.
\end{malayalam}}
\flushright{\begin{Arabic}
\quranayah[46][31]
\end{Arabic}}
\flushleft{\begin{malayalam}
ഞങ്ങളുടെ സമുദായമേ, അല്ലാഹുവിങ്കലേക്ക് വിളിക്കുന്ന ആള്‍ക്ക് നിങ്ങള്‍ ഉത്തരം നല്‍കുകയും, അദ്ദേഹത്തില്‍ നിങ്ങള്‍ വിശ്വസിക്കുകയും ചെയ്യുക. അവന്‍ നിങ്ങള്‍ക്ക് നിങ്ങളുടെ പാപങ്ങള്‍ പൊറുത്തുതരികയും വേദനയേറിയ ശിക്ഷയില്‍ നിന്ന് അവന്‍ നിങ്ങള്‍ക്ക് അഭയം നല്‍കുകയും ചെയ്യുന്നതാണ്‌.
\end{malayalam}}
\flushright{\begin{Arabic}
\quranayah[46][32]
\end{Arabic}}
\flushleft{\begin{malayalam}
അല്ലാഹുവിങ്കലേക്ക് വിളിക്കുന്ന ആള്‍ക്ക് വല്ലവനും ഉത്തരം നല്‍കാതിരിക്കുന്ന പക്ഷം ഈ ഭൂമിയില്‍ (അല്ലാഹുവെ) അവന്ന് തോല്‍പിക്കാനാവില്ല. അല്ലാഹുവിന് പുറമെ അവനു രക്ഷാധികാരികള്‍ ഉണ്ടായിരിക്കുകയുമില്ല. അത്തരക്കാര്‍ വ്യക്തമായ വഴികേടിലാകുന്നു.
\end{malayalam}}
\flushright{\begin{Arabic}
\quranayah[46][33]
\end{Arabic}}
\flushleft{\begin{malayalam}
ആകാശങ്ങളും ഭൂമിയും സൃഷ്ടിക്കുകയും അവയെ സൃഷ്ടിച്ചതുകൊണ്ട് ക്ഷീണിക്കാതിരിക്കുകയും ചെയ്ത അല്ലാഹു മരിച്ചവരെ ജീവിപ്പിക്കാന്‍ കഴിവുള്ളവന്‍ തന്നെയാണെന്ന് അവര്‍ക്ക് കണ്ടുകൂടെ? അതെ; തീര്‍ച്ചയായും അവന്‍ ഏതു കാര്യത്തിനും കഴിവുള്ളവനാകുന്നു.
\end{malayalam}}
\flushright{\begin{Arabic}
\quranayah[46][34]
\end{Arabic}}
\flushleft{\begin{malayalam}
സത്യനിഷേധികള്‍ നരകത്തിനു മുമ്പില്‍ പ്രദര്‍ശിപ്പിക്കപ്പെടുന്ന ദിവസം. (അവരോട് ചോദിക്കപ്പെടും;) ഇതു സത്യം തന്നെയല്ലേ എന്ന്‌. അവര്‍ പറയും: അതെ; ഞങ്ങളുടെ രക്ഷിതാവിനെ തന്നെയാണ! അവന്‍ പറയും: എന്നാല്‍ നിങ്ങള്‍ അവിശ്വസിച്ചിരുന്നതിന്‍റെ ഫലമായി ശിക്ഷ ആസ്വദിച്ചു കൊള്ളുക.
\end{malayalam}}
\flushright{\begin{Arabic}
\quranayah[46][35]
\end{Arabic}}
\flushleft{\begin{malayalam}
ആകയാല്‍ ദൃഢമനസ്കരായ ദൈവദൂതന്‍മാര്‍ ക്ഷമിച്ചത് പോലെ നീ ക്ഷമിക്കുക. അവരുടെ (സത്യനിഷേധികളുടെ) കാര്യത്തിന് നീ ധൃതി കാണിക്കരുത്‌. അവര്‍ക്ക് താക്കീത് നല്‍കപ്പെടുന്നത് (ശിക്ഷ) അവര്‍ നേരില്‍ കാണുന്ന ദിവസം പകലില്‍ നിന്നുള്ള ഒരു നാഴിക നേരം മാത്രമേ തങ്ങള്‍ (ഇഹലോകത്ത്‌) താമസിച്ചിട്ടുള്ളു എന്ന പോലെ അവര്‍ക്കു തോന്നും. ഇതൊരു ഉല്‍ബോധനം ആകുന്നു. എന്നാല്‍ ധിക്കാരികളായ ജനങ്ങളല്ലാതെ നശിപ്പിക്കപ്പെടുമോ?
\end{malayalam}}
\chapter{\textmalayalam{മുഹമ്മദ്}}
\begin{Arabic}
\Huge{\centerline{\basmalah}}\end{Arabic}
\flushright{\begin{Arabic}
\quranayah[47][1]
\end{Arabic}}
\flushleft{\begin{malayalam}
അവിശ്വസിക്കുകയും അല്ലാഹുവിന്‍റെ മാര്‍ഗത്തില്‍ നിന്ന് (ജനങ്ങളെ) തടയുകയും ചെയ്തവരാരോ അവരുടെ കര്‍മ്മങ്ങളെ അല്ലാഹു പാഴാക്കികളയുന്നതാണ്‌.
\end{malayalam}}
\flushright{\begin{Arabic}
\quranayah[47][2]
\end{Arabic}}
\flushleft{\begin{malayalam}
വിശ്വസിക്കുകയും സല്‍കര്‍മ്മങ്ങള്‍ പ്രവര്‍ത്തിക്കുകയും മുഹമ്മദ് നബിയുടെ മേല്‍ അവതരിപ്പിക്കപ്പെട്ടതില്‍ -അതത്രെ അവരുടെ രക്ഷിതാവിങ്കല്‍ നിന്നുള്ള സത്യം - വിശ്വസിക്കുകയും ചെയ്തവരാരോ അവരില്‍ നിന്ന് അവരുടെ തിന്‍മകള്‍ അവന്‍ (അല്ലാഹു) മായ്ച്ചുകളയുകയും അവരുടെ അവസ്ഥ അവന്‍ നന്നാക്കിതീര്‍ക്കുകയും ചെയ്യുന്നതാണ്‌.
\end{malayalam}}
\flushright{\begin{Arabic}
\quranayah[47][3]
\end{Arabic}}
\flushleft{\begin{malayalam}
അതെന്തുകൊണ്ടെന്നാല്‍ സത്യനിഷേധികള്‍ അസത്യത്തെയാണ് പിന്തുടര്‍ന്നത്‌. വിശ്വസിച്ചവരാകട്ടെ തങ്ങളുടെ രക്ഷിതാവിങ്കല്‍ നിന്നുള്ള സത്യത്തെയാണ് പിന്‍പറ്റിയത്‌. അപ്രകാരം അല്ലാഹു ജനങ്ങള്‍ക്കു വേണ്ടി അവരുടെ മാതൃകകള്‍ വിശദീകരിക്കുന്നു.
\end{malayalam}}
\flushright{\begin{Arabic}
\quranayah[47][4]
\end{Arabic}}
\flushleft{\begin{malayalam}
ആകയാല്‍ സത്യനിഷേധികളുമായി നിങ്ങള്‍ ഏറ്റുമുട്ടിയാല്‍ (നിങ്ങള്‍) പിരടികളില്‍ വെട്ടുക. അങ്ങനെ അവരെ നിങ്ങള്‍ അമര്‍ച്ച ചെയ്തു കഴിഞ്ഞാല്‍ നിങ്ങള്‍ അവരെ ശക്തിയായി ബന്ധിക്കുക. എന്നിട്ട് അതിനു ശേഷം (അവരോട്‌) ദാക്ഷിണ്യം കാണിക്കുകയോ, അല്ലെങ്കില്‍ മോചനമൂല്യം വാങ്ങി വിട്ടയക്കുകയോ ചെയ്യുക. യുദ്ധം അതിന്‍റെ ഭാരങ്ങള്‍ ഇറക്കിവെക്കുന്നത് വരെയത്രെ അത്‌. അതാണ് (യുദ്ധത്തിന്‍റെ) മുറ. അല്ലാഹു ഉദ്ദേശിച്ചിരുന്നെങ്കില്‍ അവരുടെ നേരെ അവന്‍ ശിക്ഷാനടപടി സ്വീകരിക്കുമായിരുന്നു. പക്ഷെ നിങ്ങളില്‍ ചിലരെ മറ്റു ചിലരെ കൊണ്ട് പരീക്ഷിക്കേണ്ടതിനായിട്ടാകുന്നു ഇത്‌. അല്ലാഹുവിന്‍റെ മാര്‍ഗത്തില്‍ കൊല്ലപ്പെട്ടവരാകട്ടെ അല്ലാഹു അവരുടെ കര്‍മ്മങ്ങള്‍ പാഴാക്കുകയേ ഇല്ല.
\end{malayalam}}
\flushright{\begin{Arabic}
\quranayah[47][5]
\end{Arabic}}
\flushleft{\begin{malayalam}
അവന്‍ അവരെ ലക്ഷ്യത്തിലേക്ക് നയിക്കുകയും അവരുടെ അവസ്ഥ നന്നാക്കിത്തീര്‍ക്കുകയും ചെയ്യുന്നതാണ്‌.
\end{malayalam}}
\flushright{\begin{Arabic}
\quranayah[47][6]
\end{Arabic}}
\flushleft{\begin{malayalam}
സ്വര്‍ഗത്തില്‍ അവരെ അവന്‍ പ്രവേശിപ്പിക്കുകയും ചെയ്യും. അവര്‍ക്ക് അതിനെ അവന്‍ മുമ്പേ പരിചയപ്പെടുത്തി കൊടുത്തിട്ടുണ്ട്‌.
\end{malayalam}}
\flushright{\begin{Arabic}
\quranayah[47][7]
\end{Arabic}}
\flushleft{\begin{malayalam}
സത്യവിശ്വാസികളേ, നിങ്ങള്‍ അല്ലാഹുവെ സഹായിക്കുന്ന പക്ഷം അവന്‍ നിങ്ങളെ സഹായിക്കുകയും നിങ്ങളുടെ പാദങ്ങള്‍ ഉറപ്പിച്ച് നിര്‍ത്തുകയും ചെയ്യുന്നതാണ്‌.
\end{malayalam}}
\flushright{\begin{Arabic}
\quranayah[47][8]
\end{Arabic}}
\flushleft{\begin{malayalam}
അവിശ്വസിച്ചവരാരോ, അവര്‍ക്ക് നാശം. അവന്‍ (അല്ലാഹു) അവരുടെ കര്‍മ്മങ്ങളെ പാഴാക്കികളയുന്നതുമാണ്‌.
\end{malayalam}}
\flushright{\begin{Arabic}
\quranayah[47][9]
\end{Arabic}}
\flushleft{\begin{malayalam}
അതെന്തുകൊണ്ടെന്നാല്‍ അല്ലാഹു അവതരിപ്പിച്ചതിനെ അവര്‍ വെറുത്ത് കളഞ്ഞു. അപ്പോള്‍ അവരുടെ കര്‍മ്മങ്ങളെ അവന്‍ നിഷ്ഫലമാക്കിത്തീര്‍ത്തു.
\end{malayalam}}
\flushright{\begin{Arabic}
\quranayah[47][10]
\end{Arabic}}
\flushleft{\begin{malayalam}
അവര്‍ ഭൂമിയില്‍ കൂടി സഞ്ചരിച്ചിട്ടില്ലേ? എങ്കില്‍ തങ്ങളുടെ മുന്‍ഗാമികളുടെ പര്യവസാനം എങ്ങനെയായിരുന്നു എന്നവര്‍ക്ക് നോക്കിക്കാണാമായിരുന്നു. അല്ലാഹു അവരെ തകര്‍ത്തു കളഞ്ഞു. ഈ സത്യനിഷേധികള്‍ക്കുമുണ്ട് അതു പോലെയുള്ളവ. (ശിക്ഷകള്‍)
\end{malayalam}}
\flushright{\begin{Arabic}
\quranayah[47][11]
\end{Arabic}}
\flushleft{\begin{malayalam}
അതിന്‍റെ കാരണമെന്തെന്നാല്‍ അല്ലാഹു സത്യവിശ്വാസികളുടെ രക്ഷാധികാരിയാണ്‌. സത്യനിഷേധികള്‍ക്കാകട്ടെ ഒരു രക്ഷാധികാരിയും ഇല്ല.
\end{malayalam}}
\flushright{\begin{Arabic}
\quranayah[47][12]
\end{Arabic}}
\flushleft{\begin{malayalam}
വിശ്വസിക്കുകയും സല്‍കര്‍മ്മങ്ങള്‍ പ്രവര്‍ത്തിക്കുകയും ചെയ്തവരെ താഴ്ഭാഗത്ത്കൂടി നദികള്‍ ഒഴുകുന്ന സ്വര്‍ഗത്തോപ്പുകളില്‍ അല്ലാഹു പ്രവേശിപ്പിക്കുന്നതാണ്‌; തീര്‍ച്ച. സത്യനിഷേധികളാകട്ടെ (ഇഹലോകത്ത്‌) സുഖമനുഭവിക്കുകയും നാല്‍കാലികള്‍ തിന്നുന്നത് പോലെ തിന്നുകൊണ്ടിരിക്കുകയും ചെയ്യുന്നു. നരകമാണ് അവര്‍ക്കുള്ള വാസസ്ഥലം.
\end{malayalam}}
\flushright{\begin{Arabic}
\quranayah[47][13]
\end{Arabic}}
\flushleft{\begin{malayalam}
നിന്നെ പുറത്താക്കിയ നിന്‍റെ രാജ്യത്തെക്കാള്‍ ശക്തിയേറിയ എത്രയെത്ര രാജ്യങ്ങള്‍! അവരെ നാം നശിപ്പിച്ചു. അപ്പോള്‍ അവര്‍ക്കൊരു സഹായിയുമുണ്ടായിരുന്നില്ല.
\end{malayalam}}
\flushright{\begin{Arabic}
\quranayah[47][14]
\end{Arabic}}
\flushleft{\begin{malayalam}
തന്‍റെ രക്ഷിതാവിങ്കല്‍ നിന്നുള്ള സ്പഷ്ടമായ തെളിവനുസരിച്ച് നിലകൊള്ളുന്ന ഒരാള്‍ സ്വന്തം ദുഷ് പ്രവൃത്തി അലംകൃതമായി തോന്നുകയും തന്നിഷ്ടങ്ങളെ പിന്തുടരുകയും ചെയ്ത ഒരുവനെ പോലെയാണോ?
\end{malayalam}}
\flushright{\begin{Arabic}
\quranayah[47][15]
\end{Arabic}}
\flushleft{\begin{malayalam}
സൂക്ഷ്മതയുള്ളവര്‍ക്ക് വാഗ്ദാനം ചെയ്യപ്പെട്ടിട്ടുള്ള സ്വര്‍ഗത്തിന്‍റെ അവസ്ഥ എങ്ങനെയെന്നാല്‍ അതില്‍ പകര്‍ച്ച വരാത്ത വെള്ളത്തിന്‍റെ അരുവികളുണ്ട്‌. രുചിഭേദം വരാത്ത പാലിന്‍റെ അരുവികളും, കുടിക്കുന്നവര്‍ക്ക് ആസ്വാദ്യമായ മദ്യത്തിന്‍റെ അരുവികളും, ശുദ്ധീകരിക്കപ്പെട്ട തേനിന്‍റെ അരുവികളുമുണ്ട്‌. അവര്‍ക്കതില്‍ എല്ലാതരം കായ്കനികളുമുണ്ട്‌. തങ്ങളുടെ രക്ഷിതാവിങ്കല്‍ നിന്നുള്ള പാപമോചനവുമുണ്ട്‌. (ഈ സ്വര്‍ഗവാസികളുടെ അവസ്ഥ) നരകത്തില്‍ നിത്യവാസിയായിട്ടുള്ളവനെപ്പോലെ ആയിരിക്കുമോ? അത്തരക്കാര്‍ക്കാകട്ടെ കൊടും ചൂടുള്ള വെള്ളമായിരിക്കും കുടിക്കാന്‍ നല്‍കപ്പെടുക. അങ്ങനെ അത് അവരുടെ കുടലുകളെ ഛിന്നഭിന്നമാക്കിക്കളയും.
\end{malayalam}}
\flushright{\begin{Arabic}
\quranayah[47][16]
\end{Arabic}}
\flushleft{\begin{malayalam}
അവരുടെ കൂട്ടത്തില്‍ നീ പറയുന്നത് ശ്രദ്ധിച്ച് കേള്‍ക്കുന്ന ചിലരുണ്ട്‌. എന്നാല്‍ നിന്‍റെ അടുത്ത് നിന്ന് അവര്‍ പുറത്ത് പോയാല്‍ വേദ വിജ്ഞാനം നല്‍കപ്പെട്ടവരോട് അവര്‍ (പരിഹാസപൂര്‍വ്വം) പറയും: എന്താണ് ഇദ്ദേഹം ഇപ്പോള്‍ പറഞ്ഞത്‌? അത്തരക്കാരുടെ ഹൃദയങ്ങളിന്‍മേലാകുന്നു അല്ലാഹു മുദ്രവെച്ചിരിക്കുന്നത്‌. തങ്ങളുടെ തന്നിഷ്ടങ്ങളെ പിന്‍പറ്റുകയാണവര്‍ ചെയ്തത്‌.
\end{malayalam}}
\flushright{\begin{Arabic}
\quranayah[47][17]
\end{Arabic}}
\flushleft{\begin{malayalam}
സന്‍മാര്‍ഗം സ്വീകരിച്ചവരാകട്ടെ അല്ലാഹു അവര്‍ക്ക് കൂടുതല്‍ മാര്‍ഗദര്‍ശനം നല്‍കുകയും, അവര്‍ക്ക് വേണ്ടതായ സൂക്ഷ്മത അവര്‍ക്കു നല്‍കുകയും ചെയ്യുന്നതാണ്‌.
\end{malayalam}}
\flushright{\begin{Arabic}
\quranayah[47][18]
\end{Arabic}}
\flushleft{\begin{malayalam}
ഇനി ആ (അന്ത്യ) സമയം പെട്ടെന്ന് അവര്‍ക്ക് വന്നെത്തുന്നതല്ലാതെ മറ്റുവല്ലതും അവര്‍ക്കു കാത്തിരിക്കാനുണ്ടോ? എന്നാല്‍ അതിന്‍റെ അടയാളങ്ങള്‍ വന്നു കഴിഞ്ഞിരിക്കുന്നു. അപ്പോള്‍ അത് അവര്‍ക്കു വന്നുകഴിഞ്ഞാല്‍ അവര്‍ക്കുള്ള ഉല്‍ബോധനം അവര്‍ക്കെങ്ങനെ പ്രയോജനപ്പെടും?
\end{malayalam}}
\flushright{\begin{Arabic}
\quranayah[47][19]
\end{Arabic}}
\flushleft{\begin{malayalam}
ആകയാല്‍ അല്ലാഹുവല്ലാതെ യാതൊരു ദൈവവുമില്ലെന്ന് നീ മനസ്സിലാക്കുക. നിന്‍റെ പാപത്തിന് നീ പാപമോചനം തേടുക. സത്യവിശ്വാസികള്‍ക്കും സത്യവിശ്വാസിനികള്‍ക്കും വേണ്ടിയും (പാപമോചനംതേടുക.) നിങ്ങളുടെ പോക്കുവരവും നിങ്ങളുടെ താമസവും അല്ലാഹു അറിയുന്നുണ്ട്‌
\end{malayalam}}
\flushright{\begin{Arabic}
\quranayah[47][20]
\end{Arabic}}
\flushleft{\begin{malayalam}
സത്യവിശ്വാസികള്‍ പറയും: ഒരു സൂറത്ത് അവതരിപ്പിക്കപ്പെടാത്തതെന്താണ്‌? എന്നാല്‍ ഖണ്ഡിതമായ നിയമങ്ങളുള്ള ഒരു സൂറത്ത് അവതരിപ്പിക്കപ്പെടുകയും അതില്‍ യുദ്ധത്തെപ്പറ്റി പ്രസ്താവിക്കപ്പെടുകയും ചെയ്താല്‍ ഹൃദയങ്ങളില്‍ രോഗമുള്ളവര്‍, മരണം ആസന്നമായതിനാല്‍ ബോധരഹിതനായ ആള്‍ നോക്കുന്നത് പോലെ നിന്‍റെ നേര്‍ക്ക് നോക്കുന്നതായി കാണാം. എന്നാല്‍ അവര്‍ക്ക് ഏറ്റവും അനുയോജ്യമായത് തന്നെയാണത്‌.
\end{malayalam}}
\flushright{\begin{Arabic}
\quranayah[47][21]
\end{Arabic}}
\flushleft{\begin{malayalam}
അനുസരണവും ഉചിതമായ വാക്കുമാണ് വേണ്ടത്‌. എന്നാല്‍ കാര്യം തീര്‍ച്ചപ്പെട്ടു കഴിഞ്ഞപ്പോള്‍ അവര്‍ അല്ലാഹുവോട് സത്യസന്ധത കാണിച്ചിരുന്നെങ്കില്‍ അതായിരുന്നു അവര്‍ക്ക് കൂടുതല്‍ ഉത്തമം.
\end{malayalam}}
\flushright{\begin{Arabic}
\quranayah[47][22]
\end{Arabic}}
\flushleft{\begin{malayalam}
എന്നാല്‍ നിങ്ങള്‍ കൈകാര്യകര്‍ത്തൃത്വം ഏറ്റെടുക്കുകയാണെങ്കില്‍ ഭൂമിയില്‍ നിങ്ങള്‍ കുഴപ്പമുണ്ടാക്കുകയും, നിങ്ങളുടെ കുടുംബബന്ധങ്ങള്‍ വെട്ടിമുറിക്കുകയും ചെയ്തേക്കുമോ?
\end{malayalam}}
\flushright{\begin{Arabic}
\quranayah[47][23]
\end{Arabic}}
\flushleft{\begin{malayalam}
അത്തരക്കാരെയാണ് അല്ലാഹു ശപിച്ചിട്ടുള്ളത്‌. അങ്ങനെ അവര്‍ക്ക് ബധിരത നല്‍കുകയും, അവരുടെ കണ്ണുകള്‍ക്ക് അന്ധത വരുത്തുകയും ചെയ്തിരിക്കുന്നു.
\end{malayalam}}
\flushright{\begin{Arabic}
\quranayah[47][24]
\end{Arabic}}
\flushleft{\begin{malayalam}
അപ്പോള്‍ അവര്‍ ഖുര്‍ആന്‍ ചിന്തിച്ചുമനസ്സിലാക്കുന്നില്ലേ? അതല്ല, ഹൃദയങ്ങളിന്‍മേല്‍ പൂട്ടുകളിട്ടിരിക്കയാണോ?
\end{malayalam}}
\flushright{\begin{Arabic}
\quranayah[47][25]
\end{Arabic}}
\flushleft{\begin{malayalam}
തങ്ങള്‍ക്ക് സന്‍മാര്‍ഗം വ്യക്തമായി കഴിഞ്ഞ ശേഷം പുറകോട്ട് തിരിച്ചുപോയവരാരോ, അവര്‍ക്ക് പിശാച് (തങ്ങളുടെ ചെയ്തികള്‍) അലംകൃതമായി തോന്നിച്ചിരിക്കുകയാണ്‌; തീര്‍ച്ച. അവര്‍ക്ക് അവന്‍ (വ്യാമോഹങ്ങള്‍) നീട്ടിയിട്ടു കൊടുക്കുകയും ചെയ്തിരിക്കുന്നു.
\end{malayalam}}
\flushright{\begin{Arabic}
\quranayah[47][26]
\end{Arabic}}
\flushleft{\begin{malayalam}
അത്‌, അല്ലാഹു അവതരിപ്പിച്ചത് ഇഷ്ടപ്പെടാത്തവരോട് ചില കാര്യങ്ങളില്‍ ഞങ്ങള്‍ നിങ്ങളുടെ കല്‍പന അനുസരിക്കാമെന്ന് അവര്‍ പറഞ്ഞിട്ടുള്ളത് കൊണ്ടാണ്‌. അവര്‍ രഹസ്യമാക്കി വെക്കുന്നത് അല്ലാഹു അറിയുന്നു.
\end{malayalam}}
\flushright{\begin{Arabic}
\quranayah[47][27]
\end{Arabic}}
\flushleft{\begin{malayalam}
അപ്പോള്‍ മലക്കുകള്‍ അവരുടെ മുഖത്തും പിന്‍ഭാഗത്തും അടിച്ചു കൊണ്ട് അവരെ മരിപ്പിക്കുന്ന സന്ദര്‍ഭത്തില്‍ എന്തായിരിക്കും അവരുടെ സ്ഥിതി!
\end{malayalam}}
\flushright{\begin{Arabic}
\quranayah[47][28]
\end{Arabic}}
\flushleft{\begin{malayalam}
അതെന്തുകൊണ്ടെന്നാല്‍ അല്ലാഹുവിന് വെറുപ്പുണ്ടാക്കുന്ന കാര്യത്തെ അവര്‍ പിന്തുടരുകയും, അവന്‍റെ പ്രീതി അവര്‍ ഇഷ്ടപ്പെടാതിരിക്കുകയുമാണ് ചെയ്തത്‌. അതിനാല്‍ അവരുടെ കര്‍മ്മങ്ങളെ അവന്‍ നിഷ്ഫലമാക്കികളഞ്ഞു.
\end{malayalam}}
\flushright{\begin{Arabic}
\quranayah[47][29]
\end{Arabic}}
\flushleft{\begin{malayalam}
അതല്ല, ഹൃദയങ്ങളില്‍ രോഗമുള്ള ആളുകള്‍ അല്ലാഹു അവരുടെ ഉള്ളിലെ പക വെളിപ്പെടുത്തുകയേയില്ല എന്നാണോ വിചാരിച്ചത്‌?
\end{malayalam}}
\flushright{\begin{Arabic}
\quranayah[47][30]
\end{Arabic}}
\flushleft{\begin{malayalam}
നാം ഉദ്ദേശിച്ചിരുന്നെങ്കില്‍ നിനക്ക് നാം അവരെ കാട്ടിത്തരുമായിരുന്നു. അങ്ങനെ അവരുടെ ലക്ഷണം കൊണ്ട് നിനക്ക് അവരെ മനസ്സിലാക്കാമായിരുന്നു. സംസാരശൈലിയിലൂടെയും തീര്‍ച്ചയായും നിനക്ക് അവരെ മനസ്സിലാക്കാവുന്നതാണ്‌. അല്ലാഹു നിങ്ങളുടെ പ്രവൃത്തികള്‍ അറിയുന്നു.
\end{malayalam}}
\flushright{\begin{Arabic}
\quranayah[47][31]
\end{Arabic}}
\flushleft{\begin{malayalam}
നിങ്ങളുടെ കൂട്ടത്തില്‍ സമരം ചെയ്യുന്നവരെയും ക്ഷമ കൈക്കൊള്ളുന്നവരെയും നാം തിരിച്ചറിയുകയും, നിങ്ങളുടെ വര്‍ത്തമാനങ്ങള്‍ നാം പരിശോധിച്ചു നോക്കുകയും ചെയ്യുന്നത് വരെ നിങ്ങളെ നാം പരീക്ഷിക്കുക തന്നെ ചെയ്യും.
\end{malayalam}}
\flushright{\begin{Arabic}
\quranayah[47][32]
\end{Arabic}}
\flushleft{\begin{malayalam}
അവിശ്വസിക്കുകയും അല്ലാഹുവിന്‍റെ മാര്‍ഗത്തില്‍ നിന്ന് (ജനങ്ങളെ) തടയുകയും, തങ്ങള്‍ക്ക് സന്‍മാര്‍ഗം വ്യക്തമായി കഴിഞ്ഞതിനു ശേഷം റസൂലുമായി മാത്സര്യത്തില്‍ ഏര്‍പെടുകയും ചെയ്തവരാരോ തീര്‍ച്ചയായും അവര്‍ അല്ലാഹുവിന് യാതൊരു ഉപദ്രവവും വരുത്തുകയില്ല. വഴിയെ അവന്‍ അവരുടെ കര്‍മ്മങ്ങള്‍ നിഷ്ഫലമാക്കിക്കളയുകയും ചെയ്യും.
\end{malayalam}}
\flushright{\begin{Arabic}
\quranayah[47][33]
\end{Arabic}}
\flushleft{\begin{malayalam}
സത്യവിശ്വാസികളേ, നിങ്ങള്‍ അല്ലാഹുവെ അനുസരിക്കുക. റസൂലിനെയും നിങ്ങള്‍ അനുസരിക്കുക. നിങ്ങളുടെ കര്‍മ്മങ്ങളെ നിങ്ങള്‍ നിഷ്ഫലമാക്കിക്കളയാതിരിക്കുകയും ചെയ്യുക.
\end{malayalam}}
\flushright{\begin{Arabic}
\quranayah[47][34]
\end{Arabic}}
\flushleft{\begin{malayalam}
അവിശ്വസിക്കുകയും, അല്ലാഹുവിന്‍റെ മാര്‍ഗത്തില്‍ നിന്ന് (ജനങ്ങളെ) തടയുകയും, എന്നിട്ട് സത്യനിഷേധികളായിക്കൊണ്ട് തന്നെ മരിക്കുകയും ചെയ്തവരാരോ അവര്‍ക്ക് അല്ലാഹു പൊറുത്തുകൊടുക്കുകയേ ഇല്ല.
\end{malayalam}}
\flushright{\begin{Arabic}
\quranayah[47][35]
\end{Arabic}}
\flushleft{\begin{malayalam}
ആകയാല്‍ നിങ്ങള്‍ ദൌര്‍ബല്യം കാണിക്കരുത്‌. നിങ്ങള്‍ തന്നെയാണ് ഉന്നതന്‍മാര്‍ എന്നിരിക്കെ (ശത്രുക്കളെ) നിങ്ങള്‍ സന്ധിക്കു ക്ഷണിക്കുകയും ചെയ്യരുത്‌. അല്ലാഹു നിങ്ങളുടെ കൂടെയുണ്ട്‌. നിങ്ങളുടെ കര്‍മ്മഫലങ്ങള്‍ നിങ്ങള്‍ക്ക് ഒരിക്കലും അവന്‍ നഷ്ടപ്പെടുത്തുകയില്ല.
\end{malayalam}}
\flushright{\begin{Arabic}
\quranayah[47][36]
\end{Arabic}}
\flushleft{\begin{malayalam}
ഐഹികജീവിതം കളിയും വിനോദവും മാത്രമാകുന്നു. നിങ്ങള്‍ വിശ്വസിക്കുകയും സൂക്ഷ്മത പാലിക്കുകയും ചെയ്യുന്ന പക്ഷം നിങ്ങള്‍ക്കുള്ള പ്രതിഫലം അവന്‍ നിങ്ങള്‍ക്ക് നല്‍കുന്നതാണ്‌. നിങ്ങളോട് നിങ്ങളുടെ സ്വത്തുക്കള്‍ അവന്‍ ചോദിക്കുകയുമില്ല.
\end{malayalam}}
\flushright{\begin{Arabic}
\quranayah[47][37]
\end{Arabic}}
\flushleft{\begin{malayalam}
നിങ്ങളോട് അവ (സ്വത്തുക്കള്‍) ചോദിച്ച് അവന്‍ നിങ്ങളെ ബുദ്ധിമുട്ടിച്ചിരുന്നെങ്കില്‍ നിങ്ങള്‍ പിശുക്ക് കാണിക്കുകയും നിങ്ങളുടെ ഉള്ളിലെ പക അവന്‍ വെളിയില്‍ കൊണ്ടു വരികയും ചെയ്യുമായിരുന്നു.
\end{malayalam}}
\flushright{\begin{Arabic}
\quranayah[47][38]
\end{Arabic}}
\flushleft{\begin{malayalam}
ഹേ; കൂട്ടരേ, അല്ലാഹുവിന്‍റെ മാര്‍ഗത്തില്‍ നിങ്ങള്‍ ചെലവഴിക്കുന്നതിനാണ് നിങ്ങള്‍ ആഹ്വാനം ചെയ്യപ്പെടുന്നത്‌. അപ്പോള്‍ നിങ്ങളില്‍ ചിലര്‍ പിശുക്ക് കാണിക്കുന്നു. വല്ലവനും പിശുക്കു കാണിക്കുന്ന പക്ഷം തന്നോട് തന്നെയാണ് അവന്‍ പിശുക്ക് കാണിക്കുന്നത്‌. അല്ലാഹുവാകട്ടെ പരാശ്രയമുക്തനാകുന്നു. നിങ്ങളോ ദരിദ്രന്‍മാരും. നിങ്ങള്‍ പിന്തിരിഞ്ഞു കളയുകയാണെങ്കില്‍ നിങ്ങളല്ലാത്ത ഒരു ജനതയെ അവന്‍ പകരം കൊണ്ടുവരുന്നതാണ്‌. എന്നിട്ട് അവര്‍ നിങ്ങളെപ്പോലെയായിരിക്കുകയുമില്ല.
\end{malayalam}}
\chapter{\textmalayalam{ഫതഹ് ( വിജയം )}}
\begin{Arabic}
\Huge{\centerline{\basmalah}}\end{Arabic}
\flushright{\begin{Arabic}
\quranayah[48][1]
\end{Arabic}}
\flushleft{\begin{malayalam}
തീര്‍ച്ചയായും നിനക്ക് നാം പ്രത്യക്ഷമായ ഒരു വിജയം നല്‍കിയിരിക്കുന്നു.
\end{malayalam}}
\flushright{\begin{Arabic}
\quranayah[48][2]
\end{Arabic}}
\flushleft{\begin{malayalam}
നിന്‍റെ പാപത്തില്‍ നിന്ന് മുമ്പ് കഴിഞ്ഞുപോയതും പിന്നീട് ഉണ്ടാകുന്നതും അല്ലാഹു നിനക്ക് പൊറുത്തുതരുന്നതിനു വേണ്ടിയും, അവന്‍റെ അനുഗ്രഹം നിനക്ക് നിറവേറ്റിത്തരുന്നതിനു വേണ്ടിയും, നിന്നെ നേരായ പാതയിലൂടെ നയിക്കുന്നതിന് വേണ്ടിയുമാകുന്നു അത്‌.
\end{malayalam}}
\flushright{\begin{Arabic}
\quranayah[48][3]
\end{Arabic}}
\flushleft{\begin{malayalam}
അന്തസ്സാര്‍ന്ന ഒരു സഹായം അല്ലാഹു നിനക്ക് നല്‍കാന്‍ വേണ്ടിയും.
\end{malayalam}}
\flushright{\begin{Arabic}
\quranayah[48][4]
\end{Arabic}}
\flushleft{\begin{malayalam}
അവനാകുന്നു സത്യവിശ്വാസികളുടെ ഹൃദയങ്ങളില്‍ ശാന്തി ഇറക്കികൊടുത്തത.് അവരുടെ വിശ്വാസത്തോടൊപ്പം കൂടുതല്‍ വിശ്വാസം ഉണ്ടായിത്തീരുന്നതിന് വേണ്ടി. അല്ലാഹുവിന്നുള്ളതാകുന്നു ആകാശങ്ങളിലെയും ഭൂമിയിലെയും സൈന്യങ്ങള്‍. അല്ലാഹു സര്‍വ്വജ്ഞനും യുക്തിമാനുമായിരിക്കുന്നു.
\end{malayalam}}
\flushright{\begin{Arabic}
\quranayah[48][5]
\end{Arabic}}
\flushleft{\begin{malayalam}
സത്യവിശ്വാസികളെയും സത്യവിശ്വാസിനികളെയും താഴ്ഭാഗത്തു കൂടി നദികള്‍ ഒഴുകുന്ന സ്വര്‍ഗത്തോപ്പുകളില്‍ നിത്യവാസികളെന്ന നിലയില്‍ പ്രവേശിപ്പിക്കാന്‍ വേണ്ടിയത്രെ അത്‌. അവരില്‍ നിന്ന് അവരുടെ തിന്‍മകള്‍ മായ്ച്ചുകളയുവാന്‍ വേണ്ടിയും. അല്ലാഹുവിന്‍റെ അടുക്കല്‍ അത് ഒരു മഹാഭാഗ്യമാകുന്നു.
\end{malayalam}}
\flushright{\begin{Arabic}
\quranayah[48][6]
\end{Arabic}}
\flushleft{\begin{malayalam}
അല്ലാഹുവെപ്പറ്റി തെറ്റായ ധാരണവെച്ചുപുലര്‍ത്തുന്ന കപടവിശ്വാസികളെയും കപടവിശ്വാസിനികളെയും ബഹുദൈവവിശ്വാസികളെയും ബഹുദൈവവിശ്വാസിനികളെയും ശിക്ഷിക്കുവാന്‍ വേണ്ടിയുമാണത്‌. അവരുടെ മേല്‍ തിന്‍മയുടെ വലയമുണ്ട്‌. അല്ലാഹു അവരുടെ നേരെ കോപിക്കുകയും അവരെ ശപിക്കുകയും, അവര്‍ക്ക് വേണ്ടി നരകം ഒരുക്കിവെക്കുകയും ചെയ്തിരിക്കുന്നു. ചെന്നുചേരാനുള്ള ആ സ്ഥലം എത്രമോശം.
\end{malayalam}}
\flushright{\begin{Arabic}
\quranayah[48][7]
\end{Arabic}}
\flushleft{\begin{malayalam}
അല്ലാഹുവിന്നുള്ളതാകുന്നു ആകാശങ്ങളിലെയും ഭൂമിയിലെയും സൈന്യങ്ങള്‍. അല്ലാഹു പ്രതാപിയും യുക്തിമാനുമായിരിക്കുന്നു.
\end{malayalam}}
\flushright{\begin{Arabic}
\quranayah[48][8]
\end{Arabic}}
\flushleft{\begin{malayalam}
തീര്‍ച്ചയായും നിന്നെ നാം ഒരു സാക്ഷിയായും സന്തോഷവാര്‍ത്ത നല്‍കുന്നവനായും താക്കീതുകാരനായും അയച്ചിരിക്കുന്നു.
\end{malayalam}}
\flushright{\begin{Arabic}
\quranayah[48][9]
\end{Arabic}}
\flushleft{\begin{malayalam}
അല്ലാഹുവിലും അവന്‍റെ റസൂലിലും നിങ്ങള്‍ വിശ്വസിക്കുവാനും അവനെ സഹായിക്കുവാനും ആദരിക്കുവാനും രാവിലെയും വൈകുന്നേരവും നിങ്ങള്‍ അവന്‍റെ മഹത്വം പ്രകീര്‍ത്തിക്കുവാനും വേണ്ടി.
\end{malayalam}}
\flushright{\begin{Arabic}
\quranayah[48][10]
\end{Arabic}}
\flushleft{\begin{malayalam}
തീര്‍ച്ചയായും നിന്നോട് പ്രതിജ്ഞ ചെയ്യുന്നവര്‍ അല്ലാഹുവോട് തന്നെയാണ് പ്രതിജ്ഞ ചെയ്യുന്നത്‌. അല്ലാഹുവിന്‍റെ കൈ അവരുടെ കൈകള്‍ക്കു മീതെയുണ്ട്‌. അതിനാല്‍ ആരെങ്കിലും (അത്‌) ലംഘിക്കുന്ന പക്ഷം ലംഘിക്കുന്നതിന്‍റെ ദോഷഫലം അവന് തന്നെയാകുന്നു. താന്‍ അല്ലാഹുവുമായി ഉടമ്പടിയില്‍ ഏര്‍പെട്ട കാര്യം വല്ലവനും നിറവേറ്റിയാല്‍ അവന്ന് മഹത്തായ പ്രതിഫലം നല്‍കുന്നതാണ്‌.
\end{malayalam}}
\flushright{\begin{Arabic}
\quranayah[48][11]
\end{Arabic}}
\flushleft{\begin{malayalam}
ഗ്രാമീണ അറബികളില്‍ നിന്ന് പിന്നോക്കം മാറി നിന്നവര്‍ നിന്നോട് പറഞ്ഞേക്കും: ഞങ്ങളുടെ സ്വത്തുക്കളുടെയും കുടുംബങ്ങളുടെയും കാര്യം ഞങ്ങളെ (നിങ്ങളോടൊപ്പം വരാന്‍ പറ്റാത്ത വിധം) വ്യാപൃതരാക്കികളഞ്ഞു. അത് കൊണ്ട് താങ്കള്‍ ഞങ്ങള്‍ക്കു പാപമോചനത്തിനായി പ്രാര്‍ത്ഥിക്കണം. അവരുടെ നാവുകള്‍ കൊണ്ട് അവര്‍ പറയുന്നത് അവരുടെ ഹൃദയങ്ങളിലുള്ളതല്ലാത്ത കാര്യമാണ്‌. നീ പറയുക: അപ്പോള്‍ അല്ലാഹു നിങ്ങള്‍ക്കു വല്ല ഉപദ്രവവും ചെയ്യാന്‍ ഉദ്ദേശിച്ചാല്‍ അല്ലെങ്കില്‍ അവന്‍ നിങ്ങള്‍ക്ക് വല്ല ഉപകാരവും ചെയ്യാന്‍ ഉദ്ദേശിച്ചാല്‍ അവന്‍റെ പക്കല്‍ നിന്ന് നിങ്ങള്‍ക്കു വല്ലതും അധീനപ്പെടുത്തിത്തരാന്‍ ആരുണ്ട്‌? അല്ല, നിങ്ങള്‍ പ്രവര്‍ത്തിക്കുന്നതിനെ പറ്റി അല്ലാഹു സൂക്ഷ്മമായി അറിയുന്നവനാകുന്നു.
\end{malayalam}}
\flushright{\begin{Arabic}
\quranayah[48][12]
\end{Arabic}}
\flushleft{\begin{malayalam}
അല്ല, റസൂലും സത്യവിശ്വാസികളും ഒരിക്കലും അവരുടെ കുടുംബങ്ങളിലേക്ക് തിരിച്ചെത്തുകയേ ഇല്ല എന്ന് നിങ്ങള്‍ വിചാരിച്ചു. നിങ്ങളുടെ ഹൃദയങ്ങളില്‍ അത് അലംകൃതമായി തോന്നുകയും ചെയ്തു. ദുര്‍വിചാരമാണ് നിങ്ങള്‍ വിചാരിച്ചത്‌. നിങ്ങള്‍ തുലഞ്ഞ ഒരു ജനവിഭാഗമാകുന്നു.
\end{malayalam}}
\flushright{\begin{Arabic}
\quranayah[48][13]
\end{Arabic}}
\flushleft{\begin{malayalam}
അല്ലാഹുവിലും അവന്‍റെ റസൂലിലും വല്ലവനും വിശ്വസിക്കാത്ത പക്ഷം അത്തരം സത്യനിഷേധികള്‍ക്ക് വേണ്ടി നാം ജ്വലിക്കുന്ന നരകാഗ്നി ഒരുക്കിവെച്ചിരിക്കുന്നു.
\end{malayalam}}
\flushright{\begin{Arabic}
\quranayah[48][14]
\end{Arabic}}
\flushleft{\begin{malayalam}
അല്ലാഹുവിന്നാകുന്നു ആകാശങ്ങളുടെയും ഭൂമിയുടെയും ആധിപത്യം. അവന്‍ ഉദ്ദേശിക്കുന്നവര്‍ക്ക് അവന്‍ പൊറുത്തുകൊടുക്കും. അവന്‍ ഉദ്ദേശിക്കുന്നവരെ അവന്‍ ശിക്ഷിക്കുകയും ചെയ്യും. അല്ലാഹു ഏറെ പൊറുക്കുന്നവനും കരുണാനിധിയുമാകുന്നു.
\end{malayalam}}
\flushright{\begin{Arabic}
\quranayah[48][15]
\end{Arabic}}
\flushleft{\begin{malayalam}
സ്വത്തുക്കള്‍ കൈവശപ്പെടുത്താന്‍ ഉള്ളേടത്തേക്ക് നിങ്ങള്‍ (യുദ്ധത്തിന്‌) പോകുകയാണെങ്കില്‍ ആ പിന്നോക്കം മാറി നിന്നവര്‍ പറയും: ഞങ്ങളെ നിങ്ങള്‍ (തടയാതെ) വിട്ടേക്കണം. ഞങ്ങളും നിങ്ങളെ അനുഗമിക്കാം. അല്ലാഹുവിന്‍റെ വാക്കിന് മാറ്റം വരുത്താനാണ് അവര്‍ ഉദ്ദേശിക്കുന്നത്‌. നീ പറയുക: നിങ്ങള്‍ ഒരിക്കലും ഞങ്ങളെ അനുഗമിക്കുകയില്ല, അപ്രകാരമാണ് അല്ലാഹു മുമ്പേ പറഞ്ഞിട്ടുള്ളത്‌. അപ്പോള്‍ അവര്‍ പറഞ്ഞേക്കും; അല്ല, നിങ്ങള്‍ ഞങ്ങളോട് അസൂയ കാണിക്കുകയാണ് എന്ന്‌. അങ്ങനെയല്ല. അവര്‍ (കാര്യം) ഗ്രഹിക്കാതിരിക്കുകയാകുന്നു. അല്‍പം മാത്രമല്ലാതെ.
\end{malayalam}}
\flushright{\begin{Arabic}
\quranayah[48][16]
\end{Arabic}}
\flushleft{\begin{malayalam}
ഗ്രാമീണ അറബികളില്‍ നിന്നും പിന്നോക്കം മാറി നിന്നവരോട് നീ പറയുക: കനത്ത ആക്രമണശേഷിയുള്ള ഒരു ജനവിഭാഗത്തെ നേരിടാനായി നിങ്ങള്‍ വഴിയെ വിളിക്കപ്പെടും.അവര്‍ കീഴടങ്ങുന്നത് വരെ നിങ്ങള്‍ അവരുമായി യുദ്ധം ചെയ്യേണ്ടിവരും. അപ്പോള്‍ നിങ്ങള്‍ അനുസരിക്കുന്ന പക്ഷം അല്ലാഹു നിങ്ങള്‍ക്ക് ഉത്തമമായ പ്രതിഫലം നല്‍കുന്നതാണ്‌. മുമ്പ് നിങ്ങള്‍ പിന്തിരിഞ്ഞു കളഞ്ഞതുപോലെ (ഇനിയും) പിന്തിരിഞ്ഞു കളയുന്ന പക്ഷം വേദനയേറിയ ശിക്ഷ അവന്‍ നിങ്ങള്‍ക്കു നല്‍കുന്നതുമാണ്‌.
\end{malayalam}}
\flushright{\begin{Arabic}
\quranayah[48][17]
\end{Arabic}}
\flushleft{\begin{malayalam}
അന്ധന്‍റെ മേല്‍ കുറ്റമില്ല. മുടന്തന്‍റെ മേലും കുറ്റമില്ല. രോഗിയുടെ മേലും കുറ്റമില്ല. വല്ലവനും അല്ലാഹുവെയും അവന്‍റെ റസൂലിനെയും അനുസരിക്കുന്ന പക്ഷം താഴ്ഭാഗത്ത് കൂടി നദികള്‍ ഒഴുകുന്ന സ്വര്‍ഗത്തോപ്പുകളില്‍ അവനെ പ്രവേശിപ്പിക്കുന്നതാണ്‌. വല്ലവനും പിന്തിരിഞ്ഞു കളയുന്ന പക്ഷം വേദനയേറിയ ശിക്ഷ അവന്നു നല്‍കുന്നതാണ്‌.
\end{malayalam}}
\flushright{\begin{Arabic}
\quranayah[48][18]
\end{Arabic}}
\flushleft{\begin{malayalam}
ആ മരത്തിന്‍റെ ചുവട്ടില്‍ വെച്ച് സത്യവിശ്വാസികള്‍ നിന്നോട് പ്രതിജ്ഞ ചെയ്തിരുന്ന സന്ദര്‍ഭത്തില്‍ തീര്‍ച്ചയായും അല്ലാഹു അവരെ പറ്റി തൃപ്തിപ്പെട്ടിരിക്കുന്നു. അവരുടെ ഹൃദയങ്ങളിലുള്ളത് അവന്‍ അറിയുകയും, അങ്ങനെ അവര്‍ക്ക് മനസ്സമാധാനം ഇറക്കികൊടുക്കുകയും, ആസന്നമായ വിജയം അവര്‍ക്ക് പ്രതിഫലമായി നല്‍കുകയും ചെയ്തു.
\end{malayalam}}
\flushright{\begin{Arabic}
\quranayah[48][19]
\end{Arabic}}
\flushleft{\begin{malayalam}
അവര്‍ക്ക് പിടിച്ചെടുക്കുവാന്‍ ധാരാളം സമരാര്‍ജിത സ്വത്തുകളും (അവന്‍ നല്‍കി) അല്ലാഹു പ്രതാപിയും യുക്തിമാനുമാകുന്നു.
\end{malayalam}}
\flushright{\begin{Arabic}
\quranayah[48][20]
\end{Arabic}}
\flushleft{\begin{malayalam}
നിങ്ങള്‍ക്കു പിടിച്ചെടുക്കാവുന്ന ധാരാളം സമരാര്‍ജിത സ്വത്തുകള്‍ അല്ലാഹു നിങ്ങള്‍ക്ക് വാഗ്ദാനം നല്‍കിയിരിക്കുന്നു. എന്നാല്‍ ഇത് (ഖൈബറിലെ സമരാര്‍ജിത സ്വത്ത്‌) അവന്‍ നിങ്ങള്‍ക്ക് നേരത്തെ തന്നെ തന്നിരിക്കുകയാണ്‌. ജനങ്ങളുടെ കൈകളെ നിങ്ങളില്‍ നിന്ന് അവന്‍ തടയുകയും ചെയ്തിരിക്കുന്നു. സത്യവിശ്വാസികള്‍ക്ക് അതൊരു ദൃഷ്ടാന്തമായിരിക്കുവാനും, നേരായ പാതയിലേക്ക് നിങ്ങളെ അവന്‍ നയിക്കുവാനും വേണ്ടി.
\end{malayalam}}
\flushright{\begin{Arabic}
\quranayah[48][21]
\end{Arabic}}
\flushleft{\begin{malayalam}
നിങ്ങള്‍ക്ക് നേടിയെടുക്കാന്‍ കഴിഞ്ഞിട്ടില്ലാത്ത മറ്റു നേട്ടങ്ങളും (അവന്‍ വാഗ്ദാനം ചെയ്തിരിക്കുന്നു.) അല്ലാഹു അവരെ വലയം ചെയ്തിരിക്കുകയാണ്‌. അല്ലാഹു ഏതു കാര്യത്തിനും കഴിവുള്ളവനാകുന്നു.
\end{malayalam}}
\flushright{\begin{Arabic}
\quranayah[48][22]
\end{Arabic}}
\flushleft{\begin{malayalam}
ആ സത്യനിഷേധികള്‍ നിങ്ങളോട് യുദ്ധത്തില്‍ ഏര്‍പെട്ടിരുന്നെങ്കില്‍ തന്നെ അവര്‍ പിന്തിരിഞ്ഞ് ഓടുമായിരുന്നു. പിന്നീട് ഒരു സംരക്ഷകനെയോ, സഹായിയെയോ അവര്‍ കണ്ടെത്തുകയുമില്ല.
\end{malayalam}}
\flushright{\begin{Arabic}
\quranayah[48][23]
\end{Arabic}}
\flushleft{\begin{malayalam}
മുമ്പു മുതലേ കഴിഞ്ഞുപോന്നിട്ടുള്ള അല്ലാഹുവിന്‍റെ നടപടിക്രമമാകുന്നു അത്‌. അല്ലാഹുവിന്‍റെ നടപടി ക്രമത്തിന് യാതൊരു ഭേദഗതിയും നീ കണ്ടെത്തുകയില്ല.
\end{malayalam}}
\flushright{\begin{Arabic}
\quranayah[48][24]
\end{Arabic}}
\flushleft{\begin{malayalam}
അവര്‍ക്ക് (ശത്രുക്കള്‍ക്ക്‌) എതിരില്‍ നിങ്ങള്‍ക്ക് വിജയം നല്‍കിയതിന് ശേഷം അവനാകുന്നു മക്കയുടെ ഉള്ളില്‍ വെച്ച് അവരുടെ കൈകള്‍ നിങ്ങളില്‍ നിന്നും നിങ്ങളുടെ കൈകള്‍ അവരില്‍ നിന്നും തടഞ്ഞു നിര്‍ത്തിയത്‌. അല്ലാഹു നിങ്ങള്‍ പ്രവര്‍ത്തിക്കുന്നതിനെപ്പറ്റി കണ്ടറിയുന്നവനാകുന്നു.
\end{malayalam}}
\flushright{\begin{Arabic}
\quranayah[48][25]
\end{Arabic}}
\flushleft{\begin{malayalam}
സത്യത്തെ നിഷേധിക്കുകയും, പവിത്രമായ ദേവാലയത്തില്‍ നിന്ന് നിങ്ങളെ തടയുകയും, ബലിമൃഗങ്ങളെ അവയുടെ നിശ്ചിത സ്ഥാനത്തെത്താന്‍ അനുവദിക്കാത്ത നിലയില്‍ തടഞ്ഞുനിര്‍ത്തുകയും ചെയ്തവരാകുന്നു അവര്‍. നിങ്ങള്‍ക്ക് അറിഞ്ഞ് കൂടാത്ത സത്യവിശ്വാസികളായ ചില പുരുഷന്‍മാരെയും സത്യവിശ്വാസിനികളായ ചില സ്ത്രീകളെയും നിങ്ങള്‍ ചവിട്ടിത്തേക്കുകയും, എന്നിട്ട് (നിങ്ങള്‍) അറിയാതെ തന്നെ അവര്‍ നിമിത്തം നിങ്ങള്‍ക്ക് പാപം വന്നു ഭവിക്കാന്‍ ഇടയാവുകയും ചെയ്യില്ലായിരുന്നെങ്കില്‍ (അല്ലാഹു നിങ്ങളെ ഇരുവിഭാഗത്തെയും യുദ്ധത്തില്‍ നിന്ന് തടയുമായിരുന്നില്ല.) അല്ലാഹു തന്‍റെ കാരുണ്യത്തില്‍ താന്‍ ഉദ്ദേശിക്കുന്നവരെ ഉള്‍പെടുത്തേണ്ടതിനായിട്ടാകുന്നു അത്‌. അവര്‍ (മക്കയിലെ വിശ്വാസികളും സത്യനിഷേധികളും) വേറിട്ടായിരുന്നു താമസിച്ചിരുന്നതെങ്കില്‍ അവരിലെ സത്യനിഷേധികള്‍ക്ക് വേദനയേറിയ ശിക്ഷ നാം നല്‍കുക തന്നെ ചെയ്യുമായിരുന്നു.
\end{malayalam}}
\flushright{\begin{Arabic}
\quranayah[48][26]
\end{Arabic}}
\flushleft{\begin{malayalam}
സത്യനിഷേധികള്‍ തങ്ങളുടെ ഹൃദയങ്ങളില്‍ ദുരഭിമാനം- ആ അജ്ഞാനയുഗത്തിന്‍റെ ദുരഭിമാനം -വെച്ചു പുലര്‍ത്തിയ സന്ദര്‍ഭം! അപ്പോള്‍ അല്ലാഹു അവന്‍റെ റസൂലിന്‍റെ മേലും സത്യവിശ്വാസികളുടെ മേലും അവന്‍റെ പക്കല്‍ നിന്നുള്ള മനസ്സമാധാനം ഇറക്കികൊടുത്തു. സൂക്ഷ്മത പാലിക്കാനുള്ള കല്‍പന സ്വീകരിക്കാന്‍ അവരെ നിര്‍ബന്ധിക്കുകയും ചെയ്തു. (അത് സ്വീകരിക്കാന്‍) കൂടുതല്‍ അര്‍ഹതയുള്ളവരും അതിന് അവകാശപ്പെട്ടവരുമായിരുന്നു അവര്‍. അല്ലാഹു ഏത് കാര്യത്തെപ്പറ്റിയും അറിവുള്ളവനായിരിക്കുന്നു.
\end{malayalam}}
\flushright{\begin{Arabic}
\quranayah[48][27]
\end{Arabic}}
\flushleft{\begin{malayalam}
അല്ലാഹു അവന്‍റെ ദൂതന്ന് സ്വപ്നം സത്യപ്രകാരം സാക്ഷാല്‍ക്കരിച്ചിരിക്കുന്നു. അതായത് അല്ലാഹു ഉദ്ദേശിക്കുന്ന പക്ഷം സമാധാനചിത്തരായി കൊണ്ട് തല മുണ്ഡനം ചെയ്തവരും മുടി വെട്ടിയവരും ആയികൊണ്ട് നിങ്ങള്‍ ഒന്നും ഭയപ്പെടാതെ പവിത്രമായ ദേവാലയത്തില്‍ പ്രവേശിക്കുക തന്നെ ചെയ്യുന്നതാണ് എന്ന സ്വപ്നം. എന്നാല്‍ നിങ്ങളറിയാത്തത് അവന്‍ അറിഞ്ഞിട്ടുണ്ട്‌. അതിനാല്‍ അതിന്ന് പുറമെ സമീപസ്ഥമായ ഒരു വിജയം അവന്‍ ഉണ്ടാക്കിത്തന്നു.
\end{malayalam}}
\flushright{\begin{Arabic}
\quranayah[48][28]
\end{Arabic}}
\flushleft{\begin{malayalam}
സന്‍മാര്‍ഗവും സത്യമതവുമായി തന്‍റെ റസൂലിനെ നിയോഗിച്ചത് അവനാകുന്നു. അതിനെ എല്ലാ മതത്തിനും മീതെ തെളിയിച്ചുകാണിക്കാന്‍ വേണ്ടി. സാക്ഷിയായിട്ട് അല്ലാഹു തന്നെ മതി.
\end{malayalam}}
\flushright{\begin{Arabic}
\quranayah[48][29]
\end{Arabic}}
\flushleft{\begin{malayalam}
മുഹമ്മദ് അല്ലാഹുവിന്‍റെ റസൂലാകുന്നു. അദ്ദേഹത്തോടൊപ്പമുള്ളവര്‍ സത്യനിഷേധികളുടെ നേരെ കര്‍ക്കശമായി വര്‍ത്തിക്കുന്നവരാകുന്നു. അവര്‍ അന്യോന്യം ദയാലുക്കളുമാകുന്നു. അല്ലാഹുവിങ്കല്‍ നിന്നുള്ള അനുഗ്രഹവും പ്രീതിയും തേടിക്കൊണ്ട് അവര്‍ കുമ്പിട്ടും സാഷ്ടാംഗം ചെയ്തും നമസ്കരിക്കുന്നതായി നിനക്ക് കാണാം. സുജൂദിന്‍റെ ഫലമായി അവരുടെ അടയാളം അവരുടെ മുഖങ്ങളിലുണ്ട്‌. അതാണ് തൌറാത്തില്‍ അവരെ പറ്റിയുള്ള ഉപമ. ഇന്‍ജീലില്‍ അവരെ പറ്റിയുള്ള ഉപമ ഇങ്ങനെയാകുന്നു: ഒരു വിള, അത് അതിന്‍റെ കൂമ്പ് പുറത്ത് കാണിച്ചു. എന്നിട്ടതിനെ പുഷ്ടിപ്പെടുത്തി. എന്നിട്ടത് കരുത്താര്‍ജിച്ചു. അങ്ങനെ അത് കര്‍ഷകര്‍ക്ക് കൌതുകം തോന്നിച്ചു കൊണ്ട് അതിന്‍റെ കാണ്ഡത്തിന്‍മേല്‍ നിവര്‍ന്നു നിന്നു. (സത്യവിശ്വാസികളെ ഇങ്ങനെ വളര്‍ത്തിക്കൊണ്ട് വരുന്നത്‌) അവര്‍ മൂലം സത്യനിഷേധികളെ അരിശം പിടിപ്പിക്കാന്‍ വേണ്ടിയാകുന്നു. അവരില്‍ നിന്ന് വിശ്വസിക്കുകയും സല്‍കര്‍മ്മങ്ങള്‍ പ്രവര്‍ത്തിക്കുകയും ചെയ്തവര്‍ക്കു അല്ലാഹു പാപമോചനവും മഹത്തായ പ്രതിഫലവും വാഗ്ദാനം ചെയ്തിരിക്കുന്നു.
\end{malayalam}}
\chapter{\textmalayalam{ഹുജുറാത് ( അറകള്‍ )}}
\begin{Arabic}
\Huge{\centerline{\basmalah}}\end{Arabic}
\flushright{\begin{Arabic}
\quranayah[49][1]
\end{Arabic}}
\flushleft{\begin{malayalam}
സത്യവിശ്വാസികളേ, നിങ്ങള്‍ അല്ലാഹുവിന്‍റെയും അവന്‍റെ റസൂലിന്‍റെയും മുമ്പില്‍ (യാതൊന്നും) മുങ്കടന്നു പ്രവര്‍ത്തിക്കരുത്‌. അല്ലാഹുവിനെ നിങ്ങള്‍ സൂക്ഷിക്കുക. തീര്‍ച്ചയായും അല്ലാഹു കേള്‍ക്കുന്നവനും അറിയുന്നവനുമാകുന്നു.
\end{malayalam}}
\flushright{\begin{Arabic}
\quranayah[49][2]
\end{Arabic}}
\flushleft{\begin{malayalam}
സത്യവിശ്വാസികളേ, നിങ്ങളുടെ ശബ്ദങ്ങള്‍ പ്രവാചകന്‍റെ ശബ്ദത്തിന് മീതെ ഉയര്‍ത്തരുത്‌. അദ്ദേഹത്തോട് സംസാരിക്കുമ്പോള്‍ നിങ്ങള്‍ അന്യോന്യം ഒച്ചയിടുന്നത് പോലെ ഒച്ചയിടുകയും ചെയ്യരുത്‌. നിങ്ങളറിയാതെ തന്നെ നിങ്ങളുടെ കര്‍മ്മങ്ങള്‍ നിഷ്ഫലമായി പോകാതിരിക്കാന്‍ വേണ്ടി.
\end{malayalam}}
\flushright{\begin{Arabic}
\quranayah[49][3]
\end{Arabic}}
\flushleft{\begin{malayalam}
തീര്‍ച്ചയായും തങ്ങളുടെ ശബ്ദങ്ങള്‍ അല്ലാഹുവിന്‍റെ റസൂലിന്‍റെ അടുത്ത് താഴ്ത്തുന്നവരാരോ അവരുടെ ഹൃദയങ്ങളാകുന്നു അല്ലാഹു ധപനിഷ്ഠയ്ക്കായി പരീക്ഷിച്ചെടുത്തിട്ടുള്ളത്‌. അവര്‍ക്കാകുന്നു പാപമോചനവും മഹത്തായ പ്രതിഫലവുമുള്ളത്‌.
\end{malayalam}}
\flushright{\begin{Arabic}
\quranayah[49][4]
\end{Arabic}}
\flushleft{\begin{malayalam}
(നീ താമസിക്കുന്ന) അറകള്‍ക്കു പുറത്തു നിന്ന് നിന്നെ വിളിക്കുന്നവരാരോ അവരില്‍ അധികപേരും ചിന്തിച്ചു മനസ്സിലാക്കുന്നില്ല.
\end{malayalam}}
\flushright{\begin{Arabic}
\quranayah[49][5]
\end{Arabic}}
\flushleft{\begin{malayalam}
നീ അവരുടെ അടുത്തേക്കു പുറപ്പെട്ട് ചെല്ലുന്നത് വരെ അവര്‍ ക്ഷമിച്ചിരുന്നെങ്കില്‍ അതായിരുന്നു അവര്‍ക്ക് കൂടുതല്‍ നല്ലത്‌. അല്ലാഹു ഏറെ പൊറുക്കുന്നവനും കരുണാനിധിയുമാകുന്നു.
\end{malayalam}}
\flushright{\begin{Arabic}
\quranayah[49][6]
\end{Arabic}}
\flushleft{\begin{malayalam}
സത്യവിശ്വാസികളേ, ഒരു അധര്‍മ്മകാരി വല്ല വാര്‍ത്തയും കൊണ്ട് നിങ്ങളുടെ അടുത്ത് വന്നാല്‍ നിങ്ങളതിനെപ്പറ്റി വ്യക്തമായി അന്വേഷിച്ചറിയണം. അറിയാതെ ഏതെങ്കിലും ഒരു ജനതയ്ക്ക് നിങ്ങള്‍ ആപത്തുവരുത്തുകയും, എന്നിട്ട് ആ ചെയ്തതിന്‍റെ പേരില്‍ നിങ്ങള്‍ ഖേദക്കാരായിത്തീരുകയും ചെയ്യാതിരിക്കാന്‍ വേണ്ടി.
\end{malayalam}}
\flushright{\begin{Arabic}
\quranayah[49][7]
\end{Arabic}}
\flushleft{\begin{malayalam}
അല്ലാഹുവിന്‍റെ റസൂലാണ് നിങ്ങള്‍ക്കിടയിലുള്ളതെന്ന് നിങ്ങള്‍ മനസ്സിലാക്കണം. പല കാര്യങ്ങളിലും അദ്ദേഹം നിങ്ങളെ അനുസരിച്ചിരുന്നെങ്കില്‍ നിങ്ങള്‍ വിഷമിച്ച് പോകുമായിരുന്നു. എങ്കിലും അല്ലാഹു നിങ്ങള്‍ക്ക് സത്യവിശ്വാസത്തെ പ്രിയങ്കരമാക്കിത്തീര്‍ക്കുകയും, നിങ്ങളുടെ ഹൃദയങ്ങളില്‍ അത് അലംകൃതമായി തോന്നിക്കുകയും ചെയ്തിരിക്കുന്നു. അവിശ്വാസവും അധര്‍മ്മവും അനുസരണക്കേടും നിങ്ങള്‍ക്കവന്‍ അനിഷ്ടകരമാക്കുകയും ചെയ്തിരിക്കുന്നു. അങ്ങനെയുള്ളവരാകുന്നു നേര്‍മാര്‍ഗം സ്വീകരിച്ചവര്‍.
\end{malayalam}}
\flushright{\begin{Arabic}
\quranayah[49][8]
\end{Arabic}}
\flushleft{\begin{malayalam}
അല്ലാഹുവിങ്കല്‍ നിന്നുള്ള ഒരു ഔദാര്യവും അനുഗ്രഹവുമാകുന്നു അത്‌. അല്ലാഹു സര്‍വ്വജ്ഞനും യുക്തിമാനുമാകുന്നു.
\end{malayalam}}
\flushright{\begin{Arabic}
\quranayah[49][9]
\end{Arabic}}
\flushleft{\begin{malayalam}
സത്യവിശ്വാസികളില്‍ നിന്നുള്ള രണ്ടു വിഭാഗങ്ങള്‍ പരസ്പരം പോരടിച്ചാല്‍ നിങ്ങള്‍ അവര്‍ക്കിടയില്‍ രഞ്ജിപ്പുണ്ടാക്കണം. എന്നിട്ടു രണ്ടില്‍ ഒരു വിഭാഗം മറുവിഭാഗത്തിനെതിരില്‍ അതിക്രമം കാണിച്ചാല്‍ അതിക്രമം കാണിക്കുന്ന വിഭാഗത്തോട് അവര്‍ അല്ലാഹുവിന്‍റെ കല്‍പനയിലേക്ക് മടങ്ങിവരുന്നതു വരെ നിങ്ങള്‍ സമരം നടത്തണം. അങ്ങനെ ആ വിഭാഗം മടങ്ങിവരികയാണെങ്കില്‍ നീതിപൂര്‍വ്വം ആ രണ്ടു വിഭാഗങ്ങള്‍ക്കിടയില്‍ രഞ്ജിപ്പുണ്ടാക്കുകയും നിങ്ങള്‍ നീതി പാലിക്കുകയും ചെയ്യുക. തീര്‍ച്ചയായും അല്ലാഹു നീതി പാലിക്കുന്നവരെ ഇഷ്ടപ്പെടുന്നു.
\end{malayalam}}
\flushright{\begin{Arabic}
\quranayah[49][10]
\end{Arabic}}
\flushleft{\begin{malayalam}
സത്യവിശ്വാസികള്‍ (പരസ്പരം) സഹോദരങ്ങള്‍ തന്നെയാകുന്നു. അതിനാല്‍ നിങ്ങളുടെ രണ്ടു സഹോദരങ്ങള്‍ക്കിടയില്‍ നിങ്ങള്‍ രഞ്ജിപ്പുണ്ടാക്കുക. നിങ്ങള്‍ അല്ലാഹുവെ സൂക്ഷിക്കുകയും ചെയ്യുക. നിങ്ങള്‍ക്ക് കാരുണ്യം ലഭിച്ചേക്കാം.
\end{malayalam}}
\flushright{\begin{Arabic}
\quranayah[49][11]
\end{Arabic}}
\flushleft{\begin{malayalam}
സത്യവിശ്വാസികളേ, ഒരു ജനവിഭാഗം മറ്റൊരു ജനവിഭാഗത്തെ പരിഹസിക്കരുത്‌. ഇവര്‍ (പരിഹസിക്കപ്പെടുന്നവര്‍) അവരെക്കാള്‍ നല്ലവരായിരുന്നേക്കാം. ഒരു വിഭാഗം സ്ത്രീകള്‍ മറ്റൊരു വിഭാഗം സ്ത്രീകളെയും പരിഹസിക്കരുത്‌. ഇവര്‍ (പരിഹസിക്കപ്പെടുന്ന സ്ത്രീകള്‍) മറ്റവരെക്കാള്‍ നല്ലവരായിരുന്നേക്കാം. നിങ്ങള്‍ അന്യോന്യം കുത്തുവാക്ക് പറയരുത്‌. നിങ്ങള്‍ പരിഹാസപേരുകള്‍ വിളിച്ച് പരസ്പരം അപമാനിക്കുകയും ചെയ്യരുത്‌. സത്യവിശ്വാസം കൈക്കൊണ്ടതിനു ശേഷം അധാര്‍മ്മികമായ പേര് (വിളിക്കുന്നത്‌) എത്ര ചീത്ത! വല്ലവനും പശ്ചാത്തപിക്കാത്ത പക്ഷം അത്തരക്കാര്‍ തന്നെയാകുന്നു അക്രമികള്‍.
\end{malayalam}}
\flushright{\begin{Arabic}
\quranayah[49][12]
\end{Arabic}}
\flushleft{\begin{malayalam}
സത്യവിശ്വാസികളേ, ഊഹത്തില്‍ മിക്കതും നിങ്ങള്‍ വെടിയുക. തീര്‍ച്ചയായും ഊഹത്തില്‍ ചിലത് കുറ്റമാകുന്നു. നിങ്ങള്‍ ചാരവൃത്തി നടത്തുകയും അരുത്‌. നിങ്ങളില്‍ ചിലര്‍ ചിലരെപ്പറ്റി അവരുടെ അഭാവത്തില്‍ ദുഷിച്ചുപറയുകയും അരുത്‌. തന്‍റെ സഹോദരന്‍ മരിച്ചുകിടക്കുമ്പോള്‍ അവന്‍റെ മാംസം ഭക്ഷിക്കുവാന്‍ നിങ്ങളാരെങ്കിലും ഇഷ്ടപ്പെടുമോ? എന്നാല്‍ അത് (ശവം തിന്നുന്നത്‌) നിങ്ങള്‍ വെറുക്കുകയാണു ചെയ്യുന്നത്‌. അല്ലാഹുവെ നിങ്ങള്‍ സൂക്ഷിക്കുക. തീര്‍ച്ചയായും അല്ലാഹു പശ്ചാത്താപം സ്വീകരിക്കുന്നവനും കരുണാനിധിയുമാകുന്നു.
\end{malayalam}}
\flushright{\begin{Arabic}
\quranayah[49][13]
\end{Arabic}}
\flushleft{\begin{malayalam}
ഹേ; മനുഷ്യരേ, തീര്‍ച്ചയായും നിങ്ങളെ നാം ഒരു ആണില്‍ നിന്നും ഒരു പെണ്ണില്‍ നിന്നുമായി സൃഷ്ടിച്ചിരിക്കുന്നു. നിങ്ങള്‍ അന്യോന്യം അറിയേണ്ടതിന് നിങ്ങളെ നാം വിവിധ സമുദായങ്ങളും ഗോത്രങ്ങളും ആക്കുകയും ചെയ്തിരിക്കുന്നു. തീര്‍ച്ചയായും അല്ലാഹുവിന്‍റെ അടുത്ത് നിങ്ങളില്‍ ഏറ്റവും ആദരണീയന്‍ നിങ്ങളില്‍ ഏറ്റവും ധര്‍മ്മനിഷ്ഠ പാലിക്കുന്നവനാകുന്നു. തീര്‍ച്ചയായും അല്ലാഹു സര്‍വ്വജ്ഞനും സൂക്ഷ്മജ്ഞാനിയുമാകുന്നു.
\end{malayalam}}
\flushright{\begin{Arabic}
\quranayah[49][14]
\end{Arabic}}
\flushleft{\begin{malayalam}
ഗ്രാമീണ അറബികള്‍ പറയുന്നു; ഞങ്ങള്‍ വിശ്വസിച്ചിരിക്കുന്നു എന്ന്‌. നീ പറയുക: നിങ്ങള്‍ വിശ്വസിച്ചിട്ടില്ല. എന്നാല്‍ ഞങ്ങള്‍ കീഴിപെട്ടിരിക്കുന്നു. എന്ന് നിങ്ങള്‍ പറഞ്ഞു കൊള്ളുക. വിശ്വാസം നിങ്ങളുടെ ഹൃദയങ്ങളില്‍ പ്രവേശിച്ചുകഴിഞ്ഞിട്ടില്ല. അല്ലാഹുവെയും അവന്‍റെ ദൂതനെയും നിങ്ങള്‍ അനുസരിക്കുന്ന പക്ഷം നിങ്ങള്‍ക്കു നിങ്ങളുടെ കര്‍മ്മഫലങ്ങളില്‍ നിന്ന് യാതൊന്നും അവന്‍ കുറവ് വരുത്തുകയില്ല. തീര്‍ച്ചയായും അല്ലാഹു ഏറെ പൊറുക്കുന്നവനും കരുണാനിധിയുമാകുന്നു.
\end{malayalam}}
\flushright{\begin{Arabic}
\quranayah[49][15]
\end{Arabic}}
\flushleft{\begin{malayalam}
അല്ലാഹുവിലും അവന്‍റെ ദൂതനിലും വിശ്വസിക്കുകയും പിന്നീട് സംശയിക്കാതിരിക്കുകയും, തങ്ങളുടെ സ്വത്തുക്കളും ശരീരങ്ങളും കൊണ്ട് അല്ലാഹുവിന്‍റെ മാര്‍ഗത്തില്‍ സമരം നടത്തുകയും ചെയ്തവരാരോ അവര്‍ മാത്രമാകുന്നു സത്യവിശ്വാസികള്‍. അവര്‍ തന്നെയാകുന്നു സത്യവാന്‍മാര്‍.
\end{malayalam}}
\flushright{\begin{Arabic}
\quranayah[49][16]
\end{Arabic}}
\flushleft{\begin{malayalam}
നീ പറയുക: നിങ്ങളുടെ മതത്തെപ്പറ്റി നിങ്ങള്‍ അല്ലാഹുവെ പഠിപ്പിക്കുകയാണോ? അല്ലാഹുവാകട്ടെ ആകാശങ്ങളിലുള്ളതും ഭൂമിയിലുള്ളതും അറിയുന്നു. അല്ലാഹു ഏത് കാര്യത്തെപറ്റിയും അറിയുന്നവനാകുന്നു.
\end{malayalam}}
\flushright{\begin{Arabic}
\quranayah[49][17]
\end{Arabic}}
\flushleft{\begin{malayalam}
അവര്‍ ഇസ്ലാം മതം സ്വീകരിച്ചു എന്നത് അവര്‍ നിന്നോട് കാണിച്ച ദാക്ഷിണ്യമായി അവര്‍ എടുത്തുപറയുന്നു. നീ പറയുക: നിങ്ങള്‍ ഇസ്ലാം സ്വീകരിച്ചതിനെ എന്നോട് കാണിച്ച ദാക്ഷിണ്യമായി എടുത്ത് പറയരുത്‌. പ്രത്യുത, സത്യവിശ്വാസത്തിലേക്ക് നിങ്ങള്‍ക്ക് മാര്‍ഗദര്‍ശനം നല്‍കി എന്നത് അല്ലാഹു നിങ്ങളോട് ദാക്ഷിണ്യം കാണിക്കുന്നതാകുന്നു. നിങ്ങള്‍ സത്യവാന്‍മാരാണെങ്കില്‍ (ഇത് നിങ്ങള്‍ അംഗീകരിക്കുക)
\end{malayalam}}
\flushright{\begin{Arabic}
\quranayah[49][18]
\end{Arabic}}
\flushleft{\begin{malayalam}
തീര്‍ച്ചയായും അല്ലാഹു ആകാശങ്ങളിലെയും ഭൂമിയിലെയും അദൃശ്യകാര്യം അറിയുന്നു. നിങ്ങള്‍ പ്രവര്‍ത്തിക്കുന്നത് കണ്ടറിയുന്നവനുമാകുന്നു അല്ലാഹു.
\end{malayalam}}
\chapter{\textmalayalam{ഖാഫ്}}
\begin{Arabic}
\Huge{\centerline{\basmalah}}\end{Arabic}
\flushright{\begin{Arabic}
\quranayah[50][1]
\end{Arabic}}
\flushleft{\begin{malayalam}
ഖാഫ്‌. മഹത്വമേറിയ ഖുര്‍ആന്‍ തന്നെയാണ, സത്യം.
\end{malayalam}}
\flushright{\begin{Arabic}
\quranayah[50][2]
\end{Arabic}}
\flushleft{\begin{malayalam}
എന്നാല്‍ അവരില്‍ നിന്നു തന്നെയുള്ള ഒരു താക്കീതുകാരന്‍ അവരുടെ അടുത്ത് വന്നതിനാല്‍ അവര്‍ ആശ്ചര്യപ്പെട്ടു. എന്നിട്ട് സത്യനിഷേധികള്‍ പറഞ്ഞു: ഇത് അത്ഭുതകരമായ കാര്യമാകുന്നു.
\end{malayalam}}
\flushright{\begin{Arabic}
\quranayah[50][3]
\end{Arabic}}
\flushleft{\begin{malayalam}
നാം മരിച്ച് മണ്ണായിക്കഴിഞ്ഞിട്ടോ (ഒരു പുനര്‍ ജന്‍മം?) അത് വിദൂരമായ ഒരു മടക്കമാകുന്നു.
\end{malayalam}}
\flushright{\begin{Arabic}
\quranayah[50][4]
\end{Arabic}}
\flushleft{\begin{malayalam}
അവരില്‍ നിന്ന് ഭൂമി ചുരുക്കികൊണ്ടിരിക്കുന്നത് നാം അറിഞ്ഞിട്ടുണ്ട്‌; തീര്‍ച്ച നമ്മുടെ അടുക്കല്‍ (വിവരങ്ങള്‍) സൂക്ഷ്മമായി രേഖപ്പെടുത്തിയ ഒരു ഗ്രന്ഥവുമുണ്ട്‌.
\end{malayalam}}
\flushright{\begin{Arabic}
\quranayah[50][5]
\end{Arabic}}
\flushleft{\begin{malayalam}
എന്നാല്‍ സത്യം അവര്‍ക്കു വന്നെത്തിയപ്പോള്‍ അവര്‍ അത് നിഷേധിച്ചു കളഞ്ഞു. അങ്ങനെ അവര്‍ ഇളകികൊണ്ടിരിക്കുന്ന (അനിശ്ചിതമായ) ഒരു നിലപാടിലാകുന്നു.
\end{malayalam}}
\flushright{\begin{Arabic}
\quranayah[50][6]
\end{Arabic}}
\flushleft{\begin{malayalam}
അവര്‍ക്കു മുകളിലുള്ള ആകാശത്തേക്ക് അവര്‍ നോക്കിയിട്ടില്ലേ; എങ്ങനെയാണ് നാം അതിനെ നിര്‍മിക്കുകയും അലങ്കരിക്കുകയും ചെയ്തിട്ടുള്ളതെന്ന്‌? അതിന് വിടവുകളൊന്നുമില്ല.
\end{malayalam}}
\flushright{\begin{Arabic}
\quranayah[50][7]
\end{Arabic}}
\flushleft{\begin{malayalam}
ഭൂമിയാകട്ടെ നാം അതിനെ വികസിപ്പിക്കുകയും, അതില്‍ ഉറച്ചുനില്‍ക്കുന്ന പര്‍വ്വതങ്ങള്‍ നാം സ്ഥാപിക്കുകയും കൌതുകമുള്ള എല്ലാ സസ്യവര്‍ഗങ്ങളും നാം അതില്‍ മുളപ്പിക്കുകയും ചെയ്തിരിക്കുന്നു.
\end{malayalam}}
\flushright{\begin{Arabic}
\quranayah[50][8]
\end{Arabic}}
\flushleft{\begin{malayalam}
(സത്യത്തിലേക്ക്‌) മടങ്ങുന്ന ഏതൊരു ദാസന്നും കണ്ടുമനസ്സിലാക്കുവാനും അനുസ്മരിക്കുവാനും വേണ്ടി.
\end{malayalam}}
\flushright{\begin{Arabic}
\quranayah[50][9]
\end{Arabic}}
\flushleft{\begin{malayalam}
ആകാശത്തുനിന്ന് നാം അനുഗൃഹീതമായ വെള്ളം വര്‍ഷിക്കുകയും, എന്നിട്ട് അതു മൂലം പല തരം തോട്ടങ്ങളും കൊയ്തെടുക്കുന്ന ധാന്യങ്ങളും നാം മുളപ്പിക്കുകയും ചെയ്തു.
\end{malayalam}}
\flushright{\begin{Arabic}
\quranayah[50][10]
\end{Arabic}}
\flushleft{\begin{malayalam}
അടുക്കടുക്കായി കുലകളുള്ള ഉയരമുള്ള ഈന്തപ്പനകളും.
\end{malayalam}}
\flushright{\begin{Arabic}
\quranayah[50][11]
\end{Arabic}}
\flushleft{\begin{malayalam}
(നമ്മുടെ) ദാസന്‍മാര്‍ക്ക് ഉപജീവനമായിട്ടുള്ളതത്രെ അവ. നിര്‍ജീവമായ നാടിനെ അത് മൂലം ജീവനുള്ളതാക്കുകയും ചെയ്തു. അപ്രകാരം തന്നെയാകുന്നു (ഖബ്‌റുകളില്‍ നിന്നുള്ള) പുറപ്പാട്‌.
\end{malayalam}}
\flushright{\begin{Arabic}
\quranayah[50][12]
\end{Arabic}}
\flushleft{\begin{malayalam}
ഇവരുടെ മുമ്പ് നൂഹിന്‍റെ ജനതയും റസ്സുകാരും, ഥമൂദ് സമുദായവും സത്യം നിഷേധിക്കുകയുണ്ടായി.
\end{malayalam}}
\flushright{\begin{Arabic}
\quranayah[50][13]
\end{Arabic}}
\flushleft{\begin{malayalam}
ആദ് സമുദായവും, ഫിര്‍ഔനും, ലൂത്വിന്‍റെ സഹോദരങ്ങളും,
\end{malayalam}}
\flushright{\begin{Arabic}
\quranayah[50][14]
\end{Arabic}}
\flushleft{\begin{malayalam}
മരക്കൂട്ടങ്ങള്‍ക്കിടയില്‍ വസിച്ചിരുന്നവരും, തുബ്ബഇന്‍റെ ജനതയും. ഇവരെല്ലാം ദൈവദൂതന്‍മാരെ നിഷേധിച്ചു തള്ളി. അപ്പോള്‍(അവരില്‍) എന്‍റെ താക്കീത് സത്യമായി പുലര്‍ന്നു.
\end{malayalam}}
\flushright{\begin{Arabic}
\quranayah[50][15]
\end{Arabic}}
\flushleft{\begin{malayalam}
അപ്പോള്‍ ആദ്യതവണ സൃഷ്ടിച്ചതു കൊണ്ട് നാം ക്ഷീണിച്ച് പോയോ? അല്ല, അവര്‍ പുതിയൊരു സൃഷ്ടിപ്പിനെപ്പറ്റി സംശയത്തിലാകുന്നു.
\end{malayalam}}
\flushright{\begin{Arabic}
\quranayah[50][16]
\end{Arabic}}
\flushleft{\begin{malayalam}
തീര്‍ച്ചയായും മനുഷ്യനെ നാം സൃഷ്ടിച്ചിരിക്കുന്നു. അവന്‍റെ മനസ്സ് മന്ത്രിച്ചു കൊണ്ടിരിക്കുന്നത് നാം അറിയുകയും ചെയ്യുന്നു. നാം (അവന്‍റെ) കണ്ഠനാഡി യെക്കാള്‍ അവനോട് അടുത്തവനും ആകുന്നു.
\end{malayalam}}
\flushright{\begin{Arabic}
\quranayah[50][17]
\end{Arabic}}
\flushleft{\begin{malayalam}
വലതുഭാഗത്തും ഇടതുഭാഗത്തും ഇരുന്നു കൊണ്ട് ഏറ്റുവാങ്ങുന്ന രണ്ടുപേര്‍ ഏറ്റുവാങ്ങുന്ന സന്ദര്‍ഭം.
\end{malayalam}}
\flushright{\begin{Arabic}
\quranayah[50][18]
\end{Arabic}}
\flushleft{\begin{malayalam}
അവന്‍ ഏതൊരു വാക്ക് ഉച്ചരിക്കുമ്പോഴും അവന്‍റെ അടുത്ത് തയ്യാറായി നില്‍ക്കുന്ന നിരീക്ഷകന്‍ ഉണ്ടാവാതിരിക്കുകയില്ല.
\end{malayalam}}
\flushright{\begin{Arabic}
\quranayah[50][19]
\end{Arabic}}
\flushleft{\begin{malayalam}
മരണവെപ്രാളം യാഥാര്‍ത്ഥ്യവും കൊണ്ട് വരുന്നതാണ്‌. എന്തൊന്നില്‍ നിന്ന് നീ ഒഴിഞ്ഞ് മാറികൊണ്ടിരിക്കുന്നുവോ അതത്രെ ഇത്‌.
\end{malayalam}}
\flushright{\begin{Arabic}
\quranayah[50][20]
\end{Arabic}}
\flushleft{\begin{malayalam}
കാഹളത്തില്‍ ഊതപ്പെടുകയും ചെയ്യും. അതാകുന്നു താക്കീതിന്‍റെ ദിവസം.
\end{malayalam}}
\flushright{\begin{Arabic}
\quranayah[50][21]
\end{Arabic}}
\flushleft{\begin{malayalam}
കൂടെ ഒരു ആനയിക്കുന്നവനും ഒരു സാക്ഷിയുമുള്ള നിലയിലായിരിക്കും ഏതൊരാളും (അന്ന്‌) വരുന്നത്‌.
\end{malayalam}}
\flushright{\begin{Arabic}
\quranayah[50][22]
\end{Arabic}}
\flushleft{\begin{malayalam}
(അന്ന് സത്യനിഷേധിയോടു പറയപ്പെടും:) തീര്‍ച്ചയായും നീ ഇതിനെപ്പറ്റി അശ്രദ്ധയിലായിരുന്നു. എന്നാല്‍ ഇപ്പോള്‍ നിന്നില്‍ നിന്ന് നിന്‍റെ ആ മൂടി നാം നീക്കം ചെയ്തിരിക്കുന്നു. അങ്ങനെ നിന്‍റെ ദൃഷ്ടി ഇന്ന് മൂര്‍ച്ചയുള്ളതാകുന്നു.
\end{malayalam}}
\flushright{\begin{Arabic}
\quranayah[50][23]
\end{Arabic}}
\flushleft{\begin{malayalam}
അവന്‍റെ സഹചാരി (മലക്ക്‌) പറയും: ഇതാകുന്നു എന്‍റെ പക്കല്‍ തയ്യാറുള്ളത് (രേഖ)
\end{malayalam}}
\flushright{\begin{Arabic}
\quranayah[50][24]
\end{Arabic}}
\flushleft{\begin{malayalam}
(അല്ലാഹു മലക്കുകളോട് കല്‍പിക്കും:) സത്യനിഷേധിയും ധിക്കാരിയുമായിട്ടുള്ള ഏതൊരുത്തനെയും നിങ്ങള്‍ നരകത്തില്‍ ഇട്ടേക്കുക.
\end{malayalam}}
\flushright{\begin{Arabic}
\quranayah[50][25]
\end{Arabic}}
\flushleft{\begin{malayalam}
അതായത് നന്‍മയെ മുടക്കുന്നവനും അതിക്രമകാരിയും സംശയാലുവുമായ ഏതൊരുത്തനെയും.
\end{malayalam}}
\flushright{\begin{Arabic}
\quranayah[50][26]
\end{Arabic}}
\flushleft{\begin{malayalam}
അതെ, അല്ലാഹുവോടൊപ്പം വേറെ ദൈവത്തെ സ്ഥാപിച്ച ഏതൊരുവനെയും. അതിനാല്‍ കഠിനമായ ശിക്ഷയില്‍ അവനെ നിങ്ങള്‍ ഇട്ടേക്കുക.
\end{malayalam}}
\flushright{\begin{Arabic}
\quranayah[50][27]
\end{Arabic}}
\flushleft{\begin{malayalam}
അവന്‍റെ കൂട്ടാളിപറയും: ഞങ്ങളുടെ രക്ഷിതാവേ! ഞാനവനെ വഴിതെറ്റിച്ചിട്ടില്ല. പക്ഷെ, അവന്‍ വിദൂരമായ ദുര്‍മാര്‍ഗത്തിലായിരുന്നു.
\end{malayalam}}
\flushright{\begin{Arabic}
\quranayah[50][28]
\end{Arabic}}
\flushleft{\begin{malayalam}
അവന്‍ (അല്ലാഹു) പറയും: നിങ്ങള്‍ എന്‍റെ അടുക്കല്‍ തര്‍ക്കിക്കേണ്ട. മുമ്പേ ഞാന്‍ നിങ്ങള്‍ക്ക് താക്കീത് നല്‍കിയിട്ടുണ്ട്‌.
\end{malayalam}}
\flushright{\begin{Arabic}
\quranayah[50][29]
\end{Arabic}}
\flushleft{\begin{malayalam}
എന്‍റെ അടുക്കല്‍ വാക്ക് മാറ്റപ്പെടുകയില്ല. ഞാന്‍ ദാസന്‍മാരോട് ഒട്ടും അനീതി കാണിക്കുന്നവനുമല്ല.
\end{malayalam}}
\flushright{\begin{Arabic}
\quranayah[50][30]
\end{Arabic}}
\flushleft{\begin{malayalam}
നീ നിറഞ്ഞ് കഴിഞ്ഞോ എന്ന് നാം നരകത്തോട് പറയുകയും, കൂടുതല്‍ എന്തെങ്കിലുമുണ്ടോ എന്ന് അത് (നരകം) പറയുകയും ചെയ്യുന്ന ദിവസത്തിലത്രെ അത്‌.
\end{malayalam}}
\flushright{\begin{Arabic}
\quranayah[50][31]
\end{Arabic}}
\flushleft{\begin{malayalam}
സൂക്ഷ്മത പാലിക്കുന്നവര്‍ക്ക് അകലെയല്ലാത്ത വിധത്തില്‍ സ്വര്‍ഗം അടുത്തു കൊണ്ടു വരപ്പെടുന്നതാണ്‌.
\end{malayalam}}
\flushright{\begin{Arabic}
\quranayah[50][32]
\end{Arabic}}
\flushleft{\begin{malayalam}
(അവരോട് പറയപ്പെടും:) അല്ലാഹുവിങ്കലേക്ക് ഏറ്റവും അധികം മടങ്ങുന്നവനും, (ജീവിതം) കാത്തുസൂക്ഷിക്കുന്നവനും ആയ ഏതൊരാള്‍ക്കും നല്‍കാമെന്ന് നിങ്ങളോട് വാഗ്ദാനം ചെയ്യപ്പെട്ടിരുന്നതാകുന്നു ഇത്‌.
\end{malayalam}}
\flushright{\begin{Arabic}
\quranayah[50][33]
\end{Arabic}}
\flushleft{\begin{malayalam}
അതായത് അദൃശ്യമായ നിലയില്‍ പരമകാരുണികനെ ഭയപ്പെടുകയും താഴ്മയുള്ള ഹൃദയത്തോട് കൂടി വരുകയും ചെയ്തവന്ന്‌.
\end{malayalam}}
\flushright{\begin{Arabic}
\quranayah[50][34]
\end{Arabic}}
\flushleft{\begin{malayalam}
(അവരോട് പറയപ്പെടും:) സമാധാനപൂര്‍വ്വം നിങ്ങളതില്‍ പ്രവേശിച്ച് കൊള്ളുക. ശാശ്വതവാസത്തിനുള്ള ദിവസമാകുന്നു അത്‌.
\end{malayalam}}
\flushright{\begin{Arabic}
\quranayah[50][35]
\end{Arabic}}
\flushleft{\begin{malayalam}
അവര്‍ക്കവിടെ ഉദ്ദേശിക്കുന്നതെന്തും ഉണ്ടായിരിക്കും. നമ്മുടെ പക്കലാകട്ടെ കൂടുതലായി പലതുമുണ്ട്‌.
\end{malayalam}}
\flushright{\begin{Arabic}
\quranayah[50][36]
\end{Arabic}}
\flushleft{\begin{malayalam}
ഇവര്‍ക്കു മുമ്പ് എത്ര തലമുറകളെ നാം നശിപ്പിച്ചിട്ടുണ്ട്‌! അവര്‍ ഇവരെക്കാള്‍ കടുത്ത കൈയ്യൂക്കുള്ളവരായിരുന്നു. എന്നിട്ടവര്‍ നാടുകളിലാകെ ചികഞ്ഞു നോക്കി; രക്ഷപ്രാപിക്കാന്‍ വല്ല ഇടവുമുണ്ടോ എന്ന്‌.
\end{malayalam}}
\flushright{\begin{Arabic}
\quranayah[50][37]
\end{Arabic}}
\flushleft{\begin{malayalam}
ഹൃദയമുള്ളവനായിരിക്കുകയോ, മനസ്സാന്നിധ്യത്തോടെ ചെവികൊടുത്ത് കേള്‍ക്കുകയോ ചെയ്തവന്ന് തീര്‍ച്ചയായും അതില്‍ ഒരു ഉല്‍ബോധനമുണ്ട്‌.
\end{malayalam}}
\flushright{\begin{Arabic}
\quranayah[50][38]
\end{Arabic}}
\flushleft{\begin{malayalam}
ആകാശങ്ങളും ഭൂമിയും അവയ്ക്കിടയിലുള്ളതും നാം ആറു ദിവസങ്ങളില്‍ സൃഷ്ടിച്ചിരിക്കുന്നു. യാതൊരു ക്ഷീണവും നമ്മെ ബാധിച്ചിട്ടുമില്ല.
\end{malayalam}}
\flushright{\begin{Arabic}
\quranayah[50][39]
\end{Arabic}}
\flushleft{\begin{malayalam}
അതിനാല്‍ അവര്‍ പറയുന്നതിന്‍റെ പേരില്‍ നീ ക്ഷമിച്ചു കൊള്ളുക. സൂര്യോദയത്തിനു മുമ്പും അസ്തമനത്തിനുമുമ്പും നിന്‍റെ രക്ഷിതാവിനെ സ്തുതിക്കുന്നതോടൊപ്പം (അവനെ) പ്രകീര്‍ത്തിക്കുകയും ചെയ്യുക.
\end{malayalam}}
\flushright{\begin{Arabic}
\quranayah[50][40]
\end{Arabic}}
\flushleft{\begin{malayalam}
രാത്രിയില്‍ നിന്ന് കുറച്ചു സമയവും അവനെ പ്രകീര്‍ത്തിക്കുക. സാഷ്ടാംഗ നമസ്കാരത്തിനു ശേഷമുള്ള സമയങ്ങളിലും.
\end{malayalam}}
\flushright{\begin{Arabic}
\quranayah[50][41]
\end{Arabic}}
\flushleft{\begin{malayalam}
അടുത്ത ഒരു സ്ഥലത്ത് നിന്ന് വിളിച്ചുപറയുന്നവന്‍ വിളിച്ചുപറയുന്ന ദിവസത്തെപ്പറ്റി ശ്രദ്ധിച്ചു കേള്‍ക്കുക.
\end{malayalam}}
\flushright{\begin{Arabic}
\quranayah[50][42]
\end{Arabic}}
\flushleft{\begin{malayalam}
അതായത് ആ ഘോരശബ്ദം യഥാര്‍ത്ഥമായും അവര്‍ കേള്‍ക്കുന്ന ദിവസം. അതത്രെ (ഖബ്‌റുകളില്‍ നിന്നുള്ള) പുറപ്പാടിന്‍റെ ദിവസം.
\end{malayalam}}
\flushright{\begin{Arabic}
\quranayah[50][43]
\end{Arabic}}
\flushleft{\begin{malayalam}
തീര്‍ച്ചയായും നാം ജീവിപ്പിക്കുകയും മരിപ്പിക്കുകയും ചെയ്യുന്നു. നമ്മുടെ അടുത്തേക്ക് തന്നെയാണ് തിരിച്ചെത്തലും.
\end{malayalam}}
\flushright{\begin{Arabic}
\quranayah[50][44]
\end{Arabic}}
\flushleft{\begin{malayalam}
അവരെ വിട്ടു ഭൂമി പിളര്‍ന്ന് മാറിയിട്ട് അവര്‍ അതിവേഗം വരുന്ന ദിവസം! അത് നമ്മെ സംബന്ധിച്ചിടത്തോളം എളുപ്പമുള്ള ഒരു ഒരുമിച്ചുകൂട്ടലാകുന്നു.
\end{malayalam}}
\flushright{\begin{Arabic}
\quranayah[50][45]
\end{Arabic}}
\flushleft{\begin{malayalam}
അവര്‍ പറഞ്ഞ് കൊണ്ടിരിക്കുന്നതിനെ പറ്റി നാം നല്ലവണ്ണം അറിയുന്നവനാകുന്നു. നീ അവരുടെ മേല്‍ സ്വേച്ഛാധികാരം ചെലുത്തേണ്ടവനല്ല. അതിനാല്‍ എന്‍റെ താക്കീത് ഭയപ്പെടുന്നവരെ ഖുര്‍ആന്‍ മുഖേന നീ ഉല്‍ബോധിപ്പിക്കുക.
\end{malayalam}}
\chapter{\textmalayalam{ദാരിയാത് ( വിതറുന്നവ )}}
\begin{Arabic}
\Huge{\centerline{\basmalah}}\end{Arabic}
\flushright{\begin{Arabic}
\quranayah[51][1]
\end{Arabic}}
\flushleft{\begin{malayalam}
ശക്തിയായി (പൊടി) വിതറിക്കൊണ്ടിരിക്കുന്നവ (കാറ്റുകള്‍) തന്നെയാണ, സത്യം.
\end{malayalam}}
\flushright{\begin{Arabic}
\quranayah[51][2]
\end{Arabic}}
\flushleft{\begin{malayalam}
(ജല) ഭാരം വഹിക്കുന്ന (മേഘങ്ങള്‍) തന്നെയാണ, സത്യം.
\end{malayalam}}
\flushright{\begin{Arabic}
\quranayah[51][3]
\end{Arabic}}
\flushleft{\begin{malayalam}
നിഷ്പ്രയാസം സഞ്ചരിക്കുന്നവ (കപ്പലുകള്‍) തന്നെയാണ, സത്യം!
\end{malayalam}}
\flushright{\begin{Arabic}
\quranayah[51][4]
\end{Arabic}}
\flushleft{\begin{malayalam}
കാര്യങ്ങള്‍ വിഭജിച്ചു കൊടുക്കുന്നവര്‍ (മലക്കുകള്‍) തന്നെയാണ, സത്യം.
\end{malayalam}}
\flushright{\begin{Arabic}
\quranayah[51][5]
\end{Arabic}}
\flushleft{\begin{malayalam}
തീര്‍ച്ചയായും നിങ്ങള്‍ക്കു താക്കീത് നല്‍കപ്പെടുന്ന കാര്യം സത്യമായിട്ടുള്ളത് തന്നെയാകുന്നു.
\end{malayalam}}
\flushright{\begin{Arabic}
\quranayah[51][6]
\end{Arabic}}
\flushleft{\begin{malayalam}
തീര്‍ച്ചയായും ന്യായവിധി സംഭവിക്കുന്നതു തന്നെയാകുന്നു.
\end{malayalam}}
\flushright{\begin{Arabic}
\quranayah[51][7]
\end{Arabic}}
\flushleft{\begin{malayalam}
വിവിധ പഥങ്ങളുള്ള ആകാശം തന്നെയാണ,സത്യം.
\end{malayalam}}
\flushright{\begin{Arabic}
\quranayah[51][8]
\end{Arabic}}
\flushleft{\begin{malayalam}
തീര്‍ച്ചയായും നിങ്ങള്‍ വിഭിന്നമായ അഭിപ്രായത്തിലാകുന്നു.
\end{malayalam}}
\flushright{\begin{Arabic}
\quranayah[51][9]
\end{Arabic}}
\flushleft{\begin{malayalam}
(സത്യത്തില്‍ നിന്ന്‌) തെറ്റിക്കപ്പെട്ടവന്‍ അതില്‍ നിന്ന് (ഖുര്‍ആനില്‍ നിന്ന്‌) തെറ്റിക്കപ്പെടുന്നു.
\end{malayalam}}
\flushright{\begin{Arabic}
\quranayah[51][10]
\end{Arabic}}
\flushleft{\begin{malayalam}
ഊഹാപോഹക്കാര്‍ ശപിക്കപ്പെടട്ടെ.
\end{malayalam}}
\flushright{\begin{Arabic}
\quranayah[51][11]
\end{Arabic}}
\flushleft{\begin{malayalam}
വിവരക്കേടില്‍ മതിമറന്നു കഴിയുന്നവര്‍
\end{malayalam}}
\flushright{\begin{Arabic}
\quranayah[51][12]
\end{Arabic}}
\flushleft{\begin{malayalam}
ന്യായവിധിയുടെ നാള്‍ എപ്പോഴായിരിക്കും എന്നവര്‍ ചോദിക്കുന്നു.
\end{malayalam}}
\flushright{\begin{Arabic}
\quranayah[51][13]
\end{Arabic}}
\flushleft{\begin{malayalam}
നരകാഗ്നിയില്‍ അവര്‍ പരീക്ഷണത്തിന് വിധേയരാകുന്ന ദിവസമത്രെ അത്‌.
\end{malayalam}}
\flushright{\begin{Arabic}
\quranayah[51][14]
\end{Arabic}}
\flushleft{\begin{malayalam}
(അവരോട് പറയപ്പെടും:) നിങ്ങള്‍ക്കുള്ള പരീക്ഷണം നിങ്ങള്‍ അനുഭവിച്ച് കൊള്ളുവിന്‍. നിങ്ങള്‍ എന്തൊന്നിന് ധൃതികൂട്ടിക്കൊണ്ടിരുന്നുവോ അതത്രെ ഇത്‌.
\end{malayalam}}
\flushright{\begin{Arabic}
\quranayah[51][15]
\end{Arabic}}
\flushleft{\begin{malayalam}
തീര്‍ച്ചയായും സൂക്ഷ്മത പാലിക്കുന്നവര്‍ സ്വര്‍ഗത്തോപ്പുകളിലും അരുവികളിലുമായിരിക്കും.
\end{malayalam}}
\flushright{\begin{Arabic}
\quranayah[51][16]
\end{Arabic}}
\flushleft{\begin{malayalam}
അവര്‍ക്ക് അവരുടെ രക്ഷിതാവ് നല്‍കിയത് ഏറ്റുവാങ്ങിക്കൊണ്ട്‌. തീര്‍ച്ചയായും അവര്‍ അതിനു മുമ്പ് സദ്‌വൃത്തരായിരുന്നു.
\end{malayalam}}
\flushright{\begin{Arabic}
\quranayah[51][17]
\end{Arabic}}
\flushleft{\begin{malayalam}
രാത്രിയില്‍ നിന്ന് അല്‍പഭാഗമേ അവര്‍ ഉറങ്ങാറുണ്ടായിരുന്നുള്ളൂ.
\end{malayalam}}
\flushright{\begin{Arabic}
\quranayah[51][18]
\end{Arabic}}
\flushleft{\begin{malayalam}
രാത്രിയുടെ അന്ത്യവേളകളില്‍ അവര്‍ പാപമോചനം തേടുന്നവരായിരുന്നു.
\end{malayalam}}
\flushright{\begin{Arabic}
\quranayah[51][19]
\end{Arabic}}
\flushleft{\begin{malayalam}
അവരുടെ സ്വത്തുക്കളിലാകട്ടെ ചോദിക്കുന്നവന്നും (ഉപജീവനം) തടയപ്പെട്ടവന്നും ഒരു അവകാശമുണ്ടായിരിക്കുകയും ചെയ്യും.
\end{malayalam}}
\flushright{\begin{Arabic}
\quranayah[51][20]
\end{Arabic}}
\flushleft{\begin{malayalam}
ദൃഢവിശ്വാസമുള്ളവര്‍ക്ക് ഭൂമിയില്‍ പല ദൃഷ്ടാന്തങ്ങളുമുണ്ട്‌.
\end{malayalam}}
\flushright{\begin{Arabic}
\quranayah[51][21]
\end{Arabic}}
\flushleft{\begin{malayalam}
നിങ്ങളില്‍ തന്നെയും (പല ദൃഷ്ടാന്തങ്ങളുണ്ട്‌.)എന്നിട്ട് നിങ്ങള്‍ കണ്ടറിയുന്നില്ലെ?
\end{malayalam}}
\flushright{\begin{Arabic}
\quranayah[51][22]
\end{Arabic}}
\flushleft{\begin{malayalam}
ആകാശത്ത് നിങ്ങള്‍ക്കുള്ള ഉപജീവനവും, നിങ്ങളോട് വാഗ്ദാനം ചെയ്യപ്പെടുന്ന കാര്യങ്ങളുമുണ്ട്‌.
\end{malayalam}}
\flushright{\begin{Arabic}
\quranayah[51][23]
\end{Arabic}}
\flushleft{\begin{malayalam}
എന്നാല്‍ ആകാശത്തിന്‍റെയും ഭൂമിയുടെയും രക്ഷിതാവിനെ തന്നെയാണ, സത്യം! നിങ്ങള്‍ സംസാരിക്കുന്നു എന്നതു പോലെ തീര്‍ച്ചയായും ഇത് സത്യമാകുന്നു.
\end{malayalam}}
\flushright{\begin{Arabic}
\quranayah[51][24]
\end{Arabic}}
\flushleft{\begin{malayalam}
ഇബ്രാഹീമിന്‍റെ മാന്യരായ അതിഥികളെ പറ്റിയുള്ള വാര്‍ത്ത നിനക്ക് വന്നുകിട്ടിയിട്ടുണ്ടോ?
\end{malayalam}}
\flushright{\begin{Arabic}
\quranayah[51][25]
\end{Arabic}}
\flushleft{\begin{malayalam}
അവര്‍ അദ്ദേഹത്തിന്‍റെ അടുത്തു കടന്നു വന്നിട്ട് സലാം പറഞ്ഞ സമയത്ത് അദ്ദേഹം പറഞ്ഞു: സലാം (നിങ്ങള്‍) അപരിചിതരായ ആളുകളാണല്ലോ.
\end{malayalam}}
\flushright{\begin{Arabic}
\quranayah[51][26]
\end{Arabic}}
\flushleft{\begin{malayalam}
അനന്തരം അദ്ദേഹം ധൃതിയില്‍ തന്‍റെ ഭാര്യയുടെ അടുത്തേക്ക് ചെന്നു. എന്നിട്ട് ഒരു തടിച്ച കാളക്കുട്ടിയെ (വേവിച്ചു) കൊണ്ടുവന്നു.
\end{malayalam}}
\flushright{\begin{Arabic}
\quranayah[51][27]
\end{Arabic}}
\flushleft{\begin{malayalam}
എന്നിട്ട് അത് അവരുടെ അടുത്തേക്ക് വെച്ചു. അദ്ദേഹം പറഞ്ഞു: നിങ്ങള്‍ തിന്നുന്നില്ലേ?
\end{malayalam}}
\flushright{\begin{Arabic}
\quranayah[51][28]
\end{Arabic}}
\flushleft{\begin{malayalam}
അപ്പോള്‍ അവരെപ്പറ്റി അദ്ദേഹത്തിന്‍റെ മനസ്സില്‍ ഭയം കടന്നു കൂടി. അവര്‍ പറഞ്ഞു: താങ്കള്‍ ഭയപ്പെടേണ്ട. അദ്ദേഹത്തിന് ജ്ഞാനിയായ ഒരു ആണ്‍കുട്ടിയെ പറ്റി അവര്‍ സന്തോഷവാര്‍ത്ത അറിയിക്കുകയും ചെയ്തു.
\end{malayalam}}
\flushright{\begin{Arabic}
\quranayah[51][29]
\end{Arabic}}
\flushleft{\begin{malayalam}
അപ്പോള്‍ അദ്ദേഹത്തിന്‍റെ ഭാര്യ ഉച്ചത്തില്‍ ഒരു ശബ്ദമുണ്ടാക്കിക്കൊണ്ട് മുന്നോട്ട് വന്നു. എന്നിട്ട് തന്‍റെ മുഖത്തടിച്ചുകൊണ്ട് പറഞ്ഞു: വന്ധ്യയായ ഒരു കിഴവിയാണോ (പ്രസവിക്കാന്‍ പോകുന്നത്‌?)
\end{malayalam}}
\flushright{\begin{Arabic}
\quranayah[51][30]
\end{Arabic}}
\flushleft{\begin{malayalam}
അവര്‍ (ദൂതന്‍മാര്‍) പറഞ്ഞു: അപ്രകാരം തന്നെയാകുന്നു നിന്‍റെ രക്ഷിതാവ് പറഞ്ഞിരിക്കുന്നത്‌. തീര്‍ച്ചയായും അവന്‍ തന്നെയാകുന്നു യുക്തിമാനും ജ്ഞാനിയും ആയിട്ടുള്ളവന്‍.
\end{malayalam}}
\flushright{\begin{Arabic}
\quranayah[51][31]
\end{Arabic}}
\flushleft{\begin{malayalam}
അദ്ദേഹം ചോദിച്ചു: ഹേ; ദൂതന്‍മാരേ, അപ്പോള്‍ നിങ്ങളുടെ കാര്യമെന്താണ്‌?
\end{malayalam}}
\flushright{\begin{Arabic}
\quranayah[51][32]
\end{Arabic}}
\flushleft{\begin{malayalam}
അവര്‍ പറഞ്ഞു: ഞങ്ങള്‍ കുറ്റവാളികളായ ഒരു ജനതയിലേക്ക് അയക്കപ്പെട്ടതാകുന്നു
\end{malayalam}}
\flushright{\begin{Arabic}
\quranayah[51][33]
\end{Arabic}}
\flushleft{\begin{malayalam}
കളിമണ്ണുകൊണ്ടുള്ള കല്ലുകള്‍ ഞങ്ങള്‍ അവരുടെ നേരെ അയക്കുവാന്‍ വേണ്ടി.
\end{malayalam}}
\flushright{\begin{Arabic}
\quranayah[51][34]
\end{Arabic}}
\flushleft{\begin{malayalam}
അതിക്രമകാരികള്‍ക്ക് വേണ്ടി തങ്ങളുടെ രക്ഷിതാവിന്‍റെ അടുക്കല്‍ അടയാളപ്പെടുത്തിയ (കല്ലുകള്‍)
\end{malayalam}}
\flushright{\begin{Arabic}
\quranayah[51][35]
\end{Arabic}}
\flushleft{\begin{malayalam}
അപ്പോള്‍ സത്യവിശ്വാസികളില്‍ പെട്ടവരായി അവിടെ ഉണ്ടായിരുന്നവരെ നാം പുറത്ത് കൊണ്ടു വന്നു.(രക്ഷപെടുത്തി.)
\end{malayalam}}
\flushright{\begin{Arabic}
\quranayah[51][36]
\end{Arabic}}
\flushleft{\begin{malayalam}
എന്നാല്‍ മുസ്ലിംകളുടെതായ ഒരു വീടല്ലാതെ നാം അവിടെ കണ്ടെത്തിയില്ല.
\end{malayalam}}
\flushright{\begin{Arabic}
\quranayah[51][37]
\end{Arabic}}
\flushleft{\begin{malayalam}
വേദനയേറിയ ശിക്ഷ ഭയപ്പെടുന്നവര്‍ക്ക് ഒരു ദൃഷ്ടാന്തം നാം അവിടെ അവശേഷിപ്പിക്കുകയും ചെയ്തു.
\end{malayalam}}
\flushright{\begin{Arabic}
\quranayah[51][38]
\end{Arabic}}
\flushleft{\begin{malayalam}
മൂസായുടെ ചരിത്രത്തിലുമുണ്ട് (ദൃഷ്ടാന്തങ്ങള്‍) വ്യക്തമായ ആധികാരിക പ്രമാണവുമായി ഫിര്‍ഔന്‍റെ അടുത്തേക്ക് നാം അദ്ദേഹത്തെ നിയോഗിച്ച സന്ദര്‍ഭം.
\end{malayalam}}
\flushright{\begin{Arabic}
\quranayah[51][39]
\end{Arabic}}
\flushleft{\begin{malayalam}
അപ്പോള്‍ അവന്‍ തന്‍റെ ശക്തിയില്‍ അഹങ്കരിച്ച് പിന്തിരിഞ്ഞു കളയുകയാണ് ചെയ്തത്‌. (മൂസാ) ഒരു ജാലവിദ്യക്കാരനോ അല്ലെങ്കില്‍ ഭ്രാന്തനോ എന്ന് അവന്‍ പറയുകയും ചെയ്തു.
\end{malayalam}}
\flushright{\begin{Arabic}
\quranayah[51][40]
\end{Arabic}}
\flushleft{\begin{malayalam}
അതിനാല്‍ അവനെയും അവന്‍റെ സൈന്യങ്ങളെയും നാം പിടികൂടുകയും, എന്നിട്ട് അവരെ കടലില്‍ എറിയുകയും ചെയ്തു. അവന്‍ തന്നെയായിരുന്നു ആക്ഷേപാര്‍ഹന്‍ .
\end{malayalam}}
\flushright{\begin{Arabic}
\quranayah[51][41]
\end{Arabic}}
\flushleft{\begin{malayalam}
ആദ് ജനതയിലും (ദൃഷ്ടാന്തമുണ്ട്‌) വന്ധ്യമായ കാറ്റ് നാം അവരുടെ നേരെ അയച്ച സന്ദര്‍ഭം!
\end{malayalam}}
\flushright{\begin{Arabic}
\quranayah[51][42]
\end{Arabic}}
\flushleft{\begin{malayalam}
ആ കാറ്റ് ഏതൊരു വസ്തുവിന്മേല്‍ ചെന്നെത്തിയോ, അതിനെ ദ്രവിച്ച തുരുമ്പു പോലെ ആക്കാതെ അത് വിടുമായിരുന്നില്ല.
\end{malayalam}}
\flushright{\begin{Arabic}
\quranayah[51][43]
\end{Arabic}}
\flushleft{\begin{malayalam}
ഥമൂദ് ജനതയിലും (ദൃഷ്ടാന്തമുണ്ട്‌.) ഒരു സമയം വരെ നിങ്ങള്‍ സുഖം അനുഭവിച്ച് കൊള്ളുക. എന്ന് അവരോട് പറയപ്പെട്ട സന്ദര്‍ഭം!
\end{malayalam}}
\flushright{\begin{Arabic}
\quranayah[51][44]
\end{Arabic}}
\flushleft{\begin{malayalam}
എന്നിട്ട് അവര്‍ തങ്ങളുടെ രക്ഷിതാവിന്‍റെ കല്‍പനക്കെതിരായി ധിക്കാരം കൈക്കൊണ്ടു. അതിനാല്‍ അവര്‍ നോക്കിക്കൊണ്ടിരിക്കെ ആ ഘോരനാദം അവരെ പിടികൂടി.
\end{malayalam}}
\flushright{\begin{Arabic}
\quranayah[51][45]
\end{Arabic}}
\flushleft{\begin{malayalam}
അപ്പോള്‍ അവര്‍ക്ക് എഴുന്നേറ്റു പോകാന്‍ കഴിവുണ്ടായില്ല. അവര്‍ രക്ഷാനടപടികളെടുക്കുന്നവരായതുമില്ല.
\end{malayalam}}
\flushright{\begin{Arabic}
\quranayah[51][46]
\end{Arabic}}
\flushleft{\begin{malayalam}
അതിനു മുമ്പ് നൂഹിന്‍റെ ജനതയെയും (നാം നശിപ്പിക്കുകയുണ്ടായി.) തീര്‍ച്ചയായും അവര്‍ അധര്‍മ്മകാരികളായ ഒരു ജനതയായിരുന്നു.
\end{malayalam}}
\flushright{\begin{Arabic}
\quranayah[51][47]
\end{Arabic}}
\flushleft{\begin{malayalam}
ആകാശമാകട്ടെ നാം അതിനെ ശക്തി കൊണ്ട് നിര്‍മിച്ചിരിക്കുന്നു. തീര്‍ച്ചയായും നാം വികസിപ്പിച്ചെടുക്കുന്നവനാകുന്നു.
\end{malayalam}}
\flushright{\begin{Arabic}
\quranayah[51][48]
\end{Arabic}}
\flushleft{\begin{malayalam}
ഭൂമിയാകട്ടെ നാം അതിനെ ഒരു വിരിപ്പാക്കിയിരിക്കുന്നു. എന്നാല്‍ അത് വിതാനിച്ചവന്‍ എത്ര നല്ലവന്‍!
\end{malayalam}}
\flushright{\begin{Arabic}
\quranayah[51][49]
\end{Arabic}}
\flushleft{\begin{malayalam}
എല്ലാ വസ്തുക്കളില്‍ നിന്നും ഈ രണ്ട് ഇണകളെ നാം സൃഷ്ടിച്ചിരിക്കുന്നു. നിങ്ങള്‍ ആലോചിച്ച് മനസ്സിലാക്കുവാന്‍ വേണ്ടി.
\end{malayalam}}
\flushright{\begin{Arabic}
\quranayah[51][50]
\end{Arabic}}
\flushleft{\begin{malayalam}
അതിനാല്‍ നിങ്ങള്‍ അല്ലാഹുവിങ്കലേക്ക് ഓടിച്ചെല്ലുക. തീര്‍ച്ചയായും ഞാന്‍ നിങ്ങള്‍ക്ക് അവന്‍റെ അടുക്കല്‍ നിന്നുള്ള വ്യക്തമായ താക്കീതുകാരനാകുന്നു.
\end{malayalam}}
\flushright{\begin{Arabic}
\quranayah[51][51]
\end{Arabic}}
\flushleft{\begin{malayalam}
അല്ലാഹുവോടൊപ്പം മറ്റൊരു ദൈവത്തെയും നിങ്ങള്‍ സ്ഥാപിക്കാതിരിക്കുകയും ചെയ്യുക. തീര്‍ച്ചയായും ഞാന്‍ നിങ്ങള്‍ക്ക് അവന്‍റെ അടുക്കല്‍ നിന്നുള്ള വ്യക്തമായ താക്കീതുകാരനാകുന്നു.
\end{malayalam}}
\flushright{\begin{Arabic}
\quranayah[51][52]
\end{Arabic}}
\flushleft{\begin{malayalam}
അപ്രകാരം തന്നെ ഇവരുടെ പൂര്‍വ്വികന്‍മാരുടെ അടുത്ത് ഏതൊരു റസൂല്‍ വന്നപ്പോഴും ജാലവിദ്യക്കാരനെന്നോ, ഭ്രാന്തനെന്നോ അവര്‍ പറയാതിരുന്നിട്ടില്ല.
\end{malayalam}}
\flushright{\begin{Arabic}
\quranayah[51][53]
\end{Arabic}}
\flushleft{\begin{malayalam}
അതിന് (അങ്ങനെ പറയണമെന്ന്‌) അവര്‍ അന്യോന്യം വസ്വിയ്യത്ത് ചെയ്തിരിക്കുകയാണോ? അല്ല, അവര്‍ അതിക്രമകാരികളായ ഒരു ജനതയാകുന്നു.
\end{malayalam}}
\flushright{\begin{Arabic}
\quranayah[51][54]
\end{Arabic}}
\flushleft{\begin{malayalam}
ആകയാല്‍ നീ അവരില്‍ നിന്ന് തിരിഞ്ഞുകളയുക. നീ ആക്ഷേപാര്‍ഹനല്ല.
\end{malayalam}}
\flushright{\begin{Arabic}
\quranayah[51][55]
\end{Arabic}}
\flushleft{\begin{malayalam}
നീ ഉല്‍ബോധിപ്പിക്കുക. തീര്‍ച്ചയായും ഉല്‍ബോധനം സത്യവിശ്വാസികള്‍ക്ക് പ്രയോജനം ചെയ്യും.
\end{malayalam}}
\flushright{\begin{Arabic}
\quranayah[51][56]
\end{Arabic}}
\flushleft{\begin{malayalam}
ജിന്നുകളെയും മനുഷ്യരെയും എന്നെ ആരാധിക്കുവാന്‍ വേണ്ടിയല്ലാതെ ഞാന്‍ സൃഷ്ടിച്ചിട്ടില്ല.
\end{malayalam}}
\flushright{\begin{Arabic}
\quranayah[51][57]
\end{Arabic}}
\flushleft{\begin{malayalam}
ഞാന്‍ അവരില്‍ നിന്ന് ഉപജീവനമൊന്നും ആഗ്രഹിക്കുന്നില്ല. അവര്‍ എനിക്ക് ഭക്ഷണം നല്‍കണമെന്നും ഞാന്‍ ആഗ്രഹിക്കുന്നില്ല.
\end{malayalam}}
\flushright{\begin{Arabic}
\quranayah[51][58]
\end{Arabic}}
\flushleft{\begin{malayalam}
തീര്‍ച്ചയായും അല്ലാഹു തന്നെയാണ് ഉപജീവനം നല്‍കുന്നവനും ശക്തനും പ്രബലനും.
\end{malayalam}}
\flushright{\begin{Arabic}
\quranayah[51][59]
\end{Arabic}}
\flushleft{\begin{malayalam}
തീര്‍ച്ചയായും (ഇന്ന്‌) അക്രമം ചെയ്യുന്നവര്‍ക്ക് (പൂര്‍വ്വികരായ) തങ്ങളുടെ കൂട്ടാളികള്‍ക്കു ലഭിച്ച വിഹിതം പോലെയുള്ള വിഹിതം തന്നെയുണ്ട്‌. അതിനാല്‍ എന്നോട് അവര്‍ ധൃതികൂട്ടാതിരിക്കട്ടെ.
\end{malayalam}}
\flushright{\begin{Arabic}
\quranayah[51][60]
\end{Arabic}}
\flushleft{\begin{malayalam}
അപ്പോള്‍ തങ്ങള്‍ക്ക് താക്കീത് നല്‍കപ്പെടുന്നതായ ആ ദിവസം നിമിത്തം സത്യനിഷേധികള്‍ക്കു നാശം.
\end{malayalam}}
\chapter{\textmalayalam{ത്വൂര്‍ ( ത്വൂര്‍ പര്‍വ്വതം)}}
\begin{Arabic}
\Huge{\centerline{\basmalah}}\end{Arabic}
\flushright{\begin{Arabic}
\quranayah[52][1]
\end{Arabic}}
\flushleft{\begin{malayalam}
ത്വൂര്‍ പര്‍വ്വതം തന്നെയാണ, സത്യം.
\end{malayalam}}
\flushright{\begin{Arabic}
\quranayah[52][2]
\end{Arabic}}
\flushleft{\begin{malayalam}
എഴുതപ്പെട്ട ഗ്രന്ഥം തന്നെയാണ, സത്യം.
\end{malayalam}}
\flushright{\begin{Arabic}
\quranayah[52][3]
\end{Arabic}}
\flushleft{\begin{malayalam}
നിവര്‍ത്തിവെച്ച തുകലില്‍
\end{malayalam}}
\flushright{\begin{Arabic}
\quranayah[52][4]
\end{Arabic}}
\flushleft{\begin{malayalam}
അധിവാസമുള്ള മന്ദിരം തന്നെയാണ, സത്യം.
\end{malayalam}}
\flushright{\begin{Arabic}
\quranayah[52][5]
\end{Arabic}}
\flushleft{\begin{malayalam}
ഉയര്‍ത്തപ്പെട്ട മേല്‍പുര (ആകാശം) തന്നെയാണ, സത്യം.
\end{malayalam}}
\flushright{\begin{Arabic}
\quranayah[52][6]
\end{Arabic}}
\flushleft{\begin{malayalam}
നിറഞ്ഞ സമുദ്രം തന്നെയാണ, സത്യം.
\end{malayalam}}
\flushright{\begin{Arabic}
\quranayah[52][7]
\end{Arabic}}
\flushleft{\begin{malayalam}
തീര്‍ച്ചയായും നിന്‍റെ രക്ഷിതാവിന്‍റെ ശിക്ഷ സംഭവിക്കുന്നത് തന്നെയാകുന്നു.
\end{malayalam}}
\flushright{\begin{Arabic}
\quranayah[52][8]
\end{Arabic}}
\flushleft{\begin{malayalam}
അതു തടുക്കുവാന്‍ ആരും തന്നെയില്ല.
\end{malayalam}}
\flushright{\begin{Arabic}
\quranayah[52][9]
\end{Arabic}}
\flushleft{\begin{malayalam}
ആകാശം ശക്തിയായി പ്രകമ്പനം കൊള്ളുന്ന ദിവസം.
\end{malayalam}}
\flushright{\begin{Arabic}
\quranayah[52][10]
\end{Arabic}}
\flushleft{\begin{malayalam}
പര്‍വ്വതങ്ങള്‍ (അവയുടെ സ്ഥാനങ്ങളില്‍ നിന്ന്‌) നീങ്ങി സഞ്ചരിക്കുകയും ചെയ്യുന്ന ദിവസം.
\end{malayalam}}
\flushright{\begin{Arabic}
\quranayah[52][11]
\end{Arabic}}
\flushleft{\begin{malayalam}
അന്നേ ദിവസം സത്യനിഷേധികള്‍ക്കാകുന്നു നാശം.
\end{malayalam}}
\flushright{\begin{Arabic}
\quranayah[52][12]
\end{Arabic}}
\flushleft{\begin{malayalam}
അതായത് അനാവശ്യകാര്യങ്ങളില്‍ മുഴുകി കളിച്ചുകൊണ്ടിരിക്കുന്നവര്‍ക്ക്‌
\end{malayalam}}
\flushright{\begin{Arabic}
\quranayah[52][13]
\end{Arabic}}
\flushleft{\begin{malayalam}
അവര്‍ നരകാഗ്നിയിലേക്ക് ശക്തിയായി പിടിച്ച് തള്ളപ്പെടുന്ന ദിവസം.
\end{malayalam}}
\flushright{\begin{Arabic}
\quranayah[52][14]
\end{Arabic}}
\flushleft{\begin{malayalam}
(അവരോട് പറയപ്പെടും:) ഇതത്രെ നിങ്ങള്‍ നിഷേധിച്ചു കളഞ്ഞിരുന്ന നരകം.
\end{malayalam}}
\flushright{\begin{Arabic}
\quranayah[52][15]
\end{Arabic}}
\flushleft{\begin{malayalam}
അപ്പോള്‍ ഇത് മായാജാലമാണോ? അതല്ല, നിങ്ങള്‍ കാണുന്നില്ലെന്നുണ്ടോ?
\end{malayalam}}
\flushright{\begin{Arabic}
\quranayah[52][16]
\end{Arabic}}
\flushleft{\begin{malayalam}
നിങ്ങള്‍ അതില്‍ കടന്നു എരിഞ്ഞു കൊള്ളുക. എന്നിട്ട് നിങ്ങളത് സഹിക്കുക. അല്ലെങ്കില്‍ നിങ്ങള്‍ സഹിക്കാതിരിക്കുക. അത് രണ്ടും നിങ്ങള്‍ക്ക് സമമാകുന്നു. നിങ്ങള്‍ പ്രവര്‍ത്തിച്ച് കൊണ്ടിരുന്നതിന് മാത്രമാണ് നിങ്ങള്‍ക്ക് പ്രതിഫലം നല്‍കപ്പെടുന്നത്‌.
\end{malayalam}}
\flushright{\begin{Arabic}
\quranayah[52][17]
\end{Arabic}}
\flushleft{\begin{malayalam}
തീര്‍ച്ചയായും ധര്‍മ്മനിഷ്ഠപാലിക്കുന്നവര്‍ സ്വര്‍ഗത്തോപ്പുകളിലും സുഖാനുഗ്രഹങ്ങളിലുമായിരിക്കും.
\end{malayalam}}
\flushright{\begin{Arabic}
\quranayah[52][18]
\end{Arabic}}
\flushleft{\begin{malayalam}
തങ്ങളുടെ രക്ഷിതാവ് അവര്‍ക്കു നല്‍കിയതില്‍ ആനന്ദം കൊള്ളുന്നവരായിട്ട്‌. ജ്വലിക്കുന്ന നരകത്തിലെ ശിക്ഷയില്‍ നിന്ന് അവരുടെ രക്ഷിതാവ് അവരെ കാത്തുരക്ഷിക്കുകയും ചെയ്യും.
\end{malayalam}}
\flushright{\begin{Arabic}
\quranayah[52][19]
\end{Arabic}}
\flushleft{\begin{malayalam}
(അവരോട് പറയപ്പെടും:) നിങ്ങള്‍ പ്രവര്‍ത്തിച്ചിരുന്നതിന്‍റെ ഫലമായി നിങ്ങള്‍ സുഖമായി തിന്നുകയും കുടിക്കുകയും ചെയ്തു കൊള്ളുക.
\end{malayalam}}
\flushright{\begin{Arabic}
\quranayah[52][20]
\end{Arabic}}
\flushleft{\begin{malayalam}
വരിവരിയായ് ഇട്ട കട്ടിലുകളില്‍ ചാരിയിരിക്കുന്നവരായിരിക്കും അവര്‍. വിടര്‍ന്ന കണ്ണുകളുള്ള വെളുത്ത തരുണികളെ നാം അവര്‍ക്ക് ഇണചേര്‍ത്തു കൊടുക്കുകയും ചെയ്യും.
\end{malayalam}}
\flushright{\begin{Arabic}
\quranayah[52][21]
\end{Arabic}}
\flushleft{\begin{malayalam}
ഏതൊരു കൂട്ടര്‍ വിശ്വസിക്കുകയും അവരുടെ സന്താനങ്ങള്‍ വിശ്വാസത്തില്‍ അവരെ പിന്തുടരുകയും ചെയ്തിരിക്കുന്നുവോ അവരുടെ സന്താനങ്ങളെ നാം അവരോടൊപ്പം ചേര്‍ക്കുന്നതാണ്‌. അവരുടെ കര്‍മ്മഫലത്തില്‍ നിന്ന് യാതൊന്നും നാം അവര്‍ക്കു കുറവു വരുത്തുകയുമില്ല. ഏതൊരു മനുഷ്യനും താന്‍ സമ്പാദിച്ച് വെച്ചതിന് (സ്വന്തം കര്‍മ്മങ്ങള്‍ക്ക്‌) പണയം വെക്കപ്പെട്ടവനാകുന്നു.
\end{malayalam}}
\flushright{\begin{Arabic}
\quranayah[52][22]
\end{Arabic}}
\flushleft{\begin{malayalam}
അവര്‍ കൊതിക്കുന്ന തരത്തിലുള്ള പഴവും മാംസവും നാം അവര്‍ക്ക് അധികമായി നല്‍കുകയും ചെയ്യും.
\end{malayalam}}
\flushright{\begin{Arabic}
\quranayah[52][23]
\end{Arabic}}
\flushleft{\begin{malayalam}
അവിടെ അവര്‍ പാനപാത്രം അന്യോന്യം കൈമാറികൊണ്ടിരിക്കും. അവിടെ അനാവശ്യവാക്കോ, അധാര്‍മ്മിക വൃത്തിയോ ഇല്ല.
\end{malayalam}}
\flushright{\begin{Arabic}
\quranayah[52][24]
\end{Arabic}}
\flushleft{\begin{malayalam}
അവര്‍ക്ക് (പരിചരണത്തിനായി) ചെറുപ്പക്കാര്‍ അവരുടെ അടുത്ത് ചുറ്റിത്തിരിഞ്ഞു കൊണ്ടിരിക്കും. അവര്‍ സൂക്ഷിച്ച് വെക്കപ്പെട്ട മുത്തുകള്‍ പോലെയിരിക്കും
\end{malayalam}}
\flushright{\begin{Arabic}
\quranayah[52][25]
\end{Arabic}}
\flushleft{\begin{malayalam}
പരസ്പരം പലതും ചോദിച്ചു കൊണ്ട് അവരില്‍ ചിലര്‍ ചിലരെ അഭിമുഖീകരിക്കും.
\end{malayalam}}
\flushright{\begin{Arabic}
\quranayah[52][26]
\end{Arabic}}
\flushleft{\begin{malayalam}
അവര്‍ പറയും: തീര്‍ച്ചയായും നാം മുമ്പ് നമ്മുടെ കുടുംബത്തിലായിരിക്കുമ്പോള്‍ ഭയഭക്തിയുള്ളവരായിരുന്നു
\end{malayalam}}
\flushright{\begin{Arabic}
\quranayah[52][27]
\end{Arabic}}
\flushleft{\begin{malayalam}
അതിനാല്‍ അല്ലാഹു നമുക്ക് അനുഗ്രഹം നല്‍കുകയും, രോമകൂപങ്ങളില്‍ തുളച്ചു കയറുന്ന നരകാഗ്നിയുടെ ശിക്ഷയില്‍ നിന്ന് അവന്‍ നമ്മെ കാത്തുരക്ഷിക്കുകയും ചെയ്തു.
\end{malayalam}}
\flushright{\begin{Arabic}
\quranayah[52][28]
\end{Arabic}}
\flushleft{\begin{malayalam}
തീര്‍ച്ചയായും നാം മുമ്പേ അവനോട് പ്രാര്‍ത്ഥിക്കുന്നവരായിരുന്നു. തീര്‍ച്ചയായും അവന്‍ തന്നെയാകുന്നു ഔദാര്യവാനും കരുണാനിധിയും.
\end{malayalam}}
\flushright{\begin{Arabic}
\quranayah[52][29]
\end{Arabic}}
\flushleft{\begin{malayalam}
ആകയാല്‍ നീ ഉല്‍ബോധനം ചെയ്യുക. നിന്‍റെ രക്ഷിതാവിന്‍റെ അനുഗ്രഹത്താല്‍ നീ ഒരു ജ്യോത്സ്യനോ, ഭ്രാന്തനോ അല്ല.
\end{malayalam}}
\flushright{\begin{Arabic}
\quranayah[52][30]
\end{Arabic}}
\flushleft{\begin{malayalam}
അതല്ല, (മുഹമ്മദ്‌) ഒരു കവിയാണ്‌, അവന്ന് കാലവിപത്ത് വരുന്നത് ഞങ്ങള്‍ കാത്തിരിക്കുകയാണ് എന്നാണോ അവര്‍ പറയുന്നത്‌?
\end{malayalam}}
\flushright{\begin{Arabic}
\quranayah[52][31]
\end{Arabic}}
\flushleft{\begin{malayalam}
നീ പറഞ്ഞേക്കുക: നിങ്ങള്‍ കാത്തിരുന്നോളൂ. തീര്‍ച്ചയായും ഞാനും നിങ്ങളോടൊപ്പം കാത്തിരിക്കുന്നവരുടെ കൂട്ടത്തിലാകുന്നു.
\end{malayalam}}
\flushright{\begin{Arabic}
\quranayah[52][32]
\end{Arabic}}
\flushleft{\begin{malayalam}
അതല്ല, അവരുടെ മനസ്സുകള്‍ അവരോട് ഇപ്രകാരം കല്‍പിക്കുകയാണോ? അതല്ല, അവര്‍ ധിക്കാരികളായ ഒരു ജനത തന്നെയാണോ?
\end{malayalam}}
\flushright{\begin{Arabic}
\quranayah[52][33]
\end{Arabic}}
\flushleft{\begin{malayalam}
അതല്ല, അദ്ദേഹം (നബി) അത് കെട്ടിച്ചമച്ചു പറഞ്ഞതാണ് എന്ന് അവര്‍ പറയുകയാണോ? അല്ല, അവര്‍ വിശ്വസിക്കുന്നില്ല.
\end{malayalam}}
\flushright{\begin{Arabic}
\quranayah[52][34]
\end{Arabic}}
\flushleft{\begin{malayalam}
എന്നാല്‍ അവര്‍ സത്യവാന്‍മാരാണെങ്കില്‍ ഇതു പോലുള്ള ഒരു വൃത്താന്തം അവര്‍ കൊണ്ടുവരട്ടെ.
\end{malayalam}}
\flushright{\begin{Arabic}
\quranayah[52][35]
\end{Arabic}}
\flushleft{\begin{malayalam}
അതല്ല, യാതൊരു വസ്തുവില്‍ നിന്നുമല്ലാതെ അവര്‍ സൃഷ്ടിക്കപ്പെട്ടിരിക്കുകയാണോ? അതല്ല, അവര്‍ തന്നെയാണോ സ്രഷ്ടാക്കള്‍?
\end{malayalam}}
\flushright{\begin{Arabic}
\quranayah[52][36]
\end{Arabic}}
\flushleft{\begin{malayalam}
അതല്ല, അവരാണോ ആകാശങ്ങളും ഭൂമിയും സൃഷ്ടിച്ചിരിക്കുന്നത്‌? അല്ല, അവര്‍ ദൃഢമായി വിശ്വസിക്കുന്നില്ല.
\end{malayalam}}
\flushright{\begin{Arabic}
\quranayah[52][37]
\end{Arabic}}
\flushleft{\begin{malayalam}
അതല്ല, അവരുടെ പക്കലാണോ നിന്‍റെ രക്ഷിതാവിന്‍റെ ഖജനാവുകള്‍! അതല്ല, അവരാണോ അധികാരം നടത്തുന്നവര്‍?
\end{malayalam}}
\flushright{\begin{Arabic}
\quranayah[52][38]
\end{Arabic}}
\flushleft{\begin{malayalam}
അതല്ല, അവര്‍ക്ക് (ആകാശത്തു നിന്ന്‌) വിവരങ്ങള്‍ ശ്രദ്ധിച്ചു കേള്‍ക്കാന്‍ വല്ല കോണിയുമുണ്ടോ? എന്നാല്‍ അവരിലെ ശ്രദ്ധിച്ച് കേള്‍ക്കുന്ന ആള്‍ വ്യക്തമായ വല്ല പ്രമാണവും കൊണ്ടുവരട്ടെ.
\end{malayalam}}
\flushright{\begin{Arabic}
\quranayah[52][39]
\end{Arabic}}
\flushleft{\begin{malayalam}
അതല്ല, അവന്നു (അല്ലാഹുവിനു)ള്ളത് പെണ്‍മക്കളും നിങ്ങള്‍ക്കുള്ളത് ആണ്‍മക്കളുമാണോ?
\end{malayalam}}
\flushright{\begin{Arabic}
\quranayah[52][40]
\end{Arabic}}
\flushleft{\begin{malayalam}
അതല്ല, നീ അവരോട് വല്ല പ്രതിഫലവും ചോദിച്ചിട്ട് അവര്‍ കടബാധ്യതയാല്‍ ഭാരം പേറേണ്ടവരായിരിക്കുകയാണോ?
\end{malayalam}}
\flushright{\begin{Arabic}
\quranayah[52][41]
\end{Arabic}}
\flushleft{\begin{malayalam}
അതല്ല, അവര്‍ക്ക് അദൃശ്യജ്ഞാനം കരഗതമാവുകയും, അത് അവര്‍ രേഖപ്പെടുത്തിവെക്കുകയും ചെയ്യുന്നുണ്ടോ?
\end{malayalam}}
\flushright{\begin{Arabic}
\quranayah[52][42]
\end{Arabic}}
\flushleft{\begin{malayalam}
അതല്ല, അവര്‍ വല്ല കുതന്ത്രവും നടത്താന്‍ ഉദ്ദേശിക്കുകയാണോ? എന്നാല്‍ സത്യനിഷേധികളാരോ അവര്‍ തന്നെയാണ് കുതന്ത്രത്തില്‍ അകപ്പെടുന്നവര്‍.
\end{malayalam}}
\flushright{\begin{Arabic}
\quranayah[52][43]
\end{Arabic}}
\flushleft{\begin{malayalam}
അതല്ല, അവര്‍ക്ക് അല്ലാഹുവല്ലാത്ത വല്ല ദൈവവുമുണ്ടോ? അവര്‍ പങ്കുചേര്‍ക്കുന്നതില്‍ നിന്നെല്ലാം അല്ലാഹു എത്രയോ പരിശുദ്ധനായിരിക്കുന്നു.
\end{malayalam}}
\flushright{\begin{Arabic}
\quranayah[52][44]
\end{Arabic}}
\flushleft{\begin{malayalam}
ആകാശത്തുനിന്ന് ഒരു കഷ്ണം വീഴുന്നതായി അവര്‍ കാണുകയാണെങ്കിലും അവര്‍ പറയും: അത് അടുക്കടുക്കായ മേഘമാണെന്ന്‌.
\end{malayalam}}
\flushright{\begin{Arabic}
\quranayah[52][45]
\end{Arabic}}
\flushleft{\begin{malayalam}
അതിനാല്‍ അവര്‍ ബോധരഹിതരായി വീഴ്ത്തപ്പെടുന്ന അവരുടെ ആ ദിവസം അവര്‍ കണ്ടുമുട്ടുന്നത് വരെ നീ അവരെ വിട്ടേക്കുക.
\end{malayalam}}
\flushright{\begin{Arabic}
\quranayah[52][46]
\end{Arabic}}
\flushleft{\begin{malayalam}
അവരുടെ കുതന്ത്രം അവര്‍ക്ക് ഒട്ടും പ്രയോജനം ചെയ്യാത്ത, അവര്‍ക്ക് സഹായം ലഭിക്കാത്ത ഒരു ദിവസം.
\end{malayalam}}
\flushright{\begin{Arabic}
\quranayah[52][47]
\end{Arabic}}
\flushleft{\begin{malayalam}
തീര്‍ച്ചയായും അക്രമം പ്രവര്‍ത്തിച്ചവര്‍ക്ക് അതിനു പുറമെയും ശിക്ഷയുണ്ട്‌. പക്ഷെ അവരില്‍ അധികപേരും മനസ്സിലാക്കുന്നില്ല.
\end{malayalam}}
\flushright{\begin{Arabic}
\quranayah[52][48]
\end{Arabic}}
\flushleft{\begin{malayalam}
നിന്‍റെ രക്ഷിതാവിന്‍റെ തീരുമാനത്തിന് നീ ക്ഷമാപൂര്‍വ്വം കാത്തിരിക്കുക. തീര്‍ച്ചയായും നീ നമ്മുടെ ദൃഷ്ടിയിലാകുന്നു. നീ എഴുന്നേല്‍ക്കുന്ന സമയത്ത് നിന്‍റെ രക്ഷിതാവിനെ സ്തുതിക്കുന്നതോടൊപ്പം അവന്‍റെ പരിശുദ്ധിയെ നീ പ്രകീര്‍ത്തിക്കുകയും ചെയ്യുക.
\end{malayalam}}
\flushright{\begin{Arabic}
\quranayah[52][49]
\end{Arabic}}
\flushleft{\begin{malayalam}
രാത്രിയില്‍ കുറച്ച് സമയവും നക്ഷത്രങ്ങള്‍ പിന്‍വാങ്ങുമ്പോഴും നീ അവന്‍റെ പരിശുദ്ധിയെ പ്രകീര്‍ത്തിക്കുക.
\end{malayalam}}
\chapter{\textmalayalam{നജ്മ് ( നക്ഷത്രം )}}
\begin{Arabic}
\Huge{\centerline{\basmalah}}\end{Arabic}
\flushright{\begin{Arabic}
\quranayah[53][1]
\end{Arabic}}
\flushleft{\begin{malayalam}
നക്ഷത്രം അസ്തമിക്കുമ്പോള്‍ അതിനെ തന്നെയാണ, സത്യം.
\end{malayalam}}
\flushright{\begin{Arabic}
\quranayah[53][2]
\end{Arabic}}
\flushleft{\begin{malayalam}
നിങ്ങളുടെ കൂട്ടുകാരന്‍ വഴിതെറ്റിയിട്ടില്ല. ദുര്‍മാര്‍ഗിയായിട്ടുമില്ല.
\end{malayalam}}
\flushright{\begin{Arabic}
\quranayah[53][3]
\end{Arabic}}
\flushleft{\begin{malayalam}
അദ്ദേഹം തന്നിഷ്ടപ്രകാരം സംസാരിക്കുന്നുമില്ല.
\end{malayalam}}
\flushright{\begin{Arabic}
\quranayah[53][4]
\end{Arabic}}
\flushleft{\begin{malayalam}
അത് അദ്ദേഹത്തിന് ദിവ്യസന്ദേശമായി നല്‍കപ്പെടുന്ന ഒരു ഉല്‍ബോധനം മാത്രമാകുന്നു.
\end{malayalam}}
\flushright{\begin{Arabic}
\quranayah[53][5]
\end{Arabic}}
\flushleft{\begin{malayalam}
ശക്തിമത്തായ കഴിവുള്ളവനാണ് (ജിബ്‌രീല്‍ എന്ന മലക്കാണ്‌) അദ്ദേഹത്തെ പഠിപ്പിച്ചിട്ടുള്ളത്‌.
\end{malayalam}}
\flushright{\begin{Arabic}
\quranayah[53][6]
\end{Arabic}}
\flushleft{\begin{malayalam}
കരുത്തുള്ള ഒരു വ്യക്തി. അങ്ങനെ അദ്ദേഹം (സാക്ഷാല്‍ രൂപത്തില്‍) നിലകൊണ്ടു.
\end{malayalam}}
\flushright{\begin{Arabic}
\quranayah[53][7]
\end{Arabic}}
\flushleft{\begin{malayalam}
അദ്ദേഹമാകട്ടെ അത്യുന്നതമായ മണ്ഡലത്തിലായിരുന്നു.
\end{malayalam}}
\flushright{\begin{Arabic}
\quranayah[53][8]
\end{Arabic}}
\flushleft{\begin{malayalam}
പിന്നെ അദ്ദേഹം അടുത്തു വന്നു. അങ്ങനെ കൂടുതല്‍ അടുത്തു.
\end{malayalam}}
\flushright{\begin{Arabic}
\quranayah[53][9]
\end{Arabic}}
\flushleft{\begin{malayalam}
അങ്ങനെ അദ്ദേഹം രണ്ടു വില്ലുകളുടെ അകലത്തിലോ അതിനെക്കാള്‍ അടുത്തോ ആയിരുന്നു.
\end{malayalam}}
\flushright{\begin{Arabic}
\quranayah[53][10]
\end{Arabic}}
\flushleft{\begin{malayalam}
അപ്പോള്‍ അവന്‍ (അല്ലാഹു) തന്‍റെ ദാസന് അവന്‍ ബോധനം നല്‍കിയതെല്ലാം ബോധനം നല്‍കി.
\end{malayalam}}
\flushright{\begin{Arabic}
\quranayah[53][11]
\end{Arabic}}
\flushleft{\begin{malayalam}
അദ്ദേഹം കണ്ട ആ കാഴ്ച (അദ്ദേഹത്തിന്‍റെ) ഹൃദയം നിഷേധിച്ചിട്ടില്ല.
\end{malayalam}}
\flushright{\begin{Arabic}
\quranayah[53][12]
\end{Arabic}}
\flushleft{\begin{malayalam}
എന്നിരിക്കെ അദ്ദേഹം (നേരില്‍) കാണുന്നതിന്‍റെ പേരില്‍ നിങ്ങള്‍ അദ്ദേഹത്തോട് തര്‍ക്കിക്കുകയാണോ?
\end{malayalam}}
\flushright{\begin{Arabic}
\quranayah[53][13]
\end{Arabic}}
\flushleft{\begin{malayalam}
മറ്റൊരു ഇറക്കത്തിലും അദ്ദേഹം മലക്കിനെ കണ്ടിട്ടുണ്ട്‌.
\end{malayalam}}
\flushright{\begin{Arabic}
\quranayah[53][14]
\end{Arabic}}
\flushleft{\begin{malayalam}
അറ്റത്തെ ഇലന്തമരത്തിനടുത്ത് വെച്ച്‌
\end{malayalam}}
\flushright{\begin{Arabic}
\quranayah[53][15]
\end{Arabic}}
\flushleft{\begin{malayalam}
അതിന്നടുത്താകുന്നു താമസിക്കാനുള്ള സ്വര്‍ഗം.
\end{malayalam}}
\flushright{\begin{Arabic}
\quranayah[53][16]
\end{Arabic}}
\flushleft{\begin{malayalam}
ആ ഇലന്തമരത്തെ ആവരണം ചെയ്യുന്നതൊക്കെ അതിനെ ആവരണം ചെയ്തിരുന്നപ്പോള്‍.
\end{malayalam}}
\flushright{\begin{Arabic}
\quranayah[53][17]
\end{Arabic}}
\flushleft{\begin{malayalam}
(നബിയുടെ) ദൃഷ്ടി തെറ്റിപോയിട്ടില്ല. അതിക്രമിച്ചുപോയിട്ടുമില്ല.
\end{malayalam}}
\flushright{\begin{Arabic}
\quranayah[53][18]
\end{Arabic}}
\flushleft{\begin{malayalam}
തീര്‍ച്ചയായും തന്‍റെ രക്ഷിതാവിന്‍റെ അതിമഹത്തായ ദൃഷ്ടാന്തങ്ങളില്‍ ചിലത് അദ്ദേഹം കാണുകയുണ്ടായി.
\end{malayalam}}
\flushright{\begin{Arabic}
\quranayah[53][19]
\end{Arabic}}
\flushleft{\begin{malayalam}
ലാത്തയെയും ഉസ്സയെയും പറ്റി നിങ്ങള്‍ ചിന്തിച്ചു നോക്കിയിട്ടുണ്ടോ?
\end{malayalam}}
\flushright{\begin{Arabic}
\quranayah[53][20]
\end{Arabic}}
\flushleft{\begin{malayalam}
വേറെ മൂന്നാമതായുള്ള മനാത്തയെ പറ്റിയും
\end{malayalam}}
\flushright{\begin{Arabic}
\quranayah[53][21]
\end{Arabic}}
\flushleft{\begin{malayalam}
(സന്താനമായി) നിങ്ങള്‍ക്ക് ആണും അല്ലാഹുവിന് പെണ്ണുമാണെന്നോ?
\end{malayalam}}
\flushright{\begin{Arabic}
\quranayah[53][22]
\end{Arabic}}
\flushleft{\begin{malayalam}
എങ്കില്‍ അത് നീതിയില്ലാത്ത ഒരു ഓഹരി വെക്കല്‍ തന്നെ.
\end{malayalam}}
\flushright{\begin{Arabic}
\quranayah[53][23]
\end{Arabic}}
\flushleft{\begin{malayalam}
നിങ്ങളും നിങ്ങളുടെ പിതാക്കളും നാമകരണം ചെയ്ത ചില പേരുകളല്ലാതെ മറ്റൊന്നുമല്ല അവ (ദേവതകള്‍.) അവയെപ്പറ്റി അല്ലാഹു യാതൊരു പ്രമാണവും ഇറക്കിതന്നിട്ടില്ല. ഊഹത്തെയും മനസ്സുകള്‍ ഇച്ഛിക്കുന്നതിനെയും മാത്രമാണ് അവര്‍ പിന്തുടരുന്നത്‌. അവര്‍ക്ക് തങ്ങളുടെ രക്ഷിതാവിങ്കല്‍ നിന്ന് സന്‍മാര്‍ഗം വന്നിട്ടുണ്ട് താനും.
\end{malayalam}}
\flushright{\begin{Arabic}
\quranayah[53][24]
\end{Arabic}}
\flushleft{\begin{malayalam}
അതല്ല, മനുഷ്യന് അവന്‍ മോഹിച്ചതാണോ ലഭിക്കുന്നത്‌?
\end{malayalam}}
\flushright{\begin{Arabic}
\quranayah[53][25]
\end{Arabic}}
\flushleft{\begin{malayalam}
എന്നാല്‍ അല്ലാഹുവിന്നാകുന്നു ഇഹലോകവും പരലോകവും.
\end{malayalam}}
\flushright{\begin{Arabic}
\quranayah[53][26]
\end{Arabic}}
\flushleft{\begin{malayalam}
ആകാശങ്ങളില്‍ എത്ര മലക്കുകളാണുള്ളത്‌! അവരുടെ ശുപാര്‍ശ യാതൊരു പ്രയോജനവും ചെയ്യുകയില്ല; അല്ലാഹു അവന്‍ ഉദ്ദേശിക്കുകയും തൃപ്തിപ്പെടുകയും ചെയ്യുന്നവര്‍ക്ക് (ശുപാര്‍ശയ്ക്ക്‌) അനുവാദം നല്‍കിയതിന്‍റെ ശേഷമല്ലാതെ.
\end{malayalam}}
\flushright{\begin{Arabic}
\quranayah[53][27]
\end{Arabic}}
\flushleft{\begin{malayalam}
തീര്‍ച്ചയായും പരലോകത്തില്‍ വിശ്വസിക്കാത്തവര്‍ മലക്കുകള്‍ക്ക് പേരിടുന്നത് സ്ത്രീ നാമങ്ങളാകുന്നു.
\end{malayalam}}
\flushright{\begin{Arabic}
\quranayah[53][28]
\end{Arabic}}
\flushleft{\begin{malayalam}
അവര്‍ക്ക് അതിനെ പറ്റി യാതൊരു അറിവുമില്ല. അവര്‍ ഊഹത്തെ മാത്രമാകുന്നു പിന്തുടരുന്നത്‌. തീര്‍ച്ചയായും ഊഹം സത്യത്തെ സംബന്ധിച്ചേടത്തോളം ഒട്ടും പ്രയോജനം ചെയ്യുകയില്ല.
\end{malayalam}}
\flushright{\begin{Arabic}
\quranayah[53][29]
\end{Arabic}}
\flushleft{\begin{malayalam}
ആകയാല്‍ നമ്മുടെ സ്മരണ വിട്ടു തിരിഞ്ഞുകളയുകയും ഐഹികജീവിതം മാത്രം ലക്ഷ്യമാക്കുകയും ചെയ്തവരില്‍ നിന്ന് നീ തിരിഞ്ഞുകളയുക.
\end{malayalam}}
\flushright{\begin{Arabic}
\quranayah[53][30]
\end{Arabic}}
\flushleft{\begin{malayalam}
അറിവില്‍നിന്ന് അവര്‍ ആകെ എത്തിയിട്ടുള്ളത് അത്രത്തോളമാകുന്നു. തീര്‍ച്ചയായും നിന്‍റെ രക്ഷിതാവാകുന്നു അവന്‍റെ മാര്‍ഗത്തില്‍ നിന്ന് തെറ്റിപ്പോയവരെപ്പറ്റി കൂടുതല്‍ അറിവുള്ളവന്‍. സന്‍മാര്‍ഗം പ്രാപിച്ചവരെ പറ്റി കൂടുതല്‍ അറിവുള്ളവനും അവന്‍ തന്നെയാകുന്നു.
\end{malayalam}}
\flushright{\begin{Arabic}
\quranayah[53][31]
\end{Arabic}}
\flushleft{\begin{malayalam}
അല്ലാഹുവിന്നുള്ളതാകുന്നു ആകാശങ്ങളിലുള്ളതും ഭൂമിയിലുള്ളതും. തിന്‍മ പ്രവര്‍ത്തിച്ചവര്‍ക്ക് അവര്‍ ചെയ്യുന്നതിനനുസരിച്ച് പ്രതിഫലം നല്‍കുവാന്‍ വേണ്ടിയത്രെ അത്‌. നന്‍മ പ്രവര്‍ത്തിച്ചവര്‍ക്ക് ഏറ്റവും നല്ല പ്രതിഫലം നല്‍കുവാന്‍ വേണ്ടിയും.
\end{malayalam}}
\flushright{\begin{Arabic}
\quranayah[53][32]
\end{Arabic}}
\flushleft{\begin{malayalam}
അതായത് വലിയ പാപങ്ങളില്‍ നിന്നും, നിസ്സാരമായതൊഴിച്ചുള്ള നീചവൃത്തികളില്‍ നിന്നും വിട്ടകന്നു നില്‍ക്കുന്നവര്‍ക്ക്‌. തീര്‍ച്ചയായും നിന്‍റെ രക്ഷിതാവ് വിശാലമായി പാപമോചനം നല്‍കുന്നവനാകുന്നു. നിങ്ങളെ ഭൂമിയില്‍ നിന്ന് സൃഷ്ടിച്ചുണ്ടാക്കിയ സന്ദര്‍ഭത്തിലും, നിങ്ങള്‍ നിങ്ങളുടെ ഉമ്മമാരുടെ വയറുകളില്‍ ഗര്‍ഭസ്ഥശിശുക്കളായിരിക്കുന്ന സന്ദര്‍ഭത്തിലും അവനാകുന്നു നിങ്ങളെ പറ്റി കൂടുതല്‍ അറിവുള്ളവന്‍. അതിനാല്‍ നിങ്ങള്‍ ആത്മപ്രശംസ നടത്താതിരിക്കുക. അവനാകുന്നു സൂക്ഷ്മത പാലിച്ചവരെപ്പറ്റി നന്നായി അറിയുന്നവന്‍.
\end{malayalam}}
\flushright{\begin{Arabic}
\quranayah[53][33]
\end{Arabic}}
\flushleft{\begin{malayalam}
എന്നാല്‍ പിന്‍മാറിക്കളഞ്ഞ ഒരുത്തനെ നീ കണ്ടുവോ?
\end{malayalam}}
\flushright{\begin{Arabic}
\quranayah[53][34]
\end{Arabic}}
\flushleft{\begin{malayalam}
അല്‍പമൊക്കെ അവന്‍ ദാനം നല്‍കുകയും എന്നിട്ട് അത് നിര്‍ത്തിക്കളയുകയും ചെയ്തു.
\end{malayalam}}
\flushright{\begin{Arabic}
\quranayah[53][35]
\end{Arabic}}
\flushleft{\begin{malayalam}
അവന്‍റെ അടുക്കല്‍ അദൃശ്യജ്ഞാനമുണ്ടായിട്ട് അതു മുഖേന അവന്‍ കണ്ടറിഞ്ഞ് കൊണ്ടിരിക്കുകയാണോ?
\end{malayalam}}
\flushright{\begin{Arabic}
\quranayah[53][36]
\end{Arabic}}
\flushleft{\begin{malayalam}
അതല്ല, മൂസായുടെ പത്രികകളില്‍ ഉള്ളതിനെ പറ്റി അവന് വിവരം അറിയിക്കപ്പെട്ടിട്ടില്ലേ?
\end{malayalam}}
\flushright{\begin{Arabic}
\quranayah[53][37]
\end{Arabic}}
\flushleft{\begin{malayalam}
(കടമകള്‍) നിറവേറ്റിയ ഇബ്രാഹീമിന്‍റെയും (പത്രികകളില്‍)
\end{malayalam}}
\flushright{\begin{Arabic}
\quranayah[53][38]
\end{Arabic}}
\flushleft{\begin{malayalam}
അതായത് പാപഭാരം വഹിക്കുന്ന ഒരാളും മറ്റൊരാളുടെ പാപഭാരം വഹിക്കുകയില്ലെന്നും,
\end{malayalam}}
\flushright{\begin{Arabic}
\quranayah[53][39]
\end{Arabic}}
\flushleft{\begin{malayalam}
മനുഷ്യന്ന് താന്‍ പ്രയത്നിച്ചതല്ലാതെ മറ്റൊന്നുമില്ല എന്നും.
\end{malayalam}}
\flushright{\begin{Arabic}
\quranayah[53][40]
\end{Arabic}}
\flushleft{\begin{malayalam}
അവന്‍റെ പ്രയത്നഫലം വഴിയെ കാണിച്ചുകൊടുക്കപ്പെടും എന്നുമുള്ള കാര്യം.
\end{malayalam}}
\flushright{\begin{Arabic}
\quranayah[53][41]
\end{Arabic}}
\flushleft{\begin{malayalam}
പിന്നീട് അവന് അതിന് ഏറ്റവും പൂര്‍ണ്ണമായ പ്രതിഫലം നല്‍കപ്പെടുന്നതാണെന്നും,
\end{malayalam}}
\flushright{\begin{Arabic}
\quranayah[53][42]
\end{Arabic}}
\flushleft{\begin{malayalam}
നിന്‍റെ രക്ഷിതാവിങ്കലേക്കാണ് എല്ലാം ചെന്ന് അവസാനിക്കുന്നതെന്നും,
\end{malayalam}}
\flushright{\begin{Arabic}
\quranayah[53][43]
\end{Arabic}}
\flushleft{\begin{malayalam}
അവന്‍ തന്നെയാണ് ചിരിപ്പിക്കുകയും കരയിക്കുകയും ചെയ്തതെന്നും,
\end{malayalam}}
\flushright{\begin{Arabic}
\quranayah[53][44]
\end{Arabic}}
\flushleft{\begin{malayalam}
അവന്‍ തന്നെയാണ് മരിപ്പിക്കുകയും ജീവിപ്പിക്കുകയും ചെയ്തതെന്നും,
\end{malayalam}}
\flushright{\begin{Arabic}
\quranayah[53][45]
\end{Arabic}}
\flushleft{\begin{malayalam}
ആണ്‍‍ , പെണ്‍‍ എന്നീ രണ്ട് ഇണകളെ അവനാണ് സൃഷ്ടിച്ചതെന്നും
\end{malayalam}}
\flushright{\begin{Arabic}
\quranayah[53][46]
\end{Arabic}}
\flushleft{\begin{malayalam}
ഒരു ബീജം സ്രവിക്കപ്പെടുമ്പോള്‍ അതില്‍ നിന്ന്‌
\end{malayalam}}
\flushright{\begin{Arabic}
\quranayah[53][47]
\end{Arabic}}
\flushleft{\begin{malayalam}
രണ്ടാമത് ജനിപ്പിക്കുക എന്നത് അവന്‍റെ ചുമതലയിലാണെന്നും,
\end{malayalam}}
\flushright{\begin{Arabic}
\quranayah[53][48]
\end{Arabic}}
\flushleft{\begin{malayalam}
ഐശ്വര്യം നല്‍കുകയും സംതൃപ്തി വരുത്തുകയും ചെയ്തത് അവന്‍ തന്നെയാണ് എന്നും,
\end{malayalam}}
\flushright{\begin{Arabic}
\quranayah[53][49]
\end{Arabic}}
\flushleft{\begin{malayalam}
അവന്‍ തന്നെയാണ് ശിഅ്‌റാ നക്ഷത്രത്തിന്‍റെ രക്ഷിതാവ്‌. എന്നുമുള്ള കാര്യങ്ങള്‍.
\end{malayalam}}
\flushright{\begin{Arabic}
\quranayah[53][50]
\end{Arabic}}
\flushleft{\begin{malayalam}
ആദിമ ജനതയായ ആദിനെ അവനാണ് നശിപ്പിച്ചതെന്നും,
\end{malayalam}}
\flushright{\begin{Arabic}
\quranayah[53][51]
\end{Arabic}}
\flushleft{\begin{malayalam}
ഥമൂദിനെയും. എന്നിട്ട് (ഒരാളെയും) അവന്‍ അവശേഷിപ്പിച്ചില്ല.
\end{malayalam}}
\flushright{\begin{Arabic}
\quranayah[53][52]
\end{Arabic}}
\flushleft{\begin{malayalam}
അതിന് മുമ്പ് നൂഹിന്‍റെ ജനതയെയും (അവന്‍ നശിപ്പിച്ചു.) തീര്‍ച്ചയായും അവര്‍ കൂടുതല്‍ അക്രമവും, കൂടുതല്‍ ധിക്കാരവും കാണിച്ചവരായിരുന്നു.
\end{malayalam}}
\flushright{\begin{Arabic}
\quranayah[53][53]
\end{Arabic}}
\flushleft{\begin{malayalam}
കീഴ്മേല്‍ മറിഞ്ഞ രാജ്യത്തെയും, അവന്‍ തകര്‍ത്തു കളഞ്ഞു.
\end{malayalam}}
\flushright{\begin{Arabic}
\quranayah[53][54]
\end{Arabic}}
\flushleft{\begin{malayalam}
അങ്ങനെ ആ രാജ്യത്തെ അവന്‍ ഭയങ്കരമായ ഒരു (ശിക്ഷയുടെ) ആവരണം കൊണ്ട് പൊതിഞ്ഞു.
\end{malayalam}}
\flushright{\begin{Arabic}
\quranayah[53][55]
\end{Arabic}}
\flushleft{\begin{malayalam}
അപ്പോള്‍ നിന്‍റെ രക്ഷിതാവിന്‍റെ അനുഗ്രഹങ്ങളില്‍ ഏതൊന്നിനെപ്പറ്റിയാണ് നീ തര്‍ക്കിച്ചുകൊണ്ടിരിക്കുന്നത്‌?
\end{malayalam}}
\flushright{\begin{Arabic}
\quranayah[53][56]
\end{Arabic}}
\flushleft{\begin{malayalam}
ഇദ്ദേഹം (മുഹമ്മദ് നബി) പൂര്‍വ്വികരായ താക്കീതുകാരുടെ കൂട്ടത്തില്‍ പെട്ട ഒരു താക്കീതുകാരന്‍ ആകുന്നു.
\end{malayalam}}
\flushright{\begin{Arabic}
\quranayah[53][57]
\end{Arabic}}
\flushleft{\begin{malayalam}
സമീപസ്ഥമായ ആ സംഭവം ആസന്നമായിരിക്കുന്നു.
\end{malayalam}}
\flushright{\begin{Arabic}
\quranayah[53][58]
\end{Arabic}}
\flushleft{\begin{malayalam}
അല്ലാഹുവിന് പുറമെ അതിനെ തട്ടിനീക്കാന്‍ ആരുമില്ല.
\end{malayalam}}
\flushright{\begin{Arabic}
\quranayah[53][59]
\end{Arabic}}
\flushleft{\begin{malayalam}
അപ്പോള്‍ ഈ വാര്‍ത്തയെപ്പറ്റി നിങ്ങള്‍ അത്ഭുതപ്പെടുകയും,
\end{malayalam}}
\flushright{\begin{Arabic}
\quranayah[53][60]
\end{Arabic}}
\flushleft{\begin{malayalam}
നിങ്ങള്‍ ചിരിച്ച് കൊണ്ടിരിക്കുകയും നിങ്ങള്‍ കരയാതിരിക്കുകയും,
\end{malayalam}}
\flushright{\begin{Arabic}
\quranayah[53][61]
\end{Arabic}}
\flushleft{\begin{malayalam}
നിങ്ങള്‍ അശ്രദ്ധയില്‍ കഴിയുകയുമാണോ?.
\end{malayalam}}
\flushright{\begin{Arabic}
\quranayah[53][62]
\end{Arabic}}
\flushleft{\begin{malayalam}
അതിനാല്‍ നിങ്ങള്‍ അല്ലാഹുവിന് പ്രണാമം ചെയ്യുകയും (അവനെ) ആരാധിക്കുകയും ചെയ്യുവിന്‍.
\end{malayalam}}
\chapter{\textmalayalam{ഖമര്‍ ( ചന്ദ്രന്‍ )}}
\begin{Arabic}
\Huge{\centerline{\basmalah}}\end{Arabic}
\flushright{\begin{Arabic}
\quranayah[54][1]
\end{Arabic}}
\flushleft{\begin{malayalam}
ആ (അന്ത്യ) സമയം അടുത്തു. ചന്ദ്രന്‍ പിളരുകയും ചെയ്തു.
\end{malayalam}}
\flushright{\begin{Arabic}
\quranayah[54][2]
\end{Arabic}}
\flushleft{\begin{malayalam}
ഏതൊരു ദൃഷ്ടാന്തം അവര്‍ കാണുകയാണെങ്കിലും അവര്‍ പിന്തിരിഞ്ഞു കളയുകയും, ഇത് നിലനിന്നു വരുന്ന ജാലവിദ്യയാകുന്നു എന്ന് അവര്‍ പറയുകയും ചെയ്യും.
\end{malayalam}}
\flushright{\begin{Arabic}
\quranayah[54][3]
\end{Arabic}}
\flushleft{\begin{malayalam}
അവര്‍ നിഷേധിച്ചു തള്ളുകയും തങ്ങളുടെ തന്നിഷ്ടങ്ങളെ പിന്‍പറ്റുകയും ചെയ്തിരിക്കുന്നു. ഏതൊരു കാര്യവും ഒരു നിശ്ചിത സ്ഥാനം പ്രാപിക്കുന്നതാകുന്നു.
\end{malayalam}}
\flushright{\begin{Arabic}
\quranayah[54][4]
\end{Arabic}}
\flushleft{\begin{malayalam}
(ദൈവ നിഷേധത്തില്‍ നിന്ന്‌) അവര്‍ ഒഴിഞ്ഞു നില്‍ക്കാന്‍ പര്യാപ്തമായ കാര്യങ്ങളടങ്ങിയ ചില വൃത്താന്തങ്ങള്‍ തീര്‍ച്ചയായും അവര്‍ക്ക് വന്നുകിട്ടിയിട്ടുണ്ട്‌.
\end{malayalam}}
\flushright{\begin{Arabic}
\quranayah[54][5]
\end{Arabic}}
\flushleft{\begin{malayalam}
അതെ, പരിപൂര്‍ണ്ണമായ വിജ്ഞാനം. എന്നിട്ടും താക്കീതുകള്‍ പര്യാപ്തമാകുന്നില്ല.
\end{malayalam}}
\flushright{\begin{Arabic}
\quranayah[54][6]
\end{Arabic}}
\flushleft{\begin{malayalam}
ആകയാല്‍ (നബിയേ,) നീ അവരില്‍ നിന്ന് പിന്തിരിഞ്ഞ് കളയുക. അനിഷ്ടകരമായ ഒരു കാര്യത്തിലേക്ക് വിളിക്കുന്നവന്‍ വിളിക്കുന്ന ദിവസം.
\end{malayalam}}
\flushright{\begin{Arabic}
\quranayah[54][7]
\end{Arabic}}
\flushleft{\begin{malayalam}
ദൃഷ്ടികള്‍ താഴ്ന്നു പോയവരായ നിലയില്‍ ഖബ്‌റുകളില്‍ നിന്ന് (നാലുപാടും) പരന്ന വെട്ടുകിളികളെന്നോണം അവര്‍ പുറപ്പെട്ട് വരും.
\end{malayalam}}
\flushright{\begin{Arabic}
\quranayah[54][8]
\end{Arabic}}
\flushleft{\begin{malayalam}
വിളിക്കുന്നവന്‍റെ അടുത്തേക്ക് അവര്‍ ധൃതിപ്പെട്ട് ചെല്ലുന്നവരായിരിക്കും. സത്യനിഷേധികള്‍ (അന്ന്‌) പറയും: ഇതൊരു പ്രയാസകരമായ ദിവസമാകുന്നു.
\end{malayalam}}
\flushright{\begin{Arabic}
\quranayah[54][9]
\end{Arabic}}
\flushleft{\begin{malayalam}
അവര്‍ക്ക് മുമ്പ് നൂഹിന്‍റെ ജനതയും നിഷേധിച്ചു കളഞ്ഞിട്ടുണ്ട്‌. അങ്ങനെ നമ്മുടെ ദാസനെ അവര്‍ നിഷേധിച്ച് തള്ളുകയും ഭ്രാന്തന്‍ എന്നു പറയുകയും ചെയ്തു. അദ്ദേഹം വിരട്ടി ഓടിക്കപ്പെടുകയും ചെയ്തു.
\end{malayalam}}
\flushright{\begin{Arabic}
\quranayah[54][10]
\end{Arabic}}
\flushleft{\begin{malayalam}
അപ്പോള്‍ അദ്ദേഹം തന്‍റെ രക്ഷിതാവിനെ വിളിച്ചു പ്രാര്‍ത്ഥിച്ചു: ഞാന്‍ പരാജിതനാകുന്നു. അതിനാല്‍ (എന്‍റെ) രക്ഷയ്ക്കായി നീ നടപടി സ്വീകരിക്കണമേ.
\end{malayalam}}
\flushright{\begin{Arabic}
\quranayah[54][11]
\end{Arabic}}
\flushleft{\begin{malayalam}
അപ്പോള്‍ കുത്തിച്ചൊരിയുന്ന വെള്ളവും കൊണ്ട് ആകാശത്തിന്‍റെ കവാടങ്ങള്‍ നാം തുറന്നു.
\end{malayalam}}
\flushright{\begin{Arabic}
\quranayah[54][12]
\end{Arabic}}
\flushleft{\begin{malayalam}
ഭൂമിയില്‍ നാം ഉറവുകള്‍ പൊട്ടിക്കുകയും ചെയ്തു. അങ്ങനെ നിര്‍ണയിക്കപ്പെട്ടു കഴിഞ്ഞ ഒരു കാര്യത്തിന്നായി വെള്ളം സന്ധിച്ചു.
\end{malayalam}}
\flushright{\begin{Arabic}
\quranayah[54][13]
\end{Arabic}}
\flushleft{\begin{malayalam}
പലകകളും ആണികളുമുള്ള ഒരു കപ്പലില്‍ അദ്ദേഹത്തെ നാം വഹിക്കുകയും ചെയ്തു.
\end{malayalam}}
\flushright{\begin{Arabic}
\quranayah[54][14]
\end{Arabic}}
\flushleft{\begin{malayalam}
നമ്മുടെ മേല്‍നോട്ടത്തില്‍ അത് സഞ്ചരിക്കുന്നു. നിഷേധിച്ചു തള്ളപ്പെട്ടിരുന്നവന്നു (ദൈവദൂതന്ന്‌) ഉള്ള പ്രതിഫലമത്രെ അത്‌.
\end{malayalam}}
\flushright{\begin{Arabic}
\quranayah[54][15]
\end{Arabic}}
\flushleft{\begin{malayalam}
തീര്‍ച്ചയായും അതിനെ(പ്രളയത്തെ)നാം ഒരു ദൃഷ്ടാന്തമായി അവശേഷിപ്പിച്ചിരിക്കുന്നു. എന്നാല്‍ ആലോചിച്ച് മനസ്സിലാക്കുന്നവരായി ആരെങ്കിലുമുണ്ടോ?
\end{malayalam}}
\flushright{\begin{Arabic}
\quranayah[54][16]
\end{Arabic}}
\flushleft{\begin{malayalam}
അപ്പോള്‍ എന്‍റെ ശിക്ഷയും താക്കീതുകളും എങ്ങനെയായിരുന്നു.(എന്നു നോക്കുക)
\end{malayalam}}
\flushright{\begin{Arabic}
\quranayah[54][17]
\end{Arabic}}
\flushleft{\begin{malayalam}
തീര്‍ച്ചയായും ആലോചിച്ചു മനസ്സിലാക്കാന്‍ ഖുര്‍ആന്‍ നാം എളുപ്പമുള്ളതാക്കിയിരിക്കുന്നു. എന്നാല്‍ ആലോചിച്ചു മനസ്സിലാക്കുന്നവരായി ആരെങ്കിലുമുണ്ടോ?
\end{malayalam}}
\flushright{\begin{Arabic}
\quranayah[54][18]
\end{Arabic}}
\flushleft{\begin{malayalam}
ആദ് സമുദായം (സത്യത്തെ) നിഷേധിച്ചു കളഞ്ഞു. എന്നിട്ട് എന്‍റെ ശിക്ഷയും എന്‍റെ താക്കീതുകളും എങ്ങനെയായിരുന്നു.(എന്ന് നോക്കുക.)
\end{malayalam}}
\flushright{\begin{Arabic}
\quranayah[54][19]
\end{Arabic}}
\flushleft{\begin{malayalam}
വിട്ടുമാറാത്ത ദുശ്ശകുനത്തിന്‍റെ ഒരു ദിവസത്തില്‍ ഉഗ്രമായ ഒരു കാറ്റ് നാം അവരുടെ നേര്‍ക്ക് അയക്കുക തന്നെ ചെയ്തു.
\end{malayalam}}
\flushright{\begin{Arabic}
\quranayah[54][20]
\end{Arabic}}
\flushleft{\begin{malayalam}
കടപുഴകി വീഴുന്ന ഈന്തപ്പനത്തടികളെന്നോണം അത് മനുഷ്യരെ പറിച്ചെറിഞ്ഞു കൊണ്ടിരുന്നു.
\end{malayalam}}
\flushright{\begin{Arabic}
\quranayah[54][21]
\end{Arabic}}
\flushleft{\begin{malayalam}
അപ്പോള്‍ എന്‍റെ ശിക്ഷയും എന്‍റെ താക്കീതുകളും എങ്ങനെയായിരുന്നു.(എന്നു നോക്കുക.)
\end{malayalam}}
\flushright{\begin{Arabic}
\quranayah[54][22]
\end{Arabic}}
\flushleft{\begin{malayalam}
തീര്‍ച്ചയായും ആലോചിച്ച് മനസ്സിലാക്കുവാന്‍ ഖുര്‍ആന്‍ നാം എളുപ്പമുള്ളതാക്കിയിരിക്കുന്നു. എന്നാല്‍ ആലോചിച്ച് മനസ്സിലാക്കുന്നവരായി ആരെങ്കിലുമുണ്ടോ?
\end{malayalam}}
\flushright{\begin{Arabic}
\quranayah[54][23]
\end{Arabic}}
\flushleft{\begin{malayalam}
ഥമൂദ് സമുദായം താക്കീതുകളെ നിഷേധിച്ചു കളഞ്ഞു.
\end{malayalam}}
\flushright{\begin{Arabic}
\quranayah[54][24]
\end{Arabic}}
\flushleft{\begin{malayalam}
അങ്ങനെ അവര്‍ പറഞ്ഞു. നമ്മളില്‍ പെട്ട ഒരു മനുഷ്യനെ, ഒറ്റപ്പെട്ട ഒരുത്തനെ നാം പിന്തുടരുകയോ? എങ്കില്‍ തീര്‍ച്ചയായും നാം വഴിപിഴവിലും ബുദ്ധിശൂന്യതയിലും തന്നെയായിരിക്കും
\end{malayalam}}
\flushright{\begin{Arabic}
\quranayah[54][25]
\end{Arabic}}
\flushleft{\begin{malayalam}
നമ്മുടെ കൂട്ടത്തില്‍ നിന്ന് അവന്നു പ്രത്യേകമായി ഉല്‍ബോധനം നല്‍കപ്പെട്ടു എന്നോ? അല്ല, അവന്‍ അഹങ്കാരിയായ ഒരു വ്യാജവാദിയാകുന്നു.
\end{malayalam}}
\flushright{\begin{Arabic}
\quranayah[54][26]
\end{Arabic}}
\flushleft{\begin{malayalam}
എന്നാല്‍ നാളെ അവര്‍ അറിഞ്ഞ് കൊള്ളും; ആരാണ് അഹങ്കാരിയായ വ്യാജവാദിയെന്ന്‌.
\end{malayalam}}
\flushright{\begin{Arabic}
\quranayah[54][27]
\end{Arabic}}
\flushleft{\begin{malayalam}
(അവരുടെ പ്രവാചകന്‍ സ്വാലിഹിനോട് നാം പറഞ്ഞു:) തീര്‍ച്ചയായും അവര്‍ക്ക് ഒരു പരീക്ഷണമെന്ന നിലയില്‍ നാം ഒട്ടകത്തെ അയക്കുകയാകുന്നു. അത് കൊണ്ട് നീ അവരെ നിരീക്ഷിച്ച് കൊണ്ടിരിക്കുക. ക്ഷമ കൈക്കൊള്ളുകയും ചെയ്യുക.
\end{malayalam}}
\flushright{\begin{Arabic}
\quranayah[54][28]
\end{Arabic}}
\flushleft{\begin{malayalam}
വെള്ളം അവര്‍ക്കിടയില്‍ (അവര്‍ക്കും ഒട്ടകത്തിനുമിടയില്‍) പങ്കുവെക്കപ്പെട്ടതാണ് എന്ന് നീ അവര്‍ക്ക് വിവരം അറിയിക്കുകയും ചെയ്യുക. ഓരോരുത്തരുടെയും ജലപാനത്തിന്നുള്ള ഊഴത്തില്‍ (അതിന്ന് അവകാശപ്പെട്ടവര്‍) ഹാജരാകേണ്ടതാണ്‌.
\end{malayalam}}
\flushright{\begin{Arabic}
\quranayah[54][29]
\end{Arabic}}
\flushleft{\begin{malayalam}
അപ്പോള്‍ അവര്‍ അവരുടെ ചങ്ങാതിയെ വിളിച്ചു. ങ്ങനെ അവന്‍ (ആ കൃത്യം) ഏറ്റെടുത്തു. (ആ ഒട്ടകത്തെ) അറുകൊലചെയ്തു.
\end{malayalam}}
\flushright{\begin{Arabic}
\quranayah[54][30]
\end{Arabic}}
\flushleft{\begin{malayalam}
അപ്പോള്‍ എന്‍റെ ശിക്ഷയും എന്‍റെ താക്കീതുകളും എങ്ങനെയായിരുന്നു (എന്നു നോക്കുക.)
\end{malayalam}}
\flushright{\begin{Arabic}
\quranayah[54][31]
\end{Arabic}}
\flushleft{\begin{malayalam}
നാം അവരുടെ നേരെ ഒരു ഘോരശബ്ദം അയക്കുക തന്നെ ചെയ്തു. അപ്പോള്‍ അവര്‍ ആല വളച്ച് കെട്ടുന്നവര്‍ വിട്ടേച്ചുപോയ ചുള്ളിത്തുരുമ്പുകള്‍ പോലെ ആയിത്തീര്‍ന്നു.
\end{malayalam}}
\flushright{\begin{Arabic}
\quranayah[54][32]
\end{Arabic}}
\flushleft{\begin{malayalam}
തീര്‍ച്ചയായും ആലോചിച്ചു മനസ്സിലാക്കുവാന്‍ ഖുര്‍ആന്‍ നാം എളുപ്പമുള്ളതാക്കിയിരിക്കുന്നു. എന്നാല്‍ ആലോചിച്ച് മനസ്സിലാക്കുന്നവരായി ആരെങ്കിലുമുണ്ടോ?
\end{malayalam}}
\flushright{\begin{Arabic}
\quranayah[54][33]
\end{Arabic}}
\flushleft{\begin{malayalam}
ലൂത്വിന്‍റെ ജനത താക്കീതുകളെ നിഷേധിച്ചു കളഞ്ഞു.
\end{malayalam}}
\flushright{\begin{Arabic}
\quranayah[54][34]
\end{Arabic}}
\flushleft{\begin{malayalam}
തീര്‍ച്ചയായും നാം അവരുടെ നേരെ ഒരു ചരല്‍കാറ്റ് അയച്ചു. ലൂത്വിന്‍റെ കുടുംബം അതില്‍ നിന്ന് ഒഴിവായിരുന്നു. രാത്രിയുടെ അന്ത്യവേളയില്‍ നാം അവരെ രക്ഷപ്പെടുത്തി.
\end{malayalam}}
\flushright{\begin{Arabic}
\quranayah[54][35]
\end{Arabic}}
\flushleft{\begin{malayalam}
നമ്മുടെ പക്കല്‍ നിന്നുള്ള ഒരു അനുഗ്രഹമെന്ന നിലയില്‍. അപ്രകാരമാകുന്നു നന്ദികാണിച്ചവര്‍ക്ക് നാം പ്രതിഫലം നല്‍കുന്നത്‌.
\end{malayalam}}
\flushright{\begin{Arabic}
\quranayah[54][36]
\end{Arabic}}
\flushleft{\begin{malayalam}
നമ്മുടെ ശിക്ഷയെപറ്റി അദ്ദേഹം (ലൂത്വ്‌) അവര്‍ക്കു താക്കീത് നല്‍കുകയുണ്ടായി. അപ്പോള്‍ അവര്‍ താക്കീതുകള്‍ സംശയിച്ച് തള്ളുകയാണ് ചെയ്തത്‌.
\end{malayalam}}
\flushright{\begin{Arabic}
\quranayah[54][37]
\end{Arabic}}
\flushleft{\begin{malayalam}
അദ്ദേഹത്തോട് (ലൂത്വിനോട്‌) അദ്ദേഹത്തിന്‍റെ അതിഥികളെ (ദുര്‍വൃത്തിക്കായി) വിട്ടുകൊടുക്കുവാനും അവര്‍ ആവശ്യപ്പെടുകയുണ്ടായി. അപ്പോള്‍ അവരുടെ കണ്ണുകളെ നാം തുടച്ചുനീക്കി. എന്‍റെ ശിക്ഷയും എന്‍റെ താക്കീതുകളും നിങ്ങള്‍ അനുഭവിച്ച് കൊള്ളുക (എന്ന് നാം അവരോട് പറഞ്ഞു.)
\end{malayalam}}
\flushright{\begin{Arabic}
\quranayah[54][38]
\end{Arabic}}
\flushleft{\begin{malayalam}
അതിരാവിലെ അവര്‍ക്ക് സുസ്ഥിരമായ ശിക്ഷ വന്നെത്തുക തന്നെ ചെയ്തു.
\end{malayalam}}
\flushright{\begin{Arabic}
\quranayah[54][39]
\end{Arabic}}
\flushleft{\begin{malayalam}
എന്‍റെ ശിക്ഷയും എന്‍റെ താക്കീതുകളും നിങ്ങള്‍ അനുഭവിച്ചു കൊള്ളുക.(എന്ന് നാം അവരോട് പറഞ്ഞു.)
\end{malayalam}}
\flushright{\begin{Arabic}
\quranayah[54][40]
\end{Arabic}}
\flushleft{\begin{malayalam}
തീര്‍ച്ചയായും ആലോചിച്ചു മനസ്സിലാക്കുന്നതിന് ഖുര്‍ആന്‍ നാം എളുപ്പമുള്ളതാക്കിയിരിക്കുന്നു. എന്നാല്‍ ആലോചിച്ച് മനസ്സിലാക്കുന്നവരായി ആരെങ്കിലുമുണ്ടോ?
\end{malayalam}}
\flushright{\begin{Arabic}
\quranayah[54][41]
\end{Arabic}}
\flushleft{\begin{malayalam}
ഫിര്‍ഔന്‍ കുടുംബത്തിനും താക്കീതുകള്‍ വന്നെത്തുകയുണ്ടായി.
\end{malayalam}}
\flushright{\begin{Arabic}
\quranayah[54][42]
\end{Arabic}}
\flushleft{\begin{malayalam}
അവര്‍ നമ്മുടെ ദൃഷ്ടാന്തങ്ങളെ മുഴുവന്‍ നിഷേധിച്ചു തള്ളിക്കളഞ്ഞു.അപ്പോള്‍ പ്രതാപിയും ശക്തനുമായ ഒരുത്തന്‍ പിടികൂടുന്ന വിധം നാം അവരെ പിടികൂടി.
\end{malayalam}}
\flushright{\begin{Arabic}
\quranayah[54][43]
\end{Arabic}}
\flushleft{\begin{malayalam}
(ഹേ, അറബികളേ,) നിങ്ങളുടെ കൂട്ടത്തിലെ സത്യനിഷേധികള്‍ അവരെക്കാളൊക്കെ ഉത്തമന്‍മാരാണോ? അതല്ല, വേദപ്രമാണങ്ങളില്‍ നിങ്ങള്‍ക്ക് (മാത്രം) വല്ല ഒഴിവുമുണ്ടോ?
\end{malayalam}}
\flushright{\begin{Arabic}
\quranayah[54][44]
\end{Arabic}}
\flushleft{\begin{malayalam}
അതല്ല, അവര്‍ പറയുന്നുവോ; ഞങ്ങള്‍ സംഘടിതരും സ്വയം പ്രതിരോധിക്കാന്‍ കഴിവുള്ളവരുമാണ് എന്ന്‌.
\end{malayalam}}
\flushright{\begin{Arabic}
\quranayah[54][45]
\end{Arabic}}
\flushleft{\begin{malayalam}
എന്നാല്‍ വഴിയെ ആ സംഘം തോല്‍പിക്കപ്പെടുന്നതാണ്‌. അവര്‍ പിന്തിരിഞ്ഞ് ഓടുകയും ചെയ്യും.
\end{malayalam}}
\flushright{\begin{Arabic}
\quranayah[54][46]
\end{Arabic}}
\flushleft{\begin{malayalam}
തന്നെയുമല്ല, ആ അന്ത്യസമയമാകുന്നു അവര്‍ക്കുള്ള നിശ്ചിത സന്ദര്‍ഭം. ആ അന്ത്യസമയം ഏറ്റവും ആപല്‍ക്കരവും അത്യന്തം കയ്പേറിയതുമാകുന്നു.
\end{malayalam}}
\flushright{\begin{Arabic}
\quranayah[54][47]
\end{Arabic}}
\flushleft{\begin{malayalam}
തീര്‍ച്ചയായും ആ കുറ്റവാളികള്‍ വഴിപിഴവിലും ബുദ്ധിശൂന്യതയിലുമാകുന്നു.
\end{malayalam}}
\flushright{\begin{Arabic}
\quranayah[54][48]
\end{Arabic}}
\flushleft{\begin{malayalam}
മുഖം നിലത്തു കുത്തിയനിലയില്‍ അവര്‍ നരകാഗ്നിയിലൂടെ വലിച്ചിഴക്കപ്പെടുന്ന ദിവസം. (അവരോട് പറയപ്പെടും:) നിങ്ങള്‍ നരകത്തിന്‍റെ സ്പര്‍ശനം അനുഭവിച്ച് കൊള്ളുക.
\end{malayalam}}
\flushright{\begin{Arabic}
\quranayah[54][49]
\end{Arabic}}
\flushleft{\begin{malayalam}
തീര്‍ച്ചയായും ഏതു വസ്തുവെയും നാം സൃഷ്ടിച്ചിട്ടുള്ളത് ഒരു വ്യവസ്ഥപ്രകാരമാകുന്നു.
\end{malayalam}}
\flushright{\begin{Arabic}
\quranayah[54][50]
\end{Arabic}}
\flushleft{\begin{malayalam}
നമ്മുടെ കല്‍പന ഒരൊറ്റ പ്രഖ്യാപനം മാത്രമാകുന്നു. കണ്ണിന്‍റെ ഒരു ഇമവെട്ടല്‍ പോലെ.
\end{malayalam}}
\flushright{\begin{Arabic}
\quranayah[54][51]
\end{Arabic}}
\flushleft{\begin{malayalam}
(ഹേ, സത്യനിഷേധികളേ,) തീര്‍ച്ചയായും നിങ്ങളുടെ കക്ഷിക്കാരെ നാം നശിപ്പിക്കുകയുണ്ടായിട്ടുണ്ട്‌. എന്നാല്‍ ആലോചിച്ച് മനസ്സിലാക്കുന്നവരായി ആരെങ്കിലുമുണ്ടോ?
\end{malayalam}}
\flushright{\begin{Arabic}
\quranayah[54][52]
\end{Arabic}}
\flushleft{\begin{malayalam}
അവര്‍ പ്രവര്‍ത്തിച്ച ഏത് കാര്യവും രേഖകളിലുണ്ട്‌.
\end{malayalam}}
\flushright{\begin{Arabic}
\quranayah[54][53]
\end{Arabic}}
\flushleft{\begin{malayalam}
ഏത് ചെറിയകാര്യവും വലിയ കാര്യവും രേഖപ്പെടുത്തി വെക്കപ്പെടുന്നതാണ്‌.
\end{malayalam}}
\flushright{\begin{Arabic}
\quranayah[54][54]
\end{Arabic}}
\flushleft{\begin{malayalam}
തീര്‍ച്ചയായും ധര്‍മ്മനിഷ്ഠ പാലിച്ചവര്‍ ഉദ്യാനങ്ങളിലും അരുവികളിലുമായിരിക്കും.
\end{malayalam}}
\flushright{\begin{Arabic}
\quranayah[54][55]
\end{Arabic}}
\flushleft{\begin{malayalam}
സത്യത്തിന്‍റെ ഇരിപ്പിടത്തില്‍, ശക്തനായ രാജാവിന്‍റെ അടുക്കല്‍.
\end{malayalam}}
\chapter{\textmalayalam{റഹ് മാന്‍‍ ( പരമകാരുണികന്‍ )}}
\begin{Arabic}
\Huge{\centerline{\basmalah}}\end{Arabic}
\flushright{\begin{Arabic}
\quranayah[55][1]
\end{Arabic}}
\flushleft{\begin{malayalam}
പരമകാരുണികന്‍
\end{malayalam}}
\flushright{\begin{Arabic}
\quranayah[55][2]
\end{Arabic}}
\flushleft{\begin{malayalam}
ഈ ഖുര്‍ആന്‍ പഠിപ്പിച്ചു.
\end{malayalam}}
\flushright{\begin{Arabic}
\quranayah[55][3]
\end{Arabic}}
\flushleft{\begin{malayalam}
അവന്‍ മനുഷ്യനെ സൃഷ്ടിച്ചു.
\end{malayalam}}
\flushright{\begin{Arabic}
\quranayah[55][4]
\end{Arabic}}
\flushleft{\begin{malayalam}
അവനെ അവന്‍ സംസാരിക്കാന്‍ പഠിപ്പിച്ചു.
\end{malayalam}}
\flushright{\begin{Arabic}
\quranayah[55][5]
\end{Arabic}}
\flushleft{\begin{malayalam}
സൂര്യനും ചന്ദ്രനും ഒരു കണക്കനുസരിച്ചാകുന്നു (സഞ്ചരിക്കുന്നത്‌.)
\end{malayalam}}
\flushright{\begin{Arabic}
\quranayah[55][6]
\end{Arabic}}
\flushleft{\begin{malayalam}
ചെടികളും വൃക്ഷങ്ങളും (അല്ലാഹുവിന്‌) പ്രണാമം അര്‍പ്പിച്ചു കൊണ്ടിരിക്കുന്നു.
\end{malayalam}}
\flushright{\begin{Arabic}
\quranayah[55][7]
\end{Arabic}}
\flushleft{\begin{malayalam}
ആകാശത്തെ അവന്‍ ഉയര്‍ത്തുകയും, (എല്ലാകാര്യവും തൂക്കികണക്കാക്കുവാനുള്ള) തുലാസ് അവന്‍ സ്ഥാപിക്കുകയും ചെയ്തിരിക്കുന്നു.
\end{malayalam}}
\flushright{\begin{Arabic}
\quranayah[55][8]
\end{Arabic}}
\flushleft{\begin{malayalam}
നിങ്ങള്‍ തുലാസില്‍ ക്രമക്കേട് വരുത്താതിരിക്കുവാന്‍ വേണ്ടിയാണത്‌.
\end{malayalam}}
\flushright{\begin{Arabic}
\quranayah[55][9]
\end{Arabic}}
\flushleft{\begin{malayalam}
നിങ്ങള്‍ നീതി പൂര്‍വ്വം തൂക്കം ശരിയാക്കുവിന്‍. തുലാസില്‍ നിങ്ങള്‍ കമ്മി വരുത്തരുത്‌.
\end{malayalam}}
\flushright{\begin{Arabic}
\quranayah[55][10]
\end{Arabic}}
\flushleft{\begin{malayalam}
ഭൂമിയെ അവന്‍ മനുഷ്യര്‍ക്കായി വെച്ചിരിക്കുന്നു.
\end{malayalam}}
\flushright{\begin{Arabic}
\quranayah[55][11]
\end{Arabic}}
\flushleft{\begin{malayalam}
അതില്‍ പഴങ്ങളും കൂമ്പോളകളുള്ള ഈന്തപ്പനകളുമുണ്ട്‌.
\end{malayalam}}
\flushright{\begin{Arabic}
\quranayah[55][12]
\end{Arabic}}
\flushleft{\begin{malayalam}
വൈക്കോലുള്ള ധാന്യങ്ങളും സുഗന്ധച്ചെടികളുമുണ്ട്‌.
\end{malayalam}}
\flushright{\begin{Arabic}
\quranayah[55][13]
\end{Arabic}}
\flushleft{\begin{malayalam}
അപ്പോള്‍ നിങ്ങള്‍ ഇരു വിഭാഗത്തിന്‍റെയും രക്ഷിതാവ് ചെയ്ത അനുഗ്രഹങ്ങളില്‍ ഏതിനെയാണ് നിങ്ങള്‍ നിഷേധിക്കുന്നത്‌?
\end{malayalam}}
\flushright{\begin{Arabic}
\quranayah[55][14]
\end{Arabic}}
\flushleft{\begin{malayalam}
കലം പോലെ മുട്ടിയാല്‍ മുഴക്കമുണ്ടാകുന്ന (ഉണങ്ങിയ) കളിമണ്ണില്‍ നിന്ന് മനുഷ്യനെ അവന്‍ സൃഷ്ടിച്ചു.
\end{malayalam}}
\flushright{\begin{Arabic}
\quranayah[55][15]
\end{Arabic}}
\flushleft{\begin{malayalam}
തിയ്യിന്‍റെ പുകയില്ലാത്ത ജ്വാലയില്‍ നിന്ന് ജിന്നിനെയും അവന്‍ സൃഷ്ടിച്ചു.
\end{malayalam}}
\flushright{\begin{Arabic}
\quranayah[55][16]
\end{Arabic}}
\flushleft{\begin{malayalam}
അപ്പോള്‍ നിങ്ങള്‍ ഇരുവിഭാഗത്തിന്‍റെയും രക്ഷിതാവ് ചെയ്ത അനുഗ്രഹങ്ങളില്‍ ഏതിനെയാണ് നിങ്ങള്‍ നിഷേധിക്കുന്നത്‌?
\end{malayalam}}
\flushright{\begin{Arabic}
\quranayah[55][17]
\end{Arabic}}
\flushleft{\begin{malayalam}
രണ്ട് ഉദയസ്ഥാനങ്ങളുടെ രക്ഷിതാവും രണ്ട് അസ്തമന സ്ഥാനങ്ങളുടെ രക്ഷിതാവുമാകുന്നു അവന്‍.
\end{malayalam}}
\flushright{\begin{Arabic}
\quranayah[55][18]
\end{Arabic}}
\flushleft{\begin{malayalam}
അപ്പോള്‍ നിങ്ങള്‍ ഇരു വിഭാഗത്തിന്‍റെയും രക്ഷിതാവ് ചെയ്ത അനുഗ്രഹങ്ങളില്‍ ഏതിനെയാണ് നിങ്ങള്‍ നിഷേധിക്കുന്നത്‌?
\end{malayalam}}
\flushright{\begin{Arabic}
\quranayah[55][19]
\end{Arabic}}
\flushleft{\begin{malayalam}
രണ്ട് കടലുകളെ (ജലാശയങ്ങളെ) തമ്മില്‍ കൂടിച്ചേരത്തക്ക വിധം അവന്‍ അയച്ചുവിട്ടിരിക്കുന്നു.
\end{malayalam}}
\flushright{\begin{Arabic}
\quranayah[55][20]
\end{Arabic}}
\flushleft{\begin{malayalam}
അവ രണ്ടിനുമിടക്ക് അവ അന്യോന്യം അതിക്രമിച്ച് കടക്കാതിരിക്കത്തക്കവിധം ഒരു തടസ്സമുണ്ട്‌.
\end{malayalam}}
\flushright{\begin{Arabic}
\quranayah[55][21]
\end{Arabic}}
\flushleft{\begin{malayalam}
അപ്പോള്‍ നിങ്ങള്‍ ഇരു വിഭാഗത്തിന്‍റെയും രക്ഷിതാവിന്‍റെ അനുഗ്രഹങ്ങളില്‍ ഏതിനെയാണ് നിങ്ങള്‍ നിഷേധിക്കുന്നത്‌?
\end{malayalam}}
\flushright{\begin{Arabic}
\quranayah[55][22]
\end{Arabic}}
\flushleft{\begin{malayalam}
അവ രണ്ടില്‍ നിന്നും മുത്തും പവിഴവും പുറത്തു വരുന്നു.
\end{malayalam}}
\flushright{\begin{Arabic}
\quranayah[55][23]
\end{Arabic}}
\flushleft{\begin{malayalam}
അപ്പോള്‍ നിങ്ങള്‍ ഇരു വിഭാഗത്തിന്‍റെയും രക്ഷിതാവിന്‍റെ അനുഗ്രഹങ്ങളില്‍ ഏതിനെയാണ് നിങ്ങള്‍ നിഷേധിക്കുന്നത്‌?
\end{malayalam}}
\flushright{\begin{Arabic}
\quranayah[55][24]
\end{Arabic}}
\flushleft{\begin{malayalam}
സമുദ്രത്തില്‍ (സഞ്ചരിക്കുവാന്‍) മലകള്‍ പോലെ പൊക്കി ഉണ്ടാക്കപ്പെടുന്ന കപ്പലുകളും അവന്‍റെ നിയന്ത്രണത്തിലാകുന്നു.
\end{malayalam}}
\flushright{\begin{Arabic}
\quranayah[55][25]
\end{Arabic}}
\flushleft{\begin{malayalam}
അപ്പോള്‍ നിങ്ങള്‍ ഇരു വിഭാഗത്തിന്‍റെയും രക്ഷിതാവിന്‍റെ അനുഗ്രഹങ്ങളില്‍ ഏതിനെയാണ് നിങ്ങള്‍ നിഷേധിക്കുന്നത്‌?
\end{malayalam}}
\flushright{\begin{Arabic}
\quranayah[55][26]
\end{Arabic}}
\flushleft{\begin{malayalam}
അവിടെ (ഭൂമുഖത്ത്‌)യുള്ള എല്ലാവരും നശിച്ച് പോകുന്നവരാകുന്നു.
\end{malayalam}}
\flushright{\begin{Arabic}
\quranayah[55][27]
\end{Arabic}}
\flushleft{\begin{malayalam}
മഹത്വവും ഉദാരതയും ഉള്ളവനായ നിന്‍റെ രക്ഷിതാവിന്‍റെ മുഖം അവശേഷിക്കുന്നതാണ്‌.
\end{malayalam}}
\flushright{\begin{Arabic}
\quranayah[55][28]
\end{Arabic}}
\flushleft{\begin{malayalam}
അപ്പോള്‍ നിങ്ങള്‍ ഇരു വിഭാഗത്തിന്‍റെയും രക്ഷിതാവിന്‍റെ അനുഗ്രഹങ്ങളില്‍ ഏതിനെയാണ് നിങ്ങള്‍ നിഷേധിക്കുന്നത്‌?
\end{malayalam}}
\flushright{\begin{Arabic}
\quranayah[55][29]
\end{Arabic}}
\flushleft{\begin{malayalam}
ആകാശങ്ങളിലും ഭൂമിയിലും ഉള്ളവര്‍ അവനോട് ചോദിച്ചു കൊണ്ടിരിക്കുന്നു. എല്ലാ ദിവസവും അവന്‍ കാര്യനിര്‍വഹണത്തിലാകുന്നു.
\end{malayalam}}
\flushright{\begin{Arabic}
\quranayah[55][30]
\end{Arabic}}
\flushleft{\begin{malayalam}
അപ്പോള്‍ നിങ്ങള്‍ ഇരു വിഭാഗത്തിന്‍റെയും രക്ഷിതാവിന്‍റെ അനുഗ്രഹങ്ങളില്‍ ഏതിനെയാണ് നിങ്ങള്‍ നിഷേധിക്കുന്നത്‌?
\end{malayalam}}
\flushright{\begin{Arabic}
\quranayah[55][31]
\end{Arabic}}
\flushleft{\begin{malayalam}
ഹേ; ഭാരിച്ച രണ്ട് സമൂഹങ്ങളേ, നിങ്ങളുടെ കാര്യത്തിനായി നാം ഒഴിഞ്ഞിരിക്കുന്നതാണ്‌.
\end{malayalam}}
\flushright{\begin{Arabic}
\quranayah[55][32]
\end{Arabic}}
\flushleft{\begin{malayalam}
അപ്പോള്‍ നിങ്ങള്‍ ഇരു വിഭാഗത്തിന്‍റെയും രക്ഷിതാവിന്‍റെ അനുഗ്രഹങ്ങളില്‍ ഏതിനെയാണ് നിങ്ങള്‍ നിഷേധിക്കുന്നത്‌?
\end{malayalam}}
\flushright{\begin{Arabic}
\quranayah[55][33]
\end{Arabic}}
\flushleft{\begin{malayalam}
ജിന്നുകളുടെയും മനുഷ്യരുടെയും സമൂഹമേ, ആകാശങ്ങളുടെയും ഭൂമിയുടെയും മേഖലകളില്‍ നിന്ന് പുറത്ത് കടന്നു പോകാന്‍ നിങ്ങള്‍ക്ക് സാധിക്കുന്ന പക്ഷം നിങ്ങള്‍ കടന്നു പോയിക്കൊള്ളുക. ഒരു അധികാരം ലഭിച്ചിട്ടല്ലാതെ നിങ്ങള്‍ കടന്നു പോകുകയില്ല.
\end{malayalam}}
\flushright{\begin{Arabic}
\quranayah[55][34]
\end{Arabic}}
\flushleft{\begin{malayalam}
അപ്പോള്‍ നിങ്ങള്‍ ഇരു വിഭാഗത്തിന്‍റെയും രക്ഷിതാവിന്‍റെ അനുഗ്രഹങ്ങളില്‍ ഏതിനെയാണ് നിങ്ങള്‍ നിഷേധിക്കുന്നത്‌?
\end{malayalam}}
\flushright{\begin{Arabic}
\quranayah[55][35]
\end{Arabic}}
\flushleft{\begin{malayalam}
നിങ്ങള്‍ ഇരുവിഭാഗത്തിന്‍റെയും നേര്‍ക്ക് തീജ്വാലയും പുകയും അയക്കപ്പെടും. അപ്പോള്‍ നിങ്ങള്‍ക്ക് രക്ഷാമാര്‍ഗം സ്വീകരിക്കാനാവില്ല.
\end{malayalam}}
\flushright{\begin{Arabic}
\quranayah[55][36]
\end{Arabic}}
\flushleft{\begin{malayalam}
അപ്പോള്‍ നിങ്ങള്‍ ഇരുവിഭാഗത്തിന്‍റെയും രക്ഷിതാവിന്‍റെ അനുഗ്രഹങ്ങളില്‍ ഏതിനെയാണ് നിങ്ങള്‍ നിഷേധിക്കുന്നത്‌.
\end{malayalam}}
\flushright{\begin{Arabic}
\quranayah[55][37]
\end{Arabic}}
\flushleft{\begin{malayalam}
എന്നാല്‍ ആകാശം പൊട്ടിപ്പിളരുകയും, അത് കുഴമ്പു പോലുള്ളതും റോസ് നിറമുള്ളതും ആയിത്തീരുകയും ചെയ്താല്‍
\end{malayalam}}
\flushright{\begin{Arabic}
\quranayah[55][38]
\end{Arabic}}
\flushleft{\begin{malayalam}
അപ്പോള്‍ നിങ്ങള്‍ ഇരു വിഭാഗത്തിന്‍റെയും രക്ഷിതാവിന്‍റെ അനുഗ്രഹങ്ങളില്‍ ഏതിനെയാണ് നിങ്ങള്‍ നിഷേധിക്കുന്നത്‌?
\end{malayalam}}
\flushright{\begin{Arabic}
\quranayah[55][39]
\end{Arabic}}
\flushleft{\begin{malayalam}
ഒരു മനുഷ്യനോടോ, ജിന്നിനോടോ അന്നേ ദിവസം അവന്‍റെ പാപത്തെപ്പറ്റി അന്വേഷിക്കപ്പെടുകയില്ല.
\end{malayalam}}
\flushright{\begin{Arabic}
\quranayah[55][40]
\end{Arabic}}
\flushleft{\begin{malayalam}
അപ്പോള്‍ നിങ്ങള്‍ ഇരു വിഭാഗത്തിന്‍റെയും രക്ഷിതാവിന്‍റെ അനുഗ്രഹങ്ങളില്‍ ഏതിനെയാണ് നിങ്ങള്‍ നിഷേധിക്കുന്നത്‌?
\end{malayalam}}
\flushright{\begin{Arabic}
\quranayah[55][41]
\end{Arabic}}
\flushleft{\begin{malayalam}
കുറ്റവാളികള്‍ അവരുടെ അടയാളം കൊണ്ട് തിരിച്ചറിയപ്പെടും. എന്നിട്ട് (അവരുടെ) കുടുമകളിലും പാദങ്ങളിലും പിടിക്കപ്പെടും.
\end{malayalam}}
\flushright{\begin{Arabic}
\quranayah[55][42]
\end{Arabic}}
\flushleft{\begin{malayalam}
അപ്പോള്‍ നിങ്ങള്‍ ഇരു വിഭാഗത്തിന്‍റെയും രക്ഷിതാവിന്‍റെ അനുഗ്രഹങ്ങളില്‍ ഏതിനെയാണ് നിങ്ങള്‍ നിഷേധിക്കുന്നത്‌?
\end{malayalam}}
\flushright{\begin{Arabic}
\quranayah[55][43]
\end{Arabic}}
\flushleft{\begin{malayalam}
ഇതാകുന്നു കുറ്റവാളികള്‍ നിഷേധിച്ച് തള്ളുന്നതായ നരകം.
\end{malayalam}}
\flushright{\begin{Arabic}
\quranayah[55][44]
\end{Arabic}}
\flushleft{\begin{malayalam}
അതിന്നും തിളച്ചുപൊള്ളുന്ന ചുടുവെള്ളത്തിനുമിടക്ക് അവര്‍ ചുറ്റിത്തിരിയുന്നതാണ്‌.
\end{malayalam}}
\flushright{\begin{Arabic}
\quranayah[55][45]
\end{Arabic}}
\flushleft{\begin{malayalam}
അപ്പോള്‍ നിങ്ങള്‍ ഇരു വിഭാഗത്തിന്‍റെയും രക്ഷിതാവിന്‍റെ അനുഗ്രഹങ്ങളില്‍ ഏതിനെയാണ് നിങ്ങള്‍ നിഷേധിക്കുന്നത്‌?
\end{malayalam}}
\flushright{\begin{Arabic}
\quranayah[55][46]
\end{Arabic}}
\flushleft{\begin{malayalam}
തന്‍റെ രക്ഷിതാവിന്‍റെ സന്നിധിയെ ഭയപ്പെട്ടവന്ന് രണ്ട് സ്വര്‍ഗത്തോപ്പുകളുണ്ട്‌.
\end{malayalam}}
\flushright{\begin{Arabic}
\quranayah[55][47]
\end{Arabic}}
\flushleft{\begin{malayalam}
അപ്പോള്‍ നിങ്ങള്‍ ഇരു വിഭാഗത്തിന്‍റെയും രക്ഷിതാവിന്‍റെ അനുഗ്രഹങ്ങളില്‍ ഏതിനെയാണ് നിങ്ങള്‍ നിഷേധിക്കുന്നത്‌?
\end{malayalam}}
\flushright{\begin{Arabic}
\quranayah[55][48]
\end{Arabic}}
\flushleft{\begin{malayalam}
പല തരം സുഖഐശ്വര്യങ്ങളുള്ള രണ്ടു (സ്വര്‍ഗത്തോപ്പുകള്‍)
\end{malayalam}}
\flushright{\begin{Arabic}
\quranayah[55][49]
\end{Arabic}}
\flushleft{\begin{malayalam}
അപ്പോള്‍ നിങ്ങള്‍ ഇരു വിഭാഗത്തിന്‍റെയും രക്ഷിതാവിന്‍റെ അനുഗ്രഹങ്ങളില്‍ ഏതിനെയാണ് നിങ്ങള്‍ നിഷേധിക്കുന്നത്‌?
\end{malayalam}}
\flushright{\begin{Arabic}
\quranayah[55][50]
\end{Arabic}}
\flushleft{\begin{malayalam}
അവ രണ്ടിലും ഒഴുകികൊണ്ടിരിക്കുന്ന രണ്ടു അരുവികളുണ്ട്‌.
\end{malayalam}}
\flushright{\begin{Arabic}
\quranayah[55][51]
\end{Arabic}}
\flushleft{\begin{malayalam}
അപ്പോള്‍ നിങ്ങള്‍ ഇരു വിഭാഗത്തിന്‍റെയും രക്ഷിതാവിന്‍റെ അനുഗ്രഹങ്ങളില്‍ ഏതിനെയാണ് നിങ്ങള്‍ നിഷേധിക്കുന്നത്‌?
\end{malayalam}}
\flushright{\begin{Arabic}
\quranayah[55][52]
\end{Arabic}}
\flushleft{\begin{malayalam}
അവ രണ്ടിലും ഓരോ പഴവര്‍ഗത്തില്‍ നിന്നുമുള്ള ഈ രണ്ടു ഇനങ്ങളുണ്ട്‌.
\end{malayalam}}
\flushright{\begin{Arabic}
\quranayah[55][53]
\end{Arabic}}
\flushleft{\begin{malayalam}
അപ്പോള്‍ നിങ്ങള്‍ ഇരു വിഭാഗത്തിന്‍റെയും രക്ഷിതാവിന്‍റെ അനുഗ്രഹങ്ങളില്‍ ഏതിനെയാണ് നിങ്ങള്‍ നിഷേധിക്കുന്നത്‌?
\end{malayalam}}
\flushright{\begin{Arabic}
\quranayah[55][54]
\end{Arabic}}
\flushleft{\begin{malayalam}
അവര്‍ ചില മെത്തകളില്‍ ചാരി ഇരിക്കുന്നവരായിരിക്കും. അവയുടെ ഉള്‍ഭാഗങ്ങള്‍ കട്ടികൂടിയ പട്ടുകൊണ്ട് നിര്‍മിക്കപ്പെട്ടതാകുന്നു. ആ രണ്ട് തോപ്പുകളിലെയും കായ്കനികള്‍ താഴ്ന്നു നില്‍ക്കുകയായിരിക്കും.
\end{malayalam}}
\flushright{\begin{Arabic}
\quranayah[55][55]
\end{Arabic}}
\flushleft{\begin{malayalam}
അപ്പോള്‍ നിങ്ങള്‍ ഇരു വിഭാഗത്തിന്‍റെയും രക്ഷിതാവിന്‍റെ അനുഗ്രഹങ്ങളില്‍ ഏതിനെയാണ് നിങ്ങള്‍ നിഷേധിക്കുന്നത്‌?
\end{malayalam}}
\flushright{\begin{Arabic}
\quranayah[55][56]
\end{Arabic}}
\flushleft{\begin{malayalam}
അവയില്‍ ദൃഷ്ടി നിയന്ത്രിക്കുന്നവരായ സ്ത്രീകളുണ്ടായിരിക്കും. അവര്‍ക്ക് മുമ്പ് മനുഷ്യനോ, ജിന്നോ അവരെ സ്പര്‍ശിച്ചിട്ടില്ല.
\end{malayalam}}
\flushright{\begin{Arabic}
\quranayah[55][57]
\end{Arabic}}
\flushleft{\begin{malayalam}
അപ്പോള്‍ നിങ്ങള്‍ ഇരു വിഭാഗത്തിന്‍റെയും രക്ഷിതാവിന്‍റെ അനുഗ്രഹങ്ങളില്‍ ഏതിനെയാണ് നിങ്ങള്‍ നിഷേധിക്കുന്നത്‌?
\end{malayalam}}
\flushright{\begin{Arabic}
\quranayah[55][58]
\end{Arabic}}
\flushleft{\begin{malayalam}
അവര്‍ മാണിക്യവും പവിഴവും പോലെയായിരിക്കും.
\end{malayalam}}
\flushright{\begin{Arabic}
\quranayah[55][59]
\end{Arabic}}
\flushleft{\begin{malayalam}
അപ്പോള്‍ നിങ്ങള്‍ ഇരു വിഭാഗത്തിന്‍റെയും രക്ഷിതാവിന്‍റെ അനുഗ്രഹങ്ങളില്‍ ഏതിനെയാണ് നിങ്ങള്‍ നിഷേധിക്കുന്നത്‌?
\end{malayalam}}
\flushright{\begin{Arabic}
\quranayah[55][60]
\end{Arabic}}
\flushleft{\begin{malayalam}
നല്ല പ്രവൃത്തിക്കുള്ള പ്രതിഫലം നല്ലത് ചെയ്ത് കൊടുക്കലല്ലാതെ മറ്റു വല്ലതുമാണോ?
\end{malayalam}}
\flushright{\begin{Arabic}
\quranayah[55][61]
\end{Arabic}}
\flushleft{\begin{malayalam}
അപ്പോള്‍ നിങ്ങള്‍ ഇരു വിഭാഗത്തിന്‍റെയും രക്ഷിതാവിന്‍റെ അനുഗ്രഹങ്ങളില്‍ ഏതിനെയാണ് നിങ്ങള്‍ നിഷേധിക്കുന്നത്‌?
\end{malayalam}}
\flushright{\begin{Arabic}
\quranayah[55][62]
\end{Arabic}}
\flushleft{\begin{malayalam}
അവ രണ്ടിനും പുറമെ വേറെയും രണ്ടു സ്വര്‍ഗത്തോപ്പുകളുണ്ട്‌.
\end{malayalam}}
\flushright{\begin{Arabic}
\quranayah[55][63]
\end{Arabic}}
\flushleft{\begin{malayalam}
അപ്പോള്‍ നിങ്ങള്‍ ഇരു വിഭാഗത്തിന്‍റെയും രക്ഷിതാവിന്‍റെ അനുഗ്രഹങ്ങളില്‍ ഏതിനെയാണ് നിങ്ങള്‍ നിഷേധിക്കുന്നത്‌?
\end{malayalam}}
\flushright{\begin{Arabic}
\quranayah[55][64]
\end{Arabic}}
\flushleft{\begin{malayalam}
കടും പച്ചയണിഞ്ഞ രണ്ടുസ്വര്‍ഗത്തോപ്പുകള്‍
\end{malayalam}}
\flushright{\begin{Arabic}
\quranayah[55][65]
\end{Arabic}}
\flushleft{\begin{malayalam}
അപ്പോള്‍ നിങ്ങള്‍ ഇരു വിഭാഗത്തിന്‍റെയും രക്ഷിതാവിന്‍റെ അനുഗ്രഹങ്ങളില്‍ ഏതിനെയാണ് നിങ്ങള്‍ നിഷേധിക്കുന്നത്‌?
\end{malayalam}}
\flushright{\begin{Arabic}
\quranayah[55][66]
\end{Arabic}}
\flushleft{\begin{malayalam}
അവ രണ്ടിലും കുതിച്ചൊഴുകുന്ന രണ്ടു അരുവികളുണ്ട്‌.
\end{malayalam}}
\flushright{\begin{Arabic}
\quranayah[55][67]
\end{Arabic}}
\flushleft{\begin{malayalam}
അപ്പോള്‍ നിങ്ങള്‍ ഇരു വിഭാഗത്തിന്‍റെയും രക്ഷിതാവിന്‍റെ അനുഗ്രഹങ്ങളില്‍ ഏതിനെയാണ് നിങ്ങള്‍ നിഷേധിക്കുന്നത്‌?
\end{malayalam}}
\flushright{\begin{Arabic}
\quranayah[55][68]
\end{Arabic}}
\flushleft{\begin{malayalam}
അവ രണ്ടിലും പഴവര്‍ഗങ്ങളുണ്ട്‌. ഈന്തപ്പനകളും റുമാമ്പഴവുമുണ്ട്‌.
\end{malayalam}}
\flushright{\begin{Arabic}
\quranayah[55][69]
\end{Arabic}}
\flushleft{\begin{malayalam}
അപ്പോള്‍ നിങ്ങള്‍ ഇരു വിഭാഗത്തിന്‍റെയും രക്ഷിതാവിന്‍റെ അനുഗ്രഹങ്ങളില്‍ ഏതിനെയാണ് നിങ്ങള്‍ നിഷേധിക്കുന്നത്‌?
\end{malayalam}}
\flushright{\begin{Arabic}
\quranayah[55][70]
\end{Arabic}}
\flushleft{\begin{malayalam}
അവയില്‍ സുന്ദരികളായ ഉത്തമ തരുണികളുണ്ട്‌.
\end{malayalam}}
\flushright{\begin{Arabic}
\quranayah[55][71]
\end{Arabic}}
\flushleft{\begin{malayalam}
അപ്പോള്‍ നിങ്ങള്‍ ഇരു വിഭാഗത്തിന്‍റെയും രക്ഷിതാവിന്‍റെ അനുഗ്രഹങ്ങളില്‍ ഏതിനെയാണ് നിങ്ങള്‍ നിഷേധിക്കുന്നത്‌?
\end{malayalam}}
\flushright{\begin{Arabic}
\quranayah[55][72]
\end{Arabic}}
\flushleft{\begin{malayalam}
കൂടാരങ്ങളില്‍ ഒതുക്കി നിര്‍ത്തപ്പെട്ട വെളുത്ത തരുണികള്‍!
\end{malayalam}}
\flushright{\begin{Arabic}
\quranayah[55][73]
\end{Arabic}}
\flushleft{\begin{malayalam}
അപ്പോള്‍ നിങ്ങള്‍ ഇരു വിഭാഗത്തിന്‍റെയും രക്ഷിതാവിന്‍റെ അനുഗ്രഹങ്ങളില്‍ ഏതിനെയാണ് നിങ്ങള്‍ നിഷേധിക്കുന്നത്‌?
\end{malayalam}}
\flushright{\begin{Arabic}
\quranayah[55][74]
\end{Arabic}}
\flushleft{\begin{malayalam}
അവര്‍ക്ക് മുമ്പ് മനുഷ്യനോ ജിന്നോ അവരെ സ്പര്‍ശിച്ചിട്ടില്ല.
\end{malayalam}}
\flushright{\begin{Arabic}
\quranayah[55][75]
\end{Arabic}}
\flushleft{\begin{malayalam}
അപ്പോള്‍ നിങ്ങള്‍ ഇരു വിഭാഗത്തിന്‍റെയും രക്ഷിതാവിന്‍റെ അനുഗ്രഹങ്ങളില്‍ ഏതിനെയാണ് നിങ്ങള്‍ നിഷേധിക്കുന്നത്‌?
\end{malayalam}}
\flushright{\begin{Arabic}
\quranayah[55][76]
\end{Arabic}}
\flushleft{\begin{malayalam}
പച്ചനിറമുള്ള തലയണകളിലും അഴകുള്ള പരവതാനികളിലും ചാരി കിടക്കുന്നവര്‍ ആയിരിക്കും അവര്‍.
\end{malayalam}}
\flushright{\begin{Arabic}
\quranayah[55][77]
\end{Arabic}}
\flushleft{\begin{malayalam}
അപ്പോള്‍ നിങ്ങള്‍ ഇരു വിഭാഗത്തിന്‍റെയും രക്ഷിതാവിന്‍റെ അനുഗ്രഹങ്ങളില്‍ ഏതിനെയാണ് നിങ്ങള്‍ നിഷേധിക്കുന്നത്‌?
\end{malayalam}}
\flushright{\begin{Arabic}
\quranayah[55][78]
\end{Arabic}}
\flushleft{\begin{malayalam}
മഹത്വവും ഔദാര്യവും ഉള്ളവനായ നിന്‍റെ രക്ഷിതാവിന്‍റെ നാമം ഉല്‍കൃഷ്ടമായിരിക്കുന്നു.
\end{malayalam}}
\chapter{\textmalayalam{അല്‍ വാഖിഅ ( സംഭവം )}}
\begin{Arabic}
\Huge{\centerline{\basmalah}}\end{Arabic}
\flushright{\begin{Arabic}
\quranayah[56][1]
\end{Arabic}}
\flushleft{\begin{malayalam}
ആ സംഭവം സംഭവിച്ച് കഴിഞ്ഞാല്‍.
\end{malayalam}}
\flushright{\begin{Arabic}
\quranayah[56][2]
\end{Arabic}}
\flushleft{\begin{malayalam}
അതിന്‍റെ സംഭവ്യതയെ നിഷേധിക്കുന്ന ആരും ഉണ്ടായിരിക്കുകയില്ല.
\end{malayalam}}
\flushright{\begin{Arabic}
\quranayah[56][3]
\end{Arabic}}
\flushleft{\begin{malayalam}
(ആ സംഭവം, ചിലരെ) താഴ്ത്തുന്നതും (ചിലരെ) ഉയര്‍ത്തുന്നതുമായിരിക്കും.
\end{malayalam}}
\flushright{\begin{Arabic}
\quranayah[56][4]
\end{Arabic}}
\flushleft{\begin{malayalam}
ഭൂമി കിടുകിടാ വിറപ്പിക്കപ്പെടുകയും,
\end{malayalam}}
\flushright{\begin{Arabic}
\quranayah[56][5]
\end{Arabic}}
\flushleft{\begin{malayalam}
പര്‍വ്വതങ്ങള്‍ ഇടിച്ച് പൊടിയാക്കപ്പെടുകയും;
\end{malayalam}}
\flushright{\begin{Arabic}
\quranayah[56][6]
\end{Arabic}}
\flushleft{\begin{malayalam}
അങ്ങനെ അത് പാറിപ്പറക്കുന്ന ധൂളിയായിത്തീരുകയും,
\end{malayalam}}
\flushright{\begin{Arabic}
\quranayah[56][7]
\end{Arabic}}
\flushleft{\begin{malayalam}
നിങ്ങള്‍ മൂന്ന് തരക്കാരായിത്തീരുകയും ചെയ്യുന്ന സന്ദര്‍ഭമത്രെ അത്‌.
\end{malayalam}}
\flushright{\begin{Arabic}
\quranayah[56][8]
\end{Arabic}}
\flushleft{\begin{malayalam}
അപ്പോള്‍ ഒരു വിഭാഗം വലതുപക്ഷക്കാര്‍. എന്താണ് ഈ വലതുപക്ഷക്കാരുടെ അവസ്ഥ!
\end{malayalam}}
\flushright{\begin{Arabic}
\quranayah[56][9]
\end{Arabic}}
\flushleft{\begin{malayalam}
മറ്റൊരു വിഭാഗം ഇടതുപക്ഷക്കാര്‍. എന്താണ് ഈ ഇടതുപക്ഷക്കാരുടെ അവസ്ഥ!
\end{malayalam}}
\flushright{\begin{Arabic}
\quranayah[56][10]
\end{Arabic}}
\flushleft{\begin{malayalam}
(സത്യവിശ്വാസത്തിലും സല്‍പ്രവൃത്തികളിലും) മുന്നേറിയവര്‍ (പരലോകത്തും) മുന്നോക്കക്കാര്‍ തന്നെ.
\end{malayalam}}
\flushright{\begin{Arabic}
\quranayah[56][11]
\end{Arabic}}
\flushleft{\begin{malayalam}
അവരാകുന്നു സാമീപ്യം നല്‍കപ്പെട്ടവര്‍.
\end{malayalam}}
\flushright{\begin{Arabic}
\quranayah[56][12]
\end{Arabic}}
\flushleft{\begin{malayalam}
സുഖാനുഭൂതികളുടെ സ്വര്‍ഗത്തോപ്പുകളില്‍.
\end{malayalam}}
\flushright{\begin{Arabic}
\quranayah[56][13]
\end{Arabic}}
\flushleft{\begin{malayalam}
പൂര്‍വ്വികന്‍മാരില്‍ നിന്ന് ഒരു വിഭാഗവും
\end{malayalam}}
\flushright{\begin{Arabic}
\quranayah[56][14]
\end{Arabic}}
\flushleft{\begin{malayalam}
പില്‍ക്കാലക്കാരില്‍ നിന്ന് കുറച്ചു പേരുമത്രെ ഇവര്‍.
\end{malayalam}}
\flushright{\begin{Arabic}
\quranayah[56][15]
\end{Arabic}}
\flushleft{\begin{malayalam}
സ്വര്‍ണനൂലുകൊണ്ട് മെടഞ്ഞുണ്ടാക്കപ്പെട്ട കട്ടിലുകളില്‍ ആയിരിക്കും. അവര്‍.
\end{malayalam}}
\flushright{\begin{Arabic}
\quranayah[56][16]
\end{Arabic}}
\flushleft{\begin{malayalam}
അവയില്‍ അവര്‍ പരസ്പരം അഭിമുഖമായി ചാരിയിരിക്കുന്നവരായിരിക്കും.
\end{malayalam}}
\flushright{\begin{Arabic}
\quranayah[56][17]
\end{Arabic}}
\flushleft{\begin{malayalam}
നിത്യജീവിതം നല്‍കപ്പെട്ട ബാലന്‍മാര്‍ അവരുടെ ഇടയില്‍ ചുറ്റി നടക്കും.
\end{malayalam}}
\flushright{\begin{Arabic}
\quranayah[56][18]
\end{Arabic}}
\flushleft{\begin{malayalam}
കോപ്പകളും കൂജകളും ശുദ്ധമായ ഉറവു ജലം നിറച്ച പാനപാത്രവും കൊണ്ട്‌.
\end{malayalam}}
\flushright{\begin{Arabic}
\quranayah[56][19]
\end{Arabic}}
\flushleft{\begin{malayalam}
അതു (കുടിക്കുക) മൂലം അവര്‍ക്ക് തലവേദനയുണ്ടാവുകയോ, ലഹരി ബാധിക്കുകയോ ഇല്ല.
\end{malayalam}}
\flushright{\begin{Arabic}
\quranayah[56][20]
\end{Arabic}}
\flushleft{\begin{malayalam}
അവര്‍ ഇഷ്ടപ്പെട്ടു തെരഞ്ഞെടുക്കുന്ന തരത്തില്‍ പെട്ട പഴവര്‍ഗങ്ങളും.
\end{malayalam}}
\flushright{\begin{Arabic}
\quranayah[56][21]
\end{Arabic}}
\flushleft{\begin{malayalam}
അവര്‍ കൊതിക്കുന്ന തരത്തില്‍ പെട്ട പക്ഷിമാംസവും കൊണ്ട് (അവര്‍ ചുറ്റി നടക്കും.)
\end{malayalam}}
\flushright{\begin{Arabic}
\quranayah[56][22]
\end{Arabic}}
\flushleft{\begin{malayalam}
വിശാലമായ നയനങ്ങളുള്ള വെളുത്ത തരുണികളും. (അവര്‍ക്കുണ്ട്‌.)
\end{malayalam}}
\flushright{\begin{Arabic}
\quranayah[56][23]
\end{Arabic}}
\flushleft{\begin{malayalam}
(ചിപ്പികളില്‍) ഒളിച്ചു വെക്കപ്പെട്ട മുത്തുപോലെയുള്ളവര്‍,
\end{malayalam}}
\flushright{\begin{Arabic}
\quranayah[56][24]
\end{Arabic}}
\flushleft{\begin{malayalam}
അവര്‍ പ്രവര്‍ത്തിച്ച് കൊണ്ടിരുന്നതിനുള്ള പ്രതിഫലമായികൊണ്ടാണ് (അതെല്ലാം നല്‍കപ്പെടുന്നത്‌)
\end{malayalam}}
\flushright{\begin{Arabic}
\quranayah[56][25]
\end{Arabic}}
\flushleft{\begin{malayalam}
അനാവശ്യവാക്കോ കുറ്റപ്പെടുത്തലോ അവര്‍ അവിടെ വെച്ച് കേള്‍ക്കുകയില്ല.
\end{malayalam}}
\flushright{\begin{Arabic}
\quranayah[56][26]
\end{Arabic}}
\flushleft{\begin{malayalam}
സമാധാനം! സമാധാനം! എന്നുള്ള വാക്കല്ലാതെ.
\end{malayalam}}
\flushright{\begin{Arabic}
\quranayah[56][27]
\end{Arabic}}
\flushleft{\begin{malayalam}
വലതുപക്ഷക്കാര്‍! എന്താണീ വലതുപക്ഷക്കാരുടെ അവസ്ഥ!
\end{malayalam}}
\flushright{\begin{Arabic}
\quranayah[56][28]
\end{Arabic}}
\flushleft{\begin{malayalam}
മുള്ളിലാത്ത ഇലന്തമരം,
\end{malayalam}}
\flushright{\begin{Arabic}
\quranayah[56][29]
\end{Arabic}}
\flushleft{\begin{malayalam}
അടുക്കടുക്കായി കുലകളുള്ള വാഴ,
\end{malayalam}}
\flushright{\begin{Arabic}
\quranayah[56][30]
\end{Arabic}}
\flushleft{\begin{malayalam}
വിശാലമായ തണല്‍,
\end{malayalam}}
\flushright{\begin{Arabic}
\quranayah[56][31]
\end{Arabic}}
\flushleft{\begin{malayalam}
സദാ ഒഴുക്കപ്പെട്ടു കൊണ്ടിരിക്കുന്ന വെള്ളം,
\end{malayalam}}
\flushright{\begin{Arabic}
\quranayah[56][32]
\end{Arabic}}
\flushleft{\begin{malayalam}
ധാരാളം പഴവര്‍ഗങ്ങള്‍,
\end{malayalam}}
\flushright{\begin{Arabic}
\quranayah[56][33]
\end{Arabic}}
\flushleft{\begin{malayalam}
നിലച്ചു പോവാത്തതും തടസ്സപ്പെട്ടുപോവാത്തതുമായ
\end{malayalam}}
\flushright{\begin{Arabic}
\quranayah[56][34]
\end{Arabic}}
\flushleft{\begin{malayalam}
ഉയര്‍ന്നമെത്തകള്‍ എന്നീ സുഖാനുഭവങ്ങളിലായിരിക്കും അവര്‍.
\end{malayalam}}
\flushright{\begin{Arabic}
\quranayah[56][35]
\end{Arabic}}
\flushleft{\begin{malayalam}
തീര്‍ച്ചയായും അവരെ (സ്വര്‍ഗസ്ത്രീകളെ) നാം ഒരു പ്രത്യേക പ്രകൃതിയോടെ സൃഷ്ടിച്ചുണ്ടാക്കിയിരിക്കുകയാണ്‌.
\end{malayalam}}
\flushright{\begin{Arabic}
\quranayah[56][36]
\end{Arabic}}
\flushleft{\begin{malayalam}
അങ്ങനെ അവരെ നാം കന്യകമാരാക്കിയിരിക്കുന്നു.
\end{malayalam}}
\flushright{\begin{Arabic}
\quranayah[56][37]
\end{Arabic}}
\flushleft{\begin{malayalam}
സ്നേഹവതികളും സമപ്രായക്കാരും ആക്കിയിരിക്കുന്നു.
\end{malayalam}}
\flushright{\begin{Arabic}
\quranayah[56][38]
\end{Arabic}}
\flushleft{\begin{malayalam}
വലതുപക്ഷക്കാര്‍ക്ക് വേണ്ടിയത്രെ അത്‌.
\end{malayalam}}
\flushright{\begin{Arabic}
\quranayah[56][39]
\end{Arabic}}
\flushleft{\begin{malayalam}
പൂര്‍വ്വികന്‍മാരില്‍ നിന്ന് ഒരു വിഭാഗവും
\end{malayalam}}
\flushright{\begin{Arabic}
\quranayah[56][40]
\end{Arabic}}
\flushleft{\begin{malayalam}
പിന്‍ഗാമികളില്‍ നിന്ന് ഒരു വിഭാഗവും ആയിരിക്കും അവര്‍.
\end{malayalam}}
\flushright{\begin{Arabic}
\quranayah[56][41]
\end{Arabic}}
\flushleft{\begin{malayalam}
ഇടതുപക്ഷക്കാര്‍, എന്താണീ ഇടതുപക്ഷക്കാരുടെ അവസ്ഥ!
\end{malayalam}}
\flushright{\begin{Arabic}
\quranayah[56][42]
\end{Arabic}}
\flushleft{\begin{malayalam}
തുളച്ചു കയറുന്ന ഉഷ്ണകാറ്റ്‌, ചുട്ടുതിളക്കുന്ന വെള്ളം,
\end{malayalam}}
\flushright{\begin{Arabic}
\quranayah[56][43]
\end{Arabic}}
\flushleft{\begin{malayalam}
കരിമ്പുകയുടെ തണല്‍
\end{malayalam}}
\flushright{\begin{Arabic}
\quranayah[56][44]
\end{Arabic}}
\flushleft{\begin{malayalam}
തണുപ്പുള്ളതോ, സുഖദായകമോ അല്ലാത്ത (എന്നീ ദുരിതങ്ങളിലായിരിക്കും അവര്‍.)
\end{malayalam}}
\flushright{\begin{Arabic}
\quranayah[56][45]
\end{Arabic}}
\flushleft{\begin{malayalam}
എന്തുകൊണ്ടെന്നാല്‍ തീര്‍ച്ചയായും അവര്‍ അതിനു മുമ്പ് സുഖലോലുപന്‍മാരായിരുന്നു.
\end{malayalam}}
\flushright{\begin{Arabic}
\quranayah[56][46]
\end{Arabic}}
\flushleft{\begin{malayalam}
അവര്‍ ഗുരുതരമായ പാപത്തില്‍ ശഠിച്ചുനില്‍ക്കുന്നവരുമായിരുന്നു.
\end{malayalam}}
\flushright{\begin{Arabic}
\quranayah[56][47]
\end{Arabic}}
\flushleft{\begin{malayalam}
അവര്‍ ഇപ്രകാരം പറയുകയും ചെയ്തിരുന്നു: ഞങ്ങള്‍ മരിച്ചിട്ട് മണ്ണും അസ്ഥിശകലങ്ങളുമായിക്കഴിഞ്ഞിട്ടാണോ ഞങ്ങള്‍ ഉയിര്‍ത്തെഴുന്നേല്‍പിക്കപ്പെടാന്‍ പോകുന്നത്‌?
\end{malayalam}}
\flushright{\begin{Arabic}
\quranayah[56][48]
\end{Arabic}}
\flushleft{\begin{malayalam}
ഞങ്ങളുടെ പൂര്‍വ്വികരായ പിതാക്കളും (ഉയിര്‍ത്തെഴുന്നേല്‍പിക്കപ്പെടുമെന്നോ?)
\end{malayalam}}
\flushright{\begin{Arabic}
\quranayah[56][49]
\end{Arabic}}
\flushleft{\begin{malayalam}
നീ പറയുക: തീര്‍ച്ചയായും പൂര്‍വ്വികരും പില്‍ക്കാലക്കാരും എല്ലാം-
\end{malayalam}}
\flushright{\begin{Arabic}
\quranayah[56][50]
\end{Arabic}}
\flushleft{\begin{malayalam}
ഒരു നിശ്ചിത ദിവസത്തെ കൃത്യമായ ഒരു അവധിക്ക് ഒരുമിച്ചുകൂട്ടപ്പെടുന്നവര്‍ തന്നെയാകുന്നു.
\end{malayalam}}
\flushright{\begin{Arabic}
\quranayah[56][51]
\end{Arabic}}
\flushleft{\begin{malayalam}
എന്നിട്ട്‌, ഹേ; സത്യനിഷേധികളായ ദുര്‍മാര്‍ഗികളേ,
\end{malayalam}}
\flushright{\begin{Arabic}
\quranayah[56][52]
\end{Arabic}}
\flushleft{\begin{malayalam}
തീര്‍ച്ചയായും നിങ്ങള്‍ ഒരു വൃക്ഷത്തില്‍ നിന്ന് അതായത് സഖ്ഖൂമില്‍ നിന്ന് ഭക്ഷിക്കുന്നവരാകുന്നു.
\end{malayalam}}
\flushright{\begin{Arabic}
\quranayah[56][53]
\end{Arabic}}
\flushleft{\begin{malayalam}
അങ്ങനെ അതില്‍ നിന്ന് വയറുകള്‍ നിറക്കുന്നവരും,
\end{malayalam}}
\flushright{\begin{Arabic}
\quranayah[56][54]
\end{Arabic}}
\flushleft{\begin{malayalam}
അതിന്‍റെ മീതെ തിളച്ചുപൊള്ളുന്ന വെള്ളത്തില്‍ നിന്ന് കുടിക്കുന്നവരുമാകുന്നു.
\end{malayalam}}
\flushright{\begin{Arabic}
\quranayah[56][55]
\end{Arabic}}
\flushleft{\begin{malayalam}
അങ്ങനെ ദാഹിച്ചു വലഞ്ഞ ഒട്ടകം കുടിക്കുന്നപോലെ കുടിക്കുന്നവരാകുന്നു.
\end{malayalam}}
\flushright{\begin{Arabic}
\quranayah[56][56]
\end{Arabic}}
\flushleft{\begin{malayalam}
ഇതായിരിക്കും പ്രതിഫലത്തിന്‍റെ നാളില്‍ അവര്‍ക്കുള്ള സല്‍ക്കാരം.
\end{malayalam}}
\flushright{\begin{Arabic}
\quranayah[56][57]
\end{Arabic}}
\flushleft{\begin{malayalam}
നാമാണ് നിങ്ങളെ സൃഷ്ടിച്ചിരിക്കുന്നത്‌. എന്നിരിക്കെ നിങ്ങളെന്താണ് (എന്‍റെ സന്ദേശങ്ങളെ) സത്യമായി അംഗീകരിക്കാത്തത്‌?
\end{malayalam}}
\flushright{\begin{Arabic}
\quranayah[56][58]
\end{Arabic}}
\flushleft{\begin{malayalam}
അപ്പോള്‍ നിങ്ങള്‍ സ്രവിക്കുന്ന ശുക്ലത്തെപ്പറ്റി നിങ്ങള്‍ ചിന്തിച്ചു നോക്കിയിട്ടുണ്ടോ?
\end{malayalam}}
\flushright{\begin{Arabic}
\quranayah[56][59]
\end{Arabic}}
\flushleft{\begin{malayalam}
നിങ്ങളാണോ അത് സൃഷ്ടിച്ചുണ്ടാക്കുന്നത്‌. അതല്ല, നാമാണോ സൃഷ്ടികര്‍ത്താവ്‌?
\end{malayalam}}
\flushright{\begin{Arabic}
\quranayah[56][60]
\end{Arabic}}
\flushleft{\begin{malayalam}
നാം നിങ്ങള്‍ക്കിടയില്‍ മരണം കണക്കാക്കിയിരിക്കുന്നു. നാം ഒരിക്കലും തോല്‍പിക്കപ്പെടുന്നവനല്ല.
\end{malayalam}}
\flushright{\begin{Arabic}
\quranayah[56][61]
\end{Arabic}}
\flushleft{\begin{malayalam}
(നിങ്ങള്‍ക്കു) പകരം നിങ്ങളെ പോലുള്ളവരെ കൊണ്ടുവരികയും. നിങ്ങള്‍ക്ക് അറിവില്ലാത്ത വിധത്തില്‍ നിങ്ങളെ (വീണ്ടും) സൃഷ്ടിച്ചുണ്ടാക്കുകയും ചെയ്യുന്ന കാര്യത്തില്‍
\end{malayalam}}
\flushright{\begin{Arabic}
\quranayah[56][62]
\end{Arabic}}
\flushleft{\begin{malayalam}
ആദ്യതവണ സൃഷ്ടിക്കപ്പെട്ടതിനെപ്പറ്റി തീര്‍ച്ചയായും നിങ്ങള്‍ മനസ്സിലാക്കിയിട്ടുണ്ട്‌. എന്നിട്ടും നിങ്ങള്‍ എന്തുകൊണ്ട് ആലോചിച്ചു നോക്കുന്നില്ല.
\end{malayalam}}
\flushright{\begin{Arabic}
\quranayah[56][63]
\end{Arabic}}
\flushleft{\begin{malayalam}
എന്നാല്‍ നിങ്ങള്‍ കൃഷി ചെയ്യുന്നതിനെ പറ്റി നിങ്ങള്‍ ചിന്തിച്ചു നോക്കിയിട്ടുണ്ടോ?
\end{malayalam}}
\flushright{\begin{Arabic}
\quranayah[56][64]
\end{Arabic}}
\flushleft{\begin{malayalam}
നിങ്ങളാണോ അത് മുളപ്പിച്ചു വളര്‍ത്തുന്നത്‌. അതല്ല നാമാണോ, അത് മുളപ്പിച്ച് വളര്‍ത്തുന്നവന്‍?
\end{malayalam}}
\flushright{\begin{Arabic}
\quranayah[56][65]
\end{Arabic}}
\flushleft{\begin{malayalam}
നാം ഉദ്ദേശിച്ചിരുന്നെങ്കില്‍ അത് (വിള) നാം തുരുമ്പാക്കിത്തീര്‍ക്കുമായിരുന്നു. അപ്പോള്‍ നിങ്ങള്‍ അതിശയപ്പെട്ടു പറഞ്ഞുകൊണേ്ടയിരിക്കുമായിരന്നു;
\end{malayalam}}
\flushright{\begin{Arabic}
\quranayah[56][66]
\end{Arabic}}
\flushleft{\begin{malayalam}
തീര്‍ച്ചയായും ഞങ്ങള്‍ കടബാധിതര്‍ തന്നെയാകുന്നു.
\end{malayalam}}
\flushright{\begin{Arabic}
\quranayah[56][67]
\end{Arabic}}
\flushleft{\begin{malayalam}
അല്ല, ഞങ്ങള്‍ (ഉപജീവന മാര്‍ഗം) തടയപ്പെട്ടവരാകുന്നു എന്ന്‌.
\end{malayalam}}
\flushright{\begin{Arabic}
\quranayah[56][68]
\end{Arabic}}
\flushleft{\begin{malayalam}
ഇനി, നിങ്ങള്‍ കുടിക്കുന്ന വെള്ളത്തെപ്പറ്റി നിങ്ങള്‍ ചിന്തിച്ചു നോക്കിയിട്ടുണ്ടോ?
\end{malayalam}}
\flushright{\begin{Arabic}
\quranayah[56][69]
\end{Arabic}}
\flushleft{\begin{malayalam}
നിങ്ങളാണോ അത് മേഘത്തിന്‍ നിന്ന് ഇറക്കിയത്‌? അതല്ല, നാമാണോ ഇറക്കിയവന്‍?.
\end{malayalam}}
\flushright{\begin{Arabic}
\quranayah[56][70]
\end{Arabic}}
\flushleft{\begin{malayalam}
നാം ഉദ്ദേശിച്ചിരുന്നെങ്കില്‍ അത് നാം ദുസ്സ്വാദുള്ള ഉപ്പുവെള്ളമാക്കുമായിരുന്നു. എന്നിരിക്കെ നിങ്ങള്‍ നന്ദികാണിക്കാത്തതെന്താണ്‌?
\end{malayalam}}
\flushright{\begin{Arabic}
\quranayah[56][71]
\end{Arabic}}
\flushleft{\begin{malayalam}
നിങ്ങള്‍ ഉരസികത്തിക്കുന്നതായ തീയിനെ പറ്റി നിങ്ങള്‍ ചിന്തിച്ചു നോക്കിയിട്ടുണ്ടോ?
\end{malayalam}}
\flushright{\begin{Arabic}
\quranayah[56][72]
\end{Arabic}}
\flushleft{\begin{malayalam}
നിങ്ങളാണോ അതിന്‍റെ മരം സൃഷ്ടിച്ചുണ്ടാക്കിയത്‌? അതല്ല നാമാണോ സൃഷ്ടിച്ചുണ്ടാക്കിയവന്‍?
\end{malayalam}}
\flushright{\begin{Arabic}
\quranayah[56][73]
\end{Arabic}}
\flushleft{\begin{malayalam}
നാം അതിനെ ഒരു ചിന്താവിഷയമാക്കിയിരിക്കുന്നു. ദരിദ്രരായ സഞ്ചാരികള്‍ക്ക് ഒരു ജീവിതസൌകര്യവും.
\end{malayalam}}
\flushright{\begin{Arabic}
\quranayah[56][74]
\end{Arabic}}
\flushleft{\begin{malayalam}
ആകയാല്‍ നിന്‍റെ മഹാനായ രക്ഷിതാവിന്‍റെ നാമത്തെ നീ പ്രകീര്‍ത്തിക്കുക.
\end{malayalam}}
\flushright{\begin{Arabic}
\quranayah[56][75]
\end{Arabic}}
\flushleft{\begin{malayalam}
അല്ല, നക്ഷത്രങ്ങളുടെ അസ്തമന സ്ഥാനങ്ങളെകൊണ്ട് ഞാന്‍ സത്യം ചെയ്തു പറയുന്നു.
\end{malayalam}}
\flushright{\begin{Arabic}
\quranayah[56][76]
\end{Arabic}}
\flushleft{\begin{malayalam}
തീര്‍ച്ചയായും, നിങ്ങള്‍ക്കറിയാമെങ്കില്‍, അതൊരു വമ്പിച്ച സത്യം തന്നെയാണ്‌.
\end{malayalam}}
\flushright{\begin{Arabic}
\quranayah[56][77]
\end{Arabic}}
\flushleft{\begin{malayalam}
തീര്‍ച്ചയായും ഇത് ആദരണീയമായ ഒരു ഖുര്‍ആന്‍ തന്നെയാകുന്നു.
\end{malayalam}}
\flushright{\begin{Arabic}
\quranayah[56][78]
\end{Arabic}}
\flushleft{\begin{malayalam}
ഭദ്രമായി സൂക്ഷിക്കപ്പെട്ട ഒരു രേഖയിലാകുന്നു അത്‌.
\end{malayalam}}
\flushright{\begin{Arabic}
\quranayah[56][79]
\end{Arabic}}
\flushleft{\begin{malayalam}
പരിശുദ്ധി നല്‍കപ്പെട്ടവരല്ലാതെ അത് സ്പര്‍ശിക്കുകയില്ല.
\end{malayalam}}
\flushright{\begin{Arabic}
\quranayah[56][80]
\end{Arabic}}
\flushleft{\begin{malayalam}
ലോകരക്ഷിതാവിങ്കല്‍ നിന്ന് അവതരിപ്പിക്കപ്പെട്ടതത്രെ അത്‌.
\end{malayalam}}
\flushright{\begin{Arabic}
\quranayah[56][81]
\end{Arabic}}
\flushleft{\begin{malayalam}
അപ്പോള്‍ ഈ വര്‍ത്തമാനത്തിന്‍റെ കാര്യത്തിലാണോ നിങ്ങള്‍ പുറംപൂച്ച് കാണിക്കുന്നത്‌?
\end{malayalam}}
\flushright{\begin{Arabic}
\quranayah[56][82]
\end{Arabic}}
\flushleft{\begin{malayalam}
സത്യത്തെ നിഷേധിക്കുക എന്നത് നിങ്ങള്‍ നിങ്ങളുടെ വിഹിതമാക്കുകയാണോ?
\end{malayalam}}
\flushright{\begin{Arabic}
\quranayah[56][83]
\end{Arabic}}
\flushleft{\begin{malayalam}
എന്നാല്‍ അത് (ജീവന്‍) തൊണ്ടക്കുഴിയില്‍ എത്തുമ്പോള്‍ എന്തുകൊണ്ടാണ് (നിങ്ങള്‍ക്കത് പിടിച്ചു നിര്‍ത്താനാകാത്തത്‌?)
\end{malayalam}}
\flushright{\begin{Arabic}
\quranayah[56][84]
\end{Arabic}}
\flushleft{\begin{malayalam}
നിങ്ങള്‍ അന്നേരത്ത് നോക്കിക്കൊണ്ടിരിക്കുമല്ലോ.
\end{malayalam}}
\flushright{\begin{Arabic}
\quranayah[56][85]
\end{Arabic}}
\flushleft{\begin{malayalam}
നാമാണ് ആ വ്യക്തിയെ സംബന്ധിച്ചിടത്തോളം നിങ്ങളെക്കാളും അടുത്തവന്‍. പക്ഷെ നിങ്ങള്‍ കണ്ടറിയുന്നില്ല.
\end{malayalam}}
\flushright{\begin{Arabic}
\quranayah[56][86]
\end{Arabic}}
\flushleft{\begin{malayalam}
അപ്പോള്‍ നിങ്ങള്‍ (ദൈവിക നിയമത്തിന്‌) വിധേയരല്ലാത്തവരാണെങ്കില്‍
\end{malayalam}}
\flushright{\begin{Arabic}
\quranayah[56][87]
\end{Arabic}}
\flushleft{\begin{malayalam}
നിങ്ങള്‍ക്കെന്തുകൊണ്ട് അത് (ജീവന്‍) മടക്കി എടുക്കാനാവുന്നില്ല; നിങ്ങള്‍ സത്യവാദികളാണെങ്കില്‍.
\end{malayalam}}
\flushright{\begin{Arabic}
\quranayah[56][88]
\end{Arabic}}
\flushleft{\begin{malayalam}
അപ്പോള്‍ അവന്‍ (മരിച്ചവന്‍) സാമീപ്യം സിദ്ധിച്ചവരില്‍ പെട്ടവനാണെങ്കില്‍-
\end{malayalam}}
\flushright{\begin{Arabic}
\quranayah[56][89]
\end{Arabic}}
\flushleft{\begin{malayalam}
(അവന്ന്‌) ആശ്വാസവും വിശിഷ്ടമായ ഉപജീവനവും സുഖാനുഭൂതിയുടെ സ്വര്‍ഗത്തോപ്പും ഉണ്ടായിരിക്കും.
\end{malayalam}}
\flushright{\begin{Arabic}
\quranayah[56][90]
\end{Arabic}}
\flushleft{\begin{malayalam}
എന്നാല്‍ അവന്‍ വലതുപക്ഷക്കാരില്‍ പെട്ടവനാണെങ്കിലോ,
\end{malayalam}}
\flushright{\begin{Arabic}
\quranayah[56][91]
\end{Arabic}}
\flushleft{\begin{malayalam}
വലതുപക്ഷക്കാരില്‍പെട്ട നിനക്ക് സമാധാനം എന്നായിരിക്കും (അവന്നു ലഭിക്കുന്ന അഭിവാദ്യം)
\end{malayalam}}
\flushright{\begin{Arabic}
\quranayah[56][92]
\end{Arabic}}
\flushleft{\begin{malayalam}
ഇനി അവന്‍ ദുര്‍മാര്‍ഗികളായ സത്യനിഷേധികളില്‍ പെട്ടവനാണെങ്കിലോ,
\end{malayalam}}
\flushright{\begin{Arabic}
\quranayah[56][93]
\end{Arabic}}
\flushleft{\begin{malayalam}
ചുട്ടുതിളക്കുന്ന വെള്ളം കൊണ്ടുള്ള സല്‍ക്കാരവും
\end{malayalam}}
\flushright{\begin{Arabic}
\quranayah[56][94]
\end{Arabic}}
\flushleft{\begin{malayalam}
നരകത്തില്‍ വെച്ചുള്ള ചുട്ടെരിക്കലുമാണ്‌. (അവന്നുള്ളത്‌.)
\end{malayalam}}
\flushright{\begin{Arabic}
\quranayah[56][95]
\end{Arabic}}
\flushleft{\begin{malayalam}
തീര്‍ച്ചയായും ഇതു തന്നെയാണ് ഉറപ്പുള്ള യാഥാര്‍ത്ഥ്യം.
\end{malayalam}}
\flushright{\begin{Arabic}
\quranayah[56][96]
\end{Arabic}}
\flushleft{\begin{malayalam}
ആകയാല്‍ നീ നിന്‍റെ മഹാനായ രക്ഷിതാവിന്‍റെ നാമം പ്രകീര്‍ത്തിക്കുക.
\end{malayalam}}
\chapter{\textmalayalam{ഹദീദ്  ( ഇരുമ്പ് )}}
\begin{Arabic}
\Huge{\centerline{\basmalah}}\end{Arabic}
\flushright{\begin{Arabic}
\quranayah[57][1]
\end{Arabic}}
\flushleft{\begin{malayalam}
ആകാശങ്ങളിലും ഭൂമിയിലുമുള്ളതെല്ലാം അല്ലാഹുവിന് പ്രകീര്‍ത്തനം ചെയ്തിരിക്കുന്നു. അവന്‍ പ്രതാപിയും യുക്തിമാനുമത്രെ.
\end{malayalam}}
\flushright{\begin{Arabic}
\quranayah[57][2]
\end{Arabic}}
\flushleft{\begin{malayalam}
അവന്നാകുന്നു ആകാശങ്ങളുടെയും ഭൂമിയുടെയും ആധിപത്യം. അവന്‍ ജീവിപ്പിക്കുകയും മരിപ്പിക്കുകയും ചെയ്യുന്നു. അവന്‍ സര്‍വ്വകാര്യത്തിനും കഴിവുള്ളവനുമാണ്‌.
\end{malayalam}}
\flushright{\begin{Arabic}
\quranayah[57][3]
\end{Arabic}}
\flushleft{\begin{malayalam}
അവന്‍ ആദിയും അന്തിമനും പ്രത്യക്ഷമായവനും പരോക്ഷമായവനുമാണ്‌. അവന്‍ സര്‍വ്വകാര്യങ്ങളെക്കുറിച്ചും അറിവുള്ളവനുമാണ്‌.
\end{malayalam}}
\flushright{\begin{Arabic}
\quranayah[57][4]
\end{Arabic}}
\flushleft{\begin{malayalam}
ആകാശങ്ങളും ഭൂമിയും ആറുദിവസങ്ങളിലായി സൃഷ്ടിച്ചവനാണ് അവന്‍. പിന്നീട് അവന്‍ സിംഹാസനസ്ഥനായി. ഭൂമിയില്‍ പ്രവേശിക്കുന്നതും അതില്‍ നിന്ന് പുറത്തു വരുന്നതും, ആകാശത്ത് നിന്ന് ഇറങ്ങുന്നതും അതിലേക്ക് കയറിച്ചെല്ലുന്നതും അവന്‍ അറിഞ്ഞ് കൊണ്ടിരിക്കുന്നു. നിങ്ങള്‍ എവിടെയായിരുന്നാലും അവന്‍ നിങ്ങളുടെ കൂടെയുണ്ട് താനും. അല്ലാഹു നിങ്ങള്‍ പ്രവര്‍ത്തിക്കുന്നതിനെപ്പറ്റി കണ്ടറിയുന്നവനാകുന്നു.
\end{malayalam}}
\flushright{\begin{Arabic}
\quranayah[57][5]
\end{Arabic}}
\flushleft{\begin{malayalam}
അവന്നാണ് ആകാശങ്ങളുടെയും ഭൂമിയുടെയും ആധിപത്യം. അല്ലാഹുവിങ്കലേക്ക് തന്നെ കാര്യങ്ങള്‍ മടക്കപ്പെടുകയും ചെയ്യുന്നു.
\end{malayalam}}
\flushright{\begin{Arabic}
\quranayah[57][6]
\end{Arabic}}
\flushleft{\begin{malayalam}
അവന്‍ രാത്രിയെ പകലില്‍ പ്രവേശിപ്പിക്കുന്നു. അവന്‍ പകലിനെ രാത്രിയില്‍ പ്രവേശിപ്പിക്കുകയും ചെയ്യുന്നു. അവന്‍ ഹൃദയങ്ങളിലുള്ളവയെപ്പറ്റി അറിയുന്നവനുമാകുന്നു.
\end{malayalam}}
\flushright{\begin{Arabic}
\quranayah[57][7]
\end{Arabic}}
\flushleft{\begin{malayalam}
നിങ്ങള്‍ അല്ലാഹുവിലും അവന്‍റെ ദൂതനിലും വിശ്വസിക്കുകയും, അവന്‍ നിങ്ങളെ ഏതൊരു സ്വത്തില്‍ പിന്തുടര്‍ച്ച നല്‍കപ്പെട്ടവരാക്കിയിരിക്കുന്നോ അതില്‍ നിന്നു ചെലവഴിക്കുകയും ചെയ്യുക. അങ്ങനെ നിങ്ങളുടെ കൂട്ടത്തില്‍ നിന്ന് വിശ്വസിക്കുകയും ചെലവഴിക്കുകയും ചെയ്തവരാരോ അവര്‍ക്ക് വലിയ പ്രതിഫലമുണ്ടായിരിക്കുന്നതാണ്‌.
\end{malayalam}}
\flushright{\begin{Arabic}
\quranayah[57][8]
\end{Arabic}}
\flushleft{\begin{malayalam}
അല്ലാഹുവില്‍ വിശ്വസിക്കാതിരിക്കാന്‍ നിങ്ങള്‍ക്കെന്താണ് ന്യായം? ഈ ദൂതനാകട്ടെ നിങ്ങളുടെ രക്ഷിതാവില്‍ വിശ്വസിക്കാന്‍ വേണ്ടി നിങ്ങളെ ക്ഷണിച്ചു കൊണ്ടിരിക്കുകയുമാണ്‌. അല്ലാഹു നിങ്ങളുടെ ഉറപ്പ് വാങ്ങിയിട്ടുമുണ്ട്‌. നിങ്ങള്‍ വിശ്വസിക്കുന്നവരാണെങ്കില്‍!
\end{malayalam}}
\flushright{\begin{Arabic}
\quranayah[57][9]
\end{Arabic}}
\flushleft{\begin{malayalam}
നിങ്ങളെ ഇരുട്ടില്‍ നിന്ന് പ്രകാശത്തിലേക്ക് കൊണ്ടുവരാന്‍ വേണ്ടി തന്‍റെ ദാസന്‍റെ മേല്‍ വ്യക്തമായ ദൃഷ്ടാന്തങ്ങള്‍ ഇറക്കികൊടുക്കുന്നവനാണ് അവന്‍. തീര്‍ച്ചയായും അല്ലാഹു നിങ്ങളോട് വളരെയധികം ദയാലുവും കാരുണ്യവാനും തന്നെയാണ്‌.
\end{malayalam}}
\flushright{\begin{Arabic}
\quranayah[57][10]
\end{Arabic}}
\flushleft{\begin{malayalam}
ആകാശങ്ങളുടെയും ഭൂമിയുടെയും അനന്തരാവകാശം അല്ലാഹുവിനുള്ളതായിരിക്കെ അല്ലാഹുവിന്‍റെ മാര്‍ഗത്തില്‍ ചെലവഴിക്കാതിരിക്കാന്‍ നിങ്ങള്‍ക്കെന്താണ് ന്യായം? നിങ്ങളുടെ കൂട്ടത്തില്‍ നിന്നു (മക്കാ) വിജയത്തിനു മുമ്പുള്ള കാലത്ത് ചെലവഴിക്കുകയും യുദ്ധത്തില്‍ പങ്കെടുക്കുകയും ചെയ്തവരും (അല്ലാത്തവരും) സമമാകുകയില്ല. അക്കൂട്ടര്‍ പിന്നീടു ചെലവഴിക്കുകയും യുദ്ധത്തില്‍ പങ്കുവഹിക്കുകയും ചെയ്തവരെക്കാള്‍ മഹത്തായ പദവിയുള്ളവരാകുന്നു. എല്ലാവര്‍ക്കും ഏറ്റവും നല്ല പ്രതിഫലം അല്ലാഹു വാഗ്ദാനം ചെയ്തിരിക്കുന്നു. നിങ്ങള്‍ പ്രവര്‍ത്തിക്കുന്നതിനെപ്പറ്റി സൂക്ഷ്മജ്ഞാനമുള്ളവനാണ് അല്ലാഹു.
\end{malayalam}}
\flushright{\begin{Arabic}
\quranayah[57][11]
\end{Arabic}}
\flushleft{\begin{malayalam}
ആരുണ്ട് അല്ലാഹുവിന് ഒരു നല്ല കടം കൊടുക്കുവാന്‍? എങ്കില്‍ അവനത് അയാള്‍ക്ക് വേണ്ടി ഇരട്ടിപ്പിക്കുന്നതാണ്‌. അയാള്‍ക്കാണ് മാന്യമായ പ്രതിഫലമുള്ളത്‌.
\end{malayalam}}
\flushright{\begin{Arabic}
\quranayah[57][12]
\end{Arabic}}
\flushleft{\begin{malayalam}
സത്യവിശ്വാസികളെയും സത്യവിശ്വാസിനികളെയും, അവരുടെ പ്രകാശം അവരുടെ മുന്‍ഭാഗങ്ങളിലൂടെയും വലതുഭാഗങ്ങളിലൂടെയും സഞ്ചരിക്കുന്ന നിലയില്‍ നീ കാണുന്ന ദിവസം! (അന്നവരോട് പറയപ്പെടും:) ഇന്നു നിങ്ങള്‍ക്കുള്ള സന്തോഷവാര്‍ത്ത ചില സ്വര്‍ഗത്തോപ്പുകളെ പറ്റിയാകുന്നു. അവയുടെ താഴ്ഭാഗത്തു കൂടി അരുവികള്‍ ഒഴുകികൊണ്ടിരിക്കും. നിങ്ങള്‍ അതില്‍ നിത്യവാസികളായിരിക്കും. അത് മഹത്തായ ഭാഗ്യം തന്നെയാണ്‌.
\end{malayalam}}
\flushright{\begin{Arabic}
\quranayah[57][13]
\end{Arabic}}
\flushleft{\begin{malayalam}
കപടവിശ്വാസികളും കപടവിശ്വാസിനികളും സത്യവിശ്വാസികളോട് (ഇങ്ങനെ) പറയുന്ന ദിവസം: നിങ്ങള്‍ ഞങ്ങളെ നോക്കണേ! നിങ്ങളുടെ പ്രകാശത്തില്‍ നിന്ന് ഞങ്ങള്‍ പകര്‍ത്തി എടുക്കട്ടെ. (അപ്പോള്‍ അവരോട്‌) പറയപ്പെടും: നിങ്ങള്‍ നിങ്ങളുടെ പിന്‍ഭാഗത്തേക്കു തന്നെ മടങ്ങിപ്പോകുക. എന്നിട്ട് പ്രകാശം അന്വേഷിച്ചു കൊള്ളുക! അപ്പോള്‍ അവര്‍ക്കിടയില്‍ ഒരു മതില്‍ കൊണ്ട് മറയുണ്ടാക്കപ്പെടുന്നതാണ്‌. അതിന് ഒരു വാതിലുണ്ടായിരിക്കും. അതിന്‍റെ ഉള്‍ഭാഗത്താണ് കാരുണ്യമുള്ളത്‌. അതിന്‍റെ പുറഭാഗത്താകട്ടെ ശിക്ഷയും.
\end{malayalam}}
\flushright{\begin{Arabic}
\quranayah[57][14]
\end{Arabic}}
\flushleft{\begin{malayalam}
അവരെ (സത്യവിശ്വാസികളെ) വിളിച്ച് അവര്‍ (കപടന്‍മാര്‍) പറയും: ഞങ്ങള്‍ നിങ്ങളോടൊപ്പമായിരുന്നില്ലേ? അവര്‍ (സത്യവിശ്വാസികള്‍) പറയും: അതെ; പക്ഷെ, നിങ്ങള്‍ നിങ്ങളെ തന്നെ കുഴപ്പത്തിലാക്കുകയും (മറ്റുള്ളവര്‍ക്ക് നാശം വരുന്നത്‌) പാര്‍ത്തുകൊണ്ടിരിക്കുകയും (മതത്തില്‍) സംശയിക്കുകയും അല്ലാഹുവിന്‍റെ ആജ്ഞ വന്നെത്തുന്നത് വരെ വ്യാമോഹങ്ങള്‍ നിങ്ങളെ വഞ്ചിക്കുകയും ചെയ്തു. അല്ലാഹുവിന്‍റെ കാര്യത്തില്‍ പരമവഞ്ചകനായ പിശാച് നിങ്ങളെ വഞ്ചിച്ചു കളഞ്ഞു.
\end{malayalam}}
\flushright{\begin{Arabic}
\quranayah[57][15]
\end{Arabic}}
\flushleft{\begin{malayalam}
അതുകൊണ്ട് ഇന്ന് നിങ്ങളുടെ പക്കല്‍ നിന്നോ സത്യനിഷേധികളുടെ പക്കല്‍ നിന്നോ യാതൊരു പ്രായശ്ചിത്തവും സ്വീകരിക്കപ്പെടുന്നതല്ല. നിങ്ങളുടെ വാസസ്ഥലം നരകമാകുന്നു. അതത്രെ നിങ്ങളുടെ ബന്ധു തിരിച്ചുചെല്ലാനുള്ള ആ സ്ഥലം വളരെ ചീത്ത തന്നെ.
\end{malayalam}}
\flushright{\begin{Arabic}
\quranayah[57][16]
\end{Arabic}}
\flushleft{\begin{malayalam}
വിശ്വാസികള്‍ക്ക് അവരുടെ ഹൃദയങ്ങള്‍ അല്ലാഹുവിനെ പറ്റിയുള്ള സ്മരണയിലേക്കും, അവതരിച്ചു കിട്ടിയ സത്യത്തിലേക്കും കീഴൊതുങ്ങുവാനും തങ്ങള്‍ക്ക് മുമ്പ് വേദഗ്രന്ഥം നല്‍കപ്പെട്ടവരെപ്പോലെ ആകാതിരിക്കുവാനും സമയമായില്ലേ? അങ്ങനെ ആ വേദക്കാര്‍ക്ക് കാലം ദീര്‍ഘിച്ച് പോകുകയും തന്‍മൂലം അവരുടെ ഹൃദയങ്ങള്‍ കടുത്തുപോകുകയും ചെയ്തു. അവരില്‍ അധികമാളുകളും ദുര്‍മാര്‍ഗികളാകുന്നു.
\end{malayalam}}
\flushright{\begin{Arabic}
\quranayah[57][17]
\end{Arabic}}
\flushleft{\begin{malayalam}
നിങ്ങള്‍ അറിഞ്ഞു കൊള്ളുക: തീര്‍ച്ചയായും അല്ലാഹു ഭൂമിയെ അത് നിര്‍ജീവമായതിനു ശേഷം സജീവമാക്കുന്നു. തീര്‍ച്ചയായും നാം നിങ്ങള്‍ക്ക് ദൃഷ്ടാന്തങ്ങള്‍ വിവരിച്ചുതന്നിരിക്കുന്നു. നിങ്ങള്‍ ചിന്തിക്കുവാന്‍ വേണ്ടി.
\end{malayalam}}
\flushright{\begin{Arabic}
\quranayah[57][18]
\end{Arabic}}
\flushleft{\begin{malayalam}
തീര്‍ച്ചയായും ധര്‍മ്മിഷ്ഠരായ പുരുഷന്‍മാരും സ്ത്രീകളും അല്ലാഹുവിന് നല്ല കടം കൊടുത്തവരും ആരോ അവര്‍ക്കത് ഇരട്ടിയായി നല്‍കപ്പെടുന്നതാണ്‌. അവര്‍ക്കത്രെ മാന്യമായ പ്രതിഫലമുള്ളത്‌.
\end{malayalam}}
\flushright{\begin{Arabic}
\quranayah[57][19]
\end{Arabic}}
\flushleft{\begin{malayalam}
എന്നാല്‍ അല്ലാഹുവിലും അവന്‍റെ ദൂതന്‍മാരിലും വിശ്വസിച്ചവരാരോ അവര്‍ തന്നെയാണ് തങ്ങളുടെ രക്ഷിതാവിങ്കല്‍ സത്യസന്ധന്‍മാരും സത്യസാക്ഷികളും. അവര്‍ക്ക് അവരുടെ പ്രതിഫലവും അവരുടെ പ്രകാശവുമുണ്ടായിരിക്കും. സത്യനിഷേധം കൈക്കൊള്ളുകയും നമ്മുടെ ദൃഷ്ടാന്തങ്ങളെ നിഷേധിച്ചു തള്ളുകയും ചെയ്തവരാരോ അവര്‍ തന്നെയാണ് നരകക്കാര്‍.
\end{malayalam}}
\flushright{\begin{Arabic}
\quranayah[57][20]
\end{Arabic}}
\flushleft{\begin{malayalam}
നിങ്ങള്‍ അറിയുക: ഇഹലോകജീവിതമെന്നാല്‍ കളിയും വിനോദവും അലങ്കാരവും നിങ്ങള്‍ പരസ്പരം ദുരഭിമാനം നടിക്കലും സ്വത്തുകളിലും സന്താനങ്ങളിലും പെരുപ്പം കാണിക്കലും മാത്രമാണ്‌- ഒരു മഴ പോലെ. അതു മൂലമുണ്ടായ ചെടികള്‍ കര്‍ഷകരെ ആശ്ചര്യപ്പെടുത്തി. പിന്നീടതിന് ഉണക്കം ബാധിക്കുന്നു. അപ്പോള്‍ അത് മഞ്ഞനിറം പൂണ്ടതായി നിനക്ക് കാണാം. പിന്നീടതു തുരുമ്പായിപ്പോകുന്നു. എന്നാല്‍ പരലോകത്ത് (ദുര്‍വൃത്തര്‍ക്ക്‌) കഠിനമായ ശിക്ഷയും (സദ്‌വൃത്തര്‍ക്ക്‌) അല്ലാഹുവിങ്കല്‍ നിന്നുള്ള പാപമോചനവും പ്രീതിയും ഉണ്ട്‌. ഐഹികജീവിതം വഞ്ചനയുടെ വിഭവമല്ലാതെ മറ്റൊന്നുമല്ല.
\end{malayalam}}
\flushright{\begin{Arabic}
\quranayah[57][21]
\end{Arabic}}
\flushleft{\begin{malayalam}
നിങ്ങളുടെ രക്ഷിതാവിങ്കല്‍ നിന്നുള്ള പാപമോചനത്തിലേക്കും സ്വര്‍ഗത്തിലേക്കും നിങ്ങള്‍ മുങ്കടന്നു വരുവിന്‍. അതിന്‍റെ വിസ്താരം ആകാശത്തിന്‍റെയും ഭൂമിയുടെയും വിസ്താരം പോലെയാണ്‌. അല്ലാഹുവിലും അവന്‍റെ ദൂതന്‍മാരിലും വിശ്വസിച്ചവര്‍ക്കു വേണ്ടി അത് സജ്ജീകരിക്കപ്പെട്ടിരിക്കുന്നു. അത് അല്ലാഹുവിന്‍റെ അനുഗ്രഹമത്രെ. അവന്‍ ഉദ്ദേശിക്കുന്നവര്‍ക്ക് അതവന്‍ നല്‍കുന്നു. അല്ലാഹു മഹത്തായ അനുഗ്രഹമുള്ളവനാകുന്നു.
\end{malayalam}}
\flushright{\begin{Arabic}
\quranayah[57][22]
\end{Arabic}}
\flushleft{\begin{malayalam}
ഭൂമിയിലോ നിങ്ങളുടെ ദേഹങ്ങളില്‍ തന്നെയോ യാതൊരു ആപത്തും ബാധിക്കുകയുണ്ടായിട്ടില്ല; അതിനെ നാം ഉണ്ടാക്കുന്നതിന് മുമ്പ് തന്നെ ഒരു രേഖയില്‍ ഉള്‍പെട്ടുകഴിഞ്ഞതായിട്ടല്ലാതെ. തീര്‍ച്ചയായും അത് അല്ലാഹുവെ സംബന്ധിച്ചേടത്തോളം എളുപ്പമുള്ളതാകുന്നു.
\end{malayalam}}
\flushright{\begin{Arabic}
\quranayah[57][23]
\end{Arabic}}
\flushleft{\begin{malayalam}
(ഇങ്ങനെ നാം ചെയ്തത്‌,) നിങ്ങള്‍ക്കു നഷ്ടപ്പെട്ടതിന്‍റെ പേരില്‍ നിങ്ങള്‍ ദുഃഖിക്കാതിരിക്കുവാനും, നിങ്ങള്‍ക്ക് അവന്‍ നല്‍കിയതിന്‍റെ പേരില്‍ നിങ്ങള്‍ ആഹ്ലാദിക്കാതിരിക്കുവാനും വേണ്ടിയാണ്‌. അല്ലാഹു യാതൊരു അഹങ്കാരിയെയും ദുരഭിമാനിയെയും ഇഷ്ടപ്പെടുകയില്ല.
\end{malayalam}}
\flushright{\begin{Arabic}
\quranayah[57][24]
\end{Arabic}}
\flushleft{\begin{malayalam}
അതായത് പിശുക്ക് കാണിക്കുകയും, പിശുക്ക് കാണിക്കാന്‍ ജനങ്ങളോട് കല്‍പിക്കുകയും ചെയ്യുന്നവരെ. വല്ലവനും പിന്‍തിരിഞ്ഞ് പോകുകയാണെങ്കില്‍ തീര്‍ച്ചയായും അല്ലാഹു പരാശ്രയമുക്തനും സ്തുത്യര്‍ഹനുമത്രെ.
\end{malayalam}}
\flushright{\begin{Arabic}
\quranayah[57][25]
\end{Arabic}}
\flushleft{\begin{malayalam}
തീര്‍ച്ചയായും നാം നമ്മുടെ ദൂതന്‍മാരെ വ്യക്തമായ ദൃഷ്ടാന്തങ്ങളും കൊണ്ട് അയക്കുകയുണ്ടായി. ജനങ്ങള്‍ നീതിപൂര്‍വ്വം നിലകൊള്ളുവാന്‍ വേണ്ടി അവരോടൊപ്പം വേദഗ്രന്ഥവും തുലാസും ഇറക്കികൊടുക്കുകയും ചെയ്തു. ഇരുമ്പും നാം ഇറക്കി കൊടുത്തു. അതില്‍ കഠിനമായ ആയോധന ശക്തിയും ജനങ്ങള്‍ക്ക് ഉപകാരങ്ങളുമുണ്ട്‌. അല്ലാഹുവിനെയും അവന്‍റെ ദൂതന്‍മാരെയും അദൃശ്യമായ നിലയില്‍ സഹായിക്കുന്നവരെ അവന്ന് അറിയാന്‍ വേണ്ടിയുമാണ് ഇതെല്ലാം. തീര്‍ച്ചയായും അല്ലാഹു ശക്തനും പ്രതാപിയുമാകുന്നു.
\end{malayalam}}
\flushright{\begin{Arabic}
\quranayah[57][26]
\end{Arabic}}
\flushleft{\begin{malayalam}
തീര്‍ച്ചയായും നാം നൂഹിനെയും ഇബ്രാഹീമിനെയും (ദൂതന്‍മാരായി) നിയോഗിച്ചു. അവര്‍ ഇരുവരുടെയും സന്തതികളില്‍ പ്രവാചകത്വവും വേദഗ്രന്ഥവും നാം ഏര്‍പെടുത്തുകയും ചെയ്തു. അങ്ങനെ അവരുടെ കൂട്ടത്തില്‍ സന്‍മാര്‍ഗം പ്രാപിച്ചവരുണ്ട്‌. അവരില്‍ അധികപേരും ദുര്‍മാര്‍ഗികളാകുന്നു.
\end{malayalam}}
\flushright{\begin{Arabic}
\quranayah[57][27]
\end{Arabic}}
\flushleft{\begin{malayalam}
പിന്നീട് അവരുടെ പിന്നിലായി നാം നമ്മുടെ ദൂതന്‍മാരെ തുടര്‍ന്നയച്ചു. മര്‍യമിന്‍റെ മകന്‍ ഈസായെയും നാം തുടര്‍ന്നയച്ചു. അദ്ദേഹത്തിന് നാം ഇന്‍ജീല്‍ നല്‍കുകയും ചെയ്തു. അദ്ദേഹത്തെ പിന്‍പറ്റിയവരുടെ ഹൃദയങ്ങളില്‍ നാം കൃപയും കരുണയും ഉണ്ടാക്കി. സന്യാസജീവിതത്തെ അവര്‍ സ്വയം പുതുതായി നിര്‍മിച്ചു. അല്ലാഹുവിന്‍റെ പ്രീതി തേടേണ്ടതിന് (വേണ്ടി അവരതു ചെയ്തു) എന്നല്ലാതെ, നാം അവര്‍ക്കത് നിയമമാക്കിയിട്ടുണ്ടായിരുന്നില്ല. എന്നിട്ട് അവരത് പാലിക്കേണ്ട മുറപ്രകാരം പാലിച്ചതുമില്ല. അപ്പോള്‍ അവരുടെ കൂട്ടത്തില്‍ നിന്ന് വിശ്വസിച്ചവര്‍ക്ക് അവരുടെ പ്രതിഫലം നാം നല്‍കി. അവരില്‍ അധികപേരും ദുര്‍മാര്‍ഗികളാകുന്നു.
\end{malayalam}}
\flushright{\begin{Arabic}
\quranayah[57][28]
\end{Arabic}}
\flushleft{\begin{malayalam}
സത്യവിശ്വാസികളേ, നിങ്ങള്‍ അല്ലാഹുവെ സൂക്ഷിക്കുകയും അവന്‍റെ ദൂതനില്‍ വിശ്വസിക്കുകയും ചെയ്യുക. എന്നാല്‍ അവന്‍റെ കാരുണ്യത്തില്‍ നിന്നു രണ്ട് ഓഹരി അവന്‍ നിങ്ങള്‍ക്കു നല്‍കുന്നതാണ്‌. ഒരു പ്രകാശം അവന്‍ നിങ്ങള്‍ക്ക് ഏര്‍പെടുത്തിത്തരികയും ചെയ്യും. അതുകൊണ്ട് നിങ്ങള്‍ക്ക് (ശരിയായ പാതയിലൂടെ) നടന്നു പോകാം. നിങ്ങള്‍ക്കവന്‍ പൊറുത്തുതരികയും ചെയ്യും. അല്ലാഹു വളരെയധികം പൊറുക്കുന്നവനും കരുണാനിധിയുമാണ്‌.
\end{malayalam}}
\flushright{\begin{Arabic}
\quranayah[57][29]
\end{Arabic}}
\flushleft{\begin{malayalam}
അല്ലാഹുവിന്‍റെ അനുഗ്രഹത്തില്‍ നിന്ന് യാതൊന്നും അധീനപ്പെടുത്തുവാന്‍ തങ്ങള്‍ക്ക് കഴിവില്ലെന്നും തീര്‍ച്ചയായും അനുഗ്രഹം അല്ലാഹുവിന്‍റെ കയ്യിലാണെന്നും അത് അവന്‍ ഉദ്ദേശിക്കുന്നവര്‍ക്ക് അവന്‍ നല്‍കുമെന്നും വേദക്കാര്‍ അറിയാന്‍ വേണ്ടിയാണ് ഇത്‌. അല്ലാഹു മഹത്തായ അനുഗ്രഹമുള്ളവനാകുന്നു.
\end{malayalam}}
\chapter{\textmalayalam{  മുജാദില ( തര്‍ക്കിക്കുന്നവള്‍ )}}
\begin{Arabic}
\Huge{\centerline{\basmalah}}\end{Arabic}
\flushright{\begin{Arabic}
\quranayah[58][1]
\end{Arabic}}
\flushleft{\begin{malayalam}
(നബിയേ,) തന്‍റെ ഭര്‍ത്താവിന്‍റെ കാര്യത്തില്‍ നിന്നോട് തര്‍ക്കിക്കുകയും അല്ലാഹുവിങ്കലേക്ക് സങ്കടം ബോധിപ്പിക്കുകയും ചെയ്യുന്നവളുടെ വാക്ക് അല്ലാഹു കേട്ടിട്ടുണ്ട്‌. അല്ലാഹു നിങ്ങള്‍ രണ്ടു പേരുടെയും സംഭാഷണം കേട്ടുകൊണ്ടിരിക്കുകയാണ്‌. തീര്‍ച്ചയായും അല്ലാഹു എല്ലാം കേള്‍ക്കുന്നവനും കാണുന്നവനുമാണ്‌.
\end{malayalam}}
\flushright{\begin{Arabic}
\quranayah[58][2]
\end{Arabic}}
\flushleft{\begin{malayalam}
നിങ്ങളുടെ കൂട്ടത്തില്‍ തങ്ങളുടെ ഭാര്യമാരെ മാതാക്കള്‍ക്ക് തുല്യമായി പ്രഖ്യാപിക്കുന്നവര്‍ (അബദ്ധമാകുന്നു ചെയ്യുന്നത്‌.) അവര്‍ (ഭാര്യമാര്‍) അവരുടെ മാതാക്കളല്ല. അവരുടെ മാതാക്കള്‍ അവരെ പ്രസവിച്ച സ്ത്രീകള്‍ അല്ലാതെ മറ്റാരുമല്ല. തീര്‍ച്ചയായും അവര്‍ നിഷിദ്ധമായ വാക്കും അസത്യവുമാണ് പറയുന്നത്‌. തീര്‍ച്ചയായും അല്ലാഹു അധികം മാപ്പുനല്‍കുന്നവനും പൊറുക്കുന്നവനുമാണ്‌.
\end{malayalam}}
\flushright{\begin{Arabic}
\quranayah[58][3]
\end{Arabic}}
\flushleft{\begin{malayalam}
തങ്ങളുടെ ഭാര്യമാരെ മാതാക്കള്‍ക്ക് തുല്യമായി പ്രഖ്യാപിക്കുകയും, പിന്നീട് തങ്ങള്‍ പറഞ്ഞതില്‍ നിന്ന് മടങ്ങുകയും ചെയ്യുന്നവര്‍, അവര്‍ പരസ്പരം സ്പര്‍ശിക്കുന്നതിനു മുമ്പായി ഒരു അടിമയെ മോചിപ്പിക്കേണ്ടതാണ്‌. അത് നിങ്ങള്‍ക്കു നല്‍കപ്പെടുന്ന ഉപദേശമാണ്‌. അല്ലാഹു നിങ്ങള്‍ പ്രവര്‍ത്തിക്കുന്നതിനെപ്പറ്റി സൂക്ഷ്മജ്ഞാനമുള്ളവനാകുന്നു.
\end{malayalam}}
\flushright{\begin{Arabic}
\quranayah[58][4]
\end{Arabic}}
\flushleft{\begin{malayalam}
ഇനി വല്ലവന്നും (അടിമയെ) ലഭിക്കാത്ത പക്ഷം, അവര്‍ പരസ്പരം സ്പര്‍ശിക്കുന്നതിന് മുമ്പായി തുടര്‍ച്ചയായി രണ്ടുമാസക്കാലം നോമ്പനുഷ്ഠിക്കേണ്ടതാണ്‌. വല്ലവന്നും (അത്‌) സാധ്യമാകാത്ത പക്ഷം അറുപതു അഗതികള്‍ക്ക് ആഹാരം നല്‍കേണ്ടതാണ്‌. അത് അല്ലാഹുവിലും അവന്‍റെ ദൂതനിലും നിങ്ങള്‍ വിശ്വസിക്കാന്‍ വേണ്ടിയത്രെ. അവ അല്ലാഹുവിന്‍റെ പരിധികളാകുന്നു. സത്യനിഷേധികള്‍ക്ക് വേദനയേറിയ ശിക്ഷയുണ്ട്‌.
\end{malayalam}}
\flushright{\begin{Arabic}
\quranayah[58][5]
\end{Arabic}}
\flushleft{\begin{malayalam}
തീര്‍ച്ചയായും അല്ലാഹുവെയും അവന്‍റെ ദൂതനെയും എതിര്‍ത്തു കൊണ്ടിരിക്കുന്നവര്‍ അവരുടെ മുമ്പുള്ളവര്‍ വഷളാക്കപ്പെട്ടത് പോലെ വഷളാക്കപ്പെടുന്നതാണ്‌. സുവ്യക്തമായ പല തെളിവുകളും നാം അവതരിപ്പിച്ചിട്ടുണ്ട്‌. സത്യനിഷേധികള്‍ക്ക് അപമാനകരമായ ശിക്ഷയുമുണ്ട്‌.
\end{malayalam}}
\flushright{\begin{Arabic}
\quranayah[58][6]
\end{Arabic}}
\flushleft{\begin{malayalam}
അല്ലാഹു അവരെയെല്ലാം ഉയിര്‍ത്തെഴുന്നേല്‍പിക്കുകയും, എന്നിട്ട് അവര്‍ പ്രവര്‍ത്തിച്ചതിനെപ്പറ്റി അവരെ വിവരമറിയിക്കുകയും ചെയ്യുന്ന ദിവസം. അല്ലാഹു അത് തിട്ടപ്പെടുത്തുകയും അവരത് മറന്നുപോകുകയും ചെയ്തു. അല്ലാഹു ഏത് കാര്യത്തിനും സാക്ഷിയാകുന്നു.
\end{malayalam}}
\flushright{\begin{Arabic}
\quranayah[58][7]
\end{Arabic}}
\flushleft{\begin{malayalam}
ആകാശങ്ങളിലുള്ളതും ഭൂമിയിലുള്ളതും അല്ലാഹു അറിയുന്നുണ്ടെന്ന് നീ കാണുന്നില്ലേ? മൂന്നു പേര്‍ തമ്മിലുള്ള യാതൊരു രഹസ്യസംഭാഷണവും അവന്‍ (അല്ലാഹു) അവര്‍ക്കു നാലാമനായികൊണ്ടല്ലാതെ ഉണ്ടാവുകയില്ല. അഞ്ചുപേരുടെ സംഭാഷണമാണെങ്കില്‍ അവന്‍ അവര്‍ക്കു ആറാമനായികൊണ്ടുമല്ലാതെ. അതിനെക്കാള്‍ കുറഞ്ഞവരുടെയോ, കൂടിയവരുടെയോ (സംഭാഷണം) ആണെങ്കില്‍ അവര്‍ എവിടെയായിരുന്നാലും അവന്‍ അവരോടൊപ്പമുണ്ടായിട്ടല്ലാതെ. പിന്നീട് ഉയിര്‍ത്തെഴുന്നേല്‍പിന്‍റെ നാളില്‍, അവര്‍ പ്രവര്‍ത്തിച്ചതിനെപ്പറ്റി അവരെ അവന്‍ വിവരമറിയിക്കുന്നതാണ്‌. തീര്‍ച്ചയായും അല്ലാഹു ഏത് കാര്യത്തെ പറ്റിയും അറിവുള്ളവനാകുന്നു.
\end{malayalam}}
\flushright{\begin{Arabic}
\quranayah[58][8]
\end{Arabic}}
\flushleft{\begin{malayalam}
രഹസ്യസംഭാഷണം നടത്തുന്നതില്‍ നിന്ന് വിലക്കപ്പെട്ടിട്ടുള്ളവരെ നീ കണ്ടില്ലേ? അവര്‍ ഏതൊന്നില്‍ നിന്നു വിലക്കപ്പെട്ടുവോ അതിലേക്കവര്‍ പിന്നീട് മടങ്ങുന്നു.പാപത്തിനും അതിക്രമത്തിനും റസൂലിനെ ധിക്കരിക്കുന്നതിനും അവര്‍ പരസ്പരം രഹസ്യഉപദേശം നടത്തുകയും ചെയ്യുന്നു. അവര്‍ നിന്‍റെ അടുത്ത് വന്നാല്‍ നിന്നെ അല്ലാഹു അഭിവാദ്യം ചെയ്തിട്ടില്ലാത്ത രീതിയില്‍ അവര്‍ നിനക്ക് അഭിവാദ്യമര്‍പ്പിക്കുകയും ചെയ്യും. ഞങ്ങള്‍ ഈ പറയുന്നതിന്‍റെ പേരില്‍ അല്ലാഹു ഞങ്ങളെ ശിക്ഷിക്കാതിരിക്കുന്നത് എന്തുകൊണ്ടാണെന്ന് അവര്‍ അന്യോന്യം പറയുകയും ചെയ്യും. അവര്‍ക്കു നരകം മതി. അവര്‍ അതില്‍ എരിയുന്നതാണ്‌. ആ പര്യവസാനം എത്ര ചീത്ത.
\end{malayalam}}
\flushright{\begin{Arabic}
\quranayah[58][9]
\end{Arabic}}
\flushleft{\begin{malayalam}
സത്യവിശ്വാസികളേ, നിങ്ങള്‍ രഹസ്യസംഭാഷണം നടത്തുകയാണെങ്കില്‍ അധര്‍മ്മത്തിനും അതിക്രമത്തിനും റസൂലിനെ ധിക്കരിക്കുന്നതിനും നിങ്ങള്‍ രഹസ്യസംഭാഷണം നടത്തരുത്‌. പുണ്യത്തിന്‍റെയും ഭയഭക്തിയുടെയും കാര്യത്തില്‍ നിങ്ങള്‍ രഹസ്യഉപദേശം നടത്തുക. ഏതൊരു അല്ലാഹുവിങ്കലേക്ക് നിങ്ങള്‍ ഒരുമിച്ചുകൂട്ടപ്പെടുമോ അവനെ നിങ്ങള്‍ സൂക്ഷിക്കുകയും ചെയ്യുക.
\end{malayalam}}
\flushright{\begin{Arabic}
\quranayah[58][10]
\end{Arabic}}
\flushleft{\begin{malayalam}
ആ രഹസ്യസംസാരം പിശാചില്‍ നിന്നുള്ളത് മാത്രമാകുന്നു. സത്യവിശ്വാസികളെ ദുഃഖിപ്പിക്കാന്‍ വേണ്ടിയാകുന്നു അത്‌. എന്നാല്‍ അല്ലാഹുവിന്‍റെ അനുമതികൂടാതെ അതവര്‍ക്ക് യാതൊരു ഉപദ്രവവും ചെയ്യുന്നതല്ല. സത്യവിശ്വാസികള്‍ അല്ലാഹുവിന്‍റെ മേല്‍ ഭരമേല്‍പിച്ചുകൊള്ളട്ടെ.
\end{malayalam}}
\flushright{\begin{Arabic}
\quranayah[58][11]
\end{Arabic}}
\flushleft{\begin{malayalam}
സത്യവിശ്വാസികളേ, നിങ്ങള്‍ സദസ്സുകളില്‍ സൌകര്യപ്പെടുത്തി കൊടുക്കുക എന്ന് നിങ്ങളോടു പറയപ്പെട്ടാല്‍ നിങ്ങള്‍ സൌകര്യപ്പെടുത്തികൊടുക്കണം. എങ്കില്‍ അല്ലാഹു നിങ്ങള്‍ക്കും സൌകര്യപ്പെടുത്തിത്തരുന്നതാണ്‌. നിങ്ങള്‍ എഴുന്നേറ്റ് പോകണമെന്ന് പറയപ്പെട്ടാല്‍ നിങ്ങള്‍ എഴുന്നേറ്റ് പോകണം. നിങ്ങളില്‍ നിന്ന് വിശ്വസിച്ചവരെയും വിജ്ഞാനം നല്‍കപ്പെട്ടവരെയും അല്ലാഹു പല പടികള്‍ ഉയര്‍ത്തുന്നതാണ്‌. അല്ലാഹു നിങ്ങള്‍ പ്രവര്‍ത്തിക്കുന്നതിനെ പറ്റി സൂക്ഷ്മമായി അറിയുന്നവനാകുന്നു.
\end{malayalam}}
\flushright{\begin{Arabic}
\quranayah[58][12]
\end{Arabic}}
\flushleft{\begin{malayalam}
സത്യവിശ്വാസികളേ, നിങ്ങള്‍ റസൂലുമായി രഹസ്യസംഭാഷണം നടത്തുകയാണെങ്കില്‍ നിങ്ങളുടെ രഹസ്യസംഭാഷണത്തിന്‍റെ മുമ്പായി ഏതെങ്കിലുമൊരു ദാനം നിങ്ങള്‍ അര്‍പ്പിക്കുക. അതാണു നിങ്ങള്‍ക്കു ഉത്തമവും കൂടുതല്‍ പരിശുദ്ധവുമായിട്ടുള്ളത്‌. ഇനി നിങ്ങള്‍ക്ക് (ദാനം ചെയ്യാന്‍) ഒന്നും കിട്ടിയില്ലെങ്കില്‍ തീര്‍ച്ചയായും അല്ലാഹു എറെ പൊറുക്കുന്നവനും കരുണാനിധിയുമാകുന്നു.
\end{malayalam}}
\flushright{\begin{Arabic}
\quranayah[58][13]
\end{Arabic}}
\flushleft{\begin{malayalam}
നിങ്ങളുടെ രഹസ്യസംഭാഷണത്തിനു മുമ്പായി നിങ്ങള്‍ ദാനധര്‍മ്മങ്ങള്‍ അര്‍പ്പിക്കുന്നതിനെപ്പറ്റി നിങ്ങള്‍ ഭയപ്പെട്ടിരിക്കുകയാണോ? എന്നാല്‍ നിങ്ങളത് ചെയ്യാതിരിക്കുകയും അല്ലാഹു നിങ്ങളുടെ നേരെ മടങ്ങുകയും ചെയ്തിരിക്കയാല്‍ നിങ്ങള്‍ നമസ്കാരം മുറപോലെ നിര്‍വഹിക്കുകയും സകാത്ത് നല്‍കുകയും അല്ലാഹുവെയും അവന്‍റെ റസൂലിനെയും നിങ്ങള്‍ അനുസരിക്കുകയും ചെയ്യുക. അല്ലാഹു നിങ്ങള്‍ പ്രവര്‍ത്തിക്കുന്നതിനെപ്പറ്റി സൂക്ഷ്മമായി അറിയുന്നവനാകുന്നു.
\end{malayalam}}
\flushright{\begin{Arabic}
\quranayah[58][14]
\end{Arabic}}
\flushleft{\begin{malayalam}
അല്ലാഹു കോപിച്ച ഒരു വിഭാഗ (യഹൂദര്‍) വുമായി മൈത്രിയില്‍ ഏര്‍പെട്ടവരെ (മുനാഫിഖുകളെ) നീ കണ്ടില്ലേ? അവര്‍ നിങ്ങളില്‍ പെട്ടവരല്ല. അവരില്‍ (യഹൂദരില്‍) പെട്ടവരുമല്ല. അവര്‍ അറിഞ്ഞു കൊണ്ട് കള്ള സത്യം ചെയ്യുന്നു.
\end{malayalam}}
\flushright{\begin{Arabic}
\quranayah[58][15]
\end{Arabic}}
\flushleft{\begin{malayalam}
അല്ലാഹു അവര്‍ക്ക് കഠിനമായ ശിക്ഷ ഒരുക്കിവെച്ചിരിക്കുന്നു. തീര്‍ച്ചയായും അവര്‍ ചെയ്തുകൊണ്ടിരിക്കുന്ന കാര്യം എത്രയോ ദുഷിച്ചതായിരിക്കുന്നു.
\end{malayalam}}
\flushright{\begin{Arabic}
\quranayah[58][16]
\end{Arabic}}
\flushleft{\begin{malayalam}
അവരുടെ ശപഥങ്ങളെ അവര്‍ ഒരു പരിചയാക്കിത്തീര്‍ത്തിരിക്കുന്നു. അങ്ങനെ അവര്‍ അല്ലാഹുവിന്‍റെ മാര്‍ഗത്തില്‍ നിന്ന് (ജനങ്ങളെ) തടഞ്ഞു. അതിനാല്‍ അവര്‍ക്ക് അപമാനകരമായ ശിക്ഷയുണ്ട്‌.
\end{malayalam}}
\flushright{\begin{Arabic}
\quranayah[58][17]
\end{Arabic}}
\flushleft{\begin{malayalam}
അവരുടെ സമ്പത്തുകളോ സന്താനങ്ങളോ അല്ലാഹുവിങ്കല്‍ അവര്‍ക്ക് ഒട്ടും പ്രയോജനപ്പെടുകയില്ല. അത്തരക്കാരാകുന്നു നരകാവകാശികള്‍. അവര്‍ അതില്‍ നിത്യവാസികളായിരിക്കും.
\end{malayalam}}
\flushright{\begin{Arabic}
\quranayah[58][18]
\end{Arabic}}
\flushleft{\begin{malayalam}
അല്ലാഹു അവരെയെല്ലാം ഉയിര്‍ത്തെഴുന്നേല്‍പിക്കുന്ന ദിവസം. നിങ്ങളോടവര്‍ ശപഥം ചെയ്യുന്നത് പോലെ അവനോടും അവര്‍ ശപഥം ചെയ്യും. തങ്ങള്‍ (ഈ കള്ള സത്യം മൂലം) എന്തോ ഒന്ന് നേടിയതായി അവര്‍ വിചാരിക്കുകയും ചെയ്യും. അറിയുക: തീര്‍ച്ചയായും അവര്‍ തന്നെയാകുന്നു കള്ളം പറയുന്നവര്‍
\end{malayalam}}
\flushright{\begin{Arabic}
\quranayah[58][19]
\end{Arabic}}
\flushleft{\begin{malayalam}
പിശാച് അവരെ കീഴടക്കി വെക്കുകയും അങ്ങനെ അല്ലാഹുവെ പറ്റിയുള്ള ഉല്‍ബോധനം അവര്‍ക്ക് വിസ്മരിപ്പിച്ചു കളയുകയും ചെയ്തിരിക്കുന്നു. അക്കൂട്ടരാകുന്നു പിശാചിന്‍റെ കക്ഷി. അറിയുക; തീര്‍ച്ചയായും പിശാചിന്‍റെ കക്ഷി തന്നെയാകുന്നു നഷ്ടക്കാര്‍ .
\end{malayalam}}
\flushright{\begin{Arabic}
\quranayah[58][20]
\end{Arabic}}
\flushleft{\begin{malayalam}
തീര്‍ച്ചയായും അല്ലാഹുവോടും അവന്‍റെ റസൂലിനോടും എതിര്‍ത്തുനില്‍ക്കുന്നവരാരോ അക്കൂട്ടര്‍ ഏറ്റവും നിന്ദ്യന്‍മാരായവരുടെ കൂട്ടത്തിലാകുന്നു.
\end{malayalam}}
\flushright{\begin{Arabic}
\quranayah[58][21]
\end{Arabic}}
\flushleft{\begin{malayalam}
തീര്‍ച്ചയായും ഞാനും എന്‍റെ ദൂതന്‍മാരും തന്നെയാണ് വിജയം നേടുക. എന്ന് അല്ലാഹു രേഖപ്പെടുത്തിയിരിക്കുന്നു. തീര്‍ച്ചയായും അല്ലാഹു ശക്തനും പ്രതാപിയുമാകുന്നു.
\end{malayalam}}
\flushright{\begin{Arabic}
\quranayah[58][22]
\end{Arabic}}
\flushleft{\begin{malayalam}
അല്ലാഹുവിലും അന്ത്യദിനത്തിലും വിശ്വസിക്കുന്ന ഒരു ജനത അല്ലാഹുവോടും അവന്‍റെ റസൂലിനോടും എതിര്‍ത്തു നില്‍ക്കുന്നവരുമായി സ്നേഹബന്ധം പുലര്‍ത്തുന്നത് നീ കണ്ടെത്തുകയില്ല. അവര്‍ (എതിര്‍പ്പുകാര്‍) അവരുടെ പിതാക്കളോ, പുത്രന്‍മാരോ, സഹോദരന്‍മാരോ ബന്ധുക്കളോ ആയിരുന്നാല്‍ പോലും. അത്തരക്കാരുടെ ഹൃദയങ്ങളില്‍ അല്ലാഹു വിശ്വാസം രേഖപ്പെടുത്തുകയും അവന്‍റെ പക്കല്‍ നിന്നുള്ള ഒരു ആത്മചൈതന്യം കൊണ്ട് അവന്‍ അവര്‍ക്ക് പിന്‍ബലം നല്‍കുകയും ചെയ്തിരിക്കുന്നു. താഴ്ഭാഗത്തു കൂടി അരുവികള്‍ ഒഴുകുന്ന സ്വര്‍ഗത്തോപ്പുകളില്‍ അവന്‍ അവരെ പ്രവേശിപ്പിക്കുകയും ചെയ്യും. അവരതില്‍ നിത്യവാസികളായിരിക്കും. അല്ലാഹു അവരെ പറ്റി തൃപ്തിപ്പെട്ടിരിക്കുന്നു. അവര്‍ അവനെ പറ്റിയും തൃപ്തിപ്പെട്ടിരിക്കുന്നു. അത്തരക്കാരാകുന്നു അല്ലാഹുവിന്‍റെ കക്ഷി. അറിയുക: തീര്‍ച്ചയായും അല്ലാഹുവിന്‍റെ കക്ഷിയാകുന്നു വിജയം പ്രാപിക്കുന്നവര്‍.
\end{malayalam}}
\chapter{\textmalayalam{ഹഷ്ര്‍ ( തുരത്തിയോടിക്കല്‍ )}}
\begin{Arabic}
\Huge{\centerline{\basmalah}}\end{Arabic}
\flushright{\begin{Arabic}
\quranayah[59][1]
\end{Arabic}}
\flushleft{\begin{malayalam}
ആകാശങ്ങളിലുള്ളതും ഭൂമിയിലുള്ളതും അല്ലാഹുവെ പ്രകീര്‍ത്തനം ചെയ്തിരിക്കുന്നു. അവന്‍ പ്രതാപിയും യുക്തിമാനുമാകുന്നു.
\end{malayalam}}
\flushright{\begin{Arabic}
\quranayah[59][2]
\end{Arabic}}
\flushleft{\begin{malayalam}
വേദക്കാരില്‍ പെട്ട സത്യനിഷേധികളെ ഒന്നാമത്തെ തുരത്തിയോടിക്കലില്‍ തന്നെ അവരുടെ വീടുകളില്‍ നിന്നു പുറത്തിറക്കിയവന്‍ അവനാകുന്നു. അവര്‍ പുറത്തിറങ്ങുമെന്ന് നിങ്ങള്‍ വിചാരിച്ചിരുന്നില്ല. തങ്ങളുടെ കോട്ടകള്‍ അല്ലാഹുവില്‍ നിന്ന് തങ്ങളെ പ്രതിരോധിക്കുമെന്ന് അവര്‍ വിചാരിച്ചിരുന്നു. എന്നാല്‍ അവര്‍ കണക്കാക്കാത്ത വിധത്തില്‍ അല്ലാഹു അവരുടെ അടുക്കല്‍ ചെല്ലുകയും അവരുടെ മനസ്സുകളില്‍ ഭയം ഇടുകയും ചെയ്തു. അവര്‍ സ്വന്തം കൈകള്‍കൊണ്ടും സത്യവിശ്വാസികളുടെ കൈകള്‍കൊണ്ടും അവരുടെ വീടുകള്‍ നശിപ്പിച്ചിരുന്നു. ആകയാല്‍ കണ്ണുകളുള്ളവരേ, നിങ്ങള്‍ ഗുണപാഠം ഉള്‍കൊള്ളുക.
\end{malayalam}}
\flushright{\begin{Arabic}
\quranayah[59][3]
\end{Arabic}}
\flushleft{\begin{malayalam}
അല്ലാഹു അവരുടെ മേല്‍ നാടുവിട്ടുപോക്ക് വിധിച്ചിട്ടില്ലായിരുന്നുവെങ്കില്‍ ഇഹലോകത്ത് വെച്ച് അവന്‍ അവരെ ശിക്ഷിക്കുമായിരുന്നു.പരലോകത്ത് അവര്‍ക്കു നരകശിക്ഷയുമുണ്ട്‌.
\end{malayalam}}
\flushright{\begin{Arabic}
\quranayah[59][4]
\end{Arabic}}
\flushleft{\begin{malayalam}
അത് അല്ലാഹുവോടും അവന്‍റെ റസൂലിനോടും അവര്‍ മത്സരിച്ചു നിന്നതിന്‍റെ ഫലമത്രെ. വല്ലവനും അല്ലാഹുവുമായി മത്സരിക്കുന്ന പക്ഷം തീര്‍ച്ചയായും അല്ലാഹു കഠിനമായി ശിക്ഷിക്കുന്നവനാകുന്നു.
\end{malayalam}}
\flushright{\begin{Arabic}
\quranayah[59][5]
\end{Arabic}}
\flushleft{\begin{malayalam}
നിങ്ങള്‍ വല്ല ഈന്തപ്പനയും മുറിക്കുകയോ അല്ലെങ്കില്‍ അവയെ അവയുടെ മുരടുകളില്‍ നില്‍ക്കാന്‍ വിടുകയോ ചെയ്യുന്ന പക്ഷം അത് അല്ലാഹുവിന്‍റെ അനുമതി പ്രകാരമാണ്‌. അധര്‍മ്മകാരികളെ അപമാനപ്പെടുത്തുവാന്‍ വേണ്ടിയുമാണ്‌.
\end{malayalam}}
\flushright{\begin{Arabic}
\quranayah[59][6]
\end{Arabic}}
\flushleft{\begin{malayalam}
അവരില്‍ നിന്ന് (യഹൂദരില്‍ നിന്ന്‌) അല്ലാഹു അവന്‍റെ റസൂലിന് കൈവരുത്തി കൊടുത്തതെന്തോ അതിനായി നിങ്ങള്‍ കുതിരകളെയോ ഒട്ടകങ്ങളെയോ ഓടിക്കുകയുണ്ടായിട്ടില്ല.പക്ഷെ, അല്ലാഹു അവന്‍റെ ദൂതന്‍മാരെ അവന്‍ ഉദ്ദേശിക്കുന്നവരുടെ നേര്‍ക്ക് അധികാരപ്പെടുത്തി അയക്കുന്നു. അല്ലാഹു ഏതു കാര്യത്തിനും കഴിവുള്ളവനാകുന്നു.
\end{malayalam}}
\flushright{\begin{Arabic}
\quranayah[59][7]
\end{Arabic}}
\flushleft{\begin{malayalam}
അല്ലാഹു അവന്‍റെ റസൂലിന് വിവിധ രാജ്യക്കാരില്‍ നിന്ന് കൈവരുത്തി കൊടുത്തതെന്തോ അത് അല്ലാഹുവിനും റസൂലിനും അടുത്ത കുടുംബങ്ങള്‍ക്കും അനാഥകള്‍ക്കും അഗതികള്‍ക്കും വഴിപോക്കര്‍ക്കുമുള്ളതാകുന്നു. അത് (ധനം) നിങ്ങളില്‍ നിന്നുള്ള ധനികന്‍മാര്‍ക്കിടയില്‍ മാത്രം കൈമാറ്റം ചെയ്യപ്പെടുന്ന ഒന്നാവാതിരിക്കാന്‍ വേണ്ടിയാണത്‌. നിങ്ങള്‍ക്കു റസൂല്‍ നല്‍കിയതെന്തോ അത് നിങ്ങള്‍ സ്വീകരിക്കുക. എന്തൊന്നില്‍ നിന്ന് അദ്ദേഹം നിങ്ങളെ വിലക്കിയോ അതില്‍ നിന്ന് നിങ്ങള്‍ ഒഴിഞ്ഞ് നില്‍ക്കുകയും ചെയ്യുക. നിങ്ങള്‍ അല്ലാഹുവെ സൂക്ഷിക്കുകയും ചെയ്യുക. തീര്‍ച്ചയായും അല്ലാഹു കഠിനമായി ശിക്ഷിക്കുന്നവനാണ്‌.
\end{malayalam}}
\flushright{\begin{Arabic}
\quranayah[59][8]
\end{Arabic}}
\flushleft{\begin{malayalam}
അതായത് സ്വന്തം വീടുകളില്‍ നിന്നും സ്വത്തുക്കളില്‍ നിന്നും കുടിയൊഴിപ്പിക്കപ്പെട്ട മുഹാജിറുകളായ ദരിദ്രന്‍മാര്‍ക്ക് (അവകാശപ്പെട്ടതാകുന്നു ആ ധനം.) അവര്‍ അല്ലാഹുവിങ്കല്‍ നിന്നുള്ള അനുഗ്രഹവും പ്രീതിയും തേടുകയും അല്ലാഹുവെയും അവന്‍റെ റസൂലിനെയും സഹായിക്കുകയും ചെയ്യുന്നു. അവര്‍ തന്നെയാകുന്നു സത്യവാന്‍മാര്‍.
\end{malayalam}}
\flushright{\begin{Arabic}
\quranayah[59][9]
\end{Arabic}}
\flushleft{\begin{malayalam}
അവരുടെ (മുഹാജിറുകളുടെ) വരവിനു മുമ്പായി വാസസ്ഥലവും വിശ്വാസവും സ്വീകരിച്ചുവെച്ചവര്‍ക്കും (അന്‍സാറുകള്‍ക്ക്‌). തങ്ങളുടെ അടുത്തേക്ക് സ്വദേശം വെടിഞ്ഞു വന്നവരെ അവര്‍ സ്നേഹിക്കുന്നു. അവര്‍ക്ക് (മുഹാജിറുകള്‍ക്ക്‌) നല്‍കപ്പെട്ട ധനം സംബന്ധിച്ചു തങ്ങളുടെ മനസ്സുകളില്‍ ഒരു ആവശ്യവും അവര്‍ (അന്‍സാറുകള്‍) കണ്ടെത്തുന്നുമില്ല. തങ്ങള്‍ക്ക് ദാരിദ്യ്‌രമുണ്ടായാല്‍ പോലും സ്വദേഹങ്ങളെക്കാള്‍ മറ്റുള്ളവര്‍ക്ക് അവര്‍ പ്രാധാന്യം നല്‍കുകയും ചെയ്യും. ഏതൊരാള്‍ തന്‍റെ മനസ്സിന്‍റെ പിശുക്കില്‍ നിന്ന് കാത്തുരക്ഷിക്കപ്പെടുന്നുവോ അത്തരക്കാര്‍ തന്നെയാകുന്നു വിജയം പ്രാപിച്ചവര്‍.
\end{malayalam}}
\flushright{\begin{Arabic}
\quranayah[59][10]
\end{Arabic}}
\flushleft{\begin{malayalam}
അവരുടെ ശേഷം വന്നവര്‍ക്കും. അവര്‍ പറയും: ഞങ്ങളുടെ രക്ഷിതാവേ, ഞങ്ങള്‍ക്കും വിശ്വാസത്തോടെ ഞങ്ങള്‍ക്ക് മുമ്പ് കഴിഞ്ഞുപോയിട്ടുള്ള ഞങ്ങളുടെ സഹോദരങ്ങള്‍ക്കും നീ പൊറുത്തുതരേണമേ, സത്യവിശ്വാസം സ്വീകരിച്ചവരോട് ഞങ്ങളുടെ മനസ്സുകളില്‍ നീ ഒരു വിദ്വേഷവും ഉണ്ടാക്കരുതേ. ഞങ്ങളുടെ രക്ഷിതാവേ, തീര്‍ച്ചയായും നീ ഏറെ ദയയുള്ളവനും കരുണാനിധിയുമാകുന്നു
\end{malayalam}}
\flushright{\begin{Arabic}
\quranayah[59][11]
\end{Arabic}}
\flushleft{\begin{malayalam}
ആ കാപട്യം കാണിച്ചവരെ നീ കണ്ടില്ലേ? വേദക്കാരില്‍ പെട്ട സത്യനിഷേധികളായ അവരുടെ സഹോദരന്‍മാരോട് അവര്‍ പറയുന്നു: തീര്‍ച്ചയായും നിങ്ങള്‍ പുറത്താക്കപ്പെട്ടാല്‍ ഞങ്ങളും നിങ്ങളുടെ കൂടെ പുറത്ത് പോകുന്നതാണ്‌. നിങ്ങളുടെ കാര്യത്തില്‍ ഞങ്ങള്‍ ഒരിക്കലും ഒരാളെയും അനുസരിക്കുകയില്ല. നിങ്ങള്‍ക്കെതിരില്‍ യുദ്ധമുണ്ടായാല്‍ തീര്‍ച്ചയായും ഞങ്ങള്‍ നിങ്ങളെ സഹായിക്കുന്നതാണ്‌. എന്നാല്‍ തീര്‍ച്ചയായും അവര്‍ കള്ളം പറയുന്നവരാണ് എന്നതിന് അല്ലാഹു സാക്ഷ്യം വഹിക്കുന്നു.
\end{malayalam}}
\flushright{\begin{Arabic}
\quranayah[59][12]
\end{Arabic}}
\flushleft{\begin{malayalam}
അവര്‍ യഹൂദന്‍മാര്‍ പുറത്താക്കപ്പെടുന്ന പക്ഷം ഇവര്‍ (കപടവിശ്വാസികള്‍) അവരോടൊപ്പം പുറത്തുപോകുകയില്ല തന്നെ. അവര്‍ ഒരു യുദ്ധത്തെ നേരിട്ടാല്‍ ഇവര്‍ അവരെ സഹായിക്കുകയുമില്ല. ഇനി ഇവര്‍ അവരെ സഹായിച്ചാല്‍ തന്നെ ഇവര്‍ പിന്തിരിഞ്ഞോടും തീര്‍ച്ച. പിന്നീട് അവര്‍ക്ക് ഒരു സഹായവും ലഭിക്കുകയില്ല.
\end{malayalam}}
\flushright{\begin{Arabic}
\quranayah[59][13]
\end{Arabic}}
\flushleft{\begin{malayalam}
തീര്‍ച്ചയായും അവരുടെ മനസ്സുകളില്‍ അല്ലാഹുവെക്കാള്‍ കൂടുതല്‍ ഭയമുള്ളത് നിങ്ങളെ പറ്റിയാകുന്നു. അവര്‍ കാര്യം ഗ്രഹിക്കാത്ത ഒരു ജനതയായത് കൊണ്ടാകുന്നു അത്‌.
\end{malayalam}}
\flushright{\begin{Arabic}
\quranayah[59][14]
\end{Arabic}}
\flushleft{\begin{malayalam}
കോട്ടകെട്ടിയ പട്ടണങ്ങളില്‍ വെച്ചോ മതിലുകളുടെ പിന്നില്‍ നിന്നോ അല്ലാതെ അവര്‍ ഒരുമിച്ച് നിങ്ങളോട് യുദ്ധം ചെയ്യുകയില്ല. അവര്‍ തമ്മില്‍ തന്നെയുള്ള പോരാട്ടം കടുത്തതാകുന്നു. അവര്‍ ഒരുമിച്ചാണെന്ന് നീ വിചാരിക്കുന്നു. അവരുടെ ഹൃദയങ്ങള്‍ ഭിന്നിപ്പിലാകുന്നു. അവര്‍ ചിന്തിച്ചു മനസ്സിലാക്കാത്ത ഒരു ജനതയായത് കൊണ്ടത്രെ അത്‌.
\end{malayalam}}
\flushright{\begin{Arabic}
\quranayah[59][15]
\end{Arabic}}
\flushleft{\begin{malayalam}
അവര്‍ക്കു മുമ്പ് അടുത്ത് തന്നെ കഴിഞ്ഞുപോയവരുടെ സ്ഥിതി പോലെത്തന്നെ. അവര്‍ ചെയ്തിരുന്ന കാര്യങ്ങളുടെ ദുഷ്ഫലം അവര്‍ ആസ്വദിച്ചു കഴിഞ്ഞു. അവര്‍ക്ക് വേദനയേറിയ ശിക്ഷയുമുണ്ട്‌.
\end{malayalam}}
\flushright{\begin{Arabic}
\quranayah[59][16]
\end{Arabic}}
\flushleft{\begin{malayalam}
പിശാചിന്‍റെ അവസ്ഥ പോലെ തന്നെ. മനുഷ്യനോട്‌, നീ അവിശ്വാസിയാകൂ എന്ന് അവന്‍ പറഞ്ഞ സന്ദര്‍ഭം. അങ്ങനെ അവന്‍ അവിശ്വസിച്ചു കഴിഞ്ഞപ്പോള്‍ അവന്‍ (പിശാച്‌) പറഞ്ഞു: തീര്‍ച്ചയായും ഞാന്‍ നീയുമായുള്ള ബന്ധത്തില്‍ നിന്ന് വിമുക്തനാകുന്നു. തീര്‍ച്ചയായും ലോകരക്ഷിതാവായ അല്ലാഹുവെ ഞാന്‍ ഭയപ്പെടുന്നു.
\end{malayalam}}
\flushright{\begin{Arabic}
\quranayah[59][17]
\end{Arabic}}
\flushleft{\begin{malayalam}
അങ്ങനെ അവര്‍ ഇരുവരുടെയും പര്യവസാനം അവര്‍ നരകത്തില്‍ നിത്യവാസികളായി കഴിയുക എന്നതായിത്തീര്‍ന്നു. അതത്രെ അക്രമകാരികള്‍ക്കുള്ള പ്രതിഫലം.
\end{malayalam}}
\flushright{\begin{Arabic}
\quranayah[59][18]
\end{Arabic}}
\flushleft{\begin{malayalam}
സത്യവിശ്വാസികളേ, നിങ്ങള്‍ അല്ലാഹുവെ സൂക്ഷിക്കുക. ഓരോ വ്യക്തിയും താന്‍ നാളെക്ക് വേണ്ടി എന്തൊരു മുന്നൊരുക്കമാണ് ചെയ്തു വെച്ചിട്ടുള്ളതെന്ന് നോക്കിക്കൊള്ളട്ടെ. നിങ്ങള്‍ അല്ലാഹുവെ സൂക്ഷിക്കുക. തീര്‍ച്ചയായും അല്ലാഹു നിങ്ങള്‍ പ്രവര്‍ത്തിക്കുന്നതിനെ പറ്റി സൂക്ഷ്മജ്ഞാനമുള്ളവനാകുന്നു.
\end{malayalam}}
\flushright{\begin{Arabic}
\quranayah[59][19]
\end{Arabic}}
\flushleft{\begin{malayalam}
അല്ലാഹുവെ മറന്നുകളഞ്ഞ ഒരു വിഭാഗത്തെ പോലെ നിങ്ങളാകരുത്‌. തന്‍മൂലം അല്ലാഹു അവര്‍ക്ക് അവരെ പറ്റി തന്നെ ഓര്‍മയില്ലാതാക്കി. അക്കൂട്ടര്‍ തന്നെയാകുന്നു ദുര്‍മാര്‍ഗികള്‍.
\end{malayalam}}
\flushright{\begin{Arabic}
\quranayah[59][20]
\end{Arabic}}
\flushleft{\begin{malayalam}
നരകാവകാശികളും സ്വര്‍ഗാവകാശികളും സമമാകുകയില്ല. സ്വര്‍ഗാവകാശികള്‍ തന്നെയാകുന്നു വിജയം നേടിയവര്‍.
\end{malayalam}}
\flushright{\begin{Arabic}
\quranayah[59][21]
\end{Arabic}}
\flushleft{\begin{malayalam}
ഈ ഖുര്‍ആനിനെ നാം ഒരു പര്‍വ്വതത്തിന്മേല്‍ അവതരിപ്പിച്ചിരുന്നുവെങ്കില്‍ അത് (പര്‍വ്വതം) വിനീതമാകുന്നതും അല്ലാഹുവെപ്പറ്റിയുള്ള ഭയത്താല്‍ പൊട്ടിപ്പിളരുന്നതും നിനക്കു കാണാമായിരുന്നു. ആ ഉദാഹരണങ്ങള്‍ നാം ജനങ്ങള്‍ക്ക് വേണ്ടി വിവരിക്കുന്നു. അവര്‍ ചിന്തിക്കുവാന്‍ വേണ്ടി.
\end{malayalam}}
\flushright{\begin{Arabic}
\quranayah[59][22]
\end{Arabic}}
\flushleft{\begin{malayalam}
താനല്ലാതെ യാതൊരു ആരാധ്യനുമില്ലാത്തവനായ അല്ലാഹുവാണവന്‍. അദൃശ്യവും ദൃശ്യവും അറിയുന്നവനാകുന്നു അവന്‍. അവന്‍ പരമകാരുണികനും കരുണാനിധിയുമാകുന്നു.
\end{malayalam}}
\flushright{\begin{Arabic}
\quranayah[59][23]
\end{Arabic}}
\flushleft{\begin{malayalam}
താനല്ലാതെ യാതൊരു ആരാധ്യനുമില്ലാത്തവനായ അല്ലാഹുവാണവന്‍. രാജാധികാരമുള്ളവനും പരമപരിശുദ്ധനും സമാധാനം നല്‍കുന്നവനും അഭയം നല്‍കുന്നവനും മേല്‍നോട്ടം വഹിക്കുന്നവനും പ്രതാപിയും പരമാധികാരിയും മഹത്വമുള്ളവനും ആകുന്നു അവന്‍. അവര്‍ പങ്കുചേര്‍ക്കുന്നതില്‍ നിന്നെല്ലാം അല്ലാഹു എത്രയോ പരിശുദ്ധന്‍!
\end{malayalam}}
\flushright{\begin{Arabic}
\quranayah[59][24]
\end{Arabic}}
\flushleft{\begin{malayalam}
സ്രഷ്ടാവും നിര്‍മാതാവും രൂപം നല്‍കുന്നവനുമായ അല്ലാഹുവത്രെ അവന്‍. അവന് ഏറ്റവും ഉത്തമമായ നാമങ്ങളുണ്ട്‌. ആകാശങ്ങളിലും ഭൂമിയിലുള്ളവ അവന്‍റെ മഹത്വത്തെ പ്രകീര്‍ത്തിക്കുന്നു. അവനത്രെ പ്രതാപിയും യുക്തിമാനും.
\end{malayalam}}
\chapter{\textmalayalam{മുംതഹന (. പരീക്ഷിക്കപ്പെടേണ്ടവള്‍ )}}
\begin{Arabic}
\Huge{\centerline{\basmalah}}\end{Arabic}
\flushright{\begin{Arabic}
\quranayah[60][1]
\end{Arabic}}
\flushleft{\begin{malayalam}
ഹേ; സത്യവിശ്വാസികളേ, എന്‍റെ ശത്രുവും നിങ്ങളുടെ ശത്രുവും ആയിട്ടുള്ളവരോട് സ്നേഹബന്ധം സ്ഥാപിച്ച് കൊണ്ട് നിങ്ങള്‍ അവരെ മിത്രങ്ങളാക്കി വെക്കരുത്‌. നിങ്ങള്‍ക്കു വന്നുകിട്ടിയിട്ടുള്ള സത്യത്തില്‍ അവര്‍ അവിശ്വസിച്ചിരിക്കുകയാണ്‌. നിങ്ങള്‍ നിങ്ങളുടെ രക്ഷിതാവായ അല്ലാഹുവില്‍ വിശ്വസിക്കുന്നതിനാല്‍ റസൂലിനെയും നിങ്ങളെയും അവര്‍ നാട്ടില്‍ നിന്നു പുറത്താക്കുന്നു. എന്‍റെ മാര്‍ഗത്തില്‍ സമരം ചെയ്യുവാനും എന്‍റെ പ്രീതിതേടുവാനും നിങ്ങള്‍ പുറപ്പെട്ടിരിക്കുകയാണെങ്കില്‍ (നിങ്ങള്‍ അപ്രകാരം മൈത്രീ ബന്ധം സ്ഥാപിക്കരുത്‌.) നിങ്ങള്‍ അവരുമായി രഹസ്യമായി സ്നേഹബന്ധം സ്ഥാപിക്കുന്നു. നിങ്ങള്‍ രഹസ്യമാക്കിയതും പരസ്യമാക്കിയതും ഞാന്‍ നല്ലവണ്ണം അറിയുന്നവനാണ്‌. നിങ്ങളില്‍ നിന്ന് വല്ലവനും അപ്രകാരം പ്രവര്‍ത്തിക്കുന്ന പക്ഷം അവന്‍ നേര്‍മാര്‍ഗത്തില്‍ നിന്ന് പിഴച്ചു പോയിരിക്കുന്നു.
\end{malayalam}}
\flushright{\begin{Arabic}
\quranayah[60][2]
\end{Arabic}}
\flushleft{\begin{malayalam}
അവര്‍ നിങ്ങളെ കണ്ടുമുട്ടുന്ന പക്ഷം അവര്‍ നിങ്ങള്‍ക്ക് ശത്രുക്കളായിരിക്കും. നിങ്ങളുടെ നേര്‍ക്ക് ദുഷ്ടതയും കൊണ്ട് അവരുടെ കൈകളും നാവുകളും അവര്‍ നീട്ടുകയും നിങ്ങള്‍ അവിശ്വസിച്ചിരുന്നെങ്കില്‍ എന്ന് അവര്‍ ആഗ്രഹിക്കുകയും ചെയ്യും.
\end{malayalam}}
\flushright{\begin{Arabic}
\quranayah[60][3]
\end{Arabic}}
\flushleft{\begin{malayalam}
ഉയിര്‍ത്തെഴുന്നേല്‍പിന്‍റെ നാളില്‍ നിങ്ങളുടെ രക്ത ബന്ധങ്ങളോ നിങ്ങളുടെ സന്താനങ്ങളോ നിങ്ങള്‍ക്ക് പ്രയോജനപ്പെടുകയില്ല തന്നെ. അല്ലാഹു നിങ്ങളെ തമ്മില്‍ വേര്‍പിരിക്കും. അല്ലാഹു നിങ്ങള്‍ പ്രവര്‍ത്തിക്കുന്നത് കണ്ടറിയുന്നവനാകുന്നു.
\end{malayalam}}
\flushright{\begin{Arabic}
\quranayah[60][4]
\end{Arabic}}
\flushleft{\begin{malayalam}
നിങ്ങള്‍ക്ക് ഇബ്രാഹീമിലും അദ്ദേഹത്തിന്‍റെ കൂടെയുള്ളവരിലും ഉത്തമമായ ഒരു മാതൃക ഉണ്ടായിട്ടുണ്ട്‌. അവര്‍ തങ്ങളുടെ ജനതയോട് ഇപ്രകാരം പറഞ്ഞ സന്ദര്‍ഭം: നിങ്ങളുമായും അല്ലാഹുവിന് പുറമെ നിങ്ങള്‍ ആരാധിക്കുന്നവയുമായുള്ള ബന്ധത്തില്‍ നിന്നു തീര്‍ച്ചയായും ഞങ്ങള്‍ ഒഴിവായവരാകുന്നു. നിങ്ങളില്‍ ഞങ്ങള്‍ അവിശ്വസിച്ചിരിക്കുന്നു. നിങ്ങള്‍ അല്ലാഹുവില്‍ മാത്രം വിശ്വസിക്കുന്നത് വരെ എന്നെന്നേക്കുമായി ഞങ്ങളും നിങ്ങളും തമ്മില്‍ ശത്രുതയും വിദ്വേഷവും പ്രകടമാവുകയും ചെയ്തിരിക്കുന്നു. തീര്‍ച്ചയായും ഞാന്‍ താങ്കള്‍ക്ക് വേണ്ടി പാപമോചനം തേടാം, താങ്കള്‍ക്ക് വേണ്ടി അല്ലാഹുവിങ്കല്‍ നിന്ന് യാതൊന്നും എനിക്ക് അധീനപ്പെടുത്താനാവില്ല എന്ന് ഇബ്രാഹീം തന്‍റെ പിതാവിനോട് പറഞ്ഞ വാക്കൊഴികെ. (അവര്‍ ഇപ്രകാരം പ്രാര്‍ത്ഥിച്ചിരുന്നു:) ഞങ്ങളുടെ രക്ഷിതാവേ, നിന്‍റെ മേല്‍ ഞങ്ങള്‍ ഭരമേല്‍പിക്കുകയും, നിങ്കലേക്ക് ഞങ്ങള്‍ മടങ്ങുകയും ചെയ്തിരിക്കുന്നു. നിങ്കലേക്ക് തന്നെയാണ് തിരിച്ചുവരവ്‌.
\end{malayalam}}
\flushright{\begin{Arabic}
\quranayah[60][5]
\end{Arabic}}
\flushleft{\begin{malayalam}
ഞങ്ങളുടെ രക്ഷിതാവേ, ഞങ്ങളെ സത്യനിഷേധികളുടെ പരീക്ഷണത്തിന് ഇരയാക്കരുതേ. ഞങ്ങളുടെ രക്ഷിതാവേ, ഞങ്ങള്‍ക്ക് നീ പൊറുത്തുതരികയും ചെയ്യേണമേ. തീര്‍ച്ചയായും നീ തന്നെയാണ് പ്രതാപിയും യുക്തിമാനും
\end{malayalam}}
\flushright{\begin{Arabic}
\quranayah[60][6]
\end{Arabic}}
\flushleft{\begin{malayalam}
തീര്‍ച്ചയായും നിങ്ങള്‍ക്ക് -അല്ലാഹുവെയും അന്ത്യദിനത്തെയും പ്രതീക്ഷിക്കുന്നവര്‍ക്ക് -അവരില്‍ ഉത്തമ മാതൃകയുണ്ടായിട്ടുണ്ട്‌. ആരെങ്കിലും തിരിഞ്ഞുകളയുന്ന പക്ഷം തീര്‍ച്ചയായും അല്ലാഹു തന്നെയാകുന്നു പരാശ്രയമുക്തനും സ്തുത്യര്‍ഹനുമായിട്ടുള്ളവന്‍.
\end{malayalam}}
\flushright{\begin{Arabic}
\quranayah[60][7]
\end{Arabic}}
\flushleft{\begin{malayalam}
നിങ്ങള്‍ക്കും അവരില്‍ നിന്ന് നിങ്ങള്‍ ശത്രുത പുലര്‍ത്തിയവര്‍ക്കുമിടയില്‍ അല്ലാഹു സ്നേഹബന്ധമുണ്ടാക്കിയേക്കാം. അല്ലാഹു കഴിവുള്ളവനാണ്‌. അല്ലാഹു ഏറെ പൊറുക്കുന്നവനും കരുണാനിധിയുമാകുന്നു.
\end{malayalam}}
\flushright{\begin{Arabic}
\quranayah[60][8]
\end{Arabic}}
\flushleft{\begin{malayalam}
മതകാര്യത്തില്‍ നിങ്ങളോട് യുദ്ധം ചെയ്യാതിരിക്കുകയും, നിങ്ങളുടെ വീടുകളില്‍ നിന്ന് നിങ്ങളെ പുറത്താക്കാതിരിക്കുകയും ചെയ്യുന്നവരെ സംബന്ധിച്ചിടത്തോളം നിങ്ങളവര്‍ക്ക് നന്‍മ ചെയ്യുന്നതും നിങ്ങളവരോട് നീതി കാണിക്കുന്നതും അല്ലാഹു നിങ്ങളോട് നിരോധിക്കുന്നില്ല. തീര്‍ച്ചയായും അല്ലാഹു നീതി പാലിക്കുന്നവരെ ഇഷ്ടപ്പെടുന്നു.
\end{malayalam}}
\flushright{\begin{Arabic}
\quranayah[60][9]
\end{Arabic}}
\flushleft{\begin{malayalam}
മതകാര്യത്തില്‍ നിങ്ങളോട് യുദ്ധം ചെയ്യുകയും നിങ്ങളുടെ വീടുകളില്‍ നിന്ന് നിങ്ങളെ പുറത്താക്കുകയും നിങ്ങളെ പുറത്താക്കുന്നതില്‍ പരസ്പരം സഹകരിക്കുകയും ചെയ്തവരെ സംബന്ധിച്ചുമാത്രമാണ് -അവരോട് മൈത്രികാണിക്കുന്നത് - അല്ലാഹു നിരോധിക്കുന്നത്‌. വല്ലവരും അവരോട് മൈത്രീ ബന്ധം പുലര്‍ത്തുന്ന പക്ഷം അവര്‍ തന്നെയാകുന്നു അക്രമകാരികള്‍.
\end{malayalam}}
\flushright{\begin{Arabic}
\quranayah[60][10]
\end{Arabic}}
\flushleft{\begin{malayalam}
സത്യവിശ്വാസികളേ, വിശ്വാസിനികളായ സ്ത്രീകള്‍ അഭയാര്‍ത്ഥികളായി കൊണ്ട് നിങ്ങളുടെ അടുത്ത് വന്നാല്‍ നിങ്ങള്‍ അവരെ പരീക്ഷിച്ച് നോക്കണം. അവരുടെ വിശ്വാസത്തെ പറ്റി അല്ലാഹു ഏറ്റവും അറിയുന്നവനാണ്‌. എന്നിട്ട് അവര്‍ വിശ്വാസിനികളാണെന്ന് അറിഞ്ഞ് കഴിഞ്ഞാല്‍ അവരെ നിങ്ങള്‍ സത്യനിഷേധികളുടെ അടുത്തേക്ക് മടക്കി അയക്കരുത്‌. ആ സ്ത്രീകള്‍ അവര്‍ക്ക് അനുവദനീയമല്ല. അവര്‍ക്ക് അവര്‍ ചെലവഴിച്ചത് നിങ്ങള്‍ നല്‍കുകയും വേണം. ആ സ്ത്രീകള്‍ക്ക് അവരുടെ പ്രതിഫലങ്ങള്‍ നിങ്ങള്‍ കൊടുത്താല്‍ അവരെ നിങ്ങള്‍ വിവാഹം കഴിക്കുന്നതിന് നിങ്ങള്‍ക്ക് വിരോധമില്ല. അവിശ്വാസിനികളുമായുള്ള ബന്ധം നിങ്ങള്‍ മുറുകെപിടിച്ചു കൊണ്ടിരിക്കുകയും ചെയ്യരുത്‌. നിങ്ങള്‍ ചെലവഴിച്ചതെന്തോ, അത് നിങ്ങള്‍ ചോദിച്ചു കൊള്ളുക. അവര്‍ ചെലവഴിച്ചതെന്തോ അത് അവരും ചോദിച്ച് കൊള്ളട്ടെ. അതാണ് അല്ലാഹുവിന്‍റെ വിധി. അവന്‍ നിങ്ങള്‍ക്കിടയില്‍ വിധികല്‍പിക്കുന്നു. അല്ലാഹു സര്‍വ്വജ്ഞനും യുക്തിമാനുമാകുന്നു.
\end{malayalam}}
\flushright{\begin{Arabic}
\quranayah[60][11]
\end{Arabic}}
\flushleft{\begin{malayalam}
നിങ്ങളുടെ ഭാര്യമാരില്‍ നിന്ന് വല്ലവരും അവിശ്വാസികളുടെ കൂട്ടത്തിലേക്ക് (പോയിട്ട് നിങ്ങള്‍ക്ക്‌) നഷ്ടപ്പെടുകയും എന്നിട്ട് നിങ്ങള്‍ അനന്തര നടപടിയെടുക്കുകയും ചെയ്യുകയാണെങ്കില്‍ ആരുടെ ഭാര്യമാരാണോ നഷ്ടപ്പെട്ട് പോയത്‌, അവര്‍ക്ക് അവര്‍ ചെലവഴിച്ച തുക (മഹ്ര്) പോലുള്ളത് നിങ്ങള്‍ നല്‍കുക. ഏതൊരു അല്ലാഹുവില്‍ നിങ്ങള്‍ വിശ്വസിക്കുന്നുവോ അവനെ നിങ്ങള്‍ സൂക്ഷിക്കുകയും ചെയ്യുക.
\end{malayalam}}
\flushright{\begin{Arabic}
\quranayah[60][12]
\end{Arabic}}
\flushleft{\begin{malayalam}
ഓ; നബീ, അല്ലാഹുവോട് യാതൊന്നിനെയും പങ്കുചേര്‍ക്കുകയില്ലെന്നും, മോഷ്ടിക്കുകയില്ലെന്നും, വ്യഭിചരിക്കുകയില്ലെന്നും, തങ്ങളുടെ മക്കളെ കൊന്നുകളയുകയില്ലെന്നും, തങ്ങളുടെ കൈകാലുകള്‍ക്കിടയില്‍ വ്യാജവാദം കെട്ടിച്ചമച്ചു കൊണ്ടുവരികയില്ലെന്നും, യാതൊരു നല്ലകാര്യത്തിലും നിന്നോട് അനുസരണക്കേട് കാണിക്കുകയില്ലെന്നും നിന്നോട് പ്രതിജ്ഞ ചെയ്തുകൊണ്ട് സത്യവിശ്വാസിനികള്‍ നിന്‍റെ അടുത്ത് വന്നാല്‍ നീ അവരുടെ പ്രതിജ്ഞ സ്വീകരിക്കുകയും, അവര്‍ക്കു വേണ്ടി പാപമോചനം തേടുകയും ചെയ്യുക. തീര്‍ച്ചയായും അല്ലാഹു ഏറെ പൊറുക്കുന്നവനും കരുണാനിധിയുമാകുന്നു.
\end{malayalam}}
\flushright{\begin{Arabic}
\quranayah[60][13]
\end{Arabic}}
\flushleft{\begin{malayalam}
സത്യവിശ്വാസികളേ, അല്ലാഹു കോപിച്ചിട്ടുള്ള ഒരു ജനതയോട് നിങ്ങള്‍ മൈത്രിയില്‍ ഏര്‍പെടരുത്‌. ഖബ്‌റുകളിലുള്ളവരെ സംബന്ധിച്ച് അവിശ്വാസികള്‍ നിരാശപ്പെട്ടത് പോലെ പരലോകത്തെപ്പറ്റി അവര്‍ നിരാശപ്പെട്ടുകഴിഞ്ഞിരിക്കുന്നു.
\end{malayalam}}
\chapter{\textmalayalam{സ്വഫ്ഫ് ( അണി )}}
\begin{Arabic}
\Huge{\centerline{\basmalah}}\end{Arabic}
\flushright{\begin{Arabic}
\quranayah[61][1]
\end{Arabic}}
\flushleft{\begin{malayalam}
ആകാശങ്ങളിലുള്ളതും ഭൂമിയിലുള്ളതും അല്ലാഹുവിന്‍റെ മഹത്വം പ്രകീര്‍ത്തിച്ചിരിക്കുന്നു. അവനാകുന്നു പ്രതാപിയും യുക്തിമാനും.
\end{malayalam}}
\flushright{\begin{Arabic}
\quranayah[61][2]
\end{Arabic}}
\flushleft{\begin{malayalam}
സത്യവിശ്വാസികളേ, നിങ്ങള്‍ ചെയ്യാത്തതെന്തിന് നിങ്ങള്‍ പറയുന്നു?
\end{malayalam}}
\flushright{\begin{Arabic}
\quranayah[61][3]
\end{Arabic}}
\flushleft{\begin{malayalam}
നിങ്ങള്‍ ചെയ്യാത്തത് നിങ്ങള്‍ പറയുക എന്നുള്ളത് അല്ലാഹുവിങ്കല്‍ വലിയ ക്രോധത്തിന് കാരണമായിരിക്കുന്നു.
\end{malayalam}}
\flushright{\begin{Arabic}
\quranayah[61][4]
\end{Arabic}}
\flushleft{\begin{malayalam}
(കല്ലുകള്‍) സുദൃഢമായി സംയോജിപ്പിച്ച ഒരു മതില്‍ പോലെ അണിചേര്‍ന്നുകൊണ്ട് തന്‍റെ മാര്‍ഗത്തില്‍ യുദ്ധം ചെയ്യുന്നവരെ തീര്‍ച്ചയായും അല്ലാഹു ഇഷ്ടപ്പെടുന്നു.
\end{malayalam}}
\flushright{\begin{Arabic}
\quranayah[61][5]
\end{Arabic}}
\flushleft{\begin{malayalam}
മൂസാ തന്‍റെ ജനതയോട് ഇപ്രകാരം പറഞ്ഞ സന്ദര്‍ഭം (ശ്രദ്ധേയമാകുന്നു:) എന്‍റെ ജനങ്ങളേ, നിങ്ങള്‍ എന്തിനാണ് എന്നെ ഉപദ്രവിക്കുന്നത്‌? ഞാന്‍ നിങ്ങളിലേക്കുള്ള അല്ലാഹുവിന്‍റെ ദൂതനാണെന്ന് നിങ്ങള്‍ക്കറിയാമല്ലോ. അങ്ങനെ അവര്‍ തെറ്റിയപ്പോള്‍ അല്ലാഹു അവരുടെ ഹൃദയങ്ങളെ തെറ്റിച്ചുകളഞ്ഞു. അല്ലാഹു ദുര്‍മാര്‍ഗികളായ ജനങ്ങളെ നേര്‍വഴിയിലാക്കുകയില്ല.
\end{malayalam}}
\flushright{\begin{Arabic}
\quranayah[61][6]
\end{Arabic}}
\flushleft{\begin{malayalam}
മര്‍യമിന്‍റെ മകന്‍ ഈസാ പറഞ്ഞ സന്ദര്‍ഭവും (ശ്രദ്ധേയമാകുന്നു:) ഇസ്രായീല്‍ സന്തതികളേ, എനിക്കു മുമ്പുള്ള തൌറാത്തിനെ സത്യപ്പെടുത്തുന്നവനായിക്കൊണ്ടും, എനിക്ക് ശേഷം വരുന്ന അഹ്മദ് എന്നുപേരുള്ള ഒരു ദൂതനെപ്പറ്റി സന്തോഷവാര്‍ത്ത അറിയിക്കുന്നവനായിക്കൊണ്ടും നിങ്ങളിലേക്ക് അല്ലാഹുവിന്‍റെ ദൂതനായി നിയോഗിക്കപ്പെട്ടവനാകുന്നു ഞാന്‍. അങ്ങനെ അദ്ദേഹം വ്യക്തമായ തെളിവുകളും കൊണ്ട് അവരുടെ അടുത്ത് ചെന്നപ്പോള്‍ അവര്‍ പറഞ്ഞു: ഇത് വ്യക്തമായ ജാലവിദ്യയാകുന്നു.
\end{malayalam}}
\flushright{\begin{Arabic}
\quranayah[61][7]
\end{Arabic}}
\flushleft{\begin{malayalam}
താന്‍ ഇസ്ലാമിലേക്ക് ക്ഷണിക്കപ്പെടുമ്പോള്‍ അല്ലാഹുവിന്‍റെ പേരില്‍ കള്ളം കെട്ടിച്ചമച്ചവനേക്കാള്‍ വലിയ അക്രമി ആരുണ്ട്‌? അല്ലാഹു അക്രമികളായ ജനങ്ങളെ നേര്‍വഴിയിലാക്കുകയില്ല.
\end{malayalam}}
\flushright{\begin{Arabic}
\quranayah[61][8]
\end{Arabic}}
\flushleft{\begin{malayalam}
അവര്‍ അവരുടെ വായ്കൊണ്ട് അല്ലാഹുവിന്‍റെ പ്രകാശം കെടുത്തിക്കളയാനാണ് ഉദ്ദേശിക്കുന്നത്‌. സത്യനിഷേധികള്‍ക്ക് അനിഷ്ടകരമായാലും അല്ലാഹു അവന്‍റെ പ്രകാശം പൂര്‍ത്തിയാക്കുന്നവനാകുന്നു.
\end{malayalam}}
\flushright{\begin{Arabic}
\quranayah[61][9]
\end{Arabic}}
\flushleft{\begin{malayalam}
സന്‍മാര്‍ഗവും സത്യമതവും കൊണ്ട് -എല്ലാ മതങ്ങള്‍ക്കും മീതെ അതിനെ തെളിയിച്ചു കാണിക്കുവാന്‍ വേണ്ടി-തന്‍റെ ദൂതനെ നിയോഗിച്ചവനാകുന്നു അവന്‍. ബഹുദൈവാരാധകര്‍ക്ക് (അത്‌) അനിഷ്ടകരമായാലും ശരി.
\end{malayalam}}
\flushright{\begin{Arabic}
\quranayah[61][10]
\end{Arabic}}
\flushleft{\begin{malayalam}
സത്യവിശ്വാസികളേ, വേദനാജനകമായ ശിക്ഷയില്‍ നിന്ന് നിങ്ങളെ രക്ഷിക്കുന്ന ഒരു കച്ചവടത്തെപ്പറ്റി ഞാന്‍ നിങ്ങള്‍ക്ക് അറിയിച്ച് തരട്ടെയോ?
\end{malayalam}}
\flushright{\begin{Arabic}
\quranayah[61][11]
\end{Arabic}}
\flushleft{\begin{malayalam}
നിങ്ങള്‍ അല്ലാഹുവിലും അവന്‍റെ ദൂതനിലും വിശ്വസിക്കണം.അല്ലാഹുവിന്‍റെ മാര്‍ഗത്തില്‍ നിങ്ങളുടെ സ്വത്തുക്കള്‍ കൊണ്ടും ശരീരങ്ങള്‍ കൊണ്ടും നിങ്ങള്‍ സമരം ചെയ്യുകയും വേണം. അതാണ് നിങ്ങള്‍ക്ക് ഗുണകരമായിട്ടുള്ളത്‌. നിങ്ങള്‍ അറിവുള്ളവരാണെങ്കില്‍.
\end{malayalam}}
\flushright{\begin{Arabic}
\quranayah[61][12]
\end{Arabic}}
\flushleft{\begin{malayalam}
എങ്കില്‍ അവന്‍ നിങ്ങള്‍ക്ക് നിങ്ങളുടെ പാപങ്ങള്‍ പൊറുത്തുതരികയും താഴ്ഭാഗത്ത്കൂടി അരുവികള്‍ ഒഴുകുന്ന സ്വര്‍ഗത്തോപ്പുകളിലും, സ്ഥിരവാസത്തിനുള്ള സ്വര്‍ഗത്തോപ്പുകളിലെ വിശിഷ്ടമായ വസതികളിലും അവന്‍ നിങ്ങളെ പ്രവേശിപ്പിക്കുകയും ചെയ്യുന്നതാണ്‌. അതത്രെ മഹത്തായ ഭാഗ്യം.
\end{malayalam}}
\flushright{\begin{Arabic}
\quranayah[61][13]
\end{Arabic}}
\flushleft{\begin{malayalam}
നിങ്ങള്‍ ഇഷ്ടപ്പെടുന്ന മറ്റൊരു കാര്യവും (അവന്‍ നല്‍കുന്നതാണ്‌.) അതെ, അല്ലാഹുവിങ്കല്‍ നിന്നുള്ള സഹായവും ആസന്നമായ വിജയവും. (നബിയേ,) സത്യവിശ്വാസികള്‍ക്ക് നീ സന്തോഷവാര്‍ത്ത അറിയിക്കുക.
\end{malayalam}}
\flushright{\begin{Arabic}
\quranayah[61][14]
\end{Arabic}}
\flushleft{\begin{malayalam}
സത്യവിശ്വാസികളേ, നിങ്ങള്‍ അല്ലാഹുവിന്‍റെ സഹായികളായിരിക്കുക. മര്‍യമിന്‍റെ മകന്‍ ഈസാ അല്ലാഹുവിങ്കലേക്കുള്ള മാര്‍ഗത്തില്‍ എന്‍റെ സഹായികളായി ആരുണ്ട് എന്ന് ഹവാരികളോട് ചോദിച്ചതു പോലെ. ഹവാരികള്‍ പറഞ്ഞു: ഞങ്ങള്‍ അല്ലാഹുവിന്‍റെ സഹായികളാകുന്നു. അപ്പോള്‍ ഇസ്രായീല്‍ സന്തതികളില്‍ പെട്ട ഒരു വിഭാഗം വിശ്വസിക്കുകയും മറ്റൊരു വിഭാഗം അവിശ്വസിക്കുകയും ചെയ്തു. എന്നിട്ട് വിശ്വസിച്ചവര്‍ക്ക് അവരുടെ ശത്രുവിനെതിരില്‍ നാം പിന്‍ബലം നല്‍കുകയും അങ്ങനെ അവന്‍ മികവുറ്റവരായിത്തീരുകയും ചെയ്തു.
\end{malayalam}}
\chapter{\textmalayalam{ജുമുഅ}}
\begin{Arabic}
\Huge{\centerline{\basmalah}}\end{Arabic}
\flushright{\begin{Arabic}
\quranayah[62][1]
\end{Arabic}}
\flushleft{\begin{malayalam}
രാജാവും പരമപരിശുദ്ധനും പ്രതാപശാലിയും യുക്തിമാനുമായ അല്ലാഹുവെ ആകാശങ്ങളിലുള്ളതും ഭൂമിയിലുള്ളതുമെല്ലാം പ്രകീര്‍ത്തിച്ചു കൊണ്ടിരിക്കുന്നു.
\end{malayalam}}
\flushright{\begin{Arabic}
\quranayah[62][2]
\end{Arabic}}
\flushleft{\begin{malayalam}
അക്ഷരജ്ഞാനമില്ലാത്തവര്‍ക്കിടയില്‍, തന്‍റെ ദൃഷ്ടാന്തങ്ങള്‍ അവര്‍ക്ക് വായിച്ചുകേള്‍പിക്കുകയും അവരെ സംസ്കരിക്കുകയും അവര്‍ക്ക് വേദഗ്രന്ഥവും തത്വജ്ഞാനവും പഠിപ്പിക്കുകയും ചെയ്യാന്‍ അവരില്‍ നിന്നുതന്നെയുള്ള ഒരു ദൂതനെ നിയോഗിച്ചവനാകുന്നു അവന്‍. തീര്‍ച്ചയായും അവര്‍ മുമ്പ് വ്യക്തമായ വഴികേടിലായിരുന്നു.
\end{malayalam}}
\flushright{\begin{Arabic}
\quranayah[62][3]
\end{Arabic}}
\flushleft{\begin{malayalam}
അവരില്‍പെട്ട ഇനിയും അവരോടൊപ്പം വന്നുചേര്‍ന്നിട്ടില്ലാത്ത മറ്റുള്ളവരിലേക്കും (അദ്ദേഹത്തെ നിയോഗിച്ചിരിക്കുന്നു.) അവനാകുന്നു പ്രതാപിയും യുക്തിമാനും.
\end{malayalam}}
\flushright{\begin{Arabic}
\quranayah[62][4]
\end{Arabic}}
\flushleft{\begin{malayalam}
അത് അല്ലാഹുവിന്‍റെ അനുഗ്രഹമാകുന്നു. അവന്‍ ഉദ്ദേശിക്കുന്നവര്‍ക്ക് അവന്‍ അത് നല്‍കുന്നു. അല്ലാഹു മഹത്തായ പ്രതിഫലം നല്‍കുന്നവനത്രെ.
\end{malayalam}}
\flushright{\begin{Arabic}
\quranayah[62][5]
\end{Arabic}}
\flushleft{\begin{malayalam}
തൌറാത്ത് സ്വീകരിക്കാന്‍ ചുമതല ഏല്‍പിക്കപ്പെടുകയും, എന്നിട്ട് അത് ഏറ്റെടുക്കാതിരിക്കുകയും ചെയ്തവരുടെ (യഹൂദരുടെ) ഉദാഹരണം ഗ്രന്ഥങ്ങള്‍ ചുമക്കുന്ന കഴുതയുടേത് പോലെയാകുന്നു. അല്ലാഹുവിന്‍റെ ദൃഷ്ടാന്തങ്ങള്‍ നിഷേധിച്ചു കളഞ്ഞ ജനങ്ങളുടെ ഉപമ എത്രയോ ചീത്ത! അക്രമികളായ ജനങ്ങളെ അല്ലാഹു സന്‍മാര്‍ഗത്തിലാക്കുകയില്ല.
\end{malayalam}}
\flushright{\begin{Arabic}
\quranayah[62][6]
\end{Arabic}}
\flushleft{\begin{malayalam}
(നബിയേ,) പറയുക: തീര്‍ച്ചയായും യഹൂദികളായുള്ളവരേ, മറ്റു മനുഷ്യരെ കൂടാതെ നിങ്ങള്‍ മാത്രം അല്ലാഹുവിന്‍റെ മിത്രങ്ങളാണെന്ന് നിങ്ങള്‍ വാദിക്കുകയാണെങ്കില്‍ നിങ്ങള്‍ മരണം കൊതിക്കുക. നിങ്ങള്‍ സത്യവാന്‍മാരാണെങ്കില്‍.
\end{malayalam}}
\flushright{\begin{Arabic}
\quranayah[62][7]
\end{Arabic}}
\flushleft{\begin{malayalam}
എന്നാല്‍ അവരുടെ കൈകള്‍ മുന്‍കൂട്ടി ചെയ്തുവെച്ചതിന്‍റെ ഫലമായി അവര്‍ ഒരിക്കലും അത് കൊതിക്കുകയില്ല. അല്ലാഹു അക്രമകാരികളെപ്പറ്റി അറിവുള്ളവനാകുന്നു.
\end{malayalam}}
\flushright{\begin{Arabic}
\quranayah[62][8]
\end{Arabic}}
\flushleft{\begin{malayalam}
(നബിയേ,) പറയുക: തീര്‍ച്ചയായും ഏതൊരു മരണത്തില്‍ നിന്ന് നിങ്ങള്‍ ഓടി അകലുന്നുവോ അത് തീര്‍ച്ചയായും നിങ്ങളുമായി കണ്ടുമുട്ടുന്നതാണ്‌. പിന്നീട് അദൃശ്യവും, ദൃശ്യവും അറിയുന്നവന്‍റെ അടുക്കലേക്ക് നിങ്ങള്‍ മടക്കപ്പെടുകയും ചെയ്യും. അപ്പോള്‍ നിങ്ങള്‍ പ്രവര്‍ത്തിച്ചുകൊണ്ടിരിക്കുന്നതിനെ പറ്റി അവന്‍ നിങ്ങളെ വിവരമറിയിക്കുന്നതാണ്‌.
\end{malayalam}}
\flushright{\begin{Arabic}
\quranayah[62][9]
\end{Arabic}}
\flushleft{\begin{malayalam}
സത്യവിശ്വാസികളേ, വെള്ളിയാഴ്ച നമസ്കാരത്തിന് വിളിക്കപ്പെട്ടാല്‍ അല്ലാഹുവെ പറ്റിയുള്ള സ്മരണയിലേക്ക് നിങ്ങള്‍ വേഗത്തില്‍ വരികയും, വ്യാപാരം ഒഴിവാക്കുകയും ചെയ്യുക. അതാണ് നിങ്ങള്‍ക്ക് ഉത്തമം; നിങ്ങള്‍ കാര്യം മനസ്സിലാക്കുന്നുവെങ്കില്‍.
\end{malayalam}}
\flushright{\begin{Arabic}
\quranayah[62][10]
\end{Arabic}}
\flushleft{\begin{malayalam}
അങ്ങനെ നമസ്കാരം നിര്‍വഹിക്കപ്പെട്ടു കഴിഞ്ഞാല്‍ നിങ്ങള്‍ ഭൂമിയില്‍ വ്യാപിച്ചു കൊള്ളുകയും, അല്ലാഹുവിന്‍റെ അനുഗ്രഹത്തില്‍ നിന്ന് തേടിക്കൊള്ളുകയും ചെയ്യുക. നിങ്ങള്‍ അല്ലാഹുവെ ധാരാളമായി ഓര്‍ക്കുകയും ചെയ്യുക. നിങ്ങള്‍ വിജയം പ്രാപിച്ചേക്കാം.
\end{malayalam}}
\flushright{\begin{Arabic}
\quranayah[62][11]
\end{Arabic}}
\flushleft{\begin{malayalam}
അവര്‍ ഒരു കച്ചവടമോ വിനോദമോ കണ്ടാല്‍ അവയുടെ അടുത്തേക്ക് പിരിഞ്ഞ് പോകുകയും നിന്നനില്‍പില്‍ നിന്നെ വിട്ടേക്കുകയും ചെയ്യുന്നതാണ്‌. നീ പറയുക: അല്ലാഹുവിന്‍റെ അടുക്കലുള്ളത് വിനോദത്തെക്കാളും കച്ചവടത്തെക്കാളും ഉത്തമമാകുന്നു. അല്ലാഹു ഉപജീവനം നല്‍കുന്നവരില്‍ ഏറ്റവും ഉത്തമനാകുന്നു.
\end{malayalam}}
\chapter{\textmalayalam{മുനാഫിഖൂന്‍ ( കപടവിശ്വാസികള്‍ )}}
\begin{Arabic}
\Huge{\centerline{\basmalah}}\end{Arabic}
\flushright{\begin{Arabic}
\quranayah[63][1]
\end{Arabic}}
\flushleft{\begin{malayalam}
കപട വിശ്വാസികള്‍ നിന്‍റെ അടുത്ത് വന്നാല്‍ അവര്‍ പറയും: തീര്‍ച്ചയായും താങ്കള്‍ അല്ലാഹുവിന്‍റെ ദൂതനാണെന്ന് ഞങ്ങള്‍ സാക്ഷ്യം വഹിക്കുന്നു. അല്ലാഹുവിന്നറിയാം തീര്‍ച്ചയായും നീ അവന്‍റെ ദൂതനാണെന്ന്‌. തീര്‍ച്ചയായും മുനാഫിഖുകള്‍ (കപടന്‍മാര്‍) കള്ളം പറയുന്നവരാണ് എന്ന് അല്ലാഹു സാക്ഷ്യം വഹിക്കുന്നു.
\end{malayalam}}
\flushright{\begin{Arabic}
\quranayah[63][2]
\end{Arabic}}
\flushleft{\begin{malayalam}
അവര്‍ അവരുടെ ശപഥങ്ങളെ ഒരു പരിചയാക്കിയിരിക്കയാണ്‌. അങ്ങനെ അവര്‍ അല്ലാഹുവിന്‍റെ മാര്‍ഗത്തില്‍ നിന്ന് (ജനങ്ങളെ) തടഞ്ഞിരിക്കുന്നു. തീര്‍ച്ചയായും അവര്‍ പ്രവര്‍ത്തിക്കുന്നത് എത്രയോ ചീത്ത തന്നെ.
\end{malayalam}}
\flushright{\begin{Arabic}
\quranayah[63][3]
\end{Arabic}}
\flushleft{\begin{malayalam}
അത്‌, അവര്‍ ആദ്യം വിശ്വസിക്കുകയും പിന്നീട് അവിശ്വസിക്കുകയും ചെയ്തതിന്‍റെ ഫലമത്രെ. അങ്ങനെ അവരുടെ ഹൃദയങ്ങളിന്‍മേല്‍ മുദ്രവെക്കപ്പെട്ടിരിക്കുന്നു. അതിനാല്‍ അവര്‍ (കാര്യം) ഗ്രഹിക്കുകയില്ല.
\end{malayalam}}
\flushright{\begin{Arabic}
\quranayah[63][4]
\end{Arabic}}
\flushleft{\begin{malayalam}
നീ അവരെ കാണുകയാണെങ്കില്‍ അവരുടെ ശരീരങ്ങള്‍ നിന്നെ അത്ഭുതപ്പെടുത്തും. അവര്‍ സംസാരിക്കുന്ന പക്ഷം നീ അവരുടെ വാക്ക് കേട്ടിരുന്നു പോകും. അവര്‍ ചാരിവെച്ച മരത്തടികള്‍ പോലെയാകുന്നു. എല്ലാ ഒച്ചയും തങ്ങള്‍ക്കെതിരാണെന്ന് അവര്‍ വിചാരിക്കും. അവരാകുന്നു ശത്രു. അവരെ സൂക്ഷിച്ചു കൊള്ളുക. അല്ലാഹു അവരെ നശിപ്പിക്കട്ടെ. എങ്ങനെയാണവര്‍ വഴിതെറ്റിക്കപ്പെടുന്നത്‌?
\end{malayalam}}
\flushright{\begin{Arabic}
\quranayah[63][5]
\end{Arabic}}
\flushleft{\begin{malayalam}
നിങ്ങള്‍ വരൂ. അല്ലാഹുവിന്‍റെ ദൂതന്‍ നിങ്ങള്‍ക്ക് വേണ്ടി പാപമോചനത്തിന് പ്രാര്‍ത്ഥിച്ചുകൊള്ളും എന്ന് അവരോട് പറയപ്പെട്ടാല്‍ അവര്‍ അവരുടെ തല തിരിച്ചുകളയും. അവര്‍ അഹങ്കാരം നടിച്ചു കൊണ്ട് തിരിഞ്ഞുപോകുന്നതായി നിനക്ക് കാണുകയും ചെയ്യാം.
\end{malayalam}}
\flushright{\begin{Arabic}
\quranayah[63][6]
\end{Arabic}}
\flushleft{\begin{malayalam}
നീ അവര്‍ക്ക് വേണ്ടി പാപമോചനത്തിന് പ്രാര്‍ത്ഥിച്ചാലും പ്രാര്‍ത്ഥിച്ചിട്ടില്ലെങ്കിലും അവരെ സംബന്ധിച്ചിടത്തോളം സമമാകുന്നു. അല്ലാഹു ഒരിക്കലും അവര്‍ക്കു പൊറുത്തുകൊടുക്കുകയില്ല. തീര്‍ച്ചയായും അല്ലാഹു ദുര്‍മാര്‍ഗികളായ ജനങ്ങളെ നേര്‍വഴിയിലാക്കുകയില്ല.
\end{malayalam}}
\flushright{\begin{Arabic}
\quranayah[63][7]
\end{Arabic}}
\flushleft{\begin{malayalam}
അല്ലാഹുവിന്‍റെ ദൂതന്‍റെ അടുക്കലുള്ളവര്‍ക്ക് വേണ്ടി, അവര്‍ (അവിടെ നിന്ന്‌) പിരിഞ്ഞു പോകുന്നത് വരെ നിങ്ങള്‍ ഒന്നും ചെലവ് ചെയ്യരുത് എന്ന് പറയുന്നവരാകുന്നു അവര്‍. അല്ലാഹുവിന്‍റെതാകുന്നു ആകാശങ്ങളിലെയും ഭൂമിയിലെയും ഖജനാവുകള്‍. പക്ഷെ കപടന്‍മാര്‍ കാര്യം ഗ്രഹിക്കുന്നില്ല.
\end{malayalam}}
\flushright{\begin{Arabic}
\quranayah[63][8]
\end{Arabic}}
\flushleft{\begin{malayalam}
അവര്‍ പറയുന്നു; ഞങ്ങള്‍ മദീനയിലേക്ക് മടങ്ങിച്ചെന്നാല്‍ കൂടുതല്‍ പ്രതാപമുള്ളവര്‍ നിന്ദ്യരായുള്ളവരെ പുറത്താക്കുക തന്നെ ചെയ്യുമെന്ന്‌. അല്ലാഹുവിനും അവന്‍റെ ദൂതന്നും സത്യവിശ്വാസികള്‍ക്കുമാകുന്നു പ്രതാപം. പക്ഷെ, കപടവിശ്വാസികള്‍ (കാര്യം) മനസ്സിലാക്കുന്നില്ല.
\end{malayalam}}
\flushright{\begin{Arabic}
\quranayah[63][9]
\end{Arabic}}
\flushleft{\begin{malayalam}
സത്യവിശ്വാസികളേ, നിങ്ങളുടെ സ്വത്തുക്കളും സന്താനങ്ങളും അല്ലാഹുവെപ്പറ്റിയുള്ള സ്മരണയില്‍ നിന്ന് നിങ്ങളുടെ ശ്രദ്ധതിരിച്ചുവിടാതിരിക്കട്ടെ. ആര്‍ അങ്ങനെ ചെയ്യുന്നുവോ അവര്‍ തന്നെയാണ് നഷ്ടക്കാര്‍.
\end{malayalam}}
\flushright{\begin{Arabic}
\quranayah[63][10]
\end{Arabic}}
\flushleft{\begin{malayalam}
നിങ്ങളില്‍ ഓരോരുത്തര്‍ക്കും മരണം വരുന്നതിനു മുമ്പായി നിങ്ങള്‍ക്ക് നാം നല്‍കിയതില്‍ നിന്ന് നിങ്ങള്‍ ചെലവഴിക്കുകയും ചെയ്യുക. അന്നേരത്ത് അവന്‍ ഇപ്രകാരം പറഞ്ഞേക്കും. എന്‍റെ രക്ഷിതാവേ, അടുത്ത ഒരു അവധിവരെ നീ എനിക്ക് എന്താണ് നീട്ടിത്തരാത്തത്‌? എങ്കില്‍ ഞാന്‍ ദാനം നല്‍കുകയും, സജ്ജനങ്ങളുടെ കൂട്ടത്തിലാവുകയും ചെയ്യുന്നതാണ്‌.
\end{malayalam}}
\flushright{\begin{Arabic}
\quranayah[63][11]
\end{Arabic}}
\flushleft{\begin{malayalam}
ഒരാള്‍ക്കും അയാളുടെ അവധി വന്നെത്തിയാല്‍ അല്ലാഹു നീട്ടികൊടുക്കുകയേ ഇല്ല. അല്ലാഹു നിങ്ങള്‍ പ്രവര്‍ത്തിക്കുന്നതിനെപ്പറ്റി സൂക്ഷ്മമായി അറിയുന്നവനാകുന്നു.
\end{malayalam}}
\chapter{\textmalayalam{തഗാബൂന്‍ ( നഷ്ടം വെളിപ്പെടല്‍ )}}
\begin{Arabic}
\Huge{\centerline{\basmalah}}\end{Arabic}
\flushright{\begin{Arabic}
\quranayah[64][1]
\end{Arabic}}
\flushleft{\begin{malayalam}
ആകാശങ്ങളിലുള്ളതും ഭൂമിയിലുള്ളതും അല്ലാഹുവെ പ്രകീര്‍ത്തിക്കുന്നു. അവന്നാണ് ആധിപത്യം. അവന്നാണ് സ്തുതി. അവന്‍ ഏതുകാര്യത്തിനും കഴിവുള്ളവനാകുന്നു.
\end{malayalam}}
\flushright{\begin{Arabic}
\quranayah[64][2]
\end{Arabic}}
\flushleft{\begin{malayalam}
അവനാണ് നിങ്ങളെ സൃഷ്ടിച്ചവന്‍. എന്നിട്ട് നിങ്ങളുടെ കൂട്ടത്തില്‍ സത്യനിഷേധിയുണ്ട്‌. നിങ്ങളുടെ കൂട്ടത്തില്‍ വിശ്വാസിയുമുണ്ട്‌. അല്ലാഹു നിങ്ങള്‍ പ്രവര്‍ത്തിക്കുന്നതിനെ പറ്റി കണ്ടറിയുന്നവനാകുന്നു.
\end{malayalam}}
\flushright{\begin{Arabic}
\quranayah[64][3]
\end{Arabic}}
\flushleft{\begin{malayalam}
ആകാശങ്ങളും, ഭൂമിയും അവന്‍ മുറപ്രകാരം സൃഷ്ടിച്ചിരിക്കുന്നു. നിങ്ങള്‍ക്കവന്‍ രൂപം നല്‍കുകയും, നിങ്ങളുടെ രൂപങ്ങള്‍ അവന്‍ നന്നാക്കുകയും ചെയ്തിരിക്കുന്നു. അവങ്കലേക്കാകുന്നു തിരിച്ചുപോക്ക്‌.
\end{malayalam}}
\flushright{\begin{Arabic}
\quranayah[64][4]
\end{Arabic}}
\flushleft{\begin{malayalam}
ആകാശങ്ങളിലുള്ളതും ഭൂമിയിലുള്ളതും അവന്‍ അറിയുന്നു. നിങ്ങള്‍ രഹസ്യമാക്കുന്നതും പരസ്യമാക്കുന്നതും അവന്‍ അറിയുന്നു. അല്ലാഹു ഹൃദയങ്ങളിലുള്ളത് അറിയുന്നവനാകുന്നു.
\end{malayalam}}
\flushright{\begin{Arabic}
\quranayah[64][5]
\end{Arabic}}
\flushleft{\begin{malayalam}
മുമ്പ് അവിശ്വസിച്ചവരുടെ വൃത്താന്തം നിങ്ങള്‍ക്കു വന്നുകിട്ടിയിട്ടില്ലേ? അങ്ങനെ അവരുടെ നിലപാടിന്‍റെ ഭവിഷ്യത്ത് അവര്‍ അനുഭവിച്ചു. അവര്‍ക്കു (പരലോകത്ത്‌) വേദനയേറിയ ശിക്ഷയുമുണ്ട്‌.
\end{malayalam}}
\flushright{\begin{Arabic}
\quranayah[64][6]
\end{Arabic}}
\flushleft{\begin{malayalam}
അതെന്തുകൊണ്ടെന്നാല്‍ അവരിലേക്കുള്ള ദൂതന്‍മാര്‍ വ്യക്തമായ തെളിവുകളും കൊണ്ട് അവരുടെ അടുക്കല്‍ ചെല്ലാറുണ്ടായിരുന്നു. അപ്പോള്‍ അവര്‍ പറഞ്ഞു: ഒരു മനുഷ്യന്‍ നമുക്ക് മാര്‍ഗദര്‍ശനം നല്‍കുകയോ? അങ്ങനെ അവര്‍ അവിശ്വസിക്കുകയും പിന്തിരിഞ്ഞു കളയുകയും ചെയ്തു. അല്ലാഹു സ്വയം പര്യാപ്തനായിരിക്കുന്നു. അല്ലാഹു പരാശ്രയമുക്തനും സ്തുത്യര്‍ഹനുമാകുന്നു.
\end{malayalam}}
\flushright{\begin{Arabic}
\quranayah[64][7]
\end{Arabic}}
\flushleft{\begin{malayalam}
തങ്ങള്‍ ഉയിര്‍ത്തെഴുന്നേല്‍പിക്കപ്പെടുകയില്ലെന്ന് ആ അവിശ്വാസികള്‍ ജല്‍പിച്ചു.(നബിയേ,)പറയുക: അതെ; എന്‍റെ രക്ഷിതാവിനെ തന്നെയാണ, നിങ്ങള്‍ ഉയിര്‍ത്തെഴുന്നേല്‍പിക്കപ്പെടും. പിന്നീട് നിങ്ങള്‍ പ്രവര്‍ത്തിച്ചതിനെപ്പറ്റി നിങ്ങള്‍ക്ക് വിവരമറിയിക്കപ്പെടുകയും ചെയ്യും. അത് അല്ലാഹുവെ സംബന്ധിച്ചിടത്തോളം എളുപ്പമുള്ളതാകുന്നു.
\end{malayalam}}
\flushright{\begin{Arabic}
\quranayah[64][8]
\end{Arabic}}
\flushleft{\begin{malayalam}
അതിനാല്‍ നിങ്ങള്‍ അല്ലാഹുവിലും അവന്‍റെ ദൂതനിലും നാം അവതരിപ്പിച്ച പ്രകാശത്തിലും വിശ്വസിച്ചുകൊള്ളുക. അല്ലാഹു നിങ്ങള്‍ പ്രവര്‍ത്തിക്കുന്നതിനെപ്പറ്റി സൂക്ഷ്മമായി അറിയുന്നവനാണ്‌.
\end{malayalam}}
\flushright{\begin{Arabic}
\quranayah[64][9]
\end{Arabic}}
\flushleft{\begin{malayalam}
ആ സമ്മേളനദിനത്തിന് നിങ്ങളെ അവന്‍ ഒരുമിച്ചുകൂട്ടുന്ന ദിവസം (ശ്രദ്ധേയമാകുന്നു.) അതാണ് നഷ്ടം വെളിപ്പെടുന്ന ദിവസം. ആര്‍ അല്ലാഹുവില്‍ വിശ്വസിക്കുകയും സല്‍കര്‍മ്മം പ്രവര്‍ത്തിക്കുകയും ചെയ്യുന്നുവോ അവന്‍റെ പാപങ്ങള്‍ അല്ലാഹു മായ്ച്ചുകളയുകയും താഴ്ഭാഗത്തു കൂടി അരുവികള്‍ ഒഴുകുന്ന സ്വര്‍ഗത്തോപ്പുകളില്‍ അവനെ പ്രവേശിപ്പിക്കുകയും ചെയ്യുന്നതാണ്‌. അതില്‍ (സ്വര്‍ഗത്തില്‍) അവര്‍ നിത്യവാസികളായിരിക്കും. അതത്രെ മഹത്തായ ഭാഗ്യം.
\end{malayalam}}
\flushright{\begin{Arabic}
\quranayah[64][10]
\end{Arabic}}
\flushleft{\begin{malayalam}
അവിശ്വസിക്കുകയും നമ്മുടെ ദൃഷ്ടാന്തങ്ങള്‍ നിഷേധിച്ചു തള്ളുകയും ചെയ്തവരാകട്ടെ അവരാണ് നരകാവകാശികള്‍. അവരതില്‍ നിത്യവാസികളായിരിക്കും. (അവര്‍) ചെന്നെത്തുന്ന ആ സ്ഥലം വളരെ ചീത്ത തന്നെ.
\end{malayalam}}
\flushright{\begin{Arabic}
\quranayah[64][11]
\end{Arabic}}
\flushleft{\begin{malayalam}
അല്ലാഹുവിന്‍റെ അനുമതി പ്രകാരമല്ലാതെ യാതൊരു വിപത്തും ബാധിച്ചിട്ടില്ല. വല്ലവനും അല്ലാഹുവില്‍ വിശ്വസിക്കുന്ന പക്ഷം അവന്‍റെ ഹൃദയത്തെ അവന്‍ നേര്‍വഴിയിലാക്കുന്നതാണ്‌. അല്ലാഹു ഏതു കാര്യത്തെപ്പറ്റിയും അറിവുള്ളവനാകുന്നു.
\end{malayalam}}
\flushright{\begin{Arabic}
\quranayah[64][12]
\end{Arabic}}
\flushleft{\begin{malayalam}
അല്ലാഹുവെ നിങ്ങള്‍ അനുസരിക്കുക. റസൂലിനെയും നിങ്ങള്‍ അനുസരിക്കുക. ഇനി നിങ്ങള്‍ പിന്തിരിയുന്ന പക്ഷം നമ്മുടെ റസൂലിന്‍റെ ബാധ്യത വ്യക്തമായ പ്രബോധനം മാത്രമാകുന്നു.
\end{malayalam}}
\flushright{\begin{Arabic}
\quranayah[64][13]
\end{Arabic}}
\flushleft{\begin{malayalam}
അല്ലാഹു- അവനല്ലാതെ യാതൊരു ദൈവവുമില്ല. അല്ലാഹുവിന്‍റെ മേലായിരിക്കട്ടെ സത്യവിശ്വാസികള്‍ ഭരമേല്‍പിക്കുന്നത്‌.
\end{malayalam}}
\flushright{\begin{Arabic}
\quranayah[64][14]
\end{Arabic}}
\flushleft{\begin{malayalam}
സത്യവിശ്വാസികളേ, തീര്‍ച്ചയായും നിങ്ങളുടെ ഭാര്യമാരിലും നിങ്ങളുടെ മക്കളിലും നിങ്ങള്‍ക്ക് ശത്രുവുണ്ട്‌. അതിനാല്‍ അവരെ നിങ്ങള്‍ സൂക്ഷിച്ചു കൊള്ളുക. നിങ്ങള്‍ മാപ്പുനല്‍കുകയും, വിട്ടുവീഴ്ച കാണിക്കുകയും പൊറുത്തുകൊടുക്കുകയും ചെയ്യുന്ന പക്ഷം തീര്‍ച്ചയായും അല്ലാഹു ഏറെ പൊറുക്കുന്നവനും കരുണാനിധിയുമാകുന്നു.
\end{malayalam}}
\flushright{\begin{Arabic}
\quranayah[64][15]
\end{Arabic}}
\flushleft{\begin{malayalam}
നിങ്ങളുടെ സ്വത്തുക്കളും നിങ്ങളുടെ സന്താനങ്ങളും ഒരു പരീക്ഷണം മാത്രമാകുന്നു. അല്ലാഹുവിങ്കലാകുന്നു മഹത്തായ പ്രതിഫലമുള്ളത്‌.
\end{malayalam}}
\flushright{\begin{Arabic}
\quranayah[64][16]
\end{Arabic}}
\flushleft{\begin{malayalam}
അതിനാല്‍ നിങ്ങള്‍ക്ക് സാധ്യമായ വിധം അല്ലാഹുവെ നിങ്ങള്‍ സൂക്ഷിക്കുക. നിങ്ങള്‍ കേള്‍ക്കുകയും അനുസരിക്കുകയും നിങ്ങള്‍ക്കു തന്നെ ഗുണകരമായ നിലയില്‍ ചെലവഴിക്കുകയും ചെയ്യുക. ആര്‍ മനസ്സിന്‍റെ പിശുക്കില്‍ നിന്ന് കാത്തുരക്ഷിക്കപ്പെടുന്നുവോ അവര്‍ തന്നെയാകുന്നു വിജയം പ്രാപിച്ചവര്‍.
\end{malayalam}}
\flushright{\begin{Arabic}
\quranayah[64][17]
\end{Arabic}}
\flushleft{\begin{malayalam}
നിങ്ങള്‍ അല്ലാഹുവിന് ഉത്തമമായ കടം കൊടുക്കുന്ന പക്ഷം അവനത് നിങ്ങള്‍ക്ക് ഇരട്ടിയാക്കിത്തരികയും നിങ്ങള്‍ക്ക് പൊറുത്തുതരികയും ചെയ്യുന്നതാണ്‌. അല്ലാഹു ഏറ്റവും അധികം നന്ദിയുള്ളവനും സഹനശീലനുമാകുന്നു.
\end{malayalam}}
\flushright{\begin{Arabic}
\quranayah[64][18]
\end{Arabic}}
\flushleft{\begin{malayalam}
അദൃശ്യവും ദൃശ്യവും അറിയുന്നവനും പ്രതാപിയും യുക്തിമാനുമാകുന്നു അവന്‍.
\end{malayalam}}
\chapter{\textmalayalam{ത്വലാഖ് ( വിവാഹ മോചനം )}}
\begin{Arabic}
\Huge{\centerline{\basmalah}}\end{Arabic}
\flushright{\begin{Arabic}
\quranayah[65][1]
\end{Arabic}}
\flushleft{\begin{malayalam}
നബിയേ, നിങ്ങള്‍ (വിശ്വാസികള്‍) സ്ത്രീകളെ വിവാഹമോചനം ചെയ്യുകയാണെങ്കില്‍ അവരെ നിങ്ങള്‍ അവരുടെ ഇദ്ദഃ കാലത്തിന് (കണക്കാക്കി) വിവാഹമോചനം ചെയ്യുകയും ഇദ്ദഃ കാലം നിങ്ങള്‍ എണ്ണികണക്കാക്കുകയും ചെയ്യുക. നിങ്ങളുടെ രക്ഷിതാവായ അല്ലാഹുവെ നിങ്ങള്‍ സൂക്ഷിക്കുകയും ചെയ്യുക. അവരുടെ വീടുകളില്‍ നിന്ന് അവരെ നിങ്ങള്‍ പുറത്താക്കരുത്‌. അവര്‍ പുറത്തു പോകുകയും ചെയ്യരുത്‌. പ്രത്യക്ഷമായ വല്ല നീചവൃത്തിയും അവര്‍ ചെയ്യുകയാണെങ്കിലല്ലാതെ. അവ അല്ലാഹുവിന്‍റെ നിയമപരിധികളാകാകുന്നു. അല്ലാഹുവിന്‍റെ നിയമപരിധികളെ വല്ലവനും അതിലംഘിക്കുന്ന പക്ഷം, അവന്‍ അവനോട് തന്നെ അന്യായം ചെയ്തിരിക്കുന്നു. അതിന് ശേഷം അല്ലാഹു പുതുതായി വല്ലകാര്യവും കൊണ്ടു വന്നേക്കുമോ എന്ന് നിനക്ക് അറിയില്ല.
\end{malayalam}}
\flushright{\begin{Arabic}
\quranayah[65][2]
\end{Arabic}}
\flushleft{\begin{malayalam}
അങ്ങനെ അവര്‍ (വിവാഹമുക്തകള്‍) അവരുടെ അവധിയില്‍ എത്തുമ്പോള്‍ നിങ്ങള്‍ ന്യായമായ നിലയില്‍ അവരെ പിടിച്ച് നിര്‍ത്തുകയോ, ന്യായമായ നിലയില്‍ അവരുമായി വേര്‍പിരിയുകയോ ചെയ്യുക. നിങ്ങളില്‍ നിന്നുള്ള രണ്ടു നീതിമാന്‍മാരെ നിങ്ങള്‍ സാക്ഷി നിര്‍ത്തുകയും അല്ലാഹുവിന് വേണ്ടി സാക്ഷ്യം നേരാംവണ്ണം നിലനിര്‍ത്തുകയും ചെയ്യുക. അല്ലാഹുവിലും അന്ത്യദിനത്തിലും വിശ്വസിച്ചു കൊണ്ടിരിക്കുന്നവര്‍ക്ക് ഉപദേശം നല്‍കപ്പെടുന്നതത്രെ അത്‌. അല്ലാഹുവെ വല്ലവനും സൂക്ഷിക്കുന്ന പക്ഷം അല്ലാഹു അവന്നൊരു പോംവഴി ഉണ്ടാക്കികൊടുക്കുകയും,
\end{malayalam}}
\flushright{\begin{Arabic}
\quranayah[65][3]
\end{Arabic}}
\flushleft{\begin{malayalam}
അവന്‍ കണക്കാക്കാത്ത വിധത്തില്‍ അവന്ന് ഉപജീവനം നല്‍കുകയും ചെയ്യുന്നതാണ്‌. വല്ലവനും അല്ലാഹുവില്‍ ഭരമേല്‍പിക്കുന്ന പക്ഷം അവന്ന് അല്ലാഹു തന്നെ മതിയാകുന്നതാണ്‌. തീര്‍ച്ചയായും അല്ലാഹു തന്‍റെ കാര്യം പ്രാപിക്കുന്നവനാകുന്നു. ഓരോ കാര്യത്തിനും അല്ലാഹു ഒരു ക്രമം ഏര്‍പെടുത്തിയിട്ടുണ്ട്‌.
\end{malayalam}}
\flushright{\begin{Arabic}
\quranayah[65][4]
\end{Arabic}}
\flushleft{\begin{malayalam}
നിങ്ങളുടെ സ്ത്രീകളില്‍ നിന്നും ആര്‍ത്തവത്തെ സംബന്ധിച്ച് നിരാശപ്പെട്ടിട്ടുള്ളവരെ സംബന്ധിച്ചിടത്തോളം നിങ്ങള്‍ അവരുടെ ഇദ്ദഃ യുടെ കാര്യത്തില്‍ സംശയത്തിലാണെങ്കില്‍ അത് മൂന്ന് മാസമാകുന്നു. ആര്‍ത്തവമുണ്ടായിട്ടില്ലാത്തവരുടേതും അങ്ങനെ തന്നെ. ഗര്‍ഭവതികളായ സ്ത്രീകളാകട്ടെ അവരുടെ അവധി അവര്‍ തങ്ങളുടെ ഗര്‍ഭം പ്രസവിക്കലാകുന്നു. വല്ലവനും അല്ലാഹുവെ സൂക്ഷിക്കുന്ന പക്ഷം അവന്ന് അവന്‍റെ കാര്യത്തില്‍ അല്ലാഹു എളുപ്പമുണ്ടാക്കി കൊടുക്കുന്നതാണ്‌.
\end{malayalam}}
\flushright{\begin{Arabic}
\quranayah[65][5]
\end{Arabic}}
\flushleft{\begin{malayalam}
അത് അല്ലാഹുവിന്‍റെ കല്‍പനയാകുന്നു. അവനത് നിങ്ങള്‍ക്ക് അവതരിപ്പിച്ചു തന്നിരിക്കുന്നു. വല്ലവനും അല്ലാഹുവെ സൂക്ഷിക്കുന്ന പക്ഷം അവന്‍റെ തിന്‍മകളെ അവന്‍ മായ്ച്ചുകളയുകയും അവന്നുള്ള പ്രതിഫലം അവന്‍ വലുതാക്കി കൊടുക്കുകയും ചെയ്യുന്നതാണ്‌.
\end{malayalam}}
\flushright{\begin{Arabic}
\quranayah[65][6]
\end{Arabic}}
\flushleft{\begin{malayalam}
നിങ്ങളുടെ കഴിവില്‍ പെട്ട, നിങ്ങള്‍ താമസിക്കുന്ന സ്ഥലത്ത് നിങ്ങള്‍ അവരെ താമസിപ്പിക്കണം. അവര്‍ക്കു ഞെരുക്കമുണ്ടാക്കാന്‍ വേണ്ടി നിങ്ങള്‍ അവരെ ദ്രോഹിക്കരുത്‌. അവര്‍ ഗര്‍ഭിണികളാണെങ്കില്‍ അവര്‍ പ്രസവിക്കുന്നത് വരെ നിങ്ങള്‍ അവര്‍ക്കു ചെലവുകൊടുക്കുകയും ചെയ്യുക. ഇനി അവര്‍ നിങ്ങള്‍ക്കു വേണ്ടി (കുഞ്ഞിന്‌) മുലകൊടുക്കുന്ന പക്ഷം അവര്‍ക്കു നിങ്ങള്‍ അവരുടെ പ്രതിഫലം കൊടുക്കുക. നിങ്ങള്‍ തമ്മില്‍ മര്യാദപ്രകാരം കൂടിയാലോചിക്കുകയും ചെയ്യുക. ഇനി നിങ്ങള്‍ ഇരു വിഭാഗത്തിനും ഞെരുക്കമാവുകയാണെങ്കില്‍ അയാള്‍ക്കു വേണ്ടി മറ്റൊരു സ്ത്രീ മുലകൊടുത്തു കൊള്ളട്ടെ.
\end{malayalam}}
\flushright{\begin{Arabic}
\quranayah[65][7]
\end{Arabic}}
\flushleft{\begin{malayalam}
കഴിവുള്ളവന്‍ തന്‍റെ കഴിവില്‍ നിന്ന് ചെലവിനു കൊടുക്കട്ടെ. വല്ലവന്നും തന്‍റെ ഉപജീവനം ഇടുങ്ങിയതായാല്‍ അല്ലാഹു അവന്നു കൊടുത്തതില്‍ നിന്ന് അവന്‍ ചെലവിന് കൊടുക്കട്ടെ. ഒരാളോടും അല്ലാഹു അയാള്‍ക്ക് കൊടുത്തതല്ലാതെ (നല്‍കാന്‍) നിര്‍ബന്ധിക്കുകയില്ല. അല്ലാഹു ഞെരുക്കത്തിനു ശേഷം സൌകര്യം ഏര്‍പെടുത്തികൊടുക്കുന്നതാണ്‌.
\end{malayalam}}
\flushright{\begin{Arabic}
\quranayah[65][8]
\end{Arabic}}
\flushleft{\begin{malayalam}
എത്രയെത്ര രാജ്യക്കാര്‍ അവരുടെ രക്ഷിതാവിന്‍റെയും അവന്‍റെ ദൂതന്‍മാരുടെയും കല്‍പന വിട്ട് ധിക്കാരം പ്രവര്‍ത്തിച്ചിട്ടുണ്ട്‌. അതിനാല്‍ നാം അവരോട് കര്‍ക്കശമായ നിലയില്‍ കണക്കു ചോദിക്കുകയും അവരെ നാം ഹീനമായ വിധത്തില്‍ ശിക്ഷിക്കുകയും ചെയ്തു.
\end{malayalam}}
\flushright{\begin{Arabic}
\quranayah[65][9]
\end{Arabic}}
\flushleft{\begin{malayalam}
അങ്ങനെ അവര്‍ അവരുടെ നിലപാടിന്‍റെ ദുഷ്ഫലം ആസ്വദിച്ചു. അവരുടെ നിലപാടിന്‍റെ പര്യവസാനം നഷ്ടം തന്നെയായിരുന്നു.
\end{malayalam}}
\flushright{\begin{Arabic}
\quranayah[65][10]
\end{Arabic}}
\flushleft{\begin{malayalam}
അല്ലാഹു അവര്‍ക്കു കഠിനമായ ശിക്ഷ ഒരുക്കിവെച്ചിരിക്കുന്നു. അതിനാല്‍ സത്യവിശ്വാസികളായ ബുദ്ധിമാന്‍മാരേ, നിങ്ങള്‍ അല്ലാഹുവെ സൂക്ഷിക്കുക. തീര്‍ച്ചയായും അല്ലാഹു നിങ്ങള്‍ക്ക് ഒരു ഉല്‍ബോധകനെ
\end{malayalam}}
\flushright{\begin{Arabic}
\quranayah[65][11]
\end{Arabic}}
\flushleft{\begin{malayalam}
അഥവാ അല്ലാഹുവിന്‍റെ വ്യക്തമായ ദൃഷ്ടാന്തങ്ങള്‍ നിങ്ങള്‍ക്ക് ഓതികേള്‍പിച്ചു തരുന്ന ഒരു ദൂതനെ നിങ്ങളുടെ അടുത്തേക്കിറക്കിത്തന്നിരിക്കുന്നു; വിശ്വസിക്കുകയും സല്‍കര്‍മ്മങ്ങള്‍ പ്രവര്‍ത്തിക്കുകയും ചെയ്തവരെ അന്ധകാരങ്ങളില്‍ നിന്ന് പ്രകാശത്തിലേക്ക് ആനയിക്കുവാന്‍ വേണ്ടി. വല്ലവനും അല്ലാഹുവില്‍ വിശ്വസിക്കുകയും സല്‍കര്‍മ്മം പ്രവര്‍ത്തിക്കുകയും ചെയ്യുന്ന പക്ഷം താഴ്ഭാഗത്തു കൂടി അരുവികള്‍ ഒഴുകുന്ന സ്വര്‍ഗത്തോപ്പുകളില്‍ അവനെ പ്രവേശിപ്പിക്കുന്നതാണ്‌. അവര്‍ അതില്‍ നിത്യവാസികളായിരിക്കും. അങ്ങനെയുള്ളവന്ന് അല്ലാഹു ഉപജീവനം മെച്ചപ്പെടുത്തിയിരിക്കുന്നു.
\end{malayalam}}
\flushright{\begin{Arabic}
\quranayah[65][12]
\end{Arabic}}
\flushleft{\begin{malayalam}
അല്ലാഹുവാകുന്നു ഏഴ് ആകാശങ്ങളും ഭൂമിയില്‍ നിന്ന് അവയ്ക്ക് തുല്യമായതും സൃഷ്ടിച്ചവന്‍. അവയ്ക്കിടയില്‍ (അവന്‍റെ) കല്‍പന ഇറങ്ങുന്നു. അല്ലാഹു ഏത് കാര്യത്തിനും കഴിവുള്ളവനാകുന്നു എന്നും ഏതു വസ്തുവെയും ചൂഴ്ന്ന് അറിയുന്നവനായിരിക്കുന്നു എന്നും നിങ്ങള്‍ മനസ്സിലാക്കുവാന്‍ വേണ്ടി.
\end{malayalam}}
\chapter{\textmalayalam{തഹ് രീം ( നിഷിദ്ധമാക്കല്‍ )}}
\begin{Arabic}
\Huge{\centerline{\basmalah}}\end{Arabic}
\flushright{\begin{Arabic}
\quranayah[66][1]
\end{Arabic}}
\flushleft{\begin{malayalam}
ഓ; നബീ, നീയെന്തിനാണ് നിന്‍റെ ഭാര്യമാരുടെ പ്രീതിതേടിക്കൊണ്ട്‌, അല്ലാഹു അനുവദിച്ചു തന്നത് നിഷിദ്ധമാക്കുന്നത്‌? അല്ലാഹു ഏറെ പൊറുക്കുന്നവനും കരുണാനിധിയുമാകുന്നു.
\end{malayalam}}
\flushright{\begin{Arabic}
\quranayah[66][2]
\end{Arabic}}
\flushleft{\begin{malayalam}
നിങ്ങളുടെ ശപഥങ്ങള്‍ക്കുള്ള പരിഹാരം അല്ലാഹു നിങ്ങള്‍ക്ക് നിയമമാക്കിത്തന്നിരിക്കുന്നു. അല്ലാഹു നിങ്ങളുടെ യജമാനനാകുന്നു. അവനത്രെ സര്‍വ്വജ്ഞനും യുക്തിമാനും.
\end{malayalam}}
\flushright{\begin{Arabic}
\quranayah[66][3]
\end{Arabic}}
\flushleft{\begin{malayalam}
നബി അദ്ദേഹത്തിന്‍റെ ഭാര്യമാരില്‍ ഒരാളോട് ഒരു വര്‍ത്തമാനം രഹസ്യമായി പറഞ്ഞ സന്ദര്‍ഭം (ശ്രദ്ധേയമാകുന്നു.) എന്നിട്ട് ആ ഭാര്യ അത് (മറ്റൊരാളെ) അറിയിക്കുകയും, നബിക്ക് അല്ലാഹു അത് വെളിപ്പെടുത്തി കൊടുക്കുകയും ചെയ്തപ്പോള്‍ അതിന്‍റെ ചില ഭാഗം അദ്ദേഹം (ആ ഭാര്യയ്ക്ക്‌) അറിയിച്ചുകൊടുക്കുകയും ചില ഭാഗം വിട്ടുകളയുകയും ചെയ്തു. അങ്ങനെ അവളോട് (ആ ഭാര്യയോട്‌) അദ്ദേഹം അതിനെ പറ്റി വിവരം അറിയിച്ചപ്പോള്‍ അവള്‍ പറഞ്ഞു: താങ്കള്‍ക്ക് ആരാണ് ഈ വിവരം അറിയിച്ചു തന്നത് ? നബി (സ) പറഞ്ഞു: സര്‍വ്വജ്ഞനും സൂക്ഷ്മജ്ഞാനിയുമായിട്ടുള്ളവനാണ് എനിക്ക് വിവരമറിയിച്ചു തന്നത്‌.
\end{malayalam}}
\flushright{\begin{Arabic}
\quranayah[66][4]
\end{Arabic}}
\flushleft{\begin{malayalam}
നിങ്ങള്‍ രണ്ടു പേരും അല്ലാഹുവിങ്കലേക്ക് പശ്ചാത്തപിച്ചു മടങ്ങുന്നുവെങ്കില്‍ (അങ്ങനെ ചെയ്യുക.) കാരണം നിങ്ങളുടെ രണ്ടുപേരുടെയും ഹൃദയങ്ങള്‍ (തിന്‍മയിലേക്ക്‌) ചാഞ്ഞുപോയിരിക്കുന്നു. ഇനി നിങ്ങള്‍ ഇരുവരും അദ്ദേഹത്തിനെതിരില്‍ (റസൂലിനെതിരില്‍) പരസ്പരം സഹകരിക്കുന്ന പക്ഷം തീര്‍ച്ചയായും അല്ലാഹുവാകുന്നു അദ്ദേഹത്തിന്‍റെ യജമാനന്‍. ജിബ്‌രീലും സദ്‌വൃത്തരായ സത്യവിശ്വാസികളും അതിനു പുറമെ മലക്കുകളും അദ്ദേഹത്തിന് സഹായികളായിരിക്കുന്നതാണ്‌.
\end{malayalam}}
\flushright{\begin{Arabic}
\quranayah[66][5]
\end{Arabic}}
\flushleft{\begin{malayalam}
(പ്രവാചകപത്നിമാരേ,) നിങ്ങളെ അദ്ദേഹം വിവാഹമോചനം ചെയ്യുന്ന പക്ഷം, നിങ്ങളെക്കാള്‍ നല്ലവരായ ഭാര്യമാരെ അദ്ദേഹത്തിന് അദ്ദേഹത്തിന്‍റെ രക്ഷിതാവ് പകരം നല്‍കിയേക്കാം. മുസ്ലിംകളും സത്യവിശ്വാസിനികളും ഭയഭക്തിയുള്ളവരും പശ്ചാത്താപമുള്ളവരും ആരാധനാനിരതരും വ്രതമനുഷ്ഠിക്കുന്നവരും വിധവകളും കന്യകകളുമായിട്ടുള്ള സ്ത്രീകളെ.
\end{malayalam}}
\flushright{\begin{Arabic}
\quranayah[66][6]
\end{Arabic}}
\flushleft{\begin{malayalam}
സത്യവിശ്വാസികളേ, സ്വദേഹങ്ങളെയും നിങ്ങളുടെ ബന്ധുക്കളെയും മനുഷ്യരും കല്ലുകളും ഇന്ധനമായിട്ടുള്ള നരകാഗ്നിയില്‍ നിന്ന് നിങ്ങള്‍ കാത്തുരക്ഷിക്കുക. അതിന്‍റെ മേല്‍നോട്ടത്തിന് പരുഷസ്വഭാവമുള്ളവരും അതിശക്തന്‍മാരുമായ മലക്കുകളുണ്ടായിരിക്കും. അല്ലാഹു അവരോട് കല്‍പിച്ചകാര്യത്തില്‍ അവനോടവര്‍ അനുസരണക്കേട് കാണിക്കുകയില്ല. അവരോട് കല്‍പിക്കപ്പെടുന്നത് എന്തും അവര്‍ പ്രവര്‍ത്തിക്കുകയും ചെയ്യും.
\end{malayalam}}
\flushright{\begin{Arabic}
\quranayah[66][7]
\end{Arabic}}
\flushleft{\begin{malayalam}
സത്യനിഷേധികളേ, നിങ്ങള്‍ ഇന്ന് ഒഴികഴിവ് പറയേണ്ട. നിങ്ങള്‍ ചെയ്തുകൊണ്ടിരുന്നതിനു മാത്രമാണ് നിങ്ങള്‍ക്ക് പ്രതിഫലം നല്‍കപ്പെടുന്നത്‌.
\end{malayalam}}
\flushright{\begin{Arabic}
\quranayah[66][8]
\end{Arabic}}
\flushleft{\begin{malayalam}
സത്യവിശ്വാസികളേ, നിങ്ങള്‍ അല്ലാഹുവിങ്കലേക്ക് നിഷ്കളങ്കമായ പശ്ചാത്താപം കൈക്കൊണ്ട് മടങ്ങുക. നിങ്ങളുടെ രക്ഷിതാവ് നിങ്ങളുടെ പാപങ്ങള്‍ മായ്ച്ചുകളയുകയും താഴ്ഭാഗത്തു കൂടി അരുവികള്‍ ഒഴുകുന്ന സ്വര്‍ഗത്തോപ്പുകളില്‍ നിങ്ങളെ പ്രവേശിപ്പിക്കുകയും ചെയ്തേക്കാം. അല്ലാഹു പ്രവാചകനെയും അദ്ദേഹത്തോടൊപ്പം വിശ്വസിച്ചവരെയും അപമാനിക്കാത്ത ദിവസത്തില്‍. അവരുടെ പ്രകാശം അവരുടെ മുന്നിലൂടെയും വലതുവശങ്ങളിലൂടെയും സഞ്ചരിക്കും. അവര്‍ പറയും: ഞങ്ങളുടെ രക്ഷിതാവേ, ഞങ്ങളുടെ പ്രകാശം ഞങ്ങള്‍ക്കു നീ പൂര്‍ത്തീകരിച്ച് തരികയും, ഞങ്ങള്‍ക്കു നീ പൊറുത്തുതരികയും ചെയ്യേണമേ. തീര്‍ച്ചയായും നീ ഏതു കാര്യത്തിനും കഴിവുള്ളവനാകുന്നു.
\end{malayalam}}
\flushright{\begin{Arabic}
\quranayah[66][9]
\end{Arabic}}
\flushleft{\begin{malayalam}
ഓ; നബീ, സത്യനിഷേധികളോടും കപടവിശ്വാസികളോടും നീ സമരം ചെയ്യുകയും അവരോട് പരുഷമായി പെരുമാറുകയും ചെയ്യുക. അവരുടെ സങ്കേതം നരകമാകുന്നു. എത്തിച്ചേരാനുള്ള ആ സ്ഥലം എത്രയോ ചീത്ത!
\end{malayalam}}
\flushright{\begin{Arabic}
\quranayah[66][10]
\end{Arabic}}
\flushleft{\begin{malayalam}
സത്യനിഷേധികള്‍ക്ക് ഉദാഹരണമായി നൂഹിന്‍റെ ഭാര്യയെയും, ലൂത്വിന്‍റെ ഭാര്യയെയും അല്ലാഹു ഇതാ എടുത്തുകാണിച്ചിരിക്കുന്നു. അവര്‍ രണ്ടുപേരും നമ്മുടെ ദാസന്‍മാരില്‍ പെട്ട സദ്‌വൃത്തരായ രണ്ട് ദാസന്‍മാരുടെ കീഴിലായിരുന്നു. എന്നിട്ട് അവരെ രണ്ടുപേരെയും ഇവര്‍ വഞ്ചിച്ചു കളഞ്ഞു. അപ്പോള്‍ അല്ലാഹുവിന്‍റെ ശിക്ഷയില്‍ നിന്ന് യാതൊന്നും അവര്‍ രണ്ടുപേരും ഇവര്‍ക്ക് ഒഴിവാക്കികൊടുത്തില്ല. നിങ്ങള്‍ രണ്ടുപേരും നരകത്തില്‍ കടക്കുന്നവരോടൊപ്പം കടന്നുകൊള്ളുക എന്ന് പറയപ്പെടുകയും ചെയ്തു.
\end{malayalam}}
\flushright{\begin{Arabic}
\quranayah[66][11]
\end{Arabic}}
\flushleft{\begin{malayalam}
സത്യവിശ്വാസികള്‍ക്ക് ഒരു ഉപമയായി അല്ലാഹു ഫിര്‍ഔന്‍റെ ഭാര്യയെ എടുത്തുകാണിച്ചിരിക്കുന്നു. അവള്‍ പറഞ്ഞ സന്ദര്‍ഭം: എന്‍റെ രക്ഷിതാവേ, എനിക്ക് നീ നിന്‍റെ അടുക്കല്‍ സ്വര്‍ഗത്തില്‍ ഒരു ഭവനം ഉണ്ടാക്കിത്തരികയും, ഫിര്‍ഔനില്‍ നിന്നും അവന്‍റെ പ്രവര്‍ത്തനത്തില്‍ നിന്നും എന്നെ നീ രക്ഷിക്കേണമേ. അക്രമികളായ ജനങ്ങളില്‍ നിന്നും എന്നെ നീ രക്ഷിക്കുകയും ചെയ്യേണമേ.
\end{malayalam}}
\flushright{\begin{Arabic}
\quranayah[66][12]
\end{Arabic}}
\flushleft{\begin{malayalam}
തന്‍റെ ഗുഹ്യസ്ഥാനം കാത്തുസൂക്ഷിച്ച ഇംറാന്‍റെ മകളായ മര്‍യമിനെയും (ഉപമയായി എടുത്ത് കാണിച്ചിരിക്കുന്നു.) അപ്പോള്‍ നമ്മുടെ ആത്മചൈതന്യത്തില്‍ നിന്നു നാം അതില്‍ ഊതുകയുണ്ടായി. തന്‍റെ രക്ഷിതാവിന്‍റെ വചനങ്ങളിലും ഗ്രന്ഥങ്ങളിലും അവള്‍ വിശ്വസിക്കുകയും അവള്‍ ഭയഭക്തിയുള്ളവരുടെ കൂട്ടത്തിലാവുകയും ചെയ്തു.
\end{malayalam}}
\chapter{\textmalayalam{മുല്‍ക്ക് ( അധിപത്യം )}}
\begin{Arabic}
\Huge{\centerline{\basmalah}}\end{Arabic}
\flushright{\begin{Arabic}
\quranayah[67][1]
\end{Arabic}}
\flushleft{\begin{malayalam}
ആധിപത്യം ഏതൊരുവന്‍റെ കയ്യിലാണോ അവന്‍ അനുഗ്രഹപൂര്‍ണ്ണനായിരിക്കുന്നു. അവന്‍ ഏതു കാര്യത്തിനും കഴിവുള്ളവനാകുന്നു.
\end{malayalam}}
\flushright{\begin{Arabic}
\quranayah[67][2]
\end{Arabic}}
\flushleft{\begin{malayalam}
നിങ്ങളില്‍ ആരാണ് കൂടുതല്‍ നന്നായി പ്രവര്‍ത്തിക്കുന്നവന്‍ എന്ന് പരീക്ഷിക്കുവാന്‍ വേണ്ടി മരണവും ജീവിതവും സൃഷ്ടിച്ചവനാകുന്നു അവന്‍. അവന്‍ പ്രതാപിയും ഏറെ പൊറുക്കുന്നവനുമാകുന്നു.
\end{malayalam}}
\flushright{\begin{Arabic}
\quranayah[67][3]
\end{Arabic}}
\flushleft{\begin{malayalam}
ഏഴു ആകാശങ്ങളെ അടുക്കുകളായി സൃഷ്ടിച്ചവനാകുന്നു അവന്‍. പരമകാരുണികന്‍റെ സൃഷ്ടിപ്പില്‍ യാതൊരു ഏറ്റക്കുറവും നീ കാണുകയില്ല. എന്നാല്‍ നീ ദൃഷ്ടി ഒന്നുകൂടി തിരിച്ചു കൊണ്ട് വരൂ. വല്ല വിടവും നീ കാണുന്നുണ്ടോ?
\end{malayalam}}
\flushright{\begin{Arabic}
\quranayah[67][4]
\end{Arabic}}
\flushleft{\begin{malayalam}
പിന്നീട് രണ്ടു തവണ നീ കണ്ണിനെ തിരിച്ച് കൊണ്ട് വരൂ. നിന്‍റെ അടുത്തേക്ക് ആ കണ്ണ് പരാജയപ്പെട്ട നിലയിലും പരവശമായികൊണ്ടും മടങ്ങി വരും.
\end{malayalam}}
\flushright{\begin{Arabic}
\quranayah[67][5]
\end{Arabic}}
\flushleft{\begin{malayalam}
ഏറ്റവും അടുത്ത ആകാശത്തെ നാം ചില വിളക്കുകള്‍ കൊണ്ട് അലങ്കരിച്ചിരിക്കുന്നു. അവയെ നാം പിശാചുകളെ എറിഞ്ഞോടിക്കാനുള്ളവയുമാക്കിയിരിക്കുന്നു. അവര്‍ക്കു നാം ജ്വലിക്കുന്ന നരകശിക്ഷ ഒരുക്കിവെക്കുകയും ചെയ്തിരിക്കുന്നു.
\end{malayalam}}
\flushright{\begin{Arabic}
\quranayah[67][6]
\end{Arabic}}
\flushleft{\begin{malayalam}
തങ്ങളുടെ രക്ഷിതാവില്‍ അവിശ്വസിച്ചവര്‍ക്കാണ് നരക ശിക്ഷയുള്ളത്‌. തിരിച്ചെത്തുന്ന ആ സ്ഥലം വളരെ ചീത്ത തന്നെ.
\end{malayalam}}
\flushright{\begin{Arabic}
\quranayah[67][7]
\end{Arabic}}
\flushleft{\begin{malayalam}
അവര്‍ അതില്‍ (നരകത്തില്‍) എറിയപ്പെട്ടാല്‍ അതിന്നവര്‍ ഒരു ഗര്‍ജ്ജനം കേള്‍ക്കുന്നതാണ്‌. അത് തിളച്ചു മറിഞ്ഞു കൊണ്ടിരിക്കുകയും ചെയ്യും.
\end{malayalam}}
\flushright{\begin{Arabic}
\quranayah[67][8]
\end{Arabic}}
\flushleft{\begin{malayalam}
കോപം നിമിത്തം അത് പൊട്ടിപ്പിളര്‍ന്ന് പോകുമാറാകും. അതില്‍ (നരകത്തില്‍) ഓരോ സംഘവും എറിയപ്പെടുമ്പോഴൊക്കെ അതിന്‍റെ കാവല്‍ക്കാര്‍ അവരോട് ചോദിക്കും. നിങ്ങളുടെ അടുത്ത് മുന്നറിയിപ്പുകാരന്‍ വന്നിരുന്നില്ലേ?
\end{malayalam}}
\flushright{\begin{Arabic}
\quranayah[67][9]
\end{Arabic}}
\flushleft{\begin{malayalam}
അവര്‍ പറയും: അതെ ഞങ്ങള്‍ക്ക് മുന്നറിയിപ്പുകാരന്‍ വന്നിരുന്നു. അപ്പോള്‍ ഞങ്ങള്‍ നിഷേധിച്ചു തള്ളുകയും അല്ലാഹു യാതൊന്നും ഇറക്കിയിട്ടില്ല. നിങ്ങള്‍ വലിയ വഴികേടില്‍ തന്നെയാകുന്നു എന്ന് ഞങ്ങള്‍ പറയുകയുമാണ് ചെയ്തത്‌.
\end{malayalam}}
\flushright{\begin{Arabic}
\quranayah[67][10]
\end{Arabic}}
\flushleft{\begin{malayalam}
ഞങ്ങള്‍ കേള്‍ക്കുകയോ ചിന്തിക്കുകയോ ചെയ്തിരുന്നെങ്കില്‍ ഞങ്ങള്‍ ജ്വലിക്കുന്ന നരകാഗ്നിയുടെ അവകാശികളുടെ കൂട്ടത്തിലാകുമായിരുന്നില്ല എന്നും അവര്‍ പറയും.
\end{malayalam}}
\flushright{\begin{Arabic}
\quranayah[67][11]
\end{Arabic}}
\flushleft{\begin{malayalam}
അങ്ങനെ അവര്‍ തങ്ങളുടെ കുറ്റം ഏറ്റുപറയും. അപ്പോള്‍ നരകാഗ്നിയുടെ ആള്‍ക്കാര്‍ക്കു ശാപം.
\end{malayalam}}
\flushright{\begin{Arabic}
\quranayah[67][12]
\end{Arabic}}
\flushleft{\begin{malayalam}
തീര്‍ച്ചയായും തങ്ങളുടെ രക്ഷിതാവിനെ അദൃശ്യനിലയില്‍ ഭയപ്പെടുന്നവരാരോ അവര്‍ക്ക് പാപമോചനവും വലിയ പ്രതിഫലവുമുണ്ട്‌.
\end{malayalam}}
\flushright{\begin{Arabic}
\quranayah[67][13]
\end{Arabic}}
\flushleft{\begin{malayalam}
നിങ്ങളുടെ വാക്ക് നിങ്ങള്‍ രഹസ്യമാക്കുക. അല്ലെങ്കില്‍ പരസ്യമാക്കിക്കൊള്ളുക. തീര്‍ച്ചയായും അവന്‍ (അല്ലാഹു) ഹൃദയങ്ങളിലുള്ളത് അറിയുന്നവനാകുന്നു.
\end{malayalam}}
\flushright{\begin{Arabic}
\quranayah[67][14]
\end{Arabic}}
\flushleft{\begin{malayalam}
സൃഷ്ടിച്ചുണ്ടാക്കിയവന്‍ (എല്ലാം) അറിയുകയില്ലേ? അവന്‍ നിഗൂഢരഹസ്യങ്ങള്‍ അറിയുന്നവനും സൂക്ഷ്മജ്ഞാനിയുമാകുന്നു.
\end{malayalam}}
\flushright{\begin{Arabic}
\quranayah[67][15]
\end{Arabic}}
\flushleft{\begin{malayalam}
അവനാകുന്നു നിങ്ങള്‍ക്ക് വേണ്ടി ഭൂമിയെ വിധേയമാക്കി തന്നവന്‍. അതിനാല്‍ അതിന്‍റെ ചുമലുകളിലൂടെ നിങ്ങള്‍ നടക്കുകയും അവന്‍റെ ഉപജീവനത്തില്‍ നിന്ന് ഭക്ഷിക്കുകയും ചെയ്തു കൊള്ളുക. അവങ്കലേക്ക് തന്നെയാണ് ഉയിര്‍ത്തെഴുന്നേല്‍പ്‌.
\end{malayalam}}
\flushright{\begin{Arabic}
\quranayah[67][16]
\end{Arabic}}
\flushleft{\begin{malayalam}
ആകാശത്തുള്ളവന്‍ നിങ്ങളെ ഭൂമിയില്‍ ആഴ്ത്തിക്കളയുന്നതിനെപ്പറ്റി നിങ്ങള്‍ നിര്‍ഭയരായിരിക്കുകയാണോ? അപ്പോള്‍ അത് (ഭൂമി) ഇളകിമറിഞ്ഞു കൊണ്ടിരിക്കും.
\end{malayalam}}
\flushright{\begin{Arabic}
\quranayah[67][17]
\end{Arabic}}
\flushleft{\begin{malayalam}
അതല്ല, ആകാശത്തുള്ളവന്‍ നിങ്ങളുടെ നേരെ ഒരു ചരല്‍ വര്‍ഷം അയക്കുന്നതിനെപ്പറ്റി നിങ്ങള്‍ നിര്‍ഭയരായിരിക്കുകയാണോ? എന്‍റെ താക്കീത് എങ്ങനെയുണ്ടെന്ന് നിങ്ങള്‍ വഴിയെ അറിഞ്ഞു കൊള്ളും.
\end{malayalam}}
\flushright{\begin{Arabic}
\quranayah[67][18]
\end{Arabic}}
\flushleft{\begin{malayalam}
തീര്‍ച്ചയായും അവര്‍ക്ക് മുമ്പുള്ളവരും നിഷേധിച്ചു തള്ളിയിട്ടുണ്ട്‌. അപ്പോള്‍ എന്‍റെ പ്രതിഷേധം എങ്ങനെയായിരുന്നു.
\end{malayalam}}
\flushright{\begin{Arabic}
\quranayah[67][19]
\end{Arabic}}
\flushleft{\begin{malayalam}
അവര്‍ക്കു മുകളില്‍ ചിറക് വിടര്‍ത്തിക്കൊണ്ടും ചിറകു കൂട്ടിപ്പിടിച്ചു കൊണ്ടും പറക്കുന്ന പക്ഷികളുടെ നേര്‍ക്ക് അവര്‍ നോക്കിയില്ലേ? പരമകാരുണികനല്ലാതെ (മറ്റാരും) അവയെ താങ്ങി നിറുത്തുന്നില്ല. തീര്‍ച്ചയായും അവന്‍ എല്ലാകാര്യവും കണ്ടറിയുന്നവനാകുന്നു.
\end{malayalam}}
\flushright{\begin{Arabic}
\quranayah[67][20]
\end{Arabic}}
\flushleft{\begin{malayalam}
അതല്ല പരമകാരുണികന് പുറമെ നിങ്ങളെ സഹായിക്കുവാന്‍ ഒരു പട്ടാളമായിട്ടുള്ളവന്‍ ആരുണ്ട്‌? സത്യനിഷേധികള്‍ വഞ്ചനയില്‍ അകപ്പെട്ടിരിക്കുക മാത്രമാകുന്നു.
\end{malayalam}}
\flushright{\begin{Arabic}
\quranayah[67][21]
\end{Arabic}}
\flushleft{\begin{malayalam}
അതല്ലെങ്കില്‍ അല്ലാഹു തന്‍റെ ഉപജീവനം നിര്‍ത്തിവെച്ചാല്‍ നിങ്ങള്‍ക്ക് ഉപജീവനം നല്‍കുന്നവനായി ആരുണ്ട്‌? എങ്കിലും അവര്‍ ധിക്കാരത്തിലും വെറുപ്പിലും മുഴുകിയിരിക്കയാകുന്നു.
\end{malayalam}}
\flushright{\begin{Arabic}
\quranayah[67][22]
\end{Arabic}}
\flushleft{\begin{malayalam}
അപ്പോള്‍, മുഖം നിലത്തു കുത്തിക്കൊണ്ട് നടക്കുന്നവനാണോ സന്‍മാര്‍ഗം പ്രാപിക്കുന്നവന്‍? അതല്ല നേരെയുള്ള പാതയിലൂടെ ശരിക്ക് നടക്കുന്നവനോ?
\end{malayalam}}
\flushright{\begin{Arabic}
\quranayah[67][23]
\end{Arabic}}
\flushleft{\begin{malayalam}
പറയുക: അവനാണ് നിങ്ങളെ സൃഷ്ടിച്ചുണ്ടാക്കുകയും നിങ്ങള്‍ക്ക് കേള്‍വിയും കാഴ്ചകളും ഹൃദയങ്ങളും ഏര്‍പെടുത്തിത്തരികയും ചെയ്തവന്‍. കുറച്ചു മാത്രമേ നിങ്ങള്‍ നന്ദികാണിക്കുന്നുള്ളൂ.
\end{malayalam}}
\flushright{\begin{Arabic}
\quranayah[67][24]
\end{Arabic}}
\flushleft{\begin{malayalam}
പറയുക: അവനാണ് നിങ്ങളെ ഭൂമിയില്‍ സൃഷ്ടിച്ച് വളര്‍ത്തിയവന്‍. അവങ്കലേക്കാണ് നിങ്ങള്‍ ഒരുമിച്ചുകൂട്ടപ്പെടുന്നത്‌.
\end{malayalam}}
\flushright{\begin{Arabic}
\quranayah[67][25]
\end{Arabic}}
\flushleft{\begin{malayalam}
അവര്‍ പറയുന്നു: എപ്പോഴാണ് ഈ വാഗ്ദാനം (പുലരുന്നത്‌?) നിങ്ങള്‍ സത്യവാന്‍മാരാണെങ്കില്‍ (അതൊന്ന് പറഞ്ഞുതരൂ)
\end{malayalam}}
\flushright{\begin{Arabic}
\quranayah[67][26]
\end{Arabic}}
\flushleft{\begin{malayalam}
പറയുക: ആ അറിവ് അല്ലാഹുവിന്‍റെ പക്കല്‍ മാത്രമാകുന്നു. ഞാന്‍ വ്യക്തമായ താക്കീതുകാരന്‍ മാത്രമാകുന്നു.
\end{malayalam}}
\flushright{\begin{Arabic}
\quranayah[67][27]
\end{Arabic}}
\flushleft{\begin{malayalam}
അത് (താക്കീത് നല്‍കപ്പെട്ട കാര്യം) സമീപസ്ഥമായി അവര്‍ കാണുമ്പോള്‍ സത്യനിഷേധികളുടെ മുഖങ്ങള്‍ക്ക് മ്ലാനത ബാധിക്കുന്നതാണ്‌. നിങ്ങള്‍ ഏതൊന്നിനെപ്പറ്റി വാദിച്ച് കൊണ്ടിരുന്നുവോ അതാകുന്നു ഇത് എന്ന് (അവരോട്‌) പറയപ്പെടുകയും ചെയ്യും.
\end{malayalam}}
\flushright{\begin{Arabic}
\quranayah[67][28]
\end{Arabic}}
\flushleft{\begin{malayalam}
പറയുക: നിങ്ങള്‍ ചിന്തിച്ച് നോക്കിയിട്ടുണ്ടോ? എന്നെയും എന്നോടൊപ്പമുള്ളവരെയും അല്ലാഹു നശിപ്പിക്കുകയോ അല്ലെങ്കില്‍ ഞങ്ങളോടവന്‍ കരുണ കാണിക്കുകയോ ചെയ്താല്‍ വേദനയേറിയ ശിക്ഷയില്‍ നിന്ന് സത്യനിഷേധികളെ രക്ഷിക്കാനാരുണ്ട്‌?
\end{malayalam}}
\flushright{\begin{Arabic}
\quranayah[67][29]
\end{Arabic}}
\flushleft{\begin{malayalam}
പറയുക: അവനാകുന്നു പരമകാരുണികന്‍. അവനില്‍ ഞങ്ങള്‍ വിശ്വസിച്ചിരിക്കുന്നു. അവന്‍റെ മേല്‍ ഞങ്ങള്‍ ഭരമേല്‍പിക്കുകയും ചെയ്തിരിക്കുന്നു. എന്നാല്‍ വഴിയെ നിങ്ങള്‍ക്കറിയാം; ആരാണ് വ്യക്തമായ വഴികേടിലെന്ന്‌.
\end{malayalam}}
\flushright{\begin{Arabic}
\quranayah[67][30]
\end{Arabic}}
\flushleft{\begin{malayalam}
പറയുക: നിങ്ങള്‍ ചിന്തിച്ച് നോക്കിയിട്ടുണ്ടോ? നിങ്ങളുടെ വെള്ളം വറ്റിപ്പോയാല്‍ ആരാണ് നിങ്ങള്‍ക്ക് ഒഴുകുന്ന ഉറവു വെള്ളം കൊണ്ട് വന്നു തരിക?
\end{malayalam}}
\chapter{\textmalayalam{ഖലം ( പേന )}}
\begin{Arabic}
\Huge{\centerline{\basmalah}}\end{Arabic}
\flushright{\begin{Arabic}
\quranayah[68][1]
\end{Arabic}}
\flushleft{\begin{malayalam}
നൂന്‍- പേനയും അവര്‍ എഴുതുന്നതും തന്നെയാണ സത്യം.
\end{malayalam}}
\flushright{\begin{Arabic}
\quranayah[68][2]
\end{Arabic}}
\flushleft{\begin{malayalam}
നിന്‍റെ രക്ഷിതാവിന്‍റെ അനുഗ്രഹം കൊണ്ട് നീ ഒരു ഭ്രാന്തനല്ല.
\end{malayalam}}
\flushright{\begin{Arabic}
\quranayah[68][3]
\end{Arabic}}
\flushleft{\begin{malayalam}
തീര്‍ച്ചയായും നിനക്ക് മുറിഞ്ഞ് പോകാത്ത പ്രതിഫലമുണ്ട്‌.
\end{malayalam}}
\flushright{\begin{Arabic}
\quranayah[68][4]
\end{Arabic}}
\flushleft{\begin{malayalam}
തീര്‍ച്ചയായും നീ മഹത്തായ സ്വഭാവത്തിലാകുന്നു.
\end{malayalam}}
\flushright{\begin{Arabic}
\quranayah[68][5]
\end{Arabic}}
\flushleft{\begin{malayalam}
ആകയാല്‍ വഴിയെ നീ കണ്ടറിയും; അവരും കണ്ടറിയും;
\end{malayalam}}
\flushright{\begin{Arabic}
\quranayah[68][6]
\end{Arabic}}
\flushleft{\begin{malayalam}
നിങ്ങളില്‍ ആരാണ് കുഴപ്പത്തിലകപ്പെട്ടവനെന്ന്‌
\end{malayalam}}
\flushright{\begin{Arabic}
\quranayah[68][7]
\end{Arabic}}
\flushleft{\begin{malayalam}
തീര്‍ച്ചയായും നിന്‍റെ രക്ഷിതാവ് അവന്‍റെ മാര്‍ഗം വിട്ടു പിഴച്ചുപോയവരെപ്പറ്റി നല്ലവണ്ണം അറിയുന്നവനാകുന്നു. സന്‍മാര്‍ഗം പ്രാപിച്ചവരെപ്പറ്റിയും അവന്‍ നല്ലവണ്ണം അറിയുന്നവനാകുന്നു.
\end{malayalam}}
\flushright{\begin{Arabic}
\quranayah[68][8]
\end{Arabic}}
\flushleft{\begin{malayalam}
അതിനാല്‍ സത്യനിഷേധികളെ നീ അനുസരിക്കരുത്‌?
\end{malayalam}}
\flushright{\begin{Arabic}
\quranayah[68][9]
\end{Arabic}}
\flushleft{\begin{malayalam}
നീ വഴങ്ങികൊടുത്തിരുന്നെങ്കില്‍ അവര്‍ക്കും വഴങ്ങിത്തരാമായിരുന്നു എന്നവര്‍ ആഗ്രഹിക്കുന്നു.
\end{malayalam}}
\flushright{\begin{Arabic}
\quranayah[68][10]
\end{Arabic}}
\flushleft{\begin{malayalam}
അധികമായി സത്യം ചെയ്യുന്നവനും, നീചനുമായിട്ടുള്ള യാതൊരാളെയും നീ അനുസരിച്ചു പോകരുത്‌.
\end{malayalam}}
\flushright{\begin{Arabic}
\quranayah[68][11]
\end{Arabic}}
\flushleft{\begin{malayalam}
കുത്തുവാക്ക് പറയുന്നവനും ഏഷണിയുമായി നടക്കുന്നവനുമായ
\end{malayalam}}
\flushright{\begin{Arabic}
\quranayah[68][12]
\end{Arabic}}
\flushleft{\begin{malayalam}
നന്‍മക്ക് തടസ്സം നില്‍ക്കുന്നവനും, അതിക്രമിയും മഹാപാപിയുമായ
\end{malayalam}}
\flushright{\begin{Arabic}
\quranayah[68][13]
\end{Arabic}}
\flushleft{\begin{malayalam}
ക്രൂരനും അതിനു പുറമെ ദുഷ്കീര്‍ത്തി നേടിയവനുമായ
\end{malayalam}}
\flushright{\begin{Arabic}
\quranayah[68][14]
\end{Arabic}}
\flushleft{\begin{malayalam}
അവന്‍ സ്വത്തും സന്താനങ്ങളും ഉള്ളവനായി എന്നതിനാല്‍ (അവന്‍ അത്തരം നിലപാട് സ്വീകരിച്ചു.)
\end{malayalam}}
\flushright{\begin{Arabic}
\quranayah[68][15]
\end{Arabic}}
\flushleft{\begin{malayalam}
നമ്മുടെ ദൃഷ്ടാന്തങ്ങള്‍ അവന്ന് വായിച്ചുകേള്‍പിക്കപ്പെട്ടാല്‍ അവന്‍ പറയും; പൂര്‍വ്വികന്‍മാരുടെ പുരാണകഥകള്‍ എന്ന്‌.
\end{malayalam}}
\flushright{\begin{Arabic}
\quranayah[68][16]
\end{Arabic}}
\flushleft{\begin{malayalam}
വഴിയെ (അവന്‍റെ) തുമ്പിക്കൈ മേല്‍ നാം അവന്ന് അടയാളം വെക്കുന്നതാണ്‌.
\end{malayalam}}
\flushright{\begin{Arabic}
\quranayah[68][17]
\end{Arabic}}
\flushleft{\begin{malayalam}
ആ തോട്ടക്കാരെ നാം പരീക്ഷിച്ചത് പോലെ തീര്‍ച്ചയായും അവരെയും നാം പരീക്ഷിച്ചിരിക്കുകയാണ്‌. പ്രഭാതവേളയില്‍ ആ തോട്ടത്തിലെ പഴങ്ങള്‍ അവര്‍ പറിച്ചെടുക്കുമെന്ന് അവര്‍ സത്യം ചെയ്ത സന്ദര്‍ഭം.
\end{malayalam}}
\flushright{\begin{Arabic}
\quranayah[68][18]
\end{Arabic}}
\flushleft{\begin{malayalam}
അവര്‍ (യാതൊന്നും) ഒഴിവാക്കി പറഞ്ഞിരുന്നില്ല.
\end{malayalam}}
\flushright{\begin{Arabic}
\quranayah[68][19]
\end{Arabic}}
\flushleft{\begin{malayalam}
എന്നിട്ട് അവര്‍ ഉറങ്ങിക്കൊണ്ടിരിക്കെ നിന്‍റെ രക്ഷിതാവിങ്കല്‍ നിന്നുള്ള ഒരു ശിക്ഷ ആ തോട്ടത്തെ ബാധിച്ചു.
\end{malayalam}}
\flushright{\begin{Arabic}
\quranayah[68][20]
\end{Arabic}}
\flushleft{\begin{malayalam}
അങ്ങനെ അത് മുറിച്ചെടുക്കപ്പെട്ടത് പോലെ ആയിത്തീര്‍ന്നു.
\end{malayalam}}
\flushright{\begin{Arabic}
\quranayah[68][21]
\end{Arabic}}
\flushleft{\begin{malayalam}
അങ്ങനെ പ്രഭാതവേളയില്‍ അവര്‍ പരസ്പരം വിളിച്ചുപറഞ്ഞു:
\end{malayalam}}
\flushright{\begin{Arabic}
\quranayah[68][22]
\end{Arabic}}
\flushleft{\begin{malayalam}
നിങ്ങള്‍ പറിച്ചെടുക്കാന്‍ പോകുകയാണെങ്കില്‍ നിങ്ങളുടെ കൃഷിസ്ഥലത്തേക്ക് നിങ്ങള്‍ കാലത്തുതന്നെ പുറപ്പെടുക.
\end{malayalam}}
\flushright{\begin{Arabic}
\quranayah[68][23]
\end{Arabic}}
\flushleft{\begin{malayalam}
അവര്‍ അന്യോന്യം മന്ത്രിച്ചു കൊണ്ടു പോയി.
\end{malayalam}}
\flushright{\begin{Arabic}
\quranayah[68][24]
\end{Arabic}}
\flushleft{\begin{malayalam}
ഇന്ന് ആ തോട്ടത്തില്‍ നിങ്ങളുടെ അടുത്ത് ഒരു സാധുവും കടന്നു വരാന്‍ ഇടയാവരുത് എന്ന്‌.
\end{malayalam}}
\flushright{\begin{Arabic}
\quranayah[68][25]
\end{Arabic}}
\flushleft{\begin{malayalam}
അവര്‍ (സാധുക്കളെ) തടസ്സപ്പെടുത്താന്‍ കഴിവുള്ളവരായിക്കൊണ്ടു തന്നെ കാലത്ത് പുറപ്പെടുകയും ചെയ്തു.
\end{malayalam}}
\flushright{\begin{Arabic}
\quranayah[68][26]
\end{Arabic}}
\flushleft{\begin{malayalam}
അങ്ങനെ അത് (തോട്ടം) കണ്ടപ്പോള്‍ അവര്‍ പറഞ്ഞു: തീര്‍ച്ചയായും നാം പിഴവു പറ്റിയവരാകുന്നു.
\end{malayalam}}
\flushright{\begin{Arabic}
\quranayah[68][27]
\end{Arabic}}
\flushleft{\begin{malayalam}
അല്ല, നാം നഷ്ടം നേരിട്ടവരാകുന്നു.
\end{malayalam}}
\flushright{\begin{Arabic}
\quranayah[68][28]
\end{Arabic}}
\flushleft{\begin{malayalam}
അവരുടെ കൂട്ടത്തില്‍ മദ്ധ്യനിലപാടുകാരനായ ഒരാള്‍ പറഞ്ഞു: ഞാന്‍ നിങ്ങളോട് പറഞ്ഞില്ലേ? എന്താണ് നിങ്ങള്‍ അല്ലാഹുവെ പ്രകീര്‍ത്തിക്കാതിരുന്നത്‌?
\end{malayalam}}
\flushright{\begin{Arabic}
\quranayah[68][29]
\end{Arabic}}
\flushleft{\begin{malayalam}
അവര്‍ പറഞ്ഞു: നമ്മുടെ രക്ഷിതാവ് എത്രയോ പരിശുദ്ധന്‍! തീര്‍ച്ചയായും നാം അക്രമികളായിരിക്കുന്നു.
\end{malayalam}}
\flushright{\begin{Arabic}
\quranayah[68][30]
\end{Arabic}}
\flushleft{\begin{malayalam}
അങ്ങനെ പരസ്പരം കുറ്റപ്പെടുത്തിക്കൊണ്ട് അവരില്‍ ചിലര്‍ ചിലരുടെ നേര്‍ക്ക് തിരിഞ്ഞു.
\end{malayalam}}
\flushright{\begin{Arabic}
\quranayah[68][31]
\end{Arabic}}
\flushleft{\begin{malayalam}
അവര്‍ പറഞ്ഞു: നമ്മുടെ നാശമേ! തീര്‍ച്ചയായും നാം അതിക്രമകാരികളായിരിക്കുന്നു.
\end{malayalam}}
\flushright{\begin{Arabic}
\quranayah[68][32]
\end{Arabic}}
\flushleft{\begin{malayalam}
നമ്മുടെ രക്ഷിതാവ് അതിനെക്കാള്‍ ഉത്തമമായത് നമുക്ക് പകരം തന്നേക്കാം. തീര്‍ച്ചയായും നാം നമ്മുടെ രക്ഷിതാവിങ്കലേക്ക് ആഗ്രഹിച്ചു ചെല്ലുന്നവരാകുന്നു.
\end{malayalam}}
\flushright{\begin{Arabic}
\quranayah[68][33]
\end{Arabic}}
\flushleft{\begin{malayalam}
അപ്രകാരമാകുന്നു ശിക്ഷ. പരലോകശിക്ഷയാവട്ടെ കൂടുതല്‍ ഗൌരവമുള്ളതാകുന്നു. അവര്‍ അറിഞ്ഞിരുന്നെങ്കില്‍!
\end{malayalam}}
\flushright{\begin{Arabic}
\quranayah[68][34]
\end{Arabic}}
\flushleft{\begin{malayalam}
തീര്‍ച്ചയായും സൂക്ഷ്മത പാലിക്കുന്നവര്‍ക്ക് അവരുടെ രക്ഷിതാവിങ്കല്‍ അനുഗ്രഹങ്ങളുടെ സ്വര്‍ഗത്തോപ്പുകളുണ്ട്‌.
\end{malayalam}}
\flushright{\begin{Arabic}
\quranayah[68][35]
\end{Arabic}}
\flushleft{\begin{malayalam}
അപ്പോള്‍ മുസ്ലിംകളെ നാം കുറ്റവാളികളെപോലെ ആക്കുമോ?
\end{malayalam}}
\flushright{\begin{Arabic}
\quranayah[68][36]
\end{Arabic}}
\flushleft{\begin{malayalam}
നിങ്ങള്‍ക്കെന്തു പറ്റി? നിങ്ങള്‍ എങ്ങനെയാണ് വിധികല്‍പിക്കുന്നത്‌?
\end{malayalam}}
\flushright{\begin{Arabic}
\quranayah[68][37]
\end{Arabic}}
\flushleft{\begin{malayalam}
അതല്ല, നിങ്ങള്‍ക്കു വല്ല ഗ്രന്ഥവും കിട്ടിയിട്ട് നിങ്ങളതില്‍ പഠനം നടത്തിക്കൊണ്ടിരിക്കുകയാണോ?
\end{malayalam}}
\flushright{\begin{Arabic}
\quranayah[68][38]
\end{Arabic}}
\flushleft{\begin{malayalam}
നിങ്ങള്‍ (യഥേഷ്ടം) തെരഞ്ഞെടുക്കുന്ന കാര്യങ്ങള്‍ നിങ്ങള്‍ക്ക് അതില്‍ (ആ ഗ്രന്ഥത്തില്‍) വന്നിട്ടുണ്ടോ?
\end{malayalam}}
\flushright{\begin{Arabic}
\quranayah[68][39]
\end{Arabic}}
\flushleft{\begin{malayalam}
അതല്ല, ഉയിര്‍ത്തെഴുന്നേല്‍പിന്‍റെ നാളുവരെ എത്തുന്ന - നിങ്ങള്‍ വിധിക്കുന്നതെല്ലാം നിങ്ങള്‍ക്കായിരിക്കുമെന്നതിനുള്ള-വല്ല കരാറുകളും നിങ്ങളോട് നാം ബാധ്യതയേറ്റതായി ഉണ്ടോ?
\end{malayalam}}
\flushright{\begin{Arabic}
\quranayah[68][40]
\end{Arabic}}
\flushleft{\begin{malayalam}
അവരില്‍ ആരാണ് ആ കാര്യത്തിന് ഉത്തരവാദിത്തം ഏല്‍ക്കാനുള്ളത് എന്ന് അവരോട് ചോദിച്ചു നോക്കുക.
\end{malayalam}}
\flushright{\begin{Arabic}
\quranayah[68][41]
\end{Arabic}}
\flushleft{\begin{malayalam}
അതല്ല, അവര്‍ക്ക് വല്ല പങ്കുകാരുമുണ്ടോ? എങ്കില്‍ അവരുടെ ആ പങ്കുകാരെ അവര്‍ കൊണ്ടുവരട്ടെ. അവര്‍ സത്യവാന്‍മാരാണെങ്കില്‍.
\end{malayalam}}
\flushright{\begin{Arabic}
\quranayah[68][42]
\end{Arabic}}
\flushleft{\begin{malayalam}
കണങ്കാല്‍ വെളിവാക്കപ്പെടുന്ന (ഭയങ്കരമായ) ഒരു ദിവസത്തെ നിങ്ങള്‍ ഓര്‍ക്കുക. സുജൂദ് ചെയ്യാന്‍ (അന്ന്‌) അവര്‍ ക്ഷണിക്കപ്പെടും. അപ്പോള്‍ അവര്‍ക്കതിന് സാധിക്കുകയില്ല.
\end{malayalam}}
\flushright{\begin{Arabic}
\quranayah[68][43]
\end{Arabic}}
\flushleft{\begin{malayalam}
അവരുടെ കണ്ണുകള്‍ കീഴ്പോട്ട് താഴ്ന്നിരിക്കും. നിന്ദ്യത അവരെ ആവരണം ചെയ്യും. അവര്‍ സുരക്ഷിതരായിരുന്ന സമയത്ത് സുജൂദിനായി അവര്‍ ക്ഷണിക്കപ്പെട്ടിരുന്നു.
\end{malayalam}}
\flushright{\begin{Arabic}
\quranayah[68][44]
\end{Arabic}}
\flushleft{\begin{malayalam}
ആകയാല്‍ എന്നെയും ഈ വര്‍ത്തമാനം നിഷേധിച്ചു കളയുന്നവരെയും കുടി വിട്ടേക്കുക. അവര്‍ അറിയാത്ത വിധത്തിലൂടെ നാം അവരെ പടിപടിയായി പിടികൂടിക്കൊള്ളാം.
\end{malayalam}}
\flushright{\begin{Arabic}
\quranayah[68][45]
\end{Arabic}}
\flushleft{\begin{malayalam}
ഞാന്‍ അവര്‍ക്ക് നീട്ടിയിട്ട് കൊടുക്കുകയും ചെയ്യും. തീര്‍ച്ചയായും എന്‍റെ തന്ത്രം ശക്തമാകുന്നു.
\end{malayalam}}
\flushright{\begin{Arabic}
\quranayah[68][46]
\end{Arabic}}
\flushleft{\begin{malayalam}
അതല്ല, നീ അവരോട് വല്ല പ്രതിഫലവും ചോദിച്ചിട്ട് അവര്‍ കടബാധയാല്‍ ഞെരുങ്ങിയിരിക്കുകയാണോ?
\end{malayalam}}
\flushright{\begin{Arabic}
\quranayah[68][47]
\end{Arabic}}
\flushleft{\begin{malayalam}
അതല്ല, അവരുടെ അടുക്കല്‍ അദൃശ്യജ്ഞാനമുണ്ടായിട്ട് അവര്‍ എഴുതി എടുത്തു കൊണ്ടിരിക്കുകയാണോ?
\end{malayalam}}
\flushright{\begin{Arabic}
\quranayah[68][48]
\end{Arabic}}
\flushleft{\begin{malayalam}
അതുകൊണ്ട് നിന്‍റെ രക്ഷിതാവിന്‍റെ വിധി കാത്ത് നീ ക്ഷമിച്ചു കൊള്ളുക. നീ മത്സ്യത്തിന്‍റെ ആളെപ്പോലെ (യൂനുസ് നബിയെപ്പോലെ) ആകരുത്‌. അദ്ദേഹം ദുഃഖനിമഗ്നായികൊണ്ട് വിളിച്ചു പ്രാര്‍ത്ഥിച്ച സന്ദര്‍ഭം.
\end{malayalam}}
\flushright{\begin{Arabic}
\quranayah[68][49]
\end{Arabic}}
\flushleft{\begin{malayalam}
അദ്ദേഹത്തിന്‍റെ രക്ഷിതാവിങ്കല്‍ നിന്നുള്ള അനുഗ്രഹം അദ്ദേഹത്തെ വീണ്ടെടുത്തിട്ടില്ലായിരുന്നെങ്കില്‍ അദ്ദേഹം ആ പാഴ്ഭൂമിയില്‍ ആക്ഷേപാര്‍ഹനായിക്കൊണ്ട് പുറന്തള്ളപ്പെടുമായിരുന്നു.
\end{malayalam}}
\flushright{\begin{Arabic}
\quranayah[68][50]
\end{Arabic}}
\flushleft{\begin{malayalam}
അപ്പോള്‍ അദ്ദേഹത്തിന്‍റെ രക്ഷിതാവ് അദ്ദേഹത്തെ തെരഞ്ഞെടുക്കുകയും എന്നിട്ട് അദ്ദേഹത്തെ സജ്ജനങ്ങളുടെ കൂട്ടത്തിലാക്കുകയും ചെയ്തു.
\end{malayalam}}
\flushright{\begin{Arabic}
\quranayah[68][51]
\end{Arabic}}
\flushleft{\begin{malayalam}
സത്യനിഷേധികള്‍ ഈ ഉല്‍ബോധനം കേള്‍ക്കുമ്പോള്‍ അവരുടെ കണ്ണുകള്‍കൊണ്ട് നോക്കിയിട്ട് നീ ഇടറി വീഴുമാറാക്കുക തന്നെ ചെയ്യും. തീര്‍ച്ചയായും ഇവന്‍ ഒരു ഭ്രാന്തന്‍ തന്നെയാണ് എന്നവര്‍ പറയും.
\end{malayalam}}
\flushright{\begin{Arabic}
\quranayah[68][52]
\end{Arabic}}
\flushleft{\begin{malayalam}
ഇത് ലോകര്‍ക്കുള്ള ഒരു ഉല്‍ബോധനമല്ലാതെ മറ്റൊന്നുമല്ല.
\end{malayalam}}
\chapter{\textmalayalam{ഹാഖ ( യഥാര്‍ത്ഥ സംഭവം )}}
\begin{Arabic}
\Huge{\centerline{\basmalah}}\end{Arabic}
\flushright{\begin{Arabic}
\quranayah[69][1]
\end{Arabic}}
\flushleft{\begin{malayalam}
ആ യഥാര്‍ത്ഥ സംഭവം!
\end{malayalam}}
\flushright{\begin{Arabic}
\quranayah[69][2]
\end{Arabic}}
\flushleft{\begin{malayalam}
എന്താണ് ആ യഥാര്‍ത്ഥ സംഭവം?
\end{malayalam}}
\flushright{\begin{Arabic}
\quranayah[69][3]
\end{Arabic}}
\flushleft{\begin{malayalam}
ആ യഥാര്‍ത്ഥ സംഭവം എന്താണെന്ന് നിനക്കെന്തറിയാം?
\end{malayalam}}
\flushright{\begin{Arabic}
\quranayah[69][4]
\end{Arabic}}
\flushleft{\begin{malayalam}
ഥമൂദ് സമുദായവും ആദ് സമുദായവും ആ ഭയങ്കര സംഭവത്തെ നിഷേധിച്ചു കളഞ്ഞു.
\end{malayalam}}
\flushright{\begin{Arabic}
\quranayah[69][5]
\end{Arabic}}
\flushleft{\begin{malayalam}
എന്നാല്‍ ഥമൂദ് സമുദായം അത്യന്തം ഭീകരമായ ഒരു ശിക്ഷ കൊണ്ട് നശിപ്പിക്കപ്പെട്ടു.
\end{malayalam}}
\flushright{\begin{Arabic}
\quranayah[69][6]
\end{Arabic}}
\flushleft{\begin{malayalam}
എന്നാല്‍ ആദ് സമുദായം, ആഞ്ഞു വീശുന്ന അത്യുഗ്രമായ കാറ്റ് കൊണ്ട് നശിപ്പിക്കപ്പെട്ടു.
\end{malayalam}}
\flushright{\begin{Arabic}
\quranayah[69][7]
\end{Arabic}}
\flushleft{\begin{malayalam}
തുടര്‍ച്ചയായ ഏഴു രാത്രിയും എട്ടു പകലും അത് (കാറ്റ്‌) അവരുടെ നേര്‍ക്ക് അവന്‍ തിരിച്ചുവിട്ടു. അപ്പോള്‍ കടപുഴകി വീണ ഈന്തപ്പനത്തടികള്‍ പോലെ ആ കാറ്റില്‍ ജനങ്ങള്‍ വീണുകിടക്കുന്നതായി നിനക്ക് കാണാം.
\end{malayalam}}
\flushright{\begin{Arabic}
\quranayah[69][8]
\end{Arabic}}
\flushleft{\begin{malayalam}
ഇനി അവരുടെതായി അവശേഷിക്കുന്ന വല്ലതും നീ കാണുന്നുണ്ടോ?
\end{malayalam}}
\flushright{\begin{Arabic}
\quranayah[69][9]
\end{Arabic}}
\flushleft{\begin{malayalam}
ഫിര്‍ഔനും, അവന്‍റെ മുമ്പുള്ളവരും കീഴ്മേല്‍ മറിഞ്ഞ രാജ്യങ്ങളും (തെറ്റായ പ്രവര്‍ത്തനം കൊണ്ടു വന്നു.
\end{malayalam}}
\flushright{\begin{Arabic}
\quranayah[69][10]
\end{Arabic}}
\flushleft{\begin{malayalam}
അവര്‍ അവരുടെ രക്ഷിതാവിന്‍റെ ദൂതനെ ധിക്കരിക്കുകയും, അപ്പോള്‍ അവന്‍ അവരെ ശക്തിയേറിയ ഒരു പിടുത്തം പിടിക്കുകയും ചെയ്തു.
\end{malayalam}}
\flushright{\begin{Arabic}
\quranayah[69][11]
\end{Arabic}}
\flushleft{\begin{malayalam}
തീര്‍ച്ചയായും നാം, വെള്ളം അതിരുകവിഞ്ഞ സമയത്ത് നിങ്ങളെ കപ്പലില്‍ കയറ്റി രക്ഷിക്കുകയുണ്ടായി.
\end{malayalam}}
\flushright{\begin{Arabic}
\quranayah[69][12]
\end{Arabic}}
\flushleft{\begin{malayalam}
നിങ്ങള്‍ക്ക് നാം അതൊരു സ്മരണയാക്കുവാനും ശ്രദ്ധിച്ചു മനസ്സിലാക്കുന്ന കാതുകള്‍ അത് ശ്രദ്ധിച്ചു മനസ്സിലാക്കുവാനും വേണ്ടി.
\end{malayalam}}
\flushright{\begin{Arabic}
\quranayah[69][13]
\end{Arabic}}
\flushleft{\begin{malayalam}
കാഹളത്തില്‍ ഒരു ഊത്ത് ഊതപ്പെട്ടാല്‍,
\end{malayalam}}
\flushright{\begin{Arabic}
\quranayah[69][14]
\end{Arabic}}
\flushleft{\begin{malayalam}
ഭൂമിയും പര്‍വ്വതങ്ങളും പൊക്കിയെടുക്കപ്പെടുകയും എന്നിട്ട് അവ രണ്ടും ഒരു ഇടിച്ചു തകര്‍ക്കലിന് വിധേയമാക്കപ്പെടുകയും ചെയ്താല്‍!
\end{malayalam}}
\flushright{\begin{Arabic}
\quranayah[69][15]
\end{Arabic}}
\flushleft{\begin{malayalam}
അന്നേ ദിവസം ആ സംഭവം സംഭവിക്കുകയായി.
\end{malayalam}}
\flushright{\begin{Arabic}
\quranayah[69][16]
\end{Arabic}}
\flushleft{\begin{malayalam}
ആകാശം പൊട്ടിപ്പിളരുകയും ചെയ്യും. അന്ന് അത് ദുര്‍ബലമായിരിക്കും.
\end{malayalam}}
\flushright{\begin{Arabic}
\quranayah[69][17]
\end{Arabic}}
\flushleft{\begin{malayalam}
മലക്കുകള്‍ അതിന്‍റെ നാനാഭാഗങ്ങളിലുമുണ്ടായിരിക്കും. നിന്‍റെ രക്ഷിതാവിന്‍റെ സിംഹാസനത്തെ അവരുടെ മീതെ അന്നു എട്ടുകൂട്ടര്‍ വഹിക്കുന്നതാണ്‌.
\end{malayalam}}
\flushright{\begin{Arabic}
\quranayah[69][18]
\end{Arabic}}
\flushleft{\begin{malayalam}
അന്നേ ദിവസം നിങ്ങള്‍ പ്രദര്‍ശിപ്പിക്കപ്പെടുന്നതാണ്‌. യാതൊരു മറഞ്ഞകാര്യവും നിങ്ങളില്‍ നിന്ന് മറഞ്ഞു പോകുന്നതകല്ല.
\end{malayalam}}
\flushright{\begin{Arabic}
\quranayah[69][19]
\end{Arabic}}
\flushleft{\begin{malayalam}
എന്നാല്‍ വലതുകൈയില്‍ തന്‍റെ രേഖ നല്‍കപ്പെട്ടവന്‍ പറയും: ഇതാ എന്‍റെ ഗ്രന്ഥം വായിച്ചുനോക്കൂ.
\end{malayalam}}
\flushright{\begin{Arabic}
\quranayah[69][20]
\end{Arabic}}
\flushleft{\begin{malayalam}
തീര്‍ച്ചയായും ഞാന്‍ വിചാരിച്ചിരുന്നു. ഞാന്‍ എന്‍റെ വിചാരണയെ നേരിടേണ്ടി വരുമെന്ന്‌.
\end{malayalam}}
\flushright{\begin{Arabic}
\quranayah[69][21]
\end{Arabic}}
\flushleft{\begin{malayalam}
അതിനാല്‍ അവന്‍ തൃപ്തികരമായ ജീവിതത്തിലാകുന്നു.
\end{malayalam}}
\flushright{\begin{Arabic}
\quranayah[69][22]
\end{Arabic}}
\flushleft{\begin{malayalam}
ഉന്നതമായ സ്വര്‍ഗത്തില്‍.
\end{malayalam}}
\flushright{\begin{Arabic}
\quranayah[69][23]
\end{Arabic}}
\flushleft{\begin{malayalam}
അവയിലെ പഴങ്ങള്‍ അടുത്തു വരുന്നവയാകുന്നു.
\end{malayalam}}
\flushright{\begin{Arabic}
\quranayah[69][24]
\end{Arabic}}
\flushleft{\begin{malayalam}
കഴിഞ്ഞുപോയ ദിവസങ്ങളില്‍ നിങ്ങള്‍ മുന്‍കൂട്ടി ചെയ്തതിന്‍റെ ഫലമായി നിങ്ങള്‍ ആനന്ദത്തോടെ തിന്നുകയും കുടിക്കുകയും ചെയ്തു കൊള്ളുക. (എന്ന് അവരോട് പറയപ്പെടും.)
\end{malayalam}}
\flushright{\begin{Arabic}
\quranayah[69][25]
\end{Arabic}}
\flushleft{\begin{malayalam}
എന്നാല്‍ ഇടതു കയ്യില്‍ ഗ്രന്ഥം നല്‍കപ്പെട്ടവനാകട്ടെ ഇപ്രകാരം പറയുന്നതാണ്‌. ഹാ! എന്‍റെ ഗ്രന്ഥം എനിക്ക് നല്‍കപ്പെടാതിരുന്നെങ്കില്‍,
\end{malayalam}}
\flushright{\begin{Arabic}
\quranayah[69][26]
\end{Arabic}}
\flushleft{\begin{malayalam}
എന്‍റെ വിചാരണ എന്താണെന്ന് ഞാന്‍ അറിയാതിരുന്നെങ്കില്‍ (എത്ര നന്നായിരുന്നു.)
\end{malayalam}}
\flushright{\begin{Arabic}
\quranayah[69][27]
\end{Arabic}}
\flushleft{\begin{malayalam}
അത് (മരണം) എല്ലാം അവസാനിപ്പിക്കുന്നതായിരുന്നെങ്കില്‍ (എത്ര നന്നായിരുന്നു!)
\end{malayalam}}
\flushright{\begin{Arabic}
\quranayah[69][28]
\end{Arabic}}
\flushleft{\begin{malayalam}
എന്‍റെ ധനം എനിക്ക് പ്രയോജനപ്പെട്ടില്ല.
\end{malayalam}}
\flushright{\begin{Arabic}
\quranayah[69][29]
\end{Arabic}}
\flushleft{\begin{malayalam}
എന്‍റെ അധികാരം എന്നില്‍ നിന്ന് നഷ്ടപ്പെട്ടുപോയി.
\end{malayalam}}
\flushright{\begin{Arabic}
\quranayah[69][30]
\end{Arabic}}
\flushleft{\begin{malayalam}
(അപ്പോള്‍ ഇപ്രകാരം കല്‍പനയുണ്ടാകും:) നിങ്ങള്‍ അവനെ പിടിച്ച് ബന്ധനത്തിലിടൂ.
\end{malayalam}}
\flushright{\begin{Arabic}
\quranayah[69][31]
\end{Arabic}}
\flushleft{\begin{malayalam}
പിന്നെ അവനെ നിങ്ങള്‍ ജ്വലിക്കുന്ന നരകത്തില്‍ പ്രവേശിപ്പിക്കൂ.
\end{malayalam}}
\flushright{\begin{Arabic}
\quranayah[69][32]
\end{Arabic}}
\flushleft{\begin{malayalam}
പിന്നെ, എഴുപത് മുഴം നീളമുള്ള ഒരു ചങ്ങലയില്‍ അവനെ നിങ്ങള്‍ പ്രവേശിപ്പിക്കൂ.
\end{malayalam}}
\flushright{\begin{Arabic}
\quranayah[69][33]
\end{Arabic}}
\flushleft{\begin{malayalam}
തീര്‍ച്ചയായും അവന്‍ മഹാനായ അല്ലാഹുവില്‍ വിശ്വസിച്ചിരുന്നില്ല.
\end{malayalam}}
\flushright{\begin{Arabic}
\quranayah[69][34]
\end{Arabic}}
\flushleft{\begin{malayalam}
സാധുവിന് ഭക്ഷണം കൊടുക്കുവാന്‍ അവന്‍ പ്രോത്സാഹിപ്പിച്ചിരുന്നുമില്ല.
\end{malayalam}}
\flushright{\begin{Arabic}
\quranayah[69][35]
\end{Arabic}}
\flushleft{\begin{malayalam}
അതിനാല്‍ ഇന്ന് ഇവിടെ അവന്ന് ഒരു ഉറ്റബന്ധുവുമില്ല.
\end{malayalam}}
\flushright{\begin{Arabic}
\quranayah[69][36]
\end{Arabic}}
\flushleft{\begin{malayalam}
ദുര്‍നീരുകള്‍ ഒലിച്ചു കൂടിയതില്‍ നിന്നല്ലാതെ മറ്റു ആഹാരവുമില്ല.
\end{malayalam}}
\flushright{\begin{Arabic}
\quranayah[69][37]
\end{Arabic}}
\flushleft{\begin{malayalam}
തെറ്റുകാരല്ലാതെ അതു ഭക്ഷിക്കുകയില്ല.
\end{malayalam}}
\flushright{\begin{Arabic}
\quranayah[69][38]
\end{Arabic}}
\flushleft{\begin{malayalam}
എന്നാല്‍ നിങ്ങള്‍ കാണുന്നവയെക്കൊണ്ട് ഞാന്‍ സത്യം ചെയ്ത് പറയുന്നു:
\end{malayalam}}
\flushright{\begin{Arabic}
\quranayah[69][39]
\end{Arabic}}
\flushleft{\begin{malayalam}
നിങ്ങള്‍ കാണാത്തവയെക്കൊണ്ടും
\end{malayalam}}
\flushright{\begin{Arabic}
\quranayah[69][40]
\end{Arabic}}
\flushleft{\begin{malayalam}
തീര്‍ച്ചയായും ഇത് മാന്യനായ ഒരു ദൂതന്‍റെ വാക്കു തന്നെയാകുന്നു.
\end{malayalam}}
\flushright{\begin{Arabic}
\quranayah[69][41]
\end{Arabic}}
\flushleft{\begin{malayalam}
ഇതൊരു കവിയുടെ വാക്കല്ല. വളരെ കുറച്ചേ നിങ്ങള്‍ വിശ്വസിക്കുന്നുള്ളൂ.
\end{malayalam}}
\flushright{\begin{Arabic}
\quranayah[69][42]
\end{Arabic}}
\flushleft{\begin{malayalam}
ഒരു ജ്യോത്സ്യന്‍റെ വാക്കുമല്ല. വളരെക്കുറച്ചേ നിങ്ങള്‍ ആലോചിച്ചു മനസ്സിലാക്കുന്നുള്ളൂ.
\end{malayalam}}
\flushright{\begin{Arabic}
\quranayah[69][43]
\end{Arabic}}
\flushleft{\begin{malayalam}
ഇത് ലോകരക്ഷിതാവിങ്കല്‍ നിന്ന് അവതരിപ്പിക്കപ്പെട്ടതാകുന്നു.
\end{malayalam}}
\flushright{\begin{Arabic}
\quranayah[69][44]
\end{Arabic}}
\flushleft{\begin{malayalam}
നമ്മുടെ പേരില്‍ അദ്ദേഹം (പ്രവാചകന്‍) വല്ല വാക്കും കെട്ടിച്ചമച്ചു പറഞ്ഞിരുന്നെങ്കില്‍
\end{malayalam}}
\flushright{\begin{Arabic}
\quranayah[69][45]
\end{Arabic}}
\flushleft{\begin{malayalam}
അദ്ദേഹത്തെ നാം വലതുകൈ കൊണ്ട് പിടികൂടുകയും,
\end{malayalam}}
\flushright{\begin{Arabic}
\quranayah[69][46]
\end{Arabic}}
\flushleft{\begin{malayalam}
എന്നിട്ട് അദ്ദേഹത്തിന്‍റെ ജീവനാഡി നാം മുറിച്ചുകളയുകയും ചെയ്യുമായിരുന്നു.
\end{malayalam}}
\flushright{\begin{Arabic}
\quranayah[69][47]
\end{Arabic}}
\flushleft{\begin{malayalam}
അപ്പോള്‍ നിങ്ങളില്‍ ആര്‍ക്കും അദ്ദേഹത്തില്‍ നിന്ന് (ശിക്ഷയെ) തടയാനാവില്ല.
\end{malayalam}}
\flushright{\begin{Arabic}
\quranayah[69][48]
\end{Arabic}}
\flushleft{\begin{malayalam}
തീര്‍ച്ചയായും ഇത് (ഖുര്‍ആന്‍) ഭയഭക്തിയുള്ളവര്‍ക്ക് ഒരു ഉല്‍ബോധനമാകുന്നു.
\end{malayalam}}
\flushright{\begin{Arabic}
\quranayah[69][49]
\end{Arabic}}
\flushleft{\begin{malayalam}
തീര്‍ച്ചയായും നിങ്ങളുടെ കൂട്ടത്തില്‍ (ഇതിനെ) നിഷേധിച്ചു തള്ളുന്നവരുണ്ടെന്ന് നമുക്കറിയാം.
\end{malayalam}}
\flushright{\begin{Arabic}
\quranayah[69][50]
\end{Arabic}}
\flushleft{\begin{malayalam}
തീര്‍ച്ചയായും ഇത് സത്യനിഷേധികള്‍ക്ക് ഖേദത്തിന് കാരണവുമാകുന്നു.
\end{malayalam}}
\flushright{\begin{Arabic}
\quranayah[69][51]
\end{Arabic}}
\flushleft{\begin{malayalam}
തീര്‍ച്ചയായും ഇത് ദൃഢമായ യാഥാര്‍ത്ഥ്യമാകുന്നു.
\end{malayalam}}
\flushright{\begin{Arabic}
\quranayah[69][52]
\end{Arabic}}
\flushleft{\begin{malayalam}
അതിനാല്‍ നീ നിന്‍റെ മഹാനായ രക്ഷിതാവിന്‍റെ നാമത്തെ പ്രകീര്‍ത്തിക്കുക.
\end{malayalam}}
\chapter{\textmalayalam{മആരിജ് ( കയറുന്ന വഴികള്‍ )}}
\begin{Arabic}
\Huge{\centerline{\basmalah}}\end{Arabic}
\flushright{\begin{Arabic}
\quranayah[70][1]
\end{Arabic}}
\flushleft{\begin{malayalam}
സംഭവിക്കാനിരിക്കുന്ന ഒരു ശിക്ഷയെ ഒരു ചോദ്യകര്‍ത്താവ് അതാ ആവശ്യപ്പെട്ടിരിക്കുന്നു.
\end{malayalam}}
\flushright{\begin{Arabic}
\quranayah[70][2]
\end{Arabic}}
\flushleft{\begin{malayalam}
സത്യനിഷേധികള്‍ക്ക് അത് തടുക്കുവാന്‍ ആരുമില്ല.
\end{malayalam}}
\flushright{\begin{Arabic}
\quranayah[70][3]
\end{Arabic}}
\flushleft{\begin{malayalam}
കയറിപ്പോകുന്ന വഴികളുടെ അധിപനായ അല്ലാഹുവിങ്കല്‍ നിന്ന് വരുന്ന (ശിക്ഷയെ).
\end{malayalam}}
\flushright{\begin{Arabic}
\quranayah[70][4]
\end{Arabic}}
\flushleft{\begin{malayalam}
അമ്പതിനായിരം കൊല്ലത്തിന്‍റെ അളവുള്ളതായ ഒരു ദിവസത്തില്‍ മലക്കുകളും ആത്മാവും അവങ്കലേക്ക് കയറിപ്പോകുന്നു.
\end{malayalam}}
\flushright{\begin{Arabic}
\quranayah[70][5]
\end{Arabic}}
\flushleft{\begin{malayalam}
എന്നാല്‍ (നബിയേ,) നീ ഭംഗിയായ ക്ഷമ കൈക്കൊള്ളുക.
\end{malayalam}}
\flushright{\begin{Arabic}
\quranayah[70][6]
\end{Arabic}}
\flushleft{\begin{malayalam}
തീര്‍ച്ചയായും അവര്‍ അതിനെ വിദൂരമായി കാണുന്നു.
\end{malayalam}}
\flushright{\begin{Arabic}
\quranayah[70][7]
\end{Arabic}}
\flushleft{\begin{malayalam}
നാം അതിനെ അടുത്തതായും കാണുന്നു.
\end{malayalam}}
\flushright{\begin{Arabic}
\quranayah[70][8]
\end{Arabic}}
\flushleft{\begin{malayalam}
ആകാശം ഉരുകിയ ലോഹം പോലെ ആകുന്ന ദിവസം!
\end{malayalam}}
\flushright{\begin{Arabic}
\quranayah[70][9]
\end{Arabic}}
\flushleft{\begin{malayalam}
പര്‍വ്വതങ്ങള്‍ കടഞ്ഞരോമം പോലെയും.
\end{malayalam}}
\flushright{\begin{Arabic}
\quranayah[70][10]
\end{Arabic}}
\flushleft{\begin{malayalam}
ഒരുറ്റ ബന്ധുവും മറ്റൊരു ഉറ്റബന്ധുവിനോട് (അന്ന്‌) യാതൊന്നും ചോദിക്കുകയില്ല.
\end{malayalam}}
\flushright{\begin{Arabic}
\quranayah[70][11]
\end{Arabic}}
\flushleft{\begin{malayalam}
അവര്‍ക്ക് അന്യോന്യം കാണിക്കപ്പെടും. തന്‍റെ മക്കളെ പ്രായശ്ചിത്തമായി നല്‍കി കൊണ്ട് ആ ദിവസത്തെ ശിക്ഷയില്‍ നിന്ന് മോചനം നേടാന്‍ കഴിഞ്ഞിരുന്നെങ്കില്‍ എന്ന് കുറ്റവാളി ആഗ്രഹിക്കും.
\end{malayalam}}
\flushright{\begin{Arabic}
\quranayah[70][12]
\end{Arabic}}
\flushleft{\begin{malayalam}
തന്‍റെ ഭാര്യയെയും സഹോദരനെയും
\end{malayalam}}
\flushright{\begin{Arabic}
\quranayah[70][13]
\end{Arabic}}
\flushleft{\begin{malayalam}
തനിക്ക് അഭയം നല്‍കിയിരുന്ന തന്‍റെ ബന്ധുക്കളെയും
\end{malayalam}}
\flushright{\begin{Arabic}
\quranayah[70][14]
\end{Arabic}}
\flushleft{\begin{malayalam}
ഭൂമിയിലുള്ള മുഴുവന്‍ ആളുകളെയും. എന്നിട്ട് അതവനെ രക്ഷപ്പെടുത്തുകയും ചെയ്തിരുന്നെങ്കില്‍ എന്ന്‌
\end{malayalam}}
\flushright{\begin{Arabic}
\quranayah[70][15]
\end{Arabic}}
\flushleft{\begin{malayalam}
സംശയം വേണ്ട, തീര്‍ച്ചയായും അത് ആളിക്കത്തുന്ന നരകമാകുന്നു.
\end{malayalam}}
\flushright{\begin{Arabic}
\quranayah[70][16]
\end{Arabic}}
\flushleft{\begin{malayalam}
തലയുടെ തൊലിയുരിച്ചു കളയുന്ന നരകാഗ്നി.
\end{malayalam}}
\flushright{\begin{Arabic}
\quranayah[70][17]
\end{Arabic}}
\flushleft{\begin{malayalam}
പിന്നോക്കം മാറുകയും, തിരിഞ്ഞുകളയുകയും ചെയ്തവരെ അത് ക്ഷണിക്കും.
\end{malayalam}}
\flushright{\begin{Arabic}
\quranayah[70][18]
\end{Arabic}}
\flushleft{\begin{malayalam}
ശേഖരിച്ചു സൂക്ഷിച്ചു വെച്ചവരെയും.
\end{malayalam}}
\flushright{\begin{Arabic}
\quranayah[70][19]
\end{Arabic}}
\flushleft{\begin{malayalam}
തീര്‍ച്ചയായും മനുഷ്യന്‍ സൃഷ്ടിക്കപ്പെട്ടിരിക്കുന്നത് അങ്ങേ അറ്റം അക്ഷമനായിക്കൊണ്ടാണ്‌.
\end{malayalam}}
\flushright{\begin{Arabic}
\quranayah[70][20]
\end{Arabic}}
\flushleft{\begin{malayalam}
അതായത് തിന്‍മ ബാധിച്ചാല്‍ പൊറുതികേട് കാണിക്കുന്നവനായി കൊണ്ടും,
\end{malayalam}}
\flushright{\begin{Arabic}
\quranayah[70][21]
\end{Arabic}}
\flushleft{\begin{malayalam}
നന്‍മ കൈവന്നാല്‍ തടഞ്ഞു വെക്കുന്നവനായികൊണ്ടും.
\end{malayalam}}
\flushright{\begin{Arabic}
\quranayah[70][22]
\end{Arabic}}
\flushleft{\begin{malayalam}
നമസ്കരിക്കുന്നവരൊഴികെ -
\end{malayalam}}
\flushright{\begin{Arabic}
\quranayah[70][23]
\end{Arabic}}
\flushleft{\begin{malayalam}
അതായത് തങ്ങളുടെ നമസ്കാരത്തില്‍ സ്ഥിരമായി നിഷ്ഠയുള്ളവര്‍
\end{malayalam}}
\flushright{\begin{Arabic}
\quranayah[70][24]
\end{Arabic}}
\flushleft{\begin{malayalam}
തങ്ങളുടെ സ്വത്തുക്കളില്‍ നിര്‍ണിതമായ അവകാശം നല്‍കുന്നവരും,
\end{malayalam}}
\flushright{\begin{Arabic}
\quranayah[70][25]
\end{Arabic}}
\flushleft{\begin{malayalam}
ചോദിച്ചു വരുന്നവന്നും ഉപജീവനം തടയപ്പെട്ടവന്നും
\end{malayalam}}
\flushright{\begin{Arabic}
\quranayah[70][26]
\end{Arabic}}
\flushleft{\begin{malayalam}
പ്രതിഫലദിനത്തില്‍ വിശ്വസിക്കുന്നവരും,
\end{malayalam}}
\flushright{\begin{Arabic}
\quranayah[70][27]
\end{Arabic}}
\flushleft{\begin{malayalam}
തങ്ങളുടെ രക്ഷിതാവിന്‍റെ ശിക്ഷയെപറ്റി ഭയമുള്ളവരുമൊഴികെ.
\end{malayalam}}
\flushright{\begin{Arabic}
\quranayah[70][28]
\end{Arabic}}
\flushleft{\begin{malayalam}
തീര്‍ച്ചയായും അവരുടെ രക്ഷിതാവിന്‍റെ ശിക്ഷ (വരികയില്ലെന്ന്‌) സമാധാനപ്പെടാന്‍ പറ്റാത്തതാകുന്നു.
\end{malayalam}}
\flushright{\begin{Arabic}
\quranayah[70][29]
\end{Arabic}}
\flushleft{\begin{malayalam}
തങ്ങളുടെ ഗുഹ്യാവയവങ്ങള്‍ കാത്തുസൂക്ഷിക്കുന്നവരും (ഒഴികെ)
\end{malayalam}}
\flushright{\begin{Arabic}
\quranayah[70][30]
\end{Arabic}}
\flushleft{\begin{malayalam}
തങ്ങളുടെ ഭാര്യമാരുടെയോ, വലം കൈകള്‍ ഉടമപ്പെടുത്തിയവരുടെയോ കാര്യത്തിലൊഴികെ. തീര്‍ച്ചയായും അവര്‍ ആക്ഷേപമുക്തരാകുന്നു.
\end{malayalam}}
\flushright{\begin{Arabic}
\quranayah[70][31]
\end{Arabic}}
\flushleft{\begin{malayalam}
എന്നാല്‍ അതിലപ്പുറം ആരെങ്കിലും ആഗ്രഹിക്കുന്ന പക്ഷം അത്തരക്കാര്‍ തന്നെയാകുന്നു അതിരുകവിയുന്നവര്‍.
\end{malayalam}}
\flushright{\begin{Arabic}
\quranayah[70][32]
\end{Arabic}}
\flushleft{\begin{malayalam}
തങ്ങളെ വിശ്വസിച്ചേല്‍പിച്ച കാര്യങ്ങളും തങ്ങളുടെ ഉടമ്പടികളും പാലിച്ചു പോരുന്നവരും,
\end{malayalam}}
\flushright{\begin{Arabic}
\quranayah[70][33]
\end{Arabic}}
\flushleft{\begin{malayalam}
തങ്ങളുടെ സാക്ഷ്യങ്ങള്‍ മുറപ്രകാരം നിര്‍വഹിക്കുന്നവരും,
\end{malayalam}}
\flushright{\begin{Arabic}
\quranayah[70][34]
\end{Arabic}}
\flushleft{\begin{malayalam}
തങ്ങളുടെ നമസ്കാരങ്ങള്‍ നിഷ്ഠയോടെ നിര്‍വഹിക്കുന്നവരും (ഒഴികെ).
\end{malayalam}}
\flushright{\begin{Arabic}
\quranayah[70][35]
\end{Arabic}}
\flushleft{\begin{malayalam}
അത്തരക്കാര്‍ സ്വര്‍ഗത്തോപ്പുകളില്‍ ആദരിക്കപ്പെടുന്നവരാകുന്നു.
\end{malayalam}}
\flushright{\begin{Arabic}
\quranayah[70][36]
\end{Arabic}}
\flushleft{\begin{malayalam}
അപ്പോള്‍ സത്യനിഷേധികള്‍ക്കെന്തു പറ്റി! അവര്‍ നിന്‍റെ നേരെ കഴുത്തു നീട്ടി വന്നിട്ട്‌
\end{malayalam}}
\flushright{\begin{Arabic}
\quranayah[70][37]
\end{Arabic}}
\flushleft{\begin{malayalam}
വലത്തോട്ടും ഇടത്തോട്ടും കൂട്ടങ്ങളായി ചിതറിപോകുന്നു.
\end{malayalam}}
\flushright{\begin{Arabic}
\quranayah[70][38]
\end{Arabic}}
\flushleft{\begin{malayalam}
സുഖാനുഭൂതിയുടെ സ്വര്‍ഗത്തില്‍ താന്‍ പ്രവേശിപ്പിക്കപ്പെടണമെന്ന് അവരില്‍ ഓരോ മനുഷ്യനും മോഹിക്കുന്നുണ്ടോ?
\end{malayalam}}
\flushright{\begin{Arabic}
\quranayah[70][39]
\end{Arabic}}
\flushleft{\begin{malayalam}
അതു വേണ്ട. തീര്‍ച്ചയായും അവര്‍ക്കറിയാവുന്നതില്‍ നിന്നാണ് അവരെ നാം സൃഷ്ടിച്ചിട്ടുള്ളത്‌
\end{malayalam}}
\flushright{\begin{Arabic}
\quranayah[70][40]
\end{Arabic}}
\flushleft{\begin{malayalam}
എന്നാല്‍ ഉദയസ്ഥാനങ്ങളുടെയും അസ്തമയസ്ഥാനങ്ങളുടെയും രക്ഷിതാവിന്‍റെ പേരില്‍ ഞാന്‍ സത്യം ചെയ്തു പറയുന്നു: തീര്‍ച്ചയായും നാം കഴിവുള്ളനാണെന്ന്‌.
\end{malayalam}}
\flushright{\begin{Arabic}
\quranayah[70][41]
\end{Arabic}}
\flushleft{\begin{malayalam}
അവരെക്കാള്‍ നല്ലവരെ പകരം കൊണ്ടു വരാന്‍. നാം തോല്‍പിക്കപ്പെടുന്നവനല്ല താനും.
\end{malayalam}}
\flushright{\begin{Arabic}
\quranayah[70][42]
\end{Arabic}}
\flushleft{\begin{malayalam}
ആകയാല്‍ അവര്‍ക്ക് താക്കീത് നല്‍കപ്പെടുന്ന അവരുടെ ആ ദിവസത്തെ അവര്‍ കണ്ടുമുട്ടുന്നത് വരെ അവര്‍ തോന്നിവാസത്തില്‍ മുഴുകുകയും കളിച്ചു കൊണ്ടിരിക്കുകയും ചെയ്യാന്‍ നീ അവരെ വിട്ടേക്കുക.
\end{malayalam}}
\flushright{\begin{Arabic}
\quranayah[70][43]
\end{Arabic}}
\flushleft{\begin{malayalam}
അതായത് അവര്‍ ഒരു നാട്ടക്കുറിയുടെ നേരെ ധൃതിപ്പെട്ട് പോകുന്നത് പോലെ ഖബ്‌റുകളില്‍ നിന്ന് പുറപ്പെട്ടു പോകുന്ന ദിവസം.
\end{malayalam}}
\flushright{\begin{Arabic}
\quranayah[70][44]
\end{Arabic}}
\flushleft{\begin{malayalam}
അവരുടെ കണ്ണുകള്‍ കീഴ്പോട്ട് താണിരിക്കും. അപമാനം അവരെ ആവരണം ചെയ്തിരിക്കും. അതാണ് അവര്‍ക്ക് താക്കീത് നല്‍കപ്പെട്ടിരുന്ന ദിവസം.
\end{malayalam}}
\chapter{\textmalayalam{നൂഹ്}}
\begin{Arabic}
\Huge{\centerline{\basmalah}}\end{Arabic}
\flushright{\begin{Arabic}
\quranayah[71][1]
\end{Arabic}}
\flushleft{\begin{malayalam}
തീര്‍ച്ചയായും നൂഹിനെ അദ്ദേഹത്തിന്‍റെ ജനതയിലേക്ക് നാം അയച്ചു. നിന്‍റെ ജനതയ്ക്ക് വേദനയേറിയ ശിക്ഷ വരുന്നതിന്‍റെ മുമ്പ് അവര്‍ക്ക് താക്കീത് നല്‍കുക എന്ന് നിര്‍ദേശിച്ചു കൊണ്ട്‌
\end{malayalam}}
\flushright{\begin{Arabic}
\quranayah[71][2]
\end{Arabic}}
\flushleft{\begin{malayalam}
അദ്ദേഹം പറഞ്ഞു: എന്‍റെ ജനങ്ങളെ, തീര്‍ച്ചയായും ഞാന്‍ നിങ്ങള്‍ക്കു വ്യക്തമായ താക്കീതുകാരനാകുന്നു.
\end{malayalam}}
\flushright{\begin{Arabic}
\quranayah[71][3]
\end{Arabic}}
\flushleft{\begin{malayalam}
നിങ്ങള്‍ അല്ലാഹുവെ ആരാധിക്കുകയും അവനെ സൂക്ഷിക്കുകയും എന്നെ അനുസരിക്കുകയും ചെയ്യുവിന്‍.
\end{malayalam}}
\flushright{\begin{Arabic}
\quranayah[71][4]
\end{Arabic}}
\flushleft{\begin{malayalam}
എങ്കില്‍ അവന്‍ നിങ്ങള്‍ക്കു നിങ്ങളുടെ പാപങ്ങളില്‍ ചിലത് പൊറുത്തുതരികയും, നിര്‍ണയിക്കപ്പെട്ട ഒരു അവധി വരെ നിങ്ങളെ നീട്ടിയിടുകയും ചെയ്യുന്നതാണ്‌. തീര്‍ച്ചയായും അല്ലാഹുവിന്‍റെ അവധി വന്നാല്‍ അത് നീട്ടി കൊടുക്കപ്പെടുകയില്ല. നിങ്ങള്‍ അറിഞ്ഞിരുന്നെങ്കില്‍.
\end{malayalam}}
\flushright{\begin{Arabic}
\quranayah[71][5]
\end{Arabic}}
\flushleft{\begin{malayalam}
അദ്ദേഹം പറഞ്ഞു: എന്‍റെ രക്ഷിതാവേ, തീര്‍ച്ചയായും എന്‍റെ ജനതയെ രാവും പകലും ഞാന്‍ വിളിച്ചു.
\end{malayalam}}
\flushright{\begin{Arabic}
\quranayah[71][6]
\end{Arabic}}
\flushleft{\begin{malayalam}
എന്നിട്ട് എന്‍റെ വിളി അവരുടെ ഓടിപ്പോക്ക് വര്‍ദ്ധിപ്പിക്കുക മാത്രമേ ചെയ്തുള്ളു.
\end{malayalam}}
\flushright{\begin{Arabic}
\quranayah[71][7]
\end{Arabic}}
\flushleft{\begin{malayalam}
തീര്‍ച്ചയായും, നീ അവര്‍ക്ക് പൊറുത്തുകൊടുക്കുവാന്‍ വേണ്ടി ഞാന്‍ അവരെ വിളിച്ചപ്പോഴൊക്കെയും അവര്‍ അവരുടെ വിരലുകള്‍ കാതുകളില്‍ വെക്കുകയും, അവരുടെ വസ്ത്രങ്ങള്‍ മൂടിപ്പുതക്കുകയും, അവര്‍ ശഠിച്ചു നില്‍ക്കുകയും, കടുത്ത അഹങ്കാരം നടിക്കുകയുമാണ് ചെയ്തത്‌.
\end{malayalam}}
\flushright{\begin{Arabic}
\quranayah[71][8]
\end{Arabic}}
\flushleft{\begin{malayalam}
പിന്നീട് അവരെ ഞാന്‍ ഉറക്കെ വിളിച്ചു.
\end{malayalam}}
\flushright{\begin{Arabic}
\quranayah[71][9]
\end{Arabic}}
\flushleft{\begin{malayalam}
പിന്നീട് ഞാന്‍ അവരോട് പരസ്യമായും വളരെ രഹസ്യമായും പ്രബോധനം നടത്തി.
\end{malayalam}}
\flushright{\begin{Arabic}
\quranayah[71][10]
\end{Arabic}}
\flushleft{\begin{malayalam}
അങ്ങനെ ഞാന്‍ പറഞ്ഞു: നിങ്ങള്‍ നിങ്ങളുടെ രക്ഷിതാവിനോട് പാപമോചനം തേടുക. തീര്‍ച്ചയായും അവന്‍ ഏറെ പൊറുക്കുന്നവനാകുന്നു.
\end{malayalam}}
\flushright{\begin{Arabic}
\quranayah[71][11]
\end{Arabic}}
\flushleft{\begin{malayalam}
അവന്‍ നിങ്ങള്‍ക്ക് മഴ സമൃദ്ധമായി അയച്ചുതരും.
\end{malayalam}}
\flushright{\begin{Arabic}
\quranayah[71][12]
\end{Arabic}}
\flushleft{\begin{malayalam}
സ്വത്തുക്കളും സന്താനങ്ങളും കൊണ്ട് നിങ്ങളെ അവന്‍ പോഷിപ്പിക്കുകയും, നിങ്ങള്‍ക്കവന്‍ തോട്ടങ്ങള്‍ ഉണ്ടാക്കിത്തരികയും നിങ്ങള്‍ക്കവന്‍ അരുവികള്‍ ഉണ്ടാക്കിത്തരികയും ചെയ്യും.
\end{malayalam}}
\flushright{\begin{Arabic}
\quranayah[71][13]
\end{Arabic}}
\flushleft{\begin{malayalam}
നിങ്ങള്‍ക്കെന്തു പറ്റി? അല്ലാഹുവിന് ഒരു ഗാംഭീര്യവും നിങ്ങള്‍ പ്രതീക്ഷിക്കുന്നില്ല.
\end{malayalam}}
\flushright{\begin{Arabic}
\quranayah[71][14]
\end{Arabic}}
\flushleft{\begin{malayalam}
നിങ്ങളെ അവന്‍ പല ഘട്ടങ്ങളിലായി സൃഷ്ടിച്ചിരിക്കുകയാണല്ലോ.
\end{malayalam}}
\flushright{\begin{Arabic}
\quranayah[71][15]
\end{Arabic}}
\flushleft{\begin{malayalam}
നിങ്ങള്‍ കണ്ടില്ലേ; എങ്ങനെയാണ് അല്ലാഹു അടുക്കുകളായിട്ട് ഏഴ് ആകാശങ്ങള്‍ സൃഷ്ടിച്ചിരിക്കുന്നത് എന്ന്‌.
\end{malayalam}}
\flushright{\begin{Arabic}
\quranayah[71][16]
\end{Arabic}}
\flushleft{\begin{malayalam}
ചന്ദ്രനെ അവിടെ ഒരു പ്രകാശമാക്കിയിരിക്കുന്നു.സൂര്യനെ ഒരു വിളക്കുമാക്കിയിരിക്കുന്നു.
\end{malayalam}}
\flushright{\begin{Arabic}
\quranayah[71][17]
\end{Arabic}}
\flushleft{\begin{malayalam}
അല്ലാഹു നിങ്ങളെ ഭൂമിയില്‍ നിന്ന് ഒരു മുളപ്പിക്കല്‍ മുളപ്പിച്ചിരിക്കുന്നു.
\end{malayalam}}
\flushright{\begin{Arabic}
\quranayah[71][18]
\end{Arabic}}
\flushleft{\begin{malayalam}
പിന്നെ അതില്‍ തന്നെ നിങ്ങളെ അവന്‍ മടക്കുകയും നിങ്ങളെ ഒരിക്കല്‍ അവന്‍ പുറത്തു കൊണ്ട് വരികയും ചെയ്യുന്നതാണ്‌.
\end{malayalam}}
\flushright{\begin{Arabic}
\quranayah[71][19]
\end{Arabic}}
\flushleft{\begin{malayalam}
അല്ലാഹു നിങ്ങള്‍ക്കു വേണ്ടി ഭൂമിയെ ഒരു വിരിപ്പാക്കുകയും ചെയ്തിരിക്കുന്നു.
\end{malayalam}}
\flushright{\begin{Arabic}
\quranayah[71][20]
\end{Arabic}}
\flushleft{\begin{malayalam}
അതിലെ വിസ്താരമുള്ള പാതകളില്‍ നിങ്ങള്‍ പ്രവേശിക്കുവാന്‍ വേണ്ടി.
\end{malayalam}}
\flushright{\begin{Arabic}
\quranayah[71][21]
\end{Arabic}}
\flushleft{\begin{malayalam}
നൂഹ് പറഞ്ഞു: എന്‍റെ രക്ഷിതാവേ, തീര്‍ച്ചയായും ഇവര്‍ എന്നോട് അനുസരണക്കേട് കാണിക്കുകയും ഒരു വിഭാഗത്തെ പിന്തുടരുകയും ചെയ്തിരിക്കുന്നു; അവര്‍ക്ക് (പിന്തുടരപ്പെട്ട നേതൃവിഭാഗത്തിന്‌) സ്വത്തും സന്താനവും മൂലം (ആത്മീയവും പാരത്രികവുമായ) നഷ്ടം കൂടുക മാത്രമാണുണ്ടായത്‌.
\end{malayalam}}
\flushright{\begin{Arabic}
\quranayah[71][22]
\end{Arabic}}
\flushleft{\begin{malayalam}
(പുറമെ) അവര്‍ (നേതാക്കള്‍) വലിയ കുതന്ത്രം പ്രയോഗിക്കുകയും ചെയ്തിരിക്കുന്നു.
\end{malayalam}}
\flushright{\begin{Arabic}
\quranayah[71][23]
\end{Arabic}}
\flushleft{\begin{malayalam}
അവര്‍ പറഞ്ഞു: (ജനങ്ങളേ,) നിങ്ങള്‍ നിങ്ങളുടെ ദൈവങ്ങളെ ഉപേക്ഷിക്കരുത്‌. വദ്ദ്‌, സുവാഅ്‌, യഗൂഥ്‌, യഊഖ്‌, നസ്‌റ് എന്നിവരെ നിങ്ങള്‍ ഉപേക്ഷിക്കരുത്‌.
\end{malayalam}}
\flushright{\begin{Arabic}
\quranayah[71][24]
\end{Arabic}}
\flushleft{\begin{malayalam}
അങ്ങനെ അവര്‍ വളരെയധികം ആളുകളെ വഴിപിഴപ്പിച്ചു. (രക്ഷിതാവേ,) ആ അക്രമകാരികള്‍ക്ക് വഴിപിഴവല്ലാതെ മറ്റൊന്നും നീ വര്‍ദ്ധിപ്പിക്കരുതേ.
\end{malayalam}}
\flushright{\begin{Arabic}
\quranayah[71][25]
\end{Arabic}}
\flushleft{\begin{malayalam}
അവരുടെ പാപങ്ങള്‍ നിമിത്തം അവര്‍ മുക്കി നശിപ്പിക്കപ്പെട്ടു. എന്നിട്ടവര്‍ നരകാഗ്നിയില്‍ പ്രവേശിപ്പിക്കപ്പെട്ടു. അപ്പോള്‍ തങ്ങള്‍ക്ക് അല്ലാഹുവിനു പുറമെ സഹായികളാരെയും അവര്‍ കണ്ടെത്തിയില്ല.
\end{malayalam}}
\flushright{\begin{Arabic}
\quranayah[71][26]
\end{Arabic}}
\flushleft{\begin{malayalam}
നൂഹ് പറഞ്ഞു.: എന്‍റെ രക്ഷിതാവേ, ഭൂമുഖത്ത് സത്യനിഷേധികളില്‍ പെട്ട ഒരു പൌരനെയും നീ വിട്ടേക്കരുതേ.
\end{malayalam}}
\flushright{\begin{Arabic}
\quranayah[71][27]
\end{Arabic}}
\flushleft{\begin{malayalam}
തീര്‍ച്ചയായും നീ അവരെ വിട്ടേക്കുകയാണെങ്കില്‍ നിന്‍റെ ദാസന്‍മാരെ അവര്‍ പിഴപ്പിച്ചു കളയും. ദുര്‍വൃത്തന്നും സത്യനിഷേധിക്കുമല്ലാതെ അവര്‍ ജന്‍മം നല്‍കുകയുമില്ല.
\end{malayalam}}
\flushright{\begin{Arabic}
\quranayah[71][28]
\end{Arabic}}
\flushleft{\begin{malayalam}
എന്‍റെ രക്ഷിതാവേ, എന്‍റെ മാതാപിതാക്കള്‍ക്കും എന്‍റെ വീട്ടില്‍ വിശ്വാസിയായിക്കൊണ്ട് പ്രവേശിച്ചവന്നും സത്യവിശ്വാസികള്‍ക്കും സത്യവിശ്വാസിനികള്‍ക്കും സത്യവിശ്വാസിനികള്‍ക്കും നീ പൊറുത്തുതരേണമേ. അക്രമകാരികള്‍ക്ക് നാശമല്ലാതൊന്നും നീ വര്‍ദ്ധിപ്പിക്കരുതേ.
\end{malayalam}}
\chapter{\textmalayalam{ജിന്ന് ( ജിന്ന് വര്‍ഗ്ഗം )}}
\begin{Arabic}
\Huge{\centerline{\basmalah}}\end{Arabic}
\flushright{\begin{Arabic}
\quranayah[72][1]
\end{Arabic}}
\flushleft{\begin{malayalam}
നബിയേ,) പറയുക: ജിന്നുകളില്‍ നിന്നുള്ള ഒരു സംഘം ഖുര്‍ആന്‍ ശ്രദ്ധിച്ചു കേള്‍ക്കുകയുണ്ടായി എന്ന് എനിക്ക് ദിവ്യബോധനം നല്‍കപ്പെട്ടിരിക്കുന്നു. എന്നിട്ടവര്‍ (സ്വന്തം സമൂഹത്തോട്‌) പറഞ്ഞു: തീര്‍ച്ചയായും അത്ഭുതകരമായ ഒരു ഖുര്‍ആന്‍ ഞങ്ങള്‍ കേട്ടിരിക്കുന്നു.
\end{malayalam}}
\flushright{\begin{Arabic}
\quranayah[72][2]
\end{Arabic}}
\flushleft{\begin{malayalam}
അത് സന്‍മാര്‍ഗത്തിലേക്ക് വഴി കാണിക്കുന്നു. അതു കൊണ്ട് ഞങ്ങള്‍ അതില്‍ വിശ്വസിച്ചു. മേലില്‍ ഞങ്ങളുടെ രക്ഷിതാവിനോട് ആരെയും ഞങ്ങള്‍ പങ്കുചേര്‍ക്കുകയേ ഇല്ല.
\end{malayalam}}
\flushright{\begin{Arabic}
\quranayah[72][3]
\end{Arabic}}
\flushleft{\begin{malayalam}
നമ്മുടെ രക്ഷിതാവിന്‍റെ മഹത്വം ഉന്നതമാകുന്നു. അവന്‍ കൂട്ടുകാരിയെയോ, സന്താനത്തെയോ സ്വീകരിച്ചിട്ടില്ല.
\end{malayalam}}
\flushright{\begin{Arabic}
\quranayah[72][4]
\end{Arabic}}
\flushleft{\begin{malayalam}
ഞങ്ങളിലുള്ള വിഡ്ഢികള്‍ അല്ലാഹുവെപറ്റി അതിക്രമപരമായ പരാമര്‍ശം നടത്തുമായിരുന്നു.
\end{malayalam}}
\flushright{\begin{Arabic}
\quranayah[72][5]
\end{Arabic}}
\flushleft{\begin{malayalam}
ഞങ്ങള്‍ വിചാരിച്ചു; മനുഷ്യരും ജിന്നുകളും അല്ലാഹുവിന്‍റെ പേരില്‍ ഒരിക്കലും കള്ളം പറയുകയില്ലെന്ന്‌. എന്നും (അവര്‍ പറഞ്ഞു.)
\end{malayalam}}
\flushright{\begin{Arabic}
\quranayah[72][6]
\end{Arabic}}
\flushleft{\begin{malayalam}
മനുഷ്യരില്‍പെട്ട ചില വ്യക്തികള്‍ ജിന്നുകളില്‍ പെട്ട വ്യക്തികളോട് ശരണം തേടാറുണ്ടായിരുന്നു. അങ്ങനെ അതവര്‍ക്ക് (ജിന്നുകള്‍ക്ക്‌) ഗര്‍വ്വ് വര്‍ദ്ധിപ്പിച്ചു.
\end{malayalam}}
\flushright{\begin{Arabic}
\quranayah[72][7]
\end{Arabic}}
\flushleft{\begin{malayalam}
നിങ്ങള്‍ ധരിച്ചത് പോലെ അവരും ധരിച്ചു; അല്ലാഹു ആരെയും ഉയിര്‍ത്തെഴുന്നേല്‍പിക്കുകയില്ലെന്ന് എന്നും (അവര്‍ പറഞ്ഞു.)
\end{malayalam}}
\flushright{\begin{Arabic}
\quranayah[72][8]
\end{Arabic}}
\flushleft{\begin{malayalam}
ഞങ്ങള്‍ ആകാശത്തെ സ്പര്‍ശിച്ചു നോക്കി. അപ്പോള്‍ അത് ശക്തിമത്തായ പാറാവുകാരാലും തീജ്വാലകളാലും നിറക്കപ്പെട്ടതായി ഞങ്ങള്‍ കണ്ടെത്തി എന്നും (അവര്‍ പറഞ്ഞു.)
\end{malayalam}}
\flushright{\begin{Arabic}
\quranayah[72][9]
\end{Arabic}}
\flushleft{\begin{malayalam}
(ആകാശത്തിലെ) ചില ഇരിപ്പിടങ്ങളില്‍ ഞങ്ങള്‍ കേള്‍ക്കാന്‍ വേണ്ടി ഇരിക്കാറുണ്ടായിരുന്നു. എന്നാല്‍ ഇപ്പോള്‍ ആരെങ്കിലും ശ്രദ്ധിച്ചു കേള്‍ക്കുകയാണെങ്കില്‍ കാത്തിരിക്കുന്ന അഗ്നിജ്വാലയെ അവന്ന് കണ്ടെത്താനാവും. എന്നും (അവര്‍ പറഞ്ഞു.)
\end{malayalam}}
\flushright{\begin{Arabic}
\quranayah[72][10]
\end{Arabic}}
\flushleft{\begin{malayalam}
ഭൂമിയിലുള്ളവരുടെ കാര്യത്തില്‍ തിന്‍മയാണോ, ഉദ്ദേശിക്കപ്പെട്ടിട്ടുള്ളത്‌, അതല്ല അവരുടെ രക്ഷിതാവ് അവരെ നേര്‍വഴിയിലാക്കാന്‍ ഉദ്ദേശിച്ചിരിക്കുകയാണോ എന്ന് ഞങ്ങള്‍ക്ക് അറിഞ്ഞ് കൂടാ.
\end{malayalam}}
\flushright{\begin{Arabic}
\quranayah[72][11]
\end{Arabic}}
\flushleft{\begin{malayalam}
ഞങ്ങളാകട്ടെ, ഞങ്ങളുടെ കൂട്ടത്തില്‍ സദ്‌വൃത്തന്‍മാരുണ്ട്‌. അതില്‍ താഴെയുള്ളവരും ഞങ്ങളുടെ കൂട്ടത്തിലുണ്ട്‌. ഞങ്ങള്‍ വിഭിന്ന മാര്‍ഗങ്ങളായിതീര്‍ന്നിരിക്കുന്നു. എന്നും (അവര്‍ പറഞ്ഞു.)
\end{malayalam}}
\flushright{\begin{Arabic}
\quranayah[72][12]
\end{Arabic}}
\flushleft{\begin{malayalam}
ഭൂമിയില്‍ വെച്ച് അല്ലാഹുവെ ഞങ്ങള്‍ക്ക് തോല്‍പിക്കാനാവില്ല എന്നും, ഓടി മാറിക്കളഞ്ഞിട്ട് അവനെ തോല്‍പിക്കാനാവില്ലെന്നും ഞങ്ങള്‍ ധരിച്ചിരിക്കുന്നു
\end{malayalam}}
\flushright{\begin{Arabic}
\quranayah[72][13]
\end{Arabic}}
\flushleft{\begin{malayalam}
സന്‍മാര്‍ഗം കേട്ടപ്പോള്‍ ഞങ്ങള്‍ അതില്‍ വിശ്വസിച്ചിരിക്കുന്നു. അപ്പോള്‍ ഏതൊരുത്തന്‍ തന്‍റെ രക്ഷിതാവില്‍ വിശ്വസിക്കുന്നുവോ അവന്‍ യാതൊരു നഷ്ടത്തെയും അനീതിയെയും പറ്റി ഭയപ്പെടേണ്ടി വരില്ല. എന്നും (അവര്‍ പറഞ്ഞു.)
\end{malayalam}}
\flushright{\begin{Arabic}
\quranayah[72][14]
\end{Arabic}}
\flushleft{\begin{malayalam}
ഞങ്ങളുടെ കൂട്ടത്തില്‍ കീഴ്പെട്ടു ജീവിക്കുന്നവരുണ്ട്‌. അനീതി പ്രവര്‍ത്തിക്കുന്നവരും ഞങ്ങളുടെ കൂട്ടത്തിലുണ്ട്‌. എന്നാല്‍ ആര്‍ കീഴ്പെട്ടിരിക്കുന്നുവോ അത്തരക്കാര്‍ സന്‍മാര്‍ഗം അവലംബിച്ചിരിക്കുന്നു.
\end{malayalam}}
\flushright{\begin{Arabic}
\quranayah[72][15]
\end{Arabic}}
\flushleft{\begin{malayalam}
അനീതി പ്രവര്‍ത്തിച്ചകരാകട്ടെ നരകത്തിനുള്ള വിറക് ആയി തീരുന്നതാണ്‌. (എന്നും അവര്‍ പറഞ്ഞു.)
\end{malayalam}}
\flushright{\begin{Arabic}
\quranayah[72][16]
\end{Arabic}}
\flushleft{\begin{malayalam}
ആ മാര്‍ഗത്തില്‍ (ഇസ്ലാമില്‍) അവര്‍ നേരെ നിലകൊള്ളുകയാണെങ്കില്‍ നാം അവര്‍ക്ക് ധാരാളമായി വെള്ളം കുടിക്കാന്‍ നല്‍കുന്നതാണ്‌.
\end{malayalam}}
\flushright{\begin{Arabic}
\quranayah[72][17]
\end{Arabic}}
\flushleft{\begin{malayalam}
അതിലൂടെ നാം അവരെ പരീക്ഷിക്കുവാന്‍ വേണ്ടിയത്രെ അത്‌. തന്‍റെ രക്ഷിതാവിന്‍റെ ഉല്‍ബോധനത്തെ വിട്ട് ആര്‍ തിരിഞ്ഞുകളയുന്നുവോ അവനെ അവന്‍ (രക്ഷിതാവ്‌) പ്രയാസകരമായ ശിക്ഷയില്‍ പ്രവേശിപ്പിക്കുന്നതാണ്‌. (എന്നും എനിക്ക് ബോധനം നല്‍കപ്പെട്ടിരിക്കുന്നു.)
\end{malayalam}}
\flushright{\begin{Arabic}
\quranayah[72][18]
\end{Arabic}}
\flushleft{\begin{malayalam}
പള്ളികള്‍ അല്ലാഹുവിന്നുള്ളതാകുന്നു. അതിനാല്‍ നിങ്ങള്‍ അല്ലാഹുവോടൊപ്പം ആരെയും വിളിച്ച് പ്രാര്‍ത്ഥിക്കരുത് എന്നും.
\end{malayalam}}
\flushright{\begin{Arabic}
\quranayah[72][19]
\end{Arabic}}
\flushleft{\begin{malayalam}
അല്ലാഹുവിന്‍റെ ദാസന്‍ (നബി) അവനോട് പ്രാര്‍ത്ഥിക്കുവാനായി എഴുന്നേറ്റ് നിന്നപ്പോള്‍ അവര്‍ അദ്ദേഹത്തിന് ചുറ്റും തിങ്ങിക്കൂടുവാനൊരുങ്ങി എന്നും.
\end{malayalam}}
\flushright{\begin{Arabic}
\quranayah[72][20]
\end{Arabic}}
\flushleft{\begin{malayalam}
(നബിയേ,)പറയുക: ഞാന്‍ എന്‍റെ രക്ഷിതാവിനെ മാത്രമേ വിളിച്ചു പ്രാര്‍ത്ഥിക്കുകയുള്ളൂ. അവനോട് യാതൊരാളെയും ഞാന്‍ പങ്കുചേര്‍ക്കുകയില്ല.
\end{malayalam}}
\flushright{\begin{Arabic}
\quranayah[72][21]
\end{Arabic}}
\flushleft{\begin{malayalam}
പറയുക: നിങ്ങള്‍ക്ക് ഉപദ്രവം ചെയ്യുക എന്നതോ (നിങ്ങളെ) നേര്‍വഴിയിലാക്കുക എന്നതോ എന്‍റെ അധീനതയിലല്ല.
\end{malayalam}}
\flushright{\begin{Arabic}
\quranayah[72][22]
\end{Arabic}}
\flushleft{\begin{malayalam}
പറയുക: അല്ലാഹുവി (ന്‍റെ ശിക്ഷയി) ല്‍ നിന്ന് ഒരാളും എനിക്ക് അഭയം നല്‍കുകയേ ഇല്ല; തീര്‍ച്ചയായും അവന്നു പുറമെ ഒരു അഭയസ്ഥാനവും ഞാന്‍ ഒരിക്കലും കണ്ടെത്തുകയുമില്ല.
\end{malayalam}}
\flushright{\begin{Arabic}
\quranayah[72][23]
\end{Arabic}}
\flushleft{\begin{malayalam}
അല്ലാഹുവിങ്കല്‍ നിന്നുള്ള പ്രബോധനവും അവന്‍റെ സന്ദേശങ്ങളും ഒഴികെ (മറ്റൊന്നും എന്‍റെ അധീനതയിലില്ല.) വല്ലവനും അല്ലാഹുവെയും അവന്‍റെ ദൂതനെയും ധിക്കരിക്കുന്ന പക്ഷം തീര്‍ച്ചയായും അവന്നുള്ളതാണ് നരകാഗ്നി. അത്തരക്കാര്‍ അതില്‍ നിത്യവാസികളായിരിക്കും.
\end{malayalam}}
\flushright{\begin{Arabic}
\quranayah[72][24]
\end{Arabic}}
\flushleft{\begin{malayalam}
അങ്ങനെ അവര്‍ക്ക് താക്കീത് നല്‍കപ്പെടുന്ന കാര്യം അവര്‍ കണ്ടു കഴിഞ്ഞാല്‍ ഏറ്റവും ദുര്‍ബലനായ സഹായി ആരാണെന്നും എണ്ണത്തില്‍ ഏറ്റവും കുറവ് ആരാണെന്നും അവര്‍ മനസ്സിലാക്കികൊള്ളും.
\end{malayalam}}
\flushright{\begin{Arabic}
\quranayah[72][25]
\end{Arabic}}
\flushleft{\begin{malayalam}
(നബിയേ,) പറയുക: നിങ്ങള്‍ക്ക് താക്കീത് നല്‍കപ്പെടുന്ന കാര്യം അടുത്തു തന്നെയാണോ അതല്ല എന്‍റെ രക്ഷിതാവ് അതിന് അവധി വെച്ചേക്കുമോ എന്ന് എനിക്ക് അറിയില്ല.
\end{malayalam}}
\flushright{\begin{Arabic}
\quranayah[72][26]
\end{Arabic}}
\flushleft{\begin{malayalam}
അവന്‍ അദൃശ്യം അറിയുന്നവനാണ്‌. എന്നാല്‍ അവന്‍ തന്‍റെ അദൃശ്യജ്ഞാനം യാതൊരാള്‍ക്കും വെളിപ്പെടുത്തി കൊടുക്കുകയില്ല.
\end{malayalam}}
\flushright{\begin{Arabic}
\quranayah[72][27]
\end{Arabic}}
\flushleft{\begin{malayalam}
അവന്‍ തൃപ്തിപ്പെട്ട വല്ല ദൂതന്നുമല്ലാതെ. എന്നാല്‍ അദ്ദേഹത്തിന്‍റെ (ദൂതന്‍റെ) മുന്നിലും പിന്നിലും അവന്‍ കാവല്‍ക്കാരെ ഏര്‍പെടുത്തുക തന്നെ ചെയ്യുന്നതാണ്‌.
\end{malayalam}}
\flushright{\begin{Arabic}
\quranayah[72][28]
\end{Arabic}}
\flushleft{\begin{malayalam}
അവര്‍ (ദൂതന്‍മാര്‍) തങ്ങളുടെ രക്ഷിതാവിന്‍റെ ദൌത്യങ്ങള്‍ എത്തിച്ചുകൊടുത്തിട്ടുണ്ട് എന്ന് അവന്‍ (അല്ലാഹു) അറിയാന്‍ വേണ്ടി. അവരുടെ പക്കലുള്ളതിനെ അവന്‍ പരിപൂര്‍ണ്ണമായി അറിഞ്ഞിരിക്കുന്നു. എല്ലാ വസ്തുവിന്‍റെയും എണ്ണം അവന്‍ തിട്ടപ്പെടുത്തിയിരിക്കുന്നു.
\end{malayalam}}
\chapter{\textmalayalam{മുസമ്മില്‍ ( വസ്ത്രത്താല്‍ മൂടിയവന്‍ )}}
\begin{Arabic}
\Huge{\centerline{\basmalah}}\end{Arabic}
\flushright{\begin{Arabic}
\quranayah[73][1]
\end{Arabic}}
\flushleft{\begin{malayalam}
ഹേ, വസ്ത്രം കൊണ്ട് മൂടിയവനേ,
\end{malayalam}}
\flushright{\begin{Arabic}
\quranayah[73][2]
\end{Arabic}}
\flushleft{\begin{malayalam}
രാത്രി അല്‍പസമയം ഒഴിച്ച് എഴുന്നേറ്റ് നിന്ന് പ്രാര്‍ത്ഥിക്കുക.
\end{malayalam}}
\flushright{\begin{Arabic}
\quranayah[73][3]
\end{Arabic}}
\flushleft{\begin{malayalam}
അതിന്‍റെ (രാത്രിയുടെ) പകുതി, അല്ലെങ്കില്‍ അതില്‍ നിന്നു (അല്‍പം) കുറച്ചു കൊള്ളുക.
\end{malayalam}}
\flushright{\begin{Arabic}
\quranayah[73][4]
\end{Arabic}}
\flushleft{\begin{malayalam}
അല്ലെങ്കില്‍ അതിനെക്കാള്‍ വര്‍ദ്ധിപ്പിച്ചു കൊള്ളുക. ഖുര്‍ആന്‍ സാവകാശത്തില്‍ പാരായണം നടത്തുകയും ചെയ്യുക.
\end{malayalam}}
\flushright{\begin{Arabic}
\quranayah[73][5]
\end{Arabic}}
\flushleft{\begin{malayalam}
തീര്‍ച്ചയായും നാം നിന്‍റെ മേല്‍ ഒരു കനപ്പെട്ട വാക്ക് ഇട്ടുതരുന്നതാണ്‌.
\end{malayalam}}
\flushright{\begin{Arabic}
\quranayah[73][6]
\end{Arabic}}
\flushleft{\begin{malayalam}
തീര്‍ച്ചയായും രാത്രിയില്‍ എഴുന്നേറ്റു നമസ്കരിക്കല്‍ കൂടുതല്‍ ശക്തമായ ഹൃദയസാന്നിദ്ധ്യം നല്‍കുന്നതും വാക്കിനെ കൂടുതല്‍ നേരെ നിര്‍ത്തുന്നതുമാകുന്നു.
\end{malayalam}}
\flushright{\begin{Arabic}
\quranayah[73][7]
\end{Arabic}}
\flushleft{\begin{malayalam}
തീര്‍ച്ചയായും നിനക്ക് പകല്‍ സമയത്ത് ദീര്‍ഘമായ ജോലിത്തിരക്കുണ്ട്‌.
\end{malayalam}}
\flushright{\begin{Arabic}
\quranayah[73][8]
\end{Arabic}}
\flushleft{\begin{malayalam}
നിന്‍റെ രക്ഷിതാവിന്‍റെ നാമം സ്മരിക്കുകയും, (മറ്റു ചിന്തകള്‍ വെടിഞ്ഞ്‌) അവങ്കലേങ്കു മാത്രമായി മടങ്ങുകയും ചെയ്യുക.
\end{malayalam}}
\flushright{\begin{Arabic}
\quranayah[73][9]
\end{Arabic}}
\flushleft{\begin{malayalam}
ഉദയസ്ഥാനത്തിന്‍റെയും, അസ്തമനസ്ഥാനത്തിന്‍റെയും രക്ഷിതാവാകുന്നു അവന്‍. അവനല്ലാതെ യാതൊരു ദൈവവുമില്ല. അതിനാല്‍ ഭരമേല്‍പിക്കപ്പെടേണ്ടവനായി അവനെ സ്വീകരിക്കുക.
\end{malayalam}}
\flushright{\begin{Arabic}
\quranayah[73][10]
\end{Arabic}}
\flushleft{\begin{malayalam}
അവര്‍ (അവിശ്വാസികള്‍) പറയുന്നതിനെപ്പറ്റി നീ ക്ഷമിക്കുകയും, ഭംഗിയായ വിധത്തില്‍ അവരില്‍ നിന്ന് ഒഴിഞ്ഞു നില്‍ക്കുകയും ചെയ്യുക.
\end{malayalam}}
\flushright{\begin{Arabic}
\quranayah[73][11]
\end{Arabic}}
\flushleft{\begin{malayalam}
എന്നെയും, സുഖാനുഗ്രഹങ്ങള്‍ ഉള്ളവരായ സത്യനിഷേധികളെയും വിട്ടേക്കുക. അവര്‍ക്കു അല്‍പം ഇടകൊടുക്കുകയും ചെയ്യുക.
\end{malayalam}}
\flushright{\begin{Arabic}
\quranayah[73][12]
\end{Arabic}}
\flushleft{\begin{malayalam}
തീര്‍ച്ചയായും നമ്മുടെ അടുക്കല്‍ കാല്‍ ചങ്ങലകളും ജ്വലിക്കുന്ന നരകാഗ്നിയും
\end{malayalam}}
\flushright{\begin{Arabic}
\quranayah[73][13]
\end{Arabic}}
\flushleft{\begin{malayalam}
തൊണ്ടയില്‍ അടഞ്ഞു നില്‍ക്കുന്ന ഭക്ഷണവും വേദനയേറിയ ശിക്ഷയുമുണ്ട്‌.
\end{malayalam}}
\flushright{\begin{Arabic}
\quranayah[73][14]
\end{Arabic}}
\flushleft{\begin{malayalam}
ഭൂമിയും പര്‍വ്വതങ്ങളും വിറകൊള്ളുകയും പര്‍വ്വതങ്ങള്‍ ഒലിച്ചു പോകുന്ന മണല്‍ കുന്ന് പോലെയാവുകയും ചെയ്യുന്ന ദിവസത്തില്‍.
\end{malayalam}}
\flushright{\begin{Arabic}
\quranayah[73][15]
\end{Arabic}}
\flushleft{\begin{malayalam}
തീര്‍ച്ചയായും നിങ്ങളിലേക്ക് നിങ്ങളുടെ കാര്യത്തിന് സാക്ഷിയായിട്ടുള്ള ഒരു ദൂതനെ നാം നിയോഗിച്ചിരിക്കുന്നു. ഫിര്‍ഔന്‍റെ അടുത്തേക്ക് നാം ഒരു ദൂതനെ നിയോഗിച്ചത് പോലെത്തന്നെ.
\end{malayalam}}
\flushright{\begin{Arabic}
\quranayah[73][16]
\end{Arabic}}
\flushleft{\begin{malayalam}
എന്നിട്ട് ഫിര്‍ഔന്‍ ആ ദൂതനോട് ധിക്കാരം കാണിച്ചു. അപ്പോള്‍ നാം അവനെ കടുത്ത ഒരു പിടുത്തം പിടിക്കുകയുണ്ടായി.
\end{malayalam}}
\flushright{\begin{Arabic}
\quranayah[73][17]
\end{Arabic}}
\flushleft{\begin{malayalam}
എന്നാല്‍ നിങ്ങള്‍ അവിശ്വസിക്കുകയാണെങ്കില്‍, കുട്ടികളെ നരച്ചവരാക്കിത്തീര്‍ക്കുന്ന ഒരു ദിവസത്തെ നിങ്ങള്‍ക്ക് എങ്ങനെ സൂക്ഷിക്കാനാവും?
\end{malayalam}}
\flushright{\begin{Arabic}
\quranayah[73][18]
\end{Arabic}}
\flushleft{\begin{malayalam}
അതു നിമിത്തം ആകാശം പൊട്ടിപ്പിളരുന്നതാണ്‌. അല്ലാഹുവിന്‍റെ വാഗ്ദാനം പ്രാവര്‍ത്തികമാക്കപ്പെടുന്നതാകുന്നു.
\end{malayalam}}
\flushright{\begin{Arabic}
\quranayah[73][19]
\end{Arabic}}
\flushleft{\begin{malayalam}
തീര്‍ച്ചയായും ഇതൊരു ഉല്‍ബോധനമാകുന്നു. അതിനാല്‍ വല്ലവനും ഉദ്ദേശിക്കുന്ന പക്ഷം അവന്‍ തന്‍റെ രക്ഷിതാവിങ്കലേക്ക് ഒരു മാര്‍ഗം സ്വീകരിച്ചു കൊള്ളട്ടെ.
\end{malayalam}}
\flushright{\begin{Arabic}
\quranayah[73][20]
\end{Arabic}}
\flushleft{\begin{malayalam}
നീയും നിന്‍റെ കൂടെയുള്ളവരില്‍ ഒരു വിഭാഗവും രാത്രിയുടെ മിക്കവാറും മൂന്നില്‍ രണ്ടു ഭാഗവും (ചിലപ്പോള്‍) പകുതിയും (ചിലപ്പോള്‍) മൂന്നിലൊന്നും നിന്നു നമസ്കരിക്കുന്നുണ്ട് എന്ന് തീര്‍ച്ചയായും നിന്‍റെ രക്ഷിതാവിന്നറിയാം. അല്ലാഹുവാണ് രാത്രിയെയും പകലിനെയും കണക്കാക്കുന്നത്‌. നിങ്ങള്‍ക്ക് അത് ക്ലിപ്തപ്പെടുത്താനാവുകയില്ലെന്ന് അവന്നറിയാം. അതിനാല്‍ അവന്‍ നിങ്ങള്‍ക്ക് ഇളവ് ചെയ്തിരിക്കുന്നു. ആകയാല്‍ നിങ്ങള്‍ ഖുര്‍ആനില്‍ നിന്ന് സൌകര്യപ്പെട്ടത് ഓതിക്കൊണ്ട് നമസ്കരിക്കുക. നിങ്ങളുടെ കൂട്ടത്തില്‍ രോഗികളും ഭൂമിയില്‍ സഞ്ചരിച്ച് അല്ലാഹുവിന്‍റെ അനുഗ്രഹം തേടിക്കൊണ്ടിരിക്കുന്ന വേറെ ചിലരും അല്ലാഹുവിന്‍റെ മാര്‍ഗത്തില്‍ യുദ്ധം ചെയ്യുന്ന മറ്റ് ചിലരും ഉണ്ടാകും എന്ന് അല്ലാഹുവിന്നറിയാം. അതിനാല്‍ അതില്‍ (ഖുര്‍ആനില്‍) നിന്ന് സൌകര്യപ്പെട്ടത് നിങ്ങള്‍ പാരായണം ചെയ്തു കൊള്ളുകയും നമസ്കാരം മുറപ്രകാരം നിര്‍വഹിക്കുകയും സകാത്ത് നല്‍കുകയും അല്ലാഹുവിന്ന് ഉത്തമമായ കടം നല്‍കുകയും ചെയ്യുക. സ്വദേഹങ്ങള്‍ക്ക് വേണ്ടി നിങ്ങള്‍ എന്തൊരു നന്‍മ മുന്‍കൂട്ടി ചെയ്ത് വെക്കുകയാണെങ്കിലും അല്ലാഹുവിങ്കല്‍ അത് ഗുണകരവും ഏറ്റവും മഹത്തായ പ്രതിഫലമുള്ളതുമായി നിങ്ങള്‍ കണ്ടെത്തുന്നതാണ്‌. നിങ്ങള്‍ അല്ലാഹുവോട് പാപമോചനം തേടുകയും ചെയ്യുക. തീര്‍ച്ചയായും അല്ലാഹു ഏറെ പൊറുക്കുന്നവനും കരുണാനിധിയുമാകുന്നു.
\end{malayalam}}
\chapter{\textmalayalam{മുദ്ദഥിര്‍ ( പുതച്ച് മൂടിയവന്‍ )}}
\begin{Arabic}
\Huge{\centerline{\basmalah}}\end{Arabic}
\flushright{\begin{Arabic}
\quranayah[74][1]
\end{Arabic}}
\flushleft{\begin{malayalam}
ഹേ, പുതച്ചു മൂടിയവനേ,
\end{malayalam}}
\flushright{\begin{Arabic}
\quranayah[74][2]
\end{Arabic}}
\flushleft{\begin{malayalam}
എഴുന്നേറ്റ് (ജനങ്ങളെ) താക്കീത് ചെയ്യുക.
\end{malayalam}}
\flushright{\begin{Arabic}
\quranayah[74][3]
\end{Arabic}}
\flushleft{\begin{malayalam}
നിന്‍റെ രക്ഷിതാവിനെ മഹത്വപ്പെടുത്തുകയും
\end{malayalam}}
\flushright{\begin{Arabic}
\quranayah[74][4]
\end{Arabic}}
\flushleft{\begin{malayalam}
നിന്‍റെ വസ്ത്രങ്ങള്‍ ശുദ്ധിയാക്കുകയും
\end{malayalam}}
\flushright{\begin{Arabic}
\quranayah[74][5]
\end{Arabic}}
\flushleft{\begin{malayalam}
പാപം വെടിയുകയും ചെയ്യുക.
\end{malayalam}}
\flushright{\begin{Arabic}
\quranayah[74][6]
\end{Arabic}}
\flushleft{\begin{malayalam}
കൂടുതല്‍ നേട്ടം കൊതിച്ചു കൊണ്ട് നീ ഔദാര്യം ചെയ്യരുത്‌.
\end{malayalam}}
\flushright{\begin{Arabic}
\quranayah[74][7]
\end{Arabic}}
\flushleft{\begin{malayalam}
നിന്‍റെ രക്ഷിതാവിനു വേണ്ടി നീ ക്ഷമ കൈക്കൊള്ളുക.
\end{malayalam}}
\flushright{\begin{Arabic}
\quranayah[74][8]
\end{Arabic}}
\flushleft{\begin{malayalam}
എന്നാല്‍ കാഹളത്തില്‍ മുഴക്കപ്പെട്ടാല്‍
\end{malayalam}}
\flushright{\begin{Arabic}
\quranayah[74][9]
\end{Arabic}}
\flushleft{\begin{malayalam}
അന്ന് അത് ഒരു പ്രയാസകരമായ ദിവസമായിരിക്കും.
\end{malayalam}}
\flushright{\begin{Arabic}
\quranayah[74][10]
\end{Arabic}}
\flushleft{\begin{malayalam}
സത്യനിഷേധികള്‍ക്ക് എളുപ്പമുള്ളതല്ലാത്ത ഒരു ദിവസം!
\end{malayalam}}
\flushright{\begin{Arabic}
\quranayah[74][11]
\end{Arabic}}
\flushleft{\begin{malayalam}
എന്നെയും ഞാന്‍ ഏകനായിക്കൊണ്ട് സൃഷ്ടിച്ച ഒരുത്തനെയും വിട്ടേക്കുക.
\end{malayalam}}
\flushright{\begin{Arabic}
\quranayah[74][12]
\end{Arabic}}
\flushleft{\begin{malayalam}
അവന്ന് ഞാന്‍ സമൃദ്ധമായ സമ്പത്ത് ഉണ്ടാക്കി കൊടുക്കുകയും ചെയ്തു.
\end{malayalam}}
\flushright{\begin{Arabic}
\quranayah[74][13]
\end{Arabic}}
\flushleft{\begin{malayalam}
സന്നദ്ധരായി നില്‍ക്കുന്ന സന്തതികളെയും
\end{malayalam}}
\flushright{\begin{Arabic}
\quranayah[74][14]
\end{Arabic}}
\flushleft{\begin{malayalam}
അവന്നു ഞാന്‍ നല്ല സൌകര്യങ്ങള്‍ ചെയ്തു കൊടുക്കുകയും ചെയ്തു.
\end{malayalam}}
\flushright{\begin{Arabic}
\quranayah[74][15]
\end{Arabic}}
\flushleft{\begin{malayalam}
പിന്നെയും ഞാന്‍ കൂടുതല്‍ കൊടുക്കണമെന്ന് അവന്‍ മോഹിക്കുന്നു.
\end{malayalam}}
\flushright{\begin{Arabic}
\quranayah[74][16]
\end{Arabic}}
\flushleft{\begin{malayalam}
അല്ല, തീര്‍ച്ചയായും അവന്‍ നമ്മുടെ ദൃഷ്ടാന്തങ്ങളോട് മാത്സര്യം കാണിക്കുന്നവനായിരിക്കുന്നു.
\end{malayalam}}
\flushright{\begin{Arabic}
\quranayah[74][17]
\end{Arabic}}
\flushleft{\begin{malayalam}
പ്രയാസമുള്ള ഒരു കയറ്റം കയറാന്‍ നാം വഴിയെ അവനെ നിര്‍ബന്ധിക്കുന്നതാണ്‌.
\end{malayalam}}
\flushright{\begin{Arabic}
\quranayah[74][18]
\end{Arabic}}
\flushleft{\begin{malayalam}
തീര്‍ച്ചയായും അവനൊന്നു ചിന്തിച്ചു, അവനൊന്നു കണക്കാക്കുകയും ചെയ്തു.
\end{malayalam}}
\flushright{\begin{Arabic}
\quranayah[74][19]
\end{Arabic}}
\flushleft{\begin{malayalam}
അതിനാല്‍ അവന്‍ നശിക്കട്ടെ. എങ്ങനെയാണവന്‍ കണക്കാക്കിയത്‌?
\end{malayalam}}
\flushright{\begin{Arabic}
\quranayah[74][20]
\end{Arabic}}
\flushleft{\begin{malayalam}
വീണ്ടും അവന്‍ നശിക്കട്ടെ, എങ്ങനെയാണവന്‍ കണക്കാക്കിയത്‌?
\end{malayalam}}
\flushright{\begin{Arabic}
\quranayah[74][21]
\end{Arabic}}
\flushleft{\begin{malayalam}
പിന്നീട് അവനൊന്നു നോക്കി.
\end{malayalam}}
\flushright{\begin{Arabic}
\quranayah[74][22]
\end{Arabic}}
\flushleft{\begin{malayalam}
പിന്നെ അവന്‍ മുഖം ചുളിക്കുകയും മുഖം കറുപ്പിക്കുകയും ചെയ്തു.
\end{malayalam}}
\flushright{\begin{Arabic}
\quranayah[74][23]
\end{Arabic}}
\flushleft{\begin{malayalam}
പിന്നെ അവന്‍ പിന്നോട്ട് മാറുകയും അഹങ്കാരം നടിക്കുകയും ചെയ്തു.
\end{malayalam}}
\flushright{\begin{Arabic}
\quranayah[74][24]
\end{Arabic}}
\flushleft{\begin{malayalam}
എന്നിട്ടവന്‍ പറഞ്ഞു: ഇത് (ആരില്‍ നിന്നോ) ഉദ്ധരിക്കപ്പെടുന്ന മാരണമല്ലാതെ മറ്റൊന്നുമല്ല.
\end{malayalam}}
\flushright{\begin{Arabic}
\quranayah[74][25]
\end{Arabic}}
\flushleft{\begin{malayalam}
ഇത് മനുഷ്യന്‍റെ വാക്കല്ലാതെ മറ്റൊന്നുമല്ല.
\end{malayalam}}
\flushright{\begin{Arabic}
\quranayah[74][26]
\end{Arabic}}
\flushleft{\begin{malayalam}
വഴിയെ ഞാന്‍ അവനെ സഖറില്‍ (നരകത്തില്‍) ഇട്ട് എരിക്കുന്നതാണ്‌.
\end{malayalam}}
\flushright{\begin{Arabic}
\quranayah[74][27]
\end{Arabic}}
\flushleft{\begin{malayalam}
സഖര്‍ എന്നാല്‍ എന്താണെന്ന് നിനക്കറിയുമോ?
\end{malayalam}}
\flushright{\begin{Arabic}
\quranayah[74][28]
\end{Arabic}}
\flushleft{\begin{malayalam}
അത് ഒന്നും ബാക്കിയാക്കുകയോ വിട്ടുകളയുകയോ ഇല്ല.
\end{malayalam}}
\flushright{\begin{Arabic}
\quranayah[74][29]
\end{Arabic}}
\flushleft{\begin{malayalam}
അത് തൊലി കരിച്ച് രൂപം മാറ്റിക്കളയുന്നതാണ്‌.
\end{malayalam}}
\flushright{\begin{Arabic}
\quranayah[74][30]
\end{Arabic}}
\flushleft{\begin{malayalam}
അതിന്‍റെ മേല്‍നോട്ടത്തിന് പത്തൊമ്പത് പേരുണ്ട്‌.
\end{malayalam}}
\flushright{\begin{Arabic}
\quranayah[74][31]
\end{Arabic}}
\flushleft{\begin{malayalam}
നരകത്തിന്‍റെ മേല്‍നോട്ടക്കാരായി നാം മലക്കുകളെ മാത്രമാണ് നിശ്ചയിച്ചിരിക്കുന്നത്‌. അവരുടെ എണ്ണത്തെ നാം സത്യനിഷേധികള്‍ക്ക് ഒരു പരീക്ഷണം മാത്രമാക്കിയിരിക്കുന്നു. വേദം നല്‍കപ്പെട്ടിട്ടുള്ളവര്‍ക്ക് ദൃഢബോധ്യം വരുവാനും സത്യവിശ്വാസികള്‍ക്ക് വിശ്വാസം വര്‍ദ്ധിക്കാനും വേദം നല്‍കപ്പെട്ടവരും സത്യവിശ്വാസികളും സംശയത്തിലകപ്പെടാതിരിക്കാനും അല്ലാഹു എന്തൊരു ഉപമയാണ് ഇതു കൊണ്ട് ഉദ്ദേശിച്ചിട്ടുള്ളതെന്ന് ഹൃദയങ്ങളില്‍ രോഗമുള്ളവരും സത്യനിഷേധികളും പറയുവാനും വേണ്ടിയത്രെ അത്‌. അപ്രകാരം അല്ലാഹു താന്‍ ഉദ്ദേശിക്കുന്നവരെ പിഴപ്പിക്കുകയും, താന്‍ ഉദ്ദേശിക്കുന്നവരെ നേര്‍വഴിയിലാക്കുകയും ചെയ്യുന്നു. നിന്‍റെ രക്ഷിതാവിന്‍റെ സൈന്യങ്ങളെ അവനല്ലാതെ മറ്റാരും അറിയുകയില്ല. ഇത് മനുഷ്യര്‍ക്ക് ഒരു ഉല്‍ബോധനമല്ലാതെ മറ്റൊന്നുമല്ല.
\end{malayalam}}
\flushright{\begin{Arabic}
\quranayah[74][32]
\end{Arabic}}
\flushleft{\begin{malayalam}
നിസ്സംശയം, ചന്ദ്രനെ തന്നെയാണ സത്യം.
\end{malayalam}}
\flushright{\begin{Arabic}
\quranayah[74][33]
\end{Arabic}}
\flushleft{\begin{malayalam}
രാത്രി പിന്നിട്ട് പോകുമ്പോള്‍ അതിനെ തന്നെയാണ സത്യം.
\end{malayalam}}
\flushright{\begin{Arabic}
\quranayah[74][34]
\end{Arabic}}
\flushleft{\begin{malayalam}
പ്രഭാതം പുലര്‍ന്നാല്‍ അതു തന്നെയാണ സത്യം.
\end{malayalam}}
\flushright{\begin{Arabic}
\quranayah[74][35]
\end{Arabic}}
\flushleft{\begin{malayalam}
തീര്‍ച്ചയായും അത് (നരകം) ഗൌരവമുള്ള കാര്യങ്ങളില്‍ ഒന്നാകുന്നു.
\end{malayalam}}
\flushright{\begin{Arabic}
\quranayah[74][36]
\end{Arabic}}
\flushleft{\begin{malayalam}
മനുഷ്യര്‍ക്ക് ഒരു താക്കീതെന്ന നിലയില്‍.
\end{malayalam}}
\flushright{\begin{Arabic}
\quranayah[74][37]
\end{Arabic}}
\flushleft{\begin{malayalam}
അതായത് നിങ്ങളില്‍ നിന്ന് മുന്നോട്ട് പോകുവാനോ, പിന്നോട്ട് വെക്കുവാനോ ഉദ്ദേശിക്കുന്നവര്‍ക്ക്‌.
\end{malayalam}}
\flushright{\begin{Arabic}
\quranayah[74][38]
\end{Arabic}}
\flushleft{\begin{malayalam}
ഓരോ വ്യക്തിയും താന്‍ സമ്പാദിച്ചു വെച്ചതിന് പണയപ്പെട്ടവനാകുന്നു.
\end{malayalam}}
\flushright{\begin{Arabic}
\quranayah[74][39]
\end{Arabic}}
\flushleft{\begin{malayalam}
വലതുപക്ഷക്കാരൊഴികെ.
\end{malayalam}}
\flushright{\begin{Arabic}
\quranayah[74][40]
\end{Arabic}}
\flushleft{\begin{malayalam}
ചില സ്വര്‍ഗത്തോപ്പുകളിലായിരിക്കും അവര്‍. അവര്‍ അന്വേഷിക്കും;
\end{malayalam}}
\flushright{\begin{Arabic}
\quranayah[74][41]
\end{Arabic}}
\flushleft{\begin{malayalam}
കുറ്റവാളികളെപ്പറ്റി
\end{malayalam}}
\flushright{\begin{Arabic}
\quranayah[74][42]
\end{Arabic}}
\flushleft{\begin{malayalam}
നിങ്ങളെ നരകത്തില്‍ പ്രവേശിപ്പിച്ചത് എന്തൊന്നാണെന്ന്‌.
\end{malayalam}}
\flushright{\begin{Arabic}
\quranayah[74][43]
\end{Arabic}}
\flushleft{\begin{malayalam}
അവര്‍ (കുറ്റവാളികള്‍) മറുപടി പറയും: ഞങ്ങള്‍ നമസ്കരിക്കുന്നവരുടെ കൂട്ടത്തിലായില്ല.
\end{malayalam}}
\flushright{\begin{Arabic}
\quranayah[74][44]
\end{Arabic}}
\flushleft{\begin{malayalam}
ഞങ്ങള്‍ അഗതിക്ക് ആഹാരം നല്‍കുമായിരുന്നില്ല.
\end{malayalam}}
\flushright{\begin{Arabic}
\quranayah[74][45]
\end{Arabic}}
\flushleft{\begin{malayalam}
തോന്നിവാസത്തില്‍ മുഴുകുന്നവരുടെ കൂടെ ഞങ്ങളും മുഴുകുമായിരുന്നു.
\end{malayalam}}
\flushright{\begin{Arabic}
\quranayah[74][46]
\end{Arabic}}
\flushleft{\begin{malayalam}
പ്രതിഫലത്തിന്‍റെ നാളിനെ ഞങ്ങള്‍ നിഷേധിച്ചു കളയുമായിരുന്നു.
\end{malayalam}}
\flushright{\begin{Arabic}
\quranayah[74][47]
\end{Arabic}}
\flushleft{\begin{malayalam}
അങ്ങനെയിരിക്കെ ആ ഉറപ്പായ മരണം ഞങ്ങള്‍ക്ക് വന്നെത്തി.
\end{malayalam}}
\flushright{\begin{Arabic}
\quranayah[74][48]
\end{Arabic}}
\flushleft{\begin{malayalam}
ഇനി അവര്‍ക്ക് ശുപാര്‍ശക്കാരുടെ ശുപാര്‍ശയൊന്നും പ്രയോജനപ്പെടുകയില്ല.
\end{malayalam}}
\flushright{\begin{Arabic}
\quranayah[74][49]
\end{Arabic}}
\flushleft{\begin{malayalam}
എന്നിരിക്കെ അവര്‍ക്കെന്തു പറ്റി? അവര്‍ ഉല്‍ബോധനത്തില്‍ നിന്ന് തിരിഞ്ഞുകളയുന്നവരായിരിക്കുന്നു.
\end{malayalam}}
\flushright{\begin{Arabic}
\quranayah[74][50]
\end{Arabic}}
\flushleft{\begin{malayalam}
അവര്‍ വിറളി പിടിച്ച കഴുതകളെപ്പോലിരിക്കുന്നു.
\end{malayalam}}
\flushright{\begin{Arabic}
\quranayah[74][51]
\end{Arabic}}
\flushleft{\begin{malayalam}
സിംഹത്തില്‍ നിന്ന് ഓടിരക്ഷപ്പെടുന്ന (കഴുതകള്‍)
\end{malayalam}}
\flushright{\begin{Arabic}
\quranayah[74][52]
\end{Arabic}}
\flushleft{\begin{malayalam}
അല്ല, അവരില്‍ ഓരോരുത്തരും ആഗ്രഹിക്കുന്നു; തനിക്ക് അല്ലാഹുവിങ്കല്‍ നിന്ന് നിവര്‍ത്തിയ ഏടുകള്‍ നല്‍കപ്പെടണമെന്ന്‌.
\end{malayalam}}
\flushright{\begin{Arabic}
\quranayah[74][53]
\end{Arabic}}
\flushleft{\begin{malayalam}
അല്ല; പക്ഷെ, അവര്‍ പരലോകത്തെ ഭയപ്പെടുന്നില്ല.
\end{malayalam}}
\flushright{\begin{Arabic}
\quranayah[74][54]
\end{Arabic}}
\flushleft{\begin{malayalam}
അല്ല; തീര്‍ച്ചയായും ഇത് ഒരു ഉല്‍ബോധനമാകുന്നു.
\end{malayalam}}
\flushright{\begin{Arabic}
\quranayah[74][55]
\end{Arabic}}
\flushleft{\begin{malayalam}
ആകയാല്‍ ആര്‍ ഉദ്ദേശിക്കുന്നുവോ അവരത് ഓര്‍മിച്ചു കൊള്ളട്ടെ.
\end{malayalam}}
\flushright{\begin{Arabic}
\quranayah[74][56]
\end{Arabic}}
\flushleft{\begin{malayalam}
അല്ലാഹു ഉദ്ദേശിക്കുന്നുവെങ്കിലല്ലാതെ അവര്‍ ഓര്‍മിക്കുന്നതല്ല. അവനാകുന്നു ഭക്തിക്കവകാശപ്പെട്ടവന്‍; പാപമോചനത്തിന് അവകാശപ്പെട്ടവന്‍.
\end{malayalam}}
\chapter{\textmalayalam{ഖിയാമ ( ഉയിര്‍ത്തെഴുന്നേല്‍പ്പ് )}}
\begin{Arabic}
\Huge{\centerline{\basmalah}}\end{Arabic}
\flushright{\begin{Arabic}
\quranayah[75][1]
\end{Arabic}}
\flushleft{\begin{malayalam}
ഉയിര്‍ത്തെഴുന്നേല്‍പിന്‍റെ നാളുകൊണ്ട് ഞാനിതാ സത്യം ചെയ്യുന്നു.
\end{malayalam}}
\flushright{\begin{Arabic}
\quranayah[75][2]
\end{Arabic}}
\flushleft{\begin{malayalam}
കുറ്റപ്പെടുത്തുന്ന മനസ്സിനെക്കൊണ്ടും ഞാന്‍ സത്യം ചെയ്തു പറയുന്നു.
\end{malayalam}}
\flushright{\begin{Arabic}
\quranayah[75][3]
\end{Arabic}}
\flushleft{\begin{malayalam}
മനുഷ്യന്‍ വിചാരിക്കുന്നുണ്ടോ; നാം അവന്‍റെ എല്ലുകളെ ഒരുമിച്ചുകൂട്ടുകയില്ലെന്ന്‌?
\end{malayalam}}
\flushright{\begin{Arabic}
\quranayah[75][4]
\end{Arabic}}
\flushleft{\begin{malayalam}
അതെ, നാം അവന്‍റെ വിരല്‍ത്തുമ്പുകളെ പോലും ശരിപ്പെടുത്താന്‍ കഴിവുള്ളവനായിരിക്കെ.
\end{malayalam}}
\flushright{\begin{Arabic}
\quranayah[75][5]
\end{Arabic}}
\flushleft{\begin{malayalam}
പക്ഷെ (എന്നിട്ടും) മനുഷ്യന്‍ അവന്‍റെ ഭാവി ജീവിതത്തില്‍ തോന്നിവാസം ചെയ്യാന്‍ ഉദ്ദേശിക്കുന്നു.
\end{malayalam}}
\flushright{\begin{Arabic}
\quranayah[75][6]
\end{Arabic}}
\flushleft{\begin{malayalam}
എപ്പോഴാണ് ഈ ഉയിര്‍ത്തെഴുന്നേല്‍പിന്‍റെ നാള്‍ എന്നവന്‍ ചോദിക്കുന്നു.
\end{malayalam}}
\flushright{\begin{Arabic}
\quranayah[75][7]
\end{Arabic}}
\flushleft{\begin{malayalam}
എന്നാല്‍ കണ്ണ് അഞ്ചിപ്പോകുകയും
\end{malayalam}}
\flushright{\begin{Arabic}
\quranayah[75][8]
\end{Arabic}}
\flushleft{\begin{malayalam}
ചന്ദ്രന്ന് ഗ്രഹണം ബാധിക്കുകയും
\end{malayalam}}
\flushright{\begin{Arabic}
\quranayah[75][9]
\end{Arabic}}
\flushleft{\begin{malayalam}
സൂര്യനും ചന്ദ്രനും ഒരുമിച്ചുകൂട്ടപ്പെടുകയും ചെയ്താല്‍!
\end{malayalam}}
\flushright{\begin{Arabic}
\quranayah[75][10]
\end{Arabic}}
\flushleft{\begin{malayalam}
അന്നേ ദിവസം മനുഷ്യന്‍ പറയും; എവിടെയാണ് ഓടിരക്ഷപ്പെടാനുള്ളതെന്ന്‌.
\end{malayalam}}
\flushright{\begin{Arabic}
\quranayah[75][11]
\end{Arabic}}
\flushleft{\begin{malayalam}
ഇല്ല. യാതൊരു രക്ഷയുമില്ല.
\end{malayalam}}
\flushright{\begin{Arabic}
\quranayah[75][12]
\end{Arabic}}
\flushleft{\begin{malayalam}
നിന്‍റെ രക്ഷിതാവിങ്കലേക്കാണ് അന്നേ ദിവസം ചെന്നുകൂടല്‍.
\end{malayalam}}
\flushright{\begin{Arabic}
\quranayah[75][13]
\end{Arabic}}
\flushleft{\begin{malayalam}
അന്നേ ദിവസം മനുഷ്യന്‍ മുന്‍കൂട്ടി ചെയ്തതിനെപ്പറ്റിയും നീട്ടിവെച്ചതിനെപ്പറ്റിയും അവന്ന് വിവരമറിയിക്കപ്പെടും.
\end{malayalam}}
\flushright{\begin{Arabic}
\quranayah[75][14]
\end{Arabic}}
\flushleft{\begin{malayalam}
തന്നെയുമല്ല. മനുഷ്യന്‍ തനിക്കെതിരില്‍ തന്നെ ഒരു തെളിവായിരിക്കും.
\end{malayalam}}
\flushright{\begin{Arabic}
\quranayah[75][15]
\end{Arabic}}
\flushleft{\begin{malayalam}
അവന്‍ ഒഴികഴിവുകള്‍ സമര്‍പ്പിച്ചാലും ശരി.
\end{malayalam}}
\flushright{\begin{Arabic}
\quranayah[75][16]
\end{Arabic}}
\flushleft{\begin{malayalam}
നീ അത് (ഖുര്‍ആന്‍) ധൃതിപ്പെട്ട് ഹൃദിസ്ഥമാക്കാന്‍ വേണ്ടി അതും കൊണ്ട് നിന്‍റെ നാവ് ചലിപ്പിക്കേണ്ട.
\end{malayalam}}
\flushright{\begin{Arabic}
\quranayah[75][17]
\end{Arabic}}
\flushleft{\begin{malayalam}
തീര്‍ച്ചയായും അതിന്‍റെ (ഖുര്‍ആന്‍റെ) സമാഹരണവും അത് ഓതിത്തരലും നമ്മുടെ ബാധ്യതയാകുന്നു.
\end{malayalam}}
\flushright{\begin{Arabic}
\quranayah[75][18]
\end{Arabic}}
\flushleft{\begin{malayalam}
അങ്ങനെ നാം അത് ഓതിത്തന്നാല്‍ ആ ഓത്ത് നീ പിന്തുടരുക.
\end{malayalam}}
\flushright{\begin{Arabic}
\quranayah[75][19]
\end{Arabic}}
\flushleft{\begin{malayalam}
പിന്നീട് അത് വിവരിച്ചുതരലും നമ്മുടെ ബാധ്യതയാകുന്നു.
\end{malayalam}}
\flushright{\begin{Arabic}
\quranayah[75][20]
\end{Arabic}}
\flushleft{\begin{malayalam}
അല്ല, നിങ്ങള്‍ ക്ഷണികമായ ഈ ജീവിതത്തെ ഇഷ്ടപ്പെടുന്നു.
\end{malayalam}}
\flushright{\begin{Arabic}
\quranayah[75][21]
\end{Arabic}}
\flushleft{\begin{malayalam}
പരലോകത്തെ നിങ്ങള്‍ വിട്ടേക്കുകയും ചെയ്യുന്നു.
\end{malayalam}}
\flushright{\begin{Arabic}
\quranayah[75][22]
\end{Arabic}}
\flushleft{\begin{malayalam}
ചില മുഖങ്ങള്‍ അന്ന് പ്രസന്നതയുള്ളതും
\end{malayalam}}
\flushright{\begin{Arabic}
\quranayah[75][23]
\end{Arabic}}
\flushleft{\begin{malayalam}
അവയുടെ രക്ഷിതാവിന്‍റെ നേര്‍ക്ക് ദൃഷ്ടി തിരിച്ചവയുമായിരിക്കും.
\end{malayalam}}
\flushright{\begin{Arabic}
\quranayah[75][24]
\end{Arabic}}
\flushleft{\begin{malayalam}
ചില മുഖങ്ങള്‍ അന്നു കരുവാളിച്ചതായിരിക്കും.
\end{malayalam}}
\flushright{\begin{Arabic}
\quranayah[75][25]
\end{Arabic}}
\flushleft{\begin{malayalam}
ഏതോ അത്യാപത്ത് അവയെ പിടികൂടാന്‍ പോകുകയാണ് എന്ന് അവര്‍ വിചാരിക്കും.
\end{malayalam}}
\flushright{\begin{Arabic}
\quranayah[75][26]
\end{Arabic}}
\flushleft{\begin{malayalam}
അല്ല, (പ്രാണന്‍) തൊണ്ടക്കുഴിയില്‍ എത്തുകയും,
\end{malayalam}}
\flushright{\begin{Arabic}
\quranayah[75][27]
\end{Arabic}}
\flushleft{\begin{malayalam}
മന്ത്രിക്കാനാരുണ്ട് എന്ന് പറയപ്പെടുകയും,
\end{malayalam}}
\flushright{\begin{Arabic}
\quranayah[75][28]
\end{Arabic}}
\flushleft{\begin{malayalam}
അത് (തന്‍റെ) വേര്‍പാടാണെന്ന് അവന്‍ വിചാരിക്കുകയും,
\end{malayalam}}
\flushright{\begin{Arabic}
\quranayah[75][29]
\end{Arabic}}
\flushleft{\begin{malayalam}
കണങ്കാലും കണങ്കാലുമായി കൂടിപ്പിണയുകയും ചെയ്താല്‍,
\end{malayalam}}
\flushright{\begin{Arabic}
\quranayah[75][30]
\end{Arabic}}
\flushleft{\begin{malayalam}
അന്ന് നിന്‍റെ രക്ഷിതാവിങ്കലേക്കായിരിക്കും തെളിച്ചു കൊണ്ടു പോകുന്നത്‌.
\end{malayalam}}
\flushright{\begin{Arabic}
\quranayah[75][31]
\end{Arabic}}
\flushleft{\begin{malayalam}
എന്നാല്‍ അവന്‍ വിശ്വസിച്ചില്ല. അവന്‍ നമസ്കരിച്ചതുമില്ല.
\end{malayalam}}
\flushright{\begin{Arabic}
\quranayah[75][32]
\end{Arabic}}
\flushleft{\begin{malayalam}
പക്ഷെ അവന്‍ നിഷേധിക്കുകയും പിന്തിരിയുകയും ചെയ്തു.
\end{malayalam}}
\flushright{\begin{Arabic}
\quranayah[75][33]
\end{Arabic}}
\flushleft{\begin{malayalam}
എന്നിട്ടു ദുരഭിമാനം നടിച്ചു കൊണ്ട് അവന്‍ അവന്‍റെ സ്വന്തക്കാരുടെ അടുത്തേക്ക് പോയി
\end{malayalam}}
\flushright{\begin{Arabic}
\quranayah[75][34]
\end{Arabic}}
\flushleft{\begin{malayalam}
(ശിക്ഷ) നിനക്കേറ്റവും അര്‍ഹമായതു തന്നെ. നിനക്കേറ്റവും അര്‍ഹമായതു തന്നെ.
\end{malayalam}}
\flushright{\begin{Arabic}
\quranayah[75][35]
\end{Arabic}}
\flushleft{\begin{malayalam}
വീണ്ടും നിനക്കേറ്റവും അര്‍ഹമായത് തന്നെ. നിനക്കേറ്റവും അര്‍ഹമായത് തന്നെ
\end{malayalam}}
\flushright{\begin{Arabic}
\quranayah[75][36]
\end{Arabic}}
\flushleft{\begin{malayalam}
മനുഷ്യന്‍ വിചാരിക്കുന്നുവോ; അവന്‍ വെറുതെ വിട്ടേക്കപ്പെടുമെന്ന്‌!
\end{malayalam}}
\flushright{\begin{Arabic}
\quranayah[75][37]
\end{Arabic}}
\flushleft{\begin{malayalam}
അവന്‍ സ്രവിക്കപ്പെടുന്ന ശുക്ലത്തില്‍ നിന്നുള്ള ഒരു കണമായിരുന്നില്ലേ?
\end{malayalam}}
\flushright{\begin{Arabic}
\quranayah[75][38]
\end{Arabic}}
\flushleft{\begin{malayalam}
പിന്നെ അവന്‍ ഒരു ഭ്രൂണമായി. എന്നിട്ട് അല്ലാഹു (അവനെ) സൃഷ്ടിച്ചു സംവിധാനിച്ചു.
\end{malayalam}}
\flushright{\begin{Arabic}
\quranayah[75][39]
\end{Arabic}}
\flushleft{\begin{malayalam}
അങ്ങനെ അതില്‍ നിന്ന് ആണും പെണ്ണുമാകുന്ന രണ്ടു ഇണകളെ അവന്‍ ഉണ്ടാക്കി.
\end{malayalam}}
\flushright{\begin{Arabic}
\quranayah[75][40]
\end{Arabic}}
\flushleft{\begin{malayalam}
അങ്ങനെയുള്ളവന്‍ മരിച്ചവരെ ജീവിപ്പിക്കാന്‍ കഴിവുള്ളവനല്ലെ?
\end{malayalam}}
\chapter{\textmalayalam{ഇന്‍സാന്‍ ( മനുഷ്യന്‍ )}}
\begin{Arabic}
\Huge{\centerline{\basmalah}}\end{Arabic}
\flushright{\begin{Arabic}
\quranayah[76][1]
\end{Arabic}}
\flushleft{\begin{malayalam}
മനുഷ്യന്‍ പ്രസ്താവ്യമായ ഒരു വസ്തുവേ ആയിരുന്നില്ലാത്ത ഒരു കാലഘട്ടം അവന്‍റെ മേല്‍ കഴിഞ്ഞുപോയിട്ടുണ്ടോ?
\end{malayalam}}
\flushright{\begin{Arabic}
\quranayah[76][2]
\end{Arabic}}
\flushleft{\begin{malayalam}
കൂടിച്ചേര്‍ന്നുണ്ടായ ഒരു ബീജത്തില്‍ നിന്ന് തീര്‍ച്ചയായും നാം മനുഷ്യനെ സൃഷ്ടിച്ചിരിക്കുന്നു. നാം അവനെ പരീക്ഷിക്കുവാനായിട്ട്‌. അങ്ങനെ അവനെ നാം കേള്‍വിയുള്ളവനും കാഴ്ചയുള്ളവനുമാക്കിയിരിക്കുന്നു.
\end{malayalam}}
\flushright{\begin{Arabic}
\quranayah[76][3]
\end{Arabic}}
\flushleft{\begin{malayalam}
തീര്‍ച്ചയായും നാം അവന്ന് വഴി കാണിച്ചുകൊടുത്തിരിക്കുന്നു. എന്നിട്ട് ഒന്നുകില്‍ അവന്‍ നന്ദിയുള്ളവനാകുന്നു. അല്ലെങ്കില്‍ നന്ദികെട്ടവനാകുന്നു.
\end{malayalam}}
\flushright{\begin{Arabic}
\quranayah[76][4]
\end{Arabic}}
\flushleft{\begin{malayalam}
തീര്‍ച്ചയായും സത്യനിഷേധികള്‍ക്ക് നാം ചങ്ങലകളും വിലങ്ങുകളും കത്തിജ്വലിക്കുന്ന നരകാഗ്നിയും ഒരുക്കി വെച്ചിരിക്കുന്നു.
\end{malayalam}}
\flushright{\begin{Arabic}
\quranayah[76][5]
\end{Arabic}}
\flushleft{\begin{malayalam}
തീര്‍ച്ചയായും പുണ്യവാന്‍മാര്‍ (സ്വര്‍ഗത്തില്‍) ഒരു പാനപാത്രത്തില്‍ നിന്ന് കുടിക്കുന്നതാണ്‌. അതിന്‍റെ ചേരുവ കര്‍പ്പൂരമായിരിക്കും.
\end{malayalam}}
\flushright{\begin{Arabic}
\quranayah[76][6]
\end{Arabic}}
\flushleft{\begin{malayalam}
അല്ലാഹുവിന്‍റെ ദാസന്‍മാര്‍ കുടിക്കുന്ന ഒരു ഉറവു വെള്ളമത്രെ അത്‌. അവരത് പൊട്ടിച്ചൊഴുക്കിക്കൊണ്ടിരിക്കും.
\end{malayalam}}
\flushright{\begin{Arabic}
\quranayah[76][7]
\end{Arabic}}
\flushleft{\begin{malayalam}
നേര്‍ച്ച അവര്‍ നിറവേറ്റുകയും ആപത്തു പടര്‍ന്ന് പിടിക്കുന്ന ഒരു ദിവസത്തെ അവര്‍ ഭയപ്പെടുകയും ചെയ്യും.
\end{malayalam}}
\flushright{\begin{Arabic}
\quranayah[76][8]
\end{Arabic}}
\flushleft{\begin{malayalam}
ആഹാരത്തോട് പ്രിയമുള്ളതോടൊപ്പം തന്നെ അഗതിക്കും അനാഥയ്ക്കും തടവുകാരന്നും അവരത് നല്‍കുകയും ചെയ്യും.
\end{malayalam}}
\flushright{\begin{Arabic}
\quranayah[76][9]
\end{Arabic}}
\flushleft{\begin{malayalam}
(അവര്‍ പറയും:) അല്ലാഹുവിന്‍റെ പ്രീതിക്കു വേണ്ടി മാത്രമാണ് ഞങ്ങള്‍ നിങ്ങള്‍ക്കു ആഹാരം നല്‍കുന്നത്‌. നിങ്ങളുടെ പക്കല്‍ നിന്നു യാതൊരു പ്രതിഫലവും നന്ദിയും ഞങ്ങള്‍ ആഗ്രഹിക്കുന്നില്ല.
\end{malayalam}}
\flushright{\begin{Arabic}
\quranayah[76][10]
\end{Arabic}}
\flushleft{\begin{malayalam}
മുഖം ചുളിച്ചു പോകുന്നതും, ദുസ്സഹവുമായ ഒരു ദിവസത്തെ ഞങ്ങളുടെ രക്ഷിതാവിങ്കല്‍ നിന്ന് തീര്‍ച്ചയായും ഞങ്ങള്‍ ഭയപ്പെടുന്നു.
\end{malayalam}}
\flushright{\begin{Arabic}
\quranayah[76][11]
\end{Arabic}}
\flushleft{\begin{malayalam}
അതിനാല്‍ ആ ദിവസത്തിന്‍റെ തിന്‍മയില്‍ നിന്ന് അല്ലാഹു അവരെ കാത്തുരക്ഷിക്കുകയും, പ്രസന്നതയും സന്തോഷവും അവര്‍ക്കു അവന്‍ നല്‍കുകയും ചെയ്യുന്നതാണ്‌.
\end{malayalam}}
\flushright{\begin{Arabic}
\quranayah[76][12]
\end{Arabic}}
\flushleft{\begin{malayalam}
അവര്‍ ക്ഷമിച്ചതിനാല്‍ സ്വര്‍ഗത്തോപ്പും പട്ടു വസ്ത്രങ്ങളും അവര്‍ക്കവന്‍ പ്രതിഫലമായി നല്‍കുന്നതാണ്‌.
\end{malayalam}}
\flushright{\begin{Arabic}
\quranayah[76][13]
\end{Arabic}}
\flushleft{\begin{malayalam}
അവരവിടെ സോഫകളില്‍ ചാരിയിരിക്കുന്നവരായിരിക്കും. വെയിലോ കൊടും തണുപ്പോ അവര്‍ അവിടെ കാണുകയില്ല.
\end{malayalam}}
\flushright{\begin{Arabic}
\quranayah[76][14]
\end{Arabic}}
\flushleft{\begin{malayalam}
ആ സ്വര്‍ഗത്തിലെ തണലുകള്‍ അവരുടെ മേല്‍ അടുത്തു നില്‍ക്കുന്നതായിരിക്കും. അതിലെ പഴങ്ങള്‍ പറിച്ചെടുക്കാന്‍ സൌകര്യമുള്ളതാക്കപ്പെടുകയും ചെയ്തിരിക്കുന്നു.
\end{malayalam}}
\flushright{\begin{Arabic}
\quranayah[76][15]
\end{Arabic}}
\flushleft{\begin{malayalam}
വെള്ളിയുടെ പാത്രങ്ങളും (മിനുസം കൊണ്ട്‌) സ്ഫടികം പോലെയായിതീര്‍ന്നിട്ടുള്ള കോപ്പകളുമായി അവര്‍ക്കിടയില്‍ (പരിചാരകന്‍മാര്‍) ചുറ്റി നടക്കുന്നതാണ്‌.
\end{malayalam}}
\flushright{\begin{Arabic}
\quranayah[76][16]
\end{Arabic}}
\flushleft{\begin{malayalam}
വെള്ളിക്കോപ്പകള്‍. അവര്‍ അവയ്ക്ക് (പാത്രങ്ങള്‍ക്ക്‌) ഒരു തോതനുസരിച്ച് അളവ് നിര്‍ണയിച്ചിരിക്കും.
\end{malayalam}}
\flushright{\begin{Arabic}
\quranayah[76][17]
\end{Arabic}}
\flushleft{\begin{malayalam}
ഇഞ്ചിനീരിന്‍റെ ചേരുവയുള്ള ഒരു കോപ്പ അവര്‍ക്ക് അവിടെ കുടിക്കാന്‍ നല്‍കപ്പെടുന്നതാണ്‌.
\end{malayalam}}
\flushright{\begin{Arabic}
\quranayah[76][18]
\end{Arabic}}
\flushleft{\begin{malayalam}
അതായത് അവിടത്തെ (സ്വര്‍ഗത്തിലെ) സല്‍സബീല്‍ എന്നു പേരുള്ള ഒരു ഉറവിലെ വെള്ളം.
\end{malayalam}}
\flushright{\begin{Arabic}
\quranayah[76][19]
\end{Arabic}}
\flushleft{\begin{malayalam}
അനശ്വര ജീവിതം നല്‍കപ്പെട്ട ചില കുട്ടികള്‍ അവര്‍ക്കിടയിലൂടെ ചുറ്റി നടന്നുകൊണ്ടുമിരിക്കും. അവരെ നീ കണ്ടാല്‍ വിതറിയ മുത്തുകളാണ് അവരെന്ന് നീ വിചാരിക്കും.
\end{malayalam}}
\flushright{\begin{Arabic}
\quranayah[76][20]
\end{Arabic}}
\flushleft{\begin{malayalam}
അവിടം നീ കണ്ടാല്‍ സുഖാനുഗ്രഹവും വലിയൊരു സാമ്രാജ്യവും നീ കാണുന്നതാണ്‌.
\end{malayalam}}
\flushright{\begin{Arabic}
\quranayah[76][21]
\end{Arabic}}
\flushleft{\begin{malayalam}
അവരുടെ മേല്‍ പച്ച നിറമുള്ള നേര്‍ത്ത പട്ടുവസ്ത്രങ്ങളും കട്ടിയുള്ള പട്ടു വസ്ത്രവും ഉണ്ടായിരിക്കും. വെള്ളിയുടെ വളകളും അവര്‍ക്ക് അണിയിക്കപ്പെടുന്നതാണ്‌. അവര്‍ക്ക് അവരുടെ രക്ഷിതാവ് തികച്ചും ശുദ്ധമായ പാനീയം കുടിക്കാന്‍ കൊടുക്കുന്നതുമാണ്‌.
\end{malayalam}}
\flushright{\begin{Arabic}
\quranayah[76][22]
\end{Arabic}}
\flushleft{\begin{malayalam}
(അവരോട് പറയപ്പെടും:) തീര്‍ച്ചയായും അത് നിങ്ങള്‍ക്കുള്ള പ്രതിഫലമാകുന്നു. നിങ്ങളുടെ പരിശ്രമം നന്ദിപൂര്‍വ്വം സ്വീകരിക്കപ്പെട്ടിരിക്കയാകുന്നൂ.
\end{malayalam}}
\flushright{\begin{Arabic}
\quranayah[76][23]
\end{Arabic}}
\flushleft{\begin{malayalam}
തീര്‍ച്ചയായും നാം നിനക്ക് ഈ ഖുര്‍ആനിനെ അല്‍പാല്‍പമായി അവതരിപ്പിച്ചു തന്നിരിക്കുന്നു.
\end{malayalam}}
\flushright{\begin{Arabic}
\quranayah[76][24]
\end{Arabic}}
\flushleft{\begin{malayalam}
ആകയാല്‍ നിന്‍റെ രക്ഷിതാവിന്‍റെ തീരുമാനത്തിന് നീ ക്ഷമയോടെ കാത്തിരിക്കുക. അവരുടെ കൂട്ടത്തില്‍ നിന്ന് യാതൊരു പാപിയെയും നന്ദികെട്ടവനെയും നീ അനുസരിച്ചു പോകരുത്‌.
\end{malayalam}}
\flushright{\begin{Arabic}
\quranayah[76][25]
\end{Arabic}}
\flushleft{\begin{malayalam}
നിന്‍റെ രക്ഷിതാവിന്‍റെ നാമം കാലത്തും വൈകുന്നേരവും നീ സ്മരിക്കുകയും ചെയ്യുക.
\end{malayalam}}
\flushright{\begin{Arabic}
\quranayah[76][26]
\end{Arabic}}
\flushleft{\begin{malayalam}
രാത്രിയില്‍ നീ അവനെ പ്രണമിക്കുകയും ദീര്‍ഘമായ നിശാവേളയില്‍ അവനെ പ്രകീര്‍ത്തിക്കുകയും ചെയ്യുക.
\end{malayalam}}
\flushright{\begin{Arabic}
\quranayah[76][27]
\end{Arabic}}
\flushleft{\begin{malayalam}
തീര്‍ച്ചയായും ഇക്കൂട്ടര്‍ ക്ഷണികമായ ഐഹികജീവിതത്തെ ഇഷ്ടപ്പെടുന്നു. ഭാരമേറിയ ഒരു ദിവസത്തിന്‍റെ കാര്യം അവര്‍ തങ്ങളുടെ പുറകില്‍ വിട്ടുകളയുകയും ചെയ്യുന്നു.
\end{malayalam}}
\flushright{\begin{Arabic}
\quranayah[76][28]
\end{Arabic}}
\flushleft{\begin{malayalam}
നാമാണ് അവരെ സൃഷ്ടിക്കുകയും അവരുടെ ശരീരഘടന ബലപ്പെടുത്തുകയും ചെയ്തത്‌. നാം ഉദ്ദേശിക്കുന്ന പക്ഷം അവര്‍ക്ക് തുല്യരായിട്ടുള്ളവരെ നാം അവര്‍ക്കു പകരം കൊണ്ടു വരുന്നതുമാണ്‌.
\end{malayalam}}
\flushright{\begin{Arabic}
\quranayah[76][29]
\end{Arabic}}
\flushleft{\begin{malayalam}
തീര്‍ച്ചയായും ഇത് ഒരു ഉല്‍ബോധനമാകുന്നു. ആകയാല്‍ വല്ലവനും ഉദ്ദേശിക്കുന്ന പക്ഷം തന്‍റെ രക്ഷിതാവിങ്കലേക്കുള്ള മാര്‍ഗം സ്വീകരിച്ചു കൊള്ളട്ടെ.
\end{malayalam}}
\flushright{\begin{Arabic}
\quranayah[76][30]
\end{Arabic}}
\flushleft{\begin{malayalam}
അല്ലാഹു ഉദ്ദേശിക്കുന്ന പക്ഷമല്ലാതെ നിങ്ങള്‍ ഉദ്ദേശിക്കുകയില്ല. തീര്‍ച്ചയായും അല്ലാഹു സര്‍വ്വജ്ഞനും യുക്തിമാനുമാകുന്നു.
\end{malayalam}}
\flushright{\begin{Arabic}
\quranayah[76][31]
\end{Arabic}}
\flushleft{\begin{malayalam}
അവന്‍ ഉദ്ദേശിക്കുന്നവരെ അവന്‍റെ കാരുണ്യത്തില്‍ അവന്‍ പ്രവേശിപ്പിക്കുന്നതാണ്‌. അക്രമകാരികള്‍ക്കാവട്ടെ അവന്‍ വേദനയേറിയ ശിക്ഷ ഒരുക്കി വെച്ചിരിക്കുന്നു.
\end{malayalam}}
\chapter{\textmalayalam{മുര്‍സലാത്ത് ( അയക്കപ്പെടുന്നവര്‍ )}}
\begin{Arabic}
\Huge{\centerline{\basmalah}}\end{Arabic}
\flushright{\begin{Arabic}
\quranayah[77][1]
\end{Arabic}}
\flushleft{\begin{malayalam}
തുടരെത്തുടരെ അയക്കപ്പെടുന്നവയും,
\end{malayalam}}
\flushright{\begin{Arabic}
\quranayah[77][2]
\end{Arabic}}
\flushleft{\begin{malayalam}
ശക്തിയായി ആഞ്ഞടിക്കുന്നവയും,
\end{malayalam}}
\flushright{\begin{Arabic}
\quranayah[77][3]
\end{Arabic}}
\flushleft{\begin{malayalam}
പരക്കെ വ്യാപിപ്പിക്കുന്നവയും,
\end{malayalam}}
\flushright{\begin{Arabic}
\quranayah[77][4]
\end{Arabic}}
\flushleft{\begin{malayalam}
വേര്‍തിരിച്ചു വിവേചനം ചെയ്യുന്നവയും,
\end{malayalam}}
\flushright{\begin{Arabic}
\quranayah[77][5]
\end{Arabic}}
\flushleft{\begin{malayalam}
ദിവ്യസന്ദേശം ഇട്ടുകൊടുക്കുന്നവയുമായിട്ടുള്ളവയെ തന്നെയാകുന്നു സത്യം;
\end{malayalam}}
\flushright{\begin{Arabic}
\quranayah[77][6]
\end{Arabic}}
\flushleft{\begin{malayalam}
ഒരു ഒഴികഴിവായികൊണ്ടോ താക്കീതായിക്കൊണ്ടോ
\end{malayalam}}
\flushright{\begin{Arabic}
\quranayah[77][7]
\end{Arabic}}
\flushleft{\begin{malayalam}
തീര്‍ച്ചയായും നിങ്ങളോട് താക്കീത് ചെയ്യപ്പെടുന്ന കാര്യം സംഭവിക്കുന്നതു തന്നെയാകുന്നു.
\end{malayalam}}
\flushright{\begin{Arabic}
\quranayah[77][8]
\end{Arabic}}
\flushleft{\begin{malayalam}
നക്ഷത്രങ്ങളുടെ പ്രകാശം മായ്ക്കപ്പെടുകയും,
\end{malayalam}}
\flushright{\begin{Arabic}
\quranayah[77][9]
\end{Arabic}}
\flushleft{\begin{malayalam}
ആകാശം പിളര്‍ത്തപ്പെടുകയും,
\end{malayalam}}
\flushright{\begin{Arabic}
\quranayah[77][10]
\end{Arabic}}
\flushleft{\begin{malayalam}
പര്‍വ്വതങ്ങള്‍ പൊടിക്കപ്പെടുകയും,
\end{malayalam}}
\flushright{\begin{Arabic}
\quranayah[77][11]
\end{Arabic}}
\flushleft{\begin{malayalam}
ദൂതന്‍മാര്‍ക്ക് സമയം നിര്‍ണയിച്ചു കൊടുക്കപ്പെടുകയും ചെയ്താല്‍!
\end{malayalam}}
\flushright{\begin{Arabic}
\quranayah[77][12]
\end{Arabic}}
\flushleft{\begin{malayalam}
ഏതൊരു ദിവസത്തേക്കാണ് അവര്‍ക്ക് അവധി നിശ്ചയിക്കപ്പെട്ടിരിക്കുന്നത്‌?
\end{malayalam}}
\flushright{\begin{Arabic}
\quranayah[77][13]
\end{Arabic}}
\flushleft{\begin{malayalam}
തീരുമാനത്തിന്‍റെ ദിവസത്തേക്ക്‌!
\end{malayalam}}
\flushright{\begin{Arabic}
\quranayah[77][14]
\end{Arabic}}
\flushleft{\begin{malayalam}
ആ തീരുമാനത്തിന്‍റെ ദിവസം എന്താണെന്ന് നിനക്കറിയുമോ?
\end{malayalam}}
\flushright{\begin{Arabic}
\quranayah[77][15]
\end{Arabic}}
\flushleft{\begin{malayalam}
അന്നേ ദിവസം നിഷേധിച്ചു തള്ളിയവര്‍ക്കാകുന്നു നാശം.
\end{malayalam}}
\flushright{\begin{Arabic}
\quranayah[77][16]
\end{Arabic}}
\flushleft{\begin{malayalam}
പൂര്‍വ്വികന്‍മാരെ നാം നശിപ്പിച്ചു കളഞ്ഞില്ലേ?
\end{malayalam}}
\flushright{\begin{Arabic}
\quranayah[77][17]
\end{Arabic}}
\flushleft{\begin{malayalam}
പിന്നീട് പിന്‍ഗാമികളെയും അവരുടെ പിന്നാലെ നാം അയക്കുന്നതാണ്‌.
\end{malayalam}}
\flushright{\begin{Arabic}
\quranayah[77][18]
\end{Arabic}}
\flushleft{\begin{malayalam}
അപ്രകാരമാണ് നാം കുറ്റവാളികളെക്കൊണ്ട് പ്രവര്‍ത്തിക്കുക.
\end{malayalam}}
\flushright{\begin{Arabic}
\quranayah[77][19]
\end{Arabic}}
\flushleft{\begin{malayalam}
അന്നേ ദിവസം നിഷേധിച്ചു തള്ളിയവര്‍ക്കായിരിക്കും നാശം.
\end{malayalam}}
\flushright{\begin{Arabic}
\quranayah[77][20]
\end{Arabic}}
\flushleft{\begin{malayalam}
നിസ്സാരപ്പെട്ട ഒരു ദ്രാവകത്തില്‍ നിന്ന് നിങ്ങളെ നാം സൃഷ്ടിച്ചില്ലേ?
\end{malayalam}}
\flushright{\begin{Arabic}
\quranayah[77][21]
\end{Arabic}}
\flushleft{\begin{malayalam}
എന്നിട്ട് നാം അതിനെ ഭദ്രമായ ഒരു സങ്കേതത്തില്‍ വെച്ചു.
\end{malayalam}}
\flushright{\begin{Arabic}
\quranayah[77][22]
\end{Arabic}}
\flushleft{\begin{malayalam}
നിശ്ചിതമായ ഒരു അവധി വരെ.
\end{malayalam}}
\flushright{\begin{Arabic}
\quranayah[77][23]
\end{Arabic}}
\flushleft{\begin{malayalam}
അങ്ങനെ നാം (എല്ലാം) നിര്‍ണയിച്ചു. അപ്പോള്‍ നാം എത്ര നല്ല നിര്‍ണയക്കാരന്‍!
\end{malayalam}}
\flushright{\begin{Arabic}
\quranayah[77][24]
\end{Arabic}}
\flushleft{\begin{malayalam}
അന്നേ ദിവസം നിഷേധിച്ചു തള്ളിയവര്‍ക്കാകുന്നു നാശം.
\end{malayalam}}
\flushright{\begin{Arabic}
\quranayah[77][25]
\end{Arabic}}
\flushleft{\begin{malayalam}
ഭൂമിയെ നാം ഉള്‍കൊള്ളുന്നതാക്കിയില്ലേ?
\end{malayalam}}
\flushright{\begin{Arabic}
\quranayah[77][26]
\end{Arabic}}
\flushleft{\begin{malayalam}
മരിച്ചവരെയും ജീവിച്ചിരിക്കുന്നവരെയും.
\end{malayalam}}
\flushright{\begin{Arabic}
\quranayah[77][27]
\end{Arabic}}
\flushleft{\begin{malayalam}
അതില്‍ ഉന്നതങ്ങളായി ഉറച്ചുനില്‍ക്കുന്ന പര്‍വ്വതങ്ങളെ നാം വെക്കുകയും ചെയ്തിരിക്കുന്നു. നിങ്ങള്‍ക്കു നാം സ്വച്ഛജലം കുടിക്കാന്‍ തരികയും ചെയ്തിരിക്കുന്നു.
\end{malayalam}}
\flushright{\begin{Arabic}
\quranayah[77][28]
\end{Arabic}}
\flushleft{\begin{malayalam}
അന്നേ ദിവസം നിഷേധിച്ചു തള്ളിയവര്‍ക്കാകുന്നു നാശം.
\end{malayalam}}
\flushright{\begin{Arabic}
\quranayah[77][29]
\end{Arabic}}
\flushleft{\begin{malayalam}
(ഹേ, സത്യനിഷേധികളേ,) എന്തൊന്നിനെയായിരുന്നോ നിങ്ങള്‍ നിഷേധിച്ചു തള്ളിയിരുന്നത് അതിലേക്ക് നിങ്ങള്‍ പോയി ക്കൊള്ളുക.
\end{malayalam}}
\flushright{\begin{Arabic}
\quranayah[77][30]
\end{Arabic}}
\flushleft{\begin{malayalam}
മൂന്ന് ശാഖകളുള്ള ഒരു തരം തണലിലേക്ക് നിങ്ങള്‍ പോയിക്കൊള്ളുക.
\end{malayalam}}
\flushright{\begin{Arabic}
\quranayah[77][31]
\end{Arabic}}
\flushleft{\begin{malayalam}
അത് തണല്‍ നല്‍കുന്നതല്ല. തീജ്വാലയില്‍ നിന്ന് സംരക്ഷണം നല്‍കുന്നതുമല്ല.
\end{malayalam}}
\flushright{\begin{Arabic}
\quranayah[77][32]
\end{Arabic}}
\flushleft{\begin{malayalam}
തീര്‍ച്ചയായും അത് (നരകം) വലിയ കെട്ടിടം പോലെ ഉയരമുള്ള തീപ്പൊരി തെറിപ്പിച്ചു കൊണ്ടിരിക്കും.
\end{malayalam}}
\flushright{\begin{Arabic}
\quranayah[77][33]
\end{Arabic}}
\flushleft{\begin{malayalam}
അത് (തീപ്പൊരി) മഞ്ഞനിറമുള്ള ഒട്ടക കൂട്ടങ്ങളെപ്പോലെയായിരിക്കും.
\end{malayalam}}
\flushright{\begin{Arabic}
\quranayah[77][34]
\end{Arabic}}
\flushleft{\begin{malayalam}
അന്നേ ദിവസം നിഷേധിച്ചു തള്ളിയവര്‍ക്കാകുന്നു നാശം.
\end{malayalam}}
\flushright{\begin{Arabic}
\quranayah[77][35]
\end{Arabic}}
\flushleft{\begin{malayalam}
അവര്‍ മിണ്ടാത്തതായ ദിവസമാകുന്നു ഇത്‌.
\end{malayalam}}
\flushright{\begin{Arabic}
\quranayah[77][36]
\end{Arabic}}
\flushleft{\begin{malayalam}
അവര്‍ക്ക് ഒഴികഴിവു ബോധിപ്പിക്കാന്‍ അനുവാദം നല്‍കപ്പെടുകയുമില്ല.
\end{malayalam}}
\flushright{\begin{Arabic}
\quranayah[77][37]
\end{Arabic}}
\flushleft{\begin{malayalam}
അന്നേ ദിവസം നിഷേധിച്ചു തള്ളിയവര്‍ക്കാകുന്നു നാശം.
\end{malayalam}}
\flushright{\begin{Arabic}
\quranayah[77][38]
\end{Arabic}}
\flushleft{\begin{malayalam}
(അന്നവരോട് പറയപ്പെടും:) തീരുമാനത്തിന്‍റെ ദിവസമാണിത്‌. നിങ്ങളെയും പൂര്‍വ്വികന്‍മാരെയും നാം ഇതാ ഒരുമിച്ചുകൂട്ടിയിരിക്കുന്നു.
\end{malayalam}}
\flushright{\begin{Arabic}
\quranayah[77][39]
\end{Arabic}}
\flushleft{\begin{malayalam}
ഇനി നിങ്ങള്‍ക്ക് വല്ല തന്ത്രവും പ്രയോഗിക്കാനുണ്ടെങ്കില്‍ ആ തന്ത്രം പ്രയോഗിച്ചു കൊള്ളുക.
\end{malayalam}}
\flushright{\begin{Arabic}
\quranayah[77][40]
\end{Arabic}}
\flushleft{\begin{malayalam}
അന്നേ ദിവസം നിഷേധിച്ചു തള്ളിയവര്‍ക്കാകുന്നു നാശം.
\end{malayalam}}
\flushright{\begin{Arabic}
\quranayah[77][41]
\end{Arabic}}
\flushleft{\begin{malayalam}
തീര്‍ച്ചയായും സൂക്ഷ്മത പാലിച്ചവര്‍ (സ്വര്‍ഗത്തില്‍) തണലുകളിലും അരുവികള്‍ക്കിടയിലുമാകുന്നു.
\end{malayalam}}
\flushright{\begin{Arabic}
\quranayah[77][42]
\end{Arabic}}
\flushleft{\begin{malayalam}
അവര്‍ ഇഷ്ടപ്പെടുന്ന തരത്തിലുള്ള പഴവര്‍ഗങ്ങള്‍ക്കിടയിലും.
\end{malayalam}}
\flushright{\begin{Arabic}
\quranayah[77][43]
\end{Arabic}}
\flushleft{\begin{malayalam}
(അവരോട് പറയപ്പെടും:) നിങ്ങള്‍ പ്രവര്‍ത്തിച്ചിരുന്നതിന്‍റെ ഫലമായി ആഹ്ലാദത്തോടെ നിങ്ങള്‍ തിന്നുകയും കുടിക്കുകയും ചെയ്തുകൊള്ളുക.
\end{malayalam}}
\flushright{\begin{Arabic}
\quranayah[77][44]
\end{Arabic}}
\flushleft{\begin{malayalam}
തീര്‍ച്ചയായും നാം അപ്രകാരമാകുന്നു സദ്‌വൃത്തര്‍ക്ക് പ്രതിഫലം നല്‍കുന്നത്‌.
\end{malayalam}}
\flushright{\begin{Arabic}
\quranayah[77][45]
\end{Arabic}}
\flushleft{\begin{malayalam}
അന്നേ ദിവസം നിഷേധിച്ചു തള്ളിയവര്‍ക്കാകുന്നു നാശം.
\end{malayalam}}
\flushright{\begin{Arabic}
\quranayah[77][46]
\end{Arabic}}
\flushleft{\begin{malayalam}
(അവരോട് പറയപ്പെടും:) നിങ്ങള്‍ അല്‍പം തിന്നുകയും സുഖമനുഭവിക്കുകയും ചെയ്തു കൊള്ളുക. തീര്‍ച്ചയായും നിങ്ങള്‍ കുറ്റവാളികളാകുന്നു.
\end{malayalam}}
\flushright{\begin{Arabic}
\quranayah[77][47]
\end{Arabic}}
\flushleft{\begin{malayalam}
അന്നേ ദിവസം നിഷേധിച്ചു തള്ളിയവര്‍ക്കാകുന്നു നാശം.
\end{malayalam}}
\flushright{\begin{Arabic}
\quranayah[77][48]
\end{Arabic}}
\flushleft{\begin{malayalam}
അവരോട് കുമ്പിടൂ എന്ന് പറയപ്പെട്ടാല്‍ അവര്‍ കുമ്പിടുകയില്ല.
\end{malayalam}}
\flushright{\begin{Arabic}
\quranayah[77][49]
\end{Arabic}}
\flushleft{\begin{malayalam}
അന്നേ ദിവസം നിഷേധിച്ചു തള്ളിയവര്‍ക്കാകുന്നു നാശം.
\end{malayalam}}
\flushright{\begin{Arabic}
\quranayah[77][50]
\end{Arabic}}
\flushleft{\begin{malayalam}
ഇനി ഇതിന് (ഖുര്‍ആന്ന്‌) ശേഷം ഏതൊരു വര്‍ത്തമാനത്തിലാണ് അവര്‍ വിശ്വസിക്കുന്നത്‌?
\end{malayalam}}
\chapter{\textmalayalam{നബഅ് ( വൃത്താന്തം )}}
\begin{Arabic}
\Huge{\centerline{\basmalah}}\end{Arabic}
\flushright{\begin{Arabic}
\quranayah[78][1]
\end{Arabic}}
\flushleft{\begin{malayalam}
എന്തിനെപ്പറ്റിയാണ് അവര്‍ പരസ്പരം ചോദിച്ചു കൊണ്ടിരിക്കുന്നത്‌?
\end{malayalam}}
\flushright{\begin{Arabic}
\quranayah[78][2]
\end{Arabic}}
\flushleft{\begin{malayalam}
ആ മഹത്തായ വൃത്താന്തത്തെപ്പറ്റി.
\end{malayalam}}
\flushright{\begin{Arabic}
\quranayah[78][3]
\end{Arabic}}
\flushleft{\begin{malayalam}
അവര്‍ ഏതൊരു കാര്യത്തില്‍ അഭിപ്രായവ്യത്യാസത്തിലായി ക്കൊണ്ടിരിക്കുന്നുവോ അതിനെപ്പറ്റി.
\end{malayalam}}
\flushright{\begin{Arabic}
\quranayah[78][4]
\end{Arabic}}
\flushleft{\begin{malayalam}
നിസ്സംശയം; അവര്‍ വഴിയെ അറിഞ്ഞു കൊള്ളും.
\end{malayalam}}
\flushright{\begin{Arabic}
\quranayah[78][5]
\end{Arabic}}
\flushleft{\begin{malayalam}
വീണ്ടും നിസ്സംശയം; അവര്‍ വഴിയെ അറിഞ്ഞു കൊള്ളും.
\end{malayalam}}
\flushright{\begin{Arabic}
\quranayah[78][6]
\end{Arabic}}
\flushleft{\begin{malayalam}
ഭൂമിയെ നാം ഒരു വിരിപ്പാക്കിയില്ലേ?
\end{malayalam}}
\flushright{\begin{Arabic}
\quranayah[78][7]
\end{Arabic}}
\flushleft{\begin{malayalam}
പര്‍വ്വതങ്ങളെ ആണികളാക്കുകയും (ചെയ്തില്ലേ?)
\end{malayalam}}
\flushright{\begin{Arabic}
\quranayah[78][8]
\end{Arabic}}
\flushleft{\begin{malayalam}
നിങ്ങളെ നാം ഇണകളായി സൃഷ്ടിക്കുകയും ചെയ്തിരിക്കുന്നു.
\end{malayalam}}
\flushright{\begin{Arabic}
\quranayah[78][9]
\end{Arabic}}
\flushleft{\begin{malayalam}
നിങ്ങളുടെ ഉറക്കത്തെ നാം വിശ്രമമാക്കുകയും ചെയ്തിരിക്കുന്നു.
\end{malayalam}}
\flushright{\begin{Arabic}
\quranayah[78][10]
\end{Arabic}}
\flushleft{\begin{malayalam}
രാത്രിയെ നാം ഒരു വസ്ത്രമാക്കുകയും,
\end{malayalam}}
\flushright{\begin{Arabic}
\quranayah[78][11]
\end{Arabic}}
\flushleft{\begin{malayalam}
പകലിനെ നാം ജീവസന്ധാരണവേളയാക്കുകയും ചെയ്തിരിക്കുന്നു.
\end{malayalam}}
\flushright{\begin{Arabic}
\quranayah[78][12]
\end{Arabic}}
\flushleft{\begin{malayalam}
നിങ്ങള്‍ക്ക് മീതെ ബലിഷ്ഠമായ ഏഴു ആകാശങ്ങള്‍ നാം നിര്‍മിക്കുകയും
\end{malayalam}}
\flushright{\begin{Arabic}
\quranayah[78][13]
\end{Arabic}}
\flushleft{\begin{malayalam}
കത്തിജ്വലിക്കുന്ന ഒരു വിളക്ക് നാം ഉണ്ടാക്കുകയും ചെയ്തിരിക്കുന്നു.
\end{malayalam}}
\flushright{\begin{Arabic}
\quranayah[78][14]
\end{Arabic}}
\flushleft{\begin{malayalam}
കാര്‍മേഘങ്ങളില്‍ നിന്ന് കുത്തി ഒഴുകുന്ന വെള്ളം നാം ഇറക്കുകയും ചെയ്തു.
\end{malayalam}}
\flushright{\begin{Arabic}
\quranayah[78][15]
\end{Arabic}}
\flushleft{\begin{malayalam}
അതു മുഖേന ധാന്യവും സസ്യവും നാം പുറത്തു കൊണ്ടു വരാന്‍ വേണ്ടി.
\end{malayalam}}
\flushright{\begin{Arabic}
\quranayah[78][16]
\end{Arabic}}
\flushleft{\begin{malayalam}
ഇടതൂര്‍ന്ന തോട്ടങ്ങളും
\end{malayalam}}
\flushright{\begin{Arabic}
\quranayah[78][17]
\end{Arabic}}
\flushleft{\begin{malayalam}
തീര്‍ച്ചയായും തീരുമാനത്തിന്‍റെ ദിവസം സമയം നിര്‍ണയിക്കപ്പെട്ടതായിരിക്കുന്നു.
\end{malayalam}}
\flushright{\begin{Arabic}
\quranayah[78][18]
\end{Arabic}}
\flushleft{\begin{malayalam}
അതായത് കാഹളത്തില്‍ ഊതപ്പെടുകയും, നിങ്ങള്‍ കൂട്ടംകൂട്ടമായി വന്നെത്തുകയും ചെയ്യുന്ന ദിവസം.
\end{malayalam}}
\flushright{\begin{Arabic}
\quranayah[78][19]
\end{Arabic}}
\flushleft{\begin{malayalam}
ആകാശം തുറക്കപ്പെടുകയും എന്നിട്ടത് പല കവാടങ്ങളായി തീരുകയും ചെയ്യും.
\end{malayalam}}
\flushright{\begin{Arabic}
\quranayah[78][20]
\end{Arabic}}
\flushleft{\begin{malayalam}
പര്‍വ്വതങ്ങള്‍ സഞ്ചരിപ്പിക്കപ്പെടുകയും അങ്ങനെ അവ മരീചിക പോലെ ആയിത്തീരുകയും ചെയ്യും.
\end{malayalam}}
\flushright{\begin{Arabic}
\quranayah[78][21]
\end{Arabic}}
\flushleft{\begin{malayalam}
തീര്‍ച്ചയായും നരകം കാത്തിരിക്കുന്ന സ്ഥലമാകുന്നു.
\end{malayalam}}
\flushright{\begin{Arabic}
\quranayah[78][22]
\end{Arabic}}
\flushleft{\begin{malayalam}
അതിക്രമകാരികള്‍ക്ക് മടങ്ങിച്ചെല്ലാനുള്ള സ്ഥലം.
\end{malayalam}}
\flushright{\begin{Arabic}
\quranayah[78][23]
\end{Arabic}}
\flushleft{\begin{malayalam}
അവര്‍ അതില്‍ യുഗങ്ങളോളം താമസിക്കുന്നവരായിരിക്കും.
\end{malayalam}}
\flushright{\begin{Arabic}
\quranayah[78][24]
\end{Arabic}}
\flushleft{\begin{malayalam}
കുളിര്‍മയോ കുടിനീരോ അവര്‍ അവിടെ ആസ്വദിക്കുകയില്ല.
\end{malayalam}}
\flushright{\begin{Arabic}
\quranayah[78][25]
\end{Arabic}}
\flushleft{\begin{malayalam}
കൊടുംചൂടുള്ള വെള്ളവും കൊടും തണുപ്പുള്ള വെള്ളവുമല്ലാതെ
\end{malayalam}}
\flushright{\begin{Arabic}
\quranayah[78][26]
\end{Arabic}}
\flushleft{\begin{malayalam}
അനുയോജ്യമായ പ്രതിഫലമത്രെ അത്‌.
\end{malayalam}}
\flushright{\begin{Arabic}
\quranayah[78][27]
\end{Arabic}}
\flushleft{\begin{malayalam}
തീര്‍ച്ചയായും അവര്‍ വിചാരണ പ്രതീക്ഷിക്കുന്നില്ലായിരുന്നു.
\end{malayalam}}
\flushright{\begin{Arabic}
\quranayah[78][28]
\end{Arabic}}
\flushleft{\begin{malayalam}
നമ്മുടെ ദൃഷ്ടാന്തങ്ങളെ അവര്‍ തീര്‍ത്തും നിഷേധിച്ചു തള്ളുകയും ചെയ്തു.
\end{malayalam}}
\flushright{\begin{Arabic}
\quranayah[78][29]
\end{Arabic}}
\flushleft{\begin{malayalam}
ഏതു കാര്യവും നാം എഴുതി തിട്ടപ്പെടുത്തിയിരിക്കുന്നു.
\end{malayalam}}
\flushright{\begin{Arabic}
\quranayah[78][30]
\end{Arabic}}
\flushleft{\begin{malayalam}
അതിനാല്‍ നിങ്ങള്‍ (ശിക്ഷ) ആസ്വദിച്ചു കൊള്ളുക. തീര്‍ച്ചയായും നാം നിങ്ങള്‍ക്കു ശിക്ഷയല്ലാതൊന്നും വര്‍ദ്ധിപ്പിച്ചു തരികയില്ല.
\end{malayalam}}
\flushright{\begin{Arabic}
\quranayah[78][31]
\end{Arabic}}
\flushleft{\begin{malayalam}
തീര്‍ച്ചയായും സൂക്ഷ്മത പാലിച്ചവര്‍ക്ക് വിജയമുണ്ട്‌.
\end{malayalam}}
\flushright{\begin{Arabic}
\quranayah[78][32]
\end{Arabic}}
\flushleft{\begin{malayalam}
അതായത് (സ്വര്‍ഗത്തിലെ) തോട്ടങ്ങളും മുന്തിരികളും,
\end{malayalam}}
\flushright{\begin{Arabic}
\quranayah[78][33]
\end{Arabic}}
\flushleft{\begin{malayalam}
തുടുത്ത മാര്‍വിടമുള്ള സമപ്രായക്കാരായ തരുണികളും.
\end{malayalam}}
\flushright{\begin{Arabic}
\quranayah[78][34]
\end{Arabic}}
\flushleft{\begin{malayalam}
നിറഞ്ഞ പാനപാത്രങ്ങളും.
\end{malayalam}}
\flushright{\begin{Arabic}
\quranayah[78][35]
\end{Arabic}}
\flushleft{\begin{malayalam}
അവിടെ അനാവശ്യമായ ഒരു വാക്കോ ഒരു വ്യാജവാര്‍ത്തയോ അവര്‍ കേള്‍ക്കുകയില്ല.
\end{malayalam}}
\flushright{\begin{Arabic}
\quranayah[78][36]
\end{Arabic}}
\flushleft{\begin{malayalam}
(അത്‌) നിന്‍റെ രക്ഷിതാവിങ്കല്‍ നിന്നുള്ള ഒരു പ്രതിഫലവും, കണക്കൊത്ത ഒരു സമ്മാനവുമാകുന്നു.
\end{malayalam}}
\flushright{\begin{Arabic}
\quranayah[78][37]
\end{Arabic}}
\flushleft{\begin{malayalam}
ആകാശങ്ങളുടെയും ഭൂമിയുടെയും അവക്കിടയിലുള്ളതിന്‍റെയും രക്ഷിതാവും പരമകാരുണികനുമായുള്ളവന്‍റെ (സമ്മാനം.) അവനുമായി സംഭാഷണത്തില്‍ ഏര്‍പെടാന്‍ അവര്‍ക്കു സാധിക്കുകയില്ല.
\end{malayalam}}
\flushright{\begin{Arabic}
\quranayah[78][38]
\end{Arabic}}
\flushleft{\begin{malayalam}
റൂഹും മലക്കുകളും അണിയായി നില്‍ക്കുന്ന ദിവസം. പരമകാരുണികനായ അല്ലാഹു അനുവാദം നല്‍കിയിട്ടുള്ളവനും സത്യം പറഞ്ഞിട്ടുള്ളവനുമല്ലാതെ അന്ന് സംസാരിക്കുകയില്ല.
\end{malayalam}}
\flushright{\begin{Arabic}
\quranayah[78][39]
\end{Arabic}}
\flushleft{\begin{malayalam}
അതത്രെ യഥാര്‍ത്ഥമായ ദിവസം. അതിനാല്‍ വല്ലവനും ഉദ്ദേശിക്കുന്ന പക്ഷം തന്‍റെ രക്ഷിതാവിങ്കലേക്കുള്ള മടക്കത്തിന്‍റെ മാര്‍ഗം അവന്‍ സ്വീകരിക്കട്ടെ.
\end{malayalam}}
\flushright{\begin{Arabic}
\quranayah[78][40]
\end{Arabic}}
\flushleft{\begin{malayalam}
ആസന്നമായ ഒരു ശിക്ഷയെ പറ്റി തീര്‍ച്ചയായും നിങ്ങള്‍ക്കു നാം മുന്നറിയിപ്പ് നല്‍കിയിരിക്കുന്നു. മനുഷ്യന്‍ തന്‍റെ കൈകള്‍ മുന്‍കൂട്ടി ചെയ്തു വെച്ചത് നോക്കിക്കാണുകയും, അയ്യോ ഞാന്‍ മണ്ണായിരുന്നെങ്കില്‍ എത്ര നന്നായിരുന്നേനെ എന്ന് സത്യനിഷേധി പറയുകയും ചെയ്യുന്ന ദിവസം.
\end{malayalam}}
\chapter{\textmalayalam{നാസിയാത്ത് ( ഊരിയെടുക്കുന്നവ )}}
\begin{Arabic}
\Huge{\centerline{\basmalah}}\end{Arabic}
\flushright{\begin{Arabic}
\quranayah[79][1]
\end{Arabic}}
\flushleft{\begin{malayalam}
(അവിശ്വാസികളിലേക്ക്‌) ഇറങ്ങിച്ചെന്ന് (അവരുടെ ആത്മാവുകളെ) ഊരിയെടുക്കുന്നവ തന്നെയാണ സത്യം.
\end{malayalam}}
\flushright{\begin{Arabic}
\quranayah[79][2]
\end{Arabic}}
\flushleft{\begin{malayalam}
(സത്യവിശ്വാസികളുടെ ആത്മാവുകളെ) സൌമ്യതയോടെ പുറത്തെടുക്കുന്നവ തന്നെയാണ, സത്യം.
\end{malayalam}}
\flushright{\begin{Arabic}
\quranayah[79][3]
\end{Arabic}}
\flushleft{\begin{malayalam}
ഊക്കോടെ ഒഴുകി വരുന്നവ തന്നെയാണ, സത്യം.
\end{malayalam}}
\flushright{\begin{Arabic}
\quranayah[79][4]
\end{Arabic}}
\flushleft{\begin{malayalam}
എന്നിട്ടു മുന്നോട്ടു കുതിച്ചു പോകുന്നവ തന്നെയാണ, സത്യം.
\end{malayalam}}
\flushright{\begin{Arabic}
\quranayah[79][5]
\end{Arabic}}
\flushleft{\begin{malayalam}
കാര്യം നിയന്ത്രിക്കുന്നവയും തന്നെയാണ, സത്യം.
\end{malayalam}}
\flushright{\begin{Arabic}
\quranayah[79][6]
\end{Arabic}}
\flushleft{\begin{malayalam}
ആ നടുക്കുന്ന സംഭവം നടുക്കമുണ്ടാക്കുന്ന ദിവസം.
\end{malayalam}}
\flushright{\begin{Arabic}
\quranayah[79][7]
\end{Arabic}}
\flushleft{\begin{malayalam}
അതിനെ തുടര്‍ന്ന് അതിന്‍റെ പിന്നാലെ മറ്റൊന്നും
\end{malayalam}}
\flushright{\begin{Arabic}
\quranayah[79][8]
\end{Arabic}}
\flushleft{\begin{malayalam}
ചില ഹൃദയങ്ങള്‍ അന്നു വിറച്ചു കൊണ്ടിരിക്കും.
\end{malayalam}}
\flushright{\begin{Arabic}
\quranayah[79][9]
\end{Arabic}}
\flushleft{\begin{malayalam}
അവയുടെ കണ്ണുകള്‍ അന്ന് കീഴ്പോട്ടു താഴ്ന്നിരിക്കും.
\end{malayalam}}
\flushright{\begin{Arabic}
\quranayah[79][10]
\end{Arabic}}
\flushleft{\begin{malayalam}
അവര്‍ പറയും: തീര്‍ച്ചയായും നാം (നമ്മുടെ) മുന്‍സ്ഥിതിയിലേക്ക് മടക്കപ്പെടുന്നവരാണോ?
\end{malayalam}}
\flushright{\begin{Arabic}
\quranayah[79][11]
\end{Arabic}}
\flushleft{\begin{malayalam}
നാം ജീര്‍ണിച്ച എല്ലുകളായി കഴിഞ്ഞാലും (നമുക്ക് മടക്കമോ?)
\end{malayalam}}
\flushright{\begin{Arabic}
\quranayah[79][12]
\end{Arabic}}
\flushleft{\begin{malayalam}
അവര്‍ പറയുകയാണ്‌: അങ്ങനെയാണെങ്കില്‍ നഷ്ടകരമായ ഒരു തിരിച്ചുവരവായിരിക്കും അത്‌.
\end{malayalam}}
\flushright{\begin{Arabic}
\quranayah[79][13]
\end{Arabic}}
\flushleft{\begin{malayalam}
അത് ഒരേയൊരു ഘോരശബ്ദം മാത്രമായിരിക്കും.
\end{malayalam}}
\flushright{\begin{Arabic}
\quranayah[79][14]
\end{Arabic}}
\flushleft{\begin{malayalam}
അപ്പോഴതാ അവര്‍ ഭൂമുഖത്തെത്തിക്കഴിഞ്ഞു.
\end{malayalam}}
\flushright{\begin{Arabic}
\quranayah[79][15]
\end{Arabic}}
\flushleft{\begin{malayalam}
മൂസാനബിയുടെ വര്‍ത്തമാനം നിനക്ക് വന്നെത്തിയോ?
\end{malayalam}}
\flushright{\begin{Arabic}
\quranayah[79][16]
\end{Arabic}}
\flushleft{\begin{malayalam}
ത്വുവാ എന്ന പരിശുദ്ധ താഴ്‌വരയില്‍ വെച്ച് അദ്ദേഹത്തിന്‍റെ രക്ഷിതാവ് അദ്ദേഹത്തെ വിളിച്ച് ഇപ്രകാരം പറഞ്ഞ സന്ദര്‍ഭം:
\end{malayalam}}
\flushright{\begin{Arabic}
\quranayah[79][17]
\end{Arabic}}
\flushleft{\begin{malayalam}
നീ ഫിര്‍ഔന്‍റെ അടുത്തേക്കു പോകുക. തീര്‍ച്ചയായും അവന്‍ അതിരുകവിഞ്ഞിരിക്കുന്നു.
\end{malayalam}}
\flushright{\begin{Arabic}
\quranayah[79][18]
\end{Arabic}}
\flushleft{\begin{malayalam}
എന്നിട്ട് ചോദിക്കുക: നീ പരിശുദ്ധി പ്രാപിക്കാന്‍ തയ്യാറുണ്ടോ?
\end{malayalam}}
\flushright{\begin{Arabic}
\quranayah[79][19]
\end{Arabic}}
\flushleft{\begin{malayalam}
നിന്‍റെ രക്ഷിതാവിങ്കലേക്ക് നിനക്ക് ഞാന്‍ വഴി കാണിച്ചുതരാം. എന്നിട്ട് നീ ഭയപ്പെടാനും (തയ്യാറുണ്ടോ?)
\end{malayalam}}
\flushright{\begin{Arabic}
\quranayah[79][20]
\end{Arabic}}
\flushleft{\begin{malayalam}
അങ്ങനെ അദ്ദേഹം (മൂസാ) അവന്ന് ആ മഹത്തായ ദൃഷ്ടാന്തം കാണിച്ചുകൊടുത്തു.
\end{malayalam}}
\flushright{\begin{Arabic}
\quranayah[79][21]
\end{Arabic}}
\flushleft{\begin{malayalam}
അപ്പോള്‍ അവന്‍ നിഷേധിച്ചു തള്ളുകയും ധിക്കരിക്കുകയും ചെയ്തു.
\end{malayalam}}
\flushright{\begin{Arabic}
\quranayah[79][22]
\end{Arabic}}
\flushleft{\begin{malayalam}
പിന്നെ, അവന്‍ എതിര്‍ ശ്രമങ്ങള്‍ നടത്തുവാനായി പിന്തിരിഞ്ഞു പോയി.
\end{malayalam}}
\flushright{\begin{Arabic}
\quranayah[79][23]
\end{Arabic}}
\flushleft{\begin{malayalam}
അങ്ങനെ അവന്‍ (തന്‍റെ ആള്‍ക്കാരെ) ശേഖരിച്ചു. എന്നിട്ടു വിളംബരം ചെയ്തു.
\end{malayalam}}
\flushright{\begin{Arabic}
\quranayah[79][24]
\end{Arabic}}
\flushleft{\begin{malayalam}
ഞാന്‍ നിങ്ങളുടെ അത്യുന്നതനായ രക്ഷിതാവാകുന്നു എന്ന് അവന്‍ പറഞ്ഞു.
\end{malayalam}}
\flushright{\begin{Arabic}
\quranayah[79][25]
\end{Arabic}}
\flushleft{\begin{malayalam}
അപ്പോള്‍ പരലോകത്തിലെയും ഇഹലോകത്തിലെയും ശിക്ഷയ്ക്കായി അല്ലാഹു അവനെ പിടികൂടി.
\end{malayalam}}
\flushright{\begin{Arabic}
\quranayah[79][26]
\end{Arabic}}
\flushleft{\begin{malayalam}
തീര്‍ച്ചയായും അതില്‍ ഭയപ്പെടുന്നവര്‍ക്ക് ഒരു ഗുണപാഠമുണ്ട്‌.
\end{malayalam}}
\flushright{\begin{Arabic}
\quranayah[79][27]
\end{Arabic}}
\flushleft{\begin{malayalam}
നിങ്ങളാണോ സൃഷ്ടിക്കപ്പെടാന്‍ കൂടുതല്‍ പ്രയാസമുള്ളവര്‍. അതല്ല; ആകാശമാണോ? അതിനെ അവന്‍ നിര്‍മിച്ചിരിക്കുന്നു.
\end{malayalam}}
\flushright{\begin{Arabic}
\quranayah[79][28]
\end{Arabic}}
\flushleft{\begin{malayalam}
അതിന്‍റെ വിതാനം അവന്‍ ഉയര്‍ത്തുകയും, അതിനെ അവന്‍ വ്യവസ്ഥപ്പെടുത്തുകയും ചെയ്തിരിക്കുന്നു.
\end{malayalam}}
\flushright{\begin{Arabic}
\quranayah[79][29]
\end{Arabic}}
\flushleft{\begin{malayalam}
അതിലെ രാത്രിയെ അവന്‍ ഇരുട്ടാക്കുകയും, അതിലെ പകലിനെ അവന്‍ പ്രത്യക്ഷപ്പെടുത്തുകയും ചെയ്തിരിക്കുന്നു.
\end{malayalam}}
\flushright{\begin{Arabic}
\quranayah[79][30]
\end{Arabic}}
\flushleft{\begin{malayalam}
അതിനു ശേഷം ഭൂമിയെ അവന്‍ വികസിപ്പിച്ചിരിക്കുന്നു.
\end{malayalam}}
\flushright{\begin{Arabic}
\quranayah[79][31]
\end{Arabic}}
\flushleft{\begin{malayalam}
അതില്‍ നിന്ന് അതിലെ വെള്ളവും സസ്യജാലങ്ങളും അവന്‍ പുറത്തു കൊണ്ടുവരികയും ചെയ്തിരിക്കുന്നു.
\end{malayalam}}
\flushright{\begin{Arabic}
\quranayah[79][32]
\end{Arabic}}
\flushleft{\begin{malayalam}
പര്‍വ്വതങ്ങളെ അവന്‍ ഉറപ്പിച്ചു നിര്‍ത്തുകയും ചെയ്തിരിക്കുന്നു.
\end{malayalam}}
\flushright{\begin{Arabic}
\quranayah[79][33]
\end{Arabic}}
\flushleft{\begin{malayalam}
നിങ്ങള്‍ക്കും നിങ്ങളുടെ കന്നുകാലികള്‍ക്കും ഉപയോഗത്തിനായിട്ട്‌
\end{malayalam}}
\flushright{\begin{Arabic}
\quranayah[79][34]
\end{Arabic}}
\flushleft{\begin{malayalam}
എന്നാല്‍ ആ മഹാ വിപത്ത് വരുന്ന സന്ദര്‍ഭം.
\end{malayalam}}
\flushright{\begin{Arabic}
\quranayah[79][35]
\end{Arabic}}
\flushleft{\begin{malayalam}
അതായതു മനുഷ്യന്‍ താന്‍ അദ്ധ്വാനിച്ചു വെച്ചതിനെപ്പറ്റി ഓര്‍മിക്കുന്ന ദിവസം.
\end{malayalam}}
\flushright{\begin{Arabic}
\quranayah[79][36]
\end{Arabic}}
\flushleft{\begin{malayalam}
കാണുന്നവര്‍ക്ക് വേണ്ടി നരകം വെളിവാക്കപ്പെടുന്ന ദിവസം.
\end{malayalam}}
\flushright{\begin{Arabic}
\quranayah[79][37]
\end{Arabic}}
\flushleft{\begin{malayalam}
(അന്ന്‌) ആര്‍ അതിരുകവിയുകയും
\end{malayalam}}
\flushright{\begin{Arabic}
\quranayah[79][38]
\end{Arabic}}
\flushleft{\begin{malayalam}
ഇഹലോകജീവിതത്തിനു കൂടുതല്‍ പ്രാധാന്യം നല്‍കുകയും ചെയ്തുവോ
\end{malayalam}}
\flushright{\begin{Arabic}
\quranayah[79][39]
\end{Arabic}}
\flushleft{\begin{malayalam}
(അവന്ന്‌) കത്തിജ്വലിക്കുന്ന നരകം തന്നെയാണ് സങ്കേതം.
\end{malayalam}}
\flushright{\begin{Arabic}
\quranayah[79][40]
\end{Arabic}}
\flushleft{\begin{malayalam}
അപ്പോള്‍ ഏതൊരാള്‍ തന്‍റെ രക്ഷിതാവിന്‍റെ സ്ഥാനത്തെ ഭയപ്പെടുകയും മനസ്സിനെ തന്നിഷ്ടത്തില്‍ നിന്ന് വിലക്കിനിര്‍ത്തുകയും ചെയ്തുവോ
\end{malayalam}}
\flushright{\begin{Arabic}
\quranayah[79][41]
\end{Arabic}}
\flushleft{\begin{malayalam}
(അവന്ന്‌) സ്വര്‍ഗം തന്നെയാണ് സങ്കേതം.
\end{malayalam}}
\flushright{\begin{Arabic}
\quranayah[79][42]
\end{Arabic}}
\flushleft{\begin{malayalam}
ആ അന്ത്യസമയത്തെപ്പറ്റി, അതെപ്പോഴാണ് സംഭവിക്കുക എന്ന് അവര്‍ നിന്നോട് ചോദിക്കുന്നു.
\end{malayalam}}
\flushright{\begin{Arabic}
\quranayah[79][43]
\end{Arabic}}
\flushleft{\begin{malayalam}
നിനക്ക് അതിനെപ്പറ്റി എന്ത് പറയാനാണുള്ളത്‌?
\end{malayalam}}
\flushright{\begin{Arabic}
\quranayah[79][44]
\end{Arabic}}
\flushleft{\begin{malayalam}
നിന്‍റെ രക്ഷിതാവിങ്കലേക്കാണ് അതിന്‍റെ കലാശം.
\end{malayalam}}
\flushright{\begin{Arabic}
\quranayah[79][45]
\end{Arabic}}
\flushleft{\begin{malayalam}
അതിനെ ഭയപ്പെടുന്നവര്‍ക്ക് ഒരു താക്കീതുകാരന്‍ മാത്രമാണ് നീ.
\end{malayalam}}
\flushright{\begin{Arabic}
\quranayah[79][46]
\end{Arabic}}
\flushleft{\begin{malayalam}
അതിനെ അവര്‍ കാണുന്ന ദിവസം ഒരു വൈകുന്നേരമോ ഒരു പ്രഭാതത്തിലോ അല്ലാതെ അവര്‍ (ഇവിടെ) കഴിച്ചുകൂട്ടിയിട്ടില്ലാത്ത പോലെയായിരിക്കും (അവര്‍ക്ക് തോന്നുക.)
\end{malayalam}}
\chapter{\textmalayalam{അബസ ( മുഖം ചുളിച്ചു )}}
\begin{Arabic}
\Huge{\centerline{\basmalah}}\end{Arabic}
\flushright{\begin{Arabic}
\quranayah[80][1]
\end{Arabic}}
\flushleft{\begin{malayalam}
അദ്ദേഹം മുഖം ചുളിച്ചു തിരിഞ്ഞുകളഞ്ഞു.
\end{malayalam}}
\flushright{\begin{Arabic}
\quranayah[80][2]
\end{Arabic}}
\flushleft{\begin{malayalam}
അദ്ദേഹത്തിന്‍റെ (നബിയുടെ) അടുത്ത് ആ അന്ധന്‍ വന്നതിനാല്‍.
\end{malayalam}}
\flushright{\begin{Arabic}
\quranayah[80][3]
\end{Arabic}}
\flushleft{\begin{malayalam}
(നബിയേ,) നിനക്ക് എന്തറിയാം? അയാള്‍ (അന്ധന്‍) ഒരു വേള പരിശുദ്ധി പ്രാപിച്ചേക്കാമല്ലോ?
\end{malayalam}}
\flushright{\begin{Arabic}
\quranayah[80][4]
\end{Arabic}}
\flushleft{\begin{malayalam}
അല്ലെങ്കില്‍ ഉപദേശം സ്വീകരിക്കുകയും, ആ ഉപദേശം അയാള്‍ക്ക് പ്രയോജനപ്പെടുകയും ചെയ്തേക്കാമല്ലോ.
\end{malayalam}}
\flushright{\begin{Arabic}
\quranayah[80][5]
\end{Arabic}}
\flushleft{\begin{malayalam}
എന്നാല്‍ സ്വയം പര്യാപ്തത നടിച്ചവനാകട്ടെ
\end{malayalam}}
\flushright{\begin{Arabic}
\quranayah[80][6]
\end{Arabic}}
\flushleft{\begin{malayalam}
നീ അവന്‍റെ നേരെ ശ്രദ്ധതിരിക്കുന്നു.
\end{malayalam}}
\flushright{\begin{Arabic}
\quranayah[80][7]
\end{Arabic}}
\flushleft{\begin{malayalam}
അവന്‍ പരിശുദ്ധി പ്രാപിക്കാതിരിക്കുന്നതിനാല്‍ നിനക്കെന്താണ് കുറ്റം?
\end{malayalam}}
\flushright{\begin{Arabic}
\quranayah[80][8]
\end{Arabic}}
\flushleft{\begin{malayalam}
എന്നാല്‍ നിന്‍റെ അടുക്കല്‍ ഓടിവന്നവനാകട്ടെ,
\end{malayalam}}
\flushright{\begin{Arabic}
\quranayah[80][9]
\end{Arabic}}
\flushleft{\begin{malayalam}
(അല്ലാഹുവെ) അവന്‍ ഭയപ്പെടുന്നവനായിക്കൊണ്ട്‌
\end{malayalam}}
\flushright{\begin{Arabic}
\quranayah[80][10]
\end{Arabic}}
\flushleft{\begin{malayalam}
അവന്‍റെ കാര്യത്തില്‍ നീ അശ്രദ്ധകാണിക്കുന്നു.
\end{malayalam}}
\flushright{\begin{Arabic}
\quranayah[80][11]
\end{Arabic}}
\flushleft{\begin{malayalam}
നിസ്സംശയം ഇത് (ഖുര്‍ആന്‍) ഒരു ഉല്‍ബോധനമാകുന്നു; തീര്‍ച്ച.
\end{malayalam}}
\flushright{\begin{Arabic}
\quranayah[80][12]
\end{Arabic}}
\flushleft{\begin{malayalam}
അതിനാല്‍ ആര്‍ ഉദ്ദേശിക്കുന്നുവോ അവനത് ഓര്‍മിച്ച് കൊള്ളട്ടെ.
\end{malayalam}}
\flushright{\begin{Arabic}
\quranayah[80][13]
\end{Arabic}}
\flushleft{\begin{malayalam}
ആദരണീയമായ ചില ഏടുകളിലാണത്‌.
\end{malayalam}}
\flushright{\begin{Arabic}
\quranayah[80][14]
\end{Arabic}}
\flushleft{\begin{malayalam}
ഔന്നത്യം നല്‍കപ്പെട്ടതും പരിശുദ്ധമാക്കപ്പെട്ടതുമായ (ഏടുകളില്‍)
\end{malayalam}}
\flushright{\begin{Arabic}
\quranayah[80][15]
\end{Arabic}}
\flushleft{\begin{malayalam}
ചില സന്ദേശവാഹകരുടെ കൈകളിലാണത്‌.
\end{malayalam}}
\flushright{\begin{Arabic}
\quranayah[80][16]
\end{Arabic}}
\flushleft{\begin{malayalam}
മാന്യന്‍മാരും പുണ്യവാന്‍മാരും ആയിട്ടുള്ളവരുടെ.
\end{malayalam}}
\flushright{\begin{Arabic}
\quranayah[80][17]
\end{Arabic}}
\flushleft{\begin{malayalam}
മനുഷ്യന്‍ നാശമടയട്ടെ. എന്താണവന്‍ ഇത്ര നന്ദികെട്ടവനാകാന്‍?
\end{malayalam}}
\flushright{\begin{Arabic}
\quranayah[80][18]
\end{Arabic}}
\flushleft{\begin{malayalam}
ഏതൊരു വസ്തുവില്‍ നിന്നാണ് അല്ലാഹു അവനെ സൃഷ്ടിച്ചത്‌?
\end{malayalam}}
\flushright{\begin{Arabic}
\quranayah[80][19]
\end{Arabic}}
\flushleft{\begin{malayalam}
ഒരു ബീജത്തില്‍ നിന്ന് അവനെ സൃഷ്ടിക്കുകയും, എന്നിട്ട് അവനെ (അവന്‍റെ കാര്യം) വ്യവസ്ഥപ്പെടുത്തുകയും ചെയ്തു.
\end{malayalam}}
\flushright{\begin{Arabic}
\quranayah[80][20]
\end{Arabic}}
\flushleft{\begin{malayalam}
പിന്നീട് അവന്‍ മാര്‍ഗം എളുപ്പമാക്കുകയും ചെയ്തു.
\end{malayalam}}
\flushright{\begin{Arabic}
\quranayah[80][21]
\end{Arabic}}
\flushleft{\begin{malayalam}
അനന്തരം അവനെ മരിപ്പിക്കുകയും, ഖബ്‌റില്‍ മറയ്ക്കുകയും ചെയ്തു.
\end{malayalam}}
\flushright{\begin{Arabic}
\quranayah[80][22]
\end{Arabic}}
\flushleft{\begin{malayalam}
പിന്നീട് അവന്‍ ഉദ്ദേശിക്കുമ്പോള്‍ അവനെ ഉയിര്‍ത്തെഴുന്നേല്‍പിക്കുന്നതാണ്‌.
\end{malayalam}}
\flushright{\begin{Arabic}
\quranayah[80][23]
\end{Arabic}}
\flushleft{\begin{malayalam}
നിസ്സംശയം, അവനോട് അല്ലാഹു കല്‍പിച്ചത് അവന്‍ നിര്‍വഹിച്ചില്ല.
\end{malayalam}}
\flushright{\begin{Arabic}
\quranayah[80][24]
\end{Arabic}}
\flushleft{\begin{malayalam}
എന്നാല്‍ മനുഷ്യന്‍ തന്‍റെ ഭക്ഷണത്തെപ്പറ്റി ഒന്നു ചിന്തിച്ച് നോക്കട്ടെ.
\end{malayalam}}
\flushright{\begin{Arabic}
\quranayah[80][25]
\end{Arabic}}
\flushleft{\begin{malayalam}
നാം ശക്തിയായി മഴ വെള്ളം ചൊരിഞ്ഞുകൊടുത്തു.
\end{malayalam}}
\flushright{\begin{Arabic}
\quranayah[80][26]
\end{Arabic}}
\flushleft{\begin{malayalam}
പിന്നീട് നാം ഭൂമിയെ ഒരു തരത്തില്‍ പിളര്‍ത്തി,
\end{malayalam}}
\flushright{\begin{Arabic}
\quranayah[80][27]
\end{Arabic}}
\flushleft{\begin{malayalam}
എന്നിട്ട് അതില്‍ നാം ധാന്യം മുളപ്പിച്ചു.
\end{malayalam}}
\flushright{\begin{Arabic}
\quranayah[80][28]
\end{Arabic}}
\flushleft{\begin{malayalam}
മുന്തിരിയും പച്ചക്കറികളും
\end{malayalam}}
\flushright{\begin{Arabic}
\quranayah[80][29]
\end{Arabic}}
\flushleft{\begin{malayalam}
ഒലീവും ഈന്തപ്പനയും
\end{malayalam}}
\flushright{\begin{Arabic}
\quranayah[80][30]
\end{Arabic}}
\flushleft{\begin{malayalam}
ഇടതൂര്‍ന്നു നില്‍ക്കുന്ന തോട്ടങ്ങളും.
\end{malayalam}}
\flushright{\begin{Arabic}
\quranayah[80][31]
\end{Arabic}}
\flushleft{\begin{malayalam}
പഴവര്‍ഗവും പുല്ലും.
\end{malayalam}}
\flushright{\begin{Arabic}
\quranayah[80][32]
\end{Arabic}}
\flushleft{\begin{malayalam}
നിങ്ങള്‍ക്കും നിങ്ങളുടെ കന്നുകാലികള്‍ക്കും ഉപയോഗത്തിനായിട്ട്‌.
\end{malayalam}}
\flushright{\begin{Arabic}
\quranayah[80][33]
\end{Arabic}}
\flushleft{\begin{malayalam}
എന്നാല്‍ ചെകിടടപ്പിക്കുന്ന ആ ശബ്ദം വന്നാല്‍.
\end{malayalam}}
\flushright{\begin{Arabic}
\quranayah[80][34]
\end{Arabic}}
\flushleft{\begin{malayalam}
അതായത് മനുഷ്യന്‍ തന്‍റെ സഹോദരനെ വിട്ട് ഓടിപ്പോകുന്ന ദിവസം.
\end{malayalam}}
\flushright{\begin{Arabic}
\quranayah[80][35]
\end{Arabic}}
\flushleft{\begin{malayalam}
തന്‍റെ മാതാവിനെയും പിതാവിനെയും.
\end{malayalam}}
\flushright{\begin{Arabic}
\quranayah[80][36]
\end{Arabic}}
\flushleft{\begin{malayalam}
തന്‍റെ ഭാര്യയെയും മക്കളെയും.
\end{malayalam}}
\flushright{\begin{Arabic}
\quranayah[80][37]
\end{Arabic}}
\flushleft{\begin{malayalam}
അവരില്‍പ്പെട്ട ഓരോ മനുഷ്യനും തനിക്ക് മതിയാവുന്നത്ര (ചിന്താ) വിഷയം അന്ന് ഉണ്ടായിരിക്കും.
\end{malayalam}}
\flushright{\begin{Arabic}
\quranayah[80][38]
\end{Arabic}}
\flushleft{\begin{malayalam}
അന്ന് ചില മുഖങ്ങള്‍ പ്രസന്നതയുള്ളവയായിരിക്കും
\end{malayalam}}
\flushright{\begin{Arabic}
\quranayah[80][39]
\end{Arabic}}
\flushleft{\begin{malayalam}
ചിരിക്കുന്നവയും സന്തോഷം കൊള്ളുന്നവയും.
\end{malayalam}}
\flushright{\begin{Arabic}
\quranayah[80][40]
\end{Arabic}}
\flushleft{\begin{malayalam}
വെറെ ചില മുഖങ്ങളാകട്ടെ അന്ന് പൊടി പുരണ്ടിരിക്കും.
\end{malayalam}}
\flushright{\begin{Arabic}
\quranayah[80][41]
\end{Arabic}}
\flushleft{\begin{malayalam}
അവയെ കൂരിരുട്ട് മൂടിയിരിക്കും.
\end{malayalam}}
\flushright{\begin{Arabic}
\quranayah[80][42]
\end{Arabic}}
\flushleft{\begin{malayalam}
അക്കൂട്ടരാകുന്നു അവിശ്വാസികളും അധര്‍മ്മകാരികളുമായിട്ടുള്ളവര്‍.
\end{malayalam}}
\chapter{\textmalayalam{തക് വീര്‍  ( ചുറ്റിപ്പൊതിയല്‍ )}}
\begin{Arabic}
\Huge{\centerline{\basmalah}}\end{Arabic}
\flushright{\begin{Arabic}
\quranayah[81][1]
\end{Arabic}}
\flushleft{\begin{malayalam}
സൂര്യന്‍ ചുറ്റിപ്പൊതിയപ്പെടുമ്പോള്‍,
\end{malayalam}}
\flushright{\begin{Arabic}
\quranayah[81][2]
\end{Arabic}}
\flushleft{\begin{malayalam}
നക്ഷത്രങ്ങള്‍ ഉതിര്‍ന്നു വീഴുമ്പോള്‍,
\end{malayalam}}
\flushright{\begin{Arabic}
\quranayah[81][3]
\end{Arabic}}
\flushleft{\begin{malayalam}
പര്‍വ്വതങ്ങള്‍ സഞ്ചരിപ്പിക്കപ്പെടുമ്പോള്‍,
\end{malayalam}}
\flushright{\begin{Arabic}
\quranayah[81][4]
\end{Arabic}}
\flushleft{\begin{malayalam}
പൂര്‍ണ്ണഗര്‍ഭിണികളായ ഒട്ടകങ്ങള്‍ അവഗണിക്കപ്പെടുമ്പോള്‍,
\end{malayalam}}
\flushright{\begin{Arabic}
\quranayah[81][5]
\end{Arabic}}
\flushleft{\begin{malayalam}
വന്യമൃഗങ്ങള്‍ ഒരുമിച്ചുകൂട്ടപ്പെടുമ്പോള്‍,
\end{malayalam}}
\flushright{\begin{Arabic}
\quranayah[81][6]
\end{Arabic}}
\flushleft{\begin{malayalam}
സമുദ്രങ്ങള്‍ ആളിക്കത്തിക്കപ്പെടുമ്പോള്‍,
\end{malayalam}}
\flushright{\begin{Arabic}
\quranayah[81][7]
\end{Arabic}}
\flushleft{\begin{malayalam}
ആത്മാവുകള്‍ കൂട്ടിയിണക്കപ്പെടുമ്പോള്‍,
\end{malayalam}}
\flushright{\begin{Arabic}
\quranayah[81][8]
\end{Arabic}}
\flushleft{\begin{malayalam}
(ജീവനോടെ) കുഴിച്ചു മൂടപ്പെട്ട പെണ്‍കുട്ടിയോടു ചോദിക്കപ്പെടുമ്പോള്‍,
\end{malayalam}}
\flushright{\begin{Arabic}
\quranayah[81][9]
\end{Arabic}}
\flushleft{\begin{malayalam}
താന്‍ എന്തൊരു കുറ്റത്തിനാണ് കൊല്ലപ്പെട്ടത് എന്ന്‌.
\end{malayalam}}
\flushright{\begin{Arabic}
\quranayah[81][10]
\end{Arabic}}
\flushleft{\begin{malayalam}
(കര്‍മ്മങ്ങള്‍ രേഖപ്പെടുത്തിയ) ഏടുകള്‍ തുറന്നുവെക്കപ്പെടുമ്പോള്‍.
\end{malayalam}}
\flushright{\begin{Arabic}
\quranayah[81][11]
\end{Arabic}}
\flushleft{\begin{malayalam}
ഉപരിലോകം മറ നീക്കികാണിക്കപ്പെടുമ്പോള്‍
\end{malayalam}}
\flushright{\begin{Arabic}
\quranayah[81][12]
\end{Arabic}}
\flushleft{\begin{malayalam}
ജ്വലിക്കുന്ന നരകാഗ്നി ആളിക്കത്തിക്കപ്പെടുമ്പോള്‍.
\end{malayalam}}
\flushright{\begin{Arabic}
\quranayah[81][13]
\end{Arabic}}
\flushleft{\begin{malayalam}
സ്വര്‍ഗം അടുത്തു കൊണ്ടുവരപ്പെടുമ്പോള്‍.
\end{malayalam}}
\flushright{\begin{Arabic}
\quranayah[81][14]
\end{Arabic}}
\flushleft{\begin{malayalam}
ഓരോ വ്യക്തിയും താന്‍ തയ്യാറാക്കിക്കൊണ്ടു വന്നിട്ടുള്ളത് എന്തെന്ന് അറിയുന്നതാണ്‌.
\end{malayalam}}
\flushright{\begin{Arabic}
\quranayah[81][15]
\end{Arabic}}
\flushleft{\begin{malayalam}
പിന്‍വാങ്ങിപ്പോകുന്നവയെ (നക്ഷത്രങ്ങളെ) ക്കൊണ്ട് ഞാന്‍ സത്യം ചെയ്തു പറയുന്നു.
\end{malayalam}}
\flushright{\begin{Arabic}
\quranayah[81][16]
\end{Arabic}}
\flushleft{\begin{malayalam}
സഞ്ചരിച്ചുകൊണ്ടിരിക്കുന്നവയും അപ്രത്യക്ഷമായിക്കൊണ്ടിരിക്കുന്നവയും
\end{malayalam}}
\flushright{\begin{Arabic}
\quranayah[81][17]
\end{Arabic}}
\flushleft{\begin{malayalam}
രാത്രി നീങ്ങുമ്പോള്‍ അതു കൊണ്ടും,
\end{malayalam}}
\flushright{\begin{Arabic}
\quranayah[81][18]
\end{Arabic}}
\flushleft{\begin{malayalam}
പ്രഭാതം വിടര്‍ന്ന് വരുമ്പോള്‍ അതു കൊണ്ടും (ഞാന്‍ സത്യം ചെയ്തു പറയുന്നു.)
\end{malayalam}}
\flushright{\begin{Arabic}
\quranayah[81][19]
\end{Arabic}}
\flushleft{\begin{malayalam}
തീര്‍ച്ചയായും ഇത് (ഖുര്‍ആന്‍) മാന്യനായ ഒരു ദൂതന്‍റെ വാക്കാകുന്നു.
\end{malayalam}}
\flushright{\begin{Arabic}
\quranayah[81][20]
\end{Arabic}}
\flushleft{\begin{malayalam}
ശക്തിയുള്ളവനും, സിംഹാസനസ്ഥനായ അല്ലാഹുവിങ്കല്‍ സ്ഥാനമുള്ളവനുമായ (ദൂതന്‍റെ)
\end{malayalam}}
\flushright{\begin{Arabic}
\quranayah[81][21]
\end{Arabic}}
\flushleft{\begin{malayalam}
അവിടെ അനുസരിക്കപ്പെടുന്നവനും വിശ്വസ്തനുമായ (ദൂതന്‍റെ)
\end{malayalam}}
\flushright{\begin{Arabic}
\quranayah[81][22]
\end{Arabic}}
\flushleft{\begin{malayalam}
നിങ്ങളുടെ കൂട്ടുകാരന്‍ (പ്രവാചകന്‍) ഒരു ഭ്രാന്തനല്ല തന്നെ,
\end{malayalam}}
\flushright{\begin{Arabic}
\quranayah[81][23]
\end{Arabic}}
\flushleft{\begin{malayalam}
തീര്‍ച്ചയായും അദ്ദേഹത്തെ (ജിബ്‌രീല്‍ എന്ന ദൂതനെ) പ്രത്യക്ഷമായ മണ്ഡലത്തില്‍ വെച്ച് അദ്ദേഹം കണ്ടിട്ടുണ്ട്‌.
\end{malayalam}}
\flushright{\begin{Arabic}
\quranayah[81][24]
\end{Arabic}}
\flushleft{\begin{malayalam}
അദ്ദേഹം അദൃശ്യവാര്‍ത്തയുടെ കാര്യത്തില്‍ പിശുക്ക് കാണിക്കുന്നവനുമല്ല.
\end{malayalam}}
\flushright{\begin{Arabic}
\quranayah[81][25]
\end{Arabic}}
\flushleft{\begin{malayalam}
ഇത് (ഖുര്‍ആന്‍) ശപിക്കപ്പെട്ട ഒരു പിശാചിന്‍റെ വാക്കുമല്ല.
\end{malayalam}}
\flushright{\begin{Arabic}
\quranayah[81][26]
\end{Arabic}}
\flushleft{\begin{malayalam}
അപ്പോള്‍ എങ്ങോട്ടാണ് നിങ്ങള്‍ പോകുന്നത്‌?
\end{malayalam}}
\flushright{\begin{Arabic}
\quranayah[81][27]
\end{Arabic}}
\flushleft{\begin{malayalam}
ഇത് ലോകര്‍ക്ക് വേണ്ടിയുള്ള ഒരു ഉല്‍ബോധനമല്ലാതെ മറ്റൊന്നുമല്ല.
\end{malayalam}}
\flushright{\begin{Arabic}
\quranayah[81][28]
\end{Arabic}}
\flushleft{\begin{malayalam}
അതായത് നിങ്ങളുടെ കൂട്ടത്തില്‍ നിന്ന് നേരെ നിലകൊള്ളാന്‍ ഉദ്ദേശിച്ചവര്‍ക്ക് വേണ്ടി.
\end{malayalam}}
\flushright{\begin{Arabic}
\quranayah[81][29]
\end{Arabic}}
\flushleft{\begin{malayalam}
ലോകരക്ഷിതാവായ അല്ലാഹു ഉദ്ദേശിക്കുന്നുവെങ്കിലല്ലാതെ നിങ്ങള്‍ ഉദ്ദേശിക്കുകയില്ല.
\end{malayalam}}
\chapter{\textmalayalam{ഇന്‍ഫിത്വാര്‍ ( പൊട്ടിക്കീറല്‍ )}}
\begin{Arabic}
\Huge{\centerline{\basmalah}}\end{Arabic}
\flushright{\begin{Arabic}
\quranayah[82][1]
\end{Arabic}}
\flushleft{\begin{malayalam}
ആകാശം പൊട്ടി പിളരുമ്പോള്‍.
\end{malayalam}}
\flushright{\begin{Arabic}
\quranayah[82][2]
\end{Arabic}}
\flushleft{\begin{malayalam}
നക്ഷത്രങ്ങള്‍ കൊഴിഞ്ഞു വീഴുമ്പോള്‍.
\end{malayalam}}
\flushright{\begin{Arabic}
\quranayah[82][3]
\end{Arabic}}
\flushleft{\begin{malayalam}
സമുദ്രങ്ങള്‍ പൊട്ടി ഒഴുകുമ്പോള്‍.
\end{malayalam}}
\flushright{\begin{Arabic}
\quranayah[82][4]
\end{Arabic}}
\flushleft{\begin{malayalam}
ഖബ്‌റുകള്‍ ഇളക്കിമറിക്കപ്പെടുമ്പോള്‍
\end{malayalam}}
\flushright{\begin{Arabic}
\quranayah[82][5]
\end{Arabic}}
\flushleft{\begin{malayalam}
ഓരോ വ്യക്തിയും താന്‍ മുന്‍കൂട്ടി ചെയ്തു വെച്ചതും പിന്നോട്ട് മേറ്റീവ്ച്ചതും എന്താണെന്ന് അറിയുന്നതാണ്‌.
\end{malayalam}}
\flushright{\begin{Arabic}
\quranayah[82][6]
\end{Arabic}}
\flushleft{\begin{malayalam}
ഹേ; മനുഷ്യാ, ഉദാരനായ നിന്‍റെ രക്ഷിതാവിന്‍റെ കാര്യത്തില്‍ നിന്നെ വഞ്ചിച്ചു കളഞ്ഞതെന്താണ്‌?
\end{malayalam}}
\flushright{\begin{Arabic}
\quranayah[82][7]
\end{Arabic}}
\flushleft{\begin{malayalam}
നിന്നെ സൃഷ്ടിക്കുകയും, നിന്നെ സംവിധാനിക്കുകയും , നിന്നെ ശരിയായ അവസ്ഥയിലാക്കുകയും ചെയ്തവനത്രെ അവന്‍.
\end{malayalam}}
\flushright{\begin{Arabic}
\quranayah[82][8]
\end{Arabic}}
\flushleft{\begin{malayalam}
താന്‍ ഉദ്ദേശിച്ച രൂപത്തില്‍ നിന്നെ സംഘടിപ്പിച്ചവന്‍.
\end{malayalam}}
\flushright{\begin{Arabic}
\quranayah[82][9]
\end{Arabic}}
\flushleft{\begin{malayalam}
അല്ല; പക്ഷെ, പ്രതിഫല നടപടിയെ നിങ്ങള്‍ നിഷേധിച്ചു തള്ളുന്നു.
\end{malayalam}}
\flushright{\begin{Arabic}
\quranayah[82][10]
\end{Arabic}}
\flushleft{\begin{malayalam}
തീര്‍ച്ചയായും നിങ്ങളുടെ മേല്‍ ചില മേല്‍നോട്ടക്കാരുണ്ട്‌.
\end{malayalam}}
\flushright{\begin{Arabic}
\quranayah[82][11]
\end{Arabic}}
\flushleft{\begin{malayalam}
രേഖപ്പെടുത്തിവെക്കുന്ന ചില മാന്യന്‍മാര്‍.
\end{malayalam}}
\flushright{\begin{Arabic}
\quranayah[82][12]
\end{Arabic}}
\flushleft{\begin{malayalam}
നിങ്ങള്‍ പ്രവര്‍ത്തിക്കുന്നത് അവര്‍ അറിയുന്നു.
\end{malayalam}}
\flushright{\begin{Arabic}
\quranayah[82][13]
\end{Arabic}}
\flushleft{\begin{malayalam}
തീര്‍ച്ചയായും സുകൃതവാന്‍മാര്‍ സുഖാനുഭവത്തില്‍ തന്നെയായിരിക്കും.
\end{malayalam}}
\flushright{\begin{Arabic}
\quranayah[82][14]
\end{Arabic}}
\flushleft{\begin{malayalam}
തീര്‍ച്ചയായും ദുര്‍മാര്‍ഗികള്‍ ജ്വലിക്കുന്ന നരകാഗ്നിയില്‍ തന്നെയായിരിക്കും
\end{malayalam}}
\flushright{\begin{Arabic}
\quranayah[82][15]
\end{Arabic}}
\flushleft{\begin{malayalam}
പ്രതിഫലത്തിന്‍റെ നാളില്‍ അവരതില്‍ കടന്ന് എരിയുന്നതാണ്‌.
\end{malayalam}}
\flushright{\begin{Arabic}
\quranayah[82][16]
\end{Arabic}}
\flushleft{\begin{malayalam}
അവര്‍ക്ക് അതില്‍ നിന്ന് മാറി നില്‍ക്കാനാവില്ല.
\end{malayalam}}
\flushright{\begin{Arabic}
\quranayah[82][17]
\end{Arabic}}
\flushleft{\begin{malayalam}
പ്രതിഫലനടപടിയുടെ ദിവസം എന്നാല്‍ എന്താണെന്ന് നിനക്കറിയുമോ?
\end{malayalam}}
\flushright{\begin{Arabic}
\quranayah[82][18]
\end{Arabic}}
\flushleft{\begin{malayalam}
വീണ്ടും; പ്രതിഫലനടപടിയുടെ ദിവസം എന്നാല്‍ എന്താണെന്ന് നിനക്കറിയുമോ?
\end{malayalam}}
\flushright{\begin{Arabic}
\quranayah[82][19]
\end{Arabic}}
\flushleft{\begin{malayalam}
ഒരാള്‍ക്കും മറ്റൊരാള്‍ക്കു വേണ്ടി യാതൊന്നും അധീനപ്പെടുത്താനാവാത്ത ഒരു ദിവസം. അന്നേ ദിവസം കൈകാര്യകര്‍ത്തൃത്വം അല്ലാഹുവിന്നായിരിക്കും.
\end{malayalam}}
\chapter{\textmalayalam{മുതഫ്ഫിഫീന്‍ ( അളവില്‍ കുറയ്ക്കുന്നവന്‍ )}}
\begin{Arabic}
\Huge{\centerline{\basmalah}}\end{Arabic}
\flushright{\begin{Arabic}
\quranayah[83][1]
\end{Arabic}}
\flushleft{\begin{malayalam}
അളവില്‍ കുറക്കുന്നവര്‍ക്ക് മഹാനാശം
\end{malayalam}}
\flushright{\begin{Arabic}
\quranayah[83][2]
\end{Arabic}}
\flushleft{\begin{malayalam}
അതായത് ജനങ്ങളോട് അളന്നുവാങ്ങുകയാണെങ്കില്‍ തികച്ചെടുക്കുകയും.
\end{malayalam}}
\flushright{\begin{Arabic}
\quranayah[83][3]
\end{Arabic}}
\flushleft{\begin{malayalam}
ജനങ്ങള്‍ക്ക് അളന്നുകൊടുക്കുകയോ തൂക്കികൊടുക്കുകയോ ആണെങ്കില്‍ നഷ്ടം വരുത്തുകയും ചെയ്യുന്നവര്‍ക്ക്‌.
\end{malayalam}}
\flushright{\begin{Arabic}
\quranayah[83][4]
\end{Arabic}}
\flushleft{\begin{malayalam}
അക്കൂട്ടര്‍ വിചാരിക്കുന്നില്ലേ; തങ്ങള്‍ എഴുന്നേല്‍പിക്കപ്പെടുന്നവരാണെന്ന്‌?
\end{malayalam}}
\flushright{\begin{Arabic}
\quranayah[83][5]
\end{Arabic}}
\flushleft{\begin{malayalam}
ഭയങ്കരമായ ഒരു ദിവസത്തിനായിട്ട്‌
\end{malayalam}}
\flushright{\begin{Arabic}
\quranayah[83][6]
\end{Arabic}}
\flushleft{\begin{malayalam}
അതെ, ലോകരക്ഷിതാവിങ്കലേക്ക് ജനങ്ങള്‍ എഴുന്നേറ്റ് വരുന്ന ദിവസം.
\end{malayalam}}
\flushright{\begin{Arabic}
\quranayah[83][7]
\end{Arabic}}
\flushleft{\begin{malayalam}
നിസ്സംശയം; ദുര്‍മാര്‍ഗികളുടെ രേഖ സിജ്ജീനില്‍ തന്നെയായിരിക്കും.
\end{malayalam}}
\flushright{\begin{Arabic}
\quranayah[83][8]
\end{Arabic}}
\flushleft{\begin{malayalam}
സിജ്ജീന്‍ എന്നാല്‍ എന്താണെന്ന് നിനക്കറിയാമോ?
\end{malayalam}}
\flushright{\begin{Arabic}
\quranayah[83][9]
\end{Arabic}}
\flushleft{\begin{malayalam}
എഴുതപ്പെട്ട ഒരു ഗ്രന്ഥമാകുന്നു അത്‌.
\end{malayalam}}
\flushright{\begin{Arabic}
\quranayah[83][10]
\end{Arabic}}
\flushleft{\begin{malayalam}
അന്നേ ദിവസം നിഷേധിച്ചു തള്ളുന്നവര്‍ക്കാകുന്നു നാശം.
\end{malayalam}}
\flushright{\begin{Arabic}
\quranayah[83][11]
\end{Arabic}}
\flushleft{\begin{malayalam}
അതായത് പ്രതിഫല നടപടിയുടെ ദിവസത്തെ നിഷേധിച്ചു തള്ളുന്നവര്‍ക്ക്‌.
\end{malayalam}}
\flushright{\begin{Arabic}
\quranayah[83][12]
\end{Arabic}}
\flushleft{\begin{malayalam}
എല്ലാ അതിരുവിട്ടവനും മഹാപാപിയുമായിട്ടുള്ളവനല്ലാതെ അതിനെ നിഷേധിച്ചു തള്ളുകയില്ല.
\end{malayalam}}
\flushright{\begin{Arabic}
\quranayah[83][13]
\end{Arabic}}
\flushleft{\begin{malayalam}
അവന്ന് നമ്മുടെ ദൃഷ്ടാന്തങ്ങള്‍ ഓതികേള്‍പിക്കപ്പെടുകയാണെങ്കില്‍ അവന്‍ പറയും; പൂര്‍വ്വികന്‍മാരുടെ ഐതിഹ്യങ്ങളാണെന്ന്‌.
\end{malayalam}}
\flushright{\begin{Arabic}
\quranayah[83][14]
\end{Arabic}}
\flushleft{\begin{malayalam}
അല്ല; പക്ഷെ, അവര്‍ പ്രവര്‍ത്തിച്ചുക്കൊണ്ടിരിക്കുന്നത് അവരുടെ ഹൃദയങ്ങളില്‍ കറയുണ്ടാക്കിയിരിക്കുന്നു.
\end{malayalam}}
\flushright{\begin{Arabic}
\quranayah[83][15]
\end{Arabic}}
\flushleft{\begin{malayalam}
അല്ല; തീര്‍ച്ചയായും അവര്‍ അന്നേ ദിവസം അവരുടെ രക്ഷിതാവില്‍ നിന്ന് മറയ്ക്കപ്പെടുന്നവരാകുന്നു.
\end{malayalam}}
\flushright{\begin{Arabic}
\quranayah[83][16]
\end{Arabic}}
\flushleft{\begin{malayalam}
പിന്നീടവര്‍ ജ്വലിക്കുന്ന നരകാഗ്നിയില്‍ കടന്നെരിയുന്നവരാകുന്നു.
\end{malayalam}}
\flushright{\begin{Arabic}
\quranayah[83][17]
\end{Arabic}}
\flushleft{\begin{malayalam}
പിന്നീട് പറയപ്പെടും; ഇതാണ് നിങ്ങള്‍ നിഷേധിച്ചുതള്ളിക്കൊണ്ടിരുന്ന കാര്യം.
\end{malayalam}}
\flushright{\begin{Arabic}
\quranayah[83][18]
\end{Arabic}}
\flushleft{\begin{malayalam}
നിസ്സംശയം; പുണ്യവാന്‍മാരുടെ രേഖ ഇല്ലിയ്യൂനില്‍ തന്നെയായിരിക്കും.
\end{malayalam}}
\flushright{\begin{Arabic}
\quranayah[83][19]
\end{Arabic}}
\flushleft{\begin{malayalam}
ഇല്ലിയ്യൂന്‍ എന്നാല്‍ എന്താണെന്ന് നിനക്കറിയുമോ?
\end{malayalam}}
\flushright{\begin{Arabic}
\quranayah[83][20]
\end{Arabic}}
\flushleft{\begin{malayalam}
എഴുതപ്പെട്ട ഒരു രേഖയത്രെ അത്‌.
\end{malayalam}}
\flushright{\begin{Arabic}
\quranayah[83][21]
\end{Arabic}}
\flushleft{\begin{malayalam}
സാമീപ്യം സിദ്ധിച്ചവര്‍ അതിന്‍റെ അടുക്കല്‍ സന്നിഹിതരാകുന്നതാണ്‌.
\end{malayalam}}
\flushright{\begin{Arabic}
\quranayah[83][22]
\end{Arabic}}
\flushleft{\begin{malayalam}
തീര്‍ച്ചയായും സുകൃതവാന്‍മാര്‍ സുഖാനുഭവത്തില്‍ തന്നെയായിരിക്കും.
\end{malayalam}}
\flushright{\begin{Arabic}
\quranayah[83][23]
\end{Arabic}}
\flushleft{\begin{malayalam}
സോഫകളിലിരുന്ന് അവര്‍ നോക്കിക്കൊണ്ടിരിക്കും.
\end{malayalam}}
\flushright{\begin{Arabic}
\quranayah[83][24]
\end{Arabic}}
\flushleft{\begin{malayalam}
അവരുടെ മുഖങ്ങളില്‍ സുഖാനുഭവത്തിന്‍റെ തിളക്കം നിനക്കറിയാം.
\end{malayalam}}
\flushright{\begin{Arabic}
\quranayah[83][25]
\end{Arabic}}
\flushleft{\begin{malayalam}
മുദ്രവെക്കപ്പെട്ട ശുദ്ധമായ മദ്യത്തില്‍ നിന്ന് അവര്‍ക്ക് കുടിക്കാന്‍ നല്‍കപ്പെടും.
\end{malayalam}}
\flushright{\begin{Arabic}
\quranayah[83][26]
\end{Arabic}}
\flushleft{\begin{malayalam}
അതിന്‍റെ മുദ്ര കസ്തൂരിയായിരിക്കും. വാശി കാണിക്കുന്നവര്‍ അതിന് വേണ്ടി വാശി കാണിക്കട്ടെ.
\end{malayalam}}
\flushright{\begin{Arabic}
\quranayah[83][27]
\end{Arabic}}
\flushleft{\begin{malayalam}
അതിലെ ചേരുവ തസ്നീം ആയിരിക്കും.
\end{malayalam}}
\flushright{\begin{Arabic}
\quranayah[83][28]
\end{Arabic}}
\flushleft{\begin{malayalam}
അതായത് സാമീപ്യം സിദ്ധിച്ചവര്‍ കുടിക്കുന്ന ഒരു ഉറവ് ജലം.
\end{malayalam}}
\flushright{\begin{Arabic}
\quranayah[83][29]
\end{Arabic}}
\flushleft{\begin{malayalam}
തീര്‍ച്ചയായും കുറ്റകൃത്യത്തില്‍ ഏര്‍പെട്ടവര്‍ സത്യവിശ്വാസികളെ കളിയാക്കി ചിരിക്കുമായിരുന്നു.
\end{malayalam}}
\flushright{\begin{Arabic}
\quranayah[83][30]
\end{Arabic}}
\flushleft{\begin{malayalam}
അവരുടെ (സത്യവിശ്വാസികളുടെ) മുമ്പിലൂടെ കടന്നു പോകുമ്പോള്‍ അവര്‍ പരസ്പരം കണ്ണിട്ടു കാണിക്കുമായിരുന്നു.
\end{malayalam}}
\flushright{\begin{Arabic}
\quranayah[83][31]
\end{Arabic}}
\flushleft{\begin{malayalam}
അവരുടെ സ്വന്തക്കാരുടെ അടുക്കലേക്ക് തിരിച്ചുചെല്ലുമ്പോള്‍ രസിച്ചു കൊണ്ട് അവര്‍ തിരിച്ചുചെല്ലുമായിരുന്നു.
\end{malayalam}}
\flushright{\begin{Arabic}
\quranayah[83][32]
\end{Arabic}}
\flushleft{\begin{malayalam}
അവരെ (സത്യവിശ്വാസികളെ) അവര്‍ കാണുമ്പോള്‍, തീര്‍ച്ചയായും ഇക്കൂട്ടര്‍ വഴിപിഴച്ചവര്‍ തന്നെയാണ് എന്ന് അവര്‍ പറയുകയും ചെയ്യുമായിരുന്നു.
\end{malayalam}}
\flushright{\begin{Arabic}
\quranayah[83][33]
\end{Arabic}}
\flushleft{\begin{malayalam}
അവരുടെ (സത്യവിശ്വാസികളുടെ) മേല്‍ മേല്‍നോട്ടക്കാരായിട്ട് അവര്‍ നിയോഗിക്കപ്പെട്ടിട്ടൊന്നുമില്ല.
\end{malayalam}}
\flushright{\begin{Arabic}
\quranayah[83][34]
\end{Arabic}}
\flushleft{\begin{malayalam}
എന്നാല്‍ അന്ന് (ഖിയാമത്ത് നാളില്‍) ആ സത്യവിശ്വാസികള്‍ സത്യനിഷേധികളെ കളിയാക്കി ചിരിക്കുന്നതാണ്‌.
\end{malayalam}}
\flushright{\begin{Arabic}
\quranayah[83][35]
\end{Arabic}}
\flushleft{\begin{malayalam}
സോഫകളിലിരുന്ന് അവര്‍ നോക്കിക്കൊണ്ടിരിക്കും.
\end{malayalam}}
\flushright{\begin{Arabic}
\quranayah[83][36]
\end{Arabic}}
\flushleft{\begin{malayalam}
സത്യനിഷേധികള്‍ ചെയ്തു കൊണ്ടിരുന്നതിന് അവര്‍ക്ക് പ്രതിഫലം നല്‍കപ്പെട്ടുവോ എന്ന്‌.
\end{malayalam}}
\chapter{\textmalayalam{ഇന്ഷിഖാഖ് ( പൊട്ടിപിളരല്‍ )}}
\begin{Arabic}
\Huge{\centerline{\basmalah}}\end{Arabic}
\flushright{\begin{Arabic}
\quranayah[84][1]
\end{Arabic}}
\flushleft{\begin{malayalam}
ആകാശം പിളരുമ്പോള്‍,
\end{malayalam}}
\flushright{\begin{Arabic}
\quranayah[84][2]
\end{Arabic}}
\flushleft{\begin{malayalam}
അത് അതിന്‍റെ രക്ഷിതാവിന് കീഴ്പെടുകയും ചെയ്യുമ്പോള്‍-അത് (അങ്ങനെ കീഴ്പെടാന്‍) കടപ്പെട്ടിരിക്കുന്നുതാനും.
\end{malayalam}}
\flushright{\begin{Arabic}
\quranayah[84][3]
\end{Arabic}}
\flushleft{\begin{malayalam}
ഭൂമി നീട്ടപ്പെടുമ്പോള്‍
\end{malayalam}}
\flushright{\begin{Arabic}
\quranayah[84][4]
\end{Arabic}}
\flushleft{\begin{malayalam}
അതിലുള്ളത് അത് (പുറത്തേക്ക്‌) ഇടുകയും, അത് കാലിയായിത്തീരുകയും ചെയ്യുമ്പോള്‍,
\end{malayalam}}
\flushright{\begin{Arabic}
\quranayah[84][5]
\end{Arabic}}
\flushleft{\begin{malayalam}
അതിന്‍റെ രക്ഷിതാവിന് അത് കീഴ്പെടുകയും ചെയ്യുമ്പോള്‍- അത് (അങ്ങനെ കീഴ്പെടാന്‍) കടപ്പെട്ടിരിക്കുന്നു താനും.
\end{malayalam}}
\flushright{\begin{Arabic}
\quranayah[84][6]
\end{Arabic}}
\flushleft{\begin{malayalam}
ഹേ, മനുഷ്യാ, നീ നിന്‍റെ രക്ഷിതാവിങ്കലേക്ക് കടുത്ത അദ്ധ്വാനം നടത്തി ചെല്ലുന്നവനും അങ്ങനെ അവനുമായി കണ്ടുമുട്ടുന്നവനുമാകുന്നു.
\end{malayalam}}
\flushright{\begin{Arabic}
\quranayah[84][7]
\end{Arabic}}
\flushleft{\begin{malayalam}
എന്നാല്‍ (പരലോകത്ത്‌) ഏതൊരുവന്ന് തന്‍റെ രേഖ വലതുകൈയ്യില്‍ നല്‍കപ്പെട്ടുവോ,
\end{malayalam}}
\flushright{\begin{Arabic}
\quranayah[84][8]
\end{Arabic}}
\flushleft{\begin{malayalam}
അവന്‍ ലഘുവായ വിചാരണയ്ക്ക് (മാത്രം) വിധേയനാകുന്നതാണ്‌.
\end{malayalam}}
\flushright{\begin{Arabic}
\quranayah[84][9]
\end{Arabic}}
\flushleft{\begin{malayalam}
അവന്‍ അവന്‍റെ സ്വന്തക്കാരുടെ അടുത്തേക്ക് സന്തുഷ്ടനായിക്കൊണ്ട് തിരിച്ചുപോകുകയും ചെയ്യും.
\end{malayalam}}
\flushright{\begin{Arabic}
\quranayah[84][10]
\end{Arabic}}
\flushleft{\begin{malayalam}
എന്നാല്‍ ഏതൊരുവന് തന്‍റെ രേഖ അവന്‍റെ മുതുകിന്‍റെ പിന്നിലൂടെ കൊടുക്കപ്പെട്ടുവോ
\end{malayalam}}
\flushright{\begin{Arabic}
\quranayah[84][11]
\end{Arabic}}
\flushleft{\begin{malayalam}
അവന്‍ നാശമേ എന്ന് നിലവിളിക്കുകയും,
\end{malayalam}}
\flushright{\begin{Arabic}
\quranayah[84][12]
\end{Arabic}}
\flushleft{\begin{malayalam}
ആളിക്കത്തുന്ന നരകാഗ്നിയില്‍ കടന്ന് എരിയുകയും ചെയ്യും.
\end{malayalam}}
\flushright{\begin{Arabic}
\quranayah[84][13]
\end{Arabic}}
\flushleft{\begin{malayalam}
തീര്‍ച്ചയായും അവന്‍ അവന്‍റെ സ്വന്തക്കാര്‍ക്കിടയില്‍ സന്തോഷത്തോടെ കഴിയുന്നവനായിരുന്നു.
\end{malayalam}}
\flushright{\begin{Arabic}
\quranayah[84][14]
\end{Arabic}}
\flushleft{\begin{malayalam}
തീര്‍ച്ചയായും അവന്‍ ധരിച്ചു; അവന്‍ മടങ്ങി വരുന്നതേ അല്ല എന്ന്‌.
\end{malayalam}}
\flushright{\begin{Arabic}
\quranayah[84][15]
\end{Arabic}}
\flushleft{\begin{malayalam}
അതെ, തീര്‍ച്ചയായും അവന്‍റെ രക്ഷിതാവ് അവനെപ്പറ്റി കണ്ടറിയുന്നവനായിരിക്കുന്നു.
\end{malayalam}}
\flushright{\begin{Arabic}
\quranayah[84][16]
\end{Arabic}}
\flushleft{\begin{malayalam}
എന്നാല്‍ അസ്തമയശോഭയെക്കൊണ്ട് ഞാന്‍ സത്യം ചെയ്തു പറയുന്നു:
\end{malayalam}}
\flushright{\begin{Arabic}
\quranayah[84][17]
\end{Arabic}}
\flushleft{\begin{malayalam}
രാത്രിയും അതു ഒന്നിച്ച് ചേര്‍ക്കുന്നവയും കൊണ്ടും,
\end{malayalam}}
\flushright{\begin{Arabic}
\quranayah[84][18]
\end{Arabic}}
\flushleft{\begin{malayalam}
ചന്ദ്രന്‍ പൂര്‍ണ്ണത പ്രാപിക്കുമ്പോള്‍ അതിനെ കൊണ്ടും.
\end{malayalam}}
\flushright{\begin{Arabic}
\quranayah[84][19]
\end{Arabic}}
\flushleft{\begin{malayalam}
തീര്‍ച്ചയായും നിങ്ങള്‍ ഘട്ടംഘട്ടമായി കയറിക്കൊണ്ടിരിക്കുന്നതാണ്‌.
\end{malayalam}}
\flushright{\begin{Arabic}
\quranayah[84][20]
\end{Arabic}}
\flushleft{\begin{malayalam}
എന്നാല്‍ അവര്‍ക്കെന്തുപറ്റി? അവര്‍ വിശ്വസിക്കുന്നില്ല.
\end{malayalam}}
\flushright{\begin{Arabic}
\quranayah[84][21]
\end{Arabic}}
\flushleft{\begin{malayalam}
അവര്‍ക്ക് ഖുര്‍ആന്‍ ഓതികൊടുക്കപ്പെട്ടാല്‍ അവര്‍ സുജൂദ് ചെയ്യുന്നുമില്ല.
\end{malayalam}}
\flushright{\begin{Arabic}
\quranayah[84][22]
\end{Arabic}}
\flushleft{\begin{malayalam}
പക്ഷെ അവിശ്വാസികള്‍ നിഷേധിച്ചു തള്ളുകയാണ്‌.
\end{malayalam}}
\flushright{\begin{Arabic}
\quranayah[84][23]
\end{Arabic}}
\flushleft{\begin{malayalam}
അവര്‍ മനസ്സുകളില്‍ സൂക്ഷിച്ച് വെക്കുന്നതിനെപ്പറ്റി അല്ലാഹു നല്ലവണ്ണം അറിയുന്നവനാകുന്നു.
\end{malayalam}}
\flushright{\begin{Arabic}
\quranayah[84][24]
\end{Arabic}}
\flushleft{\begin{malayalam}
ആകയാല്‍ (നബിയേ,) നീ അവര്‍ക്ക് വേദനയേറിയ ഒരു ശിക്ഷയെപ്പറ്റി സന്തോഷവാര്‍ത്ത അറിയിക്കുക.
\end{malayalam}}
\flushright{\begin{Arabic}
\quranayah[84][25]
\end{Arabic}}
\flushleft{\begin{malayalam}
വിശ്വസിക്കുകയും സല്‍കര്‍മ്മങ്ങള്‍ പ്രവര്‍ത്തിക്കുകയും ചെയ്തവര്‍ക്കൊഴികെ. അവര്‍ക്ക് മുറിഞ്ഞു പോകാത്ത പ്രതിഫലമുണ്ട്‌.
\end{malayalam}}
\chapter{\textmalayalam{ബുറൂജ് ( നക്ഷത്രമണ്ഡലങ്ങള്‍ )}}
\begin{Arabic}
\Huge{\centerline{\basmalah}}\end{Arabic}
\flushright{\begin{Arabic}
\quranayah[85][1]
\end{Arabic}}
\flushleft{\begin{malayalam}
നക്ഷത്രമണ്ഡലങ്ങളുള്ള ആകാശം തന്നെയാണ സത്യം.
\end{malayalam}}
\flushright{\begin{Arabic}
\quranayah[85][2]
\end{Arabic}}
\flushleft{\begin{malayalam}
വാഗ്ദാനം ചെയ്യപ്പെട്ട ആ ദിവസം തന്നെയാണ സത്യം.
\end{malayalam}}
\flushright{\begin{Arabic}
\quranayah[85][3]
\end{Arabic}}
\flushleft{\begin{malayalam}
സാക്ഷിയും സാക്ഷ്യം വഹിക്കപ്പെടുന്ന കാര്യവും തന്നെയാണ സത്യം.
\end{malayalam}}
\flushright{\begin{Arabic}
\quranayah[85][4]
\end{Arabic}}
\flushleft{\begin{malayalam}
ആ കിടങ്ങിന്‍റെ ആള്‍ക്കാര്‍ നശിച്ചു പോകട്ടെ.
\end{malayalam}}
\flushright{\begin{Arabic}
\quranayah[85][5]
\end{Arabic}}
\flushleft{\begin{malayalam}
അതായത് വിറകു നിറച്ച തീയുടെ ആള്‍ക്കാര്‍.
\end{malayalam}}
\flushright{\begin{Arabic}
\quranayah[85][6]
\end{Arabic}}
\flushleft{\begin{malayalam}
അവര്‍ അതിങ്കല്‍ ഇരിക്കുന്നവരായിരുന്ന സന്ദര്‍ഭം.
\end{malayalam}}
\flushright{\begin{Arabic}
\quranayah[85][7]
\end{Arabic}}
\flushleft{\begin{malayalam}
സത്യവിശ്വാസികളെക്കൊണ്ട് തങ്ങള്‍ ചെയ്യുന്നതിന് അവര്‍ ദൃക്‌സാക്ഷികളായിരുന്നു.
\end{malayalam}}
\flushright{\begin{Arabic}
\quranayah[85][8]
\end{Arabic}}
\flushleft{\begin{malayalam}
പ്രതാപശാലിയും സ്തുത്യര്‍ഹനുമായ അല്ലാഹുവില്‍ അവര്‍ വിശ്വസിക്കുന്നു എന്നത് മാത്രമായിരുന്നു അവരുടെ (സത്യവിശ്വാസികളുടെ) മേല്‍ അവര്‍ (മര്‍ദ്ദകര്‍) ചുമത്തിയ കുറ്റം.
\end{malayalam}}
\flushright{\begin{Arabic}
\quranayah[85][9]
\end{Arabic}}
\flushleft{\begin{malayalam}
ആകാശങ്ങളുടെയും ഭൂമിയുടെയും മേല്‍ ആധിപത്യം ഉള്ളവനുമായ (അല്ലാഹുവില്‍). അല്ലാഹു എല്ലാകാര്യത്തിനും സാക്ഷിയാകുന്നു.
\end{malayalam}}
\flushright{\begin{Arabic}
\quranayah[85][10]
\end{Arabic}}
\flushleft{\begin{malayalam}
സത്യവിശ്വാസികളെയും സത്യവിശ്വാസിനികളെയും മര്‍ദ്ദിക്കുകയും, പിന്നീട് പശ്ചാത്തപിക്കാതിരിക്കുകയും ചെയ്തവരാരോ അവര്‍ക്കു നരകശിക്ഷയുണ്ട്‌. തീര്‍ച്ച. അവര്‍ക്ക് ചുട്ടുകരിക്കുന്ന ശിക്ഷയുണ്ട്‌.
\end{malayalam}}
\flushright{\begin{Arabic}
\quranayah[85][11]
\end{Arabic}}
\flushleft{\begin{malayalam}
വിശ്വസിക്കുകയും സല്‍കര്‍മ്മങ്ങള്‍ പ്രവര്‍ത്തിക്കുകയും ചെയ്തവരാരോ അവര്‍ക്ക് താഴ്ഭാഗത്തുകൂടി അരുവികള്‍ ഒഴുകുന്ന സ്വര്‍ഗത്തോപ്പുകളുണ്ട്‌; തീര്‍ച്ച. അതത്രെ വലിയ വിജയം.
\end{malayalam}}
\flushright{\begin{Arabic}
\quranayah[85][12]
\end{Arabic}}
\flushleft{\begin{malayalam}
തീര്‍ച്ചയായും നിന്‍റെ രക്ഷിതാവിന്‍റെ പിടുത്തം കഠിനമായത് തന്നെയാകുന്നു.
\end{malayalam}}
\flushright{\begin{Arabic}
\quranayah[85][13]
\end{Arabic}}
\flushleft{\begin{malayalam}
തീര്‍ച്ചയായും അവന്‍ തന്നെയാണ് ആദ്യമായി ഉണ്ടാക്കുന്നതും ആവര്‍ത്തിച്ച് ഉണ്ടാക്കുന്നതും.
\end{malayalam}}
\flushright{\begin{Arabic}
\quranayah[85][14]
\end{Arabic}}
\flushleft{\begin{malayalam}
അവന്‍ ഏറെ പൊറുക്കുന്നവനും ഏറെ സ്നേഹമുള്ളവനും,
\end{malayalam}}
\flushright{\begin{Arabic}
\quranayah[85][15]
\end{Arabic}}
\flushleft{\begin{malayalam}
സിംഹാസനത്തിന്‍റെ ഉടമയും, മഹത്വമുള്ളവനും,
\end{malayalam}}
\flushright{\begin{Arabic}
\quranayah[85][16]
\end{Arabic}}
\flushleft{\begin{malayalam}
താന്‍ ഉദ്ദേശിക്കുന്നതെന്തോ അത് തികച്ചും പ്രാവര്‍ത്തികമാക്കുന്നവനുമാണ്‌.
\end{malayalam}}
\flushright{\begin{Arabic}
\quranayah[85][17]
\end{Arabic}}
\flushleft{\begin{malayalam}
ആ സൈന്യങ്ങളുടെ വര്‍ത്തമാനം നിനക്ക് വന്നുകിട്ടിയിരിക്കുന്നോ?
\end{malayalam}}
\flushright{\begin{Arabic}
\quranayah[85][18]
\end{Arabic}}
\flushleft{\begin{malayalam}
അഥവാ ഫിര്‍ഔന്‍റെയും ഥമൂദിന്‍റെയും (വര്‍ത്തമാനം).
\end{malayalam}}
\flushright{\begin{Arabic}
\quranayah[85][19]
\end{Arabic}}
\flushleft{\begin{malayalam}
അല്ല, സത്യനിഷേധികള്‍ നിഷേധിച്ചു തള്ളുന്നതിലാകുന്നു ഏര്‍പെട്ടിട്ടുള്ളത്‌.
\end{malayalam}}
\flushright{\begin{Arabic}
\quranayah[85][20]
\end{Arabic}}
\flushleft{\begin{malayalam}
അല്ലാഹു അവരുടെ പിന്‍വശത്തുകൂടി (അവരെ) വലയം ചെയ്തുകൊണ്ടിരിക്കുന്നവനാകുന്നു.
\end{malayalam}}
\flushright{\begin{Arabic}
\quranayah[85][21]
\end{Arabic}}
\flushleft{\begin{malayalam}
അല്ല, അത് മഹത്വമേറിയ ഒരു ഖുര്‍ആനാകുന്നു.
\end{malayalam}}
\flushright{\begin{Arabic}
\quranayah[85][22]
\end{Arabic}}
\flushleft{\begin{malayalam}
സംരക്ഷിതമായ ഒരു ഫലകത്തിലാണ് അതുള്ളത്‌.
\end{malayalam}}
\chapter{\textmalayalam{ത്വാരിഖ് ( രാത്രിയില്‍ വരുന്നത് )}}
\begin{Arabic}
\Huge{\centerline{\basmalah}}\end{Arabic}
\flushright{\begin{Arabic}
\quranayah[86][1]
\end{Arabic}}
\flushleft{\begin{malayalam}
ആകാശം തന്നെയാണ, രാത്രിയില്‍ വരുന്നതു തന്നെയാണ സത്യം.
\end{malayalam}}
\flushright{\begin{Arabic}
\quranayah[86][2]
\end{Arabic}}
\flushleft{\begin{malayalam}
രാത്രിയില്‍ വരുന്നത് എന്നാല്‍ എന്താണെന്ന് നിനക്കറിയുമോ?
\end{malayalam}}
\flushright{\begin{Arabic}
\quranayah[86][3]
\end{Arabic}}
\flushleft{\begin{malayalam}
തുളച്ച് കയറുന്ന നക്ഷത്രമത്രെ അത്‌.
\end{malayalam}}
\flushright{\begin{Arabic}
\quranayah[86][4]
\end{Arabic}}
\flushleft{\begin{malayalam}
തന്‍റെ കാര്യത്തില്‍ ഒരു മേല്‍നോട്ടക്കാരനുണ്ടായിക്കൊണ്ടല്ലാതെ യാതൊരാളുമില്ല.
\end{malayalam}}
\flushright{\begin{Arabic}
\quranayah[86][5]
\end{Arabic}}
\flushleft{\begin{malayalam}
എന്നാല്‍ മനുഷ്യന്‍ ചിന്തിച്ചു നോക്കട്ടെ താന്‍ എന്തില്‍ നിന്നാണ് സൃഷ്ടിക്കപ്പെട്ടിരിക്കുന്നത് എന്ന്‌
\end{malayalam}}
\flushright{\begin{Arabic}
\quranayah[86][6]
\end{Arabic}}
\flushleft{\begin{malayalam}
തെറിച്ചു വീഴുന്ന ഒരു ദ്രാവകത്തില്‍ നിന്നത്രെ അവന്‍ സൃഷ്ടിക്കപ്പെട്ടിരിക്കുന്നത്‌.
\end{malayalam}}
\flushright{\begin{Arabic}
\quranayah[86][7]
\end{Arabic}}
\flushleft{\begin{malayalam}
മുതുകെല്ലിനും, വാരിയെല്ലുകള്‍ക്കുമിടയില്‍ നിന്ന് അത് പുറത്തു വരുന്നു.
\end{malayalam}}
\flushright{\begin{Arabic}
\quranayah[86][8]
\end{Arabic}}
\flushleft{\begin{malayalam}
അവനെ (മനുഷ്യനെ) തിരിച്ചുകൊണ്ടു വരാന്‍ തീര്‍ച്ചയായും അവന്‍ (അല്ലാഹു) കഴിവുള്ളവനാകുന്നു.
\end{malayalam}}
\flushright{\begin{Arabic}
\quranayah[86][9]
\end{Arabic}}
\flushleft{\begin{malayalam}
രഹസ്യങ്ങള്‍ പരിശോധിക്കപ്പെടുന്ന ദിവസം
\end{malayalam}}
\flushright{\begin{Arabic}
\quranayah[86][10]
\end{Arabic}}
\flushleft{\begin{malayalam}
അപ്പോള്‍ അവന് (മനുഷ്യന്‌) യാതൊരു ശക്തിയോ, സഹായിയോ ഉണ്ടായിരിക്കുകയില്ല.
\end{malayalam}}
\flushright{\begin{Arabic}
\quranayah[86][11]
\end{Arabic}}
\flushleft{\begin{malayalam}
ആവര്‍ത്തിച്ച് മഴപെയ്യിക്കുന്ന ആകാശത്തെക്കൊണ്ടും
\end{malayalam}}
\flushright{\begin{Arabic}
\quranayah[86][12]
\end{Arabic}}
\flushleft{\begin{malayalam}
സസ്യലതാദികള്‍ മുളപ്പിക്കുന്ന ഭൂമിയെക്കൊണ്ടും സത്യം.
\end{malayalam}}
\flushright{\begin{Arabic}
\quranayah[86][13]
\end{Arabic}}
\flushleft{\begin{malayalam}
തീര്‍ച്ചയായും ഇതു നിര്‍ണായകമായ ഒരു വാക്കാകുന്നു.
\end{malayalam}}
\flushright{\begin{Arabic}
\quranayah[86][14]
\end{Arabic}}
\flushleft{\begin{malayalam}
ഇതു തമാശയല്ല.
\end{malayalam}}
\flushright{\begin{Arabic}
\quranayah[86][15]
\end{Arabic}}
\flushleft{\begin{malayalam}
തീര്‍ച്ചയായും അവര്‍ (വലിയ) തന്ത്രം പ്രയോഗിച്ചു കൊണ്ടിരിക്കും.
\end{malayalam}}
\flushright{\begin{Arabic}
\quranayah[86][16]
\end{Arabic}}
\flushleft{\begin{malayalam}
ഞാനും (വലിയ) തന്ത്രം പ്രയോഗിച്ചു കൊണ്ടിരിക്കും.
\end{malayalam}}
\flushright{\begin{Arabic}
\quranayah[86][17]
\end{Arabic}}
\flushleft{\begin{malayalam}
ആകയാല്‍ (നബിയേ,) നീ സത്യനിഷേധികള്‍ക്ക് കാലതാമസം നല്‍കുക. അല്‍പസമയത്തേക്ക് അവര്‍ക്ക് താമസം നല്‍കിയേക്കുക.
\end{malayalam}}
\chapter{\textmalayalam{അഅ്അലാ ( അത്യുന്നതന്‍ )}}
\begin{Arabic}
\Huge{\centerline{\basmalah}}\end{Arabic}
\flushright{\begin{Arabic}
\quranayah[87][1]
\end{Arabic}}
\flushleft{\begin{malayalam}
അത്യുന്നതനായ നിന്‍റെ രക്ഷിതാവിന്‍റെ നാമം പ്രകീര്‍ത്തിക്കുക.
\end{malayalam}}
\flushright{\begin{Arabic}
\quranayah[87][2]
\end{Arabic}}
\flushleft{\begin{malayalam}
സൃഷ്ടിക്കുകയും, സംവിധാനിക്കുകയും ചെയ്ത (രക്ഷിതാവിന്‍റെ)
\end{malayalam}}
\flushright{\begin{Arabic}
\quranayah[87][3]
\end{Arabic}}
\flushleft{\begin{malayalam}
വ്യവസ്ഥ നിര്‍ണയിച്ചു മാര്‍ഗദര്‍ശനം നല്‍കിയവനും,
\end{malayalam}}
\flushright{\begin{Arabic}
\quranayah[87][4]
\end{Arabic}}
\flushleft{\begin{malayalam}
മേച്ചില്‍ പുറങ്ങള്‍ ഉല്‍പാദിപ്പിച്ചവനും
\end{malayalam}}
\flushright{\begin{Arabic}
\quranayah[87][5]
\end{Arabic}}
\flushleft{\begin{malayalam}
എന്നിട്ട് അതിനെ ഉണങ്ങിക്കരിഞ്ഞ ചവറാക്കി തീര്‍ത്തവനുമായ (രക്ഷിതാവിന്‍റെ നാമം)
\end{malayalam}}
\flushright{\begin{Arabic}
\quranayah[87][6]
\end{Arabic}}
\flushleft{\begin{malayalam}
നിനക്ക് നാം ഓതിത്തരാം. നീ മറന്നുപോകുകയില്ല.
\end{malayalam}}
\flushright{\begin{Arabic}
\quranayah[87][7]
\end{Arabic}}
\flushleft{\begin{malayalam}
അല്ലാഹു ഉദ്ദേശിച്ചതൊഴികെ. തീര്‍ച്ചയായും അവന്‍ പരസ്യമായതും, രഹസ്യമായിരിക്കുന്നതും അറിയുന്നു.
\end{malayalam}}
\flushright{\begin{Arabic}
\quranayah[87][8]
\end{Arabic}}
\flushleft{\begin{malayalam}
കൂടുതല്‍ എളുപ്പമുള്ളതിലേക്ക് നിനക്ക് നാം സൌകര്യമുണ്ടാക്കിത്തരുന്നതുമാണ്‌.
\end{malayalam}}
\flushright{\begin{Arabic}
\quranayah[87][9]
\end{Arabic}}
\flushleft{\begin{malayalam}
അതിനാല്‍ ഉപദേശം ഫലപ്പെടുന്നുവെങ്കില്‍ നീ ഉപദേശിച്ചു കൊള്ളുക.
\end{malayalam}}
\flushright{\begin{Arabic}
\quranayah[87][10]
\end{Arabic}}
\flushleft{\begin{malayalam}
ഭയപ്പെടുന്നവര്‍ ഉപദേശം സ്വീകരിച്ചു കൊള്ളുന്നതാണ്‌.
\end{malayalam}}
\flushright{\begin{Arabic}
\quranayah[87][11]
\end{Arabic}}
\flushleft{\begin{malayalam}
ഏറ്റവും നിര്‍ഭാഗ്യവാനായിട്ടുള്ളവന്‍ അതിനെ (ഉപദേശത്തെ) വിട്ടകന്നു പോകുന്നതാണ്‌.
\end{malayalam}}
\flushright{\begin{Arabic}
\quranayah[87][12]
\end{Arabic}}
\flushleft{\begin{malayalam}
വലിയ അഗ്നിയില്‍ കടന്ന് എരിയുന്നവനത്രെ അവന്‍
\end{malayalam}}
\flushright{\begin{Arabic}
\quranayah[87][13]
\end{Arabic}}
\flushleft{\begin{malayalam}
പിന്നീട് അവന്‍ അതില്‍ മരിക്കുകയില്ല. ജീവിക്കുകയുമില്ല.
\end{malayalam}}
\flushright{\begin{Arabic}
\quranayah[87][14]
\end{Arabic}}
\flushleft{\begin{malayalam}
തീര്‍ച്ചയായും പരിശുദ്ധി നേടിയവര്‍ വിജയം പ്രാപിച്ചു.
\end{malayalam}}
\flushright{\begin{Arabic}
\quranayah[87][15]
\end{Arabic}}
\flushleft{\begin{malayalam}
തന്‍റെ രക്ഷിതാവിന്‍റെ നാമം സ്മരിക്കുകയും എന്നിട്ട് നമസ്കരിക്കുകയും (ചെയ്തവന്‍)
\end{malayalam}}
\flushright{\begin{Arabic}
\quranayah[87][16]
\end{Arabic}}
\flushleft{\begin{malayalam}
പക്ഷെ, നിങ്ങള്‍ ഐഹികജീവിതത്തിന്ന് കൂടുതല്‍ പ്രാധാന്യം നല്‍കുന്നു.
\end{malayalam}}
\flushright{\begin{Arabic}
\quranayah[87][17]
\end{Arabic}}
\flushleft{\begin{malayalam}
പരലോകമാകുന്നു ഏറ്റവും ഉത്തമവും നിലനില്‍ക്കുന്നതും.
\end{malayalam}}
\flushright{\begin{Arabic}
\quranayah[87][18]
\end{Arabic}}
\flushleft{\begin{malayalam}
തീര്‍ച്ചയായും ഇത് ആദ്യത്തെ ഏടുകളില്‍ തന്നെയുണ്ട്‌.
\end{malayalam}}
\flushright{\begin{Arabic}
\quranayah[87][19]
\end{Arabic}}
\flushleft{\begin{malayalam}
അതായത് ഇബ്രാഹീമിന്‍റെയും മൂസായുടെയും ഏടുകളില്‍.
\end{malayalam}}
\chapter{\textmalayalam{ഗാശിയ ( മൂടുന്ന സംഭവം )}}
\begin{Arabic}
\Huge{\centerline{\basmalah}}\end{Arabic}
\flushright{\begin{Arabic}
\quranayah[88][1]
\end{Arabic}}
\flushleft{\begin{malayalam}
(നബിയേ,) ആ മൂടുന്ന സംഭവത്തെ സംബന്ധിച്ച വര്‍ത്തമാനം നിനക്ക് വന്നുകിട്ടിയോ?
\end{malayalam}}
\flushright{\begin{Arabic}
\quranayah[88][2]
\end{Arabic}}
\flushleft{\begin{malayalam}
അന്നേ ദിവസം ചില മുഖങ്ങള്‍ താഴ്മകാണിക്കുന്നതും
\end{malayalam}}
\flushright{\begin{Arabic}
\quranayah[88][3]
\end{Arabic}}
\flushleft{\begin{malayalam}
പണിയെടുത്ത് ക്ഷീണിച്ചതുമായിരിക്കും.
\end{malayalam}}
\flushright{\begin{Arabic}
\quranayah[88][4]
\end{Arabic}}
\flushleft{\begin{malayalam}
ചൂടേറിയ അഗ്നിയില്‍ അവ പ്രവേശിക്കുന്നതാണ്‌.
\end{malayalam}}
\flushright{\begin{Arabic}
\quranayah[88][5]
\end{Arabic}}
\flushleft{\begin{malayalam}
ചുട്ടുതിളക്കുന്ന ഒരു ഉറവില്‍ നിന്ന് അവര്‍ക്കു കുടിപ്പിക്കപ്പെടുന്നതാണ്‌.
\end{malayalam}}
\flushright{\begin{Arabic}
\quranayah[88][6]
\end{Arabic}}
\flushleft{\begin{malayalam}
ളരീഇല്‍ നിന്നല്ലാതെ അവര്‍ക്ക് യാതൊരു ആഹാരവുമില്ല.
\end{malayalam}}
\flushright{\begin{Arabic}
\quranayah[88][7]
\end{Arabic}}
\flushleft{\begin{malayalam}
അത് പോഷണം നല്‍കുകയില്ല. വിശപ്പിന് ശമനമുണ്ടാക്കുകയുമില്ല.
\end{malayalam}}
\flushright{\begin{Arabic}
\quranayah[88][8]
\end{Arabic}}
\flushleft{\begin{malayalam}
ചില മുഖങ്ങള്‍ അന്നു തുടുത്തു മിനുത്തതായിരിക്കും.
\end{malayalam}}
\flushright{\begin{Arabic}
\quranayah[88][9]
\end{Arabic}}
\flushleft{\begin{malayalam}
അവയുടെ പ്രയത്നത്തെപ്പറ്റി തൃപ്തിയടഞ്ഞവയുമായിരിക്കും.
\end{malayalam}}
\flushright{\begin{Arabic}
\quranayah[88][10]
\end{Arabic}}
\flushleft{\begin{malayalam}
ഉന്നതമായ സ്വര്‍ഗത്തില്‍.
\end{malayalam}}
\flushright{\begin{Arabic}
\quranayah[88][11]
\end{Arabic}}
\flushleft{\begin{malayalam}
അവിടെ യാതൊരു നിരര്‍ത്ഥകമായ വാക്കും അവര്‍ കേള്‍ക്കുകയില്ല.
\end{malayalam}}
\flushright{\begin{Arabic}
\quranayah[88][12]
\end{Arabic}}
\flushleft{\begin{malayalam}
അതില്‍ ഒഴുകി കൊണ്ടിരിക്കുന്ന അരുവിയുണ്ട്‌.
\end{malayalam}}
\flushright{\begin{Arabic}
\quranayah[88][13]
\end{Arabic}}
\flushleft{\begin{malayalam}
അതില്‍ ഉയര്‍ത്തിവെക്കപ്പെട്ട കട്ടിലുകളും,
\end{malayalam}}
\flushright{\begin{Arabic}
\quranayah[88][14]
\end{Arabic}}
\flushleft{\begin{malayalam}
തയ്യാറാക്കി വെക്കപ്പെട്ട കോപ്പകളും,
\end{malayalam}}
\flushright{\begin{Arabic}
\quranayah[88][15]
\end{Arabic}}
\flushleft{\begin{malayalam}
അണിയായി വെക്കപ്പെട്ട തലയണകളും,
\end{malayalam}}
\flushright{\begin{Arabic}
\quranayah[88][16]
\end{Arabic}}
\flushleft{\begin{malayalam}
വിരിച്ചുവെക്കപ്പെട്ട പരവതാനികളുമുണ്ട്‌.
\end{malayalam}}
\flushright{\begin{Arabic}
\quranayah[88][17]
\end{Arabic}}
\flushleft{\begin{malayalam}
ഒട്ടകത്തിന്‍റെ നേര്‍ക്ക് അവര്‍ നോക്കുന്നില്ലേ? അത് എങ്ങനെ സൃഷ്ടിക്കപ്പെട്ടിരിക്കുന്നു എന്ന്‌.
\end{malayalam}}
\flushright{\begin{Arabic}
\quranayah[88][18]
\end{Arabic}}
\flushleft{\begin{malayalam}
ആകാശത്തേക്ക് (അവര്‍ നോക്കുന്നില്ലേ?) അത് എങ്ങനെ ഉയര്‍ത്തപ്പെട്ടിരിക്കുന്നു എന്ന്‌.
\end{malayalam}}
\flushright{\begin{Arabic}
\quranayah[88][19]
\end{Arabic}}
\flushleft{\begin{malayalam}
പര്‍വ്വതങ്ങളിലേക്ക് (അവര്‍ നോക്കുന്നില്ലേ?) അവ എങ്ങനെ നാട്ടിനിര്‍ത്തപ്പെട്ടിരിക്കുന്നു വെന്ന്‌.
\end{malayalam}}
\flushright{\begin{Arabic}
\quranayah[88][20]
\end{Arabic}}
\flushleft{\begin{malayalam}
ഭൂമിയിലേക്ക് (അവര്‍ നോക്കുന്നില്ലേ?) അത് എങ്ങനെ പരത്തപ്പെട്ടിരിക്കുന്നുവെന്ന്‌
\end{malayalam}}
\flushright{\begin{Arabic}
\quranayah[88][21]
\end{Arabic}}
\flushleft{\begin{malayalam}
അതിനാല്‍ (നബിയേ,) നീ ഉല്‍ബോധിപ്പിക്കുക. നീ ഒരു ഉല്‍ബോധകന്‍ മാത്രമാകുന്നു.
\end{malayalam}}
\flushright{\begin{Arabic}
\quranayah[88][22]
\end{Arabic}}
\flushleft{\begin{malayalam}
നീ അവരുടെ മേല്‍ അധികാരം ചെലുത്തേണ്ടവനല്ല.
\end{malayalam}}
\flushright{\begin{Arabic}
\quranayah[88][23]
\end{Arabic}}
\flushleft{\begin{malayalam}
പക്ഷെ, വല്ലവനും തിരിഞ്ഞുകളയുകയും, അവിശ്വസിക്കുകയും ചെയ്യുന്ന പക്ഷം
\end{malayalam}}
\flushright{\begin{Arabic}
\quranayah[88][24]
\end{Arabic}}
\flushleft{\begin{malayalam}
അല്ലാഹു അവനെ ഏറ്റവും വലിയ ശിക്ഷ ശിക്ഷിക്കുന്നതാണ്‌.
\end{malayalam}}
\flushright{\begin{Arabic}
\quranayah[88][25]
\end{Arabic}}
\flushleft{\begin{malayalam}
തീര്‍ച്ചയായും നമ്മുടെ അടുത്തേക്കാണ് അവരുടെ മടക്കം.
\end{malayalam}}
\flushright{\begin{Arabic}
\quranayah[88][26]
\end{Arabic}}
\flushleft{\begin{malayalam}
പിന്നീട്‌, തീര്‍ച്ചയായും നമ്മുടെ ബാധ്യതയാണ് അവരുടെ വിചാരണ.
\end{malayalam}}
\chapter{\textmalayalam{ഫജ്ര്‍ ( പ്രഭാതം )}}
\begin{Arabic}
\Huge{\centerline{\basmalah}}\end{Arabic}
\flushright{\begin{Arabic}
\quranayah[89][1]
\end{Arabic}}
\flushleft{\begin{malayalam}
പ്രഭാതം തന്നെയാണ സത്യം.
\end{malayalam}}
\flushright{\begin{Arabic}
\quranayah[89][2]
\end{Arabic}}
\flushleft{\begin{malayalam}
പത്തു രാത്രികള്‍ തന്നെയാണ സത്യം.
\end{malayalam}}
\flushright{\begin{Arabic}
\quranayah[89][3]
\end{Arabic}}
\flushleft{\begin{malayalam}
ഇരട്ടയും ഒറ്റയും തന്നെയാണ സത്യം
\end{malayalam}}
\flushright{\begin{Arabic}
\quranayah[89][4]
\end{Arabic}}
\flushleft{\begin{malayalam}
രാത്രി സഞ്ചരിച്ച് കൊണ്ടിരിക്കെ അത് തന്നെയാണ സത്യം.
\end{malayalam}}
\flushright{\begin{Arabic}
\quranayah[89][5]
\end{Arabic}}
\flushleft{\begin{malayalam}
അതില്‍ (മേല്‍ പറഞ്ഞവയില്‍) കാര്യബോധമുള്ളവന്ന് സത്യത്തിന് വകയുണേ്ടാ?
\end{malayalam}}
\flushright{\begin{Arabic}
\quranayah[89][6]
\end{Arabic}}
\flushleft{\begin{malayalam}
ആദ് സമുദായത്തെ കൊണ്ട് നിന്‍റെ രക്ഷിതാവ് എന്തു ചെയ്തുവെന്ന് നീ കണ്ടില്ലേ?
\end{malayalam}}
\flushright{\begin{Arabic}
\quranayah[89][7]
\end{Arabic}}
\flushleft{\begin{malayalam}
അതായത് തൂണുകളുടെ ഉടമകളായ ഇറം ഗോത്രത്തെ കൊണ്ട്‌
\end{malayalam}}
\flushright{\begin{Arabic}
\quranayah[89][8]
\end{Arabic}}
\flushleft{\begin{malayalam}
തത്തുല്യമായിട്ടൊന്ന് രാജ്യങ്ങളില്‍ സൃഷ്ടിക്കപ്പെട്ടിട്ടില്ലാത്ത ഗോത്രം.
\end{malayalam}}
\flushright{\begin{Arabic}
\quranayah[89][9]
\end{Arabic}}
\flushleft{\begin{malayalam}
താഴ്‌വരയില്‍ പാറവെട്ടി കെട്ടിടമുണ്ടാക്കിയവരായ ഥമൂദ് ഗോത്രത്തെക്കൊണ്ടും
\end{malayalam}}
\flushright{\begin{Arabic}
\quranayah[89][10]
\end{Arabic}}
\flushleft{\begin{malayalam}
ആണികളുടെ ആളായ ഫിര്‍ഔനെക്കൊണ്ടും.
\end{malayalam}}
\flushright{\begin{Arabic}
\quranayah[89][11]
\end{Arabic}}
\flushleft{\begin{malayalam}
നാടുകളില്‍ അതിക്രമം പ്രവര്‍ത്തിക്കുകയും
\end{malayalam}}
\flushright{\begin{Arabic}
\quranayah[89][12]
\end{Arabic}}
\flushleft{\begin{malayalam}
അവിടെ കുഴപ്പം വര്‍ദ്ധിപ്പിക്കുകയും ചെയ്തവരാണവര്‍.
\end{malayalam}}
\flushright{\begin{Arabic}
\quranayah[89][13]
\end{Arabic}}
\flushleft{\begin{malayalam}
അതിനാല്‍ നിന്‍റെ രക്ഷിതാവ് അവരുടെ മേല്‍ ശിക്ഷയുടെ ചമ്മട്ടി വര്‍ഷിച്ചു.
\end{malayalam}}
\flushright{\begin{Arabic}
\quranayah[89][14]
\end{Arabic}}
\flushleft{\begin{malayalam}
തീര്‍ച്ചയായും നിന്‍റെ രക്ഷിതാവ് പതിയിരിക്കുന്ന സ്ഥാനത്തു തന്നെയുണ്ട്‌.
\end{malayalam}}
\flushright{\begin{Arabic}
\quranayah[89][15]
\end{Arabic}}
\flushleft{\begin{malayalam}
എന്നാല്‍ മനുഷ്യനെ അവന്‍റെ രക്ഷിതാവ് പരീക്ഷിക്കുകയും അങ്ങനെ അവനെ ആദരിക്കുകയും അവന് സൌഖ്യം നല്‍കുകയും ചെയ്താല്‍ അവന്‍ പറയും; എന്‍റെ രക്ഷിതാവ് എന്നെ ആദരിച്ചിരിക്കുന്നു എന്ന്‌.
\end{malayalam}}
\flushright{\begin{Arabic}
\quranayah[89][16]
\end{Arabic}}
\flushleft{\begin{malayalam}
എന്നാല്‍ അവനെ (മനുഷ്യനെ) അവന്‍ പരീക്ഷിക്കുകയും എന്നിട്ടവന്‍റെ ഉപജീവനം ഇടുങ്ങിയതാക്കുകയും ചെയ്താല്‍ അവന്‍ പറയും; എന്‍റെ രക്ഷിതാവ് എന്നെ അപമാനിച്ചിരിക്കുന്നു എന്ന്‌.
\end{malayalam}}
\flushright{\begin{Arabic}
\quranayah[89][17]
\end{Arabic}}
\flushleft{\begin{malayalam}
അല്ല, പക്ഷെ നിങ്ങള്‍ അനാഥയെ ആദരിക്കുന്നില്ല.
\end{malayalam}}
\flushright{\begin{Arabic}
\quranayah[89][18]
\end{Arabic}}
\flushleft{\begin{malayalam}
പാവപ്പെട്ടവന്‍റെ ആഹാരത്തിന് നിങ്ങള്‍ പ്രോത്സാഹനം നല്‍കുന്നുമില്ല.
\end{malayalam}}
\flushright{\begin{Arabic}
\quranayah[89][19]
\end{Arabic}}
\flushleft{\begin{malayalam}
അനന്തരാവകാശ സ്വത്ത് നിങ്ങള്‍ വാരിക്കൂട്ടി തിന്നുകയും ചെയ്യുന്നു.
\end{malayalam}}
\flushright{\begin{Arabic}
\quranayah[89][20]
\end{Arabic}}
\flushleft{\begin{malayalam}
ധനത്തെ നിങ്ങള്‍ അമിതമായ തോതില്‍ സ്നേഹിക്കുകയും ചെയ്യുന്നു.
\end{malayalam}}
\flushright{\begin{Arabic}
\quranayah[89][21]
\end{Arabic}}
\flushleft{\begin{malayalam}
അല്ല, ഭൂമി പൊടിപൊടിയായി പൊടിക്കപ്പെടുകയും,
\end{malayalam}}
\flushright{\begin{Arabic}
\quranayah[89][22]
\end{Arabic}}
\flushleft{\begin{malayalam}
നിന്‍റെ രക്ഷിതാവും, അണിയണിയായി മലക്കുകളും വരുകയും,
\end{malayalam}}
\flushright{\begin{Arabic}
\quranayah[89][23]
\end{Arabic}}
\flushleft{\begin{malayalam}
അന്ന് നരകം കൊണ്ടു വരപ്പെടുകയും ചെയ്താല്‍! അന്നേ ദിവസം മനുഷ്യന്ന് ഓര്‍മ വരുന്നതാണ്‌. എവിടെനിന്നാണവന്ന് ഓര്‍മ വരുന്നത്‌?
\end{malayalam}}
\flushright{\begin{Arabic}
\quranayah[89][24]
\end{Arabic}}
\flushleft{\begin{malayalam}
അവന്‍ പറയും. അയ്യോ, ഞാന്‍ എന്‍റെ ജീവിതത്തിനു വേണ്ടി മുന്‍കൂട്ടി (സല്‍കര്‍മ്മങ്ങള്‍) ചെയ്തുവെച്ചിരുന്നെങ്കില്‍ എത്ര നന്നായിരുന്നേനെ!
\end{malayalam}}
\flushright{\begin{Arabic}
\quranayah[89][25]
\end{Arabic}}
\flushleft{\begin{malayalam}
അപ്പോള്‍ അന്നേ ദിവസം അല്ലാഹു ശിക്ഷിക്കുന്നപ്രകാരം ഒരാളും ശിക്ഷിക്കുകയില്ല.
\end{malayalam}}
\flushright{\begin{Arabic}
\quranayah[89][26]
\end{Arabic}}
\flushleft{\begin{malayalam}
അവന്‍ പിടിച്ചു ബന്ധിക്കുന്നത് പോലെ ഒരാളും പിടിച്ചു ബന്ധിക്കുന്നതുമല്ല.
\end{malayalam}}
\flushright{\begin{Arabic}
\quranayah[89][27]
\end{Arabic}}
\flushleft{\begin{malayalam}
ഹേ; സമാധാനമടഞ്ഞ ആത്മാവേ,
\end{malayalam}}
\flushright{\begin{Arabic}
\quranayah[89][28]
\end{Arabic}}
\flushleft{\begin{malayalam}
നീ നിന്‍റെ രക്ഷിതാവിങ്കലേക്ക് തൃപ്തിപ്പെട്ടുകൊണ്ടും, തൃപ്തി ലഭിച്ചു കൊണ്ടും മടങ്ങിക്കൊള്ളുക.
\end{malayalam}}
\flushright{\begin{Arabic}
\quranayah[89][29]
\end{Arabic}}
\flushleft{\begin{malayalam}
എന്നിട്ട് എന്‍റെ അടിയാന്‍മാരുടെ കൂട്ടത്തില്‍ പ്രവേശിച്ചു കൊള്ളുക.
\end{malayalam}}
\flushright{\begin{Arabic}
\quranayah[89][30]
\end{Arabic}}
\flushleft{\begin{malayalam}
എന്‍റെ സ്വര്‍ഗത്തില്‍ പ്രവേശിച്ചു കൊള്ളുക.
\end{malayalam}}
\chapter{\textmalayalam{ബലദ് ( രാജ്യം )}}
\begin{Arabic}
\Huge{\centerline{\basmalah}}\end{Arabic}
\flushright{\begin{Arabic}
\quranayah[90][1]
\end{Arabic}}
\flushleft{\begin{malayalam}
ഈ രാജ്യത്തെ (മക്കയെ) ക്കൊണ്ട് ഞാന്‍ സത്യം ചെയ്തു പറയുന്നു.
\end{malayalam}}
\flushright{\begin{Arabic}
\quranayah[90][2]
\end{Arabic}}
\flushleft{\begin{malayalam}
നീയാകട്ടെ ഈ രാജ്യത്തെ നിവാസിയാണ് താനും.
\end{malayalam}}
\flushright{\begin{Arabic}
\quranayah[90][3]
\end{Arabic}}
\flushleft{\begin{malayalam}
ജനയിതാവിനെയും, അവന്‍ ജനിപ്പിക്കുന്നതിനെയും തന്നെയാണ സത്യം.
\end{malayalam}}
\flushright{\begin{Arabic}
\quranayah[90][4]
\end{Arabic}}
\flushleft{\begin{malayalam}
തീര്‍ച്ചയായും മനുഷ്യനെ നാം സൃഷ്ടിച്ചിട്ടുള്ളത് ക്ലേശം സഹിക്കേണ്ട നിലയിലാകുന്നു.
\end{malayalam}}
\flushright{\begin{Arabic}
\quranayah[90][5]
\end{Arabic}}
\flushleft{\begin{malayalam}
അവനെ പിടികൂടാന്‍ ആര്‍ക്കും സാധിക്കുകയേ ഇല്ലെന്ന് അവന്‍ വിചാരിക്കുന്നുണേ്ടാ?
\end{malayalam}}
\flushright{\begin{Arabic}
\quranayah[90][6]
\end{Arabic}}
\flushleft{\begin{malayalam}
അവന്‍ പറയുന്നു: ഞാന്‍ മേല്‍ക്കുമേല്‍ പണം തുലച്ചിരിക്കുന്നു എന്ന്‌.
\end{malayalam}}
\flushright{\begin{Arabic}
\quranayah[90][7]
\end{Arabic}}
\flushleft{\begin{malayalam}
അവന്‍ വിചാരിക്കുന്നുണേ്ടാ; അവനെ ആരുംകണ്ടിട്ടില്ലെന്ന്‌?
\end{malayalam}}
\flushright{\begin{Arabic}
\quranayah[90][8]
\end{Arabic}}
\flushleft{\begin{malayalam}
അവന് നാം രണ്ട് കണ്ണുകള്‍ ഉണ്ടാക്കി കൊടുത്തിട്ടില്ലേ?
\end{malayalam}}
\flushright{\begin{Arabic}
\quranayah[90][9]
\end{Arabic}}
\flushleft{\begin{malayalam}
ഒരു നാവും രണ്ടു ചുണ്ടുകളും
\end{malayalam}}
\flushright{\begin{Arabic}
\quranayah[90][10]
\end{Arabic}}
\flushleft{\begin{malayalam}
തെളിഞ്ഞു നില്‍ക്കുന്ന രണ്ടു പാതകള്‍ അവന്നു നാം കാട്ടികൊടുക്കുകയും ചെയ്തിരിക്കുന്നു.
\end{malayalam}}
\flushright{\begin{Arabic}
\quranayah[90][11]
\end{Arabic}}
\flushleft{\begin{malayalam}
എന്നിട്ട് ആ മലമ്പാതയില്‍ അവന്‍ തള്ളിക്കടന്നില്ല.
\end{malayalam}}
\flushright{\begin{Arabic}
\quranayah[90][12]
\end{Arabic}}
\flushleft{\begin{malayalam}
ആ മലമ്പാത എന്താണെന്ന് നിനക്കറിയാമോ?
\end{malayalam}}
\flushright{\begin{Arabic}
\quranayah[90][13]
\end{Arabic}}
\flushleft{\begin{malayalam}
ഒരു അടിമയെ മോചിപ്പിക്കുക.
\end{malayalam}}
\flushright{\begin{Arabic}
\quranayah[90][14]
\end{Arabic}}
\flushleft{\begin{malayalam}
അല്ലെങ്കില്‍ പട്ടിണിയുള്ള നാളില്‍ ഭക്ഷണം കൊടുക്കുക.
\end{malayalam}}
\flushright{\begin{Arabic}
\quranayah[90][15]
\end{Arabic}}
\flushleft{\begin{malayalam}
കുടുംബബന്ധമുള്ള അനാഥയ്ക്ക്‌
\end{malayalam}}
\flushright{\begin{Arabic}
\quranayah[90][16]
\end{Arabic}}
\flushleft{\begin{malayalam}
അല്ലെങ്കില്‍ കടുത്ത ദാരിദ്യ്‌രമുള്ള സാധുവിന്‌
\end{malayalam}}
\flushright{\begin{Arabic}
\quranayah[90][17]
\end{Arabic}}
\flushleft{\begin{malayalam}
പുറമെ, വിശ്വസിക്കുകയും, ക്ഷമ കൊണ്ടും കാരുണ്യം കൊണ്ടും പരസ്പരം ഉപദേശിക്കുകയും ചെയ്തവരുടെ കൂട്ടത്തില്‍ അവന്‍ ആയിത്തീരുകയും ചെയ്യുക.
\end{malayalam}}
\flushright{\begin{Arabic}
\quranayah[90][18]
\end{Arabic}}
\flushleft{\begin{malayalam}
അങ്ങനെ ചെയ്യുന്നവരത്രെ വലതുപക്ഷക്കാര്‍.
\end{malayalam}}
\flushright{\begin{Arabic}
\quranayah[90][19]
\end{Arabic}}
\flushleft{\begin{malayalam}
നമ്മുടെ ദൃഷ്ടാന്തങ്ങള്‍ നിഷേധിച്ചവരാരോ അവരത്രെ ഇടതുപക്ഷത്തിന്‍റെ ആള്‍ക്കാര്‍.
\end{malayalam}}
\flushright{\begin{Arabic}
\quranayah[90][20]
\end{Arabic}}
\flushleft{\begin{malayalam}
അവരുടെ മേല്‍ അടച്ചുമൂടിയ നരകാഗ്നിയുണ്ട്‌.
\end{malayalam}}
\chapter{\textmalayalam{ശംസ് ( സൂര്യന്‍ )}}
\begin{Arabic}
\Huge{\centerline{\basmalah}}\end{Arabic}
\flushright{\begin{Arabic}
\quranayah[91][1]
\end{Arabic}}
\flushleft{\begin{malayalam}
സൂര്യനും അതിന്‍റെ പ്രഭയും തന്നെയാണ സത്യം.
\end{malayalam}}
\flushright{\begin{Arabic}
\quranayah[91][2]
\end{Arabic}}
\flushleft{\begin{malayalam}
ചന്ദ്രന്‍ തന്നെയാണ സത്യം; അത് അതിനെ തുടര്‍ന്ന് വരുമ്പോള്‍.
\end{malayalam}}
\flushright{\begin{Arabic}
\quranayah[91][3]
\end{Arabic}}
\flushleft{\begin{malayalam}
പകലിനെ തന്നെയാണ സത്യം; അത് അതിനെ (സൂര്യനെ) പ്രത്യക്ഷപ്പെടുത്തുമ്പേള്‍
\end{malayalam}}
\flushright{\begin{Arabic}
\quranayah[91][4]
\end{Arabic}}
\flushleft{\begin{malayalam}
രാത്രിയെ തന്നെയാണ സത്യം; അത് അതിനെ മൂടുമ്പോള്‍.
\end{malayalam}}
\flushright{\begin{Arabic}
\quranayah[91][5]
\end{Arabic}}
\flushleft{\begin{malayalam}
ആകാശത്തെയും, അതിനെ സ്ഥാപിച്ച രീതിയെയും തന്നെയാണ സത്യം.
\end{malayalam}}
\flushright{\begin{Arabic}
\quranayah[91][6]
\end{Arabic}}
\flushleft{\begin{malayalam}
ഭൂമിയെയും, അതിനെ വിസ്തൃതമാക്കിയ രീതിയെയും തന്നെയാണ സത്യം.
\end{malayalam}}
\flushright{\begin{Arabic}
\quranayah[91][7]
\end{Arabic}}
\flushleft{\begin{malayalam}
മനുഷ്യാസ്തിത്വത്തെയും അതിനെ സംവിധാനിച്ച രീതിയെയും തന്നെയാണ സത്യം.
\end{malayalam}}
\flushright{\begin{Arabic}
\quranayah[91][8]
\end{Arabic}}
\flushleft{\begin{malayalam}
എന്നിട്ട് അതിന്ന് അതിന്‍റെ ദുഷ്ടതയും അതിന്‍റെ സൂക്ഷ്മതയും സംബന്ധിച്ച് അവന്‍ ബോധം നല്‍കുകയും ചെയ്തിരിക്കുന്നു.
\end{malayalam}}
\flushright{\begin{Arabic}
\quranayah[91][9]
\end{Arabic}}
\flushleft{\begin{malayalam}
തീര്‍ച്ചയായും അതിനെ (അസ്തിത്വത്തെ) പരിശുദ്ധമാക്കിയവന്‍ വിജയം കൈവരിച്ചു.
\end{malayalam}}
\flushright{\begin{Arabic}
\quranayah[91][10]
\end{Arabic}}
\flushleft{\begin{malayalam}
അതിനെ കളങ്കപ്പെടുത്തിയവന്‍ തീര്‍ച്ചയായും നിര്‍ഭാഗ്യമടയുകയും ചെയ്തു.
\end{malayalam}}
\flushright{\begin{Arabic}
\quranayah[91][11]
\end{Arabic}}
\flushleft{\begin{malayalam}
ഥമൂദ് ഗോത്രം അതിന്‍റെ ധിക്കാരം മൂലം (സത്യത്തെ) നിഷേധിച്ചു തള്ളുകയുണ്ടായി.
\end{malayalam}}
\flushright{\begin{Arabic}
\quranayah[91][12]
\end{Arabic}}
\flushleft{\begin{malayalam}
അവരുടെ കൂട്ടത്തിലെ ഏറ്റവും ദുഷ്ടതയുള്ളവന്‍ ഒരുങ്ങി പുറപ്പെട്ട സന്ദര്‍ഭം .
\end{malayalam}}
\flushright{\begin{Arabic}
\quranayah[91][13]
\end{Arabic}}
\flushleft{\begin{malayalam}
അപ്പോള്‍ അല്ലാഹുവിന്‍റെ ദൂതന്‍ അവരോട് പറഞ്ഞു. അല്ലാഹുവിന്‍റെ ഒട്ടകത്തെയും അതിന്‍റെ വെള്ളം കുടിയും നിങ്ങള്‍ സൂക്ഷിക്കുക
\end{malayalam}}
\flushright{\begin{Arabic}
\quranayah[91][14]
\end{Arabic}}
\flushleft{\begin{malayalam}
അപ്പോള്‍ അവര്‍ അദ്ദേഹത്തെ നിഷേധിച്ചു തള്ളുകയും അതിനെ (ഒട്ടകത്തെ) അറുകൊല നടത്തുകയും ചെയ്തു. അപ്പോള്‍ അവരുടെ പാപം നിമിത്തം അവരുടെ രക്ഷിതാവ് അവര്‍ക്ക് സമൂല നാശം വരുത്തുകയും (അവര്‍ക്കെല്ലാം) അത് സമമാക്കുകയും ചെയ്തു.
\end{malayalam}}
\flushright{\begin{Arabic}
\quranayah[91][15]
\end{Arabic}}
\flushleft{\begin{malayalam}
അതിന്‍റെ അനന്തരഫലം അവന്‍ ഭയപ്പെട്ടിരുന്നുമില്ല.
\end{malayalam}}
\chapter{\textmalayalam{ലൈല്‍ ( രാത്രി )}}
\begin{Arabic}
\Huge{\centerline{\basmalah}}\end{Arabic}
\flushright{\begin{Arabic}
\quranayah[92][1]
\end{Arabic}}
\flushleft{\begin{malayalam}
രാവിനെതന്നെയാണ സത്യം ; അത് മൂടികൊണ്ടിരിക്കുമ്പോള്‍
\end{malayalam}}
\flushright{\begin{Arabic}
\quranayah[92][2]
\end{Arabic}}
\flushleft{\begin{malayalam}
പകലിനെ തന്നെയാണ സത്യം ; അത് പ്രത്യക്ഷപ്പെടുമ്പോള്‍
\end{malayalam}}
\flushright{\begin{Arabic}
\quranayah[92][3]
\end{Arabic}}
\flushleft{\begin{malayalam}
ആണിനെയും പെണ്ണിനെയും സൃഷ്ടിച്ച രീതിയെ തന്നെയാണ സത്യം;
\end{malayalam}}
\flushright{\begin{Arabic}
\quranayah[92][4]
\end{Arabic}}
\flushleft{\begin{malayalam}
തീര്‍ച്ചയായും നിങ്ങളുടെ പരിശ്രമം വിഭിന്ന രൂപത്തിലുള്ളതാകുന്നു.
\end{malayalam}}
\flushright{\begin{Arabic}
\quranayah[92][5]
\end{Arabic}}
\flushleft{\begin{malayalam}
എന്നാല്‍ ഏതൊരാള്‍ ദാനം നല്‍കുകയും, സൂക്ഷ്മത പാലിക്കുകയും
\end{malayalam}}
\flushright{\begin{Arabic}
\quranayah[92][6]
\end{Arabic}}
\flushleft{\begin{malayalam}
ഏറ്റവും ഉത്തമമായതിനെ സത്യപ്പെടുത്തുകയും ചെയ്തുവോ
\end{malayalam}}
\flushright{\begin{Arabic}
\quranayah[92][7]
\end{Arabic}}
\flushleft{\begin{malayalam}
അവന്നു നാം ഏറ്റവും എളുപ്പമായതിലേക്ക് സൌകര്യപ്പെടുത്തി കൊടുക്കുന്നതാണ്‌.
\end{malayalam}}
\flushright{\begin{Arabic}
\quranayah[92][8]
\end{Arabic}}
\flushleft{\begin{malayalam}
എന്നാല്‍ ആര്‍ പിശുക്കു കാണിക്കുകയും, സ്വയം പര്യാപ്തത നടിക്കുകയും,
\end{malayalam}}
\flushright{\begin{Arabic}
\quranayah[92][9]
\end{Arabic}}
\flushleft{\begin{malayalam}
ഏറ്റവും ഉത്തമമായതിനെ നിഷേധിച്ചു തള്ളുകയും ചെയ്തുവോ
\end{malayalam}}
\flushright{\begin{Arabic}
\quranayah[92][10]
\end{Arabic}}
\flushleft{\begin{malayalam}
അവന്നു നാം ഏറ്റവും ഞെരുക്കമുള്ളതിലേക്ക് സൌകര്യമൊരുക്കികൊടുക്കുന്നതാണ്‌.
\end{malayalam}}
\flushright{\begin{Arabic}
\quranayah[92][11]
\end{Arabic}}
\flushleft{\begin{malayalam}
അവന്‍ നാശത്തില്‍ പതിക്കുമ്പോള്‍ അവന്‍റെ ധനം അവന്ന് പ്രയോജനപ്പെടുന്നതല്ല.
\end{malayalam}}
\flushright{\begin{Arabic}
\quranayah[92][12]
\end{Arabic}}
\flushleft{\begin{malayalam}
തീര്‍ച്ചയായും മാര്‍ഗദര്‍ശനം നമ്മുടെ ബാധ്യതയാകുന്നു.
\end{malayalam}}
\flushright{\begin{Arabic}
\quranayah[92][13]
\end{Arabic}}
\flushleft{\begin{malayalam}
തീര്‍ച്ചയായും നമുക്കുള്ളതാകുന്നു പരലോകവും ഇഹലോകവും.
\end{malayalam}}
\flushright{\begin{Arabic}
\quranayah[92][14]
\end{Arabic}}
\flushleft{\begin{malayalam}
അതിനാല്‍ ആളിക്കത്തിക്കൊണ്ടിരിക്കുന്ന അഗ്നിയെപ്പറ്റി ഞാന്‍ നിങ്ങള്‍ക്ക് താക്കീത് നല്‍കിയിരിക്കുന്നു.
\end{malayalam}}
\flushright{\begin{Arabic}
\quranayah[92][15]
\end{Arabic}}
\flushleft{\begin{malayalam}
ഏറ്റവും ദുഷ്ടനായ വ്യക്തിയല്ലാതെ അതില്‍ പ്രവേശിക്കുകയില്ല.
\end{malayalam}}
\flushright{\begin{Arabic}
\quranayah[92][16]
\end{Arabic}}
\flushleft{\begin{malayalam}
നിഷേധിച്ചു തള്ളുകയും, പിന്തിരിഞ്ഞു കളയുകയും (വ്യക്തി)
\end{malayalam}}
\flushright{\begin{Arabic}
\quranayah[92][17]
\end{Arabic}}
\flushleft{\begin{malayalam}
ഏറ്റവും സൂക്ഷ്മതയുള്ള വ്യക്തി അതില്‍ നിന്ന് അകറ്റി നിര്‍ത്തപ്പെടുന്നതാണ്‌.
\end{malayalam}}
\flushright{\begin{Arabic}
\quranayah[92][18]
\end{Arabic}}
\flushleft{\begin{malayalam}
പരിശുദ്ധിനേടുവാനായി തന്‍റെ ധനം നല്‍കുന്ന (വ്യക്തി)
\end{malayalam}}
\flushright{\begin{Arabic}
\quranayah[92][19]
\end{Arabic}}
\flushleft{\begin{malayalam}
പ്രത്യുപകാരം നല്‍കപ്പെടേണ്ടതായ യാതൊരു അനുഗ്രഹവും അവന്‍റെ പക്കല്‍ ഒരാള്‍ക്കുമില്ല.
\end{malayalam}}
\flushright{\begin{Arabic}
\quranayah[92][20]
\end{Arabic}}
\flushleft{\begin{malayalam}
തന്‍റെ അത്യുന്നതനായ രക്ഷിതാവിന്‍റെ പ്രീതി തേടുക എന്നതല്ലാതെ.
\end{malayalam}}
\flushright{\begin{Arabic}
\quranayah[92][21]
\end{Arabic}}
\flushleft{\begin{malayalam}
വഴിയെ അവന്‍ തൃപ്തിപ്പെടുന്നതാണ്‌.
\end{malayalam}}
\chapter{\textmalayalam{ളുഹാ ( പൂര്‍വ്വാഹ്നം )}}
\begin{Arabic}
\Huge{\centerline{\basmalah}}\end{Arabic}
\flushright{\begin{Arabic}
\quranayah[93][1]
\end{Arabic}}
\flushleft{\begin{malayalam}
പൂര്‍വ്വാഹ്നം തന്നെയാണ സത്യം;
\end{malayalam}}
\flushright{\begin{Arabic}
\quranayah[93][2]
\end{Arabic}}
\flushleft{\begin{malayalam}
രാത്രി തന്നെയാണ സത്യം; അത് ശാന്തമാവുമ്പോള്‍
\end{malayalam}}
\flushright{\begin{Arabic}
\quranayah[93][3]
\end{Arabic}}
\flushleft{\begin{malayalam}
(നബിയേ,) നിന്‍റെ രക്ഷിതാവ് നിന്നെ കൈവിട്ടിട്ടില്ല. വെറുത്തിട്ടുമില്ല.
\end{malayalam}}
\flushright{\begin{Arabic}
\quranayah[93][4]
\end{Arabic}}
\flushleft{\begin{malayalam}
തീര്‍ച്ചയായും പരലോകമാണ് നിനക്ക് ഇഹലോകത്തെക്കാള്‍ ഉത്തമമായിട്ടുള്ളത്‌.
\end{malayalam}}
\flushright{\begin{Arabic}
\quranayah[93][5]
\end{Arabic}}
\flushleft{\begin{malayalam}
വഴിയെ നിനക്ക് നിന്‍റെ രക്ഷിതാവ് (അനുഗ്രഹങ്ങള്‍) നല്‍കുന്നതും അപ്പോള്‍ നീ തൃപ്തിപ്പെടുന്നതുമാണ.്‌
\end{malayalam}}
\flushright{\begin{Arabic}
\quranayah[93][6]
\end{Arabic}}
\flushleft{\begin{malayalam}
നിന്നെ അവന്‍ ഒരു അനാഥയായി കണെ്ടത്തുകയും , എന്നിട്ട് (നിനക്ക്‌) ആശ്രയം നല്‍കുകയും ചെയ്തില്ലേ?
\end{malayalam}}
\flushright{\begin{Arabic}
\quranayah[93][7]
\end{Arabic}}
\flushleft{\begin{malayalam}
നിന്നെ അവന്‍ വഴി അറിയാത്തവനായി കണെ്ടത്തുകയും എന്നിട്ട് (നിനക്ക്‌) മാര്‍ഗദര്‍ശനം നല്‍കുകയും ചെയ്തിരിക്കുന്നു.
\end{malayalam}}
\flushright{\begin{Arabic}
\quranayah[93][8]
\end{Arabic}}
\flushleft{\begin{malayalam}
നിന്നെ അവന്‍ ദരിദ്രനായി കണെ്ടത്തുകയും എന്നിട്ട് അവന്‍ ഐശ്വര്യം നല്‍കുകയും ചെയ്തിരിക്കുന്നു.
\end{malayalam}}
\flushright{\begin{Arabic}
\quranayah[93][9]
\end{Arabic}}
\flushleft{\begin{malayalam}
എന്നിരിക്കെ അനാഥയെ നീ അടിച്ചമര്‍ത്തരുത്‌
\end{malayalam}}
\flushright{\begin{Arabic}
\quranayah[93][10]
\end{Arabic}}
\flushleft{\begin{malayalam}
ചോദിച്ച് വരുന്നവനെ നീ വിരട്ടി വിടുകയും ചെയ്യരുത്‌.
\end{malayalam}}
\flushright{\begin{Arabic}
\quranayah[93][11]
\end{Arabic}}
\flushleft{\begin{malayalam}
നിന്‍റെ രക്ഷിതാവിന്‍റെ അനുഗ്രഹത്തെ സംബന്ധിച്ച് നീ സംസാരിക്കുക.
\end{malayalam}}
\chapter{\textmalayalam{ശര്‍ഹ് ( വിശാലമാക്കല്‍ )}}
\begin{Arabic}
\Huge{\centerline{\basmalah}}\end{Arabic}
\flushright{\begin{Arabic}
\quranayah[94][1]
\end{Arabic}}
\flushleft{\begin{malayalam}
നിനക്ക് നിന്‍റെ ഹൃദയം നാം വിശാലതയുള്ളതാക്കി തന്നില്ലേ?
\end{malayalam}}
\flushright{\begin{Arabic}
\quranayah[94][2]
\end{Arabic}}
\flushleft{\begin{malayalam}
നിന്നില്‍ നിന്ന് നിന്‍റെ ആ ഭാരം നാം ഇറക്കിവെക്കുകയും ചെയ്തു.
\end{malayalam}}
\flushright{\begin{Arabic}
\quranayah[94][3]
\end{Arabic}}
\flushleft{\begin{malayalam}
നിന്‍റെ മുതുകിനെ ഞെരിച്ചു കളഞ്ഞതായ (ഭാരം)
\end{malayalam}}
\flushright{\begin{Arabic}
\quranayah[94][4]
\end{Arabic}}
\flushleft{\begin{malayalam}
നിനക്ക് നിന്‍റെ കീര്‍ത്തി നാം ഉയര്‍ത്തിത്തരികയും ചെയ്തിരിക്കുന്നു.
\end{malayalam}}
\flushright{\begin{Arabic}
\quranayah[94][5]
\end{Arabic}}
\flushleft{\begin{malayalam}
എന്നാല്‍ തീര്‍ച്ചയായും ഞെരുക്കത്തിന്‍റെ കൂടെ ഒരു എളുപ്പമുണ്ടായിരിക്കും.
\end{malayalam}}
\flushright{\begin{Arabic}
\quranayah[94][6]
\end{Arabic}}
\flushleft{\begin{malayalam}
തീര്‍ച്ചയായും ഞെരുക്കത്തിന്‍റെ കൂടെ ഒരു എളുപ്പമുണ്ടായിരിക്കും.
\end{malayalam}}
\flushright{\begin{Arabic}
\quranayah[94][7]
\end{Arabic}}
\flushleft{\begin{malayalam}
ആകയാല്‍ നിനക്ക് ഒഴിവ് കിട്ടിയാല്‍ നീ അദ്ധ്വാനിക്കുക.
\end{malayalam}}
\flushright{\begin{Arabic}
\quranayah[94][8]
\end{Arabic}}
\flushleft{\begin{malayalam}
നിന്‍റെ രക്ഷിതാവിലേക്ക് തന്നെ നിന്‍റെ ആഗ്രഹം സമര്‍പ്പിക്കുകയും ചെയ്യുക.
\end{malayalam}}
\chapter{\textmalayalam{തീന്‍ ( അത്തി )}}
\begin{Arabic}
\Huge{\centerline{\basmalah}}\end{Arabic}
\flushright{\begin{Arabic}
\quranayah[95][1]
\end{Arabic}}
\flushleft{\begin{malayalam}
അത്തിയും, ഒലീവും,
\end{malayalam}}
\flushright{\begin{Arabic}
\quranayah[95][2]
\end{Arabic}}
\flushleft{\begin{malayalam}
സീനാപര്‍വ്വതവും,
\end{malayalam}}
\flushright{\begin{Arabic}
\quranayah[95][3]
\end{Arabic}}
\flushleft{\begin{malayalam}
നിര്‍ഭയത്വമുള്ള ഈ രാജ്യവും തന്നെയാണ സത്യം.
\end{malayalam}}
\flushright{\begin{Arabic}
\quranayah[95][4]
\end{Arabic}}
\flushleft{\begin{malayalam}
തീര്‍ച്ചയായും മനുഷ്യനെ നാം ഏറ്റവും നല്ല ഘടനയോടു കൂടി സൃഷ്ടിച്ചിരിക്കുന്നു.
\end{malayalam}}
\flushright{\begin{Arabic}
\quranayah[95][5]
\end{Arabic}}
\flushleft{\begin{malayalam}
പിന്നീട് അവനെ നാം അധമരില്‍ അധമനാക്കിത്തീര്‍ത്തു.
\end{malayalam}}
\flushright{\begin{Arabic}
\quranayah[95][6]
\end{Arabic}}
\flushleft{\begin{malayalam}
വിശ്വസിക്കുകയും സല്‍കര്‍മ്മങ്ങള്‍ പ്രവര്‍ത്തിക്കുകയും ചെയ്തവരൊഴികെ. എന്നാല്‍ അവര്‍ക്കാകട്ടെ മുറിഞ്ഞ് പോകാത്ത പ്രതിഫലമുണ്ടായിരിക്കും.
\end{malayalam}}
\flushright{\begin{Arabic}
\quranayah[95][7]
\end{Arabic}}
\flushleft{\begin{malayalam}
എന്നിരിക്കെ ഇതിന് ശേഷം പരലോകത്തെ പ്രതിഫല നടപടിയുടെ കാര്യത്തില്‍ (നബിയേ,) നിന്നെ നിഷേധിച്ചു തള്ളാന്‍ എന്ത് ന്യായമാണുള്ളത്‌?
\end{malayalam}}
\flushright{\begin{Arabic}
\quranayah[95][8]
\end{Arabic}}
\flushleft{\begin{malayalam}
അല്ലാഹു വിധികര്‍ത്താക്കളില്‍ വെച്ചു ഏറ്റവും വലിയ വിധികര്‍ത്താവല്ലയോ?
\end{malayalam}}
\chapter{\textmalayalam{അലഖ് ( ഭ്രൂണം )}}
\begin{Arabic}
\Huge{\centerline{\basmalah}}\end{Arabic}
\flushright{\begin{Arabic}
\quranayah[96][1]
\end{Arabic}}
\flushleft{\begin{malayalam}
സൃഷ്ടിച്ചവനായ നിന്‍റെ രക്ഷിതാവിന്‍റെ നാമത്തില്‍ വായിക്കുക.
\end{malayalam}}
\flushright{\begin{Arabic}
\quranayah[96][2]
\end{Arabic}}
\flushleft{\begin{malayalam}
മനുഷ്യനെ അവന്‍ ഭ്രൂണത്തില്‍ നിന്ന് സൃഷ്ടിച്ചിരിക്കുന്നു.
\end{malayalam}}
\flushright{\begin{Arabic}
\quranayah[96][3]
\end{Arabic}}
\flushleft{\begin{malayalam}
നീ വായിക്കുക നിന്‍റെ രക്ഷിതാവ് ഏറ്റവും വലിയ ഔദാര്യവാനാകുന്നു.
\end{malayalam}}
\flushright{\begin{Arabic}
\quranayah[96][4]
\end{Arabic}}
\flushleft{\begin{malayalam}
പേന കൊണ്ട് പഠിപ്പിച്ചവന്‍
\end{malayalam}}
\flushright{\begin{Arabic}
\quranayah[96][5]
\end{Arabic}}
\flushleft{\begin{malayalam}
മനുഷ്യന് അറിയാത്തത് അവന്‍ പഠിപ്പിച്ചിരിക്കുന്നു.
\end{malayalam}}
\flushright{\begin{Arabic}
\quranayah[96][6]
\end{Arabic}}
\flushleft{\begin{malayalam}
നിസ്സംശയം മനുഷ്യന്‍ ധിക്കാരിയായി തീരുന്നു.
\end{malayalam}}
\flushright{\begin{Arabic}
\quranayah[96][7]
\end{Arabic}}
\flushleft{\begin{malayalam}
തന്നെ സ്വയം പര്യാപ്തനായി കണ്ടതിനാല്‍
\end{malayalam}}
\flushright{\begin{Arabic}
\quranayah[96][8]
\end{Arabic}}
\flushleft{\begin{malayalam}
തീര്‍ച്ചയായും നിന്‍റെ രക്ഷിതാവിലേക്കാണ് മടക്കം.
\end{malayalam}}
\flushright{\begin{Arabic}
\quranayah[96][9]
\end{Arabic}}
\flushleft{\begin{malayalam}
വിലക്കുന്നവനെ നീ കണ്ടുവോ?
\end{malayalam}}
\flushright{\begin{Arabic}
\quranayah[96][10]
\end{Arabic}}
\flushleft{\begin{malayalam}
ഒരു അടിയനെ, അവന്‍ നമസ്കരിച്ചാല്‍.
\end{malayalam}}
\flushright{\begin{Arabic}
\quranayah[96][11]
\end{Arabic}}
\flushleft{\begin{malayalam}
അദ്ദേഹം സന്‍മാര്‍ഗത്തിലാണെങ്കില്‍ , (ആ വിലക്കുന്നവന്‍റെ അവസ്ഥ എന്തായിരിക്കുമെന്ന്‌) നീ കണ്ടുവോ?
\end{malayalam}}
\flushright{\begin{Arabic}
\quranayah[96][12]
\end{Arabic}}
\flushleft{\begin{malayalam}
അഥവാ അദ്ദേഹം സൂക്ഷ്മത കൈ കൊള്ളാന്‍ കല്‍പിച്ചിരിക്കുകയാണെങ്കില്‍
\end{malayalam}}
\flushright{\begin{Arabic}
\quranayah[96][13]
\end{Arabic}}
\flushleft{\begin{malayalam}
അവന്‍ (ആ വിലക്കുന്നവന്‍) നിഷേധിച്ചു തള്ളുകയും തിരിഞ്ഞുകളയുകയും ചെയ്തിരിക്കയാണെങ്കില്‍ (അവന്‍റെ അവസ്ഥ എന്തായിരിക്കുമെന്ന്‌) നീ കണ്ടുവോ?
\end{malayalam}}
\flushright{\begin{Arabic}
\quranayah[96][14]
\end{Arabic}}
\flushleft{\begin{malayalam}
അവന്‍ മനസ്സിലാക്കിയിട്ടില്ലേ, അല്ലാഹു കാണുന്നുണെ്ടന്ന്‌?
\end{malayalam}}
\flushright{\begin{Arabic}
\quranayah[96][15]
\end{Arabic}}
\flushleft{\begin{malayalam}
നിസ്സംശയം. അവന്‍ വിരമിച്ചിട്ടില്ലെങ്കല്‍ നാം ആ കുടുമ പിടിച്ചു വലിക്കുക തന്നെ ചെയ്യും .
\end{malayalam}}
\flushright{\begin{Arabic}
\quranayah[96][16]
\end{Arabic}}
\flushleft{\begin{malayalam}
കള്ളം പറയുന്ന , പാപം ചെയ്യുന്ന കുടുമ.
\end{malayalam}}
\flushright{\begin{Arabic}
\quranayah[96][17]
\end{Arabic}}
\flushleft{\begin{malayalam}
എന്നിട്ട് അവന്‍ അവന്‍റെ സഭയിലുള്ളവരെ വിളിച്ചുകൊള്ളട്ടെ.
\end{malayalam}}
\flushright{\begin{Arabic}
\quranayah[96][18]
\end{Arabic}}
\flushleft{\begin{malayalam}
നാം സബാനിയത്തിനെ (ശിക്ഷ നടപ്പാക്കുന്ന മലക്കുകളെ) വിളിച്ചുകൊള്ളാം.
\end{malayalam}}
\flushright{\begin{Arabic}
\quranayah[96][19]
\end{Arabic}}
\flushleft{\begin{malayalam}
നിസ്സംശയം; നീ അവനെ അനുസരിച്ചു പോകരുത് , നീ പ്രണമിക്കുകയും സാമീപ്യം നേടുകയും ചെയ്യുക.
\end{malayalam}}
\chapter{\textmalayalam{ഖദ്ര്‍ ( നിര്‍ണയം )}}
\begin{Arabic}
\Huge{\centerline{\basmalah}}\end{Arabic}
\flushright{\begin{Arabic}
\quranayah[97][1]
\end{Arabic}}
\flushleft{\begin{malayalam}
തീര്‍ച്ചയായും നാം ഇതിനെ (ഖുര്‍ആനിനെ) നിര്‍ണയത്തിന്‍റെ രാത്രിയില്‍ അവതരിപ്പിച്ചിരിക്കുന്നു.
\end{malayalam}}
\flushright{\begin{Arabic}
\quranayah[97][2]
\end{Arabic}}
\flushleft{\begin{malayalam}
നിര്‍ണയത്തിന്‍റെ രാത്രി എന്നാല്‍ എന്താണെന്ന് നിനക്കറിയാമോ?
\end{malayalam}}
\flushright{\begin{Arabic}
\quranayah[97][3]
\end{Arabic}}
\flushleft{\begin{malayalam}
നിര്‍ണയത്തിന്‍റെ രാത്രി ആയിരം മാസത്തെക്കാള്‍ ഉത്തമമാകുന്നു.
\end{malayalam}}
\flushright{\begin{Arabic}
\quranayah[97][4]
\end{Arabic}}
\flushleft{\begin{malayalam}
മലക്കുകളും ആത്മാവും അവരുടെ രക്ഷിതാവിന്‍റെ എല്ലാകാര്യത്തെ സംബന്ധിച്ചുമുള്ള ഉത്തരവുമായി ആ രാത്രിയില്‍ ഇറങ്ങി വരുന്നു.
\end{malayalam}}
\flushright{\begin{Arabic}
\quranayah[97][5]
\end{Arabic}}
\flushleft{\begin{malayalam}
പ്രഭാതോദയം വരെ അത് സമാധാനമത്രെ.
\end{malayalam}}
\chapter{\textmalayalam{ബയ്യിന ( വ്യക്തമായ തെളിവ് )}}
\begin{Arabic}
\Huge{\centerline{\basmalah}}\end{Arabic}
\flushright{\begin{Arabic}
\quranayah[98][1]
\end{Arabic}}
\flushleft{\begin{malayalam}
വേദക്കാരിലും ബഹുദൈവവിശ്വാസികളിലും പെട്ട സത്യനിഷേധികള്‍ വ്യക്തമായ തെളിവ് തങ്ങള്‍ക്ക് കിട്ടുന്നത് വരെ (അവിശ്വാസത്തില്‍ നിന്ന്‌) വേറിട്ട് പോരുന്നവരായിട്ടില്ല.
\end{malayalam}}
\flushright{\begin{Arabic}
\quranayah[98][2]
\end{Arabic}}
\flushleft{\begin{malayalam}
അതായത് പരിശുദ്ധി നല്‍കപ്പെട്ട ഏടുകള്‍ ഓതികേള്‍പിക്കുന്ന, അല്ലാഹുവിങ്കല്‍ നിന്നുള്ള ഒരു ദൂതന്‍ (വരുന്നതു വരെ)
\end{malayalam}}
\flushright{\begin{Arabic}
\quranayah[98][3]
\end{Arabic}}
\flushleft{\begin{malayalam}
അവയില്‍ (ഏടുകളില്‍) വക്രതയില്ലാത്ത രേഖകളാണുള്ളത്‌.
\end{malayalam}}
\flushright{\begin{Arabic}
\quranayah[98][4]
\end{Arabic}}
\flushleft{\begin{malayalam}
വേദം നല്‍കപ്പെട്ടവര്‍ അവര്‍ക്ക് വ്യക്തമായ തെളിവ് വന്നുകിട്ടിയതിന് ശേഷമല്ലാതെ ഭിന്നിക്കുകയുണ്ടായിട്ടില്ല.
\end{malayalam}}
\flushright{\begin{Arabic}
\quranayah[98][5]
\end{Arabic}}
\flushleft{\begin{malayalam}
കീഴ്‌വണക്കം അല്ലാഹുവിന് മാത്രം ആക്കി കൊണ്ട് ഋജുമനസ്കരായ നിലയില്‍ അവനെ ആരാധിക്കുവാനും, നമസ്കാരം നിലനിര്‍ത്തുവാനും സകാത്ത് നല്‍കുവാനും അല്ലാതെ അവരോട് കല്‍പിക്കപ്പെട്ടിട്ടില്ല. അതത്രെ വക്രതയില്ലാത്ത മതം
\end{malayalam}}
\flushright{\begin{Arabic}
\quranayah[98][6]
\end{Arabic}}
\flushleft{\begin{malayalam}
തീര്‍ച്ചയായും വേദക്കാരിലും ബഹുദൈവവിശ്വാസികളിലുംപെട്ട സത്യനിഷേധികള്‍ നരകാഗ്നിയിലാകുന്നു. അവരതില്‍ നിത്യവാസികളായിരിക്കും . അക്കൂട്ടര്‍ തന്നെയാകുന്നു സൃഷ്ടികളില്‍ മോശപ്പെട്ടവര്‍.
\end{malayalam}}
\flushright{\begin{Arabic}
\quranayah[98][7]
\end{Arabic}}
\flushleft{\begin{malayalam}
തീര്‍ച്ചയായും വിശ്വസിക്കുകയും സല്‍കര്‍മ്മങ്ങള്‍ പ്രവര്‍ത്തിക്കുകയും ചെയ്തവരാരോ അവര്‍ തന്നെയാകുന്നു സൃഷ്ടികളില്‍ ഉത്തമര്‍.
\end{malayalam}}
\flushright{\begin{Arabic}
\quranayah[98][8]
\end{Arabic}}
\flushleft{\begin{malayalam}
അവര്‍ക്ക് അവരുടെ രക്ഷിതാവിങ്കലുള്ള പ്രതിഫലം താഴ്ഭാഗത്തു കൂടി അരുവികള്‍ ഒഴുകുന്ന, സ്ഥിരവാസത്തിനുള്ള സ്വര്‍ഗത്തോപ്പുകളാകുന്നു. അവരതില്‍ നിത്യവാസികളായിരിക്കും; എന്നെന്നേക്കുമായിട്ട്‌. അല്ലാഹു അവരെ പറ്റി തൃപ്തിപ്പെട്ടിരിക്കുന്നു. അവര്‍ അവനെ പറ്റിയും തൃപ്തിപ്പെട്ടിരിക്കുന്നു. ഏതൊരുവന്‍ തന്‍റെ രക്ഷിതാവിനെ ഭയപ്പെട്ടുവോ അവന്നുള്ളതാകുന്നു അത്‌.
\end{malayalam}}
\chapter{\textmalayalam{സല്‍സല ( പ്രകമ്പനം )}}
\begin{Arabic}
\Huge{\centerline{\basmalah}}\end{Arabic}
\flushright{\begin{Arabic}
\quranayah[99][1]
\end{Arabic}}
\flushleft{\begin{malayalam}
ഭൂമി പ്രകമ്പനം കൊള്ളിക്കപ്പെട്ടാല്‍ - അതിന്‍റെ ഭയങ്കരമായ ആ പ്രകമ്പനം .
\end{malayalam}}
\flushright{\begin{Arabic}
\quranayah[99][2]
\end{Arabic}}
\flushleft{\begin{malayalam}
ഭൂമി അതിന്‍റെ ഭാരങ്ങള്‍ പുറം തള്ളുകയും,
\end{malayalam}}
\flushright{\begin{Arabic}
\quranayah[99][3]
\end{Arabic}}
\flushleft{\begin{malayalam}
അതിന് എന്തുപറ്റി എന്ന് മനുഷ്യന്‍ പറയുകയും ചെയ്താല്‍.
\end{malayalam}}
\flushright{\begin{Arabic}
\quranayah[99][4]
\end{Arabic}}
\flushleft{\begin{malayalam}
അന്നേ ദിവസം അത് (ഭൂമി) അതിന്‍റെ വര്‍ത്തമാനങ്ങള്‍ പറഞ്ഞറിയിക്കുന്നതാണ്‌.
\end{malayalam}}
\flushright{\begin{Arabic}
\quranayah[99][5]
\end{Arabic}}
\flushleft{\begin{malayalam}
നിന്‍റെ രക്ഷിതാവ് അതിന് ബോധനം നല്‍കിയത് നിമിത്തം.
\end{malayalam}}
\flushright{\begin{Arabic}
\quranayah[99][6]
\end{Arabic}}
\flushleft{\begin{malayalam}
അന്നേ ദിവസം മനുഷ്യര്‍ പല സംഘങ്ങളായി പുറപ്പെടുന്നതാണ്‌. അവര്‍ക്ക് അവരുടെ കര്‍മ്മങ്ങള്‍ കാണിക്കപ്പെടേണ്ടതിനായിട്ട്‌.
\end{malayalam}}
\flushright{\begin{Arabic}
\quranayah[99][7]
\end{Arabic}}
\flushleft{\begin{malayalam}
അപ്പോള്‍ ആര്‍ ഒരു അണുവിന്‍റെ തൂക്കം നന്‍മചെയ്തിരുന്നുവോ അവനത് കാണും.
\end{malayalam}}
\flushright{\begin{Arabic}
\quranayah[99][8]
\end{Arabic}}
\flushleft{\begin{malayalam}
ആര്‍ ഒരു അണുവിന്‍റെ തൂക്കം തിന്‍മ ചെയ്തിരുന്നുവോ അവന്‍ അതും കാണും.
\end{malayalam}}
\chapter{\textmalayalam{ആദിയാത് ( ഓടുന്നവ )}}
\begin{Arabic}
\Huge{\centerline{\basmalah}}\end{Arabic}
\flushright{\begin{Arabic}
\quranayah[100][1]
\end{Arabic}}
\flushleft{\begin{malayalam}
കിതച്ചു കൊണ്ട് ഓടുന്നവയും,
\end{malayalam}}
\flushright{\begin{Arabic}
\quranayah[100][2]
\end{Arabic}}
\flushleft{\begin{malayalam}
അങ്ങനെ (കുളമ്പ് കല്ലില്‍) ഉരസി തീപ്പൊരി പറപ്പിക്കുന്നവയും,
\end{malayalam}}
\flushright{\begin{Arabic}
\quranayah[100][3]
\end{Arabic}}
\flushleft{\begin{malayalam}
എന്നിട്ട് പ്രഭാതത്തില്‍ ആക്രമണം നടത്തുന്നവയും ,
\end{malayalam}}
\flushright{\begin{Arabic}
\quranayah[100][4]
\end{Arabic}}
\flushleft{\begin{malayalam}
അന്നേരത്ത് പൊടിപടലം ഇളക്കിവിട്ടവയും
\end{malayalam}}
\flushright{\begin{Arabic}
\quranayah[100][5]
\end{Arabic}}
\flushleft{\begin{malayalam}
അതിലൂടെ (ശത്രു) സംഘത്തിന്‍റെ നടുവില്‍ പ്രവേശിച്ചവയും (കുതിരകള്‍) തന്നെ സത്യം.
\end{malayalam}}
\flushright{\begin{Arabic}
\quranayah[100][6]
\end{Arabic}}
\flushleft{\begin{malayalam}
തീര്‍ച്ചയായും മനുഷ്യന്‍ തന്‍റെ രക്ഷിതാവിനോട് നന്ദികെട്ടവന്‍ തന്നെ.
\end{malayalam}}
\flushright{\begin{Arabic}
\quranayah[100][7]
\end{Arabic}}
\flushleft{\begin{malayalam}
തീര്‍ച്ചയായും അവന്‍ അതിന്ന് സാക്ഷ്യം വഹിക്കുന്നവനുമാകുന്നു.
\end{malayalam}}
\flushright{\begin{Arabic}
\quranayah[100][8]
\end{Arabic}}
\flushleft{\begin{malayalam}
തീര്‍ച്ചയായും അവന്‍ ധനത്തോടുള്ള സ്നേഹം കഠിനമായവനാകുന്നു.
\end{malayalam}}
\flushright{\begin{Arabic}
\quranayah[100][9]
\end{Arabic}}
\flushleft{\begin{malayalam}
എന്നാല്‍ അവന്‍ അറിയുന്നില്ലേ? ഖബ്‌റുകളിലുള്ളത് ഇളക്കിമറിച്ച് പുറത്ത് കൊണ്ട് വരപ്പെടുകയും ,
\end{malayalam}}
\flushright{\begin{Arabic}
\quranayah[100][10]
\end{Arabic}}
\flushleft{\begin{malayalam}
ഹൃദയങ്ങളിലുള്ളത് വെളിക്ക് കൊണ്ടു വരപ്പെടുകയും ചെയ്താല്‍ ,
\end{malayalam}}
\flushright{\begin{Arabic}
\quranayah[100][11]
\end{Arabic}}
\flushleft{\begin{malayalam}
തീര്‍ച്ചയായും അവരുടെ രക്ഷിതാവ് അന്നേ ദിവസം അവരെ പറ്റി സൂക്ഷ്മമായി അറിയുന്നവന്‍ തന്നെയാകുന്നു.
\end{malayalam}}
\chapter{\textmalayalam{അല്‍ ഖാരിഅ ( ഭയങ്കര സംഭവം )}}
\begin{Arabic}
\Huge{\centerline{\basmalah}}\end{Arabic}
\flushright{\begin{Arabic}
\quranayah[101][1]
\end{Arabic}}
\flushleft{\begin{malayalam}
ഭയങ്കരമായ ആ സംഭവം.
\end{malayalam}}
\flushright{\begin{Arabic}
\quranayah[101][2]
\end{Arabic}}
\flushleft{\begin{malayalam}
ഭയങ്കരമായ സംഭവം എന്നാല്‍ എന്താകുന്നു?
\end{malayalam}}
\flushright{\begin{Arabic}
\quranayah[101][3]
\end{Arabic}}
\flushleft{\begin{malayalam}
ഭയങ്കരമായ സംഭവമെന്നാല്‍ എന്താണെന്ന് നിനക്കറിയുമോ?
\end{malayalam}}
\flushright{\begin{Arabic}
\quranayah[101][4]
\end{Arabic}}
\flushleft{\begin{malayalam}
മനുഷ്യന്‍മാര്‍ ചിന്നിച്ചിതറിയ പാറ്റയെപ്പോലെ ആകുന്ന ദിവസം!
\end{malayalam}}
\flushright{\begin{Arabic}
\quranayah[101][5]
\end{Arabic}}
\flushleft{\begin{malayalam}
പര്‍വ്വതങ്ങള്‍ കടഞ്ഞ ആട്ടിന്‍ രോമം പോലെയും
\end{malayalam}}
\flushright{\begin{Arabic}
\quranayah[101][6]
\end{Arabic}}
\flushleft{\begin{malayalam}
അപ്പോള്‍ ഏതൊരാളുടെ തുലാസുകള്‍ ഘനം തൂങ്ങിയോ
\end{malayalam}}
\flushright{\begin{Arabic}
\quranayah[101][7]
\end{Arabic}}
\flushleft{\begin{malayalam}
അവന്‍ സംതൃപ്തമായ ജീവിതത്തിലായിരിക്കും.
\end{malayalam}}
\flushright{\begin{Arabic}
\quranayah[101][8]
\end{Arabic}}
\flushleft{\begin{malayalam}
എന്നാല്‍ ഏതൊരാളുടെ തുലാസുകള്‍ തൂക്കം കുറഞ്ഞതായോ
\end{malayalam}}
\flushright{\begin{Arabic}
\quranayah[101][9]
\end{Arabic}}
\flushleft{\begin{malayalam}
അവന്‍റെ സങ്കേതം ഹാവിയഃ ആയിരിക്കും.
\end{malayalam}}
\flushright{\begin{Arabic}
\quranayah[101][10]
\end{Arabic}}
\flushleft{\begin{malayalam}
ഹാവിയഃ എന്നാല്‍ എന്താണെന്ന് നിനക്കറിയുമോ?
\end{malayalam}}
\flushright{\begin{Arabic}
\quranayah[101][11]
\end{Arabic}}
\flushleft{\begin{malayalam}
ചൂടേറിയ നരകാഗ്നിയത്രെ അത്‌.
\end{malayalam}}
\chapter{\textmalayalam{തകാഥുര്‍ (പെരുമ നടിക്കല്‍ )}}
\begin{Arabic}
\Huge{\centerline{\basmalah}}\end{Arabic}
\flushright{\begin{Arabic}
\quranayah[102][1]
\end{Arabic}}
\flushleft{\begin{malayalam}
പരസ്പരം പെരുമനടിക്കുക എന്ന കാര്യം നിങ്ങളെ അശ്രദ്ധയിലാക്കിയിരിക്കുന്നു.
\end{malayalam}}
\flushright{\begin{Arabic}
\quranayah[102][2]
\end{Arabic}}
\flushleft{\begin{malayalam}
നിങ്ങള്‍ ശവകുടീരങ്ങള്‍ സന്ദര്‍ശിക്കുന്നത് വരേക്കും.
\end{malayalam}}
\flushright{\begin{Arabic}
\quranayah[102][3]
\end{Arabic}}
\flushleft{\begin{malayalam}
നിസ്സംശയം, നിങ്ങള്‍ വഴിയെ അറിഞ്ഞ് കൊള്ളും.
\end{malayalam}}
\flushright{\begin{Arabic}
\quranayah[102][4]
\end{Arabic}}
\flushleft{\begin{malayalam}
പിന്നെയും നിസ്സംശയം നിങ്ങള്‍ വഴിയെ അറിഞ്ഞ് കൊള്ളും.
\end{malayalam}}
\flushright{\begin{Arabic}
\quranayah[102][5]
\end{Arabic}}
\flushleft{\begin{malayalam}
നിസ്സംശയം, നിങ്ങള്‍ ദൃഢമായ അറിവ് അറിയുമായിരുന്നെങ്കില്‍
\end{malayalam}}
\flushright{\begin{Arabic}
\quranayah[102][6]
\end{Arabic}}
\flushleft{\begin{malayalam}
ജ്വലിക്കുന്ന നരകത്തെ നിങ്ങള്‍ കാണുക തന്നെ ചെയ്യും.
\end{malayalam}}
\flushright{\begin{Arabic}
\quranayah[102][7]
\end{Arabic}}
\flushleft{\begin{malayalam}
പിന്നെ തീര്‍ച്ചയായും നിങ്ങള്‍ അതിനെ ദൃഢമായും കണ്ണാല്‍ കാണുക തന്നെ ചെയ്യും.
\end{malayalam}}
\flushright{\begin{Arabic}
\quranayah[102][8]
\end{Arabic}}
\flushleft{\begin{malayalam}
പിന്നീട് ആ ദിവസത്തില്‍ സുഖാനുഭവങ്ങളെ പറ്റി തീര്‍ച്ചയായും നിങ്ങള്‍ ചോദ്യം ചെയ്യപ്പെടുകതന്നെ ചെയ്യും.
\end{malayalam}}
\chapter{\textmalayalam{അസ്വര്‍ ( കാലം )}}
\begin{Arabic}
\Huge{\centerline{\basmalah}}\end{Arabic}
\flushright{\begin{Arabic}
\quranayah[103][1]
\end{Arabic}}
\flushleft{\begin{malayalam}
കാലം തന്നെയാണ് സത്യം,
\end{malayalam}}
\flushright{\begin{Arabic}
\quranayah[103][2]
\end{Arabic}}
\flushleft{\begin{malayalam}
തീര്‍ച്ചയായും മനുഷ്യന്‍ നഷ്ടത്തില്‍ തന്നെയാകുന്നു;
\end{malayalam}}
\flushright{\begin{Arabic}
\quranayah[103][3]
\end{Arabic}}
\flushleft{\begin{malayalam}
വിശ്വസിക്കുകയും സല്‍കര്‍മ്മങ്ങള്‍ പ്രവര്‍ത്തിക്കുകയും, സത്യം കൈക്കൊള്ളാന്‍ അന്യോന്യം ഉപദേശിക്കുകയും ക്ഷമ കൈക്കൊള്ളാന്‍ അന്യോന്യം ഉപദേശിക്കുകയും ചെയ്തവരൊഴികെ.
\end{malayalam}}
\chapter{\textmalayalam{ഹുമസ (കുത്തിപ്പറയുന്നവര്‍ )}}
\begin{Arabic}
\Huge{\centerline{\basmalah}}\end{Arabic}
\flushright{\begin{Arabic}
\quranayah[104][1]
\end{Arabic}}
\flushleft{\begin{malayalam}
കുത്തുവാക്ക് പറയുന്നവനും അവഹേളിക്കുന്നവനുമായ ഏതൊരാള്‍ക്കും നാശം.
\end{malayalam}}
\flushright{\begin{Arabic}
\quranayah[104][2]
\end{Arabic}}
\flushleft{\begin{malayalam}
അതായത് ധനം ശേഖരിക്കുകയും അത് എണ്ണിനോക്കിക്കൊണ്ടിരിക്കുകയും ചെയ്യുന്നവന്‌.
\end{malayalam}}
\flushright{\begin{Arabic}
\quranayah[104][3]
\end{Arabic}}
\flushleft{\begin{malayalam}
അവന്‍റെ ധനം അവന് ശാശ്വത ജീവിതം നല്‍കിയിരിക്കുന്നു എന്ന് അവന്‍ വിചാരിക്കുന്നു.
\end{malayalam}}
\flushright{\begin{Arabic}
\quranayah[104][4]
\end{Arabic}}
\flushleft{\begin{malayalam}
നിസ്സംശയം, അവന്‍ ഹുത്വമയില്‍ എറിയപ്പെടുക തന്നെ ചെയ്യും.
\end{malayalam}}
\flushright{\begin{Arabic}
\quranayah[104][5]
\end{Arabic}}
\flushleft{\begin{malayalam}
ഹുത്വമ എന്നാല്‍ എന്താണെന്ന് നിനക്കറിയാമോ?
\end{malayalam}}
\flushright{\begin{Arabic}
\quranayah[104][6]
\end{Arabic}}
\flushleft{\begin{malayalam}
അത് അല്ലാഹുവിന്‍റെ ജ്വലിപ്പിക്കപ്പെട്ട അഗ്നിയാകുന്നു.
\end{malayalam}}
\flushright{\begin{Arabic}
\quranayah[104][7]
\end{Arabic}}
\flushleft{\begin{malayalam}
ഹൃദയങ്ങളിലേക്ക് കത്തിപ്പടരുന്നതായ
\end{malayalam}}
\flushright{\begin{Arabic}
\quranayah[104][8]
\end{Arabic}}
\flushleft{\begin{malayalam}
തീര്‍ച്ചയായും അത് അവരുടെ മേല്‍ അടച്ചുമൂടപ്പെടുന്നതായിരിക്കും.
\end{malayalam}}
\flushright{\begin{Arabic}
\quranayah[104][9]
\end{Arabic}}
\flushleft{\begin{malayalam}
നീട്ടിയുണ്ടാക്കപ്പെട്ട സ്തംഭങ്ങളിലായിക്കൊണ്ട്‌.
\end{malayalam}}
\chapter{\textmalayalam{ഫീല്‍ ( ആന )}}
\begin{Arabic}
\Huge{\centerline{\basmalah}}\end{Arabic}
\flushright{\begin{Arabic}
\quranayah[105][1]
\end{Arabic}}
\flushleft{\begin{malayalam}
ആനക്കാരെക്കൊണ്ട് നിന്‍റെ രക്ഷിതാവ് പ്രവര്‍ത്തിച്ചത് എങ്ങനെ എന്ന് നീ കണ്ടില്ലേ?
\end{malayalam}}
\flushright{\begin{Arabic}
\quranayah[105][2]
\end{Arabic}}
\flushleft{\begin{malayalam}
അവരുടെ തന്ത്രം അവന്‍ പിഴവിലാക്കിയില്ലേ?
\end{malayalam}}
\flushright{\begin{Arabic}
\quranayah[105][3]
\end{Arabic}}
\flushleft{\begin{malayalam}
കൂട്ടംകൂട്ടമായിക്കൊണ്ടുള്ള പക്ഷികളെ അവരുടെ നേര്‍ക്ക് അവന്‍ അയക്കുകയും ചെയ്തു.
\end{malayalam}}
\flushright{\begin{Arabic}
\quranayah[105][4]
\end{Arabic}}
\flushleft{\begin{malayalam}
ചുട്ടുപഴുപ്പിച്ച കളിമണ്‍കല്ലുകള്‍കൊണ്ട് അവരെ എറിയുന്നതായ.
\end{malayalam}}
\flushright{\begin{Arabic}
\quranayah[105][5]
\end{Arabic}}
\flushleft{\begin{malayalam}
അങ്ങനെ അവന്‍ അവരെ തിന്നൊടുക്കപ്പെട്ട വൈക്കോല്‍ തുരുമ്പുപോലെയാക്കി.
\end{malayalam}}
\chapter{\textmalayalam{ഖുറൈഷ്}}
\begin{Arabic}
\Huge{\centerline{\basmalah}}\end{Arabic}
\flushright{\begin{Arabic}
\quranayah[106][1]
\end{Arabic}}
\flushleft{\begin{malayalam}
ഖുറൈശ് ഗോത്രത്തെ കൂട്ടിയിണക്കിയതിനാല്‍.
\end{malayalam}}
\flushright{\begin{Arabic}
\quranayah[106][2]
\end{Arabic}}
\flushleft{\begin{malayalam}
ശൈത്യകാലത്തെയും ഉഷ്ണകാലത്തെയും യാത്രയുമായി അവരെ കൂട്ടിയിണക്കിയതിനാല്‍,
\end{malayalam}}
\flushright{\begin{Arabic}
\quranayah[106][3]
\end{Arabic}}
\flushleft{\begin{malayalam}
ഈ ഭവനത്തിന്‍റെ രക്ഷിതാവിനെ അവര്‍ ആരാധിച്ചുകൊള്ളട്ടെ.
\end{malayalam}}
\flushright{\begin{Arabic}
\quranayah[106][4]
\end{Arabic}}
\flushleft{\begin{malayalam}
അതായത് അവര്‍ക്ക് വിശപ്പിന്ന് ആഹാരം നല്‍കുകയും, ഭയത്തിന് പകരം സമാധാനം നല്‍കുകയും ചെയ്തവനെ.
\end{malayalam}}
\chapter{\textmalayalam{മാഊന്‍ (  പരോപകാര വസ്തുക്കള്‍ )}}
\begin{Arabic}
\Huge{\centerline{\basmalah}}\end{Arabic}
\flushright{\begin{Arabic}
\quranayah[107][1]
\end{Arabic}}
\flushleft{\begin{malayalam}
മതത്തെ വ്യാജമാക്കുന്നവന്‍ ആരെന്ന് നീ കണ്ടുവോ?
\end{malayalam}}
\flushright{\begin{Arabic}
\quranayah[107][2]
\end{Arabic}}
\flushleft{\begin{malayalam}
അനാഥക്കുട്ടിയെ തള്ളിക്കളയുന്നവനത്രെ അത്‌.
\end{malayalam}}
\flushright{\begin{Arabic}
\quranayah[107][3]
\end{Arabic}}
\flushleft{\begin{malayalam}
പാവപ്പെട്ടവന്‍റെ ഭക്ഷണത്തിന്‍റെ കാര്യത്തില്‍ പ്രോത്സാഹനം നടത്താതിരിക്കുകയും ചെയ്യുന്നവന്‍.
\end{malayalam}}
\flushright{\begin{Arabic}
\quranayah[107][4]
\end{Arabic}}
\flushleft{\begin{malayalam}
എന്നാല്‍ നമസ്കാരക്കാര്‍ക്കാകുന്നു നാശം.
\end{malayalam}}
\flushright{\begin{Arabic}
\quranayah[107][5]
\end{Arabic}}
\flushleft{\begin{malayalam}
തങ്ങളുടെ നമസ്കാരത്തെപ്പറ്റി ശ്രദ്ധയില്ലാത്തവരായ
\end{malayalam}}
\flushright{\begin{Arabic}
\quranayah[107][6]
\end{Arabic}}
\flushleft{\begin{malayalam}
ജനങ്ങളെ കാണിക്കാന്‍ വേണ്ടി പ്രവര്‍ത്തിക്കുന്നവരായ
\end{malayalam}}
\flushright{\begin{Arabic}
\quranayah[107][7]
\end{Arabic}}
\flushleft{\begin{malayalam}
പരോപകാര വസ്തുക്കള്‍ മുടക്കുന്നവരുമായ
\end{malayalam}}
\chapter{\textmalayalam{കൌഥര്‍‍ ( ധാരാളം )}}
\begin{Arabic}
\Huge{\centerline{\basmalah}}\end{Arabic}
\flushright{\begin{Arabic}
\quranayah[108][1]
\end{Arabic}}
\flushleft{\begin{malayalam}
തീര്‍ച്ചയായും നിനക്ക് നാം ധാരാളം നേട്ടം നല്‍കിയിരിക്കുന്നു.
\end{malayalam}}
\flushright{\begin{Arabic}
\quranayah[108][2]
\end{Arabic}}
\flushleft{\begin{malayalam}
ആകയാല്‍ നീ നിന്‍റെ രക്ഷിതാവിന് വേണ്ടി നമസ്കരിക്കുകയും ബലിയര്‍പ്പിക്കുകയും ചെയ്യുക.
\end{malayalam}}
\flushright{\begin{Arabic}
\quranayah[108][3]
\end{Arabic}}
\flushleft{\begin{malayalam}
തീര്‍ച്ചയായും നിന്നോട് വിദ്വേഷം വെച്ച് പുലര്‍ത്തുന്നവന്‍ തന്നെയാകുന്നു വാലറ്റവന്‍ (ഭാവിയില്ലാത്തവന്‍).
\end{malayalam}}
\chapter{\textmalayalam{കാഫിറൂന്‍ ( സത്യനിഷേധികള്‍ )}}
\begin{Arabic}
\Huge{\centerline{\basmalah}}\end{Arabic}
\flushright{\begin{Arabic}
\quranayah[109][1]
\end{Arabic}}
\flushleft{\begin{malayalam}
(നബിയേ,) പറയുക: അവിശ്വാസികളേ,
\end{malayalam}}
\flushright{\begin{Arabic}
\quranayah[109][2]
\end{Arabic}}
\flushleft{\begin{malayalam}
നിങ്ങള്‍ ആരാധിച്ചുവരുന്നതിനെ ഞാന്‍ ആരാധിക്കുന്നില്ല.
\end{malayalam}}
\flushright{\begin{Arabic}
\quranayah[109][3]
\end{Arabic}}
\flushleft{\begin{malayalam}
ഞാന്‍ ആരാധിച്ചുവരുന്നതിനെ നിങ്ങളും ആരാധിക്കുന്നവരല്ല.
\end{malayalam}}
\flushright{\begin{Arabic}
\quranayah[109][4]
\end{Arabic}}
\flushleft{\begin{malayalam}
നിങ്ങള്‍ ആരാധിച്ചുവന്നതിനെ ഞാന്‍ ആരാധിക്കാന്‍ പോകുന്നവനുമല്ല.
\end{malayalam}}
\flushright{\begin{Arabic}
\quranayah[109][5]
\end{Arabic}}
\flushleft{\begin{malayalam}
ഞാന്‍ ആരാധിച്ചു വരുന്നതിനെ നിങ്ങളും ആരാധിക്കാന്‍ പോകുന്നവരല്ല.
\end{malayalam}}
\flushright{\begin{Arabic}
\quranayah[109][6]
\end{Arabic}}
\flushleft{\begin{malayalam}
നിങ്ങള്‍ക്ക് നിങ്ങളുടെ മതം. എനിക്ക് എന്‍റെ മതവും.
\end{malayalam}}
\chapter{\textmalayalam{നസ്ര്‍ ( സഹായം )}}
\begin{Arabic}
\Huge{\centerline{\basmalah}}\end{Arabic}
\flushright{\begin{Arabic}
\quranayah[110][1]
\end{Arabic}}
\flushleft{\begin{malayalam}
അല്ലാഹുവിന്റെ സഹായവും വിജയവും വന്നുകിട്ടിയാല്‍.
\end{malayalam}}
\flushright{\begin{Arabic}
\quranayah[110][2]
\end{Arabic}}
\flushleft{\begin{malayalam}
ജനങ്ങള്‍ അല്ലാഹുവിന്റെ മതത്തില്‍ കൂട്ടംകൂട്ടമായി പ്രവേശിക്കുന്നത് നീ കാണുകയും ചെയ്താല്‍
\end{malayalam}}
\flushright{\begin{Arabic}
\quranayah[110][3]
\end{Arabic}}
\flushleft{\begin{malayalam}
നിന്റെ രക്ഷിതാവിനെ സ്തുതിക്കുന്നതോടൊപ്പം നീ അവനെ പ്രകീര്‍ത്തിക്കുകയും, നീ അവനോട് പാപമോചനം തേടുകയും ചെയ്യുക. തീര്‍ച്ചയായും അവന്‍ പശ്ചാത്താപം സ്വീകരിക്കുന്നവനാകുന്നു.
\end{malayalam}}
\chapter{\textmalayalam{മസദ് (  ഈന്തപ്പനനാര് )}}
\begin{Arabic}
\Huge{\centerline{\basmalah}}\end{Arabic}
\flushright{\begin{Arabic}
\quranayah[111][1]
\end{Arabic}}
\flushleft{\begin{malayalam}
അബൂലഹബിന്‍റെ ഇരുകൈകളും നശിച്ചിരിക്കുന്നു. അവന്‍ നാശമടയുകയും ചെയ്തിരിക്കുന്നു.
\end{malayalam}}
\flushright{\begin{Arabic}
\quranayah[111][2]
\end{Arabic}}
\flushleft{\begin{malayalam}
അവന്‍റെ ധനമോ അവന്‍ സമ്പാദിച്ചുവെച്ചതോ അവനു ഉപകാരപ്പെട്ടില്ല.
\end{malayalam}}
\flushright{\begin{Arabic}
\quranayah[111][3]
\end{Arabic}}
\flushleft{\begin{malayalam}
തീജ്വാലകളുള്ള നരകാഗ്നിയില്‍ അവന്‍ പ്രവേശിക്കുന്നതാണ്‌.
\end{malayalam}}
\flushright{\begin{Arabic}
\quranayah[111][4]
\end{Arabic}}
\flushleft{\begin{malayalam}
വിറകുചുമട്ടുകാരിയായ അവന്‍റെ ഭാര്യയും.
\end{malayalam}}
\flushright{\begin{Arabic}
\quranayah[111][5]
\end{Arabic}}
\flushleft{\begin{malayalam}
അവളുടെ കഴുത്തില്‍ ഈന്തപ്പനനാരുകൊണ്ടുള്ള ഒരു കയറുണ്ടായിരിക്കും.
\end{malayalam}}
\chapter{\textmalayalam{ഇഖ് ലാസ് ( നിഷ്കളങ്കത )}}
\begin{Arabic}
\Huge{\centerline{\basmalah}}\end{Arabic}
\flushright{\begin{Arabic}
\quranayah[112][1]
\end{Arabic}}
\flushleft{\begin{malayalam}
(നബിയേ,) പറയുക: കാര്യം അല്ലാഹു ഏകനാണ് എന്നതാകുന്നു.
\end{malayalam}}
\flushright{\begin{Arabic}
\quranayah[112][2]
\end{Arabic}}
\flushleft{\begin{malayalam}
അല്ലാഹു ഏവര്‍ക്കും ആശ്രയമായിട്ടുള്ളവനാകുന്നു.
\end{malayalam}}
\flushright{\begin{Arabic}
\quranayah[112][3]
\end{Arabic}}
\flushleft{\begin{malayalam}
അവന്‍ (ആര്‍ക്കും) ജന്‍മം നല്‍കിയിട്ടില്ല. (ആരുടെയും സന്തതിയായി) ജനിച്ചിട്ടുമില്ല.
\end{malayalam}}
\flushright{\begin{Arabic}
\quranayah[112][4]
\end{Arabic}}
\flushleft{\begin{malayalam}
അവന്ന് തുല്യനായി ആരും ഇല്ലതാനും.
\end{malayalam}}
\chapter{\textmalayalam{ഫലഖ് ( പുലരി )}}
\begin{Arabic}
\Huge{\centerline{\basmalah}}\end{Arabic}
\flushright{\begin{Arabic}
\quranayah[113][1]
\end{Arabic}}
\flushleft{\begin{malayalam}
പറയുക: പുലരിയുടെ രക്ഷിതാവിനോട് ഞാന്‍ ശരണം തേടുന്നു.
\end{malayalam}}
\flushright{\begin{Arabic}
\quranayah[113][2]
\end{Arabic}}
\flushleft{\begin{malayalam}
അവന്‍ സൃഷ്ടിച്ചുട്ടുള്ളവയുടെ കെടുതിയില്‍ നിന്ന്‌.
\end{malayalam}}
\flushright{\begin{Arabic}
\quranayah[113][3]
\end{Arabic}}
\flushleft{\begin{malayalam}
ഇരുളടയുമ്പോഴുള്ള രാത്രിയുടെ കെടുതിയില്‍നിന്നും.
\end{malayalam}}
\flushright{\begin{Arabic}
\quranayah[113][4]
\end{Arabic}}
\flushleft{\begin{malayalam}
കെട്ടുകളില്‍ ഊതുന്ന സ്ത്രീകളുടെ കെടുതിയില്‍നിന്നും
\end{malayalam}}
\flushright{\begin{Arabic}
\quranayah[113][5]
\end{Arabic}}
\flushleft{\begin{malayalam}
അസൂയാലു അസൂയപ്പെടുമ്പോള്‍ അവന്‍റെ കെടുതിയില്‍നിന്നും.
\end{malayalam}}
\chapter{\textmalayalam{നാസ് ( ജനങ്ങള്‍ )}}
\begin{Arabic}
\Huge{\centerline{\basmalah}}\end{Arabic}
\flushright{\begin{Arabic}
\quranayah[114][1]
\end{Arabic}}
\flushleft{\begin{malayalam}
പറയുക: മനുഷ്യരുടെ രക്ഷിതാവിനോട് ഞാന്‍ ശരണം തേടുന്നു.
\end{malayalam}}
\flushright{\begin{Arabic}
\quranayah[114][2]
\end{Arabic}}
\flushleft{\begin{malayalam}
മനുഷ്യരുടെ രാജാവിനോട്‌.
\end{malayalam}}
\flushright{\begin{Arabic}
\quranayah[114][3]
\end{Arabic}}
\flushleft{\begin{malayalam}
മനുഷ്യരുടെ ദൈവത്തോട്‌.
\end{malayalam}}
\flushright{\begin{Arabic}
\quranayah[114][4]
\end{Arabic}}
\flushleft{\begin{malayalam}
ദുര്‍ബോധനം നടത്തി പിന്‍മാറിക്കളയുന്നവരെക്കൊണ്ടുള്ള കെടുതിയില്‍ നിന്ന്‌.
\end{malayalam}}
\flushright{\begin{Arabic}
\quranayah[114][5]
\end{Arabic}}
\flushleft{\begin{malayalam}
മനുഷ്യരുടെ ഹൃദയങ്ങളില്‍ ദുര്‍ബോധനം നടത്തുന്നവര്‍.
\end{malayalam}}
\flushright{\begin{Arabic}
\quranayah[114][6]
\end{Arabic}}
\flushleft{\begin{malayalam}
മനുഷ്യരിലും ജിന്നുകളിലും പെട്ടവര്‍.
\end{malayalam}}
